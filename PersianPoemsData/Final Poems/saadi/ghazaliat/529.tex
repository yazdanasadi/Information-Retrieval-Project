\begin{center}
\section*{غزل ۵۲۹: سست پیمانا به یک ره دل ز ما برداشتی}
\label{sec:529}
\addcontentsline{toc}{section}{\nameref{sec:529}}
\begin{longtable}{l p{0.5cm} r}
سست پیمانا به یک ره دل ز ما برداشتی
&&
آخر ای بد عهد سنگین دل چرا برداشتی
\\
نوع تقصیری تواند بود ای سلطان عشق
&&
تا به یک ره سایه لطف از گدا برداشتی
\\
گفته بودی با تو در خواهم کشیدن جام وصل
&&
جرعه‌ای ناخورده شمشیر جفا برداشتی
\\
خاطر از مهر کسان برداشتم از بهر تو
&&
چون تو را گشتم تو خود خاطر ز ما برداشتی
\\
لعل دیدی لاجرم چشم از شبه بردوختی
&&
در پسندیدی و دست از کهربا برداشتی
\\
شمع برکردی چراغت بازنامد در نظر
&&
گل فرا دست آمدت مهر از گیا برداشتی
\\
دوست بردارد به جرمی یا خطایی دل ز دوست
&&
تو خطا کردی که بی جرم و خطا برداشتی
\\
عمرها در زیر دامن برد سعدی پای صبر
&&
سر ندیدم کز گریبان وفا برداشتی
\\
\end{longtable}
\end{center}
