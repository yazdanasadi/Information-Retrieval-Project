\begin{center}
\section*{غزل ۵۵۲: روی گشاده ای صنم طاقت خلق می‌بری}
\label{sec:552}
\addcontentsline{toc}{section}{\nameref{sec:552}}
\begin{longtable}{l p{0.5cm} r}
روی گشاده ای صنم طاقت خلق می‌بری
&&
چون پس پرده می‌روی پرده صبر می‌دری
\\
حور بهشت خوانمت ماه تمام گویمت
&&
کآدمیی ندیده‌ام چون تو پری به دلبری
\\
آینه را تو داده‌ای پرتو روی خویشتن
&&
ور نه چه زهره داشتی در نظرت برابری
\\
نسخه چشم و ابرویت پیش نگارگر برم
&&
گویمش این چنین بکن صورت قوس و مشتری
\\
چون تو درخت دل نشان تازه بهار و گلفشان
&&
حیف بود که سایه‌ای بر سر ما نگستری
\\
دیده به روی هر کسی برنکنم ز مهر تو
&&
در ز عوام بسته به چون تو به خانه اندری
\\
من نه مخیرم که چشم از تو به خویشتن کنم
&&
گر تو نظر به ما کنی ور نکنی مخیری
\\
پند حکیم بیش از این در من اثر نمی‌کند
&&
کیست که برکند یکی زمزمه قلندری
\\
عشق و دوام عافیت مختلفند سعدیا
&&
هر که سفر نمی‌کند دل ندهد به لشکری
\\
\end{longtable}
\end{center}
