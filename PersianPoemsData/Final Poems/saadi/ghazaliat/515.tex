\begin{center}
\section*{غزل ۵۱۵: چه جرم رفت که با ما سخن نمی‌گویی}
\label{sec:515}
\addcontentsline{toc}{section}{\nameref{sec:515}}
\begin{longtable}{l p{0.5cm} r}
چه جرم رفت که با ما سخن نمی‌گویی
&&
جنایت از طرف ماست یا تو بدخویی
\\
تو از نبات گرو برده‌ای به شیرینی
&&
به اتفاق ولیکن نبات خودرویی
\\
هزار جان به ارادت تو را همی‌جویند
&&
تو سنگدل به لطافت دلی نمی‌جویی
\\
ولیک با همه عیب از تو صبر نتوان کرد
&&
بیا و گر همه بد کرده‌ای که نیکویی
\\
تو بد مگوی و گر نیز خاطرت باشد
&&
بگوی از آن لب شیرین که نیک می‌گویی
\\
گلم نباید و سروم به چشم درناید
&&
مرا وصال تو باید که سرو گلبویی
\\
هزار جامه سپر ساختیم و هم بگذشت
&&
خدنگ غمزه خوبان ز دلق نه تویی
\\
به دست جهد نشاید گرفت دامن کام
&&
اگر نخواهدت ای نفس خیره می‌پویی
\\
درست شد که به یک دل دو دوست نتوان داشت
&&
به ترک خویش بگوی ای که طالب اویی
\\
همین که پای نهادی بر آستانه عشق
&&
به دست باش که دست از جهان فروشویی
\\
درازنای شب از چشم دردمندان پرس
&&
تو قدر آب چه دانی که بر لب جویی
\\
ز خاک سعدی بیچاره بوی عشق آید
&&
هزار سال پس از مرگش ار بینبویی
\\
\end{longtable}
\end{center}
