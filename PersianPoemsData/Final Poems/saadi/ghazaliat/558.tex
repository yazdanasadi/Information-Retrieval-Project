\begin{center}
\section*{غزل ۵۵۸: هر نوبتم که در نظر ای ماه بگذری}
\label{sec:558}
\addcontentsline{toc}{section}{\nameref{sec:558}}
\begin{longtable}{l p{0.5cm} r}
هر نوبتم که در نظر ای ماه بگذری
&&
بار دوم ز بار نخستین نکوتری
\\
انصاف می‌دهم که لطیفان و دلبران
&&
بسیار دیده‌ام نه بدین لطف و دلبری
\\
زنار بود هر چه همه عمر داشتم
&&
الا کمر که پیش تو بستم به چاکری
\\
از شرم چون تو آدمیان در میان خلق
&&
انصاف می‌دهد که نهان می‌شود پری
\\
شمشیر اختیار تو را سر نهاده‌ام
&&
دانم که گر تنم بکشی جان بپروری
\\
جز صورتت در آینه کس را نمی‌رسد
&&
با صورت بدیع تو کردن برابری
\\
ای مدعی گر آنچه مرا شد تو را شود
&&
بر حال من ببخشی و حالت بیاوری
\\
صید اوفتاد و پای مسافر به گل بماند
&&
هیچ افتدت که بر سر افتاده بگذری
\\
صبری که بود مایه سعدی دگر نماند
&&
سختی مکن که کیسه بپرداخت مشتری
\\
\end{longtable}
\end{center}
