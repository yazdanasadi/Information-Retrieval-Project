\begin{center}
\section*{غزل ۳۲۱: آن که هلاک من همی‌خواهد و من سلامتش}
\label{sec:321}
\addcontentsline{toc}{section}{\nameref{sec:321}}
\begin{longtable}{l p{0.5cm} r}
آن که هلاک من همی‌خواهد و من سلامتش
&&
هر چه کند ز شاهدی کس نکند ملامتش
\\
میوه نمی‌دهد به کس باغ تفرج است و بس
&&
جز به نظر نمی‌رسد سیب درخت قامتش
\\
داروی دل نمی‌کنم کان که مریض عشق شد
&&
هیچ دوا نیاورد باز به استقامتش
\\
هر که فدا نمی‌کند دنیی و دین و مال و سر
&&
گو غم نیکوان مخور تا نخوری ندامتش
\\
جنگ نمی‌کنم اگر دست به تیغ می‌برد
&&
بلکه به خون مطالبت هم نکنم قیامتش
\\
کاش که در قیامتش بار دگر بدیدمی
&&
کانچه گناه او بود من بکشم غرامتش
\\
هر که هوا گرفت و رفت از پی آرزوی دل
&&
گوش مدار سعدیا بر خبر سلامتش
\\
\end{longtable}
\end{center}
