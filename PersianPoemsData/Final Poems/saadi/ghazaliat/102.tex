\begin{center}
\section*{غزل ۱۰۲: تا دست‌ها کمر نکنی بر میان دوست}
\label{sec:102}
\addcontentsline{toc}{section}{\nameref{sec:102}}
\begin{longtable}{l p{0.5cm} r}
تا دست‌ها کمر نکنی بر میان دوست
&&
بوسی به کام دل ندهی بر دهان دوست
\\
دانی حیات کشته شمشیر عشق چیست
&&
سیبی گزیدن از رخ چون بوستان دوست
\\
بر ماجرای خسرو و شیرین قلم کشید
&&
شوری که در میان منست و میان دوست
\\
خصمی که تیر کافرش اندر غزا نکشت
&&
خونش بریخت ابروی همچون کمان دوست
\\
دل رفت و دیده خون شد و جان ضعیف ماند
&&
وان هم برای آن که کنم جان فدای دوست
\\
روزی به پای مرکب تازی درافتمش
&&
گر کبر و ناز بازنپیچد عنان دوست
\\
هیهات کام من که برآید در این طلب
&&
این بس که نام من برود بر زبان دوست
\\
چون جان سپردنیست به هر صورتی که هست
&&
در کوی عشق خوشتر و بر آستان دوست
\\
با خویشتن همی‌برم این شوق تا به خاک
&&
وز خاک سر برآرم و پرسم نشان دوست
\\
فریاد مردمان همه از دست دشمنست
&&
فریاد سعدی از دل نامهربان دوست
\\
\end{longtable}
\end{center}
