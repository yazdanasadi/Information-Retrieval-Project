\begin{center}
\section*{غزل ۲۷۴: چه سروست آن که بالا می‌نماید}
\label{sec:274}
\addcontentsline{toc}{section}{\nameref{sec:274}}
\begin{longtable}{l p{0.5cm} r}
چه سروست آن که بالا می‌نماید
&&
عنان از دست دل‌ها می‌رباید
\\
که زاد این صورت منظور محبوب
&&
از این صورت ندانم تا چه زاید
\\
اگر صد نوبتش چون قرص خورشید
&&
ببینم آب در چشم من آید
\\
کس اندر عهد ما مانند وی نیست
&&
ولی ترسم به عهد ما نپاید
\\
فراغت زان طرف چندان که خواهی
&&
وزین جانب محبت می‌فزاید
\\
حدیث عشق جانان گفتنی نیست
&&
و گر گویی کسی همدرد باید
\\
درازای شب از ناخفتگان پرس
&&
که خواب آلوده را کوته نماید
\\
مرا پای گریز از دست او نیست
&&
اگر می‌بنددم ور می‌گشاید
\\
رها کن تا بیفتد ناتوانی
&&
که با سرپنجگان زور آزماید
\\
نشاید خون سعدی بی سبب ریخت
&&
ولیکن چون مراد اوست شاید
\\
\end{longtable}
\end{center}
