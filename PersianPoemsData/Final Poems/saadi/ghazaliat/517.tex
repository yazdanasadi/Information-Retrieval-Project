\begin{center}
\section*{غزل ۵۱۷: ای حسن خط از دفتر اخلاق تو بابی}
\label{sec:517}
\addcontentsline{toc}{section}{\nameref{sec:517}}
\begin{longtable}{l p{0.5cm} r}
ای حسن خط از دفتر اخلاق تو بابی
&&
شیرینی از اوصاف تو حرفی ز کتابی
\\
از بوی تو در تاب شود آهوی مشکین
&&
گر باز کنند از شکن زلف تو تابی
\\
بر دیده صاحب نظران خواب ببستی
&&
ترسی که ببینند خیال تو به خوابی
\\
از خنده شیرین نمکدان دهانت
&&
خون می‌رود از دل چو نمک خورده کبابی
\\
تا عذر زلیخا بنهد منکر عشاق
&&
یوسف صفت از چهره برانداز نقابی
\\
بی روی توام جنت فردوس نباید
&&
کاین تشنگی از من نبرد هیچ شرابی
\\
مشغول تو را گر بگذارند به دوزخ
&&
با یاد تو دردش نکند هیچ عذابی
\\
باری به طریق کرمم بنده خود خوان
&&
تا بشنوی از هر بن موییم جوابی
\\
در من منگر تا دگران چشم ندارند
&&
کز دست گدایان نتوان کرد ثوابی
\\
آب سخنم می‌رود از طبع چو آتش
&&
چون آتش رویت که از او می‌چکد آبی
\\
یاران همه با یار و من خسته طلبکار
&&
هر کس به سر آبی و سعدی به سرابی
\\
\end{longtable}
\end{center}
