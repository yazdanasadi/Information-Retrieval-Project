\begin{center}
\section*{غزل ۳۲۰: یاری به دست کن که به امید راحتش}
\label{sec:320}
\addcontentsline{toc}{section}{\nameref{sec:320}}
\begin{longtable}{l p{0.5cm} r}
یاری به دست کن که به امید راحتش
&&
واجب کند که صبر کنی بر جراحتش
\\
ما را که ره دهد به سراپرده وصال
&&
ای باد صبحدم خبری ده ز ساحتش
\\
باران چون ستاره‌ام از دیدگان بریخت
&&
رویی که صبح خیره شود در صباحتش
\\
هر گه که گویم این دل ریشم درست شد
&&
بر وی پراکند نمکی از ملاحتش
\\
هرچ آن قبیحتر بکند یار دوست روی
&&
داند که چشم دوست نبیند قباحتش
\\
بیچاره‌ای که صورت رویت خیال بست
&&
بی دیدنت خیال مبند استراحتش
\\
با چشم نیم خواب تو خشم آیدم همی
&&
از چشم‌های نرگس و چندان وقاحتش
\\
رفتار شاهد و لب خندان و روی خوب
&&
چون آدمی طمع نکند در سماحتش
\\
سعدی که داد وصف همه نیکوان به داد
&&
عاجز بماند در تو زبان فصاحتش
\\
\end{longtable}
\end{center}
