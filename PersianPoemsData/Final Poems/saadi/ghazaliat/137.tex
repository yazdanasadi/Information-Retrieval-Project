\begin{center}
\section*{غزل ۱۳۷: کیست آن لعبت خندان که پری وار برفت}
\label{sec:137}
\addcontentsline{toc}{section}{\nameref{sec:137}}
\begin{longtable}{l p{0.5cm} r}
کیست آن لعبت خندان که پری وار برفت
&&
که قرار از دل دیوانه به یک بار برفت
\\
باد بوی گل رویش به گلستان آورد
&&
آب گلزار بشد رونق عطار برفت
\\
صورت یوسف نادیده صفت می‌کردیم
&&
چون بدیدیم زبان سخن از کار برفت
\\
بعد از این عیب و ملامت نکنم مستان را
&&
که مرا در حق این طایفه انکار برفت
\\
در سرم بود که هرگز ندهم دل به خیال
&&
به سرت کز سر من آن همه پندار برفت
\\
آخر این مور میان بسته افتان خیزان
&&
چه خطا داشت که سرکوفته چون مار برفت
\\
به خرابات چه حاجت که یکی مست شود
&&
که به دیدار تو عقل از سر هشیار برفت
\\
به نماز آمده محراب دو ابروی تو دید
&&
دلش از دست ببردند و به زنار برفت
\\
پیش تو مردن از آن به که پس از من گویند
&&
نه به صدق آمده بود این که به آزار برفت
\\
تو نه مرد گل بستان امیدی سعدی
&&
که به پهلو نتوانی به سر خار برفت
\\
\end{longtable}
\end{center}
