\begin{center}
\section*{غزل ۳۴۰: هر کسی را هوسی در سر و کاری در پیش}
\label{sec:340}
\addcontentsline{toc}{section}{\nameref{sec:340}}
\begin{longtable}{l p{0.5cm} r}
هر کسی را هوسی در سر و کاری در پیش
&&
من بی‌کار گرفتار هوای دل خویش
\\
هرگز اندیشه نکردم که تو با من باشی
&&
چون به دست آمدی ای لقمه از حوصله بیش
\\
این تویی با من و غوغای رقیبان از پس
&&
وین منم با تو گرفته ره صحرا در پیش
\\
همچنان داغ جدایی جگرم می‌سوزد
&&
مگرم دست چو مرهم بنهی بر دل ریش
\\
باور از بخت ندارم که تو مهمان منی
&&
خیمه پادشه آن گاه فضای درویش
\\
زخم شمشیر غمت را ننهم مرهم کس
&&
طشت زرینم و پیوند نگیرم به سریش
\\
عاشقان را نتوان گفت که بازآی از مهر
&&
کافران را نتوان گفت که برگرد از کیش
\\
منم امروز و تو و مطرب و ساقی و حسود
&&
خویشتن گو به در حجره بیاویز چو خیش
\\
من خود از کید عدو باک ندارم لیکن
&&
کژدم از خبث طبیعت بزند سنگ به نیش
\\
تو به آرام دل خویش رسیدی سعدی
&&
می خور و غم مخور از شنعت بیگانه و خویش
\\
ای که گفتی به هوا دل منه و مهر مبند
&&
من چنینم تو برو مصلحت خویش اندیش
\\
\end{longtable}
\end{center}
