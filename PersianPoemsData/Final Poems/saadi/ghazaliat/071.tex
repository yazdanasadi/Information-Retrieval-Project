\begin{center}
\section*{غزل ۷۱: دلی که عاشق و صابر بود مگر سنگست}
\label{sec:071}
\addcontentsline{toc}{section}{\nameref{sec:071}}
\begin{longtable}{l p{0.5cm} r}
دلی که عاشق و صابر بود مگر سنگ است
&&
ز عشق تا به صبوری هزار فرسنگ است
\\
برادران طریقت نصیحتم مکنید
&&
که توبه در ره عشق آبگینه بر سنگ است
\\
دگر به خفیه نمی‌بایدم شراب و سماع
&&
که نیکنامی در دین عاشقان ننگ است
\\
چه تربیت شنوم یا چه مصلحت بینم
&&
مرا که چشم به ساقی و گوش بر چنگ است
\\
به یادگار کسی دامن نسیم صبا
&&
گرفته‌ایم و دریغا که باد در چنگ است
\\
به خشم رفتهٔ ما را که می‌برد پیغام
&&
بیا که ما سپر انداختیم اگر جنگ است
\\
بکش چنان که توانی که بی مشاهده‌ات
&&
فراخنای جهان بر وجود ما تنگ است
\\
ملامت از دل سعدی فرونشوید عشق
&&
سیاهی از حبشی چون رود که خودرنگ است
\\
\end{longtable}
\end{center}
