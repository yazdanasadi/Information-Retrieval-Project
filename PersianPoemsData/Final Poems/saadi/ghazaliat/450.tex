\begin{center}
\section*{غزل ۴۵۰: بگذار تا بگرییم چون ابر در بهاران}
\label{sec:450}
\addcontentsline{toc}{section}{\nameref{sec:450}}
\begin{longtable}{l p{0.5cm} r}
بگذار تا بگرییم چون ابر در بهاران
&&
کز سنگ گریه  خیزد روز وداع یاران
\\
هر کو شراب فرقت روزی چشیده باشد
&&
داند که سخت باشد قطع امیدواران
\\
با ساربان بگویید احوال آب چشمم
&&
تا بر شتر نبندد محمل به روز باران
\\
بگذاشتند ما را در دیده آب حسرت
&&
گریان چو در قیامت چشم گناهکاران
\\
ای صبح شب نشینان جانم به طاقت آمد
&&
از بس که دیر ماندی چون شام روزه داران
\\
چندین که برشمردم از ماجرای عشقت
&&
اندوه دل نگفتم الا یک از هزاران
\\
سعدی به روزگاران مهری نشسته در دل
&&
بیرون نمی‌توان کرد الا به روزگاران
\\
چندت کنم حکایت شرح این قدر کفایت
&&
باقی نمی‌توان گفت الا به غمگساران
\\
\end{longtable}
\end{center}
