\begin{center}
\section*{غزل ۳۸۴: نرفت تا تو برفتی خیالت از نظرم}
\label{sec:384}
\addcontentsline{toc}{section}{\nameref{sec:384}}
\begin{longtable}{l p{0.5cm} r}
نرفت تا تو برفتی خیالت از نظرم
&&
برفت در همه عالم به بی دلی خبرم
\\
نه بخت و دولت آنم که با تو بنشینم
&&
نه صبر و طاقت آنم که از تو درگذرم
\\
من از تو روی نخواهم به دیگری آورد
&&
که زشت باشد هر روز قبله دگرم
\\
بلای عشق تو بر من چنان اثر کرده‌ست
&&
که پند عالم و عابد نمی‌کند اثرم
\\
قیامتم که به دیوان حشر پیش آرند
&&
میان آن همه تشویش در تو می‌نگرم
\\
به جان دوست که چون دوست در برم باشد
&&
هزار دشمن اگر بر سرند غم نخورم
\\
نشان پیکر خوبت نمی‌توانم داد
&&
که در تأمل او خیره می‌شود بصرم
\\
تو نیز اگر نشناسی مرا عجب نبود
&&
که هر چه در نظر آید از آن ضعیفترم
\\
به جان و سر که نگردانم از وصال تو روی
&&
و گر هزار ملامت رسد به جان و سرم
\\
مرا مگوی که سعدی چرا پریشانی
&&
خیال روی تو بر می‌کند به یک دگرم
\\
\end{longtable}
\end{center}
