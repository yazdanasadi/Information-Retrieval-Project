\begin{center}
\section*{غزل ۳۳۱: هر که هست التفات بر جانش}
\label{sec:331}
\addcontentsline{toc}{section}{\nameref{sec:331}}
\begin{longtable}{l p{0.5cm} r}
هر که هست التفات بر جانش
&&
گو مزن لاف مهر جانانش
\\
درد من بر من از طبیب من است
&&
از که جویم دوا و درمانش
\\
آن که سر در کمند وی دارد
&&
نتوان رفت جز به فرمانش
\\
چه کند بنده حقیر فقیر
&&
که نباشد به امر سلطانش
\\
ناگزیر است یار عاشق را
&&
که ملامت کنند یارانش
\\
وآن که در بحر قلزم است غریق
&&
چه تفاوت کند ز بارانش
\\
گل به غایت رسید بگذارید
&&
تا بنالد هزاردستانش
\\
عقل را گر هزار حجت هست
&&
عشق دعوی کند به بطلانش
\\
هر که را نوبتی زدند این تیر
&&
در جراحت بماند پیکانش
\\
ناله‌ای می‌کند چو گریه طفل
&&
که ندانند درد پنهانش
\\
سخن عشق زینهار مگوی
&&
یا چو گفتی بیار برهانش
\\
نرود هوشمند در آبی
&&
تا نبیند نخست پایانش
\\
سعدیا گر به یک دمت بی دوست
&&
هر دو عالم دهند مستانش
\\
\end{longtable}
\end{center}
