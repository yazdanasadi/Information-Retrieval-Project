\begin{center}
\section*{غزل ۳۶: بنده وار آمدم به زنهارت}
\label{sec:036}
\addcontentsline{toc}{section}{\nameref{sec:036}}
\begin{longtable}{l p{0.5cm} r}
بنده وار آمدم به زنهارت
&&
که ندارم سلاح پیکارت
\\
متفق می‌شوم که دل ندهم
&&
معتقد می‌شوم دگربارت
\\
مشتری را بهای روی تو نیست
&&
من بدین مفلسی خریدارت
\\
غیرتم هست و اقتدارم نیست
&&
که بپوشم ز چشم اغیارت
\\
گر چه بی طاقتم چو مور ضعیف
&&
می‌کشم نفس و می‌کشم بارت
\\
نه چنان در کمند پیچیدی
&&
که مخلص شود گرفتارت
\\
من هم اول که دیدمت گفتم
&&
حذر از چشم مست خون خوارت
\\
دیده شاید که بی تو برنکند
&&
تا نبیند فراق دیدارت
\\
تو ملولی و دوستان مشتاق
&&
تو گریزان و ما طلبکارت
\\
چشم سعدی به خواب بیند خواب
&&
که ببستی به چشم سحارت
\\
تو بدین هر دو چشم خواب آلود
&&
چه غم از چشم‌های بیدارت
\\
\end{longtable}
\end{center}
