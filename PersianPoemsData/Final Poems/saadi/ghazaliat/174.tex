\begin{center}
\section*{غزل ۱۷۴: آن شکرخنده که پرنوش دهانی دارد}
\label{sec:174}
\addcontentsline{toc}{section}{\nameref{sec:174}}
\begin{longtable}{l p{0.5cm} r}
آن شکرخنده که پرنوش دهانی دارد
&&
نه دل من که دل خلق جهانی دارد
\\
به تماشای درخت چمنش حاجت نیست
&&
هر که در خانه چنو سرو روانی دارد
\\
کافران از بت بی‌جان چه تمتع دارند
&&
باری آن بت بپرستند که جانی دارد
\\
ابرویش خم به کمان ماند و قد راست به تیر
&&
کس ندیدم که چنین تیر و کمانی دارد
\\
علت آنست که وقتی سخنی می‌گوید
&&
ور نه معلوم نبودی که دهانی دارد
\\
حجت آنست که وقتی کمری می‌بندد
&&
ور نه مفهوم نگشتی که میانی دارد
\\
ای که گفتی مرو اندر پی خون خواره خویش
&&
با کسی گوی که در دست عنانی دارد
\\
عشق داغیست که تا مرگ نیاید نرود
&&
هر که بر چهره از این داغ نشانی دارد
\\
سعدیا کشتی از این موج به در نتوان برد
&&
که نه بحریست محبت که کرانی دارد
\\
\end{longtable}
\end{center}
