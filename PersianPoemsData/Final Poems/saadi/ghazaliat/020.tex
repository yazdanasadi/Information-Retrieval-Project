\begin{center}
\section*{غزل ۲۰: لاابالی چه کند دفتر دانایی را}
\label{sec:020}
\addcontentsline{toc}{section}{\nameref{sec:020}}
\begin{longtable}{l p{0.5cm} r}
لاابالی چه کند دفتر دانایی را
&&
طاقت وعظ نباشد سر سودایی را
\\
آب را قول تو با آتش اگر جمع کند
&&
نتواند که کند عشق و شکیبایی را
\\
دیده را فایده آن است که دلبر بیند
&&
ور نبیند چه بود فایده بینایی را
\\
عاشقان را چه غم از سرزنش دشمن و دوست
&&
یا غم دوست خورد یا غم رسوایی را
\\
همه دانند که من سبزه خط دارم دوست
&&
نه چو دیگر حیوان سبزه صحرایی را
\\
من همان روز دل و صبر به یغما دادم
&&
که مقید شدم آن دلبر یغمایی را
\\
سرو بگذار که قدی و قیامی دارد
&&
گو ببین آمدن و رفتن رعنایی را
\\
گر برانی نرود ور برود باز آید
&&
ناگزیر است مگس دکه حلوایی را
\\
بر حدیث من و حسن تو نیفزاید کس
&&
حد همین است سخندانی و زیبایی را
\\
سعدیا نوبتی امشب دهل صبح نکوفت
&&
یا مگر روز نباشد شب تنهایی را
\\
\end{longtable}
\end{center}
