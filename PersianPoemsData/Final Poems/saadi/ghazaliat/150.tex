\begin{center}
\section*{غزل ۱۵۰: خوش می‌روی به تنها تن‌ها فدای جانت}
\label{sec:150}
\addcontentsline{toc}{section}{\nameref{sec:150}}
\begin{longtable}{l p{0.5cm} r}
خوش می‌روی به تنها تن‌ها فدای جانت
&&
مدهوش می‌گذاری یاران مهربانت
\\
آیینه‌ای طلب کن تا روی خود ببینی
&&
وز حسن خود بماند انگشت در دهانت
\\
قصد شکار داری یا اتفاق بستان
&&
عزمی درست باید تا می‌کشد عنانت
\\
ای گلبن خرامان با دوستان نگه کن
&&
تا بگذرد نسیمی بر ما ز بوستانت
\\
رخت سرای عقلم تاراج شوق کردی
&&
ای دزد آشکارا می‌بینم از نهانت
\\
هر دم کمند زلفت صیدی دگر بگیرد
&&
پیکان غمزه در دل ز ابروی چون کمانت
\\
دانی چرا نخفتم تو پادشاه حسنی
&&
خفتن حرام باشد بر چشم پاسبانت
\\
ما را نمی‌برازد با وصلت آشنایی
&&
مرغی لبق تر از من باید هم آشیانت
\\
من آب زندگانی بعد از تو می‌نخواهم
&&
بگذار تا بمیرم بر خاک آستانت
\\
من فتنه زمانم وان دوستان که داری
&&
بی شک نگاه دارند از فتنه زمانت
\\
سعدی چو دوست داری آزاد باش و ایمن
&&
ور دشمنی بباشد با هر که در جهانت
\\
\end{longtable}
\end{center}
