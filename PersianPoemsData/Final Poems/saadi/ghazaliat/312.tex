\begin{center}
\section*{غزل ۳۱۲: بزرگ دولت آن کز درش تو آیی باز}
\label{sec:312}
\addcontentsline{toc}{section}{\nameref{sec:312}}
\begin{longtable}{l p{0.5cm} r}
بزرگ دولت آن کز درش تو آیی باز
&&
بیا بیا که به خیر آمدی کجایی باز
\\
رخی کز او متصور نمی‌شود آرام
&&
چرا نمودی و دیگر نمی‌نمایی باز
\\
در دو لختی چشمان شوخ دلبندت
&&
چه کرده‌ام که به رویم نمی‌گشایی باز
\\
اگر تو را سر ما هست یا غم ما نیست
&&
من از تو دست ندارم به بی‌وفایی باز
\\
شراب وصل تو در کام جان من ازلیست
&&
هنوز مستم از آن جام آشنایی باز
\\
دلی که بر سر کوی تو گم کنم هیهات
&&
که جز به روی تو بینم به روشنایی باز
\\
تو را هرآینه باید به شهر دیگر رفت
&&
که دل نماند در این شهر تا ربایی باز
\\
عوام خلق ملامت کنند صوفی را
&&
کز این هوا و طبیعت چرا نیایی باز
\\
اگر حلاوت مستی بدانی ای هشیار
&&
به عمر خود نبری نام پارسایی باز
\\
گرت چو سعدی از این در نواله‌ای بخشند
&&
برو که خو نکنی هرگز از گدایی باز
\\
\end{longtable}
\end{center}
