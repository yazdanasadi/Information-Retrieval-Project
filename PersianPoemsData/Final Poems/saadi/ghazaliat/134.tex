\begin{center}
\section*{غزل ۱۳۴: دلی که دید که پیرامن خطر می‌گشت}
\label{sec:134}
\addcontentsline{toc}{section}{\nameref{sec:134}}
\begin{longtable}{l p{0.5cm} r}
دلی که دید که پیرامن خطر می‌گشت
&&
چو شمع زار و چو پروانه در به در می‌گشت
\\
هزار گونه غم از چپ و راست دامنگیر
&&
هنوز در تک و پوی غمی دگر می‌گشت
\\
سرش مدام ز شور شراب عشق خراب
&&
چو مست دایم از آن گرد شور و شر می‌گشت
\\
چو بی‌دلان همه در کار عشق می‌آویخت
&&
چو ابلهان همه از راه عقل بر می‌گشت
\\
ز بخت بی ره و آیین و پا و سر می‌زیست
&&
ز عشق بی‌دل و آرام و خواب و خور می‌گشت
\\
هزار بارش از این پند بیشتر دادم
&&
که گرد بیهده کم گرد و بیشتر می‌گشت
\\
به هر طریق که باشد نصیحتش مکنید
&&
که او به قول نصیحت کنان بتر می‌گشت
\\
\end{longtable}
\end{center}
