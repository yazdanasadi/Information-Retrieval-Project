\begin{center}
\section*{غزل ۵۷۲: این چه رفتارست کارامیدن از من می‌بری}
\label{sec:572}
\addcontentsline{toc}{section}{\nameref{sec:572}}
\begin{longtable}{l p{0.5cm} r}
این چه رفتار است کآرامیدن از من می‌بری
&&
هوشم از دل می‌ربایی عقلم از تن می‌بری
\\
باغ و لالستان چه باشد آستینی برفشان
&&
باغبان را گو بیا گر گل به دامن می‌بری
\\
روز و شب می‌باشد آن ساعت که همچون آفتاب
&&
می‌نمایی روی و دیگر باز روزن می‌بری
\\
مویت از پس تا کمرگه خوشه‌ای بر خرمن است
&&
زینهار آن خوشه پنهان کن که خرمن می‌بری
\\
دل به عیاری ببردی ناگهان از دست من
&&
دزد شب گردد تو فارغ روز روشن می‌بری
\\
گر تو برگردیدی از من بی گناه و بی سبب
&&
تا مگر من نیز برگردم غلط ظن می‌بری
\\
چون نیاید دود از آن خرمن که آتش می‌زنی
&&
یا ببندد خون از این موضع که سوزن می‌بری
\\
این طریق دشمنی باشد نه راه دوستی
&&
کآبروی دوستان در پیش دشمن می‌بری
\\
عیب مسکینی مکن افتان و خیزان در پیت
&&
کآن نمی‌آید تو زنجیرش به گردن می‌بری
\\
سعدیا گفتار شیرین پیش آن کام و دهان
&&
در به دریا می‌فرستی زر به معدن می‌بری
\\
\end{longtable}
\end{center}
