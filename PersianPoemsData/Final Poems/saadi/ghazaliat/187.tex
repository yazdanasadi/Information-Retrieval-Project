\begin{center}
\section*{غزل ۱۸۷: هشیار کسی باید کز عشق بپرهیزد}
\label{sec:187}
\addcontentsline{toc}{section}{\nameref{sec:187}}
\begin{longtable}{l p{0.5cm} r}
هشیار کسی باید کز عشق بپرهیزد
&&
وین طبع که من دارم با عقل نیامیزد
\\
آن کس که دلی دارد آراسته معنی
&&
گر هر دو جهان باشد در پای یکی ریزد
\\
گر سیل عقاب آید شوریده نیندیشد
&&
ور تیر بلا بارد دیوانه نپرهیزد
\\
آخر نه منم تنها در بادیه سودا
&&
عشق لب شیرینت بس شور برانگیزد
\\
بی بخت چه فن سازم تا برخورم از وصلت
&&
بی‌مایه زبون باشد هر چند که بستیزد
\\
فضل است اگرم خوانی عدل است اگرم رانی
&&
قدر تو نداند آن کز زجر تو بگریزد
\\
تا دل به تو پیوستم راه همه دربستم
&&
جایی که تو بنشینی بس فتنه که برخیزد
\\
سعدی نظر از رویت کوته نکند هرگز
&&
ور روی بگردانی در دامنت آویزد
\\
\end{longtable}
\end{center}
