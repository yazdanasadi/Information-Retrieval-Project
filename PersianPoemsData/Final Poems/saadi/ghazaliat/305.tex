\begin{center}
\section*{غزل ۳۰۵: از همه باشد به حقیقت گزیر}
\label{sec:305}
\addcontentsline{toc}{section}{\nameref{sec:305}}
\begin{longtable}{l p{0.5cm} r}
از همه باشد به حقیقت گزیر
&&
وز تو نباشد که نداری نظیر
\\
مشرب شیرین نبود بی زحام
&&
دعوت منعم نبود بی فقیر
\\
آن عرق است از بدنت یا گلاب
&&
آن نفس است از دهنت یا عبیر
\\
بذل تو کردم تن و هوش و روان
&&
وقف تو کردم دل و چشم و ضمیر
\\
دل چه بود جان که بدو زنده‌ام
&&
گو بده ای دوست که گویم بگیر
\\
راحت جان باشد از آن قبضه تیغ
&&
مرهم دل باشد از آن جعبه تیر
\\
درد نهانی به که گویم که نیست
&&
باخبر از درد من الا خبیر
\\
عیب کنندم که چه دیدی در او
&&
کور نداند که چه بیند بصیر
\\
چون نرود در پی صاحب کمند
&&
آهوی بیچاره به گردن اسیر
\\
هر که دل شیفته دارد چو من
&&
بس که بگوید سخن دلپذیر
\\
ناله سعدی به چه دانی خوش است
&&
بوی خوش آید چو بسوزد عبیر
\\
\end{longtable}
\end{center}
