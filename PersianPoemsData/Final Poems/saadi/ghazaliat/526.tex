\begin{center}
\section*{غزل ۵۲۶: تعالی الله چه رویست آن که گویی آفتابستی}
\label{sec:526}
\addcontentsline{toc}{section}{\nameref{sec:526}}
\begin{longtable}{l p{0.5cm} r}
تعالی الله چه روی است آن که گویی آفتابستی
&&
و گر مه را حیا بودی ز شرمش در نقابستی
\\
اگر گل را نظر بودی چو نرگس تا جهان بیند
&&
ز شرم رنگ رخسارش چو نیلوفر در آبستی
\\
شبان خوابم نمی‌گیرد نه روز آرام و آسایش
&&
ز چشم مست میگونش که پنداری به خوابستی
\\
گر آن شاهد که من دانم به هر کس روی بنماید
&&
فقیر از رقص در حالت خطیب از می خرابستی
\\
چنان مستم که پنداری نماند امید هشیاری
&&
به هش بازآمدی مجنون اگر مست شرابستی
\\
گر آن ساعد که او دارد بدی با رستم دستان
&&
به یک ساعت بیفکندی اگر افراسیابستی
\\
بیار ای لعبت ساقی اگر تلخ است و گر شیرین
&&
که از دستت شکر باشد و گر خود زهر نابستی
\\
کمال حسن رویت را مخالف نیست جز خویت
&&
دریغا آن لب شیرین اگر شیرین جوابستی
\\
اگر دانی که تا هستم نظر با جز تو پیوستم
&&
پس آنگه بر من مسکین جفا کردن صوابستی
\\
زمین تشنه را باران نبودی بعد از این حاجت
&&
اگر چندان که در چشمم سرشک اندر سحابستی
\\
ز خاکم رشک می‌آید که بر سر می‌نهی پایش
&&
که سعدی زیر نعلینت چه بودی گر ترابستی
\\
\end{longtable}
\end{center}
