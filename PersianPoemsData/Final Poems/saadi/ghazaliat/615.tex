\begin{center}
\section*{غزل ۶۱۵: ندانمت به حقیقت که در جهان به که مانی}
\label{sec:615}
\addcontentsline{toc}{section}{\nameref{sec:615}}
\begin{longtable}{l p{0.5cm} r}
ندانمت به حقیقت که در جهان به که مانی
&&
جهان و هر چه در او هست صورتند و تو جانی
\\
به پای خویشتن آیند عاشقان به کمندت
&&
که هر که را تو بگیری ز خویشتن برهانی
\\
مرا مپرس که چونی به هر صفت که تو خواهی
&&
مرا مگو که چه نامی به هر لقب که تو خوانی
\\
چنان به نظره اول ز شخص می‌ببری دل
&&
که باز می‌نتواند گرفت نظره ثانی
\\
تو پرده پیش گرفتی و ز اشتیاق جمالت
&&
ز پرده‌ها به درافتاد رازهای نهانی
\\
بر آتش تو نشستیم و دود شوق برآمد
&&
تو ساعتی ننشستی که آتشی بنشانی
\\
چو پیش خاطرم آید خیال صورت خوبت
&&
ندانمت که چه گویم ز اختلاف معانی
\\
مرا گناه نباشد نظر به روی جوانان
&&
که پیر داند مقدار روزگار جوانی
\\
تو را که دیده ز خواب و خمار باز نباشد
&&
ریاضت من شب تا سحر نشسته چه دانی
\\
من ای صبا ره رفتن به کوی دوست ندانم
&&
تو می‌روی به سلامت سلام من برسانی
\\
سر از کمند تو سعدی به هیچ روی نتابد
&&
اسیر خویش گرفتی بکش چنان که تو دانی
\\
\end{longtable}
\end{center}
