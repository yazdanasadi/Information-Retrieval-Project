\begin{center}
\section*{غزل ۵۹۴: ترحم ذلتی یا ذا المعالی}
\label{sec:594}
\addcontentsline{toc}{section}{\nameref{sec:594}}
\begin{longtable}{l p{0.5cm} r}
ترحم ذلتی یا ذا المعالی
&&
و واصلنی اذا شوشت حالی
\\
الا یا ناعس الطرفین سکری
&&
سل السهران عن طول اللیالی
\\
ندارم چون تو در عالم دگر دوست
&&
اگر چه دوستی دشمن فعالی
\\
کمال الحسن فی الدنیا مصون
&&
کمثل البدر فی حد الکمال
\\
مرکب در وجودم همچو جانی
&&
مصور در دماغم چون خیالی
\\
فما ذالنوم قیل النوم راحه
&&
و مالی النوم فی طول اللیالی
\\
دمی دلداری و صاحب دلی کن
&&
که برخور بادی از صاحب جمالی
\\
الم تنظر الی عینی و دمعی
&&
تری فی البحر اصداف اللآلی
\\
به گوشت گر رسانم ناله زار
&&
ز درد ناله زارم بنالی
\\
لقد کلفت مالم اقو حملا
&&
و مالی حیله غیر احتمالی
\\
که کوته باد چون دست من از دوست
&&
زبان دشمنان از بدسگالی
\\
الا یا سالیا عنی توقف
&&
فما قلب المعنی عنک سال
\\
به چشمانت که گر چه دوری از چشم
&&
دل از یاد تو یک دم نیست خالی
\\
منعت الناس یستسقون غیثا
&&
ان استرسلت دمعا کاللآلی
\\
جهانی تشنگان را دیده در توست
&&
چنین پاکیزه پندارم زلالی
\\
ولی فیک الاراده فوق وصف
&&
ولکن لم تردنی ما احتیالی
\\
چه دستان با تو درگیرد چو روباه
&&
که از مردم گریزان چون غزالی
\\
جرت عینای من ذکراک سیلا
&&
سل الجیران عنی ما جری لی
\\
نمایندت به هم خلقی به انگشت
&&
چو بینند آن دو ابروی هلالی
\\
حفاظی لم یزل مادمت حیا
&&
و لو انتم ضجرتم من وصالی
\\
دلت سخت است و پیمان اندکی سست
&&
دگر در هر چه گویم بر کمالی
\\
اذا کان اقتضاحی فیک حلوا
&&
فقل لی مالعذالی و مالی
\\
مرا با روزگار خویش بگذار
&&
نگیرد سرزنش در لاابالی
\\
ترانی ناظما فی الوجد بیتا
&&
و طرفی ناثر عقد اللآلی
\\
نگویم قامتت زیباست یا چشم
&&
همه لطفی و سرتاسر جمالی
\\
و ان کنتم سئمتم طول مکثی
&&
حوالیکم فقد حان ارتحالی
\\
چو سعدی خاک شد سودی ندارد
&&
اگر خاک وی اندر دیده مالی
\\
\end{longtable}
\end{center}
