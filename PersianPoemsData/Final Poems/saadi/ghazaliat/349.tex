\begin{center}
\section*{غزل ۳۴۹: بی‌دل گمان مبر که نصیحت کند قبول}
\label{sec:349}
\addcontentsline{toc}{section}{\nameref{sec:349}}
\begin{longtable}{l p{0.5cm} r}
بی‌دل گمان مبر که نصیحت کند قبول
&&
من گوش استماع ندارم لمن یقول
\\
تا عقل داشتم نگرفتم طریق عشق
&&
جایی دلم برفت که حیران شود عقول
\\
آخر نه دل به دل رود انصاف من بده
&&
چون است من به وصل تو مشتاق و تو ملول
\\
یک دم نمی‌رود که نه در خاطری ولیک
&&
بسیار فرق باشد از اندیشه تا وصول
\\
روزی سرت ببوسم و در پایت اوفتم
&&
پروانه را چه حاجت پروانه دخول
\\
گنجشک بین که صحبت شاهینش آرزوست
&&
بیچاره در هلاک تن خویشتن عجول
\\
نفسی تزول عاقبة الامر فی الهوی
&&
یا منیتی و ذکرک فی النفس لایزول
\\
ما را به جز تو در همه عالم عزیز نیست
&&
گر رد کنی بضاعت مزجاة ور قبول
\\
ای پیک نامه بر که خبر می‌بری به دوست
&&
یالیت اگر به جای تو من بودمی رسول
\\
دوران دهر و تجربتم سر سپید کرد
&&
وز سر به در نمی‌رودم همچنان فضول
\\
سعدی چو پای بند شدی بار غم ببر
&&
عیار دست بسته نباشد مگر حمول
\\
\end{longtable}
\end{center}
