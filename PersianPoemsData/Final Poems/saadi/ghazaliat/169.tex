\begin{center}
\section*{غزل ۱۶۹: تو را ز حال پریشان ما چه غم دارد}
\label{sec:169}
\addcontentsline{toc}{section}{\nameref{sec:169}}
\begin{longtable}{l p{0.5cm} r}
تو را ز حال پریشان ما چه غم دارد
&&
اگر چراغ بمیرد صبا چه غم دارد
\\
تو را که هر چه مرادست می‌رود از پیش
&&
ز بی مرادی امثال ما چه غم دارد
\\
تو پادشاهی گر چشم پاسبان همه شب
&&
به خواب درنرود پادشا چه غم دارد
\\
خطاست این که دل دوستان بیازاری
&&
ولیک قاتل عمد از خطا چه غم دارد
\\
امیر خوبان آخر گدای خیل توایم
&&
جواب ده که امیر از گدا چه غم دارد
\\
بکی العذول علی ماجری لاجفانی
&&
رفیق غافل از این ماجرا چه غم دارد
\\
هزار دشمن اگر در قفاست عارف را
&&
چو روی خوب تو دید از قفا چه غم دارد
\\
قضا به تلخی و شیرینی ای پسر رفتست
&&
تو گر ترش بنشینی قضا چه غم دارد
\\
بلای عشق عظیمست لاابالی را
&&
چو دل به مرگ نهاد از بلا چه غم دارد
\\
جفا و هر چه توانی بکن که سعدی را
&&
که ترک خویش گرفت از جفا چه غم دارد
\\
\end{longtable}
\end{center}
