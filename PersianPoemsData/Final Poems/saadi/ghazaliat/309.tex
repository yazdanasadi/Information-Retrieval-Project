\begin{center}
\section*{غزل ۳۰۹: ما در این شهر غریبیم و در این ملک فقیر}
\label{sec:309}
\addcontentsline{toc}{section}{\nameref{sec:309}}
\begin{longtable}{l p{0.5cm} r}
ما در این شهر غریبیم و در این ملک فقیر
&&
به کمند تو گرفتار و به دام تو اسیر
\\
در آفاق گشاده‌ست ولیکن بسته‌ست
&&
از سر زلف تو در پای دل ما زنجیر
\\
من نظر بازگرفتن نتوانم همه عمر
&&
از من ای خسرو خوبان تو نظر بازمگیر
\\
گر چه در خیل تو بسیار به از ما باشد
&&
ما تو را در همه عالم نشناسیم نظیر
\\
در دلم بود که جان بر تو فشانم روزی
&&
باز در خاطرم آمد که متاعیست حقیر
\\
این حدیث از سر دردیست که من می‌گویم
&&
تا بر آتش ننهی بوی نیاید ز عبیر
\\
گر بگویم که مرا حال پریشانی نیست
&&
رنگ رخسار خبر می‌دهد از سر ضمیر
\\
عشق پیرانه سر از من عجبت می‌آید
&&
چه جوانی تو که از دست ببردی دل پیر
\\
من از این هر دو کمانخانه ابروی تو چشم
&&
برنگیرم وگرم چشم بدوزند به تیر
\\
عجب از عقل کسانی که مرا پند دهند
&&
برو ای خواجه که عاشق نبود پندپذیر
\\
سعدیا پیکر مطبوع برای نظر است
&&
گر نبینی چه بود فایده چشم بصیر
\\
\end{longtable}
\end{center}
