\begin{center}
\section*{غزل ۶۰: شب فراق که داند که تا سحر چندست}
\label{sec:060}
\addcontentsline{toc}{section}{\nameref{sec:060}}
\begin{longtable}{l p{0.5cm} r}
شب فراق که داند که تا سحر چند است
&&
مگر کسی که به زندان عشق دربند است
\\
گرفتم از غم دل راه بوستان گیرم
&&
کدام سرو به بالای دوست مانند است
\\
پیام من که رساند به یار مهرگسل
&&
که برشکستی و ما را هنوز پیوند است
\\
قسم به جان تو گفتن طریق عزت نیست
&&
به خاک پای تو وان هم عظیم سوگند است
\\
که با شکستن پیمان و برگرفتن دل
&&
هنوز دیده به دیدارت آرزومند است
\\
بیا که بر سر کویت بساط چهرهٔ ماست
&&
به جای خاک که در زیر پایت افکنده‌ست
\\
خیال روی تو بیخ امید بنشانده‌ست
&&
بلای عشق تو بنیاد صبر برکنده‌ست
\\
عجب در آن که تو مجموع و گر قیاس کنی
&&
به زیر هر خم مویت دلی پراکند است
\\
اگر برهنه نباشی که شخص بنمایی
&&
گمان برند که پیراهنت گل آکند است
\\
ز دست رفته نه تنها منم در این سودا
&&
چه دستها که ز دست تو بر خداوند است
\\
فراق یار که پیش تو کاه برگی نیست
&&
بیا و بر دل من بین که کوه الوند است
\\
ز ضعف طاقت آهم نماند و ترسم خلق
&&
گمان برند که سعدی ز دوست خرسند است
\\
\end{longtable}
\end{center}
