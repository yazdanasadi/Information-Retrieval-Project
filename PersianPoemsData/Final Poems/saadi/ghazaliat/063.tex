\begin{center}
\section*{غزل ۶۳: از هر چه می‌رود سخن دوست خوشترست}
\label{sec:063}
\addcontentsline{toc}{section}{\nameref{sec:063}}
\begin{longtable}{l p{0.5cm} r}
از هر چه می‌رود سخن دوست خوشتر است
&&
پیغام آشنا نفس روح پرور است
\\
هرگز وجود حاضر غایب شنیده‌ای
&&
من در میان جمع و دلم جای دیگر است
\\
شاهد که در میان نبود شمع گو بمیر
&&
چون هست اگر چراغ نباشد منور است
\\
ابنای روزگار به صحرا روند و باغ
&&
صحرا و باغ زنده دلان کوی دلبر است
\\
جان می‌روم که در قدم اندازمش ز شوق
&&
درمانده‌ام هنوز که نزلی محقر است
\\
کاش آن به خشم رفتهٔ ما آشتی کنان
&&
بازآمدی که دیدهٔ مشتاق بر در است
\\
جانا دلم چو عود بر آتش بسوختی
&&
وین دم که می‌زنم ز غمت دود مجمر است
\\
شب‌های بی توام شب گور است در خیال
&&
ور بی تو بامداد کنم روز محشر است
\\
گیسوت عنبرینهٔ گردن تمام بود
&&
معشوق خوبروی چه محتاج زیور است
\\
سعدی خیال بیهده بستی امید وصل
&&
هجرت بکشت و وصل هنوزت مصور است
\\
زنهار از این امید درازت که در دل است
&&
هیهات از این خیال محالت که در سر است
\\
\end{longtable}
\end{center}
