\begin{center}
\section*{غزل ۴۸۵: ای طراوت برده از فردوس اعلا روی تو}
\label{sec:485}
\addcontentsline{toc}{section}{\nameref{sec:485}}
\begin{longtable}{l p{0.5cm} r}
ای طراوت برده از فردوس اعلی روی تو
&&
نادر است اندر نگارستان دنیی روی تو
\\
دختران مصر را کاسد شود بازار حسن
&&
گر چو یوسف پرده بردارد به دعوی روی تو
\\
گر چه از انگشت مانی بر نیاید چون تو نقش
&&
هر دم انگشتی نهد بر نقش مانی روی تو
\\
از گل و ماه و پری در چشم من زیباتری
&&
گل ز من دل برد یا مه یا پری نی روی تو
\\
ماه و پروین از خجالت رخ فرو پوشد اگر
&&
آفتاب آسا کند در شب تجلی روی تو
\\
مردم چشمش بدرد پرده اعمی ز شوق
&&
گر درآید در خیال چشم اعمی روی تو
\\
روی هر صاحب جمالی را به مه خواندن خطاست
&&
گر رخی را ماه باید خواند باری روی تو
\\
رسم تقوی می‌نهد در عشقبازی رای من
&&
کوس غارت می‌زند در ملک تقوی روی تو
\\
چون به هر وجهی بخواهد رفت جان از دست ما
&&
خوبتر وجهی بباید جستن اولی روی تو
\\
چشمم از زاری چو فرهاد است و شیرین لعل تو
&&
عقلم از شورش چو مجنون است و لیلی روی تو
\\
ملک زیبایی مسلم گشت فرمان تو را
&&
تا چنین خطی مزور کرد انشی روی تو
\\
داشتند اصحاب خلوت حرف‌ها بر من ز بد
&&
تا تجلی کرد در بازار تقوی روی تو
\\
خرده بر سعدی مگیر ای جان که کاری خرد نیست
&&
سوختن در عشق وانگه ساختن بی روی تو
\\
\end{longtable}
\end{center}
