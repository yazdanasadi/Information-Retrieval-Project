\begin{center}
\section*{غزل ۳۴۵: گرم بازآمدی محبوب سیم اندام سنگین دل}
\label{sec:345}
\addcontentsline{toc}{section}{\nameref{sec:345}}
\begin{longtable}{l p{0.5cm} r}
گرم بازآمدی محبوب سیم اندام سنگین دل
&&
گل از خارم برآوردی و خار از پا و پا از گل
\\
ایا باد سحرگاهی گر این شب روز می‌خواهی
&&
از آن خورشید خرگاهی برافکن دامن محمل
\\
گر او سرپنجه بگشاید که عاشق می‌کشم شاید
&&
هزارش صید پیش آید به خون خویش مستعجل
\\
گروهی همنشین من خلاف عقل و دین من
&&
بگیرند آستین من که دست از دامنش بگسل
\\
ملامتگوی عاشق را چه گوید مردم دانا
&&
که حال غرقه در دریا نداند خفته بر ساحل
\\
به خونم گر بیالاید دو دست نازنین شاید
&&
نه قتلم خوش همی‌آید که دست و پنجه قاتل
\\
اگر عاقل بود داند که مجنون صبر نتواند
&&
شتر جایی بخواباند که لیلی را بود منزل
\\
ز عقل اندیشه‌ها زاید که مردم را بفرساید
&&
گرت آسودگی باید برو عاشق شو ای عاقل
\\
مرا تا پای می‌پوید طریق وصل می‌جوید
&&
بهل تا عقل می‌گوید زهی سودای بی‌حاصل
\\
عجایب نقش‌ها بینی خلاف رومی و چینی
&&
اگر با دوست بنشینی ز دنیا و آخرت غافل
\\
در این معنی سخن باید که جز سعدی نیاراید
&&
که هرچ از جان برون آید نشیند لاجرم بر دل
\\
\end{longtable}
\end{center}
