\begin{center}
\section*{غزل ۴۸۸: ای که شمشیر جفا بر سر ما آخته‌ای}
\label{sec:488}
\addcontentsline{toc}{section}{\nameref{sec:488}}
\begin{longtable}{l p{0.5cm} r}
ای که شمشیر جفا بر سر ما آخته‌ای
&&
دشمن از دوست ندانسته و نشناخته‌ای
\\
من ز فکر تو به خود نیز نمی‌پردازم
&&
نازنینا تو دل از من به که پرداخته‌ای
\\
چند شب‌ها به غم روی تو روز آوردم
&&
که تو یک روز نپرسیده و ننواخته‌ای
\\
گفته بودم که دل از دست تو بیرون آرم
&&
باز دیدم که قوی پنجه درانداخته‌ای
\\
تا شکاری ز کمند سر زلفت نجهد
&&
ز ابروان و مژه‌ها تیر و کمان ساخته‌ای
\\
لاجرم صید دلی در همه شیراز نماند
&&
که نه با تیر و کمان در پی او تاخته‌ای
\\
ماه و خورشید و پری و آدمی اندر نظرت
&&
همه هیچند که سر بر همه افراخته‌ای
\\
با همه جلوه طاووس و خرامیدن کبک
&&
عیبت آن است که بی مهرتر از فاخته‌ای
\\
هر که می‌بیندم از جور غمت می‌گوید
&&
سعدیا بر تو چه رنج است که بگداخته‌ای
\\
بیم مات است در این بازی بیهوده مرا
&&
چه کنم دست تو بردی که دغل باخته‌ای
\\
\end{longtable}
\end{center}
