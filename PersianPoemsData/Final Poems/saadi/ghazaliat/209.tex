\begin{center}
\section*{غزل ۲۰۹: تا کی ای دلبر دل من بار تنهایی کشد}
\label{sec:209}
\addcontentsline{toc}{section}{\nameref{sec:209}}
\begin{longtable}{l p{0.5cm} r}
تا کی ای دلبر دل من بار تنهایی کشد
&&
ترسم از تنهایی احوالم به رسوایی کشد
\\
کی شکیبایی توان کردن چو عقل از دست رفت
&&
عاقلی باید که پای اندر شکیبایی کشد
\\
سروبالای منا گر چون گل آیی در چمن
&&
خاک پایت نرگس اندر چشم بینایی کشد
\\
روی تاجیکانه‌ات بنمای تا داغ حبش
&&
آسمان بر چهره ترکان یغمایی کشد
\\
شهد ریزی چون دهانت دم به شیرینی زند
&&
فتنه انگیزی چو زلفت سر به رعنایی کشد
\\
دل نماند بعد از این با کس که گر خود آهنست
&&
ساحر چشمت به مغناطیس زیبایی کشد
\\
خود هنوزت پسته خندان عقیقین نقطه‌ایست
&&
باش تا گردش قضا پرگار مینایی کشد
\\
سعدیا دم درکش ار دیوانه خوانندت که عشق
&&
گر چه از صاحب دلی خیزد به شیدایی کشد
\\
\end{longtable}
\end{center}
