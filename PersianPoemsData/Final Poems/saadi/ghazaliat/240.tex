\begin{center}
\section*{غزل ۲۴۰: چه کند بنده که بر جور تحمل نکند}
\label{sec:240}
\addcontentsline{toc}{section}{\nameref{sec:240}}
\begin{longtable}{l p{0.5cm} r}
چه کند بنده که بر جور تحمل نکند
&&
دل اگر تنگ شود مهر تبدل نکند
\\
دل و دین در سر کارت شد و بسیاری نیست
&&
سر و جان خواه که دیوانه تأمل نکند
\\
سحر گویند حرامست در این عهد ولیک
&&
چشمت آن کرد که هاروت به بابل نکند
\\
غرقه در بحر عمیق تو چنان بی‌خبرم
&&
که مبادا که چه دریام (؟) به ساحل نکند
\\
به گلستان نروم تا تو در آغوش منی
&&
بلبل ار روی تو بیند طلب گل نکند
\\
هر که با دوست چو سعدی نفسی خوش دریافت
&&
چیز و کس در نظرش باز تخیل نکند
\\
\end{longtable}
\end{center}
