\begin{center}
\section*{غزل ۲۱۶: روز برآمد بلند ای پسر هوشمند}
\label{sec:216}
\addcontentsline{toc}{section}{\nameref{sec:216}}
\begin{longtable}{l p{0.5cm} r}
روز برآمد بلند ای پسر هوشمند
&&
گرم ببود آفتاب خیمه به رویش ببند
\\
طفل گیا شیر خورد شاخ جوان گو ببال
&&
ابر بهاری گریست طرف چمن گو بخند
\\
تا به تماشای باغ میل چرا می‌کند
&&
هر که به خیلش درست قامت سرو بلند
\\
عقل روا می‌نداشت گفتن اسرار عشق
&&
قوت بازوی شوق بیخ صبوری بکند
\\
دل که بیابان گرفت چشم ندارد به راه
&&
سر که صراحی کشید گوش ندارد به پند
\\
کشته شمشیر عشق حال نگوید که چون
&&
تشنه دیدار دوست راه نپرسد که چند
\\
هر که پسند آمدش چون تو یکی در نظر
&&
بس که بخواهد شنید سرزنش ناپسند
\\
در نظر دشمنان نوش نباشد هنی
&&
وز قبل دوستان نیش نباشد گزند
\\
این که سرش در کمند جان به دهانش رسید
&&
می‌نکند التفات آن که به دستش کمند
\\
سعدی اگر عاقلی عشق طریق تو نیست
&&
با کف زورآزمای پنجه نشاید فکند
\\
\end{longtable}
\end{center}
