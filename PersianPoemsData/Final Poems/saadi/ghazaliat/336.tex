\begin{center}
\section*{غزل ۳۳۶: رفتی و نمی‌شوی فراموش}
\label{sec:336}
\addcontentsline{toc}{section}{\nameref{sec:336}}
\begin{longtable}{l p{0.5cm} r}
رفتی و نمی‌شوی فراموش
&&
می‌آیی و می‌روم من از هوش
\\
سحر است کمان ابروانت
&&
پیوسته کشیده تا بناگوش
\\
پایت بگذار تا ببوسم
&&
چون دست نمی‌رسد به آغوش
\\
جور از قبلت مقام عدل است
&&
نیش سخنت مقابل نوش
\\
بی‌کار بود که در بهاران
&&
گویند به عندلیب مخروش
\\
دوش آن غم دل که می‌نهفتم
&&
باد سحرش ببرد سرپوش
\\
آن سیل که دوش تا کمر بود
&&
امشب بگذشت خواهد از دوش
\\
شهری متحدثان حسنت
&&
الا متحیران خاموش
\\
بنشین که هزار فتنه برخاست
&&
از حلقه عارفان مدهوش
\\
آتش که تو می‌کنی محال است
&&
کاین دیگ فرونشیند از جوش
\\
بلبل که به دست شاهد افتاد
&&
یاران چمن کند فراموش
\\
ای خواجه برو به هر چه داری
&&
یاری بخر و به هیچ مفروش
\\
گر توبه دهد کسی ز عشقت
&&
از من بنیوش و پند منیوش
\\
سعدی همه ساله پند مردم
&&
می‌گوید و خود نمی‌کند گوش
\\
\end{longtable}
\end{center}
