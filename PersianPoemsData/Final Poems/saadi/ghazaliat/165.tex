\begin{center}
\section*{غزل ۱۶۵: که می‌رود به شفاعت که دوست بازآرد}
\label{sec:165}
\addcontentsline{toc}{section}{\nameref{sec:165}}
\begin{longtable}{l p{0.5cm} r}
که می‌رود به شفاعت که دوست بازآرد
&&
که عیش خلوت بی او کدورتی دارد
\\
که را مجال سخن گفتنست به حضرت او
&&
مگر نسیم صبا کاین پیام بگزارد
\\
ستیزه بردن با دوستان همین مثلست
&&
که تشنه چشمه حیوان به گل بینبارد
\\
مرا که گفت دل از یار مهربان بردار
&&
به اعتماد صبوری که شوق نگذارد
\\
که گفت هر چه ببینی ز خاطرت برود
&&
مرا تمام یقین شد که سهو پندارد
\\
حرام باد بر آن کس نشست با معشوق
&&
که از سر همه برخاستن نمی‌یارد
\\
درست ناید از آن مدعی حقیقت عشق
&&
که در مواجهه تیغش زنند و سر خارد
\\
به کام دشمنم ای دوست این چنین مگذار
&&
کس این کند که دل دوستان بیازارد
\\
بیا که در قدمت اوفتم و گر بکشی
&&
نمیرد آن که به دست تو روح بسپارد
\\
حکایت شب هجران که بازداند گفت
&&
مگر کسی که چو سعدی ستاره بشمارد
\\
\end{longtable}
\end{center}
