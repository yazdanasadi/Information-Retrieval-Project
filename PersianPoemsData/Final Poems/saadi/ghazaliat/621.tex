\begin{center}
\section*{غزل ۶۲۱: سرو ایستاده به چو تو رفتار می‌کنی}
\label{sec:621}
\addcontentsline{toc}{section}{\nameref{sec:621}}
\begin{longtable}{l p{0.5cm} r}
سرو ایستاده به چو تو رفتار می‌کنی
&&
طوطی خموش به چو تو گفتار می‌کنی
\\
کس دل به اختیار به مهرت نمی‌دهد
&&
دامی نهاده‌ای که گرفتار می‌کنی
\\
تو خود چه فتنه‌ای که به چشمان ترک مست
&&
تاراج عقل مردم هشیار می‌کنی
\\
از دوستی که دارم و غیرت که می‌برم
&&
خشم آیدم که چشم به اغیار می‌کنی
\\
گفتی نظر خطاست تو دل می‌بری رواست
&&
خود کرده جرم و خلق گنهکارمی‌کنی
\\
هرگز فرامشت نشود دفتر خلاف
&&
با دوستان چنین که تو تکرار می‌کنی
\\
دستان به خون تازه بیچارگان خضاب
&&
هرگز کس این کند که تو عیار می‌کنی
\\
با دشمنان موافق و با دوستان به خشم
&&
یاری نباشد این که تو با یار می‌کنی
\\
تا من سماع می‌شنوم پند نشنوم
&&
ای مدعی نصیحت بی‌کار می‌کنی
\\
گر تیغ می‌زنی سپر اینک وجود من
&&
صلح است از این طرف که تو پیکار می‌کنی
\\
از روی دوست تا نکنی رو به آفتاب
&&
کز آفتاب روی به دیوار می‌کنی
\\
زنهار سعدی از دل سنگین کافرش
&&
کافر چه غم خورد چو تو زنهار می‌کنی
\\
\end{longtable}
\end{center}
