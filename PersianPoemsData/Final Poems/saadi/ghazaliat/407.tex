\begin{center}
\section*{غزل ۴۰۷: تا تو به خاطر منی کس نگذشت بر دلم}
\label{sec:407}
\addcontentsline{toc}{section}{\nameref{sec:407}}
\begin{longtable}{l p{0.5cm} r}
تا تو به خاطر منی کس نگذشت بر دلم
&&
مثل تو کیست در جهان تا ز تو مهر بگسلم
\\
من چو به آخرت روم رفته به داغ دوستی
&&
داروی دوستی بود هر چه بروید از گلم
\\
میرم و همچنان رود نام تو بر زبان من
&&
ریزم و همچنان بود مهر تو در مفاصلم
\\
حاصل عمر صرف شد در طلب وصال تو
&&
با همه سعی اگر به خود ره ندهی چه حاصلم
\\
باد به دست آرزو در طلب هوای دل
&&
گر نکند معاونت دور زمان مقبلم
\\
لایق بندگی نیم بی هنری و قیمتی
&&
ور تو قبول می‌کنی با همه نقص فاضلم
\\
مثل تو را به خون من ور بکشی به باطلم
&&
کس نکند مطالبت زان که غلام قاتلم
\\
کشتی من که در میان آب گرفت و غرق شد
&&
گر بود استخوان برد باد صبا به ساحلم
\\
سرو برفت و بوستان از نظرم به جملگی
&&
می‌نرود صنوبری بیخ گرفته در دلم
\\
فکرت من کجا رسد در طلب وصال تو
&&
این همه یاد می‌رود وز تو هنوز غافلم
\\
لشکر عشق سعدیا غارت عقل می‌کند
&&
تا تو دگر به خویشتن ظن نبری که عاقلم
\\
\end{longtable}
\end{center}
