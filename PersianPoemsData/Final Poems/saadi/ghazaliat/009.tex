\begin{center}
\section*{غزل ۹: گر ماه من برافکند از رخ نقاب را}
\label{sec:009}
\addcontentsline{toc}{section}{\nameref{sec:009}}
\begin{longtable}{l p{0.5cm} r}
گر ماه من برافکند از رخ نقاب را
&&
برقع فروهلد به جمال آفتاب را
\\
گویی دو چشم جادوی عابدفریب او
&&
بر چشم من به سحر ببستند خواب را
\\
اول نظر ز دست برفتم عنان عقل
&&
وان را که عقل رفت چه داند صواب را
\\
گفتم مگر به وصل رهایی بود ز عشق
&&
بی‌حاصل است خوردن مستسقی آب را
\\
دعوی درست نیست گر از دست نازنین
&&
چون شربت شکر نخوری زهر ناب را
\\
عشق آدمیت است گر این ذوق در تو نیست
&&
همشرکتی به خوردن و خفتن دواب را
\\
آتش بیار و خرمن آزادگان بسوز
&&
تا پادشه خراج نخواهد خراب را
\\
قوم از شراب مست و ز منظور بی‌نصیب
&&
من مست از او چنان که نخواهم شراب را
\\
سعدی نگفتمت که مرو در کمند عشق
&&
تیر نظر بیفکند افراسیاب را
\\
\end{longtable}
\end{center}
