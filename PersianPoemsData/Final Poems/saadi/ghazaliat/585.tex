\begin{center}
\section*{غزل ۵۸۵: اگر تو پرده بر این زلف و رخ نمی‌پوشی}
\label{sec:585}
\addcontentsline{toc}{section}{\nameref{sec:585}}
\begin{longtable}{l p{0.5cm} r}
اگر تو پرده بر این زلف و رخ نمی‌پوشی
&&
به هتک پرده صاحب دلان همی‌کوشی
\\
چنین قیامت و قامت ندیده‌ام همه عمر
&&
تو سرو یا بدنی شمس یا بناگوشی
\\
غلام حلقه سیمین گوشوار توام
&&
که پادشاه غلامان حلقه در گوشی
\\
به کنج خلوت پاکان و پارسایان آی
&&
نظاره کن که چه مستی کنند و مدهوشی
\\
به روزگار عزیزان که یاد می‌کنمت
&&
علی الدوام نه یادی پس از فراموشی
\\
چنان موافق طبع منی و در دل من
&&
نشسته‌ای که گمان می‌برم در آغوشی
\\
چه نیکبخت کسانی که با تو هم سخنند
&&
مرا نه زهره گفت و نه صبر خاموشی
\\
رقیب نامتناسب چه اهل صحبت توست
&&
که طبع او همه نیش و تو سر به سر نوشی
\\
به تربیت به چمن گفتم ای نسیم صبا
&&
بگوی تا ندهد گل به خار چاووشی
\\
تو سوز سینه مستان ندیدی ای هشیار
&&
چو آتشیت نباشد چگونه برجوشی
\\
تو را که دل نبود عاشقی چه دانی چیست
&&
تو را که سمع نباشد سماع ننیوشی
\\
وفای یار به دنیا و دین مده سعدی
&&
دریغ باشد یوسف به هر چه بفروشی
\\
\end{longtable}
\end{center}
