\begin{center}
\section*{غزل ۴۶۲: طوطی نگوید از تو دلاویزتر سخن}
\label{sec:462}
\addcontentsline{toc}{section}{\nameref{sec:462}}
\begin{longtable}{l p{0.5cm} r}
طوطی نگوید از تو دلاویزتر سخن
&&
با شهد می‌رود ز دهانت به در سخن
\\
گر من نگویمت که تو شیرین عالمی
&&
تو خویشتن دلیل بیاری به هر سخن
\\
واجب بود که بر سخنت آفرین کنند
&&
لیکن مجال گفت نباشد تو در سخن
\\
در هیچ بوستان چو تو سروی نیامده‌ست
&&
بادام چشم و پسته دهان و شکرسخن
\\
هرگز شنیده‌ای ز بن سرو بوی مشک؟
&&
یا گوش کرده‌ای ز دهان قمر سخن؟
\\
انصاف نیست پیش تو گفتن حدیث خویش
&&
من عهد می‌کنم که نگویم دگر سخن
\\
چشمان دلبرت به نظر سحر می‌کنند
&&
من خود چگونه گویمت اندر نظر سخن
\\
ای باد اگر مجال سخن گفتنت بود
&&
در گوش آن ملول بگوی این قدر سخن
\\
وصفی چنان که لایق حسنت نمی‌رود
&&
آشفته حال را نبود معتبر سخن
\\
در می‌چکد ز منطق سعدی به جای شعر
&&
گر سیم داشتی بنوشتی به زر سخن
\\
دانندش اهل فضل که مسکین غریق بود
&&
هر گه که در سفینه ببینند تر سخن
\\
\end{longtable}
\end{center}
