\begin{center}
\section*{غزل ۲۱۸: آن سرو که گویند به بالای تو ماند}
\label{sec:218}
\addcontentsline{toc}{section}{\nameref{sec:218}}
\begin{longtable}{l p{0.5cm} r}
آن سرو که گویند به بالای تو ماند
&&
هرگز قدمی پیش تو رفتن نتواند
\\
دنبال تو بودن گنه از جانب ما نیست
&&
با غمزه بگو تا دل مردم نستاند
\\
زنهار که چون می‌گذری بر سر مجروح
&&
وز وی خبرت نیست که چون می‌گذراند
\\
بخت آن نکند با من سرگشته که یک روز
&&
همخانه من باشی و همسایه نداند
\\
هر کاو سر پیوند تو دارد به حقیقت
&&
دست از همه چیز و همه کس درگسلاند
\\
امروز چه دانی تو که در آتش و آبم
&&
چون خاک شوم باد به گوشت برساند
\\
آنان که ندانند پریشانی مشتاق
&&
گویند که نالیدن بلبل به چه ماند
\\
گل را همه کس دست گرفتند و نخوانند
&&
بلبل نتوانست که فریاد نخواند
\\
هر ساعتی این فتنه نوخاسته از جای
&&
برخیزد و خلقی متحیر بنشاند
\\
در حسرت آنم که سر و مال به یک بار
&&
در دامنش افشانم و دامن نفشاند
\\
سعدی تو در این بند بمیری و نداند
&&
فریاد بکن یا بکشد یا برهاند
\\
\end{longtable}
\end{center}
