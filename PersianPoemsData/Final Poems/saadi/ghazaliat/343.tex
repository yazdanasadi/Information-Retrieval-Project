\begin{center}
\section*{غزل ۳۴۳: نخواند بر گل رویت چه جای بلبل باغ}
\label{sec:343}
\addcontentsline{toc}{section}{\nameref{sec:343}}
\begin{longtable}{l p{0.5cm} r}
به عمر خویش ندیدم شبی که مرغ دلم
&&
نخواند بر گل رویت چه جای بلبل باغ
\\
تو را فراغت ما گر بود و گر نبود
&&
مرا به روی تو از هر که عالمست فراغ
\\
ز درد عشق تو امید رستگاری نیست
&&
گریختن نتوانند بندگان به داغ
\\
تو را که این همه بلبل نوای عشق زنند
&&
چه التفات بود بر ادای منکر زاغ
\\
دلیل روی تو هم روی توست سعدی را
&&
چراغ را نتوان دید جز به نور چراغ
\\
\end{longtable}
\end{center}
