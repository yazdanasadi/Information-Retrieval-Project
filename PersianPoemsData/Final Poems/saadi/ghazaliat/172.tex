\begin{center}
\section*{غزل ۱۷۲: هر آن ناظر که منظوری ندارد}
\label{sec:172}
\addcontentsline{toc}{section}{\nameref{sec:172}}
\begin{longtable}{l p{0.5cm} r}
هر آن ناظر که منظوری ندارد
&&
چراغ دولتش نوری ندارد
\\
چه کار اندر بهشت آن مدعی را
&&
که میل امروز با حوری ندارد
\\
چه ذوق از ذکر پیدا آید آن را
&&
که پنهان شوق مذکوری ندارد
\\
میان عارفان صاحب نظر نیست
&&
که خاطر پیش منظوری ندارد
\\
اگر سیمرغی اندر دام زلفی
&&
بماند تاب عصفوری ندارد
\\
طبیب ما یکی نامهربانست
&&
که گویی هیچ رنجوری ندارد
\\
ولیکن چون عسل بشناخت سعدی
&&
فغان از دست زنبوری ندارد
\\
\end{longtable}
\end{center}
