\begin{center}
\section*{غزل ۲۵۹: ناچار هر که صاحب روی نکو بود}
\label{sec:259}
\addcontentsline{toc}{section}{\nameref{sec:259}}
\begin{longtable}{l p{0.5cm} r}
ناچار هر که صاحب روی نکو بود
&&
هر جا که بگذرد همه چشمی در او بود
\\
ای گل تو نیز شوخی بلبل معاف دار
&&
کان جا که رنگ و بوی بود گفت و گو بود
\\
نفس آرزو کند که تو لب بر لبش نهی
&&
بعد از هزار سال که خاکش سبو بود
\\
پاکیزه روی در همه شهری بود ولیک
&&
نه چون تو پاکدامن و پاکیزه خو بود
\\
ای گوی حسن برده ز خوبان روزگار
&&
مسکین کسی که در خم چوگان چو گو بود
\\
مویی چنین دریغ نباشد گره زدن
&&
بگذار تا کنار و برت مشک بو بود
\\
پندارم آن که با تو ندارد تعلقی
&&
نه آدمی که صورتی از سنگ و رو بود
\\
من باری از تو بر نتوانم گرفت چشم
&&
گم کرده دل هرآینه در جست و جو بود
\\
بر می‌نیاید از دل تنگم نفس تمام
&&
چون ناله کسی که به چاهی فرو بود
\\
سعدی سپاس دار و جفا بین و دم مزن
&&
کز دست نیکوان همه چیزی نکو بود
\\
\end{longtable}
\end{center}
