\begin{center}
\section*{غزل ۱۶۸: کس این کند که دل از یار خویش بردارد}
\label{sec:168}
\addcontentsline{toc}{section}{\nameref{sec:168}}
\begin{longtable}{l p{0.5cm} r}
کس این کند که دل از یار خویش بردارد
&&
مگر کسی که دل از سنگ سختتر دارد
\\
که گفت من خبری دارم از حقیقت عشق
&&
دروغ گفت گر از خویشتن خبر دارد
\\
اگر نظر به دو عالم کند حرامش باد
&&
که از صفای درون با یکی نظر دارد
\\
هلاک ما به بیابان عشق خواهد بود
&&
کجاست مرد که با ما سر سفر دارد
\\
گر از مقابله شیر آید از عقب شمشیر
&&
نه عاشقست که اندیشه از خطر دارد
\\
و گر بهشت مصور کنند عارف را
&&
به غیر دوست نشاید که دیده بردارد
\\
از آن متاع که در پای دوستان ریزند
&&
مرا سریست ندانم که او چه سر دارد
\\
دریغ پای که بر خاک می‌نهد معشوق
&&
چرا نه بر سر و بر چشم ما گذر دارد
\\
عوام عیب کنندم که عاشقی همه عمر
&&
کدام عیب که سعدی خود این هنر دارد
\\
نظر به روی تو انداختن حرامش باد
&&
که جز تو در همه عالم کسی دگر دارد
\\
\end{longtable}
\end{center}
