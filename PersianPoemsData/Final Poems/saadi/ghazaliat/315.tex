\begin{center}
\section*{غزل ۳۱۵: پیوند روح می‌کند این باد مشک بیز}
\label{sec:315}
\addcontentsline{toc}{section}{\nameref{sec:315}}
\begin{longtable}{l p{0.5cm} r}
پیوند روح می‌کند این باد مشک بیز
&&
هنگام نوبت سحرست ای ندیم خیز
\\
شاهد بخوان و شمع بیفروز و می بنه
&&
عنبر بسای و عود بسوزان و گل بریز
\\
ور دوست دست می‌دهدت هیچ گو مباش
&&
خوشتر بود عروس نکوروی بی جهاز
\\
امروز باید ار کرمی می‌کند سحاب
&&
فردا که تشنه مرده بود لای گو بخیز
\\
من در وفا و عهد چنان کند نیستم
&&
کز دامن تو دست بدارم به تیغ تیز
\\
گر تیغ می‌زنی سپر اینک وجود من
&&
عیار مدعی کند از دشمن احتراز
\\
فردا که سر ز خاک برآرم اگر تو را
&&
بینم فراغتم بود از روز رستخیز
\\
تا خود کجا رسد به قیامت نماز من
&&
من روی در تو و همه کس روی در حجاز
\\
سعدی به دام عشق تو در پای بند ماند
&&
قیدی نکرده‌ای که میسر شود گریز
\\
\end{longtable}
\end{center}
