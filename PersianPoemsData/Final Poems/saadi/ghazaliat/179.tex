\begin{center}
\section*{غزل ۱۷۹: کیست آن ماه منور که چنین می‌گذرد}
\label{sec:179}
\addcontentsline{toc}{section}{\nameref{sec:179}}
\begin{longtable}{l p{0.5cm} r}
کیست آن ماه منور که چنین می‌گذرد
&&
تشنه جان می‌دهد و ماء معین می‌گذرد
\\
سرو اگر نیز تحول کند از جای به جای
&&
نتوان گفت که زیباتر از این می‌گذرد
\\
حور عین می‌گذرد در نظر سوختگان
&&
یا مه چارده یا لعبت چین می‌گذرد
\\
کام از او کس نگرفتست مگر باد بهار
&&
که بر آن زلف و بناگوش و جبین می‌گذرد
\\
مردم زیر زمین رفتن او پندارند
&&
کآفتابست که بر اوج برین می‌گذرد
\\
پای گو بر سر عاشق نه و بر دیده دوست
&&
حیف باشد که چنین کس به زمین می‌گذرد
\\
هر که در شهر دلی دارد و دینی دارد
&&
گو حذر کن که هلاک دل و دین می‌گذرد
\\
از خیال آمدن و رفتنش اندر دل و چشم
&&
با گمان افتم و گر خود به یقین می‌گذرد
\\
گر کند روی به ما یا نکند حکم او راست
&&
پادشاهیست که بر ملک یمین می‌گذرد
\\
سعدیا گوشه نشینی کن و شاهدبازی
&&
شاهد آنست که بر گوشه نشین می‌گذرد
\\
\end{longtable}
\end{center}
