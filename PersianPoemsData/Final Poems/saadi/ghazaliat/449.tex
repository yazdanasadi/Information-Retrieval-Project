\begin{center}
\section*{غزل ۴۴۹: چه خوشست بوی عشق از نفس نیازمندان}
\label{sec:449}
\addcontentsline{toc}{section}{\nameref{sec:449}}
\begin{longtable}{l p{0.5cm} r}
چه خوش است بوی عشق از نفس نیازمندان
&&
دل از انتظار خونین دهن از امید خندان
\\
مگر آن که هر دو چشمش همه عمر بسته باشد
&&
به ورع خلاص یابد ز فریب چشم بندان
\\
نظری مباح کردند و هزار خون معطل
&&
دل عارفان ببردند و قرار هوشمندان
\\
سر کوی ماه رویان همه روز فتنه باشد
&&
ز معربدان و مستان و معاشران و رندان
\\
اگر از کمند عشقت بروم کجا گریزم
&&
که خلاص بی تو بند است و حیات بی تو زندان
\\
اگرم نمی‌پسندی مدهم به دست دشمن
&&
که من از تو برنگردم به جفای ناپسندان
\\
نفسی بیا و بنشین سخنی بگوی و بشنو
&&
که قیامت است چندان سخن از دهان خندان
\\
اگر این شکر ببینند محدثان شیرین
&&
همه دست‌ها بخایند چو نیشکر به دندان
\\
همه شاهدان عالم به تو عاشقند سعدی
&&
که میان گرگ صلح است و میان گوسفندان
\\
\end{longtable}
\end{center}
