\begin{center}
\section*{غزل ۵۶۲: دو چشم مست تو برداشت رسم هشیاری}
\label{sec:562}
\addcontentsline{toc}{section}{\nameref{sec:562}}
\begin{longtable}{l p{0.5cm} r}
دو چشم مست تو برداشت رسم هشیاری
&&
و گر نه فتنه ندیدی به خواب بیداری
\\
زمانه با تو چه دعوی کند به بدمهری
&&
سپهر با تو چه پهلو زند به غداری
\\
معلمت همه شوخی و دلبری آموخت
&&
به دوستیت وصیت نکرد و دلداری
\\
چو گل لطیف ولیکن حریف اوباشی
&&
چو زر عزیز ولیکن به دست اغیاری
\\
به صید کردن دل‌ها چه شوخ و شیرینی
&&
به خیره کشتن تن‌ها چه جلد و عیاری
\\
دلم ربودی و جان می‌دهم به طیبت نفس
&&
که هست راحت درویش در سبکباری
\\
گر افتدت گذری بر وجود کشته عشق
&&
سخن بگوی که در جسم مرده جان آری
\\
گرت ارادت باشد به شورش دل خلق
&&
بشور زلف که در هر خمی دلی داری
\\
چو بت به کعبه نگونسار بر زمین افتد
&&
به پیش قبله رویت بتان فرخاری
\\
دهان پر شکرت را مثل به نقطه زنند
&&
که روی چون قمرت شمسه‌ایست پرگاری
\\
به گرد نقطه سرخت عذار سبز چنان
&&
که نیم دایره‌ای برکشند زنگاری
\\
هزار نامه پیاپی نویسمت که جواب
&&
اگر چه تلخ دهی در سخن شکرباری
\\
ز خلق گوی لطافت تو برده‌ای امروز
&&
به خوبرویی و سعدی به خوب گفتاری
\\
\end{longtable}
\end{center}
