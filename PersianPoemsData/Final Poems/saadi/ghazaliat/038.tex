\begin{center}
\section*{غزل ۳۸: چه دل‌ها بردی ای ساقی به ساق فتنه انگیزت}
\label{sec:038}
\addcontentsline{toc}{section}{\nameref{sec:038}}
\begin{longtable}{l p{0.5cm} r}
چه دل‌ها بردی ای ساقی به ساق فتنه‌انگیزت
&&
دریغا بوسه چندی بر زنخدان دلاویزت
\\
خدنگ غمزه از هر سو نهان انداختن تا کی
&&
سپر انداخت عقل از دست ناوک‌های خونریزت
\\
برآمیزی و بگریزی و بنمایی و بربایی
&&
فغان از قهر لطف اندود و زهر شکرآمیزت
\\
لب شیرینت ار شیرین بدیدی در سخن گفتن
&&
بر او شکرانه بودی گر بدادی ملک پرویزت
\\
جهان از فتنه و آشوب یک چندی برآسودی
&&
اگر نه روی شهرآشوب و چشم فتنه‌انگیزت
\\
دگر رغبت کجا ماند کسی را سوی هشیاری
&&
چو بیند دست در آغوش مستان سحرخیزت
\\
دمادم درکش ای سعدی شراب صرف و دم درکش
&&
که با مستان مجلس درنگیرد زهد و پرهیزت
\\
\end{longtable}
\end{center}
