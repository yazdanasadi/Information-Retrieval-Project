\begin{center}
\section*{غزل ۴۶۶: تا کی ای جان اثر وصل تو نتوان دیدن}
\label{sec:466}
\addcontentsline{toc}{section}{\nameref{sec:466}}
\begin{longtable}{l p{0.5cm} r}
تا کی ای جان اثر وصل تو نتوان دیدن
&&
که ندارد دل من طاقت هجران دیدن
\\
بر سر کوی تو گر خوی تو این خواهد بود
&&
دل نهادم به جفاهای فراوان دیدن
\\
عقل بی خویشتن از عشق تو دیدن تا چند
&&
خویشتن بی‌دل و دل بی سر و سامان دیدن
\\
تن به زیر قدمت خاک توان کرد ولیک
&&
گرد بر گوشه نعلین تو نتوان دیدن
\\
هر شبم زلف سیاه تو نمایند به خواب
&&
تا چه آید به من از خواب پریشان دیدن
\\
با وجود رخ و بالای تو کوته نظریست
&&
در گلستان شدن و سرو خرامان دیدن
\\
گر بر این چاه زنخدان تو ره بردی خضر
&&
بی نیاز آمدی از چشمه حیوان دیدن
\\
هر دل سوخته کاندر خم زلف تو فتاد
&&
گوی از آن به نتوان در خم چوگان دیدن
\\
آن چه از نرگس مخمور تو در چشم من است
&&
برنخیزد به گل و لاله و ریحان دیدن
\\
سعدیا حسرت بیهوده مخور دانی چیست
&&
چاره کار تو جان دادن و جانان دیدن
\\
\end{longtable}
\end{center}
