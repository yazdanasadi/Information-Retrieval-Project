\begin{center}
\section*{غزل ۴۱۰: چشم که بر تو می‌کنم چشم حسود می‌کنم}
\label{sec:410}
\addcontentsline{toc}{section}{\nameref{sec:410}}
\begin{longtable}{l p{0.5cm} r}
چشم که بر تو می‌کنم چشم حسود می‌کنم
&&
شکر خدا که باز شد دیده بخت روشنم
\\
هرگزم این گمان نبد با تو که دوستی کنم
&&
باورم این نمی‌شود با تو نشسته کاین منم
\\
دامن خیمه برفکن دشمن و دوست گو ببین
&&
کاین همه لطف می‌کند دوست به رغم دشمنم
\\
عالم شهر گو مرا وعظ مگو که نشنوم
&&
پیر محله گو مرا توبه مده که بشکنم
\\
گر بزنی به خنجرم کز پی او دگر مرو
&&
نعره شوق می‌زنم تا رمقیست در تنم
\\
این نه نصیحتی بود کز غم دوست توبه کن
&&
سخت سیه دلی بود آن که ز دوست برکنم
\\
گر همه عمر بشکنم عهد تو پس درست شد
&&
کاین همه ذکر دوستی لاف دروغ می‌زنم
\\
پیشم از این سلامتی بود و دلی و دانشی
&&
عشق تو آتشی بزد پاک بسوخت خرمنم
\\
شهری اگر به قصد من جمع شوند و متفق
&&
با همه تیغ برکشم وز تو سپر بیفکنم
\\
چند فشانی آستین بر من و روزگار من
&&
دست رها نمی‌کند مهر گرفته دامنم
\\
گر به مراد من روی ور نروی تو حاکمی
&&
من به خلاف رای تو گر نفسی زنم زنم
\\
این همه نیش می‌خورد سعدی و پیش می‌رود
&&
خون برود در این میان گر تو تویی و من منم
\\
\end{longtable}
\end{center}
