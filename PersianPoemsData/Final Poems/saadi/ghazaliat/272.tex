\begin{center}
\section*{غزل ۲۷۲: آن که نقشی دیگرش جایی مصور می‌شود}
\label{sec:272}
\addcontentsline{toc}{section}{\nameref{sec:272}}
\begin{longtable}{l p{0.5cm} r}
آن که نقشی دیگرش جایی مصور می‌شود
&&
نقش او در چشم ما هر روز خوشتر می‌شود
\\
عشق دانی چیست سلطانی که هر جا خیمه زد
&&
بی خلاف آن مملکت بر وی مقرر می‌شود
\\
دیگران را تلخ می‌آید شراب جور عشق
&&
ما ز دست دوست می‌گیریم و شکر می‌شود
\\
دل ز جان برگیر و در بر گیر یار مهربان
&&
گر بدین مقدارت آن دولت میسر می‌شود
\\
هرگزم در سر نبود اندیشه سودا ولیک
&&
پیل اگر دربند می‌افتد مسخر می‌شود
\\
عیش‌ها دارم در این آتش که بینی دم به دم
&&
کاندرونم گر چه می‌سوزد منور می‌شود
\\
تا نپنداری که با دیگر کسم خاطر خوشست
&&
ظاهرم با جمع و خاطر جای دیگر می‌شود
\\
غیرتم گوید نگویم با حریفان راز خویش
&&
باز می‌بینم که در آفاق دفتر می‌شود
\\
آب شوق از چشم سعدی می‌رود بر دست و خط
&&
لاجرم چون شعر می‌آید سخن تر می‌شود
\\
قول مطبوع از درون سوزناک آید که عود
&&
چون همی‌سوزد جهان از وی معطر می‌شود
\\
\end{longtable}
\end{center}
