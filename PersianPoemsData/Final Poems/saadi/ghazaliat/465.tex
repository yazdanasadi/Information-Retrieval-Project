\begin{center}
\section*{غزل ۴۶۵: میان باغ حرامست بی تو گردیدن}
\label{sec:465}
\addcontentsline{toc}{section}{\nameref{sec:465}}
\begin{longtable}{l p{0.5cm} r}
میان باغ حرام است بی تو گردیدن
&&
که خار با تو مرا به که بی تو گل چیدن
\\
و گر به جام برم بی تو دست در مجلس
&&
حرام صرف بود بی تو باده نوشیدن
\\
خم دو زلف تو بر لاله حلقه در حلقه
&&
به سنگ خاره درآموخت عشق ورزیدن
\\
اگر جماعت چین صورت تو بت بینند
&&
شوند جمله پشیمان ز بت پرستیدن
\\
کساد نرخ شکر در جهان پدید آید
&&
دهان چو بازگشایی به وقت خندیدن
\\
به جای خشک بمانند سروهای چمن
&&
چو قامت تو ببینند در خرامیدن
\\
من گدای که باشم که دم زنم ز لبت
&&
سعادتم چه بود خاک پات بوسیدن
\\
به عشق مستی و رسواییم خوش است از آنک
&&
نکو نباشد با عشق زهد ورزیدن
\\
نشاط زاهد از انواع طاعت است و ورع
&&
صفای عارف از ابروی نیکوان دیدن
\\
عنایت تو چو با جان سعدی است چه باک
&&
چه غم خورد گه حشر از گناه سنجیدن
\\
\end{longtable}
\end{center}
