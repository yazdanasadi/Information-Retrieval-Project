\begin{center}
\section*{غزل ۵۷۵: هر سلطنت که خواهی می‌کن که دلپذیری}
\label{sec:575}
\addcontentsline{toc}{section}{\nameref{sec:575}}
\begin{longtable}{l p{0.5cm} r}
هر سلطنت که خواهی می‌کن که دلپذیری
&&
در دست خوبرویان دولت بود اسیری
\\
جان باختن به کویت در آرزوی رویت
&&
دانسته‌ام ولیکن خونخوار ناگزیری
\\
ملک آن توست و فرمان مملوک را چه درمان
&&
گر بی‌گنه بسوزی ور بی خطا بگیری
\\
گر من سخن نگویم در وصف روی و مویت
&&
آیینه‌ات بگوید پنهان که بی‌نظیری
\\
آن کاو ندیده باشد گل در میان بستان
&&
شاید که خیره ماند در ارغوان و خیری
\\
گفتم مگر ز رفتن غایب شوی ز چشمم
&&
آن نیستی که رفتی آنی که در ضمیری
\\
ای باد صبح بستان پیغام وصل جانان
&&
می‌رو که خوش نسیمی می‌دم که خوش عبیری
\\
او را نمی‌توان دید از منتهای خوبی
&&
ما خود نمی‌نماییم از غایت حقیری
\\
گر یار با جوانان خواهد نشست و رندان
&&
ما نیز توبه کردیم از زاهدی و پیری
\\
سعدی نظر بپوشان یا خرقه در میان نه
&&
رندی روا نباشد در جامه فقیری
\\
\end{longtable}
\end{center}
