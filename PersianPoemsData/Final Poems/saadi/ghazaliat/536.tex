\begin{center}
\section*{غزل ۵۳۶: مکن سرگشته آن دل را که دست آموز غم کردی}
\label{sec:536}
\addcontentsline{toc}{section}{\nameref{sec:536}}
\begin{longtable}{l p{0.5cm} r}
مکن سرگشته آن دل را که دست آموز غم کردی
&&
به زیر پای هجرانش لگدکوب ستم کردی
\\
قلم بر بی‌دلان گفتی نخواهم راند و هم راندی
&&
جفا بر عاشقان گفتی نخواهم کرد و هم کردی
\\
بدم گفتی و خرسندم عفاک الله نکو گفتی
&&
سگم خواندی و خشنودم جزاک الله کرم کردی
\\
چه لطف است این که فرمودی مگر سبق اللسان بودت
&&
چه حرف است این که آوردی مگر سهوالقلم کردی
\\
عنایت با من اولیتر که تأدیب جفا دیدم
&&
گل افشان بر سر من کن که خارم در قدم کردی
\\
غنیمت دان اگر روزی به شادی دررسی ای دل
&&
پس از چندین تحمل‌ها که زیر بار غم کردی
\\
شب غم‌های سعدی را مگر هنگام روز آمد
&&
که تاریک و ضعیفش چون چراغ صبحدم کردی
\\
\end{longtable}
\end{center}
