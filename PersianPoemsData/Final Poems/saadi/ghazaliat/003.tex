\begin{center}
\section*{غزل ۳: روی تو خوش می‌نماید آینه ما}
\label{sec:003}
\addcontentsline{toc}{section}{\nameref{sec:003}}
\begin{longtable}{l p{0.5cm} r}
روی تو خوش می‌نماید آینه ما
&&
کآینه پاکیزه است و روی تو زیبا
\\
چون می روشن در آبگینه صافی
&&
خوی جمیل از جمال روی تو پیدا
\\
هر که دمی با تو بود یا قدمی رفت
&&
از تو نباشد به هیچ روی شکیبا
\\
صید بیابان سر از کمند بپیچد
&&
ما همه پیچیده در کمند تو عمدا
\\
طایر مسکین که مهر بست به جایی
&&
گر بکشندش نمی‌رود به دگر جا
\\
غیرتم آید شکایت از تو به هر کس
&&
درد احبا نمی‌برم به اطبا
\\
برخی جانت شوم که شمع افق را
&&
پیش بمیرد چراغدان ثریا
\\
گر تو شکرخنده آستین نفشانی
&&
هر مگسی طوطیی شوند شکرخا
\\
لعبت شیرین اگر ترش ننشیند
&&
مدعیانش طمع کنند به حلوا
\\
مرد تماشای باغ حسن تو سعدیست
&&
دست فرومایگان برند به یغما
\\
\end{longtable}
\end{center}
