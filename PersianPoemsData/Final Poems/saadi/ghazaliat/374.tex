\begin{center}
\section*{غزل ۳۷۴: از در درآمدی و من از خود به درشدم}
\label{sec:374}
\addcontentsline{toc}{section}{\nameref{sec:374}}
\begin{longtable}{l p{0.5cm} r}
از در درآمدی و من از خود به در شدم
&&
گفتی کز این جهان به جهان دگر شدم
\\
گوشم به راه تا که خبر می‌دهد ز دوست
&&
صاحب خبر بیامد و من بی‌خبر شدم
\\
چون شبنم اوفتاده بدم پیش آفتاب
&&
مهرم به جان رسید و به عیوق بر شدم
\\
گفتم ببینمش مگرم درد اشتیاق
&&
ساکن شود بدیدم و مشتاق‌تر شدم
\\
دستم نداد قوت رفتن به پیش یار
&&
چندی به پای رفتم و چندی به سر شدم
\\
تا رفتنش ببینم و گفتنش بشنوم
&&
از پای تا به سر همه سمع و بصر شدم
\\
من چشم از او چگونه توانم نگاه داشت
&&
کاول نظر به دیدن او دیده ور شدم
\\
بیزارم از وفای تو یک روز و یک زمان
&&
مجموع اگر نشستم و خرسند اگر شدم
\\
او را خود التفات نبودش به صید من
&&
من خویشتن اسیر کمند نظر شدم
\\
گویند روی سرخ تو سعدی چه زرد کرد
&&
اکسیر عشق بر مسم افتاد و زر شدم
\\
\end{longtable}
\end{center}
