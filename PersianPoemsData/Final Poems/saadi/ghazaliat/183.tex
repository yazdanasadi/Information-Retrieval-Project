\begin{center}
\section*{غزل ۱۸۳: کدام چاره سگالم که با تو درگیرد}
\label{sec:183}
\addcontentsline{toc}{section}{\nameref{sec:183}}
\begin{longtable}{l p{0.5cm} r}
کدام چاره سگالم که با تو درگیرد
&&
کجا روم که دل من دل از تو برگیرد
\\
ز چشم خلق فتادم هنوز و ممکن نیست
&&
که چشم شوخ من از عاشقی حذر گیرد
\\
دل ضعیف مرا نیست زور بازوی آن
&&
که پیش تیر غمت صابری سپر گیرد
\\
چو تلخ عیشی من بشنوی به خنده درآی
&&
که گر به خنده درآیی جهان شکر گیرد
\\
به خسته برگذری صحتش فرازآید
&&
به مرده درنگری زندگی ز سر گیرد
\\
ز سوزناکی گفتار من قلم بگریست
&&
که در نی آتش سوزنده زودتر گیرد
\\
دو چشم مست تو شهری به غمزه‌ای ببرند
&&
کرشمه تو جهانی به یک نظر گیرد
\\
گر از جفای تو در کنج خانه بنشینم
&&
خیالت از در و بامم به عنف درگیرد
\\
مکن که روز جمالت سر آید ار سعدی
&&
شبی به دست دعا دامن سحر گیرد
\\
\end{longtable}
\end{center}
