\begin{center}
\section*{غزل ۳۵: دوست دارم که بپوشی رخ همچون قمرت}
\label{sec:035}
\addcontentsline{toc}{section}{\nameref{sec:035}}
\begin{longtable}{l p{0.5cm} r}
دوست دارم که بپوشی رخ همچون قمرت
&&
تا چو خورشید نبینند به هر بام و درت
\\
جرم بیگانه نباشد که تو خود صورت خویش
&&
گر در آیینه ببینی برود دل ز برت
\\
جای خنده‌ست سخن گفتن شیرین پیشت
&&
کآب شیرین چو بخندی برود از شکرت
\\
راه آه سحر از شوق نمی‌یارم داد
&&
تا نباید که بشوراند خواب سحرت
\\
هیچ پیرایه زیادت نکند حسن تو را
&&
هیچ مشاطه نیاراید از این خوبترت
\\
بارها گفته‌ام این روی به هر کس منمای
&&
تا تأمل نکند دیده هر بی بصرت
\\
بازگویم نه که این صورت و معنی که تو راست
&&
نتواند که ببیند مگر اهل نظرت
\\
راه صد دشمنم از بهر تو می‌باید داد
&&
تا یکی دوست ببینم که بگوید خبرت
\\
آن چنان سخت نیاید سر من گر برود
&&
نازنینا که پریشانی مویی ز سرت
\\
غم آن نیست که بر خاک نشیند سعدی
&&
زحمت خویش نمی‌خواهد بر رهگذرت
\\
\end{longtable}
\end{center}
