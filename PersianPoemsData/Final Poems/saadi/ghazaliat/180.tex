\begin{center}
\section*{غزل ۱۸۰: انصاف نبود آن رخ دلبند نهان کرد}
\label{sec:180}
\addcontentsline{toc}{section}{\nameref{sec:180}}
\begin{longtable}{l p{0.5cm} r}
انصاف نبود آن رخ دلبند نهان کرد
&&
زیرا که نه روییست کز او صبر توان کرد
\\
امروز یقین شد که تو محبوب خدایی
&&
کز عالم جان این همه دل با تو روان کرد
\\
مشتاق تو را کی بود آرام و صبوری
&&
هرگز نشنیدم که کسی صبر ز جان کرد
\\
تا کوه گرفتم ز فراقت مژه‌ای آب
&&
چندان بچکانید که بر سنگ نشان کرد
\\
زنهار که از دمدمه کوس رحیلت
&&
چون رایت منصور چه دل‌ها خفقان کرد
\\
باران به بساط اول این سال ببارید
&&
ابر این همه تأخیر که کرد از پی آن کرد
\\
تا در نظرت باد صبا عذر بخواهد
&&
هر جور که بر طرف چمن باد خزان کرد
\\
گل مژده بازآمدنت در چمن انداخت
&&
سلطان صبا پر زر مصریش دهان کرد
\\
از دامن که تا به در شهر بساطی
&&
از سبزه بگسترد و بر او لاله فشان کرد
\\
شاید که زمین حله بپوشد که چو سعدی
&&
پیرانه سرش دولت روی تو جوان کرد
\\
\end{longtable}
\end{center}
