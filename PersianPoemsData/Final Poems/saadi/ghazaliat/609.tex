\begin{center}
\section*{غزل ۶۰۹: بر آنم گر تو بازآیی که در پایت کنم جانی}
\label{sec:609}
\addcontentsline{toc}{section}{\nameref{sec:609}}
\begin{longtable}{l p{0.5cm} r}
بر آنم گر تو باز آیی که در پایت کنم جانی
&&
و زین کمتر نشاید کرد در پای تو قربانی
\\
امید از بخت می‌دارم بقای عمر چندانی
&&
کز ابر لطف باز آید به خاک تشنه بارانی
\\
میان عاشق و معشوق اگر باشد بیابانی
&&
درخت ارغوان روید به جای هر مغیلانی
\\
مگر لیلی نمی‌داند که بی دیدار میمونش
&&
فراخای جهان تنگ است بر مجنون چو زندانی
\\
دریغا عهد آسانی که مقدارش ندانستم
&&
ندانی قدر وصل الا که درمانی به هجرانی
\\
نه در زلف پریشانت من تنها گرفتارم
&&
که دل در بند او دارد به هر مویی پریشانی
\\
چه فتنه‌ست این که در چشمت به غارت می‌برد دل‌ها
&&
تویی در عهد ما گر هست در شیراز فتانی
\\
نشاید خون سعدی را به باطل ریختن حقا
&&
بیا سهل است اگر داری به خط خواجه فرمانی
\\
زمان رفته باز آید ولیکن صبر می‌باید
&&
که مستخلص نمی‌گردد بهاری بی زمستانی
\\
\end{longtable}
\end{center}
