\begin{center}
\section*{غزل ۵۷۶: اگر گلاله مشکین ز رخ براندازی}
\label{sec:576}
\addcontentsline{toc}{section}{\nameref{sec:576}}
\begin{longtable}{l p{0.5cm} r}
اگر کلاله مشکین ز رخ براندازی
&&
کنند در قدمت عاشقان سراندازی
\\
اگر به رقص درآیی تو سرو سیم اندام
&&
نظاره کن که چه مستی کنند و جانبازی
\\
تو با چنین قد و بالا و صورت زیبا
&&
به سرو و لاله و شمشاد و گل نپردازی
\\
کدام باغ چو رخسار تو گلی دارد
&&
کدام سرو کند با قدت سرافرازی
\\
به حسن خال و بناگوش اگر نگاه کنی
&&
نظر تو با قد و بالای خود نیندازی
\\
غلام باد صبایم غلام باد صبا
&&
که با کلاله جعدت همی‌کند بازی
\\
بگوی مطرب یاران بیار زمزمه‌ای
&&
بنال بلبل مستان که بس خوش آوازی
\\
که گفته است که صد دل به غمزه‌ای ببری
&&
هزار صید به یک تاختن بیندازی
\\
ز لطف لفظ شکربار گفته سعدی
&&
شدم غلام همه شاعران شیرازی
\\
\end{longtable}
\end{center}
