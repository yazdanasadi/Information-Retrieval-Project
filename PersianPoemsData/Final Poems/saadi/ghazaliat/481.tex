\begin{center}
\section*{غزل ۴۸۱: صید بیابان عشق چون بخورد تیر او}
\label{sec:481}
\addcontentsline{toc}{section}{\nameref{sec:481}}
\begin{longtable}{l p{0.5cm} r}
صید بیابان عشق چون بخورد تیر او
&&
سر نتواند کشید پای ز زنجیر او
\\
گو به سنانم بدوز یا به خدنگم بزن
&&
گر به شکار آمده‌ست دولت نخجیر او
\\
گفتم از آسیب عشق روی به عالم نهم
&&
عرصه عالم گرفت حسن جهان گیر او
\\
با همه تدبیر خویش ما سپر انداختیم
&&
روی به دیوار صبر چشم به تقدیر او
\\
چاره مغلوب نیست جز سپر انداختن
&&
چون نتواند که سر در کشد از تیر او
\\
کشته معشوق را درد نباشد که خلق
&&
زنده به جانند و ما زنده به تأثیر او
\\
او به فغان آمده‌ست زین همه تعجیل ما
&&
ای عجب و ما به جان زین همه تأخیر او
\\
در همه گیتی نگاه کردم و باز آمدم
&&
صورت کس خوب نیست پیش تصاویر او
\\
سعدی شیرین زبان این همه شور از کجا
&&
شاهد ما آیتیست وین همه تفسیر او
\\
آتشی از سوز عشق در دل داوود بود
&&
تا به فلک می‌رسد بانگ مزامیر او
\\
\end{longtable}
\end{center}
