\begin{center}
\section*{غزل ۴۷: سلسله موی دوست حلقه دام بلاست}
\label{sec:047}
\addcontentsline{toc}{section}{\nameref{sec:047}}
\begin{longtable}{l p{0.5cm} r}
سلسلهٔ موی دوست حلقه دام بلاست
&&
هر که در این حلقه نیست فارغ از این ماجراست
\\
گر بزنندم به تیغ در نظرش بی‌دریغ
&&
دیدن او یک نظر صد چو منش خونبهاست
\\
گر برود جان ما در طلب وصل دوست
&&
حیف نباشد که دوست دوست‌تر از جان ماست
\\
دعوی عشاق را شرع نخواهد بیان
&&
گونهٔ زردش دلیل ناله زارش گواست
\\
مایهٔ پرهیزگار قوت صبر است و عقل
&&
عقل گرفتار عشق صبر زبون هواست
\\
دلشدهٔ پایبند گردن جان در کمند
&&
زهرهٔ گفتار نه کاین چه سبب وان چراست
\\
مالک ملک وجود حاکم رد و قبول
&&
هر چه کند جور نیست ور تو بنالی جفاست
\\
تیغ برآر از نیام زهر برافکن به جام
&&
کز قبل ما قبول وز طرف ما رضاست
\\
گر بنوازی به لطف ور بگدازی به قهر
&&
حکم تو بر من روان زجر تو بر من رواست
\\
هر که به جور رقیب یا به جفای حبیب
&&
عهد فرامش کند مدعی بی‌وفاست
\\
سعدی از اخلاق دوست هر چه برآید نکوست
&&
گو همه دشنام گو کز لب شیرین دعاست
\\
\end{longtable}
\end{center}
