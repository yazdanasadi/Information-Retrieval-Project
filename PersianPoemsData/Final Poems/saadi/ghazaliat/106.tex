\begin{center}
\section*{غزل ۱۰۶: شادی به روزگار گدایان کوی دوست}
\label{sec:106}
\addcontentsline{toc}{section}{\nameref{sec:106}}
\begin{longtable}{l p{0.5cm} r}
شادی به روزگار گدایان کوی دوست
&&
بر خاک ره نشسته به امید روی دوست
\\
گفتم به گوشه‌ای بنشینم ولی دلم
&&
ننشیند از کشیدن خاطر به سوی دوست
\\
صبرم ز روی دوست میسر نمی‌شود
&&
دانی طریق چیست تحمل ز خوی دوست
\\
ناچار هر که دل به غم روی دوست داد
&&
کارش به هم برآمده باشد چو موی دوست
\\
خاطر به باغ می‌رودم روز نوبهار
&&
تا با درخت گل بنشینم به بوی دوست
\\
فردا که خاک مرده به حشر آدمی کنند
&&
ای باد خاک من مطلب جز به کوی دوست
\\
سعدی چراغ می‌نکند در شب فراق
&&
ترسد که دیده باز کند جز به روی دوست
\\
\end{longtable}
\end{center}
