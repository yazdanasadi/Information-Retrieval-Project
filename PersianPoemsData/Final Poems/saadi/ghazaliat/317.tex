\begin{center}
\section*{غزل ۳۱۷: بوی بهار آمد بنال ای بلبل شیرین نفس}
\label{sec:317}
\addcontentsline{toc}{section}{\nameref{sec:317}}
\begin{longtable}{l p{0.5cm} r}
بوی بهار آمد بنال ای بلبل شیرین نفس
&&
ور پایبندی همچو من فریاد می‌خوان از قفس
\\
گیرند مردم دوستان نامهربان و مهربان
&&
هر روز خاطر با یکی ما خود یکی داریم و بس
\\
محمول پیش آهنگ را از من بگو ای ساربان
&&
تو خواب می‌کن بر شتر تا بانگ می‌دارد جرس
\\
شیرین بضاعت بر مگس چندان که تندی می‌کند
&&
او بادبیزن همچنان در دست و می‌آید مگس
\\
پند خردمندان چه سود اکنون که بندم سخت شد
&&
گر جستم این بار از قفس بیدار باشم زین سپس
\\
گر دوست می‌آید برم یا تیغ دشمن بر سرم
&&
من با کسی افتاده‌ام کز وی نپردازم به کس
\\
با هر که بنشینم دمی کز یاد او غافل شوم
&&
چون صبح بی خورشیدم از دل بر نمی‌آید نفس
\\
من مفلسم در کاروان گوهر که خواهی قصد کن
&&
نگذاشت مطرب دربرم چندان که بستاند عسس
\\
گر پند می‌خواهی بده ور بند می‌خواهی بنه
&&
دیوانه سر خواهد نهاد آن گه نهد از سر هوس
\\
فریاد سعدی در جهان افکندی ای آرام جان
&&
چندین به فریاد آوری باری به فریادش برس
\\
\end{longtable}
\end{center}
