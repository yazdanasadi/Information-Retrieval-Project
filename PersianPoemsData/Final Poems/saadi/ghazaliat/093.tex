\begin{center}
\section*{غزل ۹۳: یار من آن که لطف خداوند یار اوست}
\label{sec:093}
\addcontentsline{toc}{section}{\nameref{sec:093}}
\begin{longtable}{l p{0.5cm} r}
یار من آن که لطف خداوند یار اوست
&&
بیداد و داد و رد و قبول اختیار اوست
\\
دریای عشق را به حقیقت کنار نیست
&&
ور هست پیش اهل حقیقت کنار اوست
\\
در عهد لیلی این همه مجنون نبوده‌اند
&&
وین فتنه برنخاست که در روزگار اوست
\\
صاحبدلی نماند در این فصل نوبهار
&&
الا که عاشق گل و مجروح خار اوست
\\
دانی کدام خاک بر او رشک می‌برم
&&
آن خاک نیکبخت که در رهگذار اوست
\\
باور مکن که صورت او عقل من ببرد
&&
عقل من آن ببرد که صورت نگار اوست
\\
گر دیگران به منظر زیبا نظر کنند
&&
ما را نظر به قدرت پروردگار اوست
\\
اینم قبول بس که بمیرم بر آستان
&&
تا نسبتم کنند که خدمتگزار اوست
\\
بر جور و بی مرادی و درویشی و هلاک
&&
آن را که صبر نیست محبت نه کار اوست
\\
سعدی رضای دوست طلب کن نه حظ خویش
&&
عبد آن کند که رای خداوندگار اوست
\\
\end{longtable}
\end{center}
