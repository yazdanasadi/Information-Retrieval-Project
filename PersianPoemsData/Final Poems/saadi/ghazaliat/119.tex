\begin{center}
\section*{غزل ۱۱۹: هر چه خواهی کن که ما را با تو روی جنگ نیست}
\label{sec:119}
\addcontentsline{toc}{section}{\nameref{sec:119}}
\begin{longtable}{l p{0.5cm} r}
هر چه خواهی کن که ما را با تو روی جنگ نیست
&&
پنجه بر زورآوران انداختن فرهنگ نیست
\\
در که خواهم بستن آن دل کز وصالت برکنم
&&
چون تو در عالم نباشد ور نه عالم تنگ نیست
\\
شاهد ما را نه هر چشمی چنان بیند که هست
&&
صنع را آیینه‌ای باید که بر وی زنگ نیست
\\
با زمانی دیگر انداز ای که پندم می‌دهی
&&
کاین زمانم گوش بر چنگست و دل در چنگ نیست
\\
گر تو را کامی برآید دیر زود از وصل یار
&&
بعد از آن نامت به رسوایی برآید ننگ نیست
\\
سست پیمانا چرا کردی خلاف عقل و رای
&&
صلح با دشمن اگر با دوستانت جنگ نیست
\\
گر تو را آهنگ وصل ما نباشد گو مباش
&&
دوستان را جز به دیدار تو هیچ آهنگ نیست
\\
ور به سنگ از صحبت خویشم برانی عاقبت
&&
خود دلت بر من ببخشاید که آخر سنگ نیست
\\
سعدیا نامت به رندی در جهان افسانه شد
&&
از چه می‌ترسی دگر بعد از سیاهی رنگ نیست
\\
\end{longtable}
\end{center}
