\begin{center}
\section*{غزل ۶: پیش ما رسم شکستن نبود عهد وفا را}
\label{sec:006}
\addcontentsline{toc}{section}{\nameref{sec:006}}
\begin{longtable}{l p{0.5cm} r}
پیش ما رسم شکستن نبود عهد وفا را
&&
الله الله تو فراموش مکن صحبت ما را
\\
قیمت عشق نداند قدم صدق ندارد
&&
سست عهدی که تحمل نکند بار جفا را
\\
گر مخیر بکنندم به قیامت که چه خواهی
&&
دوست ما را و همه نعمت فردوس شما را
\\
گر سرم می‌رود از عهد تو سر بازنپیچم
&&
تا بگویند پس از من که به سر برد وفا را
\\
خنک آن درد که یارم به عیادت به سر آید
&&
دردمندان به چنین درد نخواهند دوا را
\\
باور از مات نباشد تو در آیینه نگه کن
&&
تا بدانی که چه بودست گرفتار بلا را
\\
از سر زلف عروسان چمن دست بدارد
&&
به سر زلف تو گر دست رسد باد صبا را
\\
سر انگشت تحیر بگزد عقل به دندان
&&
چون تأمل کند این صورت انگشت نما را
\\
آرزو می‌کندم شمع صفت پیش وجودت
&&
که سراپای بسوزند من بی سر و پا را
\\
چشم کوته نظران بر ورق صورت خوبان
&&
خط همی‌بیند و عارف قلم صنع خدا را
\\
همه را دیده به رویت نگرانست ولیکن
&&
خودپرستان ز حقیقت نشناسند هوا را
\\
مهربانی ز من آموز و گرم عمر نماند
&&
به سر تربت سعدی بطلب مهرگیا را
\\
هیچ هشیار ملامت نکند مستی ما را
&&
قل لصاح ترک الناس من الوجد سکاری
\\
\end{longtable}
\end{center}
