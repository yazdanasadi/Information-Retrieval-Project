\begin{center}
\section*{غزل ۴۸۷: پنجه با ساعد سیمین که نیندازی به}
\label{sec:487}
\addcontentsline{toc}{section}{\nameref{sec:487}}
\begin{longtable}{l p{0.5cm} r}
پنجه با ساعد سیمین که نیندازی به
&&
با توانای معربد نکنی بازی به
\\
چون دلش دادی و مهرش ستدی چاره نماند
&&
اگر او با تو نسازد تو در او سازی به
\\
جز غم یار مخور تا غم کارت بخورد
&&
تو که با مصلحت خویش نپردازی به
\\
سپر صبر تحمل نکند تیر فراق
&&
با کمان ابرو اگر جنگ نیاغازی به
\\
با چنین یار که ما عقد محبت بستیم
&&
گر همه مایه زیان می‌کند انبازی به
\\
بنده را بر خط فرمان خداوند امور
&&
سر تسلیم نهادن ز سرافرازی به
\\
گر چو چنگم بزنی پیش تو سر برنکنم
&&
این چنین یار وفادار که بنوازی به
\\
هیچ شک نیست به تیر اجل ای یار عزیز
&&
که من از پای درآیم چو تو اندازی به
\\
مجلس ما دگر امروز به بستان ماند
&&
مطرب از بلبل عاشق به خوش آوازی به
\\
گوش بر ناله مطرب کن و بلبل بگذار
&&
که نگوید سخن از سعدی شیرازی به
\\
\end{longtable}
\end{center}
