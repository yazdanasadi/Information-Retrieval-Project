\begin{center}
\section*{غزل ۳۹۷: از تو با مصلحت خویش نمی‌پردازم}
\label{sec:397}
\addcontentsline{toc}{section}{\nameref{sec:397}}
\begin{longtable}{l p{0.5cm} r}
از تو با مصلحت خویش نمی‌پردازم
&&
همچو پروانه که می‌سوزم و در پروازم
\\
گر توانی که بجویی دلم امروز بجوی
&&
ور نه بسیار بجویی و نیابی بازم
\\
نه چنان معتقدم که‌م نظری سیر کند
&&
یا چنان تشنه که جیحون بنشاند آزم
\\
همچو چنگم سر تسلیم و ارادت در پیش
&&
تو به هر ضرب که خواهی بزن و بنوازم
\\
گر به آتش بریم صد ره و بیرون آری
&&
زر نابم که همان باشم اگر بگدازم
\\
گر تو آن جور پسندی که به سنگم بزنی
&&
از من این جرم نیاید که خلاف آغازم
\\
خدمتی لایقم از دست نیاید چه کنم
&&
سر نه چیزیست که در پای عزیزان بازم
\\
من خراباتیم و عاشق و دیوانه و مست
&&
بیشتر زین چه حکایت بکند غمازم
\\
ماجرای دل دیوانه بگفتم به طبیب
&&
که همه شب در چشم است به فکرت بازم
\\
گفت از این نوع شکایت که تو داری سعدی
&&
درد عشق است ندانم که چه درمان سازم
\\
\end{longtable}
\end{center}
