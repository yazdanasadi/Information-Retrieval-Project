\begin{center}
\section*{غزل ۳۶۲: مرا دو دیده به راه و دو گوش بر پیغام}
\label{sec:362}
\addcontentsline{toc}{section}{\nameref{sec:362}}
\begin{longtable}{l p{0.5cm} r}
مرا دو دیده به راه و دو گوش بر پیغام
&&
تو مستریح و به افسوس می‌رود ایام
\\
شبی نپرسی و روزی که دوستدارانم
&&
چگونه شب به سحر می‌برند و روز به شام
\\
ببردی از دل من مهر هر کجا صنمیست
&&
مرا که قبله گرفتم چه کار با اصنام
\\
به کام دل نفسی با تو التماس من است
&&
بسا نفس که فرورفت و برنیامد کام
\\
مرا نه دولت وصل و نه احتمال فراق
&&
نه پای رفتن از این ناحیت نه جای مقام
\\
چه دشمنی تو که از عشق دست و شمشیرت
&&
مطاوعت به گریزم نمی‌کنند اقدام
\\
ملامتم نکند هر که معرفت دارد
&&
که عشق می‌بستاند ز دست عقل زمام
\\
مرا که با تو سخن گویم و سخن شنوم
&&
نه گوش فهم بماند نه هوش استفهام
\\
اگر زبان مرا روزگار دربندد
&&
به عشق در سخن آیند ریزه‌های عظام
\\
بر آتش غم سعدی کدام دل که نسوخت
&&
گر این سخن برود در جهان نماند خام
\\
\end{longtable}
\end{center}
