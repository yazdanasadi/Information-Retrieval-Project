\begin{center}
\section*{غزل ۱۶۶: هر که چیزی دوست دارد جان و دل بر وی گمارد}
\label{sec:166}
\addcontentsline{toc}{section}{\nameref{sec:166}}
\begin{longtable}{l p{0.5cm} r}
هر که چیزی دوست دارد جان و دل بر وی گمارد
&&
هر که محرابش تو باشی سر ز خلوت برنیارد
\\
روزی اندر خاکت افتم ور به بادم می‌رود سر
&&
کان که در پای تو میرد جان به شیرینی سپارد
\\
من نه آن صورت پرستم کز تمنای تو مستم
&&
هوش من دانی که بردست آن که صورت می‌نگارد
\\
عمر گویندم که ضایع می‌کنی با خوبرویان
&&
وان که منظوری ندارد عمر ضایع می‌گذارد
\\
هر که می‌ورزد درختی در سرابستان معنی
&&
بیخش اندر دل نشاند تخمش اندر جان بکارد
\\
عشق و مستوری نباشد پای گو در دامن آور
&&
کز گریبان ملامت سر برآوردن نیارد
\\
گر من از عهدت بگردم ناجوانمردم نه مردم
&&
عاشق صادق نباشد کز ملامت سر بخارد
\\
باغ می‌خواهم که روزی سرو بالایت ببیند
&&
تا گلت در پا بریزد و ارغوان بر سر ببارد
\\
آن چه رفتارست و قامت وان چه گفتار و قیامت
&&
چند خواهی گفت سعدی طیبات آخر ندارد
\\
\end{longtable}
\end{center}
