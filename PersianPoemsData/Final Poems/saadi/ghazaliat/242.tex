\begin{center}
\section*{غزل ۲۴۲: سرو بلند بین که چه رفتار می‌کند}
\label{sec:242}
\addcontentsline{toc}{section}{\nameref{sec:242}}
\begin{longtable}{l p{0.5cm} r}
سرو بلند بین که چه رفتار می‌کند
&&
وآن ماه محتشم که چه گفتار می‌کند
\\
آن چشم مست بین که به شوخی و دلبری
&&
قصد هلاک مردم هشیار می‌کند
\\
دیوانه می‌کند دل صاحب تمیز را
&&
هر گه که التفات پری وار می‌کند
\\
ما روی کرده از همه عالم به روی او
&&
وآن سست عهد روی به دیوار می‌کند
\\
عاقل خبر ندارد از اندوه عاشقان
&&
خفته‌ست و عیب مردم بیدار می‌کند
\\
من طاقت شکیب ندارم ز روی خوب
&&
صوفی به عجز خویشتن اقرار می‌کند
\\
بیچاره از مطالعه روی نیکوان
&&
صد بار توبه کرد و دگربار می‌کند
\\
سعدی نگفتمت که خم زلف شاهدان
&&
دربند او مشو که گرفتار می‌کند
\\
\end{longtable}
\end{center}
