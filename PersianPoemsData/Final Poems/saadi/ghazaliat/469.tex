\begin{center}
\section*{غزل ۴۶۹: گواهی امینست بر درد من}
\label{sec:469}
\addcontentsline{toc}{section}{\nameref{sec:469}}
\begin{longtable}{l p{0.5cm} r}
گواهی امین است بر درد من
&&
سرشک روان بر رخ زرد من
\\
ببخشای بر ناله عندلیب
&&
الا ای گل نازپرورد من
\\
که گر هم بدین نوع باشد فراق
&&
به نزد تو باد آورد گرد من
\\
که دیده‌ست هرگز چنین آتشی
&&
کز او می‌برآید دم سرد من
\\
فغان من از دست جور تو نیست
&&
که از طالع مادرآورد من
\\
من اندر خور بندگی نیستم
&&
وز اندازه بیرون تو در خورد من
\\
بداندیش نادان که مطرود باد
&&
ندانم چه می‌خواهد از طرد من
\\
و گر خود من آنم که اینم سزاست
&&
ببخش و مگیر ای جوانمرد من
\\
تو معذور داری به انعام خویش
&&
اگر زلتی آمد از کرد من
\\
تو دردی نداری که دردت مباد
&&
از آن رحمتت نیست بر درد من
\\
\end{longtable}
\end{center}
