\begin{center}
\section*{غزل ۵۵۱: رفتی و همچنان به خیال من اندری}
\label{sec:551}
\addcontentsline{toc}{section}{\nameref{sec:551}}
\begin{longtable}{l p{0.5cm} r}
رفتی و همچنان به خیال من اندری
&&
گویی که در برابر چشمم مصوری
\\
فکرم به منتهای جمالت نمی‌رسد
&&
کز هر چه در خیال من آمد نکوتری
\\
مه بر زمین نرفت و پری دیده برنداشت
&&
تا ظن برم که روی تو ماه است یا پری
\\
تو خود فرشته‌ای نه از این گل سرشته‌ای
&&
گر خلق از آب و خاک تو از مشک و عنبری
\\
ما را شکایتی ز تو گر هست هم به توست
&&
کز تو به دیگران نتوان برد داوری
\\
با دوست کنج فقر بهشت است و بوستان
&&
بی دوست خاک بر سر جاه و توانگری
\\
تا دوست در کنار نباشد به کام دل
&&
از هیچ نعمتی نتوانی که برخوری
\\
گر چشم در سرت کنم از گریه باک نیست
&&
زیرا که تو عزیزتر از چشم در سری
\\
چندان که جهد بود دویدیم در طلب
&&
کوشش چه سود چون نکند بخت یاوری
\\
سعدی به وصل دوست چو دستت نمی‌رسد
&&
باری به یاد دوست زمانی به سر بری
\\
\end{longtable}
\end{center}
