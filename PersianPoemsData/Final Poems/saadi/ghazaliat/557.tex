\begin{center}
\section*{غزل ۵۵۷: هرگز این صورت کند صورتگری}
\label{sec:557}
\addcontentsline{toc}{section}{\nameref{sec:557}}
\begin{longtable}{l p{0.5cm} r}
هرگز این صورت کند صورتگری
&&
یا چنین شاهد بود در کشوری
\\
سرورفتاری صنوبرقامتی
&&
ماه رخساری ملایک منظری
\\
می‌رود وز خویشتن بینی که هست
&&
در نمی‌آید به چشمش دیگری
\\
صد هزارش دست خاطر در رکاب
&&
پادشاهی می‌رود با لشکری
\\
عارضش باغی دهانش غنچه‌ای
&&
بل بهشتی در میانش کوثری
\\
ماهرویا مهربانی پیشه کن
&&
خوبرویی را بباید زیوری
\\
بی تو در هر گوشه پایی در گل است
&&
وز تو در هر خانه دستی بر سری
\\
چون همایم سایه‌ای بر سر فکن
&&
تا در اقبالت شوم نیک اختری
\\
در خداوندی چه نقصان آیدش
&&
گر خداوندی بپرسد چاکری
\\
مصلحت بودی شکایت گفتنم
&&
گر به غیر از خصم بودی داوری
\\
سعدیا داروی تلخ از دست دوست
&&
به که شیرینی ز دست دیگری
\\
خاکی از مردم بماند در جهان
&&
وز وجود عاشقان خاکستری
\\
\end{longtable}
\end{center}
