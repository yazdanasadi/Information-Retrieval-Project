\begin{center}
\section*{غزل ۱۷۵: بازت ندانم از سر پیمان ما که برد}
\label{sec:175}
\addcontentsline{toc}{section}{\nameref{sec:175}}
\begin{longtable}{l p{0.5cm} r}
بازت ندانم از سر پیمان ما که برد
&&
باز از نگین عهد تو نقش وفا که برد
\\
چندین وفا که کرد چو من در هوای تو
&&
وان گه ز دست هجر تو چندین جفا که برد
\\
بگریست چشم ابر بر احوال زار من
&&
جز آه من به گوش وی این ماجرا که برد
\\
گفتم لب تو را که دل من تو برده‌ای
&&
گفتا کدام دل چه نشان کی کجا که برد
\\
سودا مپز که آتش غم در دل تو نیست
&&
ما را غم تو برد به سودا تو را که برد
\\
توفیق عشق روی تو گنجیست تا که یافت
&&
باز اتفاق وصل تو گوییست تا که برد
\\
جز چشم تو که فتنه قتال عالمست
&&
صد شیخ و زاهد از سر راه خدا که برد
\\
سعدی نه مرد بازی شطرنج عشق توست
&&
دستی به کام دل ز سپهر دغا که برد
\\
\end{longtable}
\end{center}
