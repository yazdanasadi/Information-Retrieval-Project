\begin{center}
\section*{غزل ۴۵۷: چند بشاید به صبر دیده فرودوختن}
\label{sec:457}
\addcontentsline{toc}{section}{\nameref{sec:457}}
\begin{longtable}{l p{0.5cm} r}
چند بشاید به صبر دیده فرو دوختن
&&
خرمن ما را نماند حیله به جز سوختن
\\
گر نظر صدق را نام گنه می‌نهند
&&
حاصل ما هیچ نیست جز گنه اندوختن
\\
چند به شب در سماع جامه دریدن ز شوق
&&
روز دگر بامداد پاره بر او دوختن
\\
زهد نخواهد خرید چاره رنجور عشق
&&
شمع و شراب است و شید پیش تو نفروختن
\\
تا به کدام آبروی ذکر وصالت کنیم
&&
شکر خیالت هنوز می‌نتوان توختن
\\
لهجه شیرین من پیش دهان تو چیست
&&
در نظر آفتاب مشعله افروختن
\\
منطق سعدی شنید حاسد و حیران بماند
&&
چاره او خامشیست یا سخن آموختن
\\
\end{longtable}
\end{center}
