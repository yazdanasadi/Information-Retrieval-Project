\begin{center}
\section*{غزل ۲۳۲: دو چشم مست تو کز خواب صبح برخیزند}
\label{sec:232}
\addcontentsline{toc}{section}{\nameref{sec:232}}
\begin{longtable}{l p{0.5cm} r}
دو چشم مست تو کز خواب صبح برخیزند
&&
هزار فتنه به هر گوشه‌ای برانگیزند
\\
چگونه انس نگیرند با تو آدمیان
&&
که از لطافت خوی تو وحش نگریزند
\\
چنان که در رخ خوبان حلال نیست نظر
&&
حلال نیست که از تو نظر بپرهیزند
\\
غلام آن سر و پایم که از لطافت و حسن
&&
به سر سزاست که پیشش به پای برخیزند
\\
تو قدر خویش ندانی ز دردمندان پرس
&&
کز اشتیاق جمالت چه اشک می‌ریزند
\\
قرار عقل برفت و مجال صبر نماند
&&
که چشم و زلف تو از حد برون دلاویزند
\\
مرا مگوی نصیحت که پارسایی و عشق
&&
دو خصلتند که با یک دگر نیامیزند
\\
رضا به حکم قضا اختیار کن سعدی
&&
که شرط نیست که با زورمند بستیزند
\\
\end{longtable}
\end{center}
