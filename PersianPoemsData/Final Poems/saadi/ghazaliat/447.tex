\begin{center}
\section*{غزل ۴۴۷: برخیز که می‌رود زمستان}
\label{sec:447}
\addcontentsline{toc}{section}{\nameref{sec:447}}
\begin{longtable}{l p{0.5cm} r}
برخیز که می‌رود زمستان
&&
بگشای در سرای بستان
\\
نارنج و بنفشه بر طبق نه
&&
منقل بگذار در شبستان
\\
وین پرده بگوی تا به یک بار
&&
زحمت ببرد ز پیش ایوان
\\
برخیز که باد صبح نوروز
&&
در باغچه می‌کند گل افشان
\\
خاموشی بلبلان مشتاق
&&
در موسم گل ندارد امکان
\\
آواز دهل نهان نماند
&&
در زیر گلیم و عشق پنهان
\\
بوی گل بامداد نوروز
&&
و آواز خوش هزاردستان
\\
بس جامه فروخته‌ست و دستار
&&
بس خانه که سوخته‌ست و دکان
\\
ما را سر دوست بر کنار است
&&
آنک سر دشمنان و سندان
\\
چشمی که به دوست برکند دوست
&&
بر هم ننهد ز تیرباران
\\
سعدی چو به میوه می‌رسد دست
&&
سهل است جفای بوستانبان
\\
\end{longtable}
\end{center}
