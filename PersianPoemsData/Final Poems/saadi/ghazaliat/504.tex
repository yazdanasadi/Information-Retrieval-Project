\begin{center}
\section*{غزل ۵۰۴: چه رویست آن که دیدارش ببرد از من شکیبایی}
\label{sec:504}
\addcontentsline{toc}{section}{\nameref{sec:504}}
\begin{longtable}{l p{0.5cm} r}
چه روی است آن که دیدارش ببرد از من شکیبایی
&&
گواهی می‌دهد صورت بر اخلاقش به زیبایی
\\
نگارینا به هر تندی که می‌خواهی جوابم ده
&&
اگر تلخ اتفاق افتد به شیرینی بیندایی
\\
دگر چون ناشکیبایی ببینم صادقش خوانم
&&
که من در نفس خویش از تو نمی‌بینم شکیبایی
\\
از این پس عیب شیدایان نخواهم کرد و مسکینان
&&
که دانشمند از این صورت بر آرد سر به شیدایی
\\
چنانم در دلی حاضر که جان در جسم و خون در رگ
&&
فراموشم نه‌ای وقتی که دیگر وقت یاد آیی
\\
شبی خوش هر که می‌خواهد که با جانان به روز آرد
&&
بسی شب روز گرداند به تاریکی و تنهایی
\\
بیار ای لعبت ساقی بگو ای کودک مطرب
&&
که صوفی در سماع آمد دوتایی کرد یکتایی
\\
سخن پیدا بود سعدی که حدش تا کجا باشد
&&
زبان درکش که منظورت ندارد حد زیبایی
\\
\end{longtable}
\end{center}
