\begin{center}
\section*{غزل ۵۴۲: آخر نگاهی بازکن وقتی که بر ما بگذری}
\label{sec:542}
\addcontentsline{toc}{section}{\nameref{sec:542}}
\begin{longtable}{l p{0.5cm} r}
آخر نگاهی باز کن وقتی که بر ما بگذری
&&
یا کبر منعت می‌کند کز دوستان یاد آوری
\\
هرگز نبود اندر ختن بر صورتی چندین فتن
&&
هرگز نباشد در چمن سروی بدین خوش منظری
\\
صورتگر دیبای چین گو صورت رویش ببین
&&
یا صورتی برکش چنین یا توبه کن صورتگری
\\
ز ابروی زنگارین کمان گر پرده برداری عیان
&&
تا قوس باشد در جهان دیگر نبیند مشتری
\\
بالای سرو بوستان رویی ندارد دلستان
&&
خورشید با رویی چنان مویی ندارد عنبری
\\
تا نقش می‌بندد فلک کس را نبوده‌ست این نمک
&&
ماهی ندانم یا ملک فرزند آدم یا پری
\\
تا دل به مهرت داده‌ام در بحر فکر افتاده‌ام
&&
چون در نماز استاده‌ام گویی به محرابم دری
\\
دیگر نمی‌دانم طریق از دست رفتم چون غریق
&&
آنک دهانت چون عقیق از بس که خونم می‌خوری
\\
گر رفته باشم زین جهان بازآیدم رفته روان
&&
گر همچنین دامن کشان بالای خاکم بگذری
\\
از نعلش آتش می‌جهد نعلم در آتش می‌نهد
&&
گر دیگری جان می‌دهد سعدی تو جان می‌پروری
\\
هر کس که دعوی می‌کند کاو با تو انسی می‌کند
&&
در عهد موسی می‌کند آواز گاو سامری
\\
\end{longtable}
\end{center}
