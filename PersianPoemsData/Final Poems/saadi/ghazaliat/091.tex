\begin{center}
\section*{غزل ۹۱: سفر دراز نباشد به پای طالب دوست}
\label{sec:091}
\addcontentsline{toc}{section}{\nameref{sec:091}}
\begin{longtable}{l p{0.5cm} r}
سفر دراز نباشد به پای طالب دوست
&&
که زنده ابدست آدمی که کشته اوست
\\
شراب خورده معنی چو در سماع آید
&&
چه جای جامه که بر خویشتن بدرد پوست
\\
هر آن که با رخ منظور ما نظر دارد
&&
به ترک خویش بگوید که خصم عربده جوست
\\
حقیر تا نشماری تو آب چشم فقیر
&&
که قطره قطره باران چو با هم آمد جوست
\\
نمی‌رود که کمندش همی‌برد مشتاق
&&
چه جای پند نصیحت کنان بیهده گوست
\\
چو در میانه خاک اوفتاده‌ای بینی
&&
از آن بپرس که چوگان از او مپرس که گوست
\\
چرا و چون نرسد بندگان مخلص را
&&
رواست گر همه بد می‌کنی بکن که نکوست
\\
کدام سرو سهی راست با وجود تو قدر
&&
کدام غالیه را پیش خاک پای تو بوست
\\
بسی بگفت خداوند عقل و نشنیدم
&&
که دل به غمزه خوبان مده که سنگ و سبوست
\\
هزار دشمن اگر بر سرند سعدی را
&&
به دوستی که نگوید به جز حکایت دوست
\\
به آب دیده خونین نبشته قصه عشق
&&
نظر به صفحه اول مکن که تو بر توست
\\
\end{longtable}
\end{center}
