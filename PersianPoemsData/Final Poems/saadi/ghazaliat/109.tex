\begin{center}
\section*{غزل ۱۰۹: بیا بیا که مرا با تو ماجرایی هست}
\label{sec:109}
\addcontentsline{toc}{section}{\nameref{sec:109}}
\begin{longtable}{l p{0.5cm} r}
بیا بیا که مرا با تو ماجرایی هست
&&
بگوی اگر گنهی رفت و گر خطایی هست
\\
روا بود که چنین بی‌حساب دل ببری
&&
مکن که مظلمه خلق را جزایی هست
\\
توانگران را عیبی نباشد ار وقتی
&&
نظر کنند که در کوی ما گدایی هست
\\
به کام دشمن و بیگانه رفت چندین روز
&&
ز دوستان نشنیدم که آشنایی هست
\\
کسی نماند که بر درد من نبخشاید
&&
کسی نگفت که بیرون از این دوایی هست
\\
هزار نوبت اگر خاطرم بشورانی
&&
از این طرف که منم همچنان صفایی هست
\\
به دود آتش ماخولیا دماغ بسوخت
&&
هنوز جهل مصور که کیمیایی هست
\\
به کام دل نرسیدیم و جان به حلق رسید
&&
و گر به کام رسد همچنان رجایی هست
\\
به جان دوست که در اعتقاد سعدی نیست
&&
که در جهان به جز از کوی دوست جایی هست
\\
\end{longtable}
\end{center}
