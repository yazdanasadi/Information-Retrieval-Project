\begin{center}
\section*{غزل ۲۵۰: خوبرویان جفاپیشه وفا نیز کنند}
\label{sec:250}
\addcontentsline{toc}{section}{\nameref{sec:250}}
\begin{longtable}{l p{0.5cm} r}
خوبرویان جفاپیشه وفا نیز کنند
&&
به کسان درد فرستند و دوا نیز کنند
\\
پادشاهان ملاحت چو به نخجیر روند
&&
صید را پای ببندند و رها نیز کنند
\\
نظری کن به من خسته که ارباب کرم
&&
به ضعیفان نظر از بهر خدا نیز کنند
\\
عاشقان را ز بر خویش مران تا بر تو
&&
سر و زر هر دو فشانند و دعا نیز کنند
\\
گر کند میل به خوبان دل من عیب مکن
&&
کاین گناهیست که در شهر شما نیز کنند
\\
بوسه‌ای زان دهن تنگ بده یا بفروش
&&
کاین متاعیست که بخشند و بها نیز کنند
\\
تو ختایی بچه‌ای از تو خطا نیست عجب
&&
کان که از اهل صوابند خطا نیز کنند
\\
گر رود نام من اندر دهنت باکی نیست
&&
پادشاهان به غلط یاد گدا نیز کنند
\\
سعدیا گر نکند یاد تو آن ماه مرنج
&&
ما که باشیم که اندیشه ما نیز کنند
\\
\end{longtable}
\end{center}
