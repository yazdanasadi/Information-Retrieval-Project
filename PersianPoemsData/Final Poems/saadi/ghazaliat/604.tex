\begin{center}
\section*{غزل ۶۰۴: زنده بی دوست خفته در وطنی}
\label{sec:604}
\addcontentsline{toc}{section}{\nameref{sec:604}}
\begin{longtable}{l p{0.5cm} r}
زنده بی دوست خفته در وطنی
&&
مثل مرده‌ایست در کفنی
\\
عیش را بی تو عیش نتوان گفت
&&
چه بود بی وجود روح تنی
\\
تا صبا می‌رود به بستان‌ها
&&
چون تو سروی نیافت در چمنی
\\
و آفتابی خلاف امکان است
&&
که برآید ز جیب پیرهنی
\\
وآن شکن برشکن قبائل زلف
&&
که بلاییست زیر هر شکنی
\\
بر سر کوی عشق بازاریست
&&
که نیارد هزار جان ثمنی
\\
جای آن است اگر ببخشایی
&&
که نبینی فقیرتر ز منی
\\
هفت کشور نمی‌کنند امروز
&&
بی مقالات سعدی انجمنی
\\
از دو بیرون نه یا دلت سنگیست
&&
یا به گوشت نمی‌رسد سخنی
\\
\end{longtable}
\end{center}
