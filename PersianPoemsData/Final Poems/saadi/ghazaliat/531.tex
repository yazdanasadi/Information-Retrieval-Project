\begin{center}
\section*{غزل ۵۳۱: ای از بهشت جزوی و از رحمت آیتی}
\label{sec:531}
\addcontentsline{toc}{section}{\nameref{sec:531}}
\begin{longtable}{l p{0.5cm} r}
ای از بهشت جزوی و از رحمت آیتی
&&
حق را به روزگار تو با ما عنایتی
\\
گفتم نهایتی بود این درد عشق را
&&
هر بامداد می‌کند از نو بدایتی
\\
معروف شد حکایتم اندر جهان و نیست
&&
با تو مجال آن که بگویم حکایتی
\\
چندان که بی تو غایت امکان صبر بود
&&
کردیم و عشق را نه پدید است غایتی
\\
فرمان عشق و عقل به یک جای نشنوند
&&
غوغا بود دو پادشه اندر ولایتی
\\
ز ابنای روزگار به خوبی ممیزی
&&
چون در میان لشکر منصور رایتی
\\
عیبت نمی‌کنم که خداوند امر و نهی
&&
شاید که بنده‌ای بکشد بی جنایتی
\\
زان گه که عشق دست تطاول دراز کرد
&&
معلوم شد که عقل ندارد کفایتی
\\
من در پناه لطف تو خواهم گریختن
&&
فردا که هر کسی رود اندر حمایتی
\\
درمانده‌ام که از تو شکایت کجا برم
&&
هم با تو گر ز دست تو دارم شکایتی
\\
سعدی نهفته چند بماند حدیث عشق
&&
این ریش اندرون بکند هم سرایتی
\\
\end{longtable}
\end{center}
