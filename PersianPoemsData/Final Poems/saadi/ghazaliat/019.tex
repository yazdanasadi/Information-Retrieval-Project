\begin{center}
\section*{غزل ۱۹: کمان سخت که داد آن لطیف بازو را}
\label{sec:019}
\addcontentsline{toc}{section}{\nameref{sec:019}}
\begin{longtable}{l p{0.5cm} r}
کمان سخت که داد آن لطیف بازو را
&&
که تیر غمزه تمامست صید آهو را
\\
هزار صید دلت پیش تیر بازآید
&&
بدین صفت که تو داری کمان ابرو را
\\
تو خود به جوشن و برگستوان نه محتاجی
&&
که روز معرکه بر خود زره کنی مو را
\\
دیار هند و اقالیم ترک بسپارند
&&
چو چشم ترک تو بینند و زلف هندو را
\\
مغان که خدمت بت می‌کنند در فرخار
&&
ندیده‌اند مگر دلبران بت رو را
\\
حصار قلعه باغی به منجنیق مده
&&
به بام قصر برافکن کمند گیسو را
\\
مرا که عزلت عنقا گرفتمی همه عمر
&&
چنان اسیر گرفتی که باز تیهو را
\\
لبت بدیدم و لعلم بیوفتاد از چشم
&&
سخن بگفتی و قیمت برفت لؤلؤ را
\\
بهای روی تو بازار ماه و خور بشکست
&&
چنان که معجز موسی طلسم جادو را
\\
به رنج بردن بیهوده گنج نتوان برد
&&
که بخت راست فضیلت نه زور بازو را
\\
به عشق روی نکو دل کسی دهد سعدی
&&
که احتمال کند خوی زشت نیکو را
\\
\end{longtable}
\end{center}
