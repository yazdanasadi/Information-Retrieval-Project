\begin{center}
\section*{غزل ۳۵۲: جانان هزاران آفرین بر جانت از سر تا قدم}
\label{sec:352}
\addcontentsline{toc}{section}{\nameref{sec:352}}
\begin{longtable}{l p{0.5cm} r}
جانا هزاران آفرین بر جانت از سر تا قدم
&&
صانع خدایی کاین وجود آورد بیرون از عدم
\\
خورشید بر سرو روان دیگر ندیدم در جهان
&&
وصفت نگنجد در بیان نامت نیاید در قلم
\\
گفتم چو طاووسی مگر عضوی ز عضوی خوبتر
&&
می‌بینمت چون نیشکر شیرینی از سر تا قدم
\\
چندان که می‌بینم جفا امید می‌دارم وفا
&&
چشمانت می‌گویند لا ابروت می‌گوید نعم
\\
آخر نگاهی بازکن وانگه عتاب آغاز کن
&&
چندان که خواهی ناز کن چون پادشاهان بر خدم
\\
چون دل ببردی دین مبر هوش از من مسکین مبر
&&
با مهربانان کین مبر لاتقتلوا صید الحرم
\\
خار است و گل در بوستان هرچ او کند نیکوست آن
&&
سهل است پیش دوستان از دوستان بردن ستم
\\
او رفت و جان می‌پرورد این جامه بر خود می‌درد
&&
سلطان که خوابش می‌برد از پاسبانانش چه غم
\\
می‌زد به شمشیر جفا می‌رفت و می‌گفت از قفا
&&
سعدی بنالیدی ز ما مردان ننالند از الم
\\
\end{longtable}
\end{center}
