\begin{center}
\section*{غزل ۲۶۹: آن که مرا آرزوست دیر میسر شود}
\label{sec:269}
\addcontentsline{toc}{section}{\nameref{sec:269}}
\begin{longtable}{l p{0.5cm} r}
آنکه مرا آرزوست دیر میسر شود
&&
وینچه مرا در سرست عمر در این سر شود
\\
تا تو نیایی به فضل رفتن ما باطلست
&&
ور به مثل پای سعی در طلبت سر شود
\\
برق جمالی بجست خرمن خلقی بسوخت
&&
زان همه آتش نگفت دود دلی برشود
\\
ای نظر آفتاب هیچ زیان داردت
&&
گر در و دیوار ما از تو منور شود
\\
گر نگهی دوست وار بر طرف ما کنی
&&
حقه همان کیمیاست وین مس ما زر شود
\\
هوش خردمند را عشق به تاراج برد
&&
من نشنیدم که باز صید کبوتر شود
\\
گر تو چنین خوبروی بار دگر بگذری
&&
سنت پرهیزگار دین قلندر شود
\\
هر که به گل دربماند تا بنگیرند دست
&&
هر چه کند جهد بیش پای فروتر شود
\\
چون متصور شود در دل ما نقش دوست
&&
همچو بتش بشکنیم هر چه مصور شود
\\
پرتو خورشید عشق بر همه افتد ولیک
&&
سنگ به یک نوع نیست تا همه گوهر شود
\\
هر که به گوش قبول دفتر سعدی شنید
&&
دفتر وعظش به گوش همچو دف تر شود
\\
\end{longtable}
\end{center}
