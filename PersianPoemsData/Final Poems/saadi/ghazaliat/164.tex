\begin{center}
\section*{غزل ۱۶۴: دیدار یار غایب دانی چه ذوق دارد}
\label{sec:164}
\addcontentsline{toc}{section}{\nameref{sec:164}}
\begin{longtable}{l p{0.5cm} r}
دیدار یار غایب دانی چه ذوق دارد
&&
ابری که در بیابان بر تشنه‌ای ببارد
\\
ای بوی آشنایی دانستم از کجایی
&&
پیغام وصل جانان پیوند روح دارد
\\
سودای عشق پختن عقلم نمی‌پسندد
&&
فرمان عقل بردن عشقم نمی‌گذارد
\\
باشد که خود به رحمت یاد آورند ما را
&&
ور نه کدام قاصد پیغام ما گزارد
\\
هم عارفان عاشق دانند حال مسکین
&&
گر عارفی بنالد یا عاشقی بزارد
\\
زهرم چو نوشدارو از دست یار شیرین
&&
بر دل خوشست نوشم بی او نمی‌گوارد
\\
پایی که برنیارد روزی به سنگ عشقی
&&
گوییم جان ندارد یا دل نمی‌سپارد
\\
مشغول عشق جانان گر عاشقیست صادق
&&
در روز تیرباران باید که سر نخارد
\\
بی‌حاصلست یارا اوقات زندگانی
&&
الا دمی که یاری با همدمی برآرد
\\
دانی چرا نشیند سعدی به کنج خلوت
&&
کز دست خوبرویان بیرون شدن نیارد
\\
\end{longtable}
\end{center}
