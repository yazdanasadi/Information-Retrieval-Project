\begin{center}
\section*{غزل ۵۵: مجنون عشق را دگر امروز حالتست}
\label{sec:055}
\addcontentsline{toc}{section}{\nameref{sec:055}}
\begin{longtable}{l p{0.5cm} r}
مجنون عشق را دگر امروز حالت است
&&
کاسلام دین لیلی و دیگر ضلالت است
\\
فرهاد را از آن چه که شیرین ترش کند
&&
این را شکیب نیست گر آن را ملالت است
\\
عذرا که نانوشته بخواند حدیث عشق
&&
داند که آب دیدهٔ وامق رسالت است
\\
مطرب همین طریق غزل گو نگاه دار
&&
کاین ره که برگرفت به جایی دلالت است
\\
ای مدعی که می‌گذری بر کنار آب
&&
ما را که غرقه‌ایم ندانی چه حالت است
\\
زین در کجا رویم که ما را به خاک او
&&
واو را به خون ما که بریزد حوالت است
\\
گر سر قدم نمی‌کنمش پیش اهل دل
&&
سر بر نمی‌کنم که مقام خجالت است
\\
جز یاد دوست هر چه کنی عمر ضایع است
&&
جز سر عشق هر چه بگویی بطالت است
\\
ما را دگر معامله با هیچکس نماند
&&
بیعی که بی حضور تو کردم اقالت است
\\
از هر جفات بوی وفایی همی‌دهد
&&
در هر تعنتیت هزار استمالت است
\\
سعدی بشوی لوح دل از نقش غیر او
&&
علمی که ره به حق ننماید جهالت است
\\
\end{longtable}
\end{center}
