\begin{center}
\section*{غزل ۲۰۸: مرا به عاقبت این شوخ سیمتن بکشد}
\label{sec:208}
\addcontentsline{toc}{section}{\nameref{sec:208}}
\begin{longtable}{l p{0.5cm} r}
مرا به عاقبت این شوخ سیمتن بکشد
&&
چو شمع سوخته روزی در انجمن بکشد
\\
به لطف اگر بخرامد هزار دل ببرد
&&
به قهر اگر بستیزد هزار تن بکشد
\\
اگر خود آب حیاتست در دهان و لبش
&&
مرا عجب نبود کان لب و دهن بکشد
\\
گر ایستاد حریفی اسیر عشق بماند
&&
و گر گریخت خیالش به تاختن بکشد
\\
مرا که قوت کاهی نه کی دهد زنهار
&&
بلای عشق که فرهاد کوهکن بکشد
\\
کسان عتاب کنندم که ترک عشق بگوی
&&
به نقد اگر نکشد عشقم این سخن بکشد
\\
به شرع عابد اوثان اگر بباید کشت
&&
مرا چه حاجت کشتن که خود وثن بکشد
\\
به دوستی گله کردم ز چشم شوخش گفت
&&
عجب نباشد اگر مست تیغ زن بکشد
\\
به یک نفس که برآمیخت یار با اغیار
&&
بسی نماند که غیرت وجود من بکشد
\\
به خنده گفت که من شمع جمعم ای سعدی
&&
مرا از آن چه که پروانه خویشتن بکشد
\\
\end{longtable}
\end{center}
