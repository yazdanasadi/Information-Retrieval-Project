\begin{center}
\section*{غزل ۴۴۴: یا رب آن رویست یا برگ سمن}
\label{sec:444}
\addcontentsline{toc}{section}{\nameref{sec:444}}
\begin{longtable}{l p{0.5cm} r}
یا رب آن روی است یا برگ سمن
&&
یا رب آن قد است یا سرو چمن
\\
بر سمن کس دید جعد مشکبار
&&
در چمن کس دید سرو سیمتن
\\
عقل چون پروانه گردید و نیافت
&&
چون تو شمعی در هزاران انجمن
\\
سخت مشتاقیم پیمانی بکن
&&
سخت مجروحیم پیکانی بکن
\\
وه کدامت زین همه شیرین‌تر است
&&
خنده یا رفتار یا لب یا سخن
\\
گر سر ما خواهی اینک جان و سر
&&
ور سر ما داری اینک مال و تن
\\
گر نوازی ور کشی فرمان تو راست
&&
بنده‌ایم اینک سر و تیغ و کفن
\\
صعقه می‌خواهی حجابی در گذار
&&
فتنه می‌جویی نقابی بر فکن
\\
من کیم کانجا که کوی عشق توست
&&
در نمی‌گنجد حدیث ما و من
\\
ای ز وصلت خانه‌ها دار الشفا
&&
وی ز هجرت بیت‌ها بیت الحزن
\\
وقت آن آمد که خاک مرده را
&&
باد ریزد آب حیوان در دهن
\\
پاره گرداند زلیخای صبا
&&
صبحدم بر یوسف گل پیرهن
\\
نطفه شبنم در ارحام زمین
&&
شاهد گل گشت و طفل یاسمن
\\
فیح ریحان است یا بوی بهشت
&&
خاک شیراز است یا باد ختن
\\
بر گذر تا خیره گردد سروبن
&&
در نگر تا تیره گردد نسترن
\\
بارگاه زاهدان در هم نورد
&&
کارگاه صوفیان بر هم شکن
\\
شاهدان چستند ساقی گو بیار
&&
عاشقان مستند مطرب گو بزن
\\
سغبه خلقم چو صوفی در کنش
&&
شهره شهرم چو غازی بر رسن
\\
تربیت را حله گو در ما مپوش
&&
عافیت را پرده گو بر ما متن
\\
چرخ با صد چشم چون روی تو دید
&&
صد زبان می‌خواست تا گوید حسن
\\
ناسزا خواهم شنید از خاص و عام
&&
سرزنش خواهم کشید از مرد و زن
\\
سعدیا گر عاشقی پایی بکوب
&&
عاشقا گر مفلسی دستی بزن
\\
\end{longtable}
\end{center}
