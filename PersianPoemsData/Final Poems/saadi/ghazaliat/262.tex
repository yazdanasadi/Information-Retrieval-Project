\begin{center}
\section*{غزل ۲۶۲: عیبی نباشد از تو که بر ما جفا رود}
\label{sec:262}
\addcontentsline{toc}{section}{\nameref{sec:262}}
\begin{longtable}{l p{0.5cm} r}
عیبی نباشد از تو که بر ما جفا رود
&&
مجنون از آستانه لیلی کجا رود
\\
گر من فدای جان تو گردم دریغ نیست
&&
بسیار سر که در سر مهر و وفا رود
\\
ور من گدای کوی تو باشم غریب نیست
&&
قارون اگر به خیل تو آید گدا رود
\\
مجروح تیر عشق اگرش تیغ بر قفاست
&&
چون می‌رود ز پیش تو چشم از قفا رود
\\
حیف آیدم که پای همی بر زمین نهی
&&
کاین پای لایقست که بر چشم ما رود
\\
در هیچ موقفم سر گفت و شنید نیست
&&
الا در آن مقام که ذکر شما رود
\\
ای هوشیار اگر به سر مست بگذری
&&
عیبش مکن که بر سر مردم قضا رود
\\
ما چون نشانه پای به گل در بمانده‌ایم
&&
خصم آن حریف نیست که تیرش خطا رود
\\
ای آشنای کوی محبت صبور باش
&&
بیداد نیکوان همه بر آشنا رود
\\
سعدی به در نمی‌کنی از سر هوای دوست
&&
در پات لازمست که خار جفا رود
\\
\end{longtable}
\end{center}
