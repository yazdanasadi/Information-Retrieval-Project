\begin{center}
\section*{غزل ۳۸۲: من چون تو به دلبری ندیدم}
\label{sec:382}
\addcontentsline{toc}{section}{\nameref{sec:382}}
\begin{longtable}{l p{0.5cm} r}
من چون تو به دلبری ندیدم
&&
گلبرگ چنین طری ندیدم
\\
مانند تو آدمی در آفاق
&&
ممکن نبود پری ندیدم
\\
وین بوالعجبی و چشم بندی
&&
در صنعت سامری ندیدم
\\
با روی تو ماه آسمان را
&&
امکان برابری ندیدم
\\
لعلی چو لب شکرفشانت
&&
در کلبه جوهری ندیدم
\\
چون در دو رسته دهانت
&&
نظم سخن دری ندیدم
\\
مه را که خرد که من به کرات
&&
مه دیدم و مشتری ندیدم
\\
وین پرده راز پارسایان
&&
چندان که تو می‌دری ندیدم
\\
دیدم همه دلبران آفاق
&&
چون تو به دلاوری ندیدم
\\
جوری که تو می‌کنی در اسلام
&&
در ملت کافری ندیدم
\\
سعدی غم عشق خوبرویان
&&
چندان که تو می‌خوری ندیدم
\\
دیدم همه صوفیان آفاق
&&
مثل تو قلندری ندیدم
\\
\end{longtable}
\end{center}
