\begin{center}
\section*{غزل ۴۵۱: دو چشم مست میگونت ببرد آرام هشیاران}
\label{sec:451}
\addcontentsline{toc}{section}{\nameref{sec:451}}
\begin{longtable}{l p{0.5cm} r}
دو چشم مست میگونت ببرد آرام هشیاران
&&
دو خواب آلوده بربودند عقل از دست بیداران
\\
نصیحتگوی را از من بگو ای خواجه دم درکش
&&
چو سیل از سر گذشت آن را چه می‌ترسانی از باران
\\
گر آن ساقی که مستان راست هشیاران بدیدندی
&&
ز توبه توبه کردندی چو من بر دست خماران
\\
گرم با صالحان بی دوست فردا در بهشت آرند
&&
همان بهتر که در دوزخ کنندم با گنهکاران
\\
چه بوی است این که عقل از من ببرد و صبر و هشیاری
&&
ندانم باغ فردوس است یا بازار عطاران
\\
تو با این مردم کوته نظر در چاه کنعانی
&&
به مصر آ تا پدید آیند یوسف را خریداران
\\
الا ای باد شبگیری بگوی آن ماه مجلس را
&&
تو آزادی و خلقی در غم رویت گرفتاران
\\
گر آن عیار شهرآشوب روزی حال من پرسد
&&
بگو خوابش نمی‌گیرد به شب از دست عیاران
\\
گرت باری گذر باشد نگه با جانب ما کن
&&
نپندارم که بد باشد جزای خوب کرداران
\\
کسان گویند چون سعدی جفا دیدی تحول کن
&&
رها کن تا بمیرم بر سر کوی وفاداران
\\
\end{longtable}
\end{center}
