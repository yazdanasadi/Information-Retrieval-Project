\begin{center}
\section*{غزل ۸۳: مگر نسیم سحر بوی زلف یار منست}
\label{sec:083}
\addcontentsline{toc}{section}{\nameref{sec:083}}
\begin{longtable}{l p{0.5cm} r}
مگر نسیم سحر بوی زلف یار منست
&&
که راحت دل رنجور بی‌قرار منست
\\
به خواب درنرود چشم بخت من همه عمر
&&
گرش به خواب ببینم که در کنار منست
\\
اگر معاینه بینم که قصد جان دارد
&&
به جان مضایقه با دوستان نه کار منست
\\
حقیقت آن که نه درخورد اوست جان عزیز
&&
ولیک درخور امکان و اقتدار منست
\\
نه اختیار منست این معاملت لیکن
&&
رضای دوست مقدم بر اختیار منست
\\
اگر هزار غمست از جفای او بر دل
&&
هنوز بنده اویم که غمگسار منست
\\
درون خلوت ما غیر در نمی‌گنجد
&&
برو که هر که نه یار منست بار منست
\\
به لاله زار و گلستان نمی‌رود دل من
&&
که یاد دوست گلستان و لاله زار منست
\\
ستمگرا دل سعدی بسوخت در طلبت
&&
دلت نسوخت که مسکین امیدوار منست
\\
و گر مراد تو اینست بی مرادی من
&&
تفاوتی نکند چون مراد یار منست
\\
\end{longtable}
\end{center}
