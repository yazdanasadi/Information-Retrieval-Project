\begin{center}
\section*{غزل ۳۵۸: زهی سعادت من کم تو آمدی به سلام}
\label{sec:358}
\addcontentsline{toc}{section}{\nameref{sec:358}}
\begin{longtable}{l p{0.5cm} r}
زهی سعادت من که‌م تو آمدی به سلام
&&
خوش آمدی و علیک السلام و الاکرام
\\
قیام خواستمت کرد عقل می‌گوید
&&
مکن که شرط ادب نیست پیش سرو قیام
\\
اگر کساد شکر بایدت دهن بگشای
&&
ورت خجالت سرو آرزو کند بخرام
\\
تو آفتاب منیری و دیگران انجم
&&
تو روح پاکی و ابنای روزگار اجسام
\\
اگر تو آدمیی اعتقاد من این است
&&
که دیگران همه نقشند بر در حمام
\\
تنک مپوش که اندام‌های سیمینت
&&
درون جامه پدید است چون گلاب از جام
\\
از اتفاق چه خوشتر بود میان دو دوست
&&
درون پیرهنی چون دو مغز یک بادام
\\
سماع اهل دل آواز ناله سعدیست
&&
چه جای زمزمه عندلیب و سجع حمام
\\
در این سماع همه ساقیان شاهدروی
&&
بر این شراب همه صوفیان دردآشام
\\
\end{longtable}
\end{center}
