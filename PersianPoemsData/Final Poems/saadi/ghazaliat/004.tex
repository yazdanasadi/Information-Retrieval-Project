\begin{center}
\section*{غزل ۴: اگر تو فارغی از حال دوستان یارا}
\label{sec:004}
\addcontentsline{toc}{section}{\nameref{sec:004}}
\begin{longtable}{l p{0.5cm} r}
اگر تو فارغی از حال دوستان یارا
&&
فراغت از تو میسر نمی‌شود ما را
\\
تو را در آینه دیدن جمال طلعت خویش
&&
بیان کند که چه بودست ناشکیبا را
\\
بیا که وقت بهارست تا من و تو به هم
&&
به دیگران بگذاریم باغ و صحرا را
\\
به جای سرو بلند ایستاده بر لب جوی
&&
چرا نظر نکنی یار سروبالا را
\\
شمایلی که در اوصاف حسن ترکیبش
&&
مجال نطق نماند زبان گویا را
\\
که گفت در رخ زیبا نظر خطا باشد
&&
خطا بود که نبینند روی زیبا را
\\
به دوستی که اگر زهر باشد از دستت
&&
چنان به ذوق ارادت خورم که حلوا را
\\
کسی ملامت وامق کند به نادانی
&&
حبیب من که ندیدست روی عذرا را
\\
گرفتم آتش پنهان خبر نمی‌داری
&&
نگاه می‌نکنی آب چشم پیدا را
\\
نگفتمت که به یغما رود دلت سعدی
&&
چو دل به عشق دهی دلبران یغما را
\\
هنوز با همه دردم امید درمانست
&&
که آخری بود آخر شبان یلدا را
\\
\end{longtable}
\end{center}
