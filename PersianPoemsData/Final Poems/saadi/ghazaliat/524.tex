\begin{center}
\section*{غزل ۵۲۴: یارا قدحی پر کن از آن داروی مستی}
\label{sec:524}
\addcontentsline{toc}{section}{\nameref{sec:524}}
\begin{longtable}{l p{0.5cm} r}
یارا قدحی پر کن از آن داروی مستی
&&
تا از سر صوفی برود علت هستی
\\
عاقل متفکر بود و مصلحت اندیش
&&
در مذهب عشق آی و از این جمله برستی
\\
ای فتنه نوخاسته از عالم قدرت
&&
غایب مشو از دیده که در دل بنشستی
\\
آرام دلم بستدی و دست شکیبم
&&
برتافتی و پنجه صبرم بشکستی
\\
احوال دو چشم من بر هم ننهاده
&&
با تو نتوان گفت به خواب شب مستی
\\
سودازده‌ای کز همه عالم به تو پیوست
&&
دل نیک بدادت که دل از وی بگسستی
\\
در روی تو گفتم سخنی چند بگویم
&&
رو باز گشادی و در نطق ببستی
\\
گر باده از این خم بود و مطرب از این کوی
&&
ما توبه بخواهیم شکستن به درستی
\\
سعدی غرض از حقه تن آیت حق است
&&
صد تعبیه در توست و یکی باز نجستی
\\
نقاش وجود این همه صورت که بپرداخت
&&
تا نقش ببینی و مصور بپرستی
\\
\end{longtable}
\end{center}
