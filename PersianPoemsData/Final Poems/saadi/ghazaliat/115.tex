\begin{center}
\section*{غزل ۱۱۵: کیست آن کش سر پیوند تو در خاطر نیست}
\label{sec:115}
\addcontentsline{toc}{section}{\nameref{sec:115}}
\begin{longtable}{l p{0.5cm} r}
کیست آن کش سر پیوند تو در خاطر نیست
&&
یا نظر با تو ندارد مگرش ناظر نیست
\\
نه حلالست که دیدار تو بیند هر کس
&&
که حرامست بر آن کش نظری طاهر نیست
\\
همه کس را مگر این ذوق نباشد که مرا
&&
کان چه من می‌نگرم بر دگری ظاهر نیست
\\
هر شبی روزی و هر روز زوالی دارد
&&
شب وصل من و معشوق مرا آخر نیست
\\
هر که با غمزه خوبان سر و کاری دارد
&&
سست مهرست که بر داغ جفا صابر نیست
\\
هر که سرپنجه مخضوب تو بیند گوید
&&
گر بر این دست کسی کشته شود نادر نیست
\\
سر موییم نظر کن که من اندر تن خویش
&&
یک سر موی ندانم که تو را ذاکر نیست
\\
همه دانند که سودازده دلشده را
&&
چاره صبرست ولیکن چه کند قادر نیست
\\
گفته بودم غم دل با تو بگویم چندی
&&
به زبان چند بگویم که دلم حاضر نیست
\\
گر من از چشم همه خلق بیفتم سهلست
&&
تو مپندار که مخذول تو را ناصر نیست
\\
التفات از همه عالم به تو دارد سعدی
&&
همتی کان به تو مصروف بود قاصر نیست
\\
\end{longtable}
\end{center}
