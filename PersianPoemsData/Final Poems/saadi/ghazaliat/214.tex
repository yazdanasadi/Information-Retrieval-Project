\begin{center}
\section*{غزل ۲۱۴: سرمست ز کاشانه به گلزار برآمد}
\label{sec:214}
\addcontentsline{toc}{section}{\nameref{sec:214}}
\begin{longtable}{l p{0.5cm} r}
سرمست ز کاشانه به گلزار برآمد
&&
غلغل ز گل و لاله به یک بار برآمد
\\
مرغان چمن نعره زنان دیدم و گویان
&&
زین غنچه که از طرف چمنزار برآمد
\\
آب از گل رخساره او عکس پذیرفت
&&
و آتش به سر غنچه گلنار برآمد
\\
سجاده نشینی که مرید غم او شد
&&
آوازه اش از خانه خمار برآمد
\\
زاهد چو کرامات بت عارض او دید
&&
از چله میان بسته به زنار برآمد
\\
بر خاک چو من بی‌دل و دیوانه نشاندش
&&
اندر نظر هر که پری وار برآمد
\\
من مفلس از آن روز شدم کز حرم غیب
&&
دیبای جمال تو به بازار برآمد
\\
کام دلم آن بود که جان بر تو فشانم
&&
آن کام میسر شد و این کار برآمد
\\
سعدی چمن آن روز به تاراج خزان داد
&&
کز باغ دلش بوی گل یار برآمد
\\
\end{longtable}
\end{center}
