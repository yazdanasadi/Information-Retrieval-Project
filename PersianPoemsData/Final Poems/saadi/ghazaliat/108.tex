\begin{center}
\section*{غزل ۱۰۸: مرا خود با تو چیزی در میان هست}
\label{sec:108}
\addcontentsline{toc}{section}{\nameref{sec:108}}
\begin{longtable}{l p{0.5cm} r}
مرا خود با تو چیزی در میان هست
&&
و گر نه روی زیبا در جهان هست
\\
وجودی دارم از مهرت گدازان
&&
وجودم رفت و مهرت همچنان هست
\\
مبر ظن کز سرم سودای عشقت
&&
رود تا بر زمینم استخوان هست
\\
اگر پیشم نشینی دل نشانی
&&
و گر غایب شوی در دل نشان هست
\\
به گفتن راست ناید شرح حسنت
&&
ولیکن گفت خواهم تا زبان هست
\\
ندانم قامتست آن یا قیامت
&&
که می‌گوید چنین سرو روان هست
\\
توان گفتن به مه مانی ولی ماه
&&
نپندارم چنین شیرین دهان هست
\\
بجز پیشت نخواهم سر نهادن
&&
اگر بالین نباشد آستان هست
\\
برو سعدی که کوی وصل جانان
&&
نه بازاریست کان جا قدر جان هست
\\
\end{longtable}
\end{center}
