\begin{center}
\section*{غزل ۲۳۸: بخرام بالله تا صبا بیخ صنوبر برکند}
\label{sec:238}
\addcontentsline{toc}{section}{\nameref{sec:238}}
\begin{longtable}{l p{0.5cm} r}
بخرام بالله تا صبا بیخ صنوبر برکند
&&
برقع برافکن تا بهشت از حور زیور برکند
\\
زان روی و خال دلستان برکش نقاب پرنیان
&&
تا پیش رویت آسمان آن خال اختر برکند
\\
خلقی چو من بر روی تو آشفته همچون موی تو
&&
پای آن نهد در کوی تو کاول دل از سر برکند
\\
زان عارض فرخنده خو نه رنگ دارد گل نه بو
&&
انگشت غیرت را بگو تا چشم عبهر برکند
\\
ما خار غم در پای جان در کویت ای گلرخ روان
&&
وان گه که را پروای آن کز پای نشتر برکند
\\
ماه است رویت یا ملک قند است لعلت یا نمک
&&
بنمای پیکر تا فلک مهر از دوپیکر برکند
\\
باری به ناز و دلبری گر سوی صحرا بگذری
&&
واله شود کبک دری طاووس شهپر برکند
\\
سعدی چو شد هندوی تو هل تا پرستد روی تو
&&
کاو خیمه زد پهلوی تو فردای محشر برکند
\\
\end{longtable}
\end{center}
