\begin{center}
\section*{غزل ۵۱۰: نه من تنها گرفتارم به دام زلف زیبایی}
\label{sec:510}
\addcontentsline{toc}{section}{\nameref{sec:510}}
\begin{longtable}{l p{0.5cm} r}
نه من تنها گرفتارم به دام زلف زیبایی
&&
که هر کس با دلارامی سری دارند و سودایی
\\
قرین یار زیبا را چه پروای چمن باشد
&&
هزاران سرو بستانی فدای سروبالایی
\\
مرا نسبت به شیدایی کند ماه پری پیکر
&&
تو دل با خویشتن داری چه دانی حال شیدایی
\\
همی‌دانم که فریادم به گوشش می‌رسد لیکن
&&
ملولی را چه غم دارد ز حال ناشکیبایی
\\
عجب دارند یارانم که دستش را همی‌بوسم
&&
ندیدستند مسکینان سری افتاده در پایی
\\
اگر فرهاد را حاصل نشد پیوند با شیرین
&&
نه آخر جان شیرینش برآمد در تمنایی
\\
خرد با عشق می‌کوشد که وی را در کمند آرد
&&
ولیکن بر نمی‌آید ضعیفی با توانایی
\\
مرا وقتی ز نزدیکان ملامت سخت می‌آمد
&&
نترسم دیگر از باران که افتادم به دریایی
\\
تو خواهی خشم بر ما گیر و خواهی چشم بر ما کن
&&
که ما را با کسی دیگر نمانده‌ست از تو پروایی
\\
نپندارم که سعدی را بیازاری و بگذاری
&&
که بعد از سایه لطفت ندارد در جهان جایی
\\
من آن خاک وفادارم که از من بوی مهر آید
&&
و گر بادم برد چون شعر هر جزوی به اقصایی
\\
\end{longtable}
\end{center}
