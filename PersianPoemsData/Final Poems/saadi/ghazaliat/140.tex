\begin{center}
\section*{غزل ۱۴۰: چشمت چو تیغ غمزه خون خوار برگرفت}
\label{sec:140}
\addcontentsline{toc}{section}{\nameref{sec:140}}
\begin{longtable}{l p{0.5cm} r}
چشمت چو تیغ غمزه خونخوار برگرفت
&&
با عقل و هوش خلق به پیکار برگرفت
\\
عاشق ز سوز درد تو فریاد درنهاد
&&
مؤمن ز دست عشق تو زنار برگرفت
\\
عشقت بنای عقل به کلی خراب کرد
&&
جورت در امید به یک بار برگرفت
\\
شوری ز وصف روی تو در خانگه فتاد
&&
صوفی طریق خانه خمار برگرفت
\\
با هر که مشورت کنم از جور آن صنم
&&
گوید ببایدت دل از این کار برگرفت
\\
دل برتوانم از سر و جان برگرفت و چشم
&&
نتوانم از مشاهده یار برگرفت
\\
سعدی به خفیه خون جگر خورد بارها
&&
این بار پرده از سر اسرار برگرفت
\\
\end{longtable}
\end{center}
