\begin{center}
\section*{غزل ۵۹: این تویی یا سرو بستانی به رفتار آمدست}
\label{sec:059}
\addcontentsline{toc}{section}{\nameref{sec:059}}
\begin{longtable}{l p{0.5cm} r}
آن تویی یا سرو بستانی به رفتار آمده‌ست
&&
یا ملک در صورت مردم به گفتار آمده‌ست
\\
آن پری کز خلق پنهان بود چندین روزگار
&&
باز می‌بینم که در عالم پدیدار آمده‌ست
\\
عود می‌سوزند یا گل می‌دمد در بوستان
&&
دوستان یا کاروان مشک تاتار آمده‌ست
\\
تا مرا با نقش رویش آشنایی اوفتاد
&&
هر چه می‌بینم به چشمم نقش دیوار آمده‌ست
\\
ساربانا یک نظر در روی آن زیبا نگار
&&
گر به جانی می‌دهد اینک خریدار آمده‌ست
\\
من دگر در خانه ننشینم اسیر و دردمند
&&
خاصه این ساعت که گفتی گل به بازار آمده‌ست
\\
گر تو انکار نظر در آفرینش می‌کنی
&&
من همی‌گویم که چشم از بهر این کار آمده‌ست
\\
وه که گر من بازبینم روی یار خویش را
&&
مرده‌ای بینی که با دنیا دگربار آمده‌ست
\\
آن چه بر من می‌رود در بندت ای آرام جان
&&
با کسی گویم که در بندی گرفتار آمده‌ست
\\
نی که می‌نالد همی در مجلس آزادگان
&&
زان همی‌نالد که بر وی زخم بسیار آمده‌ست
\\
تا نپنداری که بعد از چشم خواب آلود تو
&&
تا برفتی خوابم اندر چشم بیدار آمده‌ست
\\
سعدیا گر همتی داری منال از جور یار
&&
تا جهان بوده‌ست جور یار بر یار آمده‌ست
\\
\end{longtable}
\end{center}
