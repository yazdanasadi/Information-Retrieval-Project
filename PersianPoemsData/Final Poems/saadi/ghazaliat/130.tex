\begin{center}
\section*{غزل ۱۳۰: دوش دور از رویت ای جان جانم از غم تاب داشت}
\label{sec:130}
\addcontentsline{toc}{section}{\nameref{sec:130}}
\begin{longtable}{l p{0.5cm} r}
دوش دور از رویت ای جان جانم از غم تاب داشت
&&
ابر چشمم بر رخ از سودای دل سیلاب داشت
\\
در تفکر عقل مسکین پایمال عشق شد
&&
با پریشانی دل شوریده چشم  خواب داشت
\\
کوس غارت زد فراقت گرد شهرستان دل
&&
شحنه عشقت سرای عقل در طبطاب داشت
\\
نقش نامت کرده دل محراب تسبیح وجود
&&
تا سحر تسبیح گویان روی در محراب داشت
\\
دیده‌ام می‌جست و گفتندم نبینی روی دوست
&&
خود درفشان بود چشمم کاندر او سیماب داشت
\\
ز آسمان آغاز کارم سخت شیرین می‌نمود
&&
کی گمان بردم که شهدآلوده زهر ناب داشت
\\
سعدی این ره مشکل افتادست در دریای عشق
&&
اول آخر در صبوری اندکی پایاب داشت
\\
\end{longtable}
\end{center}
