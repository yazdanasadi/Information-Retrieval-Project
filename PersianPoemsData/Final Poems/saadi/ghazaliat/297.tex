\begin{center}
\section*{غزل ۲۹۷: زنده کدامست بر هوشیار}
\label{sec:297}
\addcontentsline{toc}{section}{\nameref{sec:297}}
\begin{longtable}{l p{0.5cm} r}
زنده کدام است بر هوشیار
&&
آن که بمیرد به سر کوی یار
\\
عاشق دیوانه سرمست را
&&
پند خردمند نیاید به کار
\\
سر که به کشتن بنهی پیش دوست
&&
به که به گشتن بنهی در دیار
\\
ای که دلم بردی و جان سوختی
&&
در سر سودای تو شد روزگار
\\
شربت زهر ار تو دهی نیست تلخ
&&
کوه احد گر تو نهی نیست بار
\\
بندی مهر تو نیابد خلاص
&&
غرقه عشق تو نبیند کنار
\\
درد نهانی دل تنگم بسوخت
&&
لاجرمم عشق ببود آشکار
\\
در دلم آرام تصور مکن
&&
وز مژه‌ام خواب توقع مدار
\\
گر گله از ماست شکایت بگوی
&&
ور گنه از توست غرامت بیار
\\
بر سر پا عذر نباشد قبول
&&
تا ننشینی ننشیند غبار
\\
دل چه محل دارد و دینار چیست
&&
مدعیم گر نکنم جان نثار
\\
سعدی اگر زخم خوری غم مخور
&&
فخر بود داغ خداوندگار
\\
\end{longtable}
\end{center}
