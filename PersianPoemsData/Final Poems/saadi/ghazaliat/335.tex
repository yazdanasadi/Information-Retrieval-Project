\begin{center}
\section*{غزل ۳۳۵: یکی را دست حسرت بر بناگوش}
\label{sec:335}
\addcontentsline{toc}{section}{\nameref{sec:335}}
\begin{longtable}{l p{0.5cm} r}
یکی را دست حسرت بر بناگوش
&&
یکی با آن که می‌خواهد در آغوش
\\
نداند دوش بر دوش حریفان
&&
که تنها مانده چون خفت از غمش دوش
\\
نکوگویان نصیحت می‌کنندم
&&
ز من فریاد می‌آید که خاموش
\\
ز بانگ رود و آوای سرودم
&&
دگر جای نصیحت نیست در گوش
\\
مرا گویند چشم از وی بپوشان
&&
ورا گو برقعی بر خویشتن پوش
\\
نشانی زان پری تا در خیال است
&&
نیاید هرگز این دیوانه با هوش
\\
نمی‌شاید گرفتن چشمه چشم
&&
که دریای درون می‌آورد جوش
\\
بیا تا هر چه هست از دست محبوب
&&
بیاشامیم اگر زهر است اگر نوش
\\
مرا در خاک راه دوست بگذار
&&
برو گو دشمن اندر خون من کوش
\\
نه یاری سست پیمان است سعدی
&&
که در سختی کند یاری فراموش
\\
\end{longtable}
\end{center}
