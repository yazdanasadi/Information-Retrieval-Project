\begin{center}
\section*{غزل ۵۶۵: من از تو روی نپیچم گرم بیازاری}
\label{sec:565}
\addcontentsline{toc}{section}{\nameref{sec:565}}
\begin{longtable}{l p{0.5cm} r}
من از تو روی نپیچم گرم بیازاری
&&
که خوش بود ز عزیزان تحمل خواری
\\
به هر سلاح که خون مرا بخواهی ریخت
&&
حلال کردمت الا به تیغ بیزاری
\\
تو در دل من از آن خوشتری و شیرین‌تر
&&
که من ترش بنشینم ز تلخ گفتاری
\\
اگر دعات ارادت بود و گر دشنام
&&
بگوی از آن لب شیرین که شهد می‌باری
\\
اگر به صید روی وحشی از تو نگریزد
&&
که در کمند تو راحت بود گرفتاری
\\
به انتظار عیادت که دوست می‌آید
&&
خوش است بر دل رنجور عشق بیماری
\\
گرم تو زهر دهی چون عسل بیاشامم
&&
به شرط آن که به دست رقیب نسپاری
\\
تو می‌روی و مرا چشم و دل به جانب توست
&&
ولی چه سود که جانب نگه نمی‌داری
\\
گرت چو من غم عشقی زمانه پیش آرد
&&
دگر غم همه عالم به هیچ نشماری
\\
درازنای شب از چشم دردمندان پرس
&&
که هر چه پیش تو سهل است سهل پنداری
\\
حکایت من و مجنون به یکدگر ماند
&&
نیافتیم و بمردیم در طلبکاری
\\
بنال سعدی اگر چاره وصالت نیست
&&
که نیست چاره بیچارگان به جز زاری
\\
\end{longtable}
\end{center}
