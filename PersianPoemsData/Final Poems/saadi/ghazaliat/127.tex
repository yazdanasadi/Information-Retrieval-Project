\begin{center}
\section*{غزل ۱۲۷: دل نماندست که گوی خم چوگان تو نیست}
\label{sec:127}
\addcontentsline{toc}{section}{\nameref{sec:127}}
\begin{longtable}{l p{0.5cm} r}
دل نماندست که گوی خم چوگان تو نیست
&&
خصم را پای گریز از سر میدان تو نیست
\\
تا سر زلف پریشان تو در جمع آمد
&&
هیچ مجموع ندانم که پریشان تو نیست
\\
در تو حیرانم و اوصاف معانی که تو راست
&&
و اندر آن کس که بصر دارد و حیران تو نیست
\\
آن چه عیبست که در صورت زیبای تو هست
&&
وان چه سحرست که در غمزه فتان تو نیست
\\
آب حیوان نتوان گفت که در عالم هست
&&
گر چنانست که در چاه زنخدان تو نیست
\\
از خدا آمده‌ای آیت رحمت بر خلق
&&
وان کدام آیت لطفست که در شأن تو نیست
\\
گر تو را هست شکیب از من و امکان فراغ
&&
به وصالت که مرا طاقت هجران تو نیست
\\
تو کجا نالی از این خار که در پای منست
&&
یا چه غم داری از این درد که بر جان تو نیست
\\
دردی از حسرت دیدار تو دارم که طبیب
&&
عاجز آمد که مرا چاره درمان تو نیست
\\
آخر ای کعبه مقصود کجا افتادی
&&
که خود از هیچ طرف حد بیابان تو نیست
\\
گر برانی چه کند بنده که فرمان نبرد
&&
ور بخوانی عجب از غایت احسان تو نیست
\\
سعدی از بند تو هرگز به درآید هیهات
&&
بلکه حیفست بر آن کس که به زندان تو نیست
\\
\end{longtable}
\end{center}
