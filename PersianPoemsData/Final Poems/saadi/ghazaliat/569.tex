\begin{center}
\section*{غزل ۵۶۹: حدیث یا شکرست آن که در دهان داری}
\label{sec:569}
\addcontentsline{toc}{section}{\nameref{sec:569}}
\begin{longtable}{l p{0.5cm} r}
حدیث یا شکر است آن که در دهان داری
&&
دوم به لطف نگویم که در جهان داری
\\
گناه عاشق بیچاره نیست در پی تو
&&
گناه توست که رخسار دلستان داری
\\
جمال عارض خورشید و حسن قامت سرو
&&
تو را رسد که چو دعوی کنی بیان داری
\\
ندانم ای کمر این سلطنت چه لایق توست
&&
که با چنین صنمی دست در میان داری
\\
بسیست تا دل گم کرده باز می‌جستم
&&
در ابروان تو بشناختم که آن داری
\\
تو را که زلف و بناگوش و خد و قد اینست
&&
مرو به باغ که در خانه بوستان داری
\\
بدین صفت که تویی دل چه جای خدمت توست
&&
فراتر آی که ره در میان جان داری
\\
گر این روش که تو طاووس می‌کنی رفتار
&&
نه برج من که همه عالم آشیان داری
\\
قدم ز خانه چو بیرون نهی به عزت نه
&&
که خون دیده سعدی بر آستان داری
\\
\end{longtable}
\end{center}
