\begin{center}
\section*{غزل ۳۴۴: ساقی بده آن شراب گلرنگ}
\label{sec:344}
\addcontentsline{toc}{section}{\nameref{sec:344}}
\begin{longtable}{l p{0.5cm} r}
ساقی بده آن شراب گلرنگ
&&
مطرب بزن آن نوای بر چنگ
\\
کز زهد ندیده‌ام فتوحی
&&
تا کی زنم آبگینه بر سنگ
\\
خون شد دل من ندیده کامی
&&
الا که برفت نام با ننگ
\\
عشق آمد و عقل همچو بادی
&&
رفت از بر من هزار فرسنگ
\\
ای زاهد خرقه پوش تا کی
&&
با عاشق خسته دل کنی جنگ
\\
گرد دو جهان بگشته عاشق
&&
زاهد بنگر نشسته دلتنگ
\\
من خرقه فکنده‌ام ز عشقت
&&
باشد که به وصل تو زنم چنگ
\\
سعدی همه روز عشق می‌باز
&&
تا در دو جهان شوی به یک رنگ
\\
\end{longtable}
\end{center}
