\begin{center}
\section*{غزل ۳۹: بی تو حرامست به خلوت نشست}
\label{sec:039}
\addcontentsline{toc}{section}{\nameref{sec:039}}
\begin{longtable}{l p{0.5cm} r}
بی تو حرام است به خلوت نشست
&&
حیف بود در به چنین روی بست
\\
دامن دولت چو به دست اوفتاد
&&
گر بهلی بازنیاید به دست
\\
این چه نظر بود که خونم بریخت
&&
وین چه نمک بود که ریشم بخست
\\
هر که بیفتاد به تیرت نخاست
&&
وان که درآمد به کمندت نجست
\\
ما به تو یک باره مقید شدیم
&&
مرغ به دام آمد و ماهی به شست
\\
صبر قفا خورد و به راهی گریخت
&&
عقل بلا دید و به کنجی نشست
\\
بار مذلت بتوانم کشید
&&
عهد محبت نتوانم شکست
\\
وین رمقی نیز که هست از وجود
&&
پیش وجودت نتوان گفت هست
\\
هرگز اگر راه به معنی برد
&&
سجده صورت نکند بت پرست
\\
مستی خمرش نکند آرزو
&&
هر که چو سعدی شود از عشق مست
\\
\end{longtable}
\end{center}
