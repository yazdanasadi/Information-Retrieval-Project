\begin{center}
\section*{غزل ۲۰۲: جنگ از طرف دوست دل آزار نباشد}
\label{sec:202}
\addcontentsline{toc}{section}{\nameref{sec:202}}
\begin{longtable}{l p{0.5cm} r}
جنگ از طرف دوست دل آزار نباشد
&&
یاری که تحمل نکند یار نباشد
\\
گر بانگ برآید که سری در قدمی رفت
&&
بسیار مگویید که بسیار نباشد
\\
آن بار که گردون نکشد یار سبکروح
&&
گر بر دل عشاق نهد بار نباشد
\\
تا رنج تحمل نکنی گنج نبینی
&&
تا شب نرود صبح پدیدار نباشد
\\
آهنگ دراز شب رنجوری مشتاق
&&
با آن نتوان گفت که بیدار نباشد
\\
از دیده من پرس که خواب شب مستی
&&
چون خاستن و خفتن بیمار نباشد
\\
گر دست به شمشیر بری عشق همان است
&&
کانجا که ارادت بود انکار نباشد
\\
از من مشنو دوستی گل مگر آن گاه
&&
کم پای برهنه خبر از خار نباشد
\\
مرغان قفس را المی باشد و شوقی
&&
کان مرغ نداند که گرفتار نباشد
\\
دل آینه صورت غیب است ولیکن
&&
شرط است که بر آینه زنگار نباشد
\\
سعدی حیوان را که سر از خواب گران شد
&&
در بند نسیم خوش اسحار نباشد
\\
آن را که بصارت نبود یوسف صدیق
&&
جایی بفروشد که خریدار نباشد
\\
\end{longtable}
\end{center}
