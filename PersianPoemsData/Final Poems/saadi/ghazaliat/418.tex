\begin{center}
\section*{غزل ۴۱۸: گر دست دهد هزار جانم}
\label{sec:418}
\addcontentsline{toc}{section}{\nameref{sec:418}}
\begin{longtable}{l p{0.5cm} r}
گر دست دهد هزار جانم
&&
در پای مبارکت فشانم
\\
آخر به سرم گذر کن ای دوست
&&
انگار که خاک آستانم
\\
هر حکم که بر سرم برانی
&&
سهل است ز خویشتن مرانم
\\
تو خود سر وصل ما نداری
&&
من عادت بخت خویش دانم
\\
هیهات که چون تو شاهبازی
&&
تشریف دهد به آشیانم
\\
گر خانه محقر است و تاریک
&&
بر دیده روشنت نشانم
\\
گر نام تو بر سرم بگویند
&&
فریاد برآید از روانم
\\
شب نیست که در فراق رویت
&&
زاری به فلک نمی‌رسانم
\\
آخر نه من و تو دوست بودیم
&&
عهد تو شکست و من همانم
\\
من مهره مهر تو نریزم
&&
الا که بریزد استخوانم
\\
من ترک وصال تو نگویم
&&
الا به فراق جسم و جانم
\\
مجنونم اگر بهای لیلی
&&
ملک عرب و عجم ستانم
\\
شیرین زمان تویی به تحقیق
&&
من بنده خسرو زمانم
\\
شاهی که ورا رسد که گوید
&&
مولای اکابر جهانم
\\
ایوان رفیعش آسمان را
&&
گوید تو زمین من آسمانم
\\
دانی که ستم روا ندارد
&&
مگذار که بشنود فغانم
\\
هر کس به زمان خویشتن بود
&&
من سعدی آخرالزمانم
\\
\end{longtable}
\end{center}
