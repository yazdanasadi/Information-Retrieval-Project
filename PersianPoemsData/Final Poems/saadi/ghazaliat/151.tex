\begin{center}
\section*{غزل ۱۵۱: گر جان طلبی فدای جانت}
\label{sec:151}
\addcontentsline{toc}{section}{\nameref{sec:151}}
\begin{longtable}{l p{0.5cm} r}
گر جان طلبی فدای جانت
&&
سهلست جواب امتحانت
\\
سوگند به جانت ار فروشم
&&
یک موی به هر که در جهانت
\\
با آن که تو مهر کس نداری
&&
کس نیست که نیست مهربانت
\\
وین سر که تو داری ای ستمکار
&&
بس سر برود بر آستانت
\\
بس فتنه که در زمین به پا شد
&&
از روی چو ماه آسمانت
\\
من در تو رسم به جهد هیهات
&&
کز باد سبق برد عنانت
\\
بی یاد تو نیستم زمانی
&&
تا یاد کنم دگر زمانت
\\
کوته نظران کنند و حیفست
&&
تشبیه به سرو بوستانت
\\
و ابرو که تو داری ای پری زاد
&&
در صید چه حاجت کمانت
\\
گویی بدن ضعیف سعدی
&&
نقشیست گرفته از میانت
\\
گر واسطه سخن نبودی
&&
در وهم نیامدی دهانت
\\
شیرینتر از این سخن نباشد
&&
الا دهن شکرفشانت
\\
\end{longtable}
\end{center}
