\begin{center}
\section*{غزل ۵: شب فراق نخواهم دواج دیبا را}
\label{sec:005}
\addcontentsline{toc}{section}{\nameref{sec:005}}
\begin{longtable}{l p{0.5cm} r}
شب فراق نخواهم دواج دیبا را
&&
که شب دراز بود خوابگاه تنها را
\\
ز دست رفتن دیوانه عاقلان دانند
&&
که احتمال نماندست ناشکیبا را
\\
گرش ببینی و دست از ترنج بشناسی
&&
روا بود که ملامت کنی زلیخا را
\\
چنین جوان که تویی برقعی فروآویز
&&
و گر نه دل برود پیر پای برجا را
\\
تو آن درخت گلی کاعتدال قامت تو
&&
ببرد قیمت سرو بلندبالا را
\\
دگر به هر چه تو گویی مخالفت نکنم
&&
که بی تو عیش میسر نمی‌شود ما را
\\
دو چشم باز نهاده نشسته‌ام همه شب
&&
چو فرقدین و نگه می‌کنم ثریا را
\\
شبی و شمعی و جمعی چه خوش بود تا روز
&&
نظر به روی تو کوری چشم اعدا را
\\
من از تو پیش که نالم که در شریعت عشق
&&
معاف دوست بدارند قتل عمدا را
\\
تو همچنان دل شهری به غمزه‌ای ببری
&&
که بندگان بنی سعد خوان یغما را
\\
در این روش که تویی بر هزار چون سعدی
&&
جفا و جور توانی ولی مکن یارا
\\
\end{longtable}
\end{center}
