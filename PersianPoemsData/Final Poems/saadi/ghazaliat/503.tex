\begin{center}
\section*{غزل ۵۰۳: تو پری زاده ندانم ز کجا می‌آیی}
\label{sec:503}
\addcontentsline{toc}{section}{\nameref{sec:503}}
\begin{longtable}{l p{0.5cm} r}
تو پری زاده ندانم ز کجا می‌آیی
&&
کآدمیزاده نباشد به چنین زیبایی
\\
راست خواهی نه حلال است که پنهان دارند
&&
مثل این روی و نشاید که به کس بنمایی
\\
سرو با قامت زیبای تو در مجلس باغ
&&
نتواند که کند دعوی همبالایی
\\
در سراپای وجودت هنری نیست که نیست
&&
عیبت آن است که بر بنده نمی‌بخشایی
\\
به خدا بر تو که خون من بیچاره مریز
&&
که من آن قدر ندارم که تو دست آلایی
\\
بی رخت چشم ندارم که جهانی بینم
&&
به دو چشمت که ز چشمم مرو ای بینایی
\\
نه مرا حسرت جاه است و نه اندیشه مال
&&
همه اسباب مهیاست تو در می‌بایی
\\
بر من از دست تو چندان که جفا می‌آید
&&
خوشتر و خوبتر اندر نظرم می‌آیی
\\
دیگری نیست که مهر تو در او شاید بست
&&
چاره بعد از تو ندانیم به جز تنهایی
\\
ور به خواری ز در خویش برانی ما را
&&
همچنان شکر کنیمت که عزیز مایی
\\
من از این در به جفا روی نخواهم پیچید
&&
گر ببندی تو به روی من و گر بگشایی
\\
چه کند داعی دولت که قبولش نکنند
&&
ما حریصیم به خدمت تو نمی‌فرمایی
\\
سعدیا دختر انفاس تو بس دل ببرد
&&
به چنین زیور معنی که تو می‌آرایی
\\
باد نوروز که بوی گل و سنبل دارد
&&
لطف این باد ندارد که تو می‌پیمایی
\\
\end{longtable}
\end{center}
