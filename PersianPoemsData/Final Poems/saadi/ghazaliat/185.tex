\begin{center}
\section*{غزل ۱۸۵: کسی به عیب من از خویشتن نپردازد}
\label{sec:185}
\addcontentsline{toc}{section}{\nameref{sec:185}}
\begin{longtable}{l p{0.5cm} r}
کسی به عیب من از خویشتن نپردازد
&&
که هر که می‌نگرم با تو عشق می‌بازد
\\
فرشته‌ای تو بدین روشنی نه آدمیی
&&
نه آدمیست که بر تو نظر نیندازد
\\
نه آدمی که اگر آهنین بود شخصی
&&
در آفتاب جمالت چو موم بگدازد
\\
چنین پسر که تویی راحت روان پدر
&&
سزد که مادر گیتی به روی او نازد
\\
کمان چفته ابرو کشیده تا بن گوش
&&
چو لشکری که به دنبال صید می‌تازد
\\
کدام گل که به روی تو ماند اندر باغ
&&
کدام سرو که با قامتت سر افرازد
\\
درخت میوه مقصود از آن بلندترست
&&
که دست قدرت کوتاه ما بر او یازد
\\
مسلمش نبود عشق یار آتشروی
&&
مگر کسی که چو پروانه سوزد و سازد
\\
مده به دست فراقم پس از وصال چو چنگ
&&
که مطربش بزند بعد از آن که بنوازد
\\
خلاف عهد تو هرگز نیاید از سعدی
&&
دلی که از تو بپرداخت با که پردازد
\\
\end{longtable}
\end{center}
