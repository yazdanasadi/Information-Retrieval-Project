\begin{center}
\section*{غزل ۲۹۲: شیرین دهان آن بت عیار بنگرید}
\label{sec:292}
\addcontentsline{toc}{section}{\nameref{sec:292}}
\begin{longtable}{l p{0.5cm} r}
شیرین دهان آن بت عیار بنگرید
&&
در در میان لعل شکربار بنگرید
\\
بستان عارضش که تماشاگه دلست
&&
پرنرگس و بنفشه و گلنار بنگرید
\\
از ما به یک نظر بستاند هزار دل
&&
این آبروی و رونق بازار بنگرید
\\
سنبل نشانده بر گل سوری نگه کنید
&&
عنبرفشانده گرد سمن زار بنگرید
\\
امروز روی یار بسی خوبتر ز دیست
&&
امسال کار من بتر از پار بنگرید
\\
در عهد شاه عادل اگر فتنه نادرست
&&
این چشم مست و فتنه خون خوار بنگرید
\\
گفتار بشنویدش و دانم که خود ز کبر
&&
با کس سخن نگوید رفتار بنگرید
\\
آن دم که جعد زلف پریشان برافکند
&&
صد دل به زیر طره طرار بنگرید
\\
گنجیست درج در عقیقین آن پسر
&&
بالای گنج حلقه زده مار بنگرید
\\
چشمش به تیغ غمزه خون خوار خیره کش
&&
شهری گرفت قوت بیمار بنگرید
\\
آتشکدست باطن سعدی ز سوز عشق
&&
سوزی که در دلست در اشعار بنگرید
\\
دی گفت سعدیا من از آن توام به طنز
&&
این عشوه دروغ دگربار بنگرید
\\
\end{longtable}
\end{center}
