\begin{center}
\section*{غزل ۵۴۳: ای برق اگر به گوشه آن بام بگذری}
\label{sec:543}
\addcontentsline{toc}{section}{\nameref{sec:543}}
\begin{longtable}{l p{0.5cm} r}
ای برق اگر به گوشه آن بام بگذری
&&
آنجا که باد زهره ندارد خبر بری
\\
ای مرغ اگر پری به سر کوی آن صنم
&&
پیغام دوستان برسانی بدان پری
\\
آن مشتری خصال گر از ما حکایتی
&&
پرسد جواب ده که به جانند مشتری
\\
گو تشنگان بادیه را جان به لب رسید
&&
تو خفته در کجاوه به خواب خوش اندری
\\
ای ماهروی حاضر غایب که پیش دل
&&
یک روز نگذرد که تو صد بار نگذری
\\
دانی چه می‌رود به سر ما ز دست تو
&&
تا خود به پای خویش بیایی و بنگری
\\
بازآی کز صبوری و دوری بسوختیم
&&
ای غایب از نظر که به معنی برابری
\\
یا دل به ما دهی چو دل ما به دست توست
&&
یا مهر خویشتن ز دل ما به در بری
\\
تا خود برون پرده حکایت کجا رسد
&&
چون از درون پرده چنین پرده می‌دری
\\
سعدی تو کیستی که دم دوستی زنی
&&
دعوی بندگی کن و اقرار چاکری
\\
\end{longtable}
\end{center}
