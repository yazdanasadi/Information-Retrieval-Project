\begin{center}
\section*{غزل ۴۸۶: آن سرو ناز بین که چه خوش می‌رود به راه}
\label{sec:486}
\addcontentsline{toc}{section}{\nameref{sec:486}}
\begin{longtable}{l p{0.5cm} r}
آن سرو ناز بین که چه خوش می‌رود به راه
&&
وآن چشم آهوانه که چون می‌کند نگاه
\\
تو سرو دیده‌ای که کمر بست بر میان
&&
یا ماه چارده که به سر برنهد کلاه
\\
گل با وجود او چو گیاه است پیش گل
&&
مه پیش روی او چو ستاره‌ست پیش ماه
\\
سلطان صفت همی‌رود و صد هزار دل
&&
با او چنان که در پی سلطان رود سپاه
\\
گویند از او حذر کن و راه گریز گیر
&&
گویم کجا روم که ندانم گریزگاه
\\
اول نظر که چاه زنخدان بدیدمش
&&
گویی در اوفتاد دل از دست من به چاه
\\
دل خود دریغ نیست که از دست من برفت
&&
جان عزیز بر کف دست است گو بخواه
\\
ای هر دو دیده پای که بر خاک می‌نهی
&&
آخر نه بر دو دیده من به که خاک راه
\\
حیف است از آن دهن که تو داری جواب تلخ
&&
وآن سینه سفید که دارد دل سیاه
\\
بیچارگان بر آتش مهرت بسوختند
&&
آه از تو سنگدل که چه نامهربانی آه
\\
شهری به گفت و گوی تو در تنگنای شوق
&&
شب روز می‌کنند و تو در خواب صبحگاه
\\
گفتم بنالم از تو به یاران و دوستان
&&
باشد که دست ظلم بداری ز بی‌گناه
\\
بازم حفاظ دامن همت گرفت و گفت
&&
از دوست جز به دوست مبر سعدیا پناه
\\
\end{longtable}
\end{center}
