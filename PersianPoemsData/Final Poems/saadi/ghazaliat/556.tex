\begin{center}
\section*{غزل ۵۵۶: گر کنم در سر وفات سری}
\label{sec:556}
\addcontentsline{toc}{section}{\nameref{sec:556}}
\begin{longtable}{l p{0.5cm} r}
گر کنم در سر وفات سری
&&
سهل باشد زیان مختصری
\\
ای که قصد هلاک من داری
&&
صبر کن تا ببینمت نظری
\\
نه حرام است در رخ تو نظر
&&
که حرام است چشم بر دگری
\\
دوست دارم که خاک پات شوم
&&
تا مگر بر سرم کنی گذری
\\
متحیر نه در جمال توام
&&
عقل دارم به قدر خود قدری
\\
حیرتم در صفات بی چون است
&&
کاین کمال آفرید در بشری
\\
ببری هوش و طاقت زن و مرد
&&
گر تردد کنی به بام و دری
\\
حق به دست رقیب ناهموار
&&
پیش خصم ایستاده چون سپری
\\
زان که آیینه‌ای بدین خوبی
&&
حیف باشد به دست بی بصری
\\
آه سعدی اثر کند در کوه
&&
نکند در تو سنگدل اثری
\\
سنگ را سخت گفتمی همه عمر
&&
تا بدیدم ز سنگ سختتری
\\
\end{longtable}
\end{center}
