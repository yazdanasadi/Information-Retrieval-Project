\begin{center}
\section*{بخش ۱ - سر آغاز: شبی زیت فکرت همی سوختم}
\label{sec:001}
\addcontentsline{toc}{section}{\nameref{sec:001}}
\begin{longtable}{l p{0.5cm} r}
شبی زیت فکرت همی سوختم
&&
چراغ بلاغت می افروختم
\\
پراکنده گویی حدیثم شنید
&&
جز احسنت گفتن طریقی ندید
\\
هم از خبث نوعی در آن درج کرد
&&
که ناچار فریاد خیزد ز درد
\\
که فکرش بلیغ است و رایش بلند
&&
در این شیوهٔ زهد و طامات و پند
\\
نه در خشت و کوپال و گرز گران
&&
که این شیوه ختم است بر دیگران
\\
نداند که ما را سر جنگ نیست
&&
وگر نه مجال سخن تنگ نیست
\\
توانم که تیغ زبان بر کشم
&&
جهانی سخن را قلم در کشم
\\
بیا تا در این شیوه چالش کنیم
&&
سر خصم را سنگ، بالش کنیم
\\
سعادت به بخشایش داورست
&&
نه در چنگ و بازوی زور آورست
\\
چو دولت نبخشد سپهر بلند
&&
نیاید به مردانگی در کمند
\\
نه سختی رسید از ضعیفی به مور
&&
نه شیران به سرپنجه خوردند و زور
\\
چو نتوان بر افلاک دست آختن
&&
ضروری است با گردشش ساختن
\\
گرت زندگانی نبشته‌ست دیر
&&
نه مارت گزاید نه شمشیر و شیر
\\
وگر در حیاتت نمانده‌ست بهر
&&
چنانت کشد نوشدارو که زهر
\\
نه رستم چو پایان روزی بخورد
&&
شغاد از نهادش برآورد گرد؟
\\
\end{longtable}
\end{center}
