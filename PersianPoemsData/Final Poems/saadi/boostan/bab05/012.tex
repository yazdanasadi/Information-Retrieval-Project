\begin{center}
\section*{بخش ۱۲ - مثل: شتر بچه با مادر خویش گفت}
\label{sec:012}
\addcontentsline{toc}{section}{\nameref{sec:012}}
\begin{longtable}{l p{0.5cm} r}
شتر بچه با مادر خویش گفت:
&&
بس از رفتن، آخر زمانی بخفت
\\
بگفت ار به دست منستی مهار
&&
ندیدی کسم بارکش در قطار
\\
قضا کشتی آنجا که خواهد برد
&&
وگر ناخدا جامه بر تن درد
\\
مکن سعدیا دیده بر دست کس
&&
که بخشنده پروردگار است و بس
\\
اگر حق پرستی ز درها بست
&&
که گر وی براند نخواند کست
\\
گر او تاجدارت کند سر بر آر
&&
وگر نه سر نا امیدی بخار
\\
\end{longtable}
\end{center}
