\begin{center}
\section*{بخش ۱ - سرآغاز: بیا تا برآریم دستی ز دل}
\label{sec:001}
\addcontentsline{toc}{section}{\nameref{sec:001}}
\begin{longtable}{l p{0.5cm} r}
بیا تا برآریم دستی ز دل
&&
که نتوان برآورد فردا ز گل
\\
به فصل خزان در نبینی درخت
&&
که بی برگ ماند ز سرمای سخت
\\
برآرد تهی دستهای نیاز
&&
ز رحمت نگردد تهیدست باز
\\
مپندار از آن در که هرگز نبست
&&
که نومید گردد بر آورده دست
\\
قضا خلعتی نامدارش دهد
&&
قدر میوه در آستینش نهد
\\
همه طاعت آرند و مسکین نیاز
&&
بیا تا به درگاه مسکین نواز
\\
چو شاخ برهنه برآریم دست
&&
که بی برگ از این بیش نتوان نشست
\\
خداوندگارا نظر کن به جود
&&
که جرم آمد از بندگان در وجود
\\
گناه آید از بندهٔ خاکسار
&&
به امید عفو خداوندگار
\\
کریما به رزق تو پرورده‌ایم
&&
به انعام و لطف تو خو کرده‌ایم
\\
گدا چون کرم بیند و لطف و ناز
&&
نگردد ز دنبال بخشنده باز
\\
چو ما را به دنیا تو کردی عزیز
&&
به عقبی همین چشم داریم نیز
\\
عزیزی و خواری تو بخشی و بس
&&
عزیز تو خواری نبیند ز کس
\\
خدایا به عزت که خوارم مکن
&&
به ذل گنه شرمسارم مکن
\\
مسلط مکن چون منی بر سرم
&&
ز دست تو به گر عقوبت برم
\\
به گیتی بتر زین نباشد بدی
&&
جفا بردن از دست همچون خودی
\\
مرا شرمساری ز روی تو بس
&&
دگر شرمسارم مکن پیش کس
\\
گرم بر سر افتد ز تو سایه‌ای
&&
سپهرم بود کهترین پایه‌ای
\\
اگر تاج بخشی سر افرازدم
&&
تو بردار تا کس نیندازدم
\\
تنم می‌بلرزد چو یاد آورم
&&
مناجات شوریده‌ای در حرم
\\
که می‌گفت شوریدهٔ دلفگار
&&
الها ببخش و به ذلّم مدار
\\
همی‌گفت با حق به زاری بسی
&&
میفکن که دستم نگیرد کسی
\\
به لطفم بخوان و مران از درم
&&
ندارد به جز آستانت سرم
\\
تو دانی که مسکین و بیچاره‌ایم
&&
فرو مانده نفس اماره‌ایم
\\
نمی‌تازد این نفس سرکش چنان
&&
که عقلش تواند گرفتن عنان
\\
که با نفس و شیطان بر آید به زور؟
&&
مصاف پلنگان نیاید ز مور
\\
به مردان راهت که راهی بده
&&
وز این دشمنانم پناهی بده
\\
خدایا به ذات خداوندیت
&&
به اوصاف بی مثل و مانندیت
\\
به لبیک حجاج بیت‌الحرام
&&
به مدفون یثرب علیه‌السلام
\\
به تکبیر مردان شمشیر زن
&&
که مرد وغا را شمارند زن
\\
به طاعات پیران آراسته
&&
به صدق جوانان نوخاسته
\\
که ما را در آن ورطهٔ یک نفس
&&
ز ننگ دو گفتن به فریاد رس
\\
امید است از آنان که طاعت کنند
&&
که بی طاعتان را شفاعت کنند
\\
به پاکان کز آلایشم دور دار
&&
وگر زلتی رفت معذور دار
\\
به پیران پشت از عبادت دو تا
&&
ز شرم گنه دیده بر پشت پا
\\
که چشمم ز روی سعادت مبند
&&
زبانم به وقت شهادت مبند
\\
چراغ یقینم فرا راه دار
&&
ز بد کردنم دست کوتاه دار
\\
بگردان ز نادیدنی دیده‌ام
&&
مده دست بر ناپسندیده‌ام
\\
من آن ذره‌ام در هوای تو نیست
&&
وجود و عدم ز احتقارم یکی است
\\
ز خورشید لطفت شعاعی بسم
&&
که جز در شعاعت نبیند کسم
\\
بدی را نگه کن که بهتر کس است
&&
گدا را ز شاه التفاتی بس است
\\
مرا گر بگیری به انصاف و داد
&&
بنالم که عفوم نه این وعده داد
\\
خدایا به ذلت مران از درم
&&
که صورت نبندد دری دیگرم
\\
ور از جهل غایب شدم روز چند
&&
کنون کآمدم در به رویم مبند
\\
چه عذر آرم از ننگ تردامنی؟
&&
مگر عجز پیش آورم کای غنی
\\
فقیرم به جرم و گناهم مگیر
&&
غنی را ترحم بود بر فقیر
\\
چرا باید از ضعف حالم گریست؟
&&
اگر من ضعیفم پناهم قوی است
\\
خدایا به غفلت شکستیم عهد
&&
چه زور آورد با قضا دست جهد؟
\\
چه برخیزد از دست تدبیر ما؟
&&
همین نکته بس عذر تقصیر ما
\\
همه هر چه کردم تو بر هم زدی
&&
چه قوت کند با خدایی خودی؟
\\
نه من سر ز حکمت به در می‌برم
&&
که حکمت چنین می‌رود بر سرم
\\
\end{longtable}
\end{center}
