\begin{center}
\section*{بخش ۴ - حکایت: شنیدم که مستی ز تاب نبید}
\label{sec:004}
\addcontentsline{toc}{section}{\nameref{sec:004}}
\begin{longtable}{l p{0.5cm} r}
شنیدم که مستی ز تاب نبید
&&
به مقصورهٔ مسجدی در دوید
\\
بنالید بر آستان کرم
&&
که یارب به فردوس اعلی برم
\\
مؤذن گریبان گرفتش که هین
&&
سگ و مسجد! ای فارغ از عقل و دین
\\
چه شایسته کردی که خواهی بهشت؟
&&
نمی‌زیبدت ناز با روی زشت
\\
بگفت این سخن پیر و بگریست مست
&&
که مستم بدار از من ای خواجه دست
\\
عجب داری از لطف پروردگار
&&
که باشد گنهکاری امیدوار؟
\\
تو را می‌نگویم که عذرم پذیر
&&
در توبه باز است و حق دستگیر
\\
همی شرم دارم ز لطف کریم
&&
که خوانم گنه پیش عفوش عظیم
\\
کسی را که پیری در آرد ز پای
&&
چو دستش نگیری نخیزد ز جای
\\
من آنم ز پای اندر افتاده پیر
&&
خدایا به فضل خودم دست گیر
\\
نگویم بزرگی و جاهم ببخش
&&
فروماندگی و گناهم ببخش
\\
اگر یاری اندک زلل داندم
&&
به نابخردی شهره گرداندم
\\
تو بینا و ما خائف از یکدگر
&&
که تو پرده پوشی و ما پرده در
\\
برآورده مردم ز بیرون خروش
&&
تو با بنده در پرده و پرده پوش
\\
به نادانی ار بندگان سرکشند
&&
خداوندگاران قلم در کشند
\\
اگر جرم بخشی به مقدار جود
&&
نماند گنهکاری اندر وجود
\\
وگر خشم گیری به قدر گناه
&&
به دوزخ فرست و ترازو مخواه
\\
گرم دست گیری به جایی رسم
&&
وگر بفکنی بر نگیرد کسم
\\
که زور آورد گر تو یاری دهی؟
&&
که گیرد چو تو رستگاری دهی؟
\\
دو خواهند بودن به محشر فریق
&&
ندانم کدامین دهندم طریق
\\
عجب گر بود راهم از دست راست
&&
که از دست من جز کجی برنخاست
\\
دلم می‌دهد وقت وقت این امید
&&
که حق شرم دارد ز موی سپید
\\
عجب دارم ار شرم دارد ز من
&&
که شرمم نمی‌آید از خویشتن
\\
نه یوسف که چندان بلا دید و بند
&&
چو حکمش روان گشت و قدرش بلند
\\
گنه عفو کرد آل یعقوب را؟
&&
که معنی بود صورت خوب را
\\
به کردار بدشان مقید نکرد
&&
بضاعات مزجاتشان رد نکرد
\\
ز لطفت همین چشم داریم نیز
&&
بر این بی‌بضاعت ببخش ای عزیز
\\
کس از من سیه نامه تر دیده نیست
&&
که هیچم فعال پسندیده نیست
\\
جز این کاعتمادم به یاری تست
&&
امیدم به آمرزگاری تست
\\
بضاعت نیاوردم الا امید
&&
خدایا ز عفوم مکن ناامید
\\
\end{longtable}
\end{center}
