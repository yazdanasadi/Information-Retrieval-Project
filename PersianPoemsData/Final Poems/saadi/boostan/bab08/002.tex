\begin{center}
\section*{بخش ۲ - حکایت: جوانی سر از رأی مادر بتافت}
\label{sec:002}
\addcontentsline{toc}{section}{\nameref{sec:002}}
\begin{longtable}{l p{0.5cm} r}
جوانی سر از رای مادر بتافت
&&
دل دردمندش به آذر بتافت
\\
چو بیچاره شد پیشش آورد مهد
&&
که ای سست مهر فراموش عهد
\\
نه گریان و درمانده بودی و خرد
&&
که شبها ز دست تو خوابم نبرد؟
\\
نه در مهد نیروی حالت نبود
&&
مگس راندن از خود مجالت نبود؟
\\
تو آنی کز آن یک مگس رنجه‌ای
&&
که امروز سالار و سرپنجه‌ای
\\
به حالی شوی باز در قعر گور
&&
که نتوانی از خویشتن دفع مور
\\
دگر دیده چون برفروزد چراغ
&&
چو کرم لحد خورد پیه دماغ؟
\\
چو پوشیده چشمی ببینی که راه
&&
نداند همی وقت رفتن ز چاه
\\
تو گر شکر کردی که با دیده‌ای
&&
وگر نه تو هم چشم پوشیده‌ای
\\
معلم نیاموختت فهم و رای
&&
سرشت این صفت در نهادت خدای
\\
گرت منع کردی دل حق نیوش
&&
حقت عین باطل نبودی به گوش
\\
\end{longtable}
\end{center}
