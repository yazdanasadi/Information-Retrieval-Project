\begin{center}
\section*{بخش ۱۴ - گفتار در صبر بر ناتوانی به امید بهی: کمال است در نفس مرد کریم}
\label{sec:014}
\addcontentsline{toc}{section}{\nameref{sec:014}}
\begin{longtable}{l p{0.5cm} r}
کمال است در نفس مرد کریم
&&
گرش زر نباشد چه نقصان و بیم؟
\\
مپندار اگر سفله قارون شود
&&
که طبع لئیمش دگرگون شود
\\
وگر درنیابد کرم پیشه، نان
&&
نهادش توانگر بود همچنان
\\
مروت زمین است و سرمایه زرع
&&
بده کاصل خالی نماند ز فرع
\\
خدایی که از خاک مردم کند
&&
عجب باشد ار مردمی گم کند
\\
ز نعمت نهادن بلندی مجوی
&&
که ناخوش کند آب استاده بوی
\\
به بخشندگی کوش کآب روان
&&
به سیلش مدد می‌رسد ز آسمان
\\
گر از جاه و دولت بیفتد لئیم
&&
دگر باره نادر شود مستقیم
\\
وگر قیمتی گوهری غم مدار
&&
که ضایع نگرداندت روزگار
\\
کلوخ ار چه افتاده بینی به راه
&&
نبینی که در وی کند کس نگاه
\\
و گر خردهٔ زر ز دندان گاز
&&
بیفتد، به شمعش بجویند باز
\\
به در می‌کنند آبگینه ز سنگ
&&
کجا ماند آیینه در زیر زنگ؟
\\
هنر باید و فضل و دین و کمال
&&
که گاه آید و گه رود جاه و مال
\\
\end{longtable}
\end{center}
