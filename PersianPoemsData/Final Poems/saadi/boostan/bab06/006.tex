\begin{center}
\section*{بخش ۶ - حکایت: شکم صوفیی را زبون کرد و فرج}
\label{sec:006}
\addcontentsline{toc}{section}{\nameref{sec:006}}
\begin{longtable}{l p{0.5cm} r}
شکم صوفیی را زبون کرد و فرج
&&
دو دینار بر هر دوان کرد خرج
\\
یکی گفتش از دوستان در نهفت
&&
چه کردی بدین هر دو دینار؟ گفت
\\
به دیناری از پشت راندم نشاط
&&
به دیگر، شکم را کشیدم سماط
\\
فرومایگی کردم و ابلهی
&&
که این همچنان پر نشد وآن تهی
\\
غذا گر لطیف است و گر سرسری
&&
چو دیرت به دست اوفتد خوش خوری
\\
سر آنگه به بالین نهد هوشمند
&&
که خوابش به قهر آورد در کمند
\\
مجال سخن تا نیابی مگوی
&&
چو میدان نبینی نگه دار گوی
\\
وز اندازه بیرون، مرو پیش زن
&&
نه دیوانه‌ای تیغ بر خود مزن
\\
به بی رغبتی شهوت انگیختن
&&
به رغبت بود خون خود ریختن
\\
\end{longtable}
\end{center}
