\begin{center}
\section*{بخش ۴ - حکایت: یکی را تب آمد ز صاحبدلان}
\label{sec:004}
\addcontentsline{toc}{section}{\nameref{sec:004}}
\begin{longtable}{l p{0.5cm} r}
یکی را تب آمد ز صاحبدلان
&&
کسی گفت شکر بخواه از فلان
\\
بگفت ای پسر تلخی مردنم
&&
به از جور روی ترش بردنم
\\
شکر عاقل از دست آن کس نخورد
&&
که روی از تکبر بر او سرکه کرد
\\
مرو از پی هر چه دل خواهدت
&&
که تمکین تن نور جان کاهدت
\\
کند مرد را نفس اماره خوار
&&
اگر هوشمندی عزیزش مدار
\\
اگر هرچه باشد مرادت خوری
&&
ز دوران بسی نامرادی بری
\\
تنور شکم دم به دم تافتن
&&
مصیبت بود روز نایافتن
\\
به تنگی بریزاندت روی رنگ
&&
چو وقت فراخی کنی معده تنگ
\\
کشد مرد پرخواره بار شکم
&&
وگر در نیابد کشد بار غم
\\
شکم بنده بسیار بینی خجل
&&
شکم پیش من تنگ بهتر که دل
\\
\end{longtable}
\end{center}
