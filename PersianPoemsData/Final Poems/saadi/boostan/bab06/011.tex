\begin{center}
\section*{بخش ۱۱ - حکایت مرد کوته نظر و زن عالی همت: یکی طفل دندان برآورده بود}
\label{sec:011}
\addcontentsline{toc}{section}{\nameref{sec:011}}
\begin{longtable}{l p{0.5cm} r}
یکی طفل دندان برآورده بود
&&
پدر سر به فکرت فرو برده بود
\\
که من نان و برگ از کجا آرمش؟
&&
مروت نباشد که بگذارمش
\\
چو بیچاره گفت این سخن، نزد جفت
&&
نگر تا زن او را چه مردانه گفت:
\\
مخور هول ابلیس تا جان دهد
&&
همان کس که دندان دهد نان دهد
\\
تواناست آخر خداوند روز
&&
که روزی رساند، تو چندین مسوز
\\
نگارندهٔ کودک اندر شکم
&&
نویسنده عمر و روزی است هم
\\
خداوندگاری که عبدی خرید
&&
بدارد، فکیف آن که عبد آفرید
\\
تو را نیست این تکیه بر کردگار
&&
که مملوک را بر خداوندگار
\\
شنیدی که در روزگار قدیم
&&
شدی سنگ در دست ابدال سیم
\\
نپنداری این قول معقول نیست
&&
چو قانع شدی سیم و سنگت یکی است
\\
چو طفل اندرون دارد از حرص پاک
&&
چه مشتی زرش پیش همت چه خاک
\\
خبر ده به درویش سلطان پرست
&&
که سلطان ز درویش مسکین ترست
\\
گدا را کند یک درم سیم سیر
&&
فریدون به ملک عجم نیم سیر
\\
نگهبانی ملک و دولت بلاست
&&
گدا پادشاه است و نامش گداست
\\
گدایی که بر خاطرش بند نیست
&&
به از پادشاهی که خرسند نیست
\\
بخسبند خوش روستایی و جفت
&&
به ذوقی که سلطان در ایوان نخفت
\\
اگر پادشاه است و گر پینه دوز
&&
چو خفتند گردد شب هر دو روز
\\
چو سیلاب خواب آمد و مرد برد
&&
چه بر تخت سلطان، چه بر دشت کرد
\\
چو بینی توانگر سر از کبر مست
&&
برو شکر یزدان کن ای تنگدست
\\
نداری بحمدالله آن دسترس
&&
که برخیزد از دستت آزار کس
\\
\end{longtable}
\end{center}
