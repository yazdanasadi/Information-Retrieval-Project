\begin{center}
\section*{بخش ۲ - حکایت: مرا حاجیی شانهٔ عاج داد}
\label{sec:002}
\addcontentsline{toc}{section}{\nameref{sec:002}}
\begin{longtable}{l p{0.5cm} r}
مرا حاجیی شانهٔ عاج داد
&&
که رحمت بر اخلاق حجاج باد
\\
شنیدم که باری سگم خوانده بود
&&
که از من به نوعی دلش مانده بود
\\
بینداختم شانه کاین استخوان
&&
نمی‌بایدم دیگرم سگ مخوان
\\
مپندار چون سرکهٔ خود خورم
&&
که جور خداوند حلوا برم
\\
قناعت کن ای نفس بر اندکی
&&
که سلطان و درویش بینی یکی
\\
چرا پیش خسرو به خواهش روی
&&
چو یک سو نهادی طمع، خسروی
\\
وگر خود پرستی شکم طبله کن
&&
در خانهٔ این و آن قبله کن
\\
\end{longtable}
\end{center}
