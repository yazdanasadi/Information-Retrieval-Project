\begin{center}
\section*{بخش ۱۳ - حکایت: یکی سلطنت ران صاحب شکوه}
\label{sec:013}
\addcontentsline{toc}{section}{\nameref{sec:013}}
\begin{longtable}{l p{0.5cm} r}
یکی سلطنت ران صاحب شکوه
&&
فرو خواست رفت آفتابش به کوه
\\
به شیخی در آن بقعه کشور گذاشت
&&
که در دوره قائم مقامی نداشت
\\
چو خلوت نشین کوس دولت شنید
&&
دگر ذوق در کنج خلوت ندید
\\
چپ و راست لشکر کشیدن گرفت
&&
دل پردلان زو رمیدن گرفت
\\
چنان سخت بازو شد و تیز چنگ
&&
که با جنگجویان طلب کرد جنگ
\\
ز قوم پراکنده خلقی بکشت
&&
دگر جمع گشتند و هم رای و پشت
\\
چنان در حصارش کشیدند تنگ
&&
که عاجز شد از تیرباران و سنگ
\\
بر نیکمردی فرستاد کس
&&
که صعبم فرومانده، فریاد رس
\\
به همت مدد کن که شمشیر و تیر
&&
نه در هر وغایی بود دستگیر
\\
چو بشنید عابد بخندید و گفت
&&
چرا نیم نانی نخورد و نخفت؟
\\
ندانست قارون نعمت پرست
&&
که گنج سلامت به کنج اندر است
\\
\end{longtable}
\end{center}
