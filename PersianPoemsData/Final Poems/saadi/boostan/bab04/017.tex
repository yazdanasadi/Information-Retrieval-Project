\begin{center}
\section*{بخش ۱۷ - حکایت در معنی تواضع و نیازمندی: ز ویرانهٔ عارفی ژنده پوش}
\label{sec:017}
\addcontentsline{toc}{section}{\nameref{sec:017}}
\begin{longtable}{l p{0.5cm} r}
ز ویرانهٔ عارفی ژنده پوش
&&
یکی را نباح سگ آمد به گوش
\\
به دل گفت کوی سگ اینجا چراست؟
&&
درآمد که درویش صالح کجاست؟
\\
نشان سگ از پیش و از پس ندید
&&
به جز عارف آنجا دگر کس ندید
\\
خجل باز گردیدن آغاز کرد
&&
که شرم آمدش بحث این راز کرد
\\
شنید از درون عارف آواز پای
&&
هلا گفت بر در چه پایی؟ در آی
\\
مپندار ای دیدهٔ روشنم
&&
کز ایدر سگ آواز کرد، این منم
\\
چو دیدم که بیچارگی می‌خرد
&&
نهادم ز سر کبر و رای و خرد
\\
چو سگ بر درش بانگ کردم بسی
&&
که مسکین تر از سگ ندیدم کسی
\\
چو خواهی که در قدر والا رسی
&&
ز شیب تواضع به بالا رسی
\\
در این حضرت آنان گرفتند صدر
&&
که خود را فروتر نهادند قدر
\\
چو سیل اندر آمد به هول و نهیب
&&
فتاد از بلندی به سر در نشیب
\\
چو شبنم بیفتاد مسکین و خرد
&&
به مهر آسمانش به عیوق برد
\\
\end{longtable}
\end{center}
