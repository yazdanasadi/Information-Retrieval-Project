\begin{center}
\section*{بخش ۱۲ - حکایت معروف کرخی و مسافر رنجور: کسی راه معروف کرخی بجست}
\label{sec:012}
\addcontentsline{toc}{section}{\nameref{sec:012}}
\begin{longtable}{l p{0.5cm} r}
کسی راه معروف کرخی بجست
&&
که بنهاد معروفی از سر نخست
\\
شنیدم که مهمانش آمد یکی
&&
ز بیماریش تا به مرگ اندکی
\\
سرش موی و رویش صفا ریخته
&&
به موییش جان در تن آویخته
\\
شب آنجا بیفکند و بالش نهاد
&&
روان دست در بانگ و نالش نهاد
\\
نه خوابش گرفتی شبان یک نفس
&&
نه از دست فریاد او خواب کس
\\
نهادی پریشان و طبعی درشت
&&
نمی‌مرد و خلقی به حجت بکشت
\\
ز فریاد و نالیدن و خفت و خیز
&&
گرفتند از او خلق راه گریز
\\
ز دیار مردم در آن بقعه کس
&&
همان ناتوان ماند و معروف و بس
\\
شنیدم که شبها ز خدمت نخفت
&&
چو مردان میان بست و کرد آنچه گفت
\\
شبی بر سرش لشکر آورد خواب
&&
که چند آورد مرد ناخفته تاب؟
\\
به یک دم که چشمانش خفتن گرفت
&&
مسافر پراکنده گفتن گرفت
\\
که لعنت بر این نسل ناپاک باد
&&
که نامند و ناموس و زرقند و باد
\\
پلید اعتقادان پاکیزه پوش
&&
فریبندهٔ پارسایی فروش
\\
چه داند لت انبانی از خواب مست
&&
که بیچاره‌ای دیده بر هم نبست؟
\\
سخنهای منکر به معروف گفت
&&
که یک دم چرا غافل از وی بخفت
\\
فرو خورد شیخ این حدیث از کرم
&&
شنیدند پوشیدگان حرم
\\
یکی گفت معروف را در نهفت
&&
شنیدی که درویش نالان چه گفت؟
\\
برو زاین سپس گو سر خویش گیر
&&
گرانی مکن جای دیگر بمیر
\\
نکویی و رحمت به جای خود است
&&
ولی با بدان نیکمردی بد است
\\
سر سفله را گرد بالش منه
&&
سر مردم آزار بر سنگ به
\\
مکن با بدان نیکی ای نیکبخت
&&
که در شوره نادان نشاند درخت
\\
نگویم مراعات مردم مکن
&&
کرم پیش نامردمان گم مکن
\\
به اخلاق نرمی مکن با درشت
&&
که سگ را نمالند چون گربه پشت
\\
گر انصاف خواهی سگ حق شناس
&&
به سیرت به از مردم ناسپاس
\\
به برفاب رحمت مکن بر خسیس
&&
چو کردی مکافات بر یخ نویس
\\
ندیدم چنین پیچ بر پیچ کس
&&
مکن هیچ رحمت بر این هیچ کس
\\
بخندید و گفت ای دلارام جفت
&&
پریشان مشو زاین پریشان که گفت
\\
گر از ناخوشی کرد بر من خروش
&&
مرا ناخوش از وی خوش آمد به گوش
\\
جفای چنین کس نباید شنود
&&
که نتواند از بی‌قراری غنود
\\
چو خود را قوی حال بینی و خوش
&&
به شکرانه بار ضعیفان بکش
\\
اگر خود همین صورتی چون طلسم
&&
بمیری و اسمت بمیرد چو جسم
\\
وگر پرورانی درخت کرم
&&
بر نیکنامی خوری لاجرم
\\
نبینی که در کرخ تربت بسی است
&&
به جز گور معروف، معروف نیست
\\
به دولت کسانی سر افراختند
&&
که تاج تکبر بینداختند
\\
تکبر کند مرد حشمت پرست
&&
نداند که حشمت به حلم اندر است
\\
\end{longtable}
\end{center}
