\begin{center}
\section*{بخش ۲۵ - حکایت امیرالمومنین علی (ع) و سیرت پاک او: کسی مشکلی برد پیش علی}
\label{sec:025}
\addcontentsline{toc}{section}{\nameref{sec:025}}
\begin{longtable}{l p{0.5cm} r}
کسی مشکلی برد پیش علی
&&
مگر مشکلش را کند منجلی
\\
امیر عدوبند کشور گشای
&&
جوابش بگفت از سر علم و رای
\\
شنیدم که شخصی در آن انجمن
&&
بگفتا چنین نیست یا باالحسن
\\
نرنجید از او حیدر نامجوی
&&
بگفت ار تو دانی از این به بگوی
\\
بگفت آنچه دانست و بایسته گفت
&&
به گل چشمهٔ خور نشاید نهفت
\\
پسندید از او شاه مردان جواب
&&
که من بر خطا بودم او بر صواب
\\
به از ما سخنگوی دانا یکی است
&&
که بالاتر از علم او علم نیست
\\
گر امروز بودی خداوند جاه
&&
نکردی خود از کبر در وی نگاه
\\
به در کردی از بارگه حاجبش
&&
فرو کوفتندی به ناواجبش
\\
که من بعد بی آبرویی مکن
&&
ادب نیست پیش بزرگان سخن
\\
یکی را که پندار در سر بود
&&
مپندار هرگز که حق بشنود
\\
ز علمش ملال آید از وعظ ننگ
&&
شقایق به باران نروید ز سنگ
\\
گرت در دریای فضل است خیز
&&
به تذکیر در پای درویش ریز
\\
نبینی که از خاک افتاده خوار
&&
بروید گل و بشکفد نوبهار
\\
مریز ای حکیم آستینهای در
&&
چو می‌بینی از خویشتن خواجه پر
\\
به چشم کسان در نیاید کسی
&&
که از خود بزرگی نماید بسی
\\
مگو تا بگویند شکرت هزار
&&
چو خود گفتی از کس توقع مدار
\\
\end{longtable}
\end{center}
