\begin{center}
\section*{بخش ۲۸ - حکایت ذوالنون مصری: چنین یاد دارم که سقای نیل}
\label{sec:028}
\addcontentsline{toc}{section}{\nameref{sec:028}}
\begin{longtable}{l p{0.5cm} r}
چنین یاد دارم که سقای نیل
&&
نکرد آب بر مصر سالی سبیل
\\
گروهی سوی کوهساران شدند
&&
به فریاد خواهان باران شدند
\\
گرستند و از گریه جویی روان
&&
نیامد مگر گریهٔ آسمان
\\
به ذوالنون خبر برد از ایشان کسی
&&
که بر خلق رنج است و زحمت بسی
\\
فرو ماندگان را دعایی بکن
&&
که مقبول را رد نباشد سخن
\\
شنیدم که ذوالنون به مدین گریخت
&&
بسی بر نیامد که باران بریخت
\\
خبر شد به مدین پس از روز بیست
&&
که ابر سیه دل بر ایشان گریست
\\
سبک عزم باز آمدن کرد پیر
&&
که پر شد به سیل بهاران غدیر
\\
بپرسید از او عارفی در نهفت
&&
چه حکمت در این رفتنت بود؟ گفت
\\
شنیدم که بر مرغ و مور و ددان
&&
شود تنگ روزی به فعل بدان
\\
در این کشور اندیشه کردم بسی
&&
پریشان‌تر از خود ندیدم کسی
\\
برفتم مبادا که از شر من
&&
ببندد در خیر بر انجمن
\\
بهی بایدت لطف کن کان بهان
&&
ندیدندی از خود بتر در جهان
\\
تو آنگه شوی پیش مردم عزیز
&&
که مر خویشتن را نگیری به چیز
\\
بزرگی که خود را به خردی شمرد
&&
به دنیا و عقبی بزرگی ببرد
\\
از این خاکدان بنده‌ای پاک شد
&&
که در پای کمتر کسی خاک شد
\\
الا ای که بر خاک ما بگذری
&&
به خاک عزیزان که یاد آوری
\\
که گر خاک شد سعدی، او را چه غم؟
&&
که در زندگی خاک بوده‌ست هم
\\
به بیچارگی تن فرا خاک داد
&&
وگر گرد عالم برآمد چو باد
\\
بسی برنیاید که خاکش خورد
&&
دگر باره بادش به عالم برد
\\
مگر تا گلستان معنی شکفت
&&
بر او هیچ بلبل چنین خوش نگفت
\\
عجب گر بمیرد چنین بلبلی
&&
که بر استخوانش نروید گلی
\\
\end{longtable}
\end{center}
