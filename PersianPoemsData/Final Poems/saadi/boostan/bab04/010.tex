\begin{center}
\section*{بخش ۱۰ - حکایت در معنی عزت نفس مردان: سگی پای صحرانشینی گزید}
\label{sec:010}
\addcontentsline{toc}{section}{\nameref{sec:010}}
\begin{longtable}{l p{0.5cm} r}
سگی پای صحرانشینی گزید
&&
به خشمی که زهرش ز دندان چکید
\\
شب از درد بیچاره خوابش نبرد
&&
به خیل اندرش دختری بود خرد
\\
پدر را جفا کرد و تندی نمود
&&
که آخر تو را نیز دندان نبود؟
\\
پس از گریه مرد پراکنده روز
&&
بخندید کای بابک دلفروز
\\
مرا گر چه هم سلطنت بود و بیش
&&
دریغ آمدم کام و دندان خویش
\\
محال است اگر تیغ بر سر خورم
&&
که دندان به پای سگ اندر برم
\\
توان کرد با ناکسان بد رگی
&&
ولیکن نیاید ز مردم سگی
\\
\end{longtable}
\end{center}
