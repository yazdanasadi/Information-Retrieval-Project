\begin{center}
\section*{بخش ۱۳ - حکایت در معنی سفاهت نااهلان: طمع برد شوخی به صاحبدلی}
\label{sec:013}
\addcontentsline{toc}{section}{\nameref{sec:013}}
\begin{longtable}{l p{0.5cm} r}
طمع برد شوخی به صاحبدلی
&&
نبود آن زمان در میان حاصلی
\\
کمربند و دستش تهی بود و پاک
&&
که زر برفشاندی به رویش چو خاک
\\
برون تاخت خواهندهٔ خیره روی
&&
نکوهیدن آغاز کردش به کوی
\\
که زنهار از این کژدمان خموش
&&
پلنگان درندهٔ صوف پوش
\\
که چون گربه زانو به دل برنهند
&&
و گر صیدی افتد چو سگ در جهند
\\
سوی مسجد آورده دکان شید
&&
که در خانه کمتر توان یافت صید
\\
ره کاروان شیرمردان زنند
&&
ولی جامه مردم اینان کنند
\\
سپید و سیه پاره بر دوخته
&&
به سالوس و پنهان زر اندوخته
\\
زهی جو فروشان گندم نمای
&&
جهانگرد شبکوک خرمن گدای
\\
مبین در عبادت که پیرند و سست
&&
که در رقص و حالت جوانند و چست
\\
چرا کرد باید نماز از نشست
&&
چو در رقص بر می‌توانند جست؟
\\
عصای کلیمند بسیار خوار
&&
به ظاهر چنین زرد روی و نزار
\\
نه پرهیزگار و نه دانشورند
&&
همین بس که دنیا به دین می‌خورند
\\
عبایی بلیلانه در تن کنند
&&
به دخل حبش جامهٔ زن کنند
\\
ز سنت نبینی در ایشان اثر
&&
مگر خواب پیشین و نان سحر
\\
شکم تا سر آکنده از لقمه تنگ
&&
چو زنبیل دریوزه هفتاد رنگ
\\
نخواهم در این وصف از این بیش گفت
&&
که شنعت بود سیرت خویش گفت
\\
فرو گفت از این شیوه نادیده گوی
&&
نبیند هنر دیدهٔ عیبجوی
\\
یکی کرده بی آبرویی بسی
&&
چه غم داردش ز آبروی کسی؟
\\
مریدی به شیخ این سخن نقل کرد
&&
گر انصاف پرسی، نه از عقل کرد
\\
بدی در قفا عیب من کرد و خفت
&&
بتر زاو قرینی که آورد و گفت
\\
یکی تیری افکند و در ره فتاد
&&
وجودم نیازرد و رنجم نداد
\\
تو برداشتی و آمدی سوی من
&&
همی در سپوزی به پهلوی من
\\
بخندید صاحبدل نیکخوی
&&
که سهل است از این صعب تر گو بگوی
\\
هنوز آنچه گفت از بدم اندکی است
&&
از آنها که من دانم از صد یکی است
\\
ز روی گمان بر من اینها که بست
&&
من از خود یقین می‌شناسم که هست
\\
وی امسال پیوست با ما وصال
&&
کجا داندم عیب هفتاد سال؟
\\
به از من کس اندر جهان عیب من
&&
نداند به جز عالم الغیب من
\\
ندیدم چنین نیک پندار کس
&&
که پنداشت عیب من این است و بس
\\
به محشر گواه گناهم گر اوست
&&
ز دوزخ نترسم که کارم نکوست
\\
گرم عیب گوید بد اندیش من
&&
بیا گو ببر نسخه از پیش من
\\
کسان مرد راه خدا بوده‌اند
&&
که برجاس تیر بلا بوده‌اند
\\
زبون باش چون پوستینت درند
&&
که صاحبدلان بار شوخان برند
\\
گر از خاک مردان سبویی کنند
&&
به سنگش ملامت کنان بشکنند
\\
\end{longtable}
\end{center}
