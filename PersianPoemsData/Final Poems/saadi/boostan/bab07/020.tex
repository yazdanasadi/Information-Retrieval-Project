\begin{center}
\section*{بخش ۲۰ - حکایت فریدون و وزیر و غماز: فریدون وزیری پسندیده داشت}
\label{sec:020}
\addcontentsline{toc}{section}{\nameref{sec:020}}
\begin{longtable}{l p{0.5cm} r}
فریدون وزیری پسندیده داشت
&&
که روشن دل و دوربین دیده داشت
\\
رضای حق اول نگه داشتی
&&
دگر پاس فرمان شه داشتی
\\
نهد عامل سفله بر خلق رنج
&&
که تدبیر ملک است و توفیر گنج
\\
اگر جانب حق نداری نگاه
&&
گزندت رساند هم از پادشاه
\\
یکی رفت پیش ملک بامداد
&&
که هر روزت آسایش و کام باد
\\
غرض مشنو از من نصیحت پذیر
&&
تو را در نهان دشمن است این وزیر
\\
کس از خاص لشکر نمانده‌ست و عام
&&
که سیم و زر از وی ندارد به وام
\\
به شرطی که چون شاه گردن فراز
&&
بمیرد، دهند آن زر و سیم باز
\\
نخواهد تو را زنده این خودپرست
&&
مبادا که نقدش نیاید به دست
\\
یکی سوی دستور دولت پناه
&&
به چشم سیاست نگه کرد شاه
\\
که در صورت دوستان پیش من
&&
به خاطر چرایی بد اندیش من؟
\\
زمین پیش تختش ببوسید و گفت
&&
نشاید چو پرسیدی اکنون نهفت
\\
چنین خواهم ای نامور پادشاه
&&
که باشند خلقت همه نیک خواه
\\
چو مرگت بود وعدهٔ سیم من
&&
بقا بیش خواهندت از بیم من
\\
نخواهی که مردم به صدق و نیاز
&&
سرت سبز خواهند و عمرت دراز؟
\\
غنیمت شمارند مردان دعا
&&
که جوشن بود پیش تیر بلا
\\
پسندید از او شهریار آنچه گفت
&&
گل رویش از تازگی برشکفت
\\
ز قدر و مکانی که دستور داشت
&&
مکانش بیفزود و قدرش فراشت
\\
بد اندیش را زجر و تأدیب کرد
&&
پشیمانی از گفتهٔ خویش خورد
\\
ندیدم ز غماز سرگشته‌تر
&&
نگون طالع و بخت برگشته‌تر
\\
ز نادانی و تیره رایی که اوست
&&
خلاف افکند در میان دو دوست
\\
کنند این و آن خوش دگر باره دل
&&
وی اندر میان کور بخت و خجل
\\
میان دو کس آتش افروختن
&&
نه عقل است و خود در میان سوختن
\\
چو سعدی کسی ذوق خلوت چشید
&&
که از هر که عالم زبان درکشید
\\
بگوی آنچه دانی سخن سودمند
&&
وگر هیچ کس را نیاید پسند
\\
که فردا پشیمان برآرد خروش
&&
که آوخ چرا حق نکردم به گوش؟
\\
\end{longtable}
\end{center}
