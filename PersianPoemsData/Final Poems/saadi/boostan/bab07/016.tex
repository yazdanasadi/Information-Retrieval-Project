\begin{center}
\section*{بخش ۱۶ - حکایت: طریقت شناسان ثابت قدم}
\label{sec:016}
\addcontentsline{toc}{section}{\nameref{sec:016}}
\begin{longtable}{l p{0.5cm} r}
طریقت شناسان ثابت قدم
&&
به خلوت نشستند چندی به هم
\\
یکی زان میان غیبت آغاز کرد
&&
در ذکر بیچاره‌ای باز کرد
\\
کسی گفتش ای یار شوریده رنگ
&&
تو هرگز غزا کرده‌ای در فرنگ؟
\\
بگفت از پس چار دیوار خویش
&&
همه عمر ننهاده‌ام پای پیش
\\
چنین گفت درویش صادق نفس
&&
ندیدم چنین بخت برگشته کس
\\
که کافر ز پیکارش ایمن نشست
&&
مسلمان ز جور زبانش نرست
\\
چه خوش گفت دیوانهٔ مرغزی
&&
حدیثی کز او لب به دندان گزی
\\
من ار نام مردم بزشتی برم
&&
نگویم به جز غیبت مادرم
\\
که دانند پروردگان خرد
&&
که طاعت همان به که مادر برد
\\
رفیقی که غایب شد ای نیک نام
&&
دو چیزست از او بر رفیقان حرام
\\
یکی آن که مالش به باطل خورند
&&
دوم آن که نامش به غیبت برند
\\
هر آن کو برد نام مردم به عار
&&
تو خیر خود از وی توقع مدار
\\
که اندر قفای تو گوید همان
&&
که پیش تو گفت از پس مردمان
\\
کسی پیش من در جهان عاقل است
&&
که مشغول خود وز جهان غافل است
\\
\end{longtable}
\end{center}
