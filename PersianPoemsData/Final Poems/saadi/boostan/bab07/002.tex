\begin{center}
\section*{بخش ۲ - گفتار اندر فضیلت خاموشی: اگر پای در دامن آری چو کوه}
\label{sec:002}
\addcontentsline{toc}{section}{\nameref{sec:002}}
\begin{longtable}{l p{0.5cm} r}
اگر پای در دامن آری چو کوه
&&
سرت ز آسمان بگذرد در شکوه
\\
زبان درکش ای مرد بسیار دان
&&
که فردا قلم نیست بر بی زبان
\\
صدف وار گوهرشناسان راز
&&
دهان جز به لؤلؤ نکردند باز
\\
فراوان سخن باشد آکنده گوش
&&
نصیحت نگیرد مگر در خموش
\\
چو خواهی که گویی نفس بر نفس
&&
حلاوت نیابی و گفتار کس
\\
نباید سخن گفت ناساخته
&&
نشاید بریدن نینداخته
\\
تأمل کنان در خطا و صواب
&&
به از ژاژخایان حاضر جواب
\\
کمال است در نفس انسان سخن
&&
تو خود را به گفتار ناقص مکن
\\
کم آواز هرگز نبینی خجل
&&
جوی مشک بهتر که یک توده گل
\\
حذر کن ز نادان ده مرده گوی
&&
چو دانا یکی گوی و پرورده گوی
\\
صد انداختی تیر و هر صد خطاست
&&
اگر هوشمندی یک انداز و راست
\\
چرا گوید آن چیز در خفیه مرد
&&
که گر فاش گردد شود روی زرد؟
\\
مکن پیش دیوار غیبت بسی
&&
بود کز پسش گوش دارد کسی
\\
درون دلت شهربند است راز
&&
نگر تا نبیند در شهر باز
\\
از آن مرد دانا دهان دوخته‌ست
&&
که بیند که شمع از زبان سوخته‌ست
\\
\end{longtable}
\end{center}
