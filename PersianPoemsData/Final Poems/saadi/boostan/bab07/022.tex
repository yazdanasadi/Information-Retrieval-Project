\begin{center}
\section*{بخش ۲۲ - حکایت: جوانی ز ناسازگاری جفت}
\label{sec:022}
\addcontentsline{toc}{section}{\nameref{sec:022}}
\begin{longtable}{l p{0.5cm} r}
جوانی ز ناسازگاری جفت
&&
بر پیرمردی بنالید و گفت
\\
گران باری از دست این خصم چیر
&&
چنان می‌برم کآسیا سنگ زیر
\\
به سختی بنه گفتش، ای خواجه، دل
&&
کس از صبر کردن نگردد خجل
\\
به شب سنگ بالایی ای خانه سوز
&&
چرا سنگ زیرین نباشی به روز؟
\\
چو از گلبنی دیده باشی خوشی
&&
روا باشد ار بار خارش کشی
\\
درختی که پیوسته بارش خوری
&&
تحمل کن آنگه که خارش خوری
\\
\end{longtable}
\end{center}
