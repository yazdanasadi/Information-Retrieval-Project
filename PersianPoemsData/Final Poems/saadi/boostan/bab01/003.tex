\begin{center}
\section*{بخش ۳ - گفتار اندر بخشایش بر ضعیفان: نه بر حکم شرع آب خوردن خطاست}
\label{sec:003}
\addcontentsline{toc}{section}{\nameref{sec:003}}
\begin{longtable}{l p{0.5cm} r}
نه بر حکم شرع آب خوردن خطاست
&&
وگر خون به فتوی بریزی رواست
\\
که را شرع فتوی دهد بر هلاک
&&
الا تا نداری ز کشتنش باک
\\
وگر دانی اندر تبارش کسان
&&
بر ایشان ببخشای و راحت رسان
\\
گنه بود مرد ستمکاره را
&&
چه تاوان زن و طفل بیچاره را؟
\\
تنت زورمند است و لشکر گران
&&
ولیکن در اقلیم دشمن مران
\\
که وی بر حصاری گریزد بلند
&&
رسد کشوری بی گنه را گزند
\\
نظر کن در احوال زندانیان
&&
که ممکن بود بی‌گنه در میان
\\
چو بازارگان در دیارت بمرد
&&
به مالش خساست بود دستبرد
\\
کز آن پس که بر وی بگریند زار
&&
به هم باز گویند خویش و تبار
\\
که مسکین در اقلیم غربت بمرد
&&
متاعی کز او ماند ظالم ببرد
\\
بیندیش از آن طفلک بی پدر
&&
وز آه دل دردمندش حذر
\\
بسا نام نیکوی پنجاه سال
&&
که یک نام زشتش کند پایمال
\\
پسندیده کاران جاوید نام
&&
تطاول نکردند بر مال عام
\\
بر آفاق اگر سر به سر پادشاست
&&
چو مال از توانگر ستاند گداست
\\
بمرد از تهیدستی آزاد مرد
&&
ز پهلوی مسکین شکم پر نکرد
\\
\end{longtable}
\end{center}
