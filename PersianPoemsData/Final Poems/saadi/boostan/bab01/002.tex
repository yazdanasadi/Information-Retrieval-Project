\begin{center}
\section*{بخش ۲ - حکایت در تدبیر و تأخیر در سیاست: ز دریای عمان برآمد کسی}
\label{sec:002}
\addcontentsline{toc}{section}{\nameref{sec:002}}
\begin{longtable}{l p{0.5cm} r}
ز دریای عمان برآمد کسی
&&
سفر کرده هامون و دریا بسی
\\
عرب دیده و ترک و تاجیک و روم
&&
ز هر جنس در نفس پاکش علوم
\\
جهان گشته و دانش اندوخته
&&
سفر کرده و صحبت آموخته
\\
به هیکل قوی چون تناور درخت
&&
ولیکن فرو مانده بی برگ سخت
\\
دو صد رقعه بالای هم دوخته
&&
ز حراق و او در میان سوخته
\\
به شهری در آمد ز دریا کنار
&&
بزرگی در آن ناحیت شهریار
\\
که طبعی نکونامی اندیش داشت
&&
سر عجز در پای درویش داشت
\\
بشستند خدمتگزاران شاه
&&
سر و تن به حمامش از گرد راه
\\
چو بر آستان ملک سر نهاد
&&
نیایش کنان دست بر بر نهاد
\\
درآمد به ایوان شاهنشهی
&&
که بختت جوان باد و دولت رهی
\\
نرفتم در این مملکت منزلی
&&
کز آسیب آزرده دیدم دلی
\\
ندیدم کسی سرگران از شراب
&&
مگر هم خرابات دیدم خراب
\\
ملک را همین ملک پیرایه بس
&&
که راضی نگرد به آزار کس
\\
سخن گفت و دامان گوهر فشاند
&&
به نطقی که شه آستین برفشاند
\\
پسند آمدش حسن گفتار مرد
&&
به نزد خودش خواند و اکرام کرد
\\
زرش داد و گوهر به شکر قدوم
&&
بپرسیدش از گوهر و زاد و بوم
\\
بگفت آنچه پرسیدش از سرگذشت
&&
به قربت ز دیگر کسان بر گذشت
\\
ملک با دل خویش با گفت و گو
&&
که دست وزارت سپارد بدو
\\
ولیکن بتدریج تا انجمن
&&
به سستی نخندند بر رای من
\\
به عقلش بباید نخست آزمود
&&
به قدر هنر پایگاهش فزود
\\
برد بر دل از جور غم بارها
&&
که نا آزموده کند کارها
\\
چو قاضی به فکرت نویسد سجل
&&
نگردد ز دستاربندان خجل
\\
نظر کن چو سوفار داری به شست
&&
نه آنگه که پرتاب کردی ز دست
\\
چو یوسف کسی در صلاح و تمیز
&&
به یک سال باید که گردد عزیز
\\
به ایام تا بر نیاید بسی
&&
نشاید رسیدن به غور کسی
\\
ز هر نوع اخلاق او کشف کرد
&&
خردمند و پاکیزه دین بود مرد
\\
نکو سیرتش دید و روشن قیاس
&&
سخن سنج و مقدار مردم شناس
\\
به رای از بزرگان مهش دید و بیش
&&
نشاندش زبردست دستور خویش
\\
چنان حکمت و معرفت کار بست
&&
که از امر و نهیش درونی نخست
\\
در آورد ملکی به زیر قلم
&&
کز او بر وجودی نیامد الم
\\
زبان همه حرف گیران ببست
&&
که حرفی بدش بر نیامد ز دست
\\
حسودی که یک جو خیانت ندید
&&
به کارش نیامد چو گندم تپید
\\
ز روشن دلش ملک پرتو گرفت
&&
وزیر کهن را غم نو گرفت
\\
ندید آن خردمند را رخنه‌ای
&&
که در وی تواند زدن طعنه‌ای
\\
امین و بد اندیش طشتند و مور
&&
نشاید در او رخنه کردن به زور
\\
ملک را دو خورشید طلعت غلام
&&
به سر بر، کمر بسته بودی مدام
\\
دو پاکیزه پیکر چو حور و پری
&&
چو خورشید و ماه از سدیگر بری
\\
دو صورت که گفتی یکی نیست بیش
&&
نموده در آیینه همتای خویش
\\
سخنهای دانای شیرین سخن
&&
گرفت اندر آن هر دو شمشاد بن
\\
چو دیدند کاوصاف و خلقش نکوست
&&
به طبعش هواخواه گشتند و دوست
\\
در او هم اثر کرد میل بشر
&&
نه میلی چو کوتاه‌بینان به شر
\\
از آسایش آنگه خبر داشتی
&&
که در روی ایشان نظر داشتی
\\
چو خواهی که قدرت بماند بلند
&&
دل، ای خواجه، در ساده رویان مبند
\\
وگر خود نباشد غرض در میان
&&
حذر کن که دارد به هیبت زیان
\\
وزیر اندر این شمه‌ای راه برد
&&
به خبث این حکایت بر شاه برد
\\
که این را ندانم چه خوانند و کیست!
&&
نخواهد به سامان در این ملک زیست
\\
سفر کردگان لاابالی زیند
&&
که پروردهٔ ملک و دولت نیند
\\
شنیدم که با بندگانش سر است
&&
خیانت پسند است و شهوت پرست
\\
نشاید چنین خیره روی تباه
&&
که بد نامی آرد در ایوان شاه
\\
مگر نعمت شه فرامش کنم
&&
که بینم تباهی و خامش کنم
\\
به پندار نتوان سخن گفت زود
&&
نگفتم تو را تا یقینم نبود
\\
ز فرمانبرانم کسی گوش داشت
&&
که آغوش را اندر آغوش داشت
\\
من این گفتم اکنون ملک راست رای
&&
چو من آزمودم تو نیز آزمای
\\
به ناخوب تر صورتی شرح داد
&&
که بد مرد را نیکروزی مباد
\\
بداندیش بر خرده چون دست یافت
&&
درون بزرگان به آتش بتافت
\\
به خرده توان آتش افروختن
&&
پس آنگه درخت کهن سوختن
\\
ملک را چنان گرم کرد این خبر
&&
که جوشش برآمد چو مرجل به سر
\\
غضب دست در خون درویش داشت
&&
ولیکن سکون دست در پیش داشت
\\
که پرورده کشتن نه مردی بود
&&
ستم در پی داد، سردی بود
\\
میازار پروردهٔ خویشتن
&&
چو تیر تو دارد به تیرش مزن
\\
به نعمت نبایست پروردنش
&&
چو خواهی به بیداد خون خوردنش
\\
از او تا هنرها یقینت نشد
&&
در ایوان شاهی قرینت نشد
\\
کنون تا یقینت نگردد گناه
&&
به گفتار دشمن گزندش مخواه
\\
ملک در دل این راز پوشیده داشت
&&
که قول حکیمان نیوشیده داشت
\\
دل است، ای خردمند، زندان راز
&&
چو گفتی نیاید به زنجیر باز
\\
نظر کرد پوشیده در کار مرد
&&
خلل دید در رای هشیار مرد
\\
که ناگه نظر زی یکی بنده کرد
&&
پری چهره در زیر لب خنده کرد
\\
دو کس را که با هم بود جان و هوش
&&
حکایت کنانند و ایشان خموش
\\
چو دیده به دیدار کردی دلیر
&&
نگردی چو مستسقی از دجله سیر
\\
ملک را گمان بدی راست شد
&&
ز سودا بر او خشمگین خواست شد
\\
هم از حسن تدبیر و رای تمام
&&
به آهستگی گفتش ای نیک نام
\\
تو را من خردمند پنداشتم
&&
بر اسرار ملکت امین داشتم
\\
گمان بردمت زیرک و هوشمند
&&
ندانستمت خیره و ناپسند
\\
چنین مرتفع پایه جای تو نیست
&&
گناه از من آمد خطای تو نیست
\\
که چون بدگهر پرورم لاجرم
&&
خیانت روا داردم در حرم
\\
برآورد سر مرد بسیاردان
&&
چنین گفت با خسرو کاردان
\\
مرا چون بود دامن از جرم پاک
&&
نباشد ز خبث بداندیش باک
\\
به خاطر درم هرگز این ظن نرفت
&&
ندانم که گفت آنچه بر من نرفت
\\
شهنشاه گفت: آنچه گفتم برت
&&
بگویند خصمان به روی اندرت
\\
چنین گفت با من وزیر کهن
&&
تو نیز آنچه دانی بگوی و بکن
\\
تسبم کنان دست بر لب گرفت
&&
کز او هر چه آید نیاید شگفت
\\
حسودی که بیند به جای خودم
&&
کجا بر زبان آورد جز بدم
\\
من آن ساعت انگاشتم دشمنش
&&
که بنشاند شه زیردست منش
\\
چو سلطان فضیلت نهد بر ویم
&&
ندانی که دشمن بود در پیم؟
\\
مرا تا قیامت نگیرد به دوست
&&
چو بیند که در عز من ذل اوست
\\
بر اینت بگویم حدیثی درست
&&
اگر گوش با بنده داری نخست
\\
ندانم کجا دیده‌ام در کتاب
&&
که ابلیس را دید شخصی به خواب
\\
به بالا صنوبر، به دیدن چو حور
&&
چو خورشیدش از چهره می‌تافت نور
\\
فرا رفت و گفت: ای عجب، این تویی
&&
فرشته نباشد بدین نیکویی
\\
تو کاین روی داری به حسن قمر
&&
چرا در جهانی به زشتی سمر؟
\\
چرا نقش بندت در ایوان شاه
&&
دژم روی کرده‌ست و زشت و تباه؟
\\
شنید این سخن بخت برگشته دیو
&&
به زاری برآورد بانگ و غریو
\\
که ای نیکبخت این نه شکل من است
&&
ولیکن قلم در کف دشمن است
\\
مرا همچنین نام نیک است لیک
&&
ز علت نگوید بداندیش نیک
\\
وزیری که جاه من آبش بریخت
&&
به فرسنگ باید ز مکرش گریخت
\\
ولیکن نیندیشم از خشم شاه
&&
دلاور بود در سخن، بی‌گناه
\\
اگر محتسب گردد آن را غم است
&&
که سنگ ترازوی بارش کم است
\\
چو حرفم برآید درست از قلم
&&
مرا از همه حرف گیران چه غم؟
\\
ملک در سخن گفتنش خیره ماند
&&
سر دست فرماندهی برفشاند
\\
که مجرم به زرق و زبان آوری
&&
ز جرمی که دارد نگردد بری
\\
ز خصمت همانا که نشنیده‌ام
&&
نه آخر به چشم خودم دیده‌ام؟
\\
کز این زمره خلق در بارگاه
&&
نمی‌باشدت جز در اینان نگاه
\\
بخندید مرد سخنگوی و گفت
&&
حق است این سخن، حق نشاید نهفت
\\
در این نکته‌ای هست اگر بشنوی
&&
که حکمت روان باد و دولت قوی
\\
نبینی که درویش بی دستگاه
&&
به حسرت کند در توانگر نگاه
\\
مرا دستگاه جوانی برفت
&&
به لهو و لعب زندگانی برفت
\\
ز دیدار اینان ندارم شکیب
&&
که سرمایه داران حسنند و زیب
\\
مرا همچنین چهره گلفام بود
&&
بلورینم از خوبی اندام بود
\\
در این غایتم رشت باید کفن
&&
که مویم چو پنبه‌ست و دوکم بدن
\\
مرا همچنین جعد شبرنگ بود
&&
قبا در بر از نازکی تنگ بود
\\
دو رسته درم در دهن داشت جای
&&
چو دیواری از خشت سیمین بپای
\\
کنونم نگه کن به وقت سخن
&&
بیفتاده یک یک چو سور کهن
\\
در اینان به حسرت چرا ننگرم؟
&&
که عمر تلف کرده یاد آورم
\\
برفت از من آن روزهای عزیز
&&
به پایان رسد ناگه این روز نیز
\\
چو دانشور این در معنی بسفت
&&
بگفت این کز این به محال است گفت
\\
در ارکان دولت نگه کرد شاه
&&
کز این خوبتر لفظ و معنی مخواه
\\
کسی را نظر سوی شاهد رواست
&&
که داند بدین شاهدی عذر خواست
\\
به عقل ار نه آهستگی کردمی
&&
به گفتار خصمش بیازردمی
\\
به تندی سبک دست بردن به تیغ
&&
به دندان برد پشت دست دریغ
\\
ز صاحب غرض تا سخن نشنوی
&&
که گر کار بندی پشیمان شوی
\\
نکونام را جاه و تشریف و مال
&&
بیفزود و، بدگوی را گوشمال
\\
به تدبیر دستور دانشورش
&&
به نیکی بشد نام در کشورش
\\
به عدل و کرم سالها ملک راند
&&
برفت و نکونامی از وی بماند
\\
چنین پادشاهان که دین پرورند
&&
به بازوی دین، گوی دولت برند
\\
از آنان نبینم در این عهد کس
&&
وگر هست بوبکر سعد است و بس
\\
بهشتی درختی تو، ای پادشاه
&&
که افکنده‌ای سایه یک ساله راه
\\
طمع بود از بخت نیک اخترم
&&
که بال همای افکند بر سرم
\\
خرد گفت دولت نبخشد همای
&&
گر اقبال خواهی در این سایه آی
\\
خدایا به رحمت نظر کرده‌ای
&&
که این سایه بر خلق گسترده‌ای
\\
دعا گوی این دولتم بنده‌وار
&&
خدایا تو این سایه پاینده دار
\\
صواب است پیش از کشش بند کرد
&&
که نتوان سر کشته پیوند کرد
\\
خداوند فرمان و رای و شکوه
&&
ز غوغای مردم نگردد ستوه
\\
سر پر غرور از تحمل تهی
&&
حرامش بود تاج شاهنشهی
\\
نگویم چو جنگ آوری پای دار
&&
چو خشم آیدت عقل بر جای دار
\\
تحمل کند هر که را عقل هست
&&
نه عقلی که خشمش کند زیردست
\\
چو لشکر برون تاخت خشم از کمین
&&
نه انصاف ماند نه تقوی نه دین
\\
ندیدم چنین دیو زیر فلک
&&
که از وی گریزند چندین ملک
\\
\end{longtable}
\end{center}
