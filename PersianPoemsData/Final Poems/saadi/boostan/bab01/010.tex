\begin{center}
\section*{بخش ۱۰ - حکایت ملک روم با دانشمند: شنیدم که بگریست سلطان روم}
\label{sec:010}
\addcontentsline{toc}{section}{\nameref{sec:010}}
\begin{longtable}{l p{0.5cm} r}
شنیدم که بگریست سلطان روم
&&
بر نیکمردی ز اهل علوم
\\
که پایابم از دست دشمن نماند
&&
جز این قلعه و شهر با من نماند
\\
بسی جهد کردم که فرزند من
&&
پس از من بود سرور انجمن
\\
کنون دشمن بدگهر دست یافت
&&
سر دست مردی و جهدم بتافت
\\
چه تدبیر سازم، چه درمان کنم؟
&&
که از غم بفرسود جان در تنم
\\
بگفت ای برادر غم خویش خور
&&
که از عمر بهتر شد و بیشتر
\\
تو را این قدر تا بمانی بس است
&&
چو رفتی جهان جای دیگر کس است
\\
اگر هوشمند است و گر بی‌خرد
&&
غم او مخور کاو غم خود خورد
\\
مشقت نیرزد جهان داشتن
&&
گرفتن به شمشیر و بگذاشتن
\\
بدین پنج روزه اقامت مناز
&&
به اندیشه تدبیر رفتن بساز
\\
که را دانی از خسروان عجم
&&
ز عهد فریدون و ضحاک و جم
\\
که بر تخت و ملکش نیامد زوال؟
&&
نماند به جز ملک ایزد تعال
\\
که را جاودان ماندن امید ماند
&&
چو کس را نبینی که جاوید ماند؟
\\
که را سیم و زر ماند و گنج و مال
&&
پس از وی به چندی شود پایمال
\\
وز آن کس که خیری بماند روان
&&
دمادم رسد رحمتش بر روان
\\
بزرگی کز او نام نیکو نماند
&&
توان گفت با اهل دل کاو نماند
\\
الا تا درخت کرم پروری
&&
گر امیدواری کز او بر خوری
\\
کرم کن که فردا که دیوان نهند
&&
منازل به مقدار احسان دهند
\\
یکی را که سعی قدم پیشتر
&&
به درگاه حق، منزلت بیشتر
\\
یکی باز پس خائن و شرمسار
&&
بترسد همی مرد ناکرده کار
\\
بهل تا به دندان گزد پشت دست
&&
تنوری چنین گرم و نانی نبست
\\
بدانی گه غله برداشتن
&&
که سستی بود تخم ناکاشتن
\\
\end{longtable}
\end{center}
