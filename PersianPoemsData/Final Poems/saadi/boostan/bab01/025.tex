\begin{center}
\section*{بخش ۲۵ - در تغیر روزگار و انتقال مملکت: شنیدم که در مصر میری اجل}
\label{sec:025}
\addcontentsline{toc}{section}{\nameref{sec:025}}
\begin{longtable}{l p{0.5cm} r}
شنیدم که در مصر میری اجل
&&
سپه تاخت بر روزگارش اجل
\\
جمالش برفت از رخ دل فروز
&&
چو خور زرد شد بس نماند ز روز
\\
گزیدند فرزانگان دست فوت
&&
که در طب ندیدند داروی موت
\\
همه تخت و ملکی پذیرد زوال
&&
به جز ملک فرمانده لایزال
\\
چو نزدیک شد روز عمرش به شب
&&
شنیدند می‌گفت در زیر لب
\\
که در مصر چون من عزیزی نبود
&&
چو حاصل همین بود چیزی نبود
\\
جهان گرد کردم نخوردم برش
&&
برفتم چو بیچارگان از سرش
\\
پسندیده رایی که بخشید و خورد
&&
جهان از پی خویشتن گرد کرد
\\
در این کوش تا با تو ماند مقیم
&&
که هرچ از تو ماند دریغ است و بیم
\\
کند خواجه بر بستر جان‌گداز
&&
یکی دست کوتاه و دیگر دراز
\\
در آن دم تو را می‌نماید به دست
&&
که دهشت زبانش ز گفتن ببست
\\
که دستی به جود و کرم کن دراز
&&
دگر دست کوته کن از ظلم و آز
\\
کنونت که دست است خاری بکن
&&
دگر کی بر آری تو دست از کفن؟
\\
بتابد بسی ماه و پروین و هور
&&
که سر بر نداری ز بالین گور
\\
\end{longtable}
\end{center}
