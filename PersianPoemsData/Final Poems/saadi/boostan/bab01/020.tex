\begin{center}
\section*{بخش ۲۰ - حکایت شحنه مردم آزار: گزیری به چاهی در افتاده بود}
\label{sec:020}
\addcontentsline{toc}{section}{\nameref{sec:020}}
\begin{longtable}{l p{0.5cm} r}
گزیری به چاهی در افتاده بود
&&
که از هول او شیر نر ماده بود
\\
بداندیش مردم به جز بد ندید
&&
بیافتاد و عاجزتر از خود ندید
\\
همه شب ز فریاد و زاری نخفت
&&
یکی بر سرش کوفت سنگی و گفت:
\\
تو هرگز رسیدی به فریاد کس
&&
که می‌خواهی امروز فریادرس؟
\\
همه تخم نامردمی کاشتی
&&
ببین لاجرم بر که برداشتی
\\
که بر جان ریشت نهد مرهمی
&&
که دلها ز ریشت بنالد همی؟
\\
تو ما را همی چاه کندی به راه
&&
به سر لاجرم در فتادی به چاه
\\
دو کس چه کنند از پی خاص و عام
&&
یکی نیک محضر، دگر زشت نام
\\
یکی تشنه را تا کند تازه حلق
&&
دگر تا به گردن در افتند خلق
\\
اگر بد کنی چشم نیکی مدار
&&
که هرگز نیارد گز انگور بار
\\
نپندارم ای در خزان کشته جو
&&
که گندم ستانی به وقت درو
\\
درخت زقوم ار به جان پروری
&&
مپندار هرگز کز او بر خوری
\\
رطب ناورد چوب خرزهره بار
&&
چو تخم افکنی، بر همان چشم دار
\\
\end{longtable}
\end{center}
