\begin{center}
\section*{بخش ۳۷ - گفتار اندر حذر کردن از دشمنان: نگویم ز جنگ بد اندیش ترس}
\label{sec:037}
\addcontentsline{toc}{section}{\nameref{sec:037}}
\begin{longtable}{l p{0.5cm} r}
نگویم ز جنگ بد اندیش ترس
&&
در آوازهٔ صلح از او بیش ترس
\\
بسا کس به روز آیت صلح خواند
&&
چو شب شد سپه بر سر خفته راند
\\
زره پوش خسبند مرد اوژنان
&&
که بستر بود خوابگاه زنان
\\
به خیمه درون مرد شمشیر زن
&&
برهنه نخسبد چو در خانه زن
\\
بباید نهان جنگ را ساختن
&&
که دشمن نهان آورد تاختن
\\
حذر کار مردان کار آگه است
&&
یزک سد رویین لشکر گه است
\\
\end{longtable}
\end{center}
