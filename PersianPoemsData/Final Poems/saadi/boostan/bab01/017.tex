\begin{center}
\section*{بخش ۱۷ - صفت جمعیت اوقات درویشان راضی: مگو جاهی از سلطنت بیش نیست}
\label{sec:017}
\addcontentsline{toc}{section}{\nameref{sec:017}}
\begin{longtable}{l p{0.5cm} r}
مگو جاهی از سلطنت بیش نیست
&&
که ایمن‌تر از ملک درویش نیست
\\
سبکبار مردم سبک‌تر روند
&&
حق این است و صاحبدلان بشنوند
\\
تهیدست تشویش نانی خورد
&&
جهانبان به قدر جهانی خورد
\\
گدا را چو حاصل شود نان شام
&&
چنان خوش بخسبد که سلطان شام
\\
غم و شادمانی به سر می‌رود
&&
به مرگ این دو از سر به در می‌رود
\\
چه آن را که بر سر نهادند تاج
&&
چه آن را که بر گردن آمد خراج
\\
اگر سرفرازی به کیوان بر است
&&
وگر تنگدستی به زندان در است
\\
چو خیل اجل بر سر هر دو تاخت
&&
نمی شاید از یکدگرشان شناخت
\\
\end{longtable}
\end{center}
