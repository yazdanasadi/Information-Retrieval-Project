\begin{center}
\section*{بخش ۲۳ - حکایت در این معنی: یکی را حکایت کنند از ملوک}
\label{sec:023}
\addcontentsline{toc}{section}{\nameref{sec:023}}
\begin{longtable}{l p{0.5cm} r}
یکی را حکایت کنند از ملوک
&&
که بیماری رشته کردش چو دوک
\\
چنانش در انداخت ضعف جسد
&&
که می‌برد بر زیردستان حسد
\\
که شاه ار چه بر عرصه نام آور است
&&
چو ضعف آمد از بیدقی کمتر است
\\
ندیمی زمین ملک بوسه داد
&&
که ملک خداوند جاوید باد
\\
در این شهر مردی مبارک دم است
&&
که در پارسایی چنویی کم است
\\
نرفته‌ست هرگز ره ناصواب
&&
دلی روشن و دعوتی مستجاب
\\
نبردند پیشش مهمات کس
&&
که مقصود حاصل نشد در نفس
\\
بخوان تا بخواند دعایی بر این
&&
که رحمت رسد ز آسمان برین
\\
بفرمود تا مهتران خدم
&&
بخواندند پیر مبارک قدم
\\
برفتند و گفتند و آمد فقیر
&&
تنی محتشم در لباسی حقیر
\\
بگفتا دعایی کن ای هوشمند
&&
که در رشته چون سوزنم پای‌بند
\\
شنید این سخن پیر خم بوده پشت
&&
به تندی برآورد بانگی درشت
\\
که حق مهربان است بر دادگر
&&
ببخشای و بخشایش حق نگر
\\
دعای منت کی شود سودمند
&&
اسیران محتاج در چاه و بند؟
\\
تو ناکرده بر خلق بخشایشی
&&
کجا بینی از دولت آسایشی؟
\\
ببایدت عذر خطا خواستن
&&
پس از شیخ صالح دعا خواستن
\\
کجا دست گیرد دعای ویت
&&
دعای ستمدیدگان در پیت؟
\\
شنید این سخن شهریار عجم
&&
ز خشم و خجالت بر آمد بهم
\\
برنجید و پس با دل خویش گفت
&&
چه رنجم؟ حق است این که درویش گفت
\\
بفرمود تا هر که در بند بود
&&
به فرمانش آزاد کردند زود
\\
جهاندیده بعد از دو رکعت نماز
&&
به داور برآورد دست نیاز
\\
که ای برفرازندهٔ آسمان
&&
به جنگش گرفتی به صلحش بمان
\\
ولی همچنان بر دعا داشت دست
&&
که شه سر برآورد و بر پای جست
\\
تو گفتی ز شادی بخواهد پرید
&&
چو طاووس، چون رشته در پا ندید
\\
بفرمود گنجینهٔ گوهرش
&&
فشاندند در پای و زر بر سرش
\\
حق از بهر باطل نشاید نهفت
&&
از آن جمله دامن بیفشاند و گفت
\\
مرو با سر رشته بار دگر
&&
مبادا که دیگر کند رشته سر
\\
چو باری فتادی نگه‌دار پای
&&
که یک بار دیگر بلغزد ز جای
\\
ز سعدی شنو کاین سخن راست است
&&
نه هر باری افتاده برخاسته‌ست
\\
\end{longtable}
\end{center}
