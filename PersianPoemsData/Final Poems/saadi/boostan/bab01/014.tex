\begin{center}
\section*{بخش ۱۴ - حکایت: شبی دود خلق آتشی برفروخت}
\label{sec:014}
\addcontentsline{toc}{section}{\nameref{sec:014}}
\begin{longtable}{l p{0.5cm} r}
شبی دود خلق آتشی برفروخت
&&
شنیدم که بغداد نیمی بسوخت
\\
یکی شکر گفت اندران خاک و دود
&&
که دکان ما را گزندی نبود
\\
جهاندیده‌ای گفتش ای بوالهوس
&&
تو را خود غم خویشتن بود و بس؟
\\
پسندی که شهری بسوزد به نار
&&
اگر چه سرایت بود بر کنار؟
\\
به جز سنگدل ناکند معده تنگ
&&
چو بیند کسان بر شکم بسته سنگ
\\
توانگر خود آن لقمه چون می‌خورد
&&
چو بیند که درویش خون می‌خورد؟
\\
مگو تندرست است رنجوردار
&&
که می‌پیچد از غصه رنجوروار
\\
تنکدل چو یاران به منزل رسند
&&
نخسبد که واماندگان از پسند
\\
دل پادشاهان شود بارکش
&&
چو بینند در گل خر خارکش
\\
اگر در سرای سعادت کس است
&&
ز گفتار سعدیش حرفی بس است
\\
همینت بسنده‌ست اگر بشنوی
&&
که گر خار کاری سمن ندروی
\\
\end{longtable}
\end{center}
