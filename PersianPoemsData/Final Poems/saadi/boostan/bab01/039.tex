\begin{center}
\section*{بخش ۳۹ - گفتار اندر ملاطفت با دشمن از روی عاقبت اندیشی: چو شمشیر پیکار برداشتی}
\label{sec:039}
\addcontentsline{toc}{section}{\nameref{sec:039}}
\begin{longtable}{l p{0.5cm} r}
چو شمشیر پیکار برداشتی
&&
نگه دار پنهان ره آشتی
\\
که لشکر شکوفان مغفر شکاف
&&
نهان صلح جستند و پیدا مصاف
\\
دل مرد میدان نهانی بجوی
&&
که باشد که در پایت افتد چو گوی
\\
چو سالاری از دشمن افتد به چنگ
&&
به کشتن درش کرد باید درنگ
\\
که افتد کز این نیمه هم سروری
&&
بماند گرفتار در چنبری
\\
اگر کشتی این بندی ریش را
&&
نبینی دگر بندی خویش را
\\
نترسد که دورانش بندی کند
&&
که بر بندیان زورمندی کند؟
\\
کسی بندیان را بود دستگیر
&&
که خود بوده باشد به بندی اسیر
\\
اگر سر نهد بر خطت سروری
&&
چو نیکش بداری، نهد دیگری
\\
اگر خفیه ده دل بدست آوری
&&
از آن به که صد ره شبیخون بری
\\
\end{longtable}
\end{center}
