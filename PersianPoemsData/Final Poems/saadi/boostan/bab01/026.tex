\begin{center}
\section*{بخش ۲۶ - حکایت قزل ارسلان با دانشمند: قزل ارسلان قلعه‌ای سخت داشت}
\label{sec:026}
\addcontentsline{toc}{section}{\nameref{sec:026}}
\begin{longtable}{l p{0.5cm} r}
قزل ارسلان قلعه‌ای سخت داشت
&&
که گردن به الوند بر می‌فراشت
\\
نه اندیشه از کس نه حاجت به هیچ
&&
چو زلف عروسان رهش پیچ پیچ
\\
چنان نادر افتاده در روضه‌ای
&&
که بر لاجوردی طبق بیضه‌ای
\\
شنیدم که مردی مبارک حضور
&&
به نزدیک شاه آمد از راه دور
\\
حقایق شناسی، جهاندیده‌ای
&&
هنرمندی، آفاق گردیده‌ای
\\
بزرگی، زبان آوری کاردان
&&
حکیمی، سخنگوی بسیاردان
\\
قزل گفت چندین که گردیده‌ای
&&
چنین جای محکم دگر دیده‌ای؟
\\
بخندید کاین قلعه‌ای خرم است
&&
ولیکن نپندارمش محکم است
\\
نه پیش از تو گردن کشان داشتند
&&
دمی چند بودند و بگذاشتند؟
\\
نه بعد از تو شاهان دیگر برند
&&
درخت امید تو را بر خورند؟
\\
ز دوران ملک پدر یاد کن
&&
دل از بند اندیشه آزاد کن
\\
چنان روزگارش به کنجی نشاند
&&
که بر یک پشیزش تصرف نماند
\\
چو نومید ماند از همه چیز و کس
&&
امیدش به فضل خدا ماند و بس
\\
بر مرد هشیار دنیا خس است
&&
که هر مدتی جای دیگر کس است
\\
چنین گفت شوریده‌ای در عجم
&&
به کسری که ای وارث ملک جم
\\
اگر ملک بر جم بماندی و بخت
&&
تو را کی میسر شدی تاج و تخت؟
\\
اگر گنج قارون به دست آوری
&&
نماند مگر آنچه بخشی، بری
\\
\end{longtable}
\end{center}
