\begin{center}
\section*{بخش ۵ - مدح محمد بن سعد بن ابوبکر: اتابک محمد شه نیکبخت}
\label{sec:005}
\addcontentsline{toc}{section}{\nameref{sec:005}}
\begin{longtable}{l p{0.5cm} r}
اتابک محمد شه نیکبخت
&&
خداوند تاج و خداوند تخت
\\
جوان جوان‌بخت روشن‌ضمیر
&&
به دولت جوان و به تدبیر پیر
\\
به دانش بزرگ و به همت بلند
&&
به بازو دلیر و به دل هوشمند
\\
زهی دولت مادر روزگار
&&
که رودی چنین پرورد در کنار
\\
به دست کرم آب دریا ببرد
&&
به رفعت محل ثریا ببرد
\\
زهی چشم دولت به روی تو باز
&&
سر شهریاران گردن فراز
\\
صدف را که بینی ز دردانه پر
&&
نه آن قدر دارد که یکدانه در
\\
تو آن در مکنون یکدانه‌ای
&&
که پیرایهٔ سلطنت خانه‌ای
\\
نگه دار یارب به چشم خودش
&&
بپرهیز از آسیب چشم بدش
\\
خدایا در آفاق نامی کنش
&&
به توفیق طاعت گرامی کنش
\\
مقیمش در انصاف و تقوی بدار
&&
مرادش به دنیا و عقبی برآر
\\
غم از دشمن ناپسندش مباد
&&
وز اندیشه بر دل گزندش مباد
\\
بهشتی درخت آورد چون تو بار
&&
پسر نامجوی و پدر نامدار
\\
از آن خاندان خیر بیگانه دان
&&
که باشند بدخواه این خاندان
\\
زهی دین و دانش، زهی عدل و داد
&&
زهی ملک و دولت که پاینده باد
\\
نگنجد کرمهای حق در قیاس
&&
چه خدمت گزارد زبان سپاس؟
\\
خدایا تو این شاه درویش دوست
&&
که آسایش خلق در ظل اوست
\\
بسی بر سر خلق پاینده دار
&&
به توفیق طاعت دلش زنده دار
\\
برومند دارش درخت امید
&&
سرش سبز و رویش به رحمت سفید
\\
به راه تکلف مرو سعدیا
&&
اگر صدق داری بیار و بیا
\\
تو منزل شناسی و شه راهرو
&&
تو حقگوی و خسرو حقایق شنو
\\
چه حاجت که نه کرسی آسمان
&&
نهی زیر پای قزل ارسلان
\\
مگو پای عزت بر افلاک نه
&&
بگو روی اخلاص بر خاک نه
\\
بطاعت بنه چهره بر آستان
&&
که این است سر جاده راستان
\\
اگر بنده‌ای سر بر این در بنه
&&
کلاه خداوندی از سر بنه
\\
به درگاه فرمانده ذوالجلال
&&
چو درویش پیش توانگر بنال
\\
چو طاعت کنی لبس شاهی مپوش
&&
چو درویش مخلص برآور خروش
\\
که پروردگارا توانگر تویی
&&
توانا و درویش پرور تویی
\\
نه کشور خدایم نه فرماندهم
&&
یکی از گدایان این درگهم
\\
تو بر خیر و نیکی دهم دسترس
&&
وگر نه چه خیر آید از من به کس؟
\\
دعا کن به شب چون گدایان به سوز
&&
اگر می‌کنی پادشاهی به روز
\\
کمر بسته گردن کشان بر درت
&&
تو بر آستان عبادت سرت
\\
زهی بندگان را خداوندگار
&&
خداوند را بندهٔ حق گزار
\\
\end{longtable}
\end{center}
