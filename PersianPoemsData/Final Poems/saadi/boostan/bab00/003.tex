\begin{center}
\section*{بخش ۳ - سبب نظم کتاب: در اقصای عالم بگشتم بسی}
\label{sec:003}
\addcontentsline{toc}{section}{\nameref{sec:003}}
\begin{longtable}{l p{0.5cm} r}
در اقصای عالم بگشتم بسی
&&
به سر بردم ایام با هر کسی
\\
تمتع به هر گوشه‌ای یافتم
&&
ز هر خرمنی خوشه‌ای یافتم
\\
چو پاکان شیراز، خاکی نهاد
&&
ندیدم که رحمت بر این خاک باد
\\
تولای مردان این پاک بوم
&&
برانگیختم خاطر از شام و روم
\\
دریغ آمدم زآن همه بوستان
&&
تهیدست رفتن سوی دوستان
\\
به دل گفتم از مصر قند آورند
&&
بر دوستان ارمغانی برند
\\
مرا گر تهی بود از آن قند دست
&&
سخنهای شیرین‌تر از قند هست
\\
نه قندی که مردم به صورت خورند
&&
که ارباب معنی به کاغذ برند
\\
چو این کاخ دولت بپرداختم
&&
بر او ده در از تربیت ساختم
\\
یکی باب عدل است و تدبیر و رای
&&
نگهبانی خلق و ترس خدای
\\
دوم باب احسان نهادم اساس
&&
که منعم کند فضل حق را سپاس
\\
سوم باب عشق است و مستی و شور
&&
نه عشقی که بندند بر خود بزور
\\
چهارم تواضع، رضا پنجمین
&&
ششم ذکر مرد قناعت گزین
\\
به هفتم در از عالم تربیت
&&
به هشتم در از شکر بر عافیت
\\
نهم باب توبه است و راه صواب
&&
دهم در مناجات و ختم کتاب
\\
به روز همایون و سال سعید
&&
به تاریخ فرخ میان دو عید
\\
ز ششصد فزون بود پنجاه و پنج
&&
که پر در شد این نامبردار گنج
\\
بمانده‌ست با دامنی گوهرم
&&
هنوز از خجالت به زانو برم
\\
که در بحر لؤلؤ صدف نیز هست
&&
درخت بلند است در باغ و پست
\\
الا ای خردمند پاکیزه خوی
&&
خردمند نشنیده‌ام عیب جوی
\\
قبا گر حریر است و گر پرنیان
&&
به ناچار حشوش بود در میان
\\
تو گر پرنیانی نیابی مجوش
&&
کرم کار فرما و حشوش بپوش
\\
ننازم به سرمایهٔ فضل خویش
&&
به دریوزه آورده‌ام دست پیش
\\
شنیدم که در روز امید و بیم
&&
بدان را به نیکان ببخشد کریم
\\
تو نیز ار بدی بینیم در سخن
&&
به خلق جهان آفرین کار کن
\\
چو بیتی پسند آیدت از هزار
&&
به مردی که دست از تعنت بدار
\\
همانا که در فارس انشای من
&&
چو مشک است بی قیمت اندر ختن
\\
چو بانگ دهل هولم از دور بود
&&
به غیبت درم عیب مستور بود
\\
گل آورد سعدی سوی بوستان
&&
به شوخی و فلفل به هندوستان
\\
چو خرما به شیرینی اندوده پوست
&&
چو بازش کنی استخوانی در اوست
\\
\end{longtable}
\end{center}
