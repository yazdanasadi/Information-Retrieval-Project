\begin{center}
\section*{بخش ۱۱ - حکایت در صبر بر جفای آن که از او صبر نتوان کرد: شکایت کند نوعروسی جوان}
\label{sec:011}
\addcontentsline{toc}{section}{\nameref{sec:011}}
\begin{longtable}{l p{0.5cm} r}
شکایت کند نوعروسی جوان
&&
به پیری ز داماد نامهربان
\\
که مپسند چندین که با این پسر
&&
به تلخی رود روزگارم به سر
\\
کسانی که با ما در این منزلند
&&
نبینم که چون من پریشان دلند
\\
زن و مرد با هم چنان دوستند
&&
که گویی دو مغز و یکی پوستند
\\
ندیدم در این مدت از شوی من
&&
که باری بخندید در روی من
\\
شنید این سخن پیر فرخنده فال
&&
سخندان بود مرد دیرینه سال
\\
یکی پاسخش داد شیرین و خوش
&&
که گر خوبروی است بارش بکش
\\
دریغ است روی از کسی تافتن
&&
که دیگر نشاید چنو یافتن
\\
چرا سر کشی زان که گر سر کشد
&&
به حرف وجودت قلم در کشد؟
\\
یکم روز بر بنده‌ای دل بسوخت
&&
که می‌گفت و فرماندهش می‌فروخت
\\
تو را بنده از من به افتد بسی
&&
مرا چون تو دیگر نیفتد کسی
\\
\end{longtable}
\end{center}
