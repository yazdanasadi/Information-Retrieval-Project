\begin{center}
\section*{بخش ۲ - تقریر عشق مجازی و قوت آن: تو را عشق همچون خودی ز آب و گل}
\label{sec:002}
\addcontentsline{toc}{section}{\nameref{sec:002}}
\begin{longtable}{l p{0.5cm} r}
تو را عشق همچون خودی ز آب و گل
&&
رباید همی صبر و آرام دل
\\
به بیداریش فتنه بر خد و خال
&&
به خواب اندرش پای بند خیال
\\
به صدقش چنان سر نهی در قدم
&&
که بینی جهان با وجودش عدم
\\
چو در چشم شاهد نیاید زرت
&&
زر و خاک یکسان نماید برت
\\
دگر با کست بر نیاید نفس
&&
که با او نماند دگر جای کس
\\
تو گویی به چشم اندرش منزل است
&&
وگر دیده بر هم نهی در دل است
\\
نه اندیشه از کس که رسوا شوی
&&
نه قوت که یک دم شکیبا شوی
\\
گرت جان بخواهد به لب بر نهی
&&
ورت تیغ بر سر نهد سر نهی
\\
\end{longtable}
\end{center}
