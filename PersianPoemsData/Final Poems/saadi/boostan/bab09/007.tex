\begin{center}
\section*{بخش ۷ - حکایت در معنی بیداری از خواب غفلت: فرو رفت جم را یکی نازنین}
\label{sec:007}
\addcontentsline{toc}{section}{\nameref{sec:007}}
\begin{longtable}{l p{0.5cm} r}
فرو رفت جم را یکی نازنین
&&
کفن کرد چون کرمش ابریشمین
\\
به دخمه برآمد پس از چند روز
&&
که بر وی بگرید به زاری و سوز
\\
چو پوسیده دیدش حریرین کفن
&&
به فکرت چنین گفت با خویشتن
\\
من از کرم برکنده بودم به زور
&&
بکندند از او باز کرمان گور
\\
در این باغ سروی نیامد بلند
&&
که باد اجل بیخش از بن نکند
\\
قضا نقش یوسف جمالی نکرد
&&
که ماهی گورش چو یونس نخورد
\\
دو بیتم جگر کرد روزی کباب
&&
که می‌گفت گوینده‌ای با رباب:
\\
دریغا که بی ما بسی روزگار
&&
بروید گل و بشکفد نوبهار
\\
بسی تیر و دی ماه و اردیبهشت
&&
برآید که ما خاک باشیم و خشت
\\
\end{longtable}
\end{center}
