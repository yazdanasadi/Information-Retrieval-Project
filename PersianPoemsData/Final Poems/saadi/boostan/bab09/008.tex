\begin{center}
\section*{بخش ۸ - حکایت: یکی پارسا سیرت حق پرست}
\label{sec:008}
\addcontentsline{toc}{section}{\nameref{sec:008}}
\begin{longtable}{l p{0.5cm} r}
یکی پارسا سیرت حق پرست
&&
فتادش یکی خشت زرین به دست
\\
سر هوشمندش چنان خیره کرد
&&
که سودا دل روشنش تیره کرد
\\
همه شب در اندیشه کاین گنج و مال
&&
در او تا زیم ره نیابد زوال
\\
دگر قامت عجزم از بهر خواست
&&
نباید بر کس دوتا کرد و راست
\\
سرایی کنم پای بستش رخام
&&
درختان سقفش همه عود خام
\\
یکی حجره خاص از پی دوستان
&&
در حجره اندر سرا بوستان
\\
بفرسودم از رقعه بر رقعه دوخت
&&
تف دیگدان چشم و مغزم بسوخت
\\
دگر زیردستان پزندم خورش
&&
به راحت دهم روح را پرورش
\\
به سختی بکشت این نمد بسترم
&&
روم زین سپس عبقری گسترم
\\
خیالش خرف کرد و کالیوه رنگ
&&
به مغزش فرو برده خرچنگ چنگ
\\
فراغ مناجات و رازش نماند
&&
خور و خواب و ذکر و نمازش نماند
\\
به صحرا برآمد سر از عشوه مست
&&
که جایی نبودش قرار نشست
\\
یکی بر سر گور گل می سرشت
&&
که حاصل کند زآن گل گور خشت
\\
به اندیشه لختی فرو رفت پیر
&&
که ای نفس کوته نظر پند گیر
\\
چه بندی در این خشت زرین دلت
&&
که یک روز خشتی کنند از گلت؟
\\
طمع را نه چندان دهان است باز
&&
که بازش نشیند به یک لقمه آز
\\
بدار ای فرومایه زین خشت دست
&&
که جیحون نشاید به یک خشت بست
\\
تو غافل در اندیشهٔ سود و مال
&&
که سرمایهٔ عمر شد پایمال
\\
غبار هوا چشم عقلت بدوخت
&&
سموم هوس کشت عمرت بسوخت
\\
بکن سرمهٔ غفلت از چشم پاک
&&
که فردا شوی سرمه در چشم خاک
\\
\end{longtable}
\end{center}
