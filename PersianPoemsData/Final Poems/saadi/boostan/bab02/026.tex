\begin{center}
\section*{بخش ۲۶ - حکایت: ز تاج ملکزاده‌ای در مناخ}
\label{sec:026}
\addcontentsline{toc}{section}{\nameref{sec:026}}
\begin{longtable}{l p{0.5cm} r}
ز تاج ملکزاده‌ای در مناخ
&&
شبی لعلی افتاد در سنگلاخ
\\
پدر گفتش اندر شب تیره رنگ
&&
چه دانی که گوهر کدام است و سنگ؟
\\
همه سنگها پاس دار ای پسر
&&
که لعل از میانش نباشد به در
\\
در اوباش، پاکان شوریده رنگ
&&
همان جای تاریک و لعلند و سنگ
\\
چو پاکیزه نفسان و صاحبدلان
&&
بر آمیختستند با جاهلان
\\
به رغبت بکش بار هر جاهلی
&&
که افتی به سر وقت صاحبدلی
\\
کسی را که با دوستی سرخوش است
&&
نبینی که چون بار دشمن کش است؟
\\
بدرد چو گل جامه از دست خار
&&
که خون در دل افتاده خندد چو نار
\\
غم جمله خور در هوای یکی
&&
مراعات صد کن برای یکی
\\
گرت خاکپایان شوریده سر
&&
حقیر و فقیر آید اندر نظر
\\
به مردی کز ایشان به در نیست آن
&&
به خدمت کمر بندشان بر میان
\\
تو هرگز مبینشان به چشم پسند
&&
که ایشان پسندیده حق بسند
\\
کسی را که نزدیک ظنت بد اوست
&&
چه دانی که صاحب ولایت خود اوست؟
\\
در معرفت بر کسانی است باز
&&
که درهاست بر روی ایشان فراز
\\
بسا تلخ عیشان تلخی چشان
&&
که آیند در حله دامن کشان
\\
ببوسی گرت عقل و تدبیر هست
&&
ملکزاده را در نواخانه دست
\\
که روزی برون آید از شهربند
&&
بلندیت بخشد چو گردد بلند
\\
مسوزان درخت گل اندر خریف
&&
که در نوبهارت نماید ظریف
\\
\end{longtable}
\end{center}
