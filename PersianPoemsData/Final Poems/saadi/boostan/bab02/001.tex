\begin{center}
\section*{بخش ۱ - سر آغاز: اگر هوشمندی به معنی گرای}
\label{sec:001}
\addcontentsline{toc}{section}{\nameref{sec:001}}
\begin{longtable}{l p{0.5cm} r}
اگر هوشمندی به معنی گرای
&&
که معنی بماند ز صورت به جای
\\
که را دانش و جود و تقوی نبود
&&
به صورت درش هیچ معنی نبود
\\
کسی خسبد آسوده در زیر گل
&&
که خسبند از او مردم آسوده دل
\\
غم خویش در زندگی خور که خویش
&&
به مرده نپردازد از حرص خویش
\\
زر و نعمت اکنون بده کان تست
&&
که بعد از تو بیرون ز فرمان تست
\\
نخواهی که باشی پراکنده دل
&&
پراکندگان را ز خاطر مهل
\\
پریشان کن امروز گنجینه چست
&&
که فردا کلیدش نه در دست تست
\\
تو با خود ببر توشه خویشتن
&&
که شفقت نیاید ز فرزند و زن
\\
کسی گوی دولت ز دنیا برد
&&
که با خود نصیبی به عقبی برد
\\
به غمخوارگی چون سرانگشت من
&&
نخارد کس اندر جهان پشت من
\\
مکن، بر کف دست نه هر چه هست
&&
که فردا به دندان بری پشت دست
\\
به پوشیدن ستر درویش کوش
&&
که ستر خدایت بود پرده پوش
\\
مگردان غریب از درت بی نصیب
&&
مبادا که گردی به درها غریب
\\
بزرگی رساند به محتاج خیر
&&
که ترسد که محتاج گردد به غیر
\\
به حال دل خستگان در نگر
&&
که روزی تو دلخسته باشی مگر
\\
درون فروماندگان شاد کن
&&
ز روز فروماندگی یاد کن
\\
نه خواهنده‌ای بر در دیگران؟
&&
به شکرانه خواهنده از در مران
\\
\end{longtable}
\end{center}
