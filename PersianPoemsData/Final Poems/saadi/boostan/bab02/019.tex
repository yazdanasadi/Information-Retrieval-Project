\begin{center}
\section*{بخش ۱۹ - حکایت حاتم طائی و صفت جوانمردی او: شنیدم در ایام حاتم که بود}
\label{sec:019}
\addcontentsline{toc}{section}{\nameref{sec:019}}
\begin{longtable}{l p{0.5cm} r}
شنیدم در ایام حاتم که بود
&&
به خیل اندرش بادپایی چو دود
\\
صبا سرعتی، رعد بانگ ادهمی
&&
که بر برق پیشی گرفتی همی
\\
به تک ژاله می‌ریخت بر کوه و دشت
&&
تو گفتی مگر ابر نیسان گذشت
\\
یکی سیل رفتار هامون نورد
&&
که باد از پیش باز ماندی چو گرد
\\
ز اوصاف حاتم به هر مرز و بوم
&&
بگفتند برخی به سلطان روم
\\
که همتای او در کرم مرد نیست
&&
چو اسبش به جولان و ناورد نیست
\\
بیابان نوردی چو کشتی بر آب
&&
که بالای سیرش نپرد عقاب
\\
به دستور دانا چنین گفت شاه
&&
که دعوی خجالت بود بی گواه
\\
من از حاتم آن اسب تازی نژاد
&&
بخواهم، گر او مکرمت کرد و داد
\\
بدانم که در وی شکوه مهی است
&&
وگر رد کند بانگ طبل تهی است
\\
رسولی هنرمند عالم به طی
&&
روان کرد و ده مرد همراه وی
\\
زمین مرده و ابر گریان بر او
&&
صبا کرده بار دگر جان در او
\\
به منزلگه حاتم آمد فرود
&&
بر آسود چون تشنه بر زنده رود
\\
سماطی بیفکند و اسبی بکشت
&&
به دامن شکر دادشان زر به مشت
\\
شب آن جا ببودند و روز دگر
&&
بگفت آنچه دانست صاحب خبر
\\
همی گفت حاتم پریشان چو مست
&&
به دندان ز حسرت همی کند دست
\\
که ای بهره ور موبد نیک نام
&&
چرا پیش از اینم نگفتی پیام؟
\\
من آن باد رفتار دلدل شتاب
&&
ز بهر شما دوش کردم کباب
\\
که دانستم از هول باران و سیل
&&
نشاید شدن در چراگاه خیل
\\
به نوعی دگر روی و راهم نبود
&&
جز او بر در بارگاهم نبود
\\
مروت ندیدم در آیین خویش
&&
که مهمان بخسبد دل از فاقه ریش
\\
مرا نام باید در اقلیم فاش
&&
دگر مرکب نامور گو مباش
\\
کسان را درم داد و تشریف و اسب
&&
طبیعی است اخلاق نیکو نه کسب
\\
خبر شد به روم از جوانمرد طی
&&
هزار آفرین گفت بر طبع وی
\\
ز حاتم بدین نکته راضی مشو
&&
از این خوب تر ماجرایی شنو
\\
\end{longtable}
\end{center}
