\begin{center}
\section*{بخش ۲۷ - حکایت پدر بخیل و پسر لاابالی: یکی زهرهٔ خرج کردن نداشت}
\label{sec:027}
\addcontentsline{toc}{section}{\nameref{sec:027}}
\begin{longtable}{l p{0.5cm} r}
یکی زهرهٔ خرج کردن نداشت
&&
زرش بود و یارای خوردن نداشت
\\
نه خوردی، که خاطر بر آسایدش
&&
نه دادی، که فردا بکار آیدش
\\
شب و روز در بند زر بود و سیم
&&
زر و سیم در بند مرد لئیم
\\
بدانست روزی پسر در کمین
&&
که ممسک کجا کرد زر در زمین
\\
ز خاکش بر آورد و بر باد داد
&&
شنیدم که سنگی در آن جا نهاد
\\
جوانمرد را زر بقایی نکرد
&&
به یک دستش آمد، به دیگر بخورد
\\
کز این کم زنی بود ناپاکرو
&&
کلاهش به بازار و میزر گرو
\\
نهاده پدر چنگ در نای خویش
&&
پسر چنگی و نایی آورده پیش
\\
پدر زار و گریان همه شب نخفت
&&
پسر بامدادان بخندید و گفت
\\
زر از بهر خوردن بود ای پدر
&&
ز بهر نهادن چه سنگ و چه زر
\\
زر از سنگ خارا برون آورند
&&
که با دوستان و عزیزان خورند
\\
زر اندر کف مرد دنیاپرست
&&
هنوز ای برادر به سنگ اندرست
\\
چو در زندگانی بدی با عیال
&&
گرت مرگ خواهند، از ایشان منال
\\
چو خشم آری آن گه خورند از تو سیر
&&
که از بام پنجه گز افتی به زیر
\\
بخیل توانگر به دینار و سیم
&&
طلسمی است بالای گنجی مقیم
\\
از آن سالها می‌بماند زرش
&&
که لرزد طلسمی چنین بر سرش
\\
به سنگ اجل ناگهش بشکنند
&&
به اسودگی گنج قسمت کنند
\\
پس از بردن و گرد کردن چو مور
&&
بخور پیش از آن که‌ت خورد کرم گور
\\
سخنهای سعدی مثال است و پند
&&
به کار آیدت گر شوی کار بند
\\
دریغ است از این روی برتافتن
&&
کز این روی دولت توان یافتن
\\
\end{longtable}
\end{center}
