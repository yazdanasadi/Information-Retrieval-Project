\begin{center}
\section*{بخش ۱۱ - حکایت: یکی در بیابان سگی تشنه یافت}
\label{sec:011}
\addcontentsline{toc}{section}{\nameref{sec:011}}
\begin{longtable}{l p{0.5cm} r}
یکی در بیابان سگی تشنه یافت
&&
برون از رمق در حیاتش نیافت
\\
کله دلو کرد آن پسندیده کیش
&&
چو حبل اندر آن بست دستار خویش
\\
به خدمت میان بست و بازو گشاد
&&
سگ ناتوان را دمی آب داد
\\
خبر داد پیغمبر از حال مرد
&&
که داور گناهان از او عفو کرد
\\
الا گر جفاکاری اندیشه کن
&&
وفا پیش گیر و کرم پیشه کن
\\
کسی با سگی نیکویی گم نکرد
&&
کجا گم شود خیر با نیکمرد؟
\\
کرم کن چنان که‌ت برآید ز دست
&&
جهانبان در خیر بر کس نبست
\\
به قنطار زر بخش کردن ز گنج
&&
نباشد چو قیراطی از دسترنج
\\
برد هر کسی بار در خورد زور
&&
گران است پای ملخ پیش مور
\\
\end{longtable}
\end{center}
