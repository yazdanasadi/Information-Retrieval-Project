\begin{center}
\section*{بخش ۱۳ - حکایت در معنی رحمت بر ضعیفان و اندیشه در عاقبت: بنالید درویشی از ضعف حال}
\label{sec:013}
\addcontentsline{toc}{section}{\nameref{sec:013}}
\begin{longtable}{l p{0.5cm} r}
بنالید درویشی از ضعف حال
&&
بر تندرویی خداوند مال
\\
نه دینار دادش سیه دل نه دانگ
&&
بر او زد به سر باری از طیره بانگ
\\
دل سائل از جور او خون گرفت
&&
سر از غم بر آورد و گفت ای شگفت
\\
توانگر ترش روی، باری، چراست؟
&&
مگر می‌نترسد ز تلخی خواست؟
\\
بفرمود کوته نظر تا غلام
&&
براندش به خواری و زجر تمام
\\
به ناکردن شکر پروردگار
&&
شنیدم که برگشت از او روزگار
\\
بزرگیش سر در تباهی نهاد
&&
عطارد قلم در سیاهی نهاد
\\
شقاوت برهنه نشاندش چو سیر
&&
نه بارش رها کرد و نه بارگیر
\\
فشاندش قضا بر سر از فاقه خاک
&&
مشعبد صفت، کیسه و دست پاک
\\
سراپای حالش دگرگونه گشت
&&
بر این ماجرا مدتی بر گذشت
\\
غلامش به دست کریمی فتاد
&&
توانگر دل و دست و روشن نهاد
\\
به دیدار مسکین آشفته حال
&&
چنان شاد بودی که مسکین به مال
\\
شبانگه یکی بر درش لقمه جست
&&
ز سختی کشیدن قدمهاش سست
\\
بفرمود صاحب نظر بنده را
&&
که خشنود کن مرد درمنده را
\\
چو نزدیک بردش ز خوان بهره‌ای
&&
برآورد بی خویشتن نعره‌ای
\\
شکسته دل آمد بر خواجه باز
&&
عیان کرده اشکش به دیباجه راز
\\
بپرسید سالار فرخنده خوی
&&
که اشکت ز جور که آمد به روی؟
\\
بگفت اندرونم بشورید سخت
&&
بر احوال این پیر شوریده بخت
\\
که مملوک وی بودم اندر قدیم
&&
خداوند املاک و اسباب و سیم
\\
چو کوتاه شد دستش از عز و ناز
&&
کند دست خواهش به درها دراز
\\
بخندید و گفت ای پسر جور نیست
&&
ستم بر کس از گردش دور نیست
\\
نه آن تندروی است بازارگان
&&
که بردی سر از کبر بر آسمان؟
\\
من آنم که آن روزم از در براند
&&
به روز منش دور گیتی نشاند
\\
نگه کرد باز آسمان سوی من
&&
فرو شست گرد غم از روی من
\\
خدای ار به حکمت ببندد دری
&&
گشاید به فضل و کرم دیگری
\\
بسا مفلس بینوا سیر شد
&&
بسا کار منعم زبر زیر شد
\\
\end{longtable}
\end{center}
