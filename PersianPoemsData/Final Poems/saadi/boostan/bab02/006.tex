\begin{center}
\section*{بخش ۶ - حکایت ممسک و فرزند ناخلف: یکی رفت و دینار از او صد هزار}
\label{sec:006}
\addcontentsline{toc}{section}{\nameref{sec:006}}
\begin{longtable}{l p{0.5cm} r}
یکی رفت و دینار از او صد هزار
&&
خلف برد صاحبدلی هوشیار
\\
نه چون ممسکان دست بر زر گرفت
&&
چو آزادگان دست از او بر گرفت
\\
ز درویش خالی نبودی درش
&&
مسافر به مهمانسرای اندرش
\\
دل خویش و بیگانه خرسند کرد
&&
نه همچون پدر سیم و زر بند کرد
\\
ملامت کنی گفتش ای باد دست
&&
به یک ره پریشان مکن هر چه هست
\\
به سالی توان خرمن اندوختن
&&
به یک دم نه مردی بود سوختن
\\
چو در تنگدستی نداری شکیب
&&
نگه دار وقت فراخی حسیب
\\
به دختر چه خوش گفت بانوی ده
&&
که روز نوا برگ سختی بنه
\\
همه وقت بر دار مشک و سبوی
&&
که پیوسته در ده روان نیست جوی
\\
به دنیا توان آخرت یافتن
&&
به زر پنجه شیر بر تافتن
\\
به یک بار بر دوستان زر مپاش
&&
وز آسیب دشمن به اندیشه باش
\\
اگر تنگدستی مرو پیش یار
&&
وگر سیم داری بیا و بیار
\\
اگر روی بر خاک پایش نهی
&&
جوابت نگوید به دست تهی
\\
خداوند زر بر کند چشم دیو
&&
به دام آورد صخر جنی به ریو
\\
تهی دست در خوبرویان مپیچ
&&
که بی سیم مردم نیرزند هیچ
\\
به دست تهی بر نیاد امید
&&
به زر برکنی چشم دیو سپید
\\
وگر هرچه یابی به کف بر نهی
&&
کفت وقت حاجت بماند تهی
\\
گدایان به سعی تو هرگز قوی
&&
نگردند، ترسم تو لاغر شوی
\\
چو مناع خیر این حکایت بگفت
&&
ز غیرت جوانمرد را رگ نخفت
\\
پراکنده دل گشت از آن عیب جوی
&&
بر آشفت و گفت ای پراکنده گوی
\\
مرا دستگاهی که پیرامن است
&&
پدر گفت میراث جد من است
\\
نه ایشان به خست نگه داشتند
&&
به حسرت بمردند و بگذاشتند؟
\\
به دستم نیفتاد مال پدر
&&
که بعد از من افتد به دست پسر
\\
همان به که امروز مردم خورند
&&
که فردا پس از من به یغما برند
\\
خور و پوش و بخشای و راحت رسان
&&
نگه می چه داری ز بهر کسان؟
\\
برند از جهان با خود اصحاب رای
&&
فرو مایه ماند به حسرت بجای
\\
زر و نعمت اکنون بده کان توست
&&
که بعد از تو بیرون ز فرمان توست
\\
به دنیا توانی که عقبی خری
&&
بخر، جان من، ور نه حسرت بری
\\
\end{longtable}
\end{center}
