\begin{center}
\section*{بخش ۱۷ - حکایت درویش با روباه: یکی روبهی دید بی دست و پای}
\label{sec:017}
\addcontentsline{toc}{section}{\nameref{sec:017}}
\begin{longtable}{l p{0.5cm} r}
یکی روبهی دید بی دست و پای
&&
فرو ماند در لطف و صنع خدای
\\
که چون زندگانی به سر می‌برد؟
&&
بدین دست و پای از کجا می‌خورد؟
\\
در این بود درویش شوریده رنگ
&&
که شیری در آمد شغالی به چنگ
\\
شغال نگون بخت را شیر خورد
&&
بماند آنچه روباه از آن سیر خورد
\\
دگر روز باز اتفاق اوفتاد
&&
که روزی رسان قوت روزش بداد
\\
یقین، مرد را دیده بیننده کرد
&&
شد و تکیه بر آفریننده کرد
\\
کز این پس به کنجی نشینم چو مور
&&
که روزی نخوردند پیلان به زور
\\
زنخدان فرو برد چندی به جیب
&&
که بخشنده روزی فرستد ز غیب
\\
نه بیگانه تیمار خوردش نه دوست
&&
چو چنگش رگ و استخوان ماند و پوست
\\
چو صبرش نماند از ضعیفی و هوش
&&
ز دیوار محرابش آمد به گوش
\\
برو شیر درنده باش، ای دغل
&&
مینداز خود را چو روباه شل
\\
چنان سعی کن کز تو ماند چو شیر
&&
چه باشی چو روبه به وامانده سیر؟
\\
چو شیر آن که را گردنی فربه است
&&
گر افتد چو روبه، سگ از وی به است
\\
به چنگ آر و با دیگران نوش کن
&&
نه بر فضلهٔ دیگران گوش کن
\\
بخور تا توانی به بازوی خویش
&&
که سعیت بود در ترازوی خویش
\\
چو مردان ببر رنج و راحت رسان
&&
مخنث خورد دسترنج کسان
\\
بگیر ای جوان دست درویش پیر
&&
نه خود را بیفکن که دستم بگیر
\\
خدا را بر آن بنده بخشایش است
&&
که خلق از وجودش در آسایش است
\\
کرم ورزد آن سر که مغزی در اوست
&&
که دون همتانند بی مغز و پوست
\\
کسی نیک بیند به هر دو سرای
&&
که نیکی رساند به خلق خدای
\\
\end{longtable}
\end{center}
