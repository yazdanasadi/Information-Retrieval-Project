\begin{center}
\section*{غزل شماره ۱۱۹: دلی که غیب نمای است و جام جم دارد}
\label{sec:sh119}
\addcontentsline{toc}{section}{\nameref{sec:sh119}}
\begin{longtable}{l p{0.5cm} r}
دلی که غیب نمای است و جام جم دارد
&&
ز خاتمی که دمی گم شود چه غم دارد
\\
به خط و خال گدایان مده خزینه دل
&&
به دست شاهوشی ده که محترم دارد
\\
نه هر درخت تحمل کند جفای خزان
&&
غلام همت سروم که این قدم دارد
\\
رسید موسم آن کز طرب چو نرگس مست
&&
نهد به پای قدح هر که شش درم دارد
\\
زر از بهای می اکنون چو گل دریغ مدار
&&
که عقل کل به صدت عیب متهم دارد
\\
ز سر غیب کس آگاه نیست قصه مخوان
&&
کدام محرم دل ره در این حرم دارد
\\
دلم که لاف تجرد زدی کنون صد شغل
&&
به بوی زلف تو با باد صبحدم دارد
\\
مراد دل ز که پرسم که نیست دلداری
&&
که جلوه نظر و شیوه کرم دارد
\\
ز جیب خرقه حافظ چه طرف بتوان بست
&&
که ما صمد طلبیدیم و او صنم دارد
\\
\end{longtable}
\end{center}
