\begin{center}
\section*{غزل شماره ۱۲۰: بتی دارم که گرد گل ز سنبل سایه بان دارد}
\label{sec:sh120}
\addcontentsline{toc}{section}{\nameref{sec:sh120}}
\begin{longtable}{l p{0.5cm} r}
بتی دارم که گرد گل ز سنبل سایه بان دارد
&&
بهار عارضش خطی به خون ارغوان دارد
\\
غبار خط بپوشانید خورشید رخش یا رب
&&
بقای جاودانش ده که حسن جاودان دارد
\\
چو عاشق می‌شدم گفتم که بردم گوهر مقصود
&&
ندانستم که این دریا چه موج خون فشان دارد
\\
ز چشمت جان نشاید برد کز هر سو که می‌بینم
&&
کمین از گوشه‌ای کرده‌ست و تیر اندر کمان دارد
\\
چو دام طره افشاند ز گرد خاطر عشاق
&&
به غماز صبا گوید که راز ما نهان دارد
\\
بیفشان جرعه‌ای بر خاک و حال اهل دل بشنو
&&
که از جمشید و کیخسرو فراوان داستان دارد
\\
چو در رویت بخندد گل مشو در دامش ای بلبل
&&
که بر گل اعتمادی نیست گر حسن جهان دارد
\\
خدا را داد من بستان از او ای شحنه مجلس
&&
که می با دیگری خورده‌ست و با من سر گران دارد
\\
به فتراک ار همی‌بندی خدا را زود صیدم کن
&&
که آفت‌هاست در تاخیر و طالب را زیان دارد
\\
ز سروقد دلجویت مکن محروم چشمم را
&&
بدین سرچشمه‌اش بنشان که خوش آبی روان دارد
\\
ز خوف هجرم ایمن کن اگر امید آن داری
&&
که از چشم بداندیشان خدایت در امان دارد
\\
چه عذر بخت خود گویم که آن عیار شهرآشوب
&&
به تلخی کشت حافظ را و شکر در دهان دارد
\\
\end{longtable}
\end{center}
