\begin{center}
\section*{غزل شماره ۳۲۲: خیال نقش تو در کارگاه دیده کشیدم}
\label{sec:sh322}
\addcontentsline{toc}{section}{\nameref{sec:sh322}}
\begin{longtable}{l p{0.5cm} r}
خیال نقش تو در کارگاه دیده کشیدم
&&
به صورت تو نگاری ندیدم و نشنیدم
\\
اگر چه در طلبت همعنان باد شمالم
&&
به گرد سرو خرامان قامتت نرسیدم
\\
امید در شب زلفت به روز عمر نبستم
&&
طمع به دور دهانت ز کام دل ببریدم
\\
به شوق چشمه نوشت چه قطره‌ها که فشاندم
&&
ز لعل باده فروشت چه عشوه‌ها که خریدم
\\
ز غمزه بر دل ریشم چه تیرها که گشادی
&&
ز غصه بر سر کویت چه بارها که کشیدم
\\
ز کوی یار بیار ای نسیم صبح غباری
&&
که بوی خون دل ریش از آن تراب شنیدم
\\
گناه چشم سیاه تو بود و گردن دلخواه
&&
که من چو آهوی وحشی ز آدمی برمیدم
\\
چو غنچه بر سرم از کوی او گذشت نسیمی
&&
که پرده بر دل خونین به بوی او بدریدم
\\
به خاک پای تو سوگند و نور دیده حافظ
&&
که بی رخ تو فروغ از چراغ دیده ندیدم
\\
\end{longtable}
\end{center}
