\begin{center}
\section*{غزل شماره ۳۱۸: مرا می‌بینی و هر دم زیادت می‌کنی دردم}
\label{sec:sh318}
\addcontentsline{toc}{section}{\nameref{sec:sh318}}
\begin{longtable}{l p{0.5cm} r}
مرا می‌بینی و هر دم زیادت می‌کنی دردم
&&
تو را می‌بینم و میلم زیادت می‌شود هر دم
\\
به سامانم نمی‌پرسی نمی‌دانم چه سر داری
&&
به درمانم نمی‌کوشی نمی‌دانی مگر دردم
\\
نه راه است این که بگذاری مرا بر خاک و بگریزی
&&
گذاری آر و بازم پرس تا خاک رهت گردم
\\
ندارم دستت از دامن به جز در خاک و آن دم هم
&&
که بر خاکم روان گردی بگیرد دامنت گردم
\\
فرو رفت از غم عشقت دمم دم می‌دهی تا کی
&&
دمار از من برآوردی نمی‌گویی برآوردم
\\
شبی دل را به تاریکی ز زلفت باز می‌جستم
&&
رخت می‌دیدم و جامی هلالی باز می‌خوردم
\\
کشیدم در برت ناگاه و شد در تاب گیسویت
&&
نهادم بر لبت لب را و جان و دل فدا کردم
\\
تو خوش می‌باش با حافظ برو گو خصم جان می‌ده
&&
چو گرمی از تو می‌بینم چه باک از خصم دم سردم
\\
\end{longtable}
\end{center}
