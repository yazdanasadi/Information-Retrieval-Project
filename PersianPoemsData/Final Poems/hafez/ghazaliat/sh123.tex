\begin{center}
\section*{غزل شماره ۱۲۳: مطرب عشق عجب ساز و نوایی دارد}
\label{sec:sh123}
\addcontentsline{toc}{section}{\nameref{sec:sh123}}
\begin{longtable}{l p{0.5cm} r}
مطرب عشق عجب ساز و نوایی دارد
&&
نقش هر نغمه که زد راه به جایی دارد
\\
عالم از ناله عشاق مبادا خالی
&&
که خوش آهنگ و فرح بخش هوایی دارد
\\
پیر دردی کش ما گر چه ندارد زر و زور
&&
خوش عطابخش و خطاپوش خدایی دارد
\\
محترم دار دلم کاین مگس قندپرست
&&
تا هواخواه تو شد فر همایی دارد
\\
از عدالت نبود دور گرش پرسد حال
&&
پادشاهی که به همسایه گدایی دارد
\\
اشک خونین بنمودم به طبیبان گفتند
&&
درد عشق است و جگرسوز دوایی دارد
\\
ستم از غمزه میاموز که در مذهب عشق
&&
هر عمل اجری و هر کرده جزایی دارد
\\
نغز گفت آن بت ترسابچه باده پرست
&&
شادی روی کسی خور که صفایی دارد
\\
خسروا حافظ درگاه نشین فاتحه خواند
&&
و از زبان تو تمنای دعایی دارد
\\
\end{longtable}
\end{center}
