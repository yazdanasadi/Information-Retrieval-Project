\begin{center}
\section*{غزل شماره ۳۴۸: دیده دریا کنم و صبر به صحرا فکنم}
\label{sec:sh348}
\addcontentsline{toc}{section}{\nameref{sec:sh348}}
\begin{longtable}{l p{0.5cm} r}
دیده دریا کنم و صبر به صحرا فکنم
&&
و اندر این کار دل خویش به دریا فکنم
\\
از دل تنگ گنهکار برآرم آهی
&&
کآتش اندر گنه آدم و حوا فکنم
\\
مایه خوشدلی آن جاست که دلدار آن جاست
&&
می‌کنم جهد که خود را مگر آن جا فکنم
\\
بگشا بند قبا ای مه خورشیدکلاه
&&
تا چو زلفت سر سودازده در پا فکنم
\\
خورده‌ام تیر فلک باده بده تا سرمست
&&
عقده دربند کمر ترکش جوزا فکنم
\\
جرعه جام بر این تخت روان افشانم
&&
غلغل چنگ در این گنبد مینا فکنم
\\
حافظا تکیه بر ایام چو سهو است و خطا
&&
من چرا عشرت امروز به فردا فکنم
\\
\end{longtable}
\end{center}
