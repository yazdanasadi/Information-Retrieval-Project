\begin{center}
\section*{غزل شماره ۲۴۰: ابر آذاری برآمد باد نوروزی وزید}
\label{sec:sh240}
\addcontentsline{toc}{section}{\nameref{sec:sh240}}
\begin{longtable}{l p{0.5cm} r}
ابر آذاری برآمد باد نوروزی وزید
&&
وجه می می‌خواهم و مطرب که می‌گوید رسید
\\
شاهدان در جلوه و من شرمسار کیسه‌ام
&&
بار عشق و مفلسی صعب است می‌باید کشید
\\
قحط جود است آبروی خود نمی‌باید فروخت
&&
باده و گل از بهای خرقه می‌باید خرید
\\
گوییا خواهد گشود از دولتم کاری که دوش
&&
من همی‌کردم دعا و صبح صادق می‌دمید
\\
با لبی و صد هزاران خنده آمد گل به باغ
&&
از کریمی گوییا در گوشه‌ای بویی شنید
\\
دامنی گر چاک شد در عالم رندی چه باک
&&
جامه‌ای در نیک نامی نیز می‌باید درید
\\
این لطایف کز لب لعل تو من گفتم که گفت
&&
وین تطاول کز سر زلف تو من دیدم که دید
\\
عدل سلطان گر نپرسد حال مظلومان عشق
&&
گوشه گیران را ز آسایش طمع باید برید
\\
تیر عاشق کش ندانم بر دل حافظ که زد
&&
این قدر دانم که از شعر ترش خون می‌چکید
\\
\end{longtable}
\end{center}
