\begin{center}
\section*{غزل شماره ۲۴۳: بوی خوش تو هر که ز باد صبا شنید}
\label{sec:sh243}
\addcontentsline{toc}{section}{\nameref{sec:sh243}}
\begin{longtable}{l p{0.5cm} r}
بوی خوش تو هر که ز باد صبا شنید
&&
از یار آشنا سخن آشنا شنید
\\
ای شاه حسن چشم به حال گدا فکن
&&
کاین گوش بس حکایت شاه و گدا شنید
\\
خوش می‌کنم به باده مشکین مشام جان
&&
کز دلق پوش صومعه بوی ریا شنید
\\
سر خدا که عارف سالک به کس نگفت
&&
در حیرتم که باده فروش از کجا شنید
\\
یا رب کجاست محرم رازی که یک زمان
&&
دل شرح آن دهد که چه گفت و چه‌ها شنید
\\
اینش سزا نبود دل حق گزار من
&&
کز غمگسار خود سخن ناسزا شنید
\\
محروم اگر شدم ز سر کوی او چه شد
&&
از گلشن زمانه که بوی وفا شنید
\\
ساقی بیا که عشق ندا می‌کند بلند
&&
کان کس که گفت قصه ما هم ز ما شنید
\\
ما باده زیر خرقه نه امروز می‌خوریم
&&
صد بار پیر میکده این ماجرا شنید
\\
ما می به بانگ چنگ نه امروز می‌کشیم
&&
بس دور شد که گنبد چرخ این صدا شنید
\\
پند حکیم محض صواب است و عین خیر
&&
فرخنده آن کسی که به سمع رضا شنید
\\
حافظ وظیفه تو دعا گفتن است و بس
&&
دربند آن مباش که نشنید یا شنید
\\
\end{longtable}
\end{center}
