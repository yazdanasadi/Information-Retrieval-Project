\begin{center}
\section*{غزل شماره ۴۵۶: نوبهار است در آن کوش که خوشدل باشی}
\label{sec:sh456}
\addcontentsline{toc}{section}{\nameref{sec:sh456}}
\begin{longtable}{l p{0.5cm} r}
نوبهار است در آن کوش که خوشدل باشی
&&
که بسی گل بدمد باز و تو در گل باشی
\\
من نگویم که کنون با که نشین و چه بنوش
&&
که تو خود دانی اگر زیرک و عاقل باشی
\\
چنگ در پرده همین می‌دهدت پند ولی
&&
وعظت آن گاه کند سود که قابل باشی
\\
در چمن هر ورقی دفتر حالی دگر است
&&
حیف باشد که ز کار همه غافل باشی
\\
نقد عمرت ببرد غصه دنیا به گزاف
&&
گر شب و روز در این قصه مشکل باشی
\\
گر چه راهیست پر از بیم ز ما تا بر دوست
&&
رفتن آسان بود ار واقف منزل باشی
\\
حافظا گر مدد از بخت بلندت باشد
&&
صید آن شاهد مطبوع شمایل باشی
\\
\end{longtable}
\end{center}
