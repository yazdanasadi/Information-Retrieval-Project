\begin{center}
\section*{غزل شماره ۷۳: روشن از پرتو رویت نظری نیست که نیست}
\label{sec:sh073}
\addcontentsline{toc}{section}{\nameref{sec:sh073}}
\begin{longtable}{l p{0.5cm} r}
روشن از پرتو رویت نظری نیست که نیست
&&
منت خاک درت بر بصری نیست که نیست
\\
ناظر روی تو صاحب نظرانند آری
&&
سر گیسوی تو در هیچ سری نیست که نیست
\\
اشک غماز من ار سرخ برآمد چه عجب
&&
خجل از کرده خود پرده دری نیست که نیست
\\
تا به دامن ننشیند ز نسیمش گردی
&&
سیل خیز از نظرم رهگذری نیست که نیست
\\
تا دم از شام سر زلف تو هر جا نزنند
&&
با صبا گفت و شنیدم سحری نیست که نیست
\\
من از این طالع شوریده برنجم ور نی
&&
بهره مند از سر کویت دگری نیست که نیست
\\
از حیای لب شیرین تو ای چشمه نوش
&&
غرق آب و عرق اکنون شکری نیست که نیست
\\
مصلحت نیست که از پرده برون افتد راز
&&
ور نه در مجلس رندان خبری نیست که نیست
\\
شیر در بادیه عشق تو روباه شود
&&
آه از این راه که در وی خطری نیست که نیست
\\
آب چشمم که بر او منت خاک در توست
&&
زیر صد منت او خاک دری نیست که نیست
\\
از وجودم قدری نام و نشان هست که هست
&&
ور نه از ضعف در آن جا اثری نیست که نیست
\\
غیر از این نکته که حافظ ز تو ناخشنود است
&&
در سراپای وجودت هنری نیست که نیست
\\
\end{longtable}
\end{center}
