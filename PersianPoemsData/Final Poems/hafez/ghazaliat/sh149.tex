\begin{center}
\section*{غزل شماره ۱۴۹: دلم جز مهر مه رویان طریقی بر نمی‌گیرد}
\label{sec:sh149}
\addcontentsline{toc}{section}{\nameref{sec:sh149}}
\begin{longtable}{l p{0.5cm} r}
دلم جز مهر مه رویان طریقی بر نمی‌گیرد
&&
ز هر در می‌دهم پندش ولیکن در نمی‌گیرد
\\
خدا را ای نصیحتگو حدیث ساغر و می گو
&&
که نقشی در خیال ما از این خوشتر نمی‌گیرد
\\
بیا ای ساقی گلرخ بیاور باده رنگین
&&
که فکری در درون ما از این بهتر نمی‌گیرد
\\
صراحی می‌کشم پنهان و مردم دفتر انگارند
&&
عجب گر آتش این زرق در دفتر نمی‌گیرد
\\
من این دلق مرقع را بخواهم سوختن روزی
&&
که پیر می فروشانش به جامی بر نمی‌گیرد
\\
از آن رو هست یاران را صفاها با می لعلش
&&
که غیر از راستی نقشی در آن جوهر نمی‌گیرد
\\
سر و چشمی چنین دلکش تو گویی چشم از او بردوز
&&
برو کاین وعظ بی‌معنی مرا در سر نمی‌گیرد
\\
نصیحتگوی رندان را که با حکم قضا جنگ است
&&
دلش بس تنگ می‌بینم مگر ساغر نمی‌گیرد
\\
میان گریه می‌خندم که چون شمع اندر این مجلس
&&
زبان آتشینم هست لیکن در نمی‌گیرد
\\
چه خوش صید دلم کردی بنازم چشم مستت را
&&
که کس مرغان وحشی را از این خوشتر نمی‌گیرد
\\
سخن در احتیاج ما و استغنای معشوق است
&&
چه سود افسونگری ای دل که در دلبر نمی‌گیرد
\\
من آن آیینه را روزی به دست آرم سکندروار
&&
اگر می‌گیرد این آتش زمانی ور نمی‌گیرد
\\
خدا را رحمی ای منعم که درویش سر کویت
&&
دری دیگر نمی‌داند رهی دیگر نمی‌گیرد
\\
بدین شعر تر شیرین ز شاهنشه عجب دارم
&&
که سر تا پای حافظ را چرا در زر نمی‌گیرد
\\
\end{longtable}
\end{center}
