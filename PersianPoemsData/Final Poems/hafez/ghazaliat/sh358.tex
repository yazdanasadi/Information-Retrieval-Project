\begin{center}
\section*{غزل شماره ۳۵۸: غم زمانه که هیچش کران نمی‌بینم}
\label{sec:sh358}
\addcontentsline{toc}{section}{\nameref{sec:sh358}}
\begin{longtable}{l p{0.5cm} r}
غم زمانه که هیچش کران نمی‌بینم
&&
دواش جز می چون ارغوان نمی‌بینم
\\
به ترک خدمت پیر مغان نخواهم گفت
&&
چرا که مصلحت خود در آن نمی‌بینم
\\
ز آفتاب قدح ارتفاع عیش بگیر
&&
چرا که طالع وقت آن چنان نمی‌بینم
\\
نشان اهل خدا عاشقیست با خود دار
&&
که در مشایخ شهر این نشان نمی‌بینم
\\
بدین دو دیده حیران من هزار افسوس
&&
که با دو آینه رویش عیان نمی‌بینم
\\
قد تو تا بشد از جویبار دیده من
&&
به جای سرو جز آب روان نمی‌بینم
\\
در این خمار کسم جرعه‌ای نمی‌بخشد
&&
ببین که اهل دلی در میان نمی‌بینم
\\
نشان موی میانش که دل در او بستم
&&
ز من مپرس که خود در میان نمی‌بینم
\\
من و سفینه حافظ که جز در این دریا
&&
بضاعت سخن درفشان نمی‌بینم
\\
\end{longtable}
\end{center}
