\begin{center}
\section*{غزل شماره ۴۴۶: صبا تو نکهت آن زلف مشک بو داری}
\label{sec:sh446}
\addcontentsline{toc}{section}{\nameref{sec:sh446}}
\begin{longtable}{l p{0.5cm} r}
صبا تو نکهت آن زلف مشک بو داری
&&
به یادگار بمانی که بوی او داری
\\
دلم که گوهر اسرار حسن و عشق در اوست
&&
توان به دست تو دادن گرش نکو داری
\\
در آن شمایل مطبوع هیچ نتوان گفت
&&
جز این قدر که رقیبان تندخو داری
\\
نوای بلبلت ای گل کجا پسند افتد
&&
که گوش و هوش به مرغان هرزه گو داری
\\
به جرعه تو سرم مست گشت نوشت باد
&&
خود از کدام خم است این که در سبو داری
\\
به سرکشی خود ای سرو جویبار مناز
&&
که گر بدو رسی از شرم سر فروداری
\\
دم از ممالک خوبی چو آفتاب زدن
&&
تو را رسد که غلامان ماه رو داری
\\
قبای حسن فروشی تو را برازد و بس
&&
که همچو گل همه آیین رنگ و بو داری
\\
ز کنج صومعه حافظ مجوی گوهر عشق
&&
قدم برون نه اگر میل جست و جو داری
\\
\end{longtable}
\end{center}
