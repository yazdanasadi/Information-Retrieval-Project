\begin{center}
\section*{غزل شماره ۲۵: شکفته شد گل حمرا و گشت بلبل مست}
\label{sec:sh025}
\addcontentsline{toc}{section}{\nameref{sec:sh025}}
\begin{longtable}{l p{0.5cm} r}
شکفته شد گل حمرا و گشت بلبل مست
&&
صلای سرخوشی ای صوفیان باده پرست
\\
اساس توبه که در محکمی چو سنگ نمود
&&
ببین که جام زجاجی چه طرفه‌اش بشکست
\\
بیار باده که در بارگاه استغنا
&&
چه پاسبان و چه سلطان چه هوشیار و چه مست
\\
از این رباط دودر چون ضرورت است رحیل
&&
رواق و طاق معیشت چه سربلند و چه پست
\\
مقام عیش میسر نمی‌شود بی‌رنج
&&
بلی به حکم بلا بسته‌اند عهد الست
\\
به هست و نیست مرنجان ضمیر و خوش می‌باش
&&
که نیستیست سرانجام هر کمال که هست
\\
شکوه آصفی و اسب باد و منطق طیر
&&
به باد رفت و از او خواجه هیچ طرف نبست
\\
به بال و پر مرو از ره که تیر پرتابی
&&
هوا گرفت زمانی ولی به خاک نشست
\\
زبان کلک تو حافظ چه شکر آن گوید
&&
که گفته سخنت می‌برند دست به دست
\\
\end{longtable}
\end{center}
