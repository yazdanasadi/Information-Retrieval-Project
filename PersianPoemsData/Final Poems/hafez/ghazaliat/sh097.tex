\begin{center}
\section*{غزل شماره ۹۷: تویی که بر سر خوبان کشوری چون تاج}
\label{sec:sh097}
\addcontentsline{toc}{section}{\nameref{sec:sh097}}
\begin{longtable}{l p{0.5cm} r}
تویی که بر سر خوبان کشوری چون تاج
&&
سزد اگر همه دلبران دهندت باج
\\
دو چشم شوخ تو برهم زده خطا و حبش
&&
به چین زلف تو ماچین و هند داده خراج
\\
بیاض روی تو روشن چو عارض رخ روز
&&
سواد زلف سیاه تو هست ظلمت داج
\\
دهان شهد تو داده رواج آب خضر
&&
لب چو قند تو برد از نبات مصر رواج
\\
از این مرض به حقیقت شفا نخواهم یافت
&&
که از تو درد دل ای جان نمی‌رسد به علاج
\\
چرا همی‌شکنی جان من ز سنگ دلی
&&
دل ضعیف که باشد به نازکی چو زجاج
\\
لب تو خضر و دهان تو آب حیوان است
&&
قد تو سرو و میان موی و بر به هیئت عاج
\\
فتاد در دل حافظ هوای چون تو شهی
&&
کمینه ذره خاک در تو بودی کاج
\\
\end{longtable}
\end{center}
