\begin{center}
\section*{غزل شماره ۲۵۸: هزار شکر که دیدم به کام خویشت باز}
\label{sec:sh258}
\addcontentsline{toc}{section}{\nameref{sec:sh258}}
\begin{longtable}{l p{0.5cm} r}
هزار شکر که دیدم به کام خویشت باز
&&
ز روی صدق و صفا گشته با دلم دمساز
\\
روندگان طریقت ره بلا سپرند
&&
رفیق عشق چه غم دارد از نشیب و فراز
\\
غم حبیب نهان به ز گفت و گوی رقیب
&&
که نیست سینه ارباب کینه محرم راز
\\
اگر چه حسن تو از عشق غیر مستغنیست
&&
من آن نیم که از این عشقبازی آیم باز
\\
چه گویمت که ز سوز درون چه می‌بینم
&&
ز اشک پرس حکایت که من نیم غماز
\\
چه فتنه بود که مشاطه قضا انگیخت
&&
که کرد نرگس مستش سیه به سرمه ناز
\\
بدین سپاس که مجلس منور است به دوست
&&
گرت چو شمع جفایی رسد بسوز و بساز
\\
غرض کرشمه حسن است ور نه حاجت نیست
&&
جمال دولت محمود را به زلف ایاز
\\
غزل سرایی ناهید صرفه‌ای نبرد
&&
در آن مقام که حافظ برآورد آواز
\\
\end{longtable}
\end{center}
