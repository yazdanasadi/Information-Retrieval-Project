\begin{center}
\section*{غزل شماره ۴۷۲: احمد الله علی معدله السلطان}
\label{sec:sh472}
\addcontentsline{toc}{section}{\nameref{sec:sh472}}
\begin{longtable}{l p{0.5cm} r}
احمد الله علی معدلة السلطان
&&
احمد شیخ اویس حسن ایلخانی
\\
خان بن خان و شهنشاه شهنشاه نژاد
&&
آن که می‌زیبد اگر جان جهانش خوانی
\\
دیده نادیده به اقبال تو ایمان آورد
&&
مرحبا ای به چنین لطف خدا ارزانی
\\
ماه اگر بی تو برآید به دو نیمش بزنند
&&
دولت احمدی و معجزه سبحانی
\\
جلوه بخت تو دل می‌برد از شاه و گدا
&&
چشم بد دور که هم جانی و هم جانانی
\\
برشکن کاکل ترکانه که در طالع توست
&&
بخشش و کوشش خاقانی و چنگزخانی
\\
گر چه دوریم به یاد تو قدح می‌گیریم
&&
بعد منزل نبود در سفر روحانی
\\
از گل پارسیم غنچه عیشی نشکفت
&&
حبذا دجله بغداد و می ریحانی
\\
سر عاشق که نه خاک در معشوق بود
&&
کی خلاصش بود از محنت سرگردانی
\\
ای نسیم سحری خاک در یار بیار
&&
که کند حافظ از او دیده دل نورانی
\\
\end{longtable}
\end{center}
