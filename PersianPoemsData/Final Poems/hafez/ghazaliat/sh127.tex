\begin{center}
\section*{غزل شماره ۱۲۷: روشنی طلعت تو ماه ندارد}
\label{sec:sh127}
\addcontentsline{toc}{section}{\nameref{sec:sh127}}
\begin{longtable}{l p{0.5cm} r}
روشنی طلعت تو ماه ندارد
&&
پیش تو گل رونق گیاه ندارد
\\
گوشه ابروی توست منزل جانم
&&
خوشتر از این گوشه پادشاه ندارد
\\
تا چه کند با رخ تو دود دل من
&&
آینه دانی که تاب آه ندارد
\\
شوخی نرگس نگر که پیش تو بشکفت
&&
چشم دریده ادب نگاه ندارد
\\
دیدم و آن چشم دل سیه که تو داری
&&
جانب هیچ آشنا نگاه ندارد
\\
رطل گرانم ده ای مرید خرابات
&&
شادی شیخی که خانقاه ندارد
\\
خون خور و خامش نشین که آن دل نازک
&&
طاقت فریاد دادخواه ندارد
\\
گو برو و آستین به خون جگر شوی
&&
هر که در این آستانه راه ندارد
\\
نی من تنها کشم تطاول زلفت
&&
کیست که او داغ آن سیاه ندارد
\\
حافظ اگر سجده تو کرد مکن عیب
&&
کافر عشق ای صنم گناه ندارد
\\
\end{longtable}
\end{center}
