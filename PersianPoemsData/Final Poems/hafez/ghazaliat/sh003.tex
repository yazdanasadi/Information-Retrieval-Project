\begin{center}
\section*{غزل شماره ۳: اگر آن ترک شیرازی به دست آرد دل ما را}
\label{sec:sh003}
\addcontentsline{toc}{section}{\nameref{sec:sh003}}
\begin{longtable}{l p{0.5cm} r}
اگر آن ترک شیرازی به دست آرد دل ما را
&&
به خال هندویش بخشم سمرقند و بخارا را
\\
بده ساقی می باقی که در جنت نخواهی یافت
&&
کنار آب رکن آباد و گلگشت مصلا را
\\
فغان کاین لولیان شوخ شیرین کار شهرآشوب
&&
چنان بردند صبر از دل که ترکان خوان یغما را
\\
ز عشق ناتمام ما جمال یار مستغنی است
&&
به آب و رنگ و خال و خط چه حاجت روی زیبا را
\\
من از آن حسن روزافزون که یوسف داشت دانستم
&&
که عشق از پرده عصمت برون آرد زلیخا را
\\
اگر دشنام فرمایی و گر نفرین دعا گویم
&&
جواب تلخ می‌زیبد لب لعل شکرخا را
\\
نصیحت گوش کن جانا که از جان دوست‌تر دارند
&&
جوانان سعادتمند پند پیر دانا را
\\
حدیث از مطرب و می گو و راز دهر کمتر جو
&&
که کس نگشود و نگشاید به حکمت این معما را
\\
غزل گفتی و در سفتی بیا و خوش بخوان حافظ
&&
که بر نظم تو افشاند فلک عقد ثریا را
\\
\end{longtable}
\end{center}
