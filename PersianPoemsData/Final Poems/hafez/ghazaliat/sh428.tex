\begin{center}
\section*{غزل شماره ۴۲۸: سحرگاهان که مخمور شبانه}
\label{sec:sh428}
\addcontentsline{toc}{section}{\nameref{sec:sh428}}
\begin{longtable}{l p{0.5cm} r}
سحرگاهان که مخمور شبانه
&&
گرفتم باده با چنگ و چغانه
\\
نهادم عقل را ره توشه از می
&&
ز شهر هستیش کردم روانه
\\
نگار می فروشم عشوه‌ای داد
&&
که ایمن گشتم از مکر زمانه
\\
ز ساقی کمان ابرو شنیدم
&&
که ای تیر ملامت را نشانه
\\
نبندی زان میان طرفی کمروار
&&
اگر خود را ببینی در میانه
\\
برو این دام بر مرغی دگر نه
&&
که عنقا را بلند است آشیانه
\\
که بندد طرف وصل از حسن شاهی
&&
که با خود عشق بازد جاودانه
\\
ندیم و مطرب و ساقی همه اوست
&&
خیال آب و گل در ره بهانه
\\
بده کشتی می تا خوش برانیم
&&
از این دریای ناپیداکرانه
\\
وجود ما معماییست حافظ
&&
که تحقیقش فسون است و فسانه
\\
\end{longtable}
\end{center}
