\begin{center}
\section*{غزل شماره ۲۶۳: بیا و کشتی ما در شط شراب انداز}
\label{sec:sh263}
\addcontentsline{toc}{section}{\nameref{sec:sh263}}
\begin{longtable}{l p{0.5cm} r}
بیا و کشتی ما در شط شراب انداز
&&
خروش و ولوله در جان شیخ و شاب انداز
\\
مرا به کشتی باده درافکن ای ساقی
&&
که گفته‌اند نکویی کن و در آب انداز
\\
ز کوی میکده برگشته‌ام ز راه خطا
&&
مرا دگر ز کرم با ره صواب انداز
\\
بیار زان می گلرنگ مشک بو جامی
&&
شرار رشک و حسد در دل گلاب انداز
\\
اگر چه مست و خرابم تو نیز لطفی کن
&&
نظر بر این دل سرگشته خراب انداز
\\
به نیمشب اگرت آفتاب می‌باید
&&
ز روی دختر گلچهر رز نقاب انداز
\\
مهل که روز وفاتم به خاک بسپارند
&&
مرا به میکده بر در خم شراب انداز
\\
ز جور چرخ چو حافظ به جان رسید دلت
&&
به سوی دیو محن ناوک شهاب انداز
\\
\end{longtable}
\end{center}
