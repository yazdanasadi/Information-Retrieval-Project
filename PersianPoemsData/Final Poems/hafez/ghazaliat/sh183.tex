\begin{center}
\section*{غزل شماره ۱۸۳: دوش وقت سحر از غصه نجاتم دادند}
\label{sec:sh183}
\addcontentsline{toc}{section}{\nameref{sec:sh183}}
\begin{longtable}{l p{0.5cm} r}
دوش وقت سحر از غصه نجاتم دادند
&&
واندر آن ظلمت شب آب حیاتم دادند
\\
بیخود از شعشعه پرتو ذاتم کردند
&&
باده از جام تجلی صفاتم دادند
\\
چه مبارک سحری بود و چه فرخنده شبی
&&
آن شب قدر که این تازه براتم دادند
\\
بعد از این روی من و آینه وصف جمال
&&
که در آن جا خبر از جلوه ذاتم دادند
\\
من اگر کامروا گشتم و خوشدل چه عجب
&&
مستحق بودم و این‌ها به زکاتم دادند
\\
هاتف آن روز به من مژده این دولت داد
&&
که بدان جور و جفا صبر و ثباتم دادند
\\
این همه شهد و شکر کز سخنم می‌ریزد
&&
اجر صبریست کز آن شاخ نباتم دادند
\\
همت حافظ و انفاس سحرخیزان بود
&&
که ز بند غم ایام نجاتم دادند
\\
\end{longtable}
\end{center}
