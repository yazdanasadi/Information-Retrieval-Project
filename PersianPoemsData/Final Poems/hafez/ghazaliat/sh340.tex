\begin{center}
\section*{غزل شماره ۳۴۰: من که از آتش دل چون خم می در جوشم}
\label{sec:sh340}
\addcontentsline{toc}{section}{\nameref{sec:sh340}}
\begin{longtable}{l p{0.5cm} r}
من که از آتش دل چون خم می در جوشم
&&
مهر بر لب زده خون می‌خورم و خاموشم
\\
قصد جان است طمع در لب جانان کردن
&&
تو مرا بین که در این کار به جان می‌کوشم
\\
من کی آزاد شوم از غم دل چون هر دم
&&
هندوی زلف بتی حلقه کند در گوشم
\\
حاش لله که نیم معتقد طاعت خویش
&&
این قدر هست که گه گه قدحی می نوشم
\\
هست امیدم که علیرغم عدو روز جزا
&&
فیض عفوش ننهد بار گنه بر دوشم
\\
پدرم روضه رضوان به دو گندم بفروخت
&&
من چرا ملک جهان را به جوی نفروشم
\\
خرقه پوشی من از غایت دین داری نیست
&&
پرده‌ای بر سر صد عیب نهان می‌پوشم
\\
من که خواهم که ننوشم به جز از راوق خم
&&
چه کنم گر سخن پیر مغان ننیوشم
\\
گر از این دست زند مطرب مجلس ره عشق
&&
شعر حافظ ببرد وقت سماع از هوشم
\\
\end{longtable}
\end{center}
