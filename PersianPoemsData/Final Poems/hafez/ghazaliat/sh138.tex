\begin{center}
\section*{غزل شماره ۱۳۸: یاد باد آن که ز ما وقت سفر یاد نکرد}
\label{sec:sh138}
\addcontentsline{toc}{section}{\nameref{sec:sh138}}
\begin{longtable}{l p{0.5cm} r}
یاد باد آن که ز ما وقت سفر یاد نکرد
&&
به وداعی دل غمدیده ما شاد نکرد
\\
آن جوان بخت که می‌زد رقم خیر و قبول
&&
بنده پیر ندانم ز چه آزاد نکرد
\\
کاغذین جامه به خوناب بشویم که فلک
&&
رهنمونیم به پای علم داد نکرد
\\
دل به امید صدایی که مگر در تو رسد
&&
ناله‌ها کرد در این کوه که فرهاد نکرد
\\
سایه تا بازگرفتی ز چمن مرغ سحر
&&
آشیان در شکن طره شمشاد نکرد
\\
شاید ار پیک صبا از تو بیاموزد کار
&&
زان که چالاکتر از این حرکت باد نکرد
\\
کلک مشاطه صنعش نکشد نقش مراد
&&
هر که اقرار بدین حسن خداداد نکرد
\\
مطربا پرده بگردان و بزن راه عراق
&&
که بدین راه بشد یار و ز ما یاد نکرد
\\
غزلیات عراقیست سرود حافظ
&&
که شنید این ره دلسوز که فریاد نکرد
\\
\end{longtable}
\end{center}
