\begin{center}
\section*{غزل شماره ۱۷۸: هر که شد محرم دل در حرم یار بماند}
\label{sec:sh178}
\addcontentsline{toc}{section}{\nameref{sec:sh178}}
\begin{longtable}{l p{0.5cm} r}
هر که شد محرم دل در حرم یار بماند
&&
وان که این کار ندانست در انکار بماند
\\
اگر از پرده برون شد دل من عیب مکن
&&
شکر ایزد که نه در پرده پندار بماند
\\
صوفیان واستدند از گرو می همه رخت
&&
دلق ما بود که در خانه خمار بماند
\\
محتسب شیخ شد و فسق خود از یاد ببرد
&&
قصه ماست که در هر سر بازار بماند
\\
هر می لعل کز آن دست بلورین ستدیم
&&
آب حسرت شد و در چشم گهربار بماند
\\
جز دل من کز ازل تا به ابد عاشق رفت
&&
جاودان کس نشنیدیم که در کار بماند
\\
گشت بیمار که چون چشم تو گردد نرگس
&&
شیوه تو نشدش حاصل و بیمار بماند
\\
از صدای سخن عشق ندیدم خوشتر
&&
یادگاری که در این گنبد دوار بماند
\\
داشتم دلقی و صد عیب مرا می‌پوشید
&&
خرقه رهن می و مطرب شد و زنار بماند
\\
بر جمال تو چنان صورت چین حیران شد
&&
که حدیثش همه جا در در و دیوار بماند
\\
به تماشاگه زلفش دل حافظ روزی
&&
شد که بازآید و جاوید گرفتار بماند
\\
\end{longtable}
\end{center}
