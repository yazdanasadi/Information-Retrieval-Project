\begin{center}
\section*{غزل شماره ۲۵۶: نصیحتی کنمت بشنو و بهانه مگیر}
\label{sec:sh256}
\addcontentsline{toc}{section}{\nameref{sec:sh256}}
\begin{longtable}{l p{0.5cm} r}
نصیحتی کنمت بشنو و بهانه مگیر
&&
هر آنچه ناصح مشفق بگویدت بپذیر
\\
ز وصل روی جوانان تمتعی بردار
&&
که در کمینگه عمر است مکر عالم پیر
\\
نعیم هر دو جهان پیش عاشقان به جوی
&&
که این متاع قلیل است و آن عطای کثیر
\\
معاشری خوش و رودی بساز می‌خواهم
&&
که درد خویش بگویم به ناله بم و زیر
\\
بر آن سرم که ننوشم می و گنه نکنم
&&
اگر موافق تدبیر من شود تقدیر
\\
چو قسمت ازلی بی حضور ما کردند
&&
گر اندکی نه به وفق رضاست خرده مگیر
\\
چو لاله در قدحم ریز ساقیا می و مشک
&&
که نقش خال نگارم نمی‌رود ز ضمیر
\\
بیار ساغر در خوشاب ای ساقی
&&
حسود گو کرم آصفی ببین و بمیر
\\
به عزم توبه نهادم قدح ز کف صد بار
&&
ولی کرشمه ساقی نمی‌کند تقصیر
\\
می دوساله و محبوب چارده ساله
&&
همین بس است مرا صحبت صغیر و کبیر
\\
دل رمیده ما را که پیش می‌گیرد
&&
خبر دهید به مجنون خسته از زنجیر
\\
حدیث توبه در این بزمگه مگو حافظ
&&
که ساقیان کمان ابرویت زنند به تیر
\\
\end{longtable}
\end{center}
