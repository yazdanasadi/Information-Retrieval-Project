\begin{center}
\section*{غزل شماره ۲۲۰: از دیده خون دل همه بر روی ما رود}
\label{sec:sh220}
\addcontentsline{toc}{section}{\nameref{sec:sh220}}
\begin{longtable}{l p{0.5cm} r}
از دیده خون دل همه بر روی ما رود
&&
بر روی ما ز دیده چه گویم چه‌ها رود
\\
ما در درون سینه هوایی نهفته‌ایم
&&
بر باد اگر رود دل ما زان هوا رود
\\
خورشید خاوری کند از رشک جامه چاک
&&
گر ماه مهرپرور من در قبا رود
\\
بر خاک راه یار نهادیم روی خویش
&&
بر روی ما رواست اگر آشنا رود
\\
سیل است آب دیده و هر کس که بگذرد
&&
گر خود دلش ز سنگ بود هم ز جا رود
\\
ما را به آب دیده شب و روز ماجراست
&&
زان رهگذر که بر سر کویش چرا رود
\\
حافظ به کوی میکده دایم به صدق دل
&&
چون صوفیان صومعه دار از صفا رود
\\
\end{longtable}
\end{center}
