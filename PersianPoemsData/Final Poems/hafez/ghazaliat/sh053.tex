\begin{center}
\section*{غزل شماره ۵۳: منم که گوشه میخانه خانقاه من است}
\label{sec:sh053}
\addcontentsline{toc}{section}{\nameref{sec:sh053}}
\begin{longtable}{l p{0.5cm} r}
منم که گوشه میخانه خانقاه من است
&&
دعای پیر مغان ورد صبحگاه من است
\\
گرم ترانه چنگ صبوح نیست چه باک
&&
نوای من به سحر آه عذرخواه من است
\\
ز پادشاه و گدا فارغم بحمدالله
&&
گدای خاک در دوست پادشاه من است
\\
غرض ز مسجد و میخانه‌ام وصال شماست
&&
جز این خیال ندارم خدا گواه من است
\\
مگر به تیغ اجل خیمه برکنم ور نی
&&
رمیدن از در دولت نه رسم و راه من است
\\
از آن زمان که بر این آستان نهادم روی
&&
فراز مسند خورشید تکیه گاه من است
\\
گناه اگر چه نبود اختیار ما حافظ
&&
تو در طریق ادب باش گو گناه من است
\\
\end{longtable}
\end{center}
