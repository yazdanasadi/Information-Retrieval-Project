\begin{center}
\section*{غزل شماره ۸۴: ساقی بیار باده که ماه صیام رفت}
\label{sec:sh084}
\addcontentsline{toc}{section}{\nameref{sec:sh084}}
\begin{longtable}{l p{0.5cm} r}
ساقی بیار باده که ماه صیام رفت
&&
درده قدح که موسم ناموس و نام رفت
\\
وقت عزیز رفت بیا تا قضا کنیم
&&
عمری که بی حضور صراحی و جام رفت
\\
مستم کن آن چنان که ندانم ز بیخودی
&&
در عرصه خیال که آمد کدام رفت
\\
بر بوی آن که جرعه جامت به ما رسد
&&
در مصطبه دعای تو هر صبح و شام رفت
\\
دل را که مرده بود حیاتی به جان رسید
&&
تا بویی از نسیم می‌اش در مشام رفت
\\
زاهد غرور داشت سلامت نبرد راه
&&
رند از ره نیاز به دارالسلام رفت
\\
نقد دلی که بود مرا صرف باده شد
&&
قلب سیاه بود از آن در حرام رفت
\\
در تاب توبه چند توان سوخت همچو عود
&&
می ده که عمر در سر سودای خام رفت
\\
دیگر مکن نصیحت حافظ که ره نیافت
&&
گمگشته‌ای که باده نابش به کام رفت
\\
\end{longtable}
\end{center}
