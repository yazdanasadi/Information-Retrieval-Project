\begin{center}
\section*{غزل شماره ۴۵: در این زمانه رفیقی که خالی از خلل است}
\label{sec:sh045}
\addcontentsline{toc}{section}{\nameref{sec:sh045}}
\begin{longtable}{l p{0.5cm} r}
در این زمانه رفیقی که خالی از خلل است
&&
صراحی می ناب و سفینه غزل است
\\
جریده رو که گذرگاه عافیت تنگ است
&&
پیاله گیر که عمر عزیز بی‌بدل است
\\
نه من ز بی عملی در جهان ملولم و بس
&&
ملالت علما هم ز علم بی عمل است
\\
به چشم عقل در این رهگذار پرآشوب
&&
جهان و کار جهان بی‌ثبات و بی‌محل است
\\
بگیر طره مه چهره‌ای و قصه مخوان
&&
که سعد و نحس ز تاثیر زهره و زحل است
\\
دلم امید فراوان به وصل روی تو داشت
&&
ولی اجل به ره عمر رهزن امل است
\\
به هیچ دور نخواهند یافت هشیارش
&&
چنین که حافظ ما مست باده ازل است
\\
\end{longtable}
\end{center}
