\begin{center}
\section*{غزل شماره ۳۱۷: فاش می‌گویم و از گفته خود دلشادم}
\label{sec:sh317}
\addcontentsline{toc}{section}{\nameref{sec:sh317}}
\begin{longtable}{l p{0.5cm} r}
فاش می‌گویم و از گفته خود دلشادم
&&
بنده عشقم و از هر دو جهان آزادم
\\
طایر گلشن قدسم چه دهم شرح فراق
&&
که در این دامگه حادثه چون افتادم
\\
من ملک بودم و فردوس برین جایم بود
&&
آدم آورد در این دیر خراب آبادم
\\
سایه طوبی و دلجویی حور و لب حوض
&&
به هوای سر کوی تو برفت از یادم
\\
نیست بر لوح دلم جز الف قامت دوست
&&
چه کنم حرف دگر یاد نداد استادم
\\
کوکب بخت مرا هیچ منجم نشناخت
&&
یا رب از مادر گیتی به چه طالع زادم
\\
تا شدم حلقه به گوش در میخانه عشق
&&
هر دم آید غمی از نو به مبارک بادم
\\
می‌خورد خون دلم مردمک دیده سزاست
&&
که چرا دل به جگرگوشه مردم دادم
\\
پاک کن چهره حافظ به سر زلف ز اشک
&&
ور نه این سیل دمادم ببرد بنیادم
\\
\end{longtable}
\end{center}
