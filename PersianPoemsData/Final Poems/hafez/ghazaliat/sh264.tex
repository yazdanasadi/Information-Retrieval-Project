\begin{center}
\section*{غزل شماره ۲۶۴: خیز و در کاسه زر آب طربناک انداز}
\label{sec:sh264}
\addcontentsline{toc}{section}{\nameref{sec:sh264}}
\begin{longtable}{l p{0.5cm} r}
خیز و در کاسه زر آب طربناک انداز
&&
پیشتر زان که شود کاسه سر خاک انداز
\\
عاقبت منزل ما وادی خاموشان است
&&
حالیا غلغله در گنبد افلاک انداز
\\
چشم آلوده نظر از رخ جانان دور است
&&
بر رخ او نظر از آینه پاک انداز
\\
به سر سبز تو ای سرو که گر خاک شوم
&&
ناز از سر بنه و سایه بر این خاک انداز
\\
دل ما را که ز مار سر زلف تو بخست
&&
از لب خود به شفاخانه تریاک انداز
\\
ملک این مزرعه دانی که ثباتی ندهد
&&
آتشی از جگر جام در املاک انداز
\\
غسل در اشک زدم کاهل طریقت گویند
&&
پاک شو اول و پس دیده بر آن پاک انداز
\\
یا رب آن زاهد خودبین که به جز عیب ندید
&&
دود آهیش در آیینه ادراک انداز
\\
چون گل از نکهت او جامه قبا کن حافظ
&&
وین قبا در ره آن قامت چالاک انداز
\\
\end{longtable}
\end{center}
