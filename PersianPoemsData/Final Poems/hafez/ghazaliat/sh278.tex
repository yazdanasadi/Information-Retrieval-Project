\begin{center}
\section*{غزل شماره ۲۷۸: شراب تلخ می‌خواهم که مردافکن بود زورش}
\label{sec:sh278}
\addcontentsline{toc}{section}{\nameref{sec:sh278}}
\begin{longtable}{l p{0.5cm} r}
شراب تلخ می‌خواهم که مردافکن بود زورش
&&
که تا یک دم بیاسایم ز دنیا و شر و شورش
\\
سماط دهر دون پرور ندارد شهد آسایش
&&
مذاق حرص و آز ای دل بشو از تلخ و از شورش
\\
بیاور می که نتوان شد ز مکر آسمان ایمن
&&
به لعب زهره چنگی و مریخ سلحشورش
\\
کمند صید بهرامی بیفکن جام جم بردار
&&
که من پیمودم این صحرا نه بهرام است و نه گورش
\\
بیا تا در می صافیت راز دهر بنمایم
&&
به شرط آن که ننمایی به کج طبعان دل کورش
\\
نظر کردن به درویشان منافی بزرگی نیست
&&
سلیمان با چنان حشمت نظرها بود با مورش
\\
کمان ابروی جانان نمی‌پیچد سر از حافظ
&&
ولیکن خنده می‌آید بدین بازوی بی زورش
\\
\end{longtable}
\end{center}
