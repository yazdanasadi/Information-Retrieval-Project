\begin{center}
\section*{غزل شماره ۳۳: خلوت گزیده را به تماشا چه حاجت است}
\label{sec:sh033}
\addcontentsline{toc}{section}{\nameref{sec:sh033}}
\begin{longtable}{l p{0.5cm} r}
خلوت گزیده را به تماشا چه حاجت است
&&
چون کوی دوست هست به صحرا چه حاجت است
\\
جانا به حاجتی که تو را هست با خدا
&&
کآخر دمی بپرس که ما را چه حاجت است
\\
ای پادشاه حسن خدا را بسوختیم
&&
آخر سؤال کن که گدا را چه حاجت است
\\
ارباب حاجتیم و زبان سؤال نیست
&&
در حضرت کریم تمنا چه حاجت است
\\
محتاج قصه نیست گرت قصد خون ماست
&&
چون رخت از آن توست به یغما چه حاجت است
\\
جام جهان نماست ضمیر منیر دوست
&&
اظهار احتیاج خود آن جا چه حاجت است
\\
آن شد که بار منت ملاح بردمی
&&
گوهر چو دست داد به دریا چه حاجت است
\\
ای مدعی برو که مرا با تو کار نیست
&&
احباب حاضرند به اعدا چه حاجت است
\\
ای عاشق گدا چو لب روح بخش یار
&&
می‌داندت وظیفه تقاضا چه حاجت است
\\
حافظ تو ختم کن که هنر خود عیان شود
&&
با مدعی نزاع و محاکا چه حاجت است
\\
\end{longtable}
\end{center}
