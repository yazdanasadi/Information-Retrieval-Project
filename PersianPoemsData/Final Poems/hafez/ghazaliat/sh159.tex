\begin{center}
\section*{غزل شماره ۱۵۹: نقد صوفی نه همه صافی بی‌غش باشد}
\label{sec:sh159}
\addcontentsline{toc}{section}{\nameref{sec:sh159}}
\begin{longtable}{l p{0.5cm} r}
نقد صوفی نه همه صافی بی‌غش باشد
&&
ای بسا خرقه که مستوجب آتش باشد
\\
صوفی ما که ز ورد سحری مست شدی
&&
شامگاهش نگران باش که سرخوش باشد
\\
خوش بود گر محک تجربه آید به میان
&&
تا سیه روی شود هر که در او غش باشد
\\
خط ساقی گر از این گونه زند نقش بر آب
&&
ای بسا رخ که به خونابه منقش باشد
\\
ناز پرورد تنعم نبرد راه به دوست
&&
عاشقی شیوه رندان بلاکش باشد
\\
غم دنیای دنی چند خوری باده بخور
&&
حیف باشد دل دانا که مشوش باشد
\\
دلق و سجاده حافظ ببرد باده فروش
&&
گر شرابش ز کف ساقی مه وش باشد
\\
\end{longtable}
\end{center}
