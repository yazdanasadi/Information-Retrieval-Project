\begin{center}
\section*{غزل شماره ۲۴۵: الا ای طوطی گویای اسرار}
\label{sec:sh245}
\addcontentsline{toc}{section}{\nameref{sec:sh245}}
\begin{longtable}{l p{0.5cm} r}
الا ای طوطی گویای اسرار
&&
مبادا خالیت شکر ز منقار
\\
سرت سبز و دلت خوش باد جاوید
&&
که خوش نقشی نمودی از خط یار
\\
سخن سربسته گفتی با حریفان
&&
خدا را زین معما پرده بردار
\\
به روی ما زن از ساغر گلابی
&&
که خواب آلوده‌ایم ای بخت بیدار
\\
چه ره بود این که زد در پرده مطرب
&&
که می‌رقصند با هم مست و هشیار
\\
از آن افیون که ساقی در می افکند
&&
حریفان را نه سر ماند نه دستار
\\
سکندر را نمی‌بخشند آبی
&&
به زور و زر میسر نیست این کار
\\
بیا و حال اهل درد بشنو
&&
به لفظ اندک و معنی بسیار
\\
بت چینی عدوی دین و دل‌هاست
&&
خداوندا دل و دینم نگه دار
\\
به مستوران مگو اسرار مستی
&&
حدیث جان مگو با نقش دیوار
\\
به یمن دولت منصور شاهی
&&
علم شد حافظ اندر نظم اشعار
\\
خداوندی به جای بندگان کرد
&&
خداوندا ز آفاتش نگه دار
\\
\end{longtable}
\end{center}
