\begin{center}
\section*{غزل شماره ۴۷۹: صبح است و ژاله می‌چکد از ابر بهمنی}
\label{sec:sh479}
\addcontentsline{toc}{section}{\nameref{sec:sh479}}
\begin{longtable}{l p{0.5cm} r}
صبح است و ژاله می‌چکد از ابر بهمنی
&&
برگ صبوح ساز و بده جام یک منی
\\
در بحر مایی و منی افتاده‌ام بیار
&&
می تا خلاص بخشدم از مایی و منی
\\
خون پیاله خور که حلال است خون او
&&
در کار یار باش که کاریست کردنی
\\
ساقی به دست باش که غم در کمین ماست
&&
مطرب نگاه دار همین ره که می‌زنی
\\
می ده که سر به گوش من آورد چنگ و گفت
&&
خوش بگذران و بشنو از این پیر منحنی
\\
ساقی به بی‌نیازی رندان که می بده
&&
تا بشنوی ز صوت مغنی هوالغنی
\\
\end{longtable}
\end{center}
