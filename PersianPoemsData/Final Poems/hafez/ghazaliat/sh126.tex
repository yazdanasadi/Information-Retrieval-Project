\begin{center}
\section*{غزل شماره ۱۲۶: جان بی جمال جانان میل جهان ندارد}
\label{sec:sh126}
\addcontentsline{toc}{section}{\nameref{sec:sh126}}
\begin{longtable}{l p{0.5cm} r}
جان بی جمال جانان میل جهان ندارد
&&
هر کس که این ندارد حقا که آن ندارد
\\
با هیچ کس نشانی زان دلستان ندیدم
&&
یا من خبر ندارم یا او نشان ندارد
\\
هر شبنمی در این ره صد بحر آتشین است
&&
دردا که این معما شرح و بیان ندارد
\\
سرمنزل فراغت نتوان ز دست دادن
&&
ای ساروان فروکش کاین ره کران ندارد
\\
چنگ خمیده قامت می‌خواندت به عشرت
&&
بشنو که پند پیران هیچت زیان ندارد
\\
ای دل طریق رندی از محتسب بیاموز
&&
مست است و در حق او کس این گمان ندارد
\\
احوال گنج قارون کایام داد بر باد
&&
در گوش دل فروخوان تا زر نهان ندارد
\\
گر خود رقیب شمع است اسرار از او بپوشان
&&
کان شوخ سربریده بند زبان ندارد
\\
کس در جهان ندارد یک بنده همچو حافظ
&&
زیرا که چون تو شاهی کس در جهان ندارد
\\
\end{longtable}
\end{center}
