\begin{center}
\section*{غزل شماره ۹۰: ای هدهد صبا به سبا می‌فرستمت}
\label{sec:sh090}
\addcontentsline{toc}{section}{\nameref{sec:sh090}}
\begin{longtable}{l p{0.5cm} r}
ای هدهد صبا به سبا می‌فرستمت
&&
بنگر که از کجا به کجا می‌فرستمت
\\
حیف است طایری چو تو در خاکدان غم
&&
زین جا به آشیان وفا می‌فرستمت
\\
در راه عشق مرحله قرب و بعد نیست
&&
می‌بینمت عیان و دعا می‌فرستمت
\\
هر صبح و شام قافله‌ای از دعای خیر
&&
در صحبت شمال و صبا می‌فرستمت
\\
تا لشکر غمت نکند ملک دل خراب
&&
جان عزیز خود به نوا می‌فرستمت
\\
ای غایب از نظر که شدی همنشین دل
&&
می‌گویمت دعا و ثنا می‌فرستمت
\\
در روی خود تفرج صنع خدای کن
&&
کآیینهٔ خدای نما می‌فرستمت
\\
تا مطربان ز شوق منت آگهی دهند
&&
قول و غزل به ساز و نوا می‌فرستمت
\\
ساقی بیا که هاتف غیبم به مژده گفت
&&
با درد صبر کن که دوا می‌فرستمت
\\
حافظ سرود مجلس ما ذکر خیر توست
&&
بشتاب هان که اسب و قبا می‌فرستمت
\\
\end{longtable}
\end{center}
