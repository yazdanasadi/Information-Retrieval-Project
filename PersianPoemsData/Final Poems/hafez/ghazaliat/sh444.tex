\begin{center}
\section*{غزل شماره ۴۴۴: شهریست پرظریفان و از هر طرف نگاری}
\label{sec:sh444}
\addcontentsline{toc}{section}{\nameref{sec:sh444}}
\begin{longtable}{l p{0.5cm} r}
شهریست پرظریفان و از هر طرف نگاری
&&
یاران صلای عشق است گر می‌کنید کاری
\\
چشم فلک نبیند زین طرفه‌تر جوانی
&&
در دست کس نیفتد زین خوبتر نگاری
\\
هرگز که دیده باشد جسمی ز جان مرکب
&&
بر دامنش مبادا زین خاکیان غباری
\\
چون من شکسته‌ای را از پیش خود چه رانی
&&
کم غایت توقع بوسیست یا کناری
\\
می بی‌غش است دریاب وقتی خوش است بشتاب
&&
سال دگر که دارد امید نوبهاری
\\
در بوستان حریفان مانند لاله و گل
&&
هر یک گرفته جامی بر یاد روی یاری
\\
چون این گره گشایم وین راز چون نمایم
&&
دردی و سخت دردی کاری و صعب کاری
\\
هر تار موی حافظ در دست زلف شوخی
&&
مشکل توان نشستن در این چنین دیاری
\\
\end{longtable}
\end{center}
