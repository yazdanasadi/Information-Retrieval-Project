\begin{center}
\section*{غزل شماره ۱۶۲: خوش آمد گل وز آن خوشتر نباشد}
\label{sec:sh162}
\addcontentsline{toc}{section}{\nameref{sec:sh162}}
\begin{longtable}{l p{0.5cm} r}
خوش آمد گل وز آن خوشتر نباشد
&&
که در دستت به جز ساغر نباشد
\\
زمان خوشدلی دریاب و در یاب
&&
که دایم در صدف گوهر نباشد
\\
غنیمت دان و می خور در گلستان
&&
که گل تا هفته دیگر نباشد
\\
ایا پرلعل کرده جام زرین
&&
ببخشا بر کسی کش زر نباشد
\\
بیا ای شیخ و از خمخانه ما
&&
شرابی خور که در کوثر نباشد
\\
بشوی اوراق اگر همدرس مایی
&&
که علم عشق در دفتر نباشد
\\
ز من بنیوش و دل در شاهدی بند
&&
که حسنش بسته زیور نباشد
\\
شرابی بی خمارم بخش یا رب
&&
که با وی هیچ درد سر نباشد
\\
من از جان بنده سلطان اویسم
&&
اگر چه یادش از چاکر نباشد
\\
به تاج عالم آرایش که خورشید
&&
چنین زیبنده افسر نباشد
\\
کسی گیرد خطا بر نظم حافظ
&&
که هیچش لطف در گوهر نباشد
\\
\end{longtable}
\end{center}
