\begin{center}
\section*{غزل شماره ۲۹۳: بامدادان که ز خلوتگه کاخ ابداع}
\label{sec:sh293}
\addcontentsline{toc}{section}{\nameref{sec:sh293}}
\begin{longtable}{l p{0.5cm} r}
بامدادان که ز خلوتگه کاخ ابداع
&&
شمع خاور فکند بر همه اطراف شعاع
\\
برکشد آینه از جیب افق چرخ و در آن
&&
بنماید رخ گیتی به هزاران انواع
\\
در زوایای طربخانه جمشید فلک
&&
ارغنون ساز کند زهره به آهنگ سماع
\\
چنگ در غلغله آید که کجا شد منکر
&&
جام در قهقهه آید که کجا شد مناع
\\
وضع دوران بنگر ساغر عشرت بر گیر
&&
که به هر حالتی این است بهین اوضاع
\\
طره شاهد دنیی همه بند است و فریب
&&
عارفان بر سر این رشته نجویند نزاع
\\
عمر خسرو طلب ار نفع جهان می‌خواهی
&&
که وجودیست عطابخش کریم نفاع
\\
مظهر لطف ازل روشنی چشم امل
&&
جامع علم و عمل جان جهان شاه شجاع
\\
\end{longtable}
\end{center}
