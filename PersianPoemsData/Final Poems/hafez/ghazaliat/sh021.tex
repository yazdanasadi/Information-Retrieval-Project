\begin{center}
\section*{غزل شماره ۲۱: دل و دینم شد و دلبر به ملامت برخاست}
\label{sec:sh021}
\addcontentsline{toc}{section}{\nameref{sec:sh021}}
\begin{longtable}{l p{0.5cm} r}
دل و دینم شد و دلبر به ملامت برخاست
&&
گفت با ما منشین کز تو سلامت برخاست
\\
که شنیدی که در این بزم دمی خوش بنشست
&&
که نه در آخر صحبت به ندامت برخاست
\\
شمع اگر زان لب خندان به زبان لافی زد
&&
پیش عشاق تو شب‌ها به غرامت برخاست
\\
در چمن باد بهاری ز کنار گل و سرو
&&
به هواداری آن عارض و قامت برخاست
\\
مست بگذشتی و از خلوتیان ملکوت
&&
به تماشای تو آشوب قیامت برخاست
\\
پیش رفتار تو پا برنگرفت از خجلت
&&
سرو سرکش که به ناز از قد و قامت برخاست
\\
حافظ این خرقه بینداز مگر جان ببری
&&
کاتش از خرقه سالوس و کرامت برخاست
\\
\end{longtable}
\end{center}
