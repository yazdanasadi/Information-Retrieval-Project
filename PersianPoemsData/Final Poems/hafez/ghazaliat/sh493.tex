\begin{center}
\section*{غزل شماره ۴۹۳: ای پادشه خوبان داد از غم تنهایی}
\label{sec:sh493}
\addcontentsline{toc}{section}{\nameref{sec:sh493}}
\begin{longtable}{l p{0.5cm} r}
ای پادشه خوبان داد از غم تنهایی
&&
دل بی تو به جان آمد وقت است که بازآیی
\\
دایم گل این بستان شاداب نمی‌ماند
&&
دریاب ضعیفان را در وقت توانایی
\\
دیشب گله زلفش با باد همی‌کردم
&&
گفتا غلطی بگذر زین فکرت سودایی
\\
صد باد صبا این جا با سلسله می‌رقصند
&&
این است حریف ای دل تا باد نپیمایی
\\
مشتاقی و مهجوری دور از تو چنانم کرد
&&
کز دست بخواهد شد پایاب شکیبایی
\\
یا رب به که شاید گفت این نکته که در عالم
&&
رخساره به کس ننمود آن شاهد هرجایی
\\
ساقی چمن گل را بی روی تو رنگی نیست
&&
شمشاد خرامان کن تا باغ بیارایی
\\
ای درد توام درمان در بستر ناکامی
&&
و ای یاد توام مونس در گوشه تنهایی
\\
در دایره قسمت ما نقطه تسلیمیم
&&
لطف آن چه تو اندیشی حکم آن چه تو فرمایی
\\
فکر خود و رای خود در عالم رندی نیست
&&
کفر است در این مذهب خودبینی و خودرایی
\\
زین دایره مینا خونین جگرم می ده
&&
تا حل کنم این مشکل در ساغر مینایی
\\
حافظ شب هجران شد بوی خوش وصل آمد
&&
شادیت مبارک باد ای عاشق شیدایی
\\
\end{longtable}
\end{center}
