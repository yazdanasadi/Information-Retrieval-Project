\begin{center}
\section*{غزل شماره ۳۰۵: به وقت گل شدم از توبه شراب خجل}
\label{sec:sh305}
\addcontentsline{toc}{section}{\nameref{sec:sh305}}
\begin{longtable}{l p{0.5cm} r}
به وقت گل شدم از توبهٔ شراب خجل
&&
که کس مباد ز کردار ناصواب خجل
\\
صلاح ما همه دام ره است و من زین بحث
&&
نیم ز شاهد و ساقی به هیچ باب خجل
\\
بود که یار نرنجد ز ما به خلق کریم
&&
که از سؤال ملولیم و از جواب خجل
\\
ز خون که رفت شب دوش از سراچهٔ چشم
&&
شدیم در نظر رهروان خواب خجل
\\
رواست نرگس مست ار فکند سر در پیش
&&
که شد ز شیوهٔ آن چشم پر عتاب خجل
\\
تویی که خوبتری ز آفتاب و شکر خدا
&&
که نیستم ز تو در روی آفتاب خجل
\\
حجاب ظلمت از آن بست آب خضر که گشت
&&
ز شعر حافظ و آن طبع همچو آب خجل
\\
\end{longtable}
\end{center}
