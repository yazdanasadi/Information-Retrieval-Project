\begin{center}
\section*{غزل شماره ۱۵۷: هر که را با خط سبزت سر سودا باشد}
\label{sec:sh157}
\addcontentsline{toc}{section}{\nameref{sec:sh157}}
\begin{longtable}{l p{0.5cm} r}
هر که را با خط سبزت سر سودا باشد
&&
پای از این دایره بیرون ننهد تا باشد
\\
من چو از خاک لحد لاله صفت برخیزم
&&
داغ سودای توام سر سویدا باشد
\\
تو خود ای گوهر یک دانه کجایی آخر
&&
کز غمت دیده مردم همه دریا باشد
\\
از بن هر مژه‌ام آب روان است بیا
&&
اگرت میل لب جوی و تماشا باشد
\\
چون گل و می دمی از پرده برون آی و درآ
&&
که دگرباره ملاقات نه پیدا باشد
\\
ظل ممدود خم زلف توام بر سر باد
&&
کاندر این سایه قرار دل شیدا باشد
\\
چشمت از ناز به حافظ نکند میل آری
&&
سرگرانی صفت نرگس رعنا باشد
\\
\end{longtable}
\end{center}
