\begin{center}
\section*{غزل شماره ۱۷۷: نه هر که چهره برافروخت دلبری داند}
\label{sec:sh177}
\addcontentsline{toc}{section}{\nameref{sec:sh177}}
\begin{longtable}{l p{0.5cm} r}
نه هر که چهره برافروخت دلبری داند
&&
نه هر که آینه سازد سکندری داند
\\
نه هر که طرف کله کج نهاد و تند نشست
&&
کلاه داری و آیین سروری داند
\\
تو بندگی چو گدایان به شرط مزد مکن
&&
که دوست خود روش بنده پروری داند
\\
غلام همت آن رند عافیت سوزم
&&
که در گداصفتی کیمیاگری داند
\\
وفا و عهد نکو باشد ار بیاموزی
&&
وگرنه هر که تو بینی ستمگری داند
\\
بباختم دل دیوانه و ندانستم
&&
که آدمی بچه‌ای شیوه پری داند
\\
هزار نکته باریکتر ز مو این جاست
&&
نه هر که سر بتراشد قلندری داند
\\
مدار نقطه بینش ز خال توست مرا
&&
که قدر گوهر یک دانه جوهری داند
\\
به قد و چهره هر آن کس که شاه خوبان شد
&&
جهان بگیرد اگر دادگستری داند
\\
ز شعر دلکش حافظ کسی بود آگاه
&&
که لطف طبع و سخن گفتن دری داند
\\
\end{longtable}
\end{center}
