\begin{center}
\section*{غزل شماره ۲۲۲: از سر کوی تو هر کو به ملالت برود}
\label{sec:sh222}
\addcontentsline{toc}{section}{\nameref{sec:sh222}}
\begin{longtable}{l p{0.5cm} r}
از سر کوی تو هر کو به ملالت برود
&&
نرود کارش و آخر به خجالت برود
\\
کاروانی که بود بدرقه‌اش حفظ خدا
&&
به تجمل بنشیند به جلالت برود
\\
سالک از نور هدایت ببرد راه به دوست
&&
که به جایی نرسد گر به ضلالت برود
\\
کام خود آخر عمر از می و معشوق بگیر
&&
حیف اوقات که یک سر به بطالت برود
\\
ای دلیل دل گمگشته خدا را مددی
&&
که غریب ار نبرد ره به دلالت برود
\\
حکم مستوری و مستی همه بر خاتمت است
&&
کس ندانست که آخر به چه حالت برود
\\
حافظ از چشمه حکمت به کف آور جامی
&&
بو که از لوح دلت نقش جهالت برود
\\
\end{longtable}
\end{center}
