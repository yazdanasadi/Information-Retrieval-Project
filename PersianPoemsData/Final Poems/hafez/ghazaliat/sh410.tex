\begin{center}
\section*{غزل شماره ۴۱۰: ای قبای پادشاهی راست بر بالای تو}
\label{sec:sh410}
\addcontentsline{toc}{section}{\nameref{sec:sh410}}
\begin{longtable}{l p{0.5cm} r}
ای قبای پادشاهی راست بر بالای تو
&&
زینت تاج و نگین از گوهر والای تو
\\
آفتاب فتح را هر دم طلوعی می‌دهد
&&
از کلاه خسروی رخسار مه سیمای تو
\\
جلوه گاه طایر اقبال باشد هر کجا
&&
سایه‌اندازد همای چتر گردون سای تو
\\
از رسوم شرع و حکمت با هزاران اختلاف
&&
نکته‌ای هرگز نشد فوت از دل دانای تو
\\
آب حیوانش ز منقار بلاغت می‌چکد
&&
طوطی خوش لهجه یعنی کلک شکرخای تو
\\
گر چه خورشید فلک چشم و چراغ عالم است
&&
روشنایی بخش چشم اوست خاک پای تو
\\
آن چه اسکندر طلب کرد و ندادش روزگار
&&
جرعه‌ای بود از زلال جام جان افزای تو
\\
عرض حاجت در حریم حضرتت محتاج نیست
&&
راز کس مخفی نماند با فروغ رای تو
\\
خسروا پیرانه سر حافظ جوانی می‌کند
&&
بر امید عفو جان بخش گنه فرسای تو
\\
\end{longtable}
\end{center}
