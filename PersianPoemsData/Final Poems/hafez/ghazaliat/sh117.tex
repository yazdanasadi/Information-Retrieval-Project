\begin{center}
\section*{غزل شماره ۱۱۷: دل ما به دور رویت ز چمن فراغ دارد}
\label{sec:sh117}
\addcontentsline{toc}{section}{\nameref{sec:sh117}}
\begin{longtable}{l p{0.5cm} r}
دل ما به دور رویت ز چمن فراغ دارد
&&
که چو سرو پایبند است و چو لاله داغ دارد
\\
سر ما فرونیاید به کمان ابروی کس
&&
که درون گوشه گیران ز جهان فراغ دارد
\\
ز بنفشه تاب دارم که ز زلف او زند دم
&&
تو سیاه کم بها بین که چه در دماغ دارد
\\
به چمن خرام و بنگر بر تخت گل که لاله
&&
به ندیم شاه ماند که به کف ایاغ دارد
\\
شب ظلمت و بیابان به کجا توان رسیدن
&&
مگر آن که شمع رویت به رهم چراغ دارد
\\
من و شمع صبحگاهی سزد ار به هم بگرییم
&&
که بسوختیم و از ما بت ما فراغ دارد
\\
سزدم چو ابر بهمن که بر این چمن بگریم
&&
طرب آشیان بلبل بنگر که زاغ دارد
\\
سر درس عشق دارد دل دردمند حافظ
&&
که نه خاطر تماشا نه هوای باغ دارد
\\
\end{longtable}
\end{center}
