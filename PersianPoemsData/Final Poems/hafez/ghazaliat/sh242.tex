\begin{center}
\section*{غزل شماره ۲۴۲: بیا که رایت منصور پادشاه رسید}
\label{sec:sh242}
\addcontentsline{toc}{section}{\nameref{sec:sh242}}
\begin{longtable}{l p{0.5cm} r}
بیا که رایت منصور پادشاه رسید
&&
نوید فتح و بشارت به مهر و ماه رسید
\\
جمال بخت ز روی ظفر نقاب انداخت
&&
کمال عدل به فریاد دادخواه رسید
\\
سپهر دور خوش اکنون کند که ماه آمد
&&
جهان به کام دل اکنون رسد که شاه رسید
\\
ز قاطعان طریق این زمان شوند ایمن
&&
قوافل دل و دانش که مرد راه رسید
\\
عزیز مصر به رغم برادران غیور
&&
ز قعر چاه برآمد به اوج ماه رسید
\\
کجاست صوفی دجال فعل ملحدشکل
&&
بگو بسوز که مهدی دین پناه رسید
\\
صبا بگو که چه‌ها بر سرم در این غم عشق
&&
ز آتش دل سوزان و دود آه رسید
\\
ز شوق روی تو شاها بدین اسیر فراق
&&
همان رسید کز آتش به برگ کاه رسید
\\
مرو به خواب که حافظ به بارگاه قبول
&&
ز ورد نیم شب و درس صبحگاه رسید
\\
\end{longtable}
\end{center}
