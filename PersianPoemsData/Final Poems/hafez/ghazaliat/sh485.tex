\begin{center}
\section*{غزل شماره ۴۸۵: ساقیا سایه ابر است و بهار و لب جوی}
\label{sec:sh485}
\addcontentsline{toc}{section}{\nameref{sec:sh485}}
\begin{longtable}{l p{0.5cm} r}
ساقیا سایه ابر است و بهار و لب جوی
&&
من نگویم چه کن ار اهل دلی خود تو بگوی
\\
بوی یک رنگی از این نقش نمی‌آید خیز
&&
دلق آلوده صوفی به می ناب بشوی
\\
سفله طبع است جهان بر کرمش تکیه مکن
&&
ای جهان دیده ثبات قدم از سفله مجوی
\\
دو نصیحت کنمت بشنو و صد گنج ببر
&&
از در عیش درآ و به ره عیب مپوی
\\
شکر آن را که دگربار رسیدی به بهار
&&
بیخ نیکی بنشان و ره تحقیق بجوی
\\
روی جانان طلبی آینه را قابل ساز
&&
ور نه هرگز گل و نسرین ندمد ز آهن و روی
\\
گوش بگشای که بلبل به فغان می‌گوید
&&
خواجه تقصیر مفرما گل توفیق ببوی
\\
گفتی از حافظ ما بوی ریا می‌آید
&&
آفرین بر نفست باد که خوش بردی بوی
\\
\end{longtable}
\end{center}
