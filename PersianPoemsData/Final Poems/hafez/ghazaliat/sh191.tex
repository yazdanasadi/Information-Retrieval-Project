\begin{center}
\section*{غزل شماره ۱۹۱: آن کیست کز روی کرم با ما وفاداری کند}
\label{sec:sh191}
\addcontentsline{toc}{section}{\nameref{sec:sh191}}
\begin{longtable}{l p{0.5cm} r}
آن کیست کز روی کرم با ما وفاداری کند
&&
بر جای بدکاری چو من یک دم نکوکاری کند
\\
اول به بانگ نای و نی آرد به دل پیغام وی
&&
وان گه به یک پیمانه می با من وفاداری کند
\\
دلبر که جان فرسود از او کام دلم نگشود از او
&&
نومید نتوان بود از او باشد که دلداری کند
\\
گفتم گره نگشوده‌ام زان طره تا من بوده‌ام
&&
گفتا منش فرموده‌ام تا با تو طراری کند
\\
پشمینه پوش تندخو از عشق نشنیده‌است بو
&&
از مستیش رمزی بگو تا ترک هشیاری کند
\\
چون من گدای بی‌نشان مشکل بود یاری چنان
&&
سلطان کجا عیش نهان با رند بازاری کند
\\
زان طره پرپیچ و خم سهل است اگر بینم ستم
&&
از بند و زنجیرش چه غم هر کس که عیاری کند
\\
شد لشکر غم بی عدد از بخت می‌خواهم مدد
&&
تا فخر دین عبدالصمد باشد که غمخواری کند
\\
با چشم پرنیرنگ او حافظ مکن آهنگ او
&&
کان طره شبرنگ او بسیار طراری کند
\\
\end{longtable}
\end{center}
