\begin{center}
\section*{غزل شماره ۱۷۵: صبا به تهنیت پیر می فروش آمد}
\label{sec:sh175}
\addcontentsline{toc}{section}{\nameref{sec:sh175}}
\begin{longtable}{l p{0.5cm} r}
صبا به تهنیت پیر می فروش آمد
&&
که موسم طرب و عیش و ناز و نوش آمد
\\
هوا مسیح نفس گشت و باد نافه گشای
&&
درخت سبز شد و مرغ در خروش آمد
\\
تنور لاله چنان برفروخت باد بهار
&&
که غنچه غرق عرق گشت و گل به جوش آمد
\\
به گوش هوش نیوش از من و به عشرت کوش
&&
که این سخن سحر از هاتفم به گوش آمد
\\
ز فکر تفرقه بازآی تا شوی مجموع
&&
به حکم آن که چو شد اهرمن سروش آمد
\\
ز مرغ صبح ندانم که سوسن آزاد
&&
چه گوش کرد که با ده زبان خموش آمد
\\
چه جای صحبت نامحرم است مجلس انس
&&
سر پیاله بپوشان که خرقه پوش آمد
\\
ز خانقاه به میخانه می‌رود حافظ
&&
مگر ز مستی زهد ریا به هوش آمد
\\
\end{longtable}
\end{center}
