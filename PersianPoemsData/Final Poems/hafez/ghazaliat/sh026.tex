\begin{center}
\section*{غزل شماره ۲۶: زلف آشفته و خوی کرده و خندان لب و مست}
\label{sec:sh026}
\addcontentsline{toc}{section}{\nameref{sec:sh026}}
\begin{longtable}{l p{0.5cm} r}
زلف آشفته و خوی کرده و خندان لب و مست
&&
پیرهن چاک و غزل خوان و صراحی در دست
\\
نرگسش عربده جوی و لبش افسوس کنان
&&
نیم شب دوش به بالین من آمد بنشست
\\
سر فرا گوش من آورد به آواز حزین
&&
گفت ای عاشق دیرینه من خوابت هست
\\
عاشقی را که چنین باده شبگیر دهند
&&
کافر عشق بود گر نشود باده پرست
\\
برو ای زاهد و بر دردکشان خرده مگیر
&&
که ندادند جز این تحفه به ما روز الست
\\
آن چه او ریخت به پیمانه ما نوشیدیم
&&
اگر از خمر بهشت است وگر باده مست
\\
خنده جام می و زلف گره گیر نگار
&&
ای بسا توبه که چون توبه حافظ بشکست
\\
\end{longtable}
\end{center}
