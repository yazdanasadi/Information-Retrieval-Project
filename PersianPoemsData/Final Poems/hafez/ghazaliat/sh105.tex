\begin{center}
\section*{غزل شماره ۱۰۵: صوفی ار باده به اندازه خورد نوشش باد}
\label{sec:sh105}
\addcontentsline{toc}{section}{\nameref{sec:sh105}}
\begin{longtable}{l p{0.5cm} r}
صوفی ار باده به اندازه خورد نوشش باد
&&
ور نه اندیشه این کار فراموشش باد
\\
آن که یک جرعه می از دست تواند دادن
&&
دست با شاهد مقصود در آغوشش باد
\\
پیر ما گفت خطا بر قلم صنع نرفت
&&
آفرین بر نظر پاک خطاپوشش باد
\\
شاه ترکان سخن مدعیان می‌شنود
&&
شرمی از مظلمه خون سیاووشش باد
\\
گر چه از کبر سخن با من درویش نگفت
&&
جان فدای شکرین پسته خاموشش باد
\\
چشمم از آینه داران خط و خالش گشت
&&
لبم از بوسه ربایان بر و دوشش باد
\\
نرگس مست نوازش کن مردم دارش
&&
خون عاشق به قدح گر بخورد نوشش باد
\\
به غلامی تو مشهور جهان شد حافظ
&&
حلقه بندگی زلف تو در گوشش باد
\\
\end{longtable}
\end{center}
