\begin{center}
\section*{غزل شماره ۳۵: برو به کار خود ای واعظ این چه فریادست}
\label{sec:sh035}
\addcontentsline{toc}{section}{\nameref{sec:sh035}}
\begin{longtable}{l p{0.5cm} r}
برو به کار خود ای واعظ این چه فریادست
&&
مرا فتاد دل از ره تو را چه افتادست
\\
میان او که خدا آفریده است از هیچ
&&
دقیقه‌ایست که هیچ آفریده نگشادست
\\
به کام تا نرساند مرا لبش چون نای
&&
نصیحت همه عالم به گوش من بادست
\\
گدای کوی تو از هشت خلد مستغنیست
&&
اسیر عشق تو از هر دو عالم آزادست
\\
اگر چه مستی عشقم خراب کرد ولی
&&
اساس هستی من زان خراب آبادست
\\
دلا منال ز بیداد و جور یار که یار
&&
تو را نصیب همین کرد و این از آن دادست
\\
برو فسانه مخوان و فسون مدم حافظ
&&
کز این فسانه و افسون مرا بسی یادست
\\
\end{longtable}
\end{center}
