\begin{center}
\section*{بخش ۱۳۲ - فرق میان آنک درویش است به خدا و تشنهٔ خدا و میان آنک درویش است از خدا و تشنهٔ غیرست}
\label{sec:sh132}
\addcontentsline{toc}{section}{\nameref{sec:sh132}}
\begin{longtable}{l p{0.5cm} r}
نقش درویشست او نه اهل نان
&&
نقش سگ را تو مینداز استخوان
\\
فقر لقمه دارد او نه فقر حق
&&
پیش نقش مرده‌ای کم نه طبق
\\
ماهی خاکی بود درویش نان
&&
شکل ماهی لیک از دریا رمان
\\
مرغ خانه‌ست او نه سیمرغ هوا
&&
لوت نوشد او ننوشد از خدا
\\
عاشق حقست او بهر نوال
&&
نیست جانش عاشق حسن و جمال
\\
گر توهم می‌کند او عشق ذات
&&
ذات نبود وهم اسما و صفات
\\
وهم مخلوقست و مولود آمدست
&&
حق نزاییده‌ست او لم یولدست
\\
عاشق تصویر و وهم خویشتن
&&
کی بود از عاشقان ذوالمنن
\\
عاشق آن وهم اگر صادق بود
&&
آن مجاز او حقیقت‌کش شود
\\
شرح می‌خواهد بیان این سخن
&&
لیک می‌ترسم ز افهام کهن
\\
فهمهای کهنهٔ کوته‌نظر
&&
صد خیال بد در آرد در فکر
\\
بر سماع راست هر کس چیر نیست
&&
لقمهٔ هر مرغکی انجیر نیست
\\
خاصه مرغی مرده‌ای پوسیده‌ای
&&
پرخیالی اعمیی بی‌دیده‌ای
\\
نقش ماهی را چه دریا و چه خاک
&&
رنگ هندو را چه صابون و چه زاک
\\
نقش اگر غمگین نگاری بر ورق
&&
او ندارد از غم و شادی سبق
\\
صورتش غمگین و او فارغ از آن
&&
صورتش خندان و او زان بی‌نشان
\\
وین غم و شادی که اندر دل حظیست
&&
پیش آن شادی و غم جز نقش نیست
\\
صورت غمگین نقش از بهر ماست
&&
تا که ما را یاد آید راه راست
\\
صورت خندان نقش از بهر تست
&&
تا از آن صورت شود معنی درست
\\
نقشهایی کاندرین حمامهاست
&&
از برون جامه‌کن چون جامه‌هاست
\\
تا برونی جامه‌ها بینی و بس
&&
جامه بیرون کن درآ ای هم‌نفس
\\
زانک با جامه درون سو راه نیست
&&
تن ز جان جامه ز تن آگاه نیست
\\
\end{longtable}
\end{center}
