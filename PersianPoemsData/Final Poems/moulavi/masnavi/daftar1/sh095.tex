\begin{center}
\section*{بخش ۹۵ - مضرت تعظیم خلق و انگشت‌نمای شدن}
\label{sec:sh095}
\addcontentsline{toc}{section}{\nameref{sec:sh095}}
\begin{longtable}{l p{0.5cm} r}
تن قفس‌شکلست تن شد خار جان
&&
در فریب داخلان و خارجان
\\
اینش گوید من شوم همراز تو
&&
وآنش گوید نی منم انباز تو
\\
اینش گوید نیست چون تو در وجود
&&
در جمال و فضل و در احسان و جود
\\
آنش گوید هر دو عالم آن تست
&&
جمله جانهامان طفیل جان تست
\\
او چو بیند خلق را سرمست خویش
&&
از تکبر می‌رود از دست خویش
\\
او نداند که هزاران را چو او
&&
دیو افکندست اندر آب جو
\\
لطف و سالوس جهان خوش لقمه‌ایست
&&
کمترش خور کان پر آتش لقمه‌ایست
\\
آتشش پنهان و ذوقش آشکار
&&
دود او ظاهر شود پایان کار
\\
تو مگو آن مدح را من کی خورم
&&
از طمع می‌گوید او پی می‌برم
\\
مادحت گر هجو گوید بر ملا
&&
روزها سوزد دلت زان سوزها
\\
گر چه دانی کو ز حرمان گفت آن
&&
کان طمع که داشت از تو شد زیان
\\
آن اثر می‌ماندت در اندرون
&&
در مدیح این حالتت هست آزمون
\\
آن اثر هم روزها باقی بود
&&
مایهٔ کبر و خداع جان شود
\\
لیک ننماید چو شیرینست مدح
&&
بد نماید زانک تلخ افتاد قدح
\\
همچو مطبوخست و حب کان را خوری
&&
تا بدیری شورش و رنج اندری
\\
ور خوری حلوا بود ذوقش دمی
&&
این اثر چون آن نمی‌پاید همی
\\
چون نمی‌پاید همی‌پاید نهان
&&
هر ضدی را تو به ضد او بدان
\\
چون شکر پاید نهان تاثیر او
&&
بعد حینی دمل آرد نیش‌جو
\\
نفس از بس مدحها فرعون شد
&&
کن ذلیل النفس هونا لا تسد
\\
تا توانی بنده شو سلطان مباش
&&
زخم کش چون گوی شو چوگان مباش
\\
ورنه چون لطفت نماند وین جمال
&&
از تو آید آن حریفان را ملال
\\
آن جماعت کت همی‌دادند ریو
&&
چون ببینندت بگویندت که دیو
\\
جمله گویندت چو بینندت بدر
&&
مرده‌ای از گور خود بر کرد سر
\\
همچو امرد که خدا نامش کنند
&&
تا بدین سالوس در دامش کنند
\\
چونک در بدنامی آمد ریش او
&&
دیو را ننگ آید از تفتیش او
\\
دیو سوی آدمی شد بهر شر
&&
سوی تو ناید که از دیوی بتر
\\
تا تو بودی آدمی دیو از پیت
&&
می‌دوید و می‌چشانید او میت
\\
چون شدی در خوی دیوی استوار
&&
می‌گریزد از تو دیو نابکار
\\
آنک اندر دامنت آویخت او
&&
چون چنین گشتی ز تو بگریخت او
\\
\end{longtable}
\end{center}
