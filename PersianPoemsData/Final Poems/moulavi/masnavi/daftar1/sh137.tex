\begin{center}
\section*{بخش ۱۳۷ - حکایت ماجرای نحوی و کشتیبان}
\label{sec:sh137}
\addcontentsline{toc}{section}{\nameref{sec:sh137}}
\begin{longtable}{l p{0.5cm} r}
آن یکی نحوی به کشتی در نشست
&&
رو به کشتیبان نهاد آن خودپرست
\\
گفت هیچ از نحو خواندی گفت لا
&&
گفت نیم عمر تو شد در فنا
\\
دل‌شکسته گشت کشتیبان ز تاب
&&
لیک آن دم کرد خامش از جواب
\\
باد کشتی را به گردابی فکند
&&
گفت کشتیبان بدان نحوی بلند
\\
هیچ دانی آشنا کردن بگو
&&
گفت نی ای خوش‌جواب خوب‌رو
\\
گفت کل عمرت ای نحوی فناست
&&
زانک کشتی غرق این گردابهاست
\\
محو می‌باید نه نحو اینجا بدان
&&
گر تو محوی بی‌خطر در آب ران
\\
آب دریا مرده را بر سر نهد
&&
ور بود زنده ز دریا کی رهد
\\
چون بمردی تو ز اوصاف بشر
&&
بحر اسرارت نهد بر فرق سر
\\
ای که خلقان را تو خر می‌خوانده‌ای
&&
این زمان چون خر برین یخ مانده‌ای
\\
گر تو علامه زمانی در جهان
&&
نک فنای این جهان بین وین زمان
\\
مرد نحوی را از آن در دوختیم
&&
تا شما را نحو محو آموختیم
\\
فقه فقه و نحو نحو و صرف صرف
&&
در کم آمد یابی ای یار شگرف
\\
آن سبوی آب دانشهای ماست
&&
وان خلیفه دجلهٔ علم خداست
\\
ما سبوها پر به دجله می‌بریم
&&
گرنه خر دانیم خود را ما خریم
\\
باری اعرابی بدان معذور بود
&&
کو ز دجله غافل و بس دور بود
\\
گر ز دجله با خبر بودی چو ما
&&
او نبردی آن سبو را جا بجا
\\
بلک از دجله چو واقف آمدی
&&
آن سبو را بر سر سنگی زدی
\\
\end{longtable}
\end{center}
