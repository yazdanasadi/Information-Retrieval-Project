\begin{center}
\section*{بخش ۱۴۵ - ادب کردن شیر گرگ را کی در قسمت بی‌ادبی کرده بود}
\label{sec:sh145}
\addcontentsline{toc}{section}{\nameref{sec:sh145}}
\begin{longtable}{l p{0.5cm} r}
گرگ را بر کند سر آن سرفراز
&&
تا نماند دوسری و امتیاز
\\
فانتقمنا منهم است ای گرگ پیر
&&
چون نبودی مرده در پیش امیر
\\
بعد از آن رو شیر با روباه کرد
&&
گفت این را بخش کن از بهر خورد
\\
سجده کرد و گفت کین گاو سمین
&&
چاشت‌خوردت باشد ای شاه گزین
\\
وان بز از بهر میان روز را
&&
یخنیی باشد شه پیروز را
\\
و آن دگر خرگوش بهر شام هم
&&
شب‌چرهٔ این شاه با لطف و کرم
\\
گفت ای روبه تو عدل افروختی
&&
این چنین قسمت ز کی آموختی
\\
از کجا آموختی این ای بزرگ
&&
گفت ای شاه جهان از حال گرگ
\\
گفت چون در عشق ما گشتی گرو
&&
هر سه را بر گیر و بستان و برو
\\
روبها چون جملگی ما را شدی
&&
چونت آزاریم چون تو ما شدی
\\
ما ترا و جمله اشکاران ترا
&&
پای بر گردون هفتم نه بر آ
\\
چون گرفتی عبرت از گرگ دنی
&&
پس تو روبه نیستی شیر منی
\\
عاقل آن باشد که عبرت گیرد از
&&
مرگ یاران در بلای محترز
\\
روبه آن دم بر زبان صد شکر راند
&&
که مرا شیر از پی آن گرگ خواند
\\
گر مرا اول بفرمودی که تو
&&
بخش کن این را که بردی جان ازو
\\
پس سپاس او را که ما را در جهان
&&
کرد پیدا از پس پیشینیان
\\
تا شنیدیم آن سیاستهای حق
&&
بر قرون ماضیه اندر سبق
\\
تا که ما از حال آن گرگان پیش
&&
همچو روبه پاس خود داریم بیش
\\
امت مرحومه زین رو خواندمان
&&
آن رسول حق و صادق در بیان
\\
استخوان و پشم آن گرگان عیان
&&
بنگرید و پند گیرید ای مهان
\\
عاقل از سر بنهد این هستی و باد
&&
چون شنید انجام فرعونان و عاد
\\
ور بننهد دیگران از حال او
&&
عبرتی گیرند از اضلال او
\\
\end{longtable}
\end{center}
