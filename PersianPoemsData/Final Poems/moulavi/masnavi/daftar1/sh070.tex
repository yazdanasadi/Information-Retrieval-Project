\begin{center}
\section*{بخش ۷۰ - پا واپس کشیدن خرگوش از شیر چون نزدیک چاه رسید}
\label{sec:sh070}
\addcontentsline{toc}{section}{\nameref{sec:sh070}}
\begin{longtable}{l p{0.5cm} r}
چونک نزد چاه آمد شیر دید
&&
کز ره آن خرگوش ماند و پا کشید
\\
گفت پا واپس کشیدی تو چرا
&&
پای را واپس مکش پیش اندر آ
\\
گفت کو پایم که دست و پای رفت
&&
جان من لرزید و دل از جای رفت
\\
رنگ رویم را نمی‌بینی چو زر
&&
ز اندرون خود می‌دهد رنگم خبر
\\
حق چو سیما را معرف خوانده‌ست
&&
چشم عارف سوی سیما مانده‌ست
\\
رنگ و بو غماز آمد چون جرس
&&
از فرس آگه کند بانگ فرس
\\
بانگ هر چیزی رساند زو خبر
&&
تا بدانی بانگ خر از بانگ در
\\
گفت پیغامبر به تمییز کسان
&&
مرء مخفی لدی طی‌اللسان
\\
رنگ رو از حال دل دارد نشان
&&
رحمتم کن مهر من در دل نشان
\\
رنگ روی سرخ دارد بانگ شکر
&&
بانگ روی زرد دارد صبر و نکر
\\
در من آمد آنک دست و پا برد
&&
رنگ رو و قوت و سیما برد
\\
آنک در هر چه در آید بشکند
&&
هر درخت از بیخ و بن او بر کند
\\
در من آمد آنک از وی گشت مات
&&
آدمی و جانور جامد نبات
\\
این خود اجزا اند کلیات ازو
&&
زرد کرده رنگ و فاسد کرده بو
\\
تا جهان گه صابرست و گه شکور
&&
بوستان گه حله پوشد گاه عور
\\
آفتابی کو بر آید نارگون
&&
ساعتی دیگر شود او سرنگون
\\
اختران تافته بر چار طاق
&&
لحظه لحظه مبتلای احتراق
\\
ماه کو افزود ز اختر در جمال
&&
شد ز رنج دق او همچون خیال
\\
این زمین با سکون با ادب
&&
اندر آرد زلزله‌ش در لرز تب
\\
ای بسا که زین بلای مر دریگ
&&
گشته است اندر جهان او خرد و ریگ
\\
این هوا با روح آمد مقترن
&&
چون قضا آید وبا گشت و عفن
\\
آب خوش کو روح را همشیره شد
&&
در غدیری زرد و تلخ و تیره شد
\\
آتشی کو باد دارد در بروت
&&
هم یکی بادی برو خواند یموت
\\
حال دریا ز اضطراب و جوش او
&&
فهم کن تبدیلهای هوش او
\\
چرخ سرگردان که اندر جست و جوست
&&
حال او چون حال فرزندان اوست
\\
گه حضیض و گه میانه گاه اوج
&&
اندرو از سعد و نحسی فوج فوج
\\
از خود ای جزوی ز کلها مختلط
&&
فهم می‌کن حالت هر منبسط
\\
چونک کلیات را رنجست و درد
&&
جزو ایشان چون نباشد روی‌زرد
\\
خاصه جزوی کو ز اضدادست جمع
&&
ز آب و خاک و آتش و بادست جمع
\\
این عجب نبود که میش از گرگ جست
&&
این عجب کین میش دل در گرگ بست
\\
زندگانی آشتی ضدهاست
&&
مرگ آنک اندر میانش جنگ خاست
\\
لطف حق این شیر را و گور را
&&
الف دادست این دو ضد دور را
\\
چون جهان رنجور و زندانی بود
&&
چه عجب رنجور اگر فانی بود
\\
خواند بر شیر او ازین رو پندها
&&
گفت من پس مانده‌ام زین بندها
\\
\end{longtable}
\end{center}
