\begin{center}
\section*{بخش ۴۶ - ترجیح نهادن نخچیران توکل را بر اجتهاد}
\label{sec:sh046}
\addcontentsline{toc}{section}{\nameref{sec:sh046}}
\begin{longtable}{l p{0.5cm} r}
قوم گفتندش که کسب از ضعف خلق
&&
لقمهٔ تزویر دان بر قدر حلق
\\
نیست کسبی از توکل خوب‌تر
&&
چیست از تسلیم خود محبوب‌تر
\\
بس گریزند از بلا سوی بلا
&&
بس جهند از مار سوی اژدها
\\
حیله کرد انسان و حیله‌ش دام بود
&&
آنک جان پنداشت خون‌آشام بود
\\
در ببست و دشمن اندر خانه بود
&&
حیلهٔ فرعون زین افسانه بود
\\
صد هزاران طفل کشت آن کینه‌کش
&&
وانک او می‌جست اندر خانه‌اش
\\
دیدهٔ ما چون بسی علت دروست
&&
رو فنا کن دید خود در دید دوست
\\
دید ما را دید او نعم العوض
&&
یابی اندر دید او کل غرض
\\
طفل تا گیرا و تا پویا نبود
&&
مرکبش جز گردن بابا نبود
\\
چون فضولی گشت و دست و پا نمود
&&
در عنا افتاد و در کور و کبود
\\
جانهای خلق پیش از دست و پا
&&
می‌پریدند از وفا اندر صفا
\\
چون بامر اهبطوا بندی شدند
&&
حبس خشم و حرص و خرسندی شدند
\\
ما عیال حضرتیم و شیرخواه
&&
گفت الخلق عیال للاله
\\
آنک او از آسمان باران دهد
&&
هم تواند کو ز رحمت نان دهد
\\
\end{longtable}
\end{center}
