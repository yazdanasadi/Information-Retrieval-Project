\begin{center}
\section*{بخش ۱۱۵ - نصحیت کردن زن مر شوی را کی سخن افزون از قدم و از مقام خود مگو لم تقولون ما لا تفعلون کی این سخنها اگرچه راستست این مقام توکل ترا نیست و این سخن گفتن فوق مقام و معاملهٔ خود زیان دارد و کبر مقتا عند الله باشد}
\label{sec:sh115}
\addcontentsline{toc}{section}{\nameref{sec:sh115}}
\begin{longtable}{l p{0.5cm} r}
زن برو زد بانگ کای ناموس‌کیش
&&
من فسون تو نخواهم خورد بیش
\\
ترهات از دعوی و دعوت مگو
&&
رو سخن از کبر و از نخوت مگو
\\
چند حرف طمطراق و کار بار
&&
کار و حال خود ببین و شرم‌دار
\\
کبر زشت و از گدایان زشت‌تر
&&
روز سرد و برف وانگه جامه تر
\\
چند دعوی و دم و باد و بروت
&&
ای ترا خانه چو بیت العنکبوت
\\
از قناعت کی تو جان افروختی
&&
از قناعتها تو نام آموختی
\\
گفت پیغامبر قناعت چیست گنج
&&
گنج را تو وا نمی‌دانی ز رنج
\\
این قناعت نیست جز گنج روان
&&
تو مزن لاف ای غم و رنج روان
\\
تو مخوانم جفت کمتر زن بغل
&&
جفت انصافم نیم جفت دغل
\\
چون قدم با میر و با بگ می‌زنی
&&
چون ملخ را در هوا رگ می‌زنی
\\
با سگان زین استخوان در چالشی
&&
چون نی اشکم تهی در نالشی
\\
سوی من منگر بخواری سست سست
&&
تا نگویم آنچ در رگهای تست
\\
عقل خود را از من افزون دیده‌ای
&&
مر من کم‌عقل را چون دیده‌ای
\\
همچو گرگ غافل اندر ما مجه
&&
ای ز ننگ عقل تو بی‌عقل به
\\
چونک عقل تو عقیلهٔ مردمست
&&
آن نه عقلست آن که مار و کزدمست
\\
خصم ظلم و مکر تو الله باد
&&
فضل و عقل تو ز ما کوتاه باد
\\
هم تو ماری هم فسون‌گر این عجب
&&
مارگیر و ماری ای ننگ عرب
\\
زاغ اگر زشتی خود بشناختی
&&
همچو برف از درد و غم بگداختی
\\
مرد افسونگر بخواند چون عدو
&&
او فسون بر مار و مار افسون برو
\\
گر نبودی دام او افسون مار
&&
کی فسون مار را گشتی شکار
\\
مرد افسون‌گر ز حرص کسب و کار
&&
در نیابد آن زمان افسون مار
\\
مار گوید ای فسون‌گر هین و هین
&&
آن خود دیدی فسون من ببین
\\
تو به نام حق فریبی مر مرا
&&
تا کنی رسوای شور و شر مرا
\\
نام حقم بست نی آن رای تو
&&
نام حق را دام کردی وای تو
\\
نام حق بستاند از تو داد من
&&
من به نام حق سپردم جان و تن
\\
یا به زخم من رگ جانت برد
&&
یا که همچون من به زندانت برد
\\
زن ازین گونه خشن گفتارها
&&
خواند بر شوی جوان طومارها
\\
\end{longtable}
\end{center}
