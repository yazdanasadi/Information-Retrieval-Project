\begin{center}
\section*{بخش ۲۲ - تخلیط وزیر در احکام انجیل}
\label{sec:sh022}
\addcontentsline{toc}{section}{\nameref{sec:sh022}}
\begin{longtable}{l p{0.5cm} r}
ساخت طوماری به نام هر یکی
&&
نقش هر طومار دیگر مسلکی
\\
حکمهای هر یکی نوعی دگر
&&
این خلاف آن ز پایان تا به سر
\\
در یکی راه ریاضت را و جوع
&&
رکن توبه کرده و شرط رجوع
\\
در یکی گفته ریاضت سود نیست
&&
اندرین ره مخلصی جز جود نیست
\\
در یکی گفته که جوع و جود تو
&&
شرک باشد از تو با معبود تو
\\
جز توکل جز که تسلیم تمام
&&
در غم و راحت همه مکرست و دام
\\
در یکی گفته که واجب خدمتست
&&
ور نه اندیشهٔ توکل تهمتست
\\
در یکی گفته که امر و نهیهاست
&&
بهر کردن نیست شرح عجز ماست
\\
تا که عجز خود بینیم اندر آن
&&
قدرت او را بدانیم آن زمان
\\
در یکی گفته که عجز خود مبین
&&
کفر نعمت کردنست آن عجز هین
\\
قدرت خود بین که این قدرت ازوست
&&
قدرت تو نعمت او دان که هوست
\\
در یکی گفته کزین دو بر گذر
&&
بت بود هر چه بگنجد در نظر
\\
در یکی گفته مکش این شمع را
&&
کین نظر چون شمع آمد جمع را
\\
از نظر چون بگذری و از خیال
&&
کشته باشی نیم شب شمع وصال
\\
در یکی گفته بکش باکی مدار
&&
تا عوض بینی نظر را صد هزار
\\
که ز کشتن شمع جان افزون شود
&&
لیلی‌ات از صبر تو مجنون شود
\\
ترک دنیا هر که کرد از زهد خویش
&&
بیش آید پیش او دنیا و بیش
\\
در یکی گفته که آنچت داد حق
&&
بر تو شیرین کرد در ایجاد حق
\\
بر تو آسان کرد و خوش آن را بگیر
&&
خویشتن را در میفکن در زحیر
\\
در یکی گفته که بگذار آن خود
&&
کان قبول طبع تو ردست و بد
\\
راههای مختلف آسان شدست
&&
هر یکی را ملتی چون جان شدست
\\
گر میسر کردن حق ره بدی
&&
هر جهود و گبر ازو آگه بدی
\\
در یکی گفته میسر آن بود
&&
که حیات دل غذای جان بود
\\
هر چه ذوق طبع باشد چون گذشت
&&
بر نه آرد همچو شوره ریع و کشت
\\
جز پشیمانی نباشد ریع او
&&
جز خسارت پیش نارد بیع او
\\
آن میسر نبود اندر عاقبت
&&
نام او باشد معسر عاقبت
\\
تو معسر از میسر بازدان
&&
عاقبت بنگر جمال این و آن
\\
در یکی گفته که استادی طلب
&&
عاقبت‌بینی نیابی در حسب
\\
عاقبت دیدند هر گون ملتی
&&
لاجرم گشتند اسیر زلتی
\\
عاقبت دیدن نباشد دست‌باف
&&
ورنه کی بودی ز دینها اختلاف
\\
در یکی گفته که استا هم توی
&&
زانک استا را شناسا هم توی
\\
مرد باش و سخرهٔ مردان مشو
&&
رو سر خود گیر و سرگردان مشو
\\
در یکی گفته که این جمله یکیست
&&
هر که او دو بیند احول مردکیست
\\
در یکی گفته که صد یک چون بود
&&
این کی اندیشد مگر مجنون بود
\\
هر یکی قولیست ضد هم‌دگر
&&
چون یکی باشد یکی زهر و شکر
\\
تا ز زهر و از شکر در نگذری
&&
کی تو از گلزار وحدت بو بری
\\
این نمط وین نوع ده طومار و دو
&&
بر نوشت آن دین عیسی را عدو
\\
\end{longtable}
\end{center}
