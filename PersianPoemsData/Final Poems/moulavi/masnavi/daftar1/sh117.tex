\begin{center}
\section*{بخش ۱۱۷ - در بیان آنک جنبیدن هر کسی از آنجا کی ویست هر کس را از چنبرهٔ وجود خود بیند تابهٔ کبود آفتاب را کبود نماید و سرخ سرخ نماید چون تابه‌ها از رنگها بیرون آید سپید شود از همه تابه‌های دیگر او راست‌گوتر باشد و امام باشد}
\label{sec:sh117}
\addcontentsline{toc}{section}{\nameref{sec:sh117}}
\begin{longtable}{l p{0.5cm} r}
دید احمد را ابوجهل و بگفت
&&
زشت نقشی کز بنی‌هاشم شکفت
\\
گفت احمد مر ورا که راستی
&&
راست گفتی گرچه کار افزاستی
\\
دید صدیقش بگفت ای آفتاب
&&
نی ز شرقی نی ز غربی خوش بتاب
\\
گفت احمد راست گفتی ای عزیز
&&
ای رهیده تو ز دنیای نه چیز
\\
حاضران گفتند ای صدر الوری
&&
راست‌گو گفتی دو ضدگو را چرا
\\
گفت من آیینه‌ام مصقول دست
&&
ترک و هندو در من آن بیند که هست
\\
ای زن ار طماع می‌بینی مرا
&&
زین تحری زنانه برتر آ
\\
آن طمع را ماند و رحمت بود
&&
کو طمع آنجا که آن نعمت بود
\\
امتحان کن فقر را روزی دو تو
&&
تا به فقر اندر غنا بینی دوتو
\\
صبر کن با فقر و بگذار این ملال
&&
زانک در فقرست عز ذوالجلال
\\
سرکه مفروش و هزاران جان ببین
&&
از قناعت غرق بحر انگبین
\\
صد هزاران جان تلخی‌کش نگر
&&
همچو گل آغشته اندر گلشکر
\\
ای دریغا مر ترا گنجا بدی
&&
تا ز جانم شرح دل پیدا شدی
\\
این سخن شیرست در پستان جان
&&
بی کشنده خوش نمی‌گردد روان
\\
مستمع چون تشنه و جوینده شد
&&
واعظ ار مرده بود گوینده شد
\\
مستمع چون تازه آمد بی‌ملال
&&
صدزبان گردد به گفتن گنگ و لال
\\
چونک نامحرم در آید از درم
&&
پرده در پنهان شوند اهل حرم
\\
ور در آید محرمی دور از گزند
&&
برگشایند آن ستیران روی‌بند
\\
هرچه را خوب و خوش و زیبا کنند
&&
از برای دیدهٔ بینا کنند
\\
کی بود آواز چنگ و زیر و بم
&&
از برای گوش بی‌حس اصم
\\
مشک را بیهوده حق خوش‌دم نکرد
&&
بهر حس کرد و پی اخشم نکرد
\\
حق زمین و آسمان بر ساخته‌ست
&&
در میان بس نار و نور افراخته‌ست
\\
این زمین را از برای خاکیان
&&
آسمان را مسکن افلاکیان
\\
مرد سفلی دشمن بالا بود
&&
مشتری هر مکان پیدا بود
\\
ای ستیره هیچ تو بر خاستی
&&
خویشتن را بهر کور آراستی
\\
گر جهان را پر در مکنون کنم
&&
روزی تو چون نباشد چون کنم
\\
ترک جنگ و ره‌زنی ای زن بگو
&&
ور نمی‌گویی به ترک من بگو
\\
مر مرا چه جای جنگ نیک و بد
&&
کین دلم از صلحها هم می‌رمد
\\
گر خمش گردی و گر نه آن کنم
&&
که همین دم ترک خان و مان کنم
\\
\end{longtable}
\end{center}
