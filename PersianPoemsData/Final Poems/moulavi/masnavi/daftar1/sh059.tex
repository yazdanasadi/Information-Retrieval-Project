\begin{center}
\section*{بخش ۵۹ - قصهٔ مکر خرگوش}
\label{sec:sh059}
\addcontentsline{toc}{section}{\nameref{sec:sh059}}
\begin{longtable}{l p{0.5cm} r}
ساعتی تاخیر کرد اندر شدن
&&
بعد از آن شد پیش شیر پنجه‌زن
\\
زان سبب کاندر شدن او ماند دیر
&&
خاک را می‌کند و می‌غرید شیر
\\
گفت من گفتم که عهد آن خسان
&&
خام باشد خام و سست و نارسان
\\
دمدمهٔ ایشان مرا از خر فکند
&&
چند بفریبد مرا این دهر چند
\\
سخت در ماند امیر سست ریش
&&
چون نه پس بیند نه پیش از احمقیش
\\
راه هموارست زیرش دامها
&&
قحط معنی درمیان نامها
\\
لفظها و نامها چون دامهاست
&&
لفظ شیرین ریگ آب عمر ماست
\\
آن یکی ریگی که جوشد آب ازو
&&
سخت کم‌یابست رو آن را بجو
\\
منبع حکمت شود حکمت‌طلب
&&
فارغ آید او ز تحصیل و سبب
\\
لوح حافظ لوح محفوظی شود
&&
عقل او از روح محظوظی شود
\\
چون معلم بود عقلش ز ابتدا
&&
بعد ازین شد عقل شاگردی ورا
\\
عقل چون جبریل گوید احمدا
&&
گر یکی گامی نهم سوزد مرا
\\
تو مرا بگذار زین پس پیش ران
&&
حد من این بود ای سلطان جان
\\
هر که ماند از کاهلی بی‌شکر و صبر
&&
او همین داند که گیرد پای جبر
\\
هر که جبر آورد خود رنجور کرد
&&
تا همان رنجوریش در گور کرد
\\
گفت پیغمبر که رنجوری بلاغ
&&
رنج آرد تا بمیرد چون چراغ
\\
جبر چه بود بستن اشکسته را
&&
یا بپیوستن رگی بگسسته را
\\
چون درین ره پای خود نشکسته‌ای
&&
بر کی می‌خندی چه پا را بسته‌ای
\\
وانک پایش در ره کوشش شکست
&&
در رسید او را براق و بر نشست
\\
حامل دین بود او محمول شد
&&
قابل فرمان بد او مقبول شد
\\
تاکنون فرمان پذیرفتی ز شاه
&&
بعد ازین فرمان رساند بر سپاه
\\
تاکنون اختر اثر کردی درو
&&
بعد ازین باشد امیر اختر او
\\
گر ترا اشکال آید در نظر
&&
پس تو شک داری در انشق القمر
\\
تازه کن ایمان نی از گفت زبان
&&
ای هوا را تازه کرده در نهان
\\
تا هوا تازه‌ست ایمان تازه نیست
&&
کین هوا جز قفل آن دروازه نیست
\\
کرده‌ای تاویل حرف بکر را
&&
خویش را تاویل کن نه ذکر را
\\
بر هوا تاویل قرآن می‌کنی
&&
پست و کژ شد از تو معنی سنی
\\
\end{longtable}
\end{center}
