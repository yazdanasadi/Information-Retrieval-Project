\begin{center}
\section*{بخش ۷۳ - مژده بردن خرگوش سوی نخچیران کی شیر در چاه فتاد}
\label{sec:sh073}
\addcontentsline{toc}{section}{\nameref{sec:sh073}}
\begin{longtable}{l p{0.5cm} r}
چونک خرگوش از رهایی شاد گشت
&&
سوی نخچیران دوان شد تا به دشت
\\
شیر را چون دید در چه کشته زار
&&
چرخ می‌زد شادمان تا مرغزار
\\
دست می‌زد چون رهید از دست مرگ
&&
سبز و رقصان در هوا چون شاخ و برگ
\\
شاخ و برگ از حبس خاک آزاد شد
&&
سر برآورد و حریف باد شد
\\
برگها چون شاخ را بکشافتند
&&
تا به بالای درخت اشتافتند
\\
با زبان شطاه شکر خدا
&&
می‌سراید هر بر و برگی جدا
\\
که بپرورد اصل ما را ذوالعطا
&&
تا درخت استغلظ آمد و استوی
\\
جانهای بسته اندر آب و گل
&&
چون رهند از آب و گلها شاددل
\\
در هوای عشق حق رقصان شوند
&&
همچو قرص بدر بی‌نقصان شوند
\\
چشمان در رقص و جانها خود مپرس
&&
وانک گرد جان از آنها خود مپرس
\\
شیر را خرگوش در زندان نشاند
&&
ننگ شیری کو ز خرگوشی بماند
\\
درچنان ننگی و آنگه این عجب
&&
فخر دین خواهد که گویندش لقب
\\
ای تو شیری در تک این چاه فرد
&&
نقش چون خرگوش خونت‌ریخت و خورد
\\
نفس خرگوشت به صحرا در چرا
&&
تو بقعر این چه چون و چرا
\\
سوی نخچیران دوید آن شیرگیر
&&
کابشروا یا قوم اذ جاء البشیر
\\
مژده مژده ای گروه عیش‌ساز
&&
کان سگ دوزخ به دوزخ رفت باز
\\
مژده مژده کان عدو جانها
&&
کند قهر خالقش دندانها
\\
آنک از پنجه بسی سرها بکوفت
&&
همچو خس جاروب مرگش هم بروفت
\\
\end{longtable}
\end{center}
