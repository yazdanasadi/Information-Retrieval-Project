\begin{center}
\section*{بخش ۱۲ - داستان آن پادشاه جهود کی نصرانیان را می‌کشت از بهر تعصب}
\label{sec:sh012}
\addcontentsline{toc}{section}{\nameref{sec:sh012}}
\begin{longtable}{l p{0.5cm} r}
بود شاهی در جهودان ظلم‌ساز
&&
دشمن عیسی و نصرانی گداز
\\
عهد عیسی بود و نوبت آن او
&&
جان موسی او و موسی جان او
\\
شاه احول کرد در راه خدا
&&
آن دو دمساز خدایی را جدا
\\
گفت استاد احولی را کاندر آ
&&
زو برون آر از وثاق آن شیشه را
\\
گفت احول زان دو شیشه من کدام
&&
پیش تو آرم بکن شرح تمام
\\
گفت استاد آن دو شیشه نیست رو
&&
احولی بگذار و افزون‌بین مشو
\\
گفت ای استا مرا طعنه مزن
&&
گفت استا زان دو یک را در شکن
\\
چون یک بشکست هر دو شد ز چشم
&&
مرد احول گردد از میلان و خشم
\\
شیشه یک بود و به چشمش دو نمود
&&
چون شکست او شیشه را دیگر نبود
\\
خشم و شهوت مرد را احول کند
&&
ز استقامت روح را مبدل کند
\\
چون غرض آمد هنر پوشیده شد
&&
صد حجاب از دل به سوی دیده شد
\\
چون دهد قاضی به دل رشوت قرار
&&
کی شناسد ظالم از مظلوم زار
\\
شاه از حقد جهودانه چنان
&&
گشت احول کالامان یا رب امان
\\
صد هزاران مؤمن مظلوم کشت
&&
که پناهم دین موسی را و پشت
\\
\end{longtable}
\end{center}
