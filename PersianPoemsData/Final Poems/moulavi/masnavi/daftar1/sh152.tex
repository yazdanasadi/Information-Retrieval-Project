\begin{center}
\section*{بخش ۱۵۲ - اعتماد کردن هاروت و ماروت بر عصمت خویش و امیری اهل دنیا خواستن و در فتنه افتادن}
\label{sec:sh152}
\addcontentsline{toc}{section}{\nameref{sec:sh152}}
\begin{longtable}{l p{0.5cm} r}
همچو هاروت و چو ماروت شهیر
&&
از بطر خوردند زهرآلود تیر
\\
اعتمادی بودشان بر قدس خویش
&&
چیست بر شیر اعتماد گاومیش
\\
گرچه او با شاخ صد چاره کند
&&
شاخ شاخش شیر نر پاره کند
\\
گر شود پر شاخ همچون خارپشت
&&
شیر خواهد گاو را ناچار کشت
\\
گرچه صرصر پس درختان می‌کند
&&
با گیاه تر وی احسان می‌کند
\\
بر ضعیفی گیاه آن باد تند
&&
رحم کرد ای دل تو از قوت ملند
\\
تیشه را ز انبوهی شاخ درخت
&&
کی هراس آید ببرد لخت لخت
\\
لیک بر برگی نکوبد خویش را
&&
جز که بر نیشی نکوبد نیش را
\\
شعله را ز انبوهی هیزم چه غم
&&
کی رمد قصاب از خیل غنم
\\
پیش معنی چیست صورت بس زبون
&&
چرخ را معنیش می‌دارد نگون
\\
تو قیاس از چرخ دولابی بگیر
&&
گردشش از کیست از عقل مشیر
\\
گردش این قالب همچون سپر
&&
هست از روح مستر ای پسر
\\
گردش این باد از معنی اوست
&&
همچو چرخی کان اسیر آب جوست
\\
جر و مد و دخل و خرج این نفس
&&
از کی باشد جز ز جان پر هوس
\\
گاه جیمش می‌کند گه حا و دال
&&
گاه صلحش می‌کند گاهی جدال
\\
گه یمینش می‌برد گاهی یسار
&&
که گلستانش کند گاهیش خار
\\
همچنین این باد را یزدان ما
&&
کرده بد بر عاد همچون اژدها
\\
باز هم آن باد را بر مؤمنان
&&
کرده بد صلح و مراعات و امان
\\
گفت المعنی هوالله شیخ دین
&&
بحر معنیهای رب العالمین
\\
جمله اطباق زمین و آسمان
&&
همچو خاشاکی در آن بحر روان
\\
حمله‌ها و رقص خاشاک اندر آب
&&
هم ز آب آمد به وقت اضطراب
\\
چونک ساکن خواهدش کرد از مرا
&&
سوی ساحل افکند خاشاک را
\\
چون کشد از ساحلش در موج‌گاه
&&
آن کند با او که آتش با گیاه
\\
این حدیث آخر ندارد باز ران
&&
جانب هاروت و ماروت ای جوان
\\
\end{longtable}
\end{center}
