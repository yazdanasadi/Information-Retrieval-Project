\begin{center}
\section*{بخش ۲۵ - مکر دیگر انگیختن وزیر در اضلال قوم}
\label{sec:sh025}
\addcontentsline{toc}{section}{\nameref{sec:sh025}}
\begin{longtable}{l p{0.5cm} r}
مکر دیگر آن وزیر از خود ببست
&&
وعظ را بگذاشت و در خلوت نشست
\\
در مریدان در فکند از شوق سوز
&&
بود در خلوت چهل پنجاه روز
\\
خلق دیوانه شدند از شوق او
&&
از فراق حال و قال و ذوق او
\\
لابه و زاری همی کردند و او
&&
از ریاضت گشته در خلوت دوتو
\\
گفته ایشان نیست ما را بی تو نور
&&
بی عصاکش چون بود احوال کور
\\
از سر اکرام و از بهر خدا
&&
بیش ازین ما را مدار از خود جدا
\\
ما چو طفلانیم و ما را دایه تو
&&
بر سر ما گستران آن سایه تو
\\
گفت جانم از محبان دور نیست
&&
لیک بیرون آمدن دستور نیست
\\
آن امیران در شفاعت آمدند
&&
وان مریدان در شناعت آمدند
\\
کین چه بدبختیست ما را ای کریم
&&
از دل و دین مانده ما بی تو یتیم
\\
تو بهانه می‌کنی و ما ز درد
&&
می‌زنیم از سوز دل دمهای سرد
\\
ما به گفتار خوشت خو کرده‌ایم
&&
ما ز شیر حکمت تو خورده‌ایم
\\
الله الله این جفا با ما مکن
&&
خیر کن امروز را فردا مکن
\\
می‌دهد دل مر ترا کین بی‌دلان
&&
بی تو گردند آخر از بی‌حاصلان
\\
جمله در خشکی چو ماهی می‌طپند
&&
آب را بگشا ز جو بر دار بند
\\
ای که چون تو در زمانه نیست کس
&&
الله الله خلق را فریاد رس
\\
\end{longtable}
\end{center}
