\begin{center}
\section*{بخش ۱۷ - قصهٔ دیدن خلیفه لیلی را}
\label{sec:sh017}
\addcontentsline{toc}{section}{\nameref{sec:sh017}}
\begin{longtable}{l p{0.5cm} r}
گفت لیلی را خلیفه کان توی
&&
کز تو مجنون شد پریشان و غوی
\\
از دگر خوبان تو افزون نیستی
&&
گفت خامش چون تو مجنون نیستی
\\
هر که بیدارست او در خواب‌تر
&&
هست بیداریش از خوابش بتر
\\
چون بحق بیدار نبود جان ما
&&
هست بیداری چو در بندان ما
\\
جان همه روز از لگدکوب خیال
&&
وز زیان و سود وز خوف زوال
\\
نی صفا می‌ماندش نی لطف و فر
&&
نی بسوی آسمان راه سفر
\\
خفته آن باشد که او از هر خیال
&&
دارد اومید و کند با او مقال
\\
دیو را چون حور بیند او به خواب
&&
پس ز شهوت ریزد او با دیو آب
\\
چونک تخم نسل را در شوره ریخت
&&
او به خویش آمد خیال از وی گریخت
\\
ضعف سر بیند از آن و تن پلید
&&
آه از آن نقش پدید ناپدید
\\
مرغ بر بالا و زیر آن سایه‌اش
&&
می‌دود بر خاک پران مرغ‌وش
\\
ابلهی صیاد آن سایه شود
&&
می‌دود چندانک بی‌مایه شود
\\
بی‌خبر کان عکس آن مرغ هواست
&&
بی‌خبر که اصل آن سایه کجاست
\\
تیر اندازد به سوی سایه او
&&
ترکشش خالی شود از جست و جو
\\
ترکش عمرش تهی شد عمر رفت
&&
از دویدن در شکار سایه تفت
\\
سایهٔ یزدان چو باشد دایه‌اش
&&
وا رهاند از خیال و سایه‌اش
\\
سایهٔ یزدان بود بندهٔ خدا
&&
مرده او زین عالم و زندهٔ خدا
\\
دامن او گیر زوتر بی‌گمان
&&
تا رهی در دامن آخر زمان
\\
کیف مد الظل نقش اولیاست
&&
کو دلیل نور خورشید خداست
\\
اندرین وادی مرو بی این دلیل
&&
لا احب افلین گو چون خلیل
\\
رو ز سایه آفتابی را بیاب
&&
دامن شه شمس تبریزی بتاب
\\
ره ندانی جانب این سور و عرس
&&
از ضیاء الحق حسام الدین بپرس
\\
ور حسد گیرد ترا در ره گلو
&&
در حسد ابلیس را باشد غلو
\\
کو ز آدم ننگ دارد از حسد
&&
با سعادت جنگ دارد از حسد
\\
عقبه‌ای زین صعب‌تر در راه نیست
&&
ای خنک آنکش حسد همراه نیست
\\
این جسد خانهٔ حسد آمد بدان
&&
از حسد آلوده باشد خاندان
\\
گر جسد خانهٔ حسد باشد ولیک
&&
آن جسد را پاک کرد الله نیک
\\
طهرا بیتی بیان پاکیست
&&
گنج نورست ار طلسمش خاکیست
\\
چون کنی بر بی‌حسد مکر و حسد
&&
زان حسد دل را سیاهیها رسد
\\
خاک شو مردان حق را زیر پا
&&
خاک بر سر کن حسد را همچو ما
\\
\end{longtable}
\end{center}
