\begin{center}
\section*{بخش ۴ - از خداوند ولی‌التوفیق در خواستن توفیق رعایت ادب در همه حالها و بیان کردن وخامت ضررهای بی‌ادبی}
\label{sec:sh004}
\addcontentsline{toc}{section}{\nameref{sec:sh004}}
\begin{longtable}{l p{0.5cm} r}
از خدا جوییم توفیق ادب
&&
بی‌ادب محروم گشت از لطف رب
\\
بی‌ادب تنها نه خود را داشت بد
&&
بلک آتش در همه آفاق زد
\\
مایده از آسمان در می‌رسید
&&
بی‌شری و بیع و بی‌گفت و شنید
\\
درمیان قوم موسی چند کس
&&
بی‌ادب گفتند کو سیر و عدس
\\
منقطع شد خوان و نان از آسمان
&&
ماند رنج زرع و بیل و داس‌مان
\\
باز عیسی چون شفاعت کرد حق
&&
خوان فرستاد و غنیمت بر طبق
\\
مائده از آسمان شد عائده
&&
 چون که گفت انزل علینا مائده
\\
باز گستاخان ادب بگذاشتند
&&
چون گدایان زله‌ها برداشتند
\\
لابه کرده عیسی ایشان را که این
&&
دایمست و کم نگردد از زمین
\\
بدگمانی کردن و حرص‌آوری
&&
کفر باشد پیش خوان مهتری
\\
زان گدارویان نادیده ز آز
&&
آن در رحمت بریشان شد فراز
\\
ابر بر ناید پی منع زکات
&&
وز زنا افتد وبا اندر جهات
\\
هر چه بر تو آید از ظلمات و غم
&&
آن ز بی‌باکی و گستاخیست هم
\\
هر که بی‌باکی کند در راه دوست
&&
ره‌زن مردان شد و نامرد اوست
\\
از ادب پرنور گشته‌ست این فلک
&&
وز ادب معصوم و پاک آمد ملک
\\
بد ز گستاخی کسوف آفتاب
&&
شد عزازیلی ز جرات رد باب
\\
\end{longtable}
\end{center}
