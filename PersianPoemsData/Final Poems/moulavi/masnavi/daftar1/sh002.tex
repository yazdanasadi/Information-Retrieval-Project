\begin{center}
\section*{بخش ۲ - عاشق شدن پادشاه بر کنیزک رنجور و تدبیر کردن در صحت او}
\label{sec:sh002}
\addcontentsline{toc}{section}{\nameref{sec:sh002}}
\begin{longtable}{l p{0.5cm} r}
بشنوید ای دوستان این داستان
&&
خود حقیقت نقد حال ماست آن
\\
بود شاهی در زمانی پیش ازین
&&
ملک دنیا بودش و هم ملک دین
\\
اتفاقا شاه روزی شد سوار
&&
با خواص خویش از بهر شکار
\\
یک کنیزک دید شه بر شاه‌راه
&&
شد غلام آن کنیزک پادشاه
\\
مرغ جانش در قفس چون می‌طپید
&&
داد مال و آن کنیزک را خرید
\\
چون خرید او را و برخوردار شد
&&
آن کنیزک از قضا بیمار شد
\\
آن یکی خر داشت و پالانش نبود
&&
یافت پالان گرگ خر را در ربود
\\
کوزه بودش آب می‌نامد بدست
&&
آب را چون یافت خود کوزه شکست
\\
شه طبیبان جمع کرد از چپ و راست
&&
گفت جان هر دو در دست شماست
\\
جان من سهلست جان جانم اوست
&&
دردمند و خسته‌ام درمانم اوست
\\
هر که درمان کرد مر جان مرا
&&
برد گنج و در و مرجان مرا
\\
جمله گفتندش که جانبازی کنیم
&&
فهم گرد آریم و انبازی کنیم
\\
هر یکی از ما مسیح عالمیست
&&
هر الم را در کف ما مرهمیست
\\
گر خدا خواهد نگفتند از بطر
&&
پس خدا بنمودشان عجز بشر
\\
ترک استثنا مرادم قسوتیست
&&
نه همین گفتن که عارض حالتیست
\\
ای بسا ناورده استثنا بگفت
&&
جان او با جان استثناست جفت
\\
هرچه کردند از علاج و از دوا
&&
گشت رنج افزون و حاجت ناروا
\\
آن کنیزک از مرض چون موی شد
&&
چشم شه از اشک خون چون جوی شد
\\
از قضا سرکنگبین صفرا فزود
&&
روغن بادام خشکی می‌نمود
\\
از هلیله قبض شد اطلاق رفت
&&
آب آتش را مدد شد همچو نفت
\\
\end{longtable}
\end{center}
