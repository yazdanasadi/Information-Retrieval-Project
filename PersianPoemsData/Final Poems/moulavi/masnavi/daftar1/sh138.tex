\begin{center}
\section*{بخش ۱۳۸ - قبول کردن خلیفه هدیه را و عطا فرمودن با کمال بی‌نیازی از آن هدیه و از آن سبو}
\label{sec:sh138}
\addcontentsline{toc}{section}{\nameref{sec:sh138}}
\begin{longtable}{l p{0.5cm} r}
چون خلیفه دید و احوالش شنید
&&
آن سبو را پر ز زر کرد و مزید
\\
آن عرب را کرد از فاقه خلاص
&&
داد بخششها و خلعتهای خاص
\\
کین سبو پر زر به دست او دهید
&&
چونک واگردد سوی دجله‌ش برید
\\
از ره خشک آمدست و از سفر
&&
از ره دجله‌ش بود نزدیکتر
\\
چون به کشتی در نشست و دجله دید
&&
سجده می‌کرد از حیا و می‌خمید
\\
کای عجب لطف این شه وهاب را
&&
وان عجب‌تر کو ستد آن آب را
\\
چون پذیرفت از من آن دریای جود
&&
آنچنان نقد دغل را زود زود
\\
کل عالم را سبو دان ای پسر
&&
کو بود از علم و خوبی تا بسر
\\
قطره‌ای از دجلهٔ خوبی اوست
&&
کان نمی‌گنجد ز پری زیر پوست
\\
گنج مخفی بد ز پری چاک کرد
&&
خاک را تابان‌تر از افلاک کرد
\\
گنج مخفی بد ز پری جوش کرد
&&
خاک را سلطان اطلس‌پوش کرد
\\
ور بدیدی شاخی از دجلهٔ خدا
&&
آن سبو را او فنا کردی فنا
\\
آنک دیدندش همیشه بی خودند
&&
بی‌خودانه بر سبو سنگی زدند
\\
ای ز غیرت بر سبو سنگی زده
&&
وان شکستت خود درستی آمده
\\
خم شکسته آب ازو ناریخته
&&
صد درستی زین شکست انگیخته
\\
جزو جزو خم برقصست و بحال
&&
عقل جزوی را نموده این محال
\\
نه سبو پیدا درین حالت نه آب
&&
خوش ببین والله اعلم بالصواب
\\
چون در معنی زنی بازت کنند
&&
پر فکرت زن که شهبازت کنند
\\
پر فکرت شد گل‌آلود و گران
&&
زانک گل‌خواری ترا گل شد چو نان
\\
نان گلست و گوشت کمتر خور ازین
&&
تا نمانی همچو گل اندر زمین
\\
چون گرسنه می‌شوی سگ می‌شوی
&&
تند و بد پیوند و بدرگ می‌شوی
\\
چون شدی تو سیر مرداری شدی
&&
بی‌خبر بی پا چو دیواری شدی
\\
پس دمی مردار و دیگر دم سگی
&&
چون کنی در راه شیران خوش‌تگی
\\
آلت اشکار خود جز سگ مدان
&&
کمترک انداز سگ را استخوان
\\
زانک سگ چون سیر شد سرکش شود
&&
کی سوی صید و شکار خوش دود
\\
آن عرب را بی‌نوایی می‌کشید
&&
تا بدان درگاه و آن دولت رسید
\\
در حکایت گفته‌ایم احسان شاه
&&
در حق آن بی‌نوای بی‌پناه
\\
هر چه گوید مرد عاشق بوی عشق
&&
از دهانش می‌جهد در کوی عشق
\\
گر بگوید فقه فقر آید همه
&&
بوی فقر آید از آن خوش دمدمه
\\
ور بگوید کفر دارد بوی دین
&&
آید از گفت شکش بوی یقین
\\
کف کژ کز بهر صدقی خاستست
&&
اصل صاف آن فرع را آراستست
\\
آن کفش را صافی و محقوق دان
&&
همچو دشنام لب معشوق دان
\\
گشته آن دشنام نامطلوب او
&&
خوش ز بهر عارض محبوب او
\\
گر بگوید کژ نماید راستی
&&
ای کژی که راست را آراستی
\\
از شکر گر شکل نانی می‌پزی
&&
طعم قند آید نه نان چون می‌مزی
\\
ور بیابد مؤمنی زرین وثن
&&
کی هلد آن را برای هر شمن
\\
بلک گیرد اندر آتش افکند
&&
صورت عاریتش را بشکند
\\
تا نماند بر ذهب شکل وثن
&&
زانک صورت مانعست و راه‌زن
\\
ذات زرش داد ربانیتست
&&
نقش بت بر نقد زر عاریتست
\\
بهر کیکی تو گلیمی را مسوز
&&
وز صداع هر مگس مگذار روز
\\
بت‌پرستی چون بمانی در صور
&&
صورتش بگذار و در معنی نگر
\\
مرد حجی همره حاجی طلب
&&
خواه هندو خواه ترک و یا عرب
\\
منگر اندر نقش و اندر رنگ او
&&
بنگر اندر عزم و در آهنگ او
\\
گر سیاهست او هم‌آهنگ توست
&&
تو سپیدش خوان که همرنگ توست
\\
این حکایت گفته شد زیر و زبر
&&
همچو فکر عاشقان بی پا و سر
\\
سر ندارد چون ز ازل بودست پیش
&&
پا ندارد با ابد بودست خویش
\\
بلک چون آبست هر قطره از آن
&&
هم سرست و پا و هم بی هر دوان
\\
حاش لله این حکایت نیست هین
&&
نقد حال ما و تست این خوش ببین
\\
زانک صوفی با کر و با فر بود
&&
هرچ آن ماضیست لا یذکر بود
\\
هم عرب ما هم سبو ما هم ملک
&&
جمله ما یؤفک عنه من افک
\\
عقل را شو دان و زن این نفس و طمع
&&
این دو ظلمانی و منکر عقل شمع
\\
بشنو اکنون اصل انکار از چه خاست
&&
زانک کل را گونه‌گونه جزوهاست
\\
جزو کل نی جزوها نسبت به کل
&&
نی چو بوی گل که باشد جزو گل
\\
لطف سبزه جزو لطف گل بود
&&
بانگ قمری جزو آن بلبل بود
\\
گر شوم مشغول اشکال و جواب
&&
تشنگان را کی توانم داد آب
\\
گر تو اشکالی بکلی و حرج
&&
صبر کن الصبر مفتاح الفرج
\\
احتما کن احتما ز اندیشه‌ها
&&
فکر شیر و گور و دلها بیشه‌ها
\\
احتماها بر دواها سرورست
&&
زانک خاریدن فزونی گرست
\\
احتما اصل دوا آمد یقین
&&
احتما کن قوت جانت ببین
\\
قابل این گفته‌ها شو گوش‌وار
&&
تا که از زر سازمت من گوش‌وار
\\
حلقه در گوش مه زرگر شوی
&&
تا به ماه و تا ثریا بر شوی
\\
اولا بشنو که خلق مختلف
&&
مختلف جانند تا یا از الف
\\
در حروف مختلف شور و شکیست
&&
گرچه از یک رو ز سر تا پا یکیست
\\
از یکی رو ضد و یک رو متحد
&&
از یکی رو هزل و از یک روی جد
\\
پس قیامت روز عرض اکبرست
&&
عرض او خواهد که با زیب و فرست
\\
هر که چون هندوی بدسوداییست
&&
روز عرضش نوبت رسواییست
\\
چون ندارد روی همچون آفتاب
&&
او نخواهد جز شبی همچون نقاب
\\
برگ یک گل چون ندارد خار او
&&
شد بهاران دشمن اسرار او
\\
وانک سر تا پا گلست و سوسنست
&&
پس بهار او را دو چشم روشنست
\\
خار بی‌معنی خزان خواهد خزان
&&
تا زند پهلوی خود با گلستان
\\
تا بپوشد حسن آن و ننگ این
&&
تا نبینی رنگ آن و زنگ این
\\
پس خزان او را بهارست و حیات
&&
یک نماید سنگ و یاقوت زکات
\\
باغبان هم داند آن را در خزان
&&
لیک دید یک به از دید جهان
\\
خود جهان آن یک کس است او ابلهست
&&
هر ستاره بر فلک جزو مهست
\\
پس همی‌گویند هر نقش و نگار
&&
مژده مژده نک همی آید بهار
\\
تا بود تابان شکوفه چون زره
&&
کی کنند آن میوه‌ها پیدا گره
\\
چون شکوفه ریخت میوه سر کند
&&
چونک تن بشکست جان سر بر زند
\\
میوه معنی و شکوفه صورتش
&&
آن شکوفه مژده میوه نعمتش
\\
چون شکوفه ریخت میوه شد پدید
&&
چونک آن کم شد شد این اندر مزید
\\
تا که نان نشکست قوت کی دهد
&&
ناشکسته خوشه‌ها کی می‌دهد
\\
تا هلیله نشکند با ادویه
&&
کی شود خود صحت‌افزا ادویه
\\
\end{longtable}
\end{center}
