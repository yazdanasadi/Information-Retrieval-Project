\begin{center}
\section*{بخش ۱۸ - بیان حسد وزیر}
\label{sec:sh018}
\addcontentsline{toc}{section}{\nameref{sec:sh018}}
\begin{longtable}{l p{0.5cm} r}
آن وزیرک از حسد بودش نژاد
&&
تا به باطل گوش و بینی باد داد
\\
بر امید آنک از نیش حسد
&&
زهر او در جان مسکینان رسد
\\
هر کسی کو از حسد بینی کند
&&
خویش را بی‌گوش و بی بینی کند
\\
بینی آن باشد که او بویی برد
&&
بوی او را جانب کویی برد
\\
هر که بویش نیست بی بینی بود
&&
بوی آن بویست کان دینی بود
\\
چونک بویی برد و شکر آن نکرد
&&
کفر نعمت آمد و بینیش خورد
\\
شکر کن مر شاکران را بنده باش
&&
پیش ایشان مرده شو پاینده باش
\\
چون وزیر از ره‌زنی مایه مساز
&&
خلق را تو بر میاور از نماز
\\
ناصح دین گشته آن کافر وزیر
&&
کرده او از مکر در گوزینه سیر
\\
\end{longtable}
\end{center}
