\begin{center}
\section*{بخش ۱۰۵ - نالیدن ستون حنانه چون برای پیغامبر صلی الله علیه و سلم منبر ساختند کی جماعت انبوه شد گفتند ما روی مبارک ترا بهنگام وعظ نمی‌بینیم و شنیدن رسول و صحابه آن ناله را و سال و جواب مصطفی صلی الله علیه و سلم با ستون صریح}
\label{sec:sh105}
\addcontentsline{toc}{section}{\nameref{sec:sh105}}
\begin{longtable}{l p{0.5cm} r}
استن حنانه از هجر رسول
&&
ناله می‌زد همچو ارباب عقول
\\
گفت پیغامبر چه خواهی ای ستون
&&
گفت جانم از فراقت گشت خون
\\
مسندت من بودم از من تاختی
&&
بر سر منبر تو مسند ساختی
\\
گفت خواهی که ترا نخلی کنند
&&
شرقی و غربی ز تو میوه چنند
\\
یا در آن عالم حقت سروی کند
&&
تا تر و تازه بمانی تا ابد
\\
گفت آن خواهم که دایم شد بقاش
&&
بشنو ای غافل کم از چوبی مباش
\\
آن ستون را دفن کرد اندر زمین
&&
تا چو مردم حشر گردد یوم دین
\\
تا بدانی هر که را یزدان بخواند
&&
از همه کار جهان بی کار ماند
\\
هر که را باشد ز یزدان کار و بار
&&
یافت بار آنجا و بیرون شد ز کار
\\
آنک او را نبود از اسرار داد
&&
کی کند تصدیق او نالهٔ جماد
\\
گوید آری نه ز دل بهر وفاق
&&
تا نگویندش که هست اهل نفاق
\\
گر نیندی واقفان امر کن
&&
در جهان رد گشته بودی این سخن
\\
صد هزاران ز اهل تقلید و نشان
&&
افکندشان نیم وهمی در گمان
\\
که بظن تقلید و استدلالشان
&&
قایمست و جمله پر و بالشان
\\
شبهه‌ای انگیزد آن شیطان دون
&&
در فتند این جمله کوران سرنگون
\\
پای استدلالیان چوبین بود
&&
پای چوبین سخت بی تمکین بود
\\
غیر آن قطب زمان دیده‌ور
&&
کز ثباتش کوه گردد خیره‌سر
\\
پای نابینا عصا باشد عصا
&&
تا نیفتد سرنگون او بر حصا
\\
آن سواری کو سپه را شد ظفر
&&
اهل دین را کیست سلطان بصر
\\
با عصا کوران اگر ره دیده‌اند
&&
در پناه خلق روشن‌دیده‌اند
\\
گر نه بینایان بدندی و شهان
&&
جمله کوران مرده‌اندی در جهان
\\
نه ز کوران کشت آید نه درود
&&
نه عمارت نه تجارتها و سود
\\
گر نکردی رحمت و افضالتان
&&
در شکستی چوب استدلالتان
\\
این عصا چه بود قیاسات و دلیل
&&
آن عصا که دادشان بینا جلیل
\\
چون عصا شد آلت جنگ و نفیر
&&
آن عصا را خرد بشکن ای ضریر
\\
او عصاتان داد تا پیش آمدیت
&&
آن عصا از خشم هم بر وی زدیت
\\
حلقهٔ کوران به چه کار اندرید
&&
دیدبان را در میانه آورید
\\
دامن او گیر کو دادت عصا
&&
در نگر کادم چه‌ها دید از عصا
\\
معجزهٔ موسی و احمد را نگر
&&
چون عصا شد مار و استن با خبر
\\
از عصا ماری و از استن حنین
&&
پنج نوبت می‌زنند از بهر دین
\\
گرنه نامعقول بودی این مزه
&&
کی بدی حاجت به چندین معجزه
\\
هرچه معقولست عقلش می‌خورد
&&
بی بیان معجزه بی جر و مد
\\
این طریق بکر نامعقول بین
&&
در دل هر مقبلی مقبول بین
\\
همچنان کز بیم آدم دیو و دد
&&
در جزایر در رمیدند از حسد
\\
هم ز بیم معجزات انبیا
&&
سر کشیده منکران زیر گیا
\\
تا به ناموس مسلمانی زیند
&&
در تسلس تا ندانی که کیند
\\
همچو قلابان بر آن نقد تباه
&&
نقره می‌مالند و نام پادشاه
\\
ظاهر الفاظشان توحید و شرع
&&
باطن آن همچو در نان تخم صرع
\\
فلسفی را زهره نه تا دم زند
&&
دم زند دین حقش بر هم زند
\\
دست و پای او جماد و جان او
&&
هر چه گوید آن دو در فرمان او
\\
با زبان گر چه تهمت می‌نهند
&&
دست و پاهاشان گواهی می‌دهند
\\
\end{longtable}
\end{center}
