\begin{center}
\section*{بخش ۸۷ - تفسیر قول فریدالدین عطار قدس الله روحه تو صاحب نفسی ای غافل میان خاک خون می‌خور که صاحب‌دل اگر زهری خورد آن انگبین باشد}
\label{sec:sh087}
\addcontentsline{toc}{section}{\nameref{sec:sh087}}
\begin{longtable}{l p{0.5cm} r}
صاحب دل را ندارد آن زیان
&&
گر خورد او زهر قاتل را عیان
\\
زانک صحت یافت و از پرهیز رست
&&
طالب مسکین میان تب درست
\\
گفت پیغامبر که ای مرد جری
&&
هان مکن با هیچ مطلوبی مری
\\
در تو نمرودیست آتش در مرو
&&
رفت خواهی اول ابراهیم شو
\\
چون نه‌ای سباح و نه دریایی
&&
در میفکن خویش از خودراییی
\\
او ز آتش ورد احمر آورد
&&
از زیانها سود بر سر آورد
\\
کاملی گر خاک گیرد زر شود
&&
ناقص ار زر برد خاکستر شود
\\
چون قبول حق بود آن مرد راست
&&
دست او در کارها دست خداست
\\
دست ناقص دست شیطانست و دیو
&&
زانک اندر دام تکلیفست و ریو
\\
جهل آید پیش او دانش شود
&&
جهل شد علمی که در منکر رود
\\
هرچه گیرد علتی علت شود
&&
کفر گیرد کاملی ملت شود
\\
ای مری کرده پیاده با سوار
&&
سر نخواهی برد اکنون پای دار
\\
\end{longtable}
\end{center}
