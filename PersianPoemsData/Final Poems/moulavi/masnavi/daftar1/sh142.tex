\begin{center}
\section*{بخش ۱۴۲ - رفتن گرگ و روباه در خدمت شیر به شکار}
\label{sec:sh142}
\addcontentsline{toc}{section}{\nameref{sec:sh142}}
\begin{longtable}{l p{0.5cm} r}
شیر و گرگ و روبهی بهر شکار
&&
رفته بودند از طلب در کوهسار
\\
تا به پشت همدگر بر صیدها
&&
سخت بر بندند بار قیدها
\\
هر سه با هم اندر آن صحرای ژرف
&&
صیدها گیرند بسیار و شگرف
\\
گرچه زیشان شیر نر را ننگ بود
&&
لیک کرد اکرام و همراهی نمود
\\
این چنین شه را ز لشکر زحمتست
&&
لیک همره شد جماعت رحمتست
\\
این چنین مه را ز اختر ننگهاست
&&
او میان اختران بهر سخاست
\\
امر شاورهم پیمبر را رسید
&&
گرچه رایی نیست رایش را ندید
\\
در ترازو جو رفیق زر شدست
&&
نه از آن که جو چو زر جوهر شدست
\\
روح قالب را کنون همره شدست
&&
مدتی سگ حارس درگه شدست
\\
چونک رفتند این جماعت سوی کوه
&&
در رکاب شیر با فر و شکوه
\\
گاو کوهی و بز و خرگوش زفت
&&
یافتند و کار ایشان پیش رفت
\\
هر که باشد در پی شیر حراب
&&
کم نیاید روز و شب او را کباب
\\
چون ز که در پیشه آوردندشان
&&
کشته و مجروح و اندر خون کشان
\\
گرگ و روبه را طمع بود اندر آن
&&
که رود قسمت به عدل خسروان
\\
عکس طمع هر دوشان بر شیر زد
&&
شیر دانست آن طمعها را سند
\\
هر که باشد شیر اسرار و امیر
&&
او بداند هر چه اندیشد ضمیر
\\
هین نگه دار ای دل اندیشه‌خو
&&
دل ز اندیشهٔ بدی در پیش او
\\
داند و خر را همی‌راند خموش
&&
در رخت خندد برای روی‌پوش
\\
شیر چون دانست آن وسواسشان
&&
وا نگفت و داشت آن دم پاسشان
\\
لیک با خود گفت بنمایم سزا
&&
مر شما را ای خسیسان گدا
\\
مر شما را بس نیامد رای من
&&
ظنتان اینست در اعطای من
\\
ای عقول و رایتان از رای من
&&
از عطاهای جهان‌آرای من
\\
نقش با نقاش چه سگالد دگر
&&
چون سگالش اوش بخشید و خبر
\\
این چنین ظن خسیسانه بمن
&&
مر شما را بود ننگان زمن
\\
ظانین بالله ظن السؤ را
&&
گر نبرم سر بود عین خطا
\\
وا رهانم چرخ را از ننگتان
&&
تا بماند در جهان این داستان
\\
شیر با این فکر می‌زد خنده فاش
&&
بر تبسمهای شیر ایمن مباش
\\
مال دنیا شد تبسمهای حق
&&
کرد ما را مست و مغرور و خلق
\\
فقر و رنجوری بهستت ای سند
&&
کان تبسم دام خود را بر کند
\\
\end{longtable}
\end{center}
