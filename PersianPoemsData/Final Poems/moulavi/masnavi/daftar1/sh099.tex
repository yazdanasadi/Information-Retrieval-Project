\begin{center}
\section*{بخش ۹۹ - قصهٔ سوال کردن عایشه رضی الله عنها از مصطفی صلی‌الله علیه و سلم کی امروز باران بارید چون تو سوی گورستان رفتی جامه‌های تو چون تر نیست}
\label{sec:sh099}
\addcontentsline{toc}{section}{\nameref{sec:sh099}}
\begin{longtable}{l p{0.5cm} r}
مصطفی روزی به گورستان برفت
&&
با جنازهٔ مردی از یاران برفت
\\
خاک را در گور او آگنده کرد
&&
زیر خاک آن دانه‌اش را زنده کرد
\\
این درختانند همچون خاکیان
&&
دستها بر کرده‌اند از خاکدان
\\
سوی خلقان صد اشارت می‌کنند
&&
وانک گوشستش عبارت می‌کنند
\\
با زبان سبز و با دست دراز
&&
از ضمیر خاک می‌گویند راز
\\
همچو بطان سر فرو برده بب
&&
گشته طاووسان و بوده چون غراب
\\
در زمستانشان اگر محبوس کرد
&&
آن غرابان را خدا طاووس کرد
\\
در زمستانشان اگر چه داد مرگ
&&
زنده‌شان کرد از بهار و داد برگ
\\
منکران گویند خود هست این قدیم
&&
این چرا بندیم بر رب کریم
\\
کوری ایشان درون دوستان
&&
حق برویانید باغ و بوستان
\\
هر گلی کاندر درون بویا بود
&&
آن گل از اسرار کل گویا بود
\\
بوی ایشان رغم آنف منکران
&&
گرد عالم می‌رود پرده‌دران
\\
منکران همچون جعل زان بوی گل
&&
یا چو نازک مغز در بانگ دهل
\\
خویشتن مشغول می‌سازند و غرق
&&
چشم می‌دزدند ازین لمعان برق
\\
چشم می‌دزدند و آنجا چشم نی
&&
چشم آن باشد که بیند مامنی
\\
چون ز گورستان پیمبر باز گشت
&&
سوی صدیقه شد و همراز گشت
\\
چشم صدیقه چو بر رویش فتاد
&&
پیش آمد دست بر وی می‌نهاد
\\
بر عمامه و روی او و موی او
&&
بر گریبان و بر و بازوی او
\\
گفت پیغامبر چه می‌جویی شتاب
&&
گفت باران آمد امروز از سحاب
\\
جامه‌هاات می‌بجویم در طلب
&&
تر نمی‌یابم ز باران ای عجب
\\
گفت چه بر سر فکندی از ازار
&&
گفت کردم آن ردای تو خمار
\\
گفت بهر آن نمود ای پاک‌جیب
&&
چشم پاکت را خدا باران غیب
\\
نیست آن باران ازین ابر شما
&&
هست ابری دیگر و دیگر سما
\\
\end{longtable}
\end{center}
