\begin{center}
\section*{بخش ۱۳۹ - در صفت پیر و مطاوعت وی}
\label{sec:sh139}
\addcontentsline{toc}{section}{\nameref{sec:sh139}}
\begin{longtable}{l p{0.5cm} r}
ای ضیاء الحق حسام الدین بگیر
&&
یک دو کاغذ بر فزا در وصف پیر
\\
گرچه جسم نازکت را زور نیست
&&
لیک بی خورشید ما را نور نیست
\\
گرچه مصباح و زجاجه گشته‌ای
&&
لیک سرخیل دلی سررشته‌ای
\\
چون سر رشته به دست و کام تست
&&
درهای عقد دل ز انعام تست
\\
بر نویس احوال پیر راه‌دان
&&
پیر را بگزین و عین راه دان
\\
پیر تابستان و خلقان تیر ماه
&&
خلق مانند شبند و پیر ماه
\\
کرده‌ام بخت جوان را نام پیر
&&
کو ز حق پیرست نه از ایام پیر
\\
او چنان پیرست کش آغاز نیست
&&
با چنان در یتیم انباز نیست
\\
خود قوی‌تر می‌شود خمر کهن
&&
خاصه آن خمری که باشد من لدن
\\
پیر را بگزین که بی پیر این سفر
&&
هست بس پر آفت و خوف و خطر
\\
آن رهی که بارها تو رفته‌ای
&&
بی قلاوز اندر آن آشفته‌ای
\\
پس رهی را که ندیدستی تو هیچ
&&
هین مرو تنها ز رهبر سر مپیچ
\\
گر نباشد سایهٔ او بر تو گول
&&
پس ترا سرگشته دارد بانگ غول
\\
غولت از ره افکند اندر گزند
&&
از تو داهی‌تر درین ره بس بدند
\\
از نبی بشنو ضلال ره‌روان
&&
که چه شان کرد آن بلیس بدروان
\\
صد هزاران ساله راه از جاده دور
&&
بردشان و کردشان ادبیر و عور
\\
استخوانهاشان ببین و مویشان
&&
عبرتی گیر و مران خر سویشان
\\
گردن خر گیر و سوی راه کش
&&
سوی ره‌بانان و ره‌دانان خوش
\\
هین مهل خر را و دست از وی مدار
&&
زانک عشق اوست سوی سبزه‌زار
\\
گر یکی دم تو به غفلت وا هلیش
&&
او رود فرسنگها سوی حشیش
\\
دشمن راهست خر مست علف
&&
ای که بس خر بنده را کرد او تلف
\\
گر ندانی ره هر آنچ خر بخواست
&&
عکس آن کن خود بود آن راه راست
\\
شاوروهن و آنگه خالفوا
&&
ان من لم یعصهن تالف
\\
با هوا و آرزو کم باش دوست
&&
چون یضلک عن سبیل الله اوست
\\
این هوا را نشکند اندر جهان
&&
هیچ چیزی همچو سایهٔ همرهان
\\
\end{longtable}
\end{center}
