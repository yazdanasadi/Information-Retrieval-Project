\begin{center}
\section*{بخش ۱۶۷ - گفتن پیغامبر صلی الله علیه و سلم به گوش رکابدار امیر المومنین علی کرم الله  وجهه کی کشتن علی بر دست تو خواهد بودن خبرت کردم}
\label{sec:sh167}
\addcontentsline{toc}{section}{\nameref{sec:sh167}}
\begin{longtable}{l p{0.5cm} r}
من چنان مردم که بر خونی خویش
&&
نوش لطف من نشد در قهر نیش
\\
گفت پیغامبر به گوش چاکرم
&&
کو برد روزی ز گردن این سرم
\\
کرد آگه آن رسول از وحی دوست
&&
که هلاکم عاقبت بر دست اوست
\\
او همی‌گوید بکش پیشین مرا
&&
تا نیاید از من این منکر خطا
\\
من همی‌گویم چو مرگ من ز تست
&&
با قضا من چون توانم حیله جست
\\
او همی‌افتد به پیشم کای کریم
&&
مر مرا کن از برای حق دو نیم
\\
تا نه آید بر من این انجام بد
&&
تا نسوزد جان من بر جان خود
\\
من همی گویم برو جف القلم
&&
زان قلم بس سرنگون گردد علم
\\
هیچ بغضی نیست در جانم ز تو
&&
زانک این را من نمی‌دانم ز تو
\\
آلت حقی تو فاعل دست حق
&&
چون زنم بر آلت حق طعن و دق
\\
گفت او پس آن قصاص از بهر چیست
&&
گفت هم از حق و آن سر خفیست
\\
گر کند بر فعل خود او اعتراض
&&
ز اعتراض خود برویاند ریاض
\\
اعتراض او را رسد بر فعل خود
&&
زانک در قهرست و در لطف او احد
\\
اندرین شهر حوادث میر اوست
&&
در ممالک مالک تدبیر اوست
\\
آلت خود را اگر او بشکند
&&
آن شکسته گشته را نیکو کند
\\
رمز ننسخ آیة او ننسها
&&
نات خیرا در عقب می‌دان مها
\\
هر شریعت را که حق منسوخ کرد
&&
او گیا برد و عوض آورد ورد
\\
شب کند منسوخ شغل روز را
&&
بین جمادی خرد افروز را
\\
باز شب منسوخ شد از نور روز
&&
تا جمادی سوخت زان آتش‌فروز
\\
گرچه ظلمت آمد آن نوم و سبات
&&
نه درون ظلمتست آب حیات
\\
نه در آن ظلمت خردها تازه شد
&&
سکته‌ای سرمایهٔ آوازه شد
\\
که ز ضدها ضدها آمد پدید
&&
در سویدا روشنایی آفرید
\\
جنگ پیغامبر مدار صلح شد
&&
صلح این آخر زمان زان جنگ بد
\\
صد هزاران سر برید آن دلستان
&&
تا امان یابد سر اهل جهان
\\
باغبان زان می‌برد شاخ مضر
&&
تا بیابد نخل قامتها و بر
\\
می‌کند از باغ دانا آن حشیش
&&
تا نماید باغ و میوه خرمیش
\\
می‌کند دندان بد را آن طبیب
&&
تا رهد از درد و بیماری حبیب
\\
پس زیادتها درون نقصهاست
&&
مر شهیدان را حیات اندر فناست
\\
چون بریده گشت حلق رزق‌خوار
&&
یرزقون فرحین شد گوار
\\
حلق حیوان چون بریده شد بعدل
&&
حلق انسان رست و افزونید فضل
\\
حلق انسان چون ببرد هین ببین
&&
تا چه زاید کن قیاس آن برین
\\
حلق ثالث زاید و تیمار او
&&
شربت حق باشد و انوار او
\\
حلق ببریده خورد شربت ولی
&&
حلق از لا رسته مرده در بلی
\\
بس کن ای دون‌همت کوته‌بنان
&&
تا کیت باشد حیات جان به نان
\\
زان نداری میوه‌ای مانند بید
&&
کب رو بردی پی نان سپید
\\
گر ندارد صبر زین نان جان حس
&&
کیمیا را گیر و زر گردان تو مس
\\
جامه‌شویی کرد خواهی ای فلان
&&
رو مگردان از محلهٔ گازران
\\
گرچه نان بشکست مر روزهٔ ترا
&&
در شکسته‌بند پیچ و برتر آ
\\
چون شکسته‌بند آمد دست او
&&
پس رفو باشد یقین اشکست او
\\
گر تو آن را بشکنی گوید بیا
&&
تو درستش کن نداری دست و پا
\\
پس شکستن حق او باشد که او
&&
مر شکسته گشته را داند رفو
\\
آنک داند دوخت او داند درید
&&
هر چه را بفروخت نیکوتر خرید
\\
خانه را ویران کند زیر و زبر
&&
پس بیک ساعت کند معمورتر
\\
گر یکی سر را ببرد از بدن
&&
صد هزاران سر بر آرد در زمن
\\
گر نفرمودی قصاصی بر جنات
&&
یا نگفتی فی القصاص آمد حیات
\\
خود که را زهره بدی تا او ز خود
&&
بر اسیر حکم حق تیغی زند
\\
زانک داند هر که چشمش را گشود
&&
کان کشنده سخرهٔ تقدیر بود
\\
هر که را آن حکم بر سر آمدی
&&
بر سر فرزند هم تیغی زدی
\\
رو بترس و طعنه کم زن بر بدان
&&
پیش دام حکم عجز خود بدان
\\
\end{longtable}
\end{center}
