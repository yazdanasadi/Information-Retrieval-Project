\begin{center}
\section*{بخش ۶۵ - جواب گفتن شیر خرگوش را و روان شدن با او}
\label{sec:sh065}
\addcontentsline{toc}{section}{\nameref{sec:sh065}}
\begin{longtable}{l p{0.5cm} r}
گفت بسم الله بیا تا او کجاست
&&
پیش در شو گر همی گویی تو راست
\\
تا سزای او و صد چون او دهم
&&
ور دروغست این سزای تو دهم
\\
اندر آمد چون قلاووزی به پیش
&&
تا برد او را به سوی دام خویش
\\
سوی چاهی کو نشانش کرده بود
&&
چاه مغ را دام جانش کرده بود
\\
می‌شدند این هر دو تا نزدیک چاه
&&
اینت خرگوشی چو آبی زیر کاه
\\
آب کاهی را به هامون می‌برد
&&
آب کوهی را عجب چون می‌برد
\\
دام مکر او کمند شیر بود
&&
طرفه خرگوشی که شیری می‌ربود
\\
موسیی فرعون را با رود نیل
&&
می‌کشد با لشکر و جمع ثقیل
\\
پشه‌ای نمرود را با نیم پر
&&
می‌شکافد بی‌محابا درز سر
\\
حال آن کو قول دشمن را شنود
&&
بین جزای آنک شد یار حسود
\\
حال فرعونی که هامان را شنود
&&
حال نمرودی که شیطان را شنود
\\
دشمن ار چه دوستانه گویدت
&&
دام دان گر چه ز دانه گویدت
\\
گر ترا قندی دهد آن زهر دان
&&
گر بتن لطفی کند آن قهر دان
\\
چون قضا آید نبینی غیر پوست
&&
دشمنان را باز نشناسی ز دوست
\\
چون چنین شد ابتهال آغاز کن
&&
ناله و تسبیح و روزه ساز کن
\\
ناله می‌کن کای تو علام الغیوب
&&
زیر سنگ مکر بد ما را مکوب
\\
گر سگی کردیم ای شیرآفرین
&&
شیر را مگمار بر ما زین کمین
\\
آب خوش را صورت آتش مده
&&
اندر آتش صورت آبی منه
\\
از شراب قهر چون مستی دهی
&&
نیستها را صورت هستی دهی
\\
چیست مستی بند چشم از دید چشم
&&
تا نماند سنگ گوهر پشم یشم
\\
چیست مستی حسها مبدل شدن
&&
چوب گز اندر نظر صندل شدن
\\
\end{longtable}
\end{center}
