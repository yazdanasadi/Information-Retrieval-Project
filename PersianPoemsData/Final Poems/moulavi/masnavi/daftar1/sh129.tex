\begin{center}
\section*{بخش ۱۲۹ - هدیه بردن عرب سبوی آب باران از میان بادیه سوی بغداد به امیرالممنین بر پنداشت آنک آنجا هم قحط آبست}
\label{sec:sh129}
\addcontentsline{toc}{section}{\nameref{sec:sh129}}
\begin{longtable}{l p{0.5cm} r}
گفت زن صدق آن بود کز بود خویش
&&
پاک برخیزی تو از مجهود خویش
\\
آب بارانست ما را در سبو
&&
ملکت و سرمایه و اسباب تو
\\
این سبوی آب را بردار و رو
&&
هدیه ساز و پیش شاهنشاه شو
\\
گو که ما را غیر این اسباب نیست
&&
در مفازه هیچ به زین آب نیست
\\
گر خزینه‌ش پر متاع فاخرست
&&
این چنین آبش نباشد نادرست
\\
چیست آن کوزه تن محصور ما
&&
اندرو آب حواس شور ما
\\
ای خداوند این خم و کوزهٔ مرا
&&
در پذیر از فضل الله اشتری
\\
کوزه‌ای با پنج لولهٔ پنج حس
&&
پاک دار این آب را از هر نجس
\\
تا شود زین کوزه منفذ سوی بحر
&&
تا بگیرد کوزهٔ من خوی بحر
\\
تا چو هدیه پیش سلطانش بری
&&
پاک بیند باشدش شه مشتری
\\
بی‌نهایت گردد آبش بعد از آن
&&
پر شود از کوزهٔ من صد جهان
\\
لوله‌ها بر بند و پر دارش ز خم
&&
گفت غضوا عن هوا ابصارکم
\\
ریش او پر باد کین هدیه کراست
&&
لایق چون او شهی اینست راست
\\
زن نمی‌دانست کانجا برگذر
&&
هست جاری دجله‌ای همچون شکر
\\
در میان شهر چون دریا روان
&&
پر ز کشتیها و شست ماهیان
\\
رو بر سلطان و کار و بار بین
&&
حس تجری تحتها الانهار بین
\\
این چنین حسها و ادراکات ما
&&
قطره‌ای باشد در آن نهر صفا
\\
\end{longtable}
\end{center}
