\begin{center}
\section*{بخش ۱۴۴ - قصه آنکس کی در یاری بکوفت از درون گفت کیست آن گفت منم گفت چون تو توی در نمی‌گشایم هیچ کس را از یاران نمی‌شناسم کی او من باشد برو}
\label{sec:sh144}
\addcontentsline{toc}{section}{\nameref{sec:sh144}}
\begin{longtable}{l p{0.5cm} r}
آن یکی آمد در یاری بزد
&&
گفت یارش کیستی ای معتمد
\\
گفت من گفتش برو هنگام نیست
&&
بر چنین خوانی مقام خام نیست
\\
خام را جز آتش هجر و فراق
&&
کی پزد کی وا رهاند از نفاق
\\
رفت آن مسکین و سالی در سفر
&&
در فراق دوست سوزید از شرر
\\
پخته گشت آن سوخته پس باز گشت
&&
باز گرد خانهٔ همباز گشت
\\
حلقه زد بر در بصد ترس و ادب
&&
تا بنجهد بی‌ادب لفظی ز لب
\\
بانگ زد یارش که بر در کیست آن
&&
گفت بر در هم توی ای دلستان
\\
گفت اکنون چون منی ای من در آ
&&
نیست گنجایی دو من را در سرا
\\
نیست سوزن را سر رشتهٔ دوتا
&&
چونک یکتایی درین سوزن در آ
\\
رشته را با سوزن آمد ارتباط
&&
نیست در خور با جمل سم الخیاط
\\
کی شود باریک هستی جمل
&&
جز بمقراض ریاضات و عمل
\\
دست حق باید مر آن را ای فلان
&&
کو بود بر هر محالی کن فکان
\\
هر محال از دست او ممکن شود
&&
هر حرون از بیم او ساکن شود
\\
اکمه و ابرص چه باشد مرده نیز
&&
زنده گردد از فسون آن عزیز
\\
و آن عدم کز مرده مرده‌تر بود
&&
در کف ایجاد او مضطر بود
\\
کل یوم هو فی شان بخوان
&&
مر ورا بی کار و بی‌فعلی مدان
\\
کمترین کاریش هر روزست آن
&&
کو سه لشکر را کند این سو روان
\\
لشکری ز اصلاب سوی امهات
&&
بهر آن تا در رحم روید نبات
\\
لشکری ز ارحام سوی خاکدان
&&
تا ز نر و ماده پر گردد جهان
\\
لشکری از خاک زان سوی اجل
&&
تا ببیند هر کسی حسن عمل
\\
این سخن پایان ندارد هین بتاز
&&
سوی آن دو یار پاک پاک‌باز
\\
گفت یارش کاندر آ ای جمله من
&&
نی مخالف چون گل و خار چمن
\\
رشته یکتا شد غلط کم شو کنون
&&
گر دوتا بینی حروف کاف و نون
\\
کاف و نون همچون کمند آمد جذوب
&&
تا کشاند مر عدم را در خطوب
\\
پس دوتا باید کمند اندر صور
&&
گرچه یکتا باشد آن دو در اثر
\\
گر دو پا گر چار پا ره را برد
&&
همچو مقراض دو تا یکتا برد
\\
آن دو همبازان گازر را ببین
&&
هست در ظاهر خلافی زان و زین
\\
آن یکی کرباس را در آب زد
&&
وان دگر همباز خشکش می‌کند
\\
باز او آن خشک را تر می‌کند
&&
گوییا ز استیزه ضد بر می‌تند
\\
لیک این دو ضد استیزه‌نما
&&
یک‌دل و یک‌کار باشد در رضا
\\
هر نبی و هر ولی را ملکیست
&&
لیک تا حق می‌برد جمله یکیست
\\
چونک جمع مستمع را خواب برد
&&
سنگهای آسیا را آب برد
\\
رفتن این آب فوق آسیاست
&&
رفتنش در آسیا بهر شماست
\\
چون شما را حاجت طاحون نماند
&&
آب را در جوی اصلی باز راند
\\
ناطقه سوی دهان تعلیم راست
&&
ورنه خود آن نطق را جویی جداست
\\
می‌رود بی بانگ و بی تکرارها
&&
تحتها الانهار تا گلزارها
\\
ای خدا جان را تو بنما آن مقام
&&
کاندرو بی‌حرف می‌روید کلام
\\
تا که سازد جان پاک از سر قدم
&&
سوی عرصهٔ دور و پنهای عدم
\\
عرصه‌ای بس با گشاد و با فضا
&&
وین خیال و هست یابد زو نوا
\\
تنگ‌تر آمد خیالات از عدم
&&
زان سبب باشد خیال اسباب غم
\\
باز هستی تنگ‌تر بود از خیال
&&
زان شود در وی قمر همچون هلال
\\
باز هستی جهان حس و رنگ
&&
تنگ‌تر آمد که زندانیست تنگ
\\
علت تنگیست ترکیب و عدد
&&
جانب ترکیب حسها می‌کشد
\\
زان سوی حس عالم توحید دان
&&
گر یکی خواهی بدان جانب بران
\\
امر کن یک فعل بود و نون و کاف
&&
در سخن افتاد و معنی بود صاف
\\
این سخن پایان ندارد باز گرد
&&
تا چه شد احوال گرگ اندر نبرد
\\
\end{longtable}
\end{center}
