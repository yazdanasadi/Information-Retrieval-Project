\begin{center}
\section*{بخش ۱۰۲ - پرسیدن صدیقه رضی‌الله عنها از مصطفی صلی‌الله علیه و سلم کی سر باران امروزینه چه بود}
\label{sec:sh102}
\addcontentsline{toc}{section}{\nameref{sec:sh102}}
\begin{longtable}{l p{0.5cm} r}
گفت صدیقه که ای زبدهٔ وجود
&&
حکمت باران امروزین چه بود
\\
این ز بارانهای رحمت بود یا
&&
بهر تهدیدست و عدل کبریا
\\
این از آن لطف بهاریات بود
&&
یا ز پاییزی پر آفات بود
\\
گفت این از بهر تسکین غمست
&&
کز مصیبت بر نژاد آدمست
\\
گر بر آن آتش بماندی آدمی
&&
بس خرابی در فتادی و کمی
\\
این جهان ویران شدی اندر زمان
&&
حرصها بیرون شدی از مردمان
\\
استن این عالم ای جان غفلتست
&&
هوشیاری این جهان را آفتست
\\
هوشیاری زان جهانست و چو آن
&&
غالب آید پست گردد این جهان
\\
هوشیاری آفتاب و حرص یخ
&&
هوشیاری آب و این عالم وسخ
\\
زان جهان اندک ترشح می‌رسد
&&
تا نغرد در جهان حرص و حسد
\\
گر ترشح بیشتر گردد ز غیب
&&
نه هنر ماند درین عالم نه عیب
\\
این ندارد حد سوی آغاز رو
&&
سوی قصهٔ مرد مطرب باز رو
\\
\end{longtable}
\end{center}
