\begin{center}
\section*{بخش ۱۶۳ - آتش افتادن در شهر بایام عمر رضی الله عنه}
\label{sec:sh163}
\addcontentsline{toc}{section}{\nameref{sec:sh163}}
\begin{longtable}{l p{0.5cm} r}
آتشی افتاد در عهد عمر
&&
همچو چوب خشک می‌خورد او حجر
\\
در فتاد اندر بنا و خانه‌ها
&&
تا زد اندر پر مرغ و لانه‌ها
\\
نیم شهر از شعله‌ها آتش گرفت
&&
آب می‌ترسید از آن و می‌شکفت
\\
مشکهای آب و سرکه می‌زدند
&&
بر سر آتش کسان هوشمند
\\
آتش از استیزه افزون می‌شدی
&&
می‌رسید او را مدد از بی حدی
\\
خلق آمد جانب عمر شتاب
&&
کآتش ما می‌نمیرد هیچ از آب
\\
گفت آن آتش ز آیات خداست
&&
شعله‌ای از آتش بخل شماست
\\
آب و سرکه چیست نان قسمت کنید
&&
بخل بگذارید اگر آل منید
\\
خلق گفتندش که در بگشوده‌ایم
&&
ما سخی و اهل فتوت بوده‌ایم
\\
گفت نان در رسم و عادت داده‌اید
&&
دست از بهر خدا نگشاده‌اید
\\
بهر فخر و بهر بوش و بهر ناز
&&
نه از برای ترس و تقوی و نیاز
\\
مال تخمست و بهر شوره منه
&&
تیغ را در دست هر ره‌زن مده
\\
اهل دین را باز دان از اهل کین
&&
همنشین حق بجو با او نشین
\\
هر کسی بر قوم خود ایثار کرد
&&
کاغه پندارد که او خود کار کرد
\\
\end{longtable}
\end{center}
