\begin{center}
\section*{بخش ۱۶۱ - گفتن پیغامبر صلی الله علیه و سلم مر زید را کی این سر را فاش‌تر ازین مگو و متابعت نگهدار}
\label{sec:sh161}
\addcontentsline{toc}{section}{\nameref{sec:sh161}}
\begin{longtable}{l p{0.5cm} r}
گفت پیغامبر که اصحابی نجوم
&&
ره‌روان را شمع و شیطان را رجوم
\\
هر کسی را گر بدی آن چشم و زور
&&
کو گرفتی ز آفتاب چرخ نور
\\
کی ستاره حاجتستی ای ذلیل
&&
که بدی بر نور خورشید او دلیل
\\
ماه می‌گوید به خاک و ابر و فی
&&
من بشر بودم ولی یوحی الی
\\
چون شما تاریک بودم در نهاد
&&
وحی خورشیدم چنین نوری بداد
\\
ظلمتی دارم به نسبت با شموس
&&
نور دارم بهر ظلمات نفوس
\\
زان ضعیفم تا تو تابی آوری
&&
که نه مرد آفتاب انوری
\\
همچو شهد و سرکه در هم بافتم
&&
تا سوی رنج جگر ره یافتم
\\
چون ز علت وا رهیدی ای رهین
&&
سرکه را بگذار و می‌خور انگبین
\\
تخت دل معمور شد پاک از هوا
&&
بین که الرحمن علی العرش استوی
\\
حکم بر دل بعد ازین بی واسطه
&&
حق کند چون یافت دل این رابطه
\\
این سخن پایان ندارد زید کو
&&
تا دهم پندش که رسوایی مجو
\\
\end{longtable}
\end{center}
