\begin{center}
\section*{بخش ۱۶۶ - جواب گفتن امیر الممنین کی سبب افکندن شمشیر از دست چه بوده است  در آن حالت}
\label{sec:sh166}
\addcontentsline{toc}{section}{\nameref{sec:sh166}}
\begin{longtable}{l p{0.5cm} r}
گفت من تیغ از پی حق می‌زنم
&&
بندهٔ حقم نه مامور تنم
\\
شیر حقم نیستم شیر هوا
&&
فعل من بر دین من باشد گوا
\\
ما رمیت اذ رمیتم در حراب
&&
من چو تیغم وان زننده آفتاب
\\
رخت خود را من ز ره بر داشتم
&&
غیر حق را من عدم انگاشتم
\\
سایه‌ای‌ام کدخداام آفتاب
&&
حاجبم من نیستم او را حجاب
\\
من چو تیغم پر گهرهای وصال
&&
زنده گردانم نه کشته در قتال
\\
خون نپوشد گوهر تیغ مرا
&&
باد از جا کی برد میغ مرا
\\
که نیم کوهم ز حلم و صبر و داد
&&
کوه را کی در رباید تند باد
\\
آنک از بادی رود از جا خسیست
&&
زانک باد ناموافق خود بسیست
\\
باد خشم و باد شهوت باد آز
&&
برد او را که نبود اهل نماز
\\
کوهم و هستی من بنیاد اوست
&&
ور شوم چون کاه بادم یاد اوست
\\
جز به باد او نجنبد میل من
&&
نیست جز عشق احد سرخیل من
\\
خشم بر شاهان شه و ما را غلام
&&
خشم را هم بسته‌ام زیر لگام
\\
تیغ حلمم گردن خشمم زدست
&&
خشم حق بر من چو رحمت آمدست
\\
غرق نورم گرچه سقفم شد خراب
&&
روضه گشتم گرچه هستم بوتراب
\\
چون در آمد علتی اندر غزا
&&
تیغ را دیدم نهان کردن سزا
\\
تا احب لله آید نام من
&&
تا که ابغض لله آید کام من
\\
تا که اعطا لله آید جود من
&&
تا که امسک لله آید بود من
\\
بخل من لله عطا لله و بس
&&
جمله لله‌ام نیم من آن کس
\\
وانچ لله می‌کنم تقلید نیست
&&
نیست تخییل و گمان جز دید نیست
\\
ز اجتهاد و از تحری رسته‌ام
&&
آستین بر دامن حق بسته‌ام
\\
گر همی‌پرم همی‌بینم مطار
&&
ور همی‌گردم همی‌بینم مدار
\\
ور کشم باری بدانم تا کجا
&&
ماهم و خورشید پیشم پیشوا
\\
بیش ازین با خلق گفتن روی نیست
&&
بحر را گنجایی اندر جوی نیست
\\
پست می‌گویم به اندازهٔ عقول
&&
عیب نبود این بود کار رسول
\\
از غرض حرم گواهی حر شنو
&&
که گواهی بندگان نه ارزد دو جو
\\
در شریعت مر گواهی بنده را
&&
نیست قدری وقت دعوی و قضا
\\
گر هزاران بنده باشندت گواه
&&
بر نسنجد شرع ایشان را به کاه
\\
بندهٔ شهوت بتر نزدیک حق
&&
از غلام و بندگان مسترق
\\
کین بیک لفظی شود از خواجه حر
&&
وان زید شیرین میرد سخت مر
\\
بندهٔ شهوت ندارد خود خلاص
&&
جز به فضل ایزد و انعام خاص
\\
در چهی افتاد کان را غور نیست
&&
وان گناه اوست جبر و جور نیست
\\
در چهی انداخت او خود را که من
&&
درخور قعرش نمی‌یابم رسن
\\
بس کنم گر این سخن افزون شود
&&
خود جگر چه بود که خارا خون شود
\\
این جگرها خون نشد نه از سختی است
&&
غفلت و مشغولی و بدبختی است
\\
خون شود روزی که خونش سود نیست
&&
خون شو آن وقتی که خون مردود نیست
\\
چون گواهی بندگان مقبول نیست
&&
عدل او باشد که بندهٔ غول نیست
\\
گشت ارسلناک شاهد در نذر
&&
زانک بود از کون او حر بن حر
\\
چونک حرم خشم کی بندد مرا
&&
نیست اینجا جز صفات حق در آ
\\
اندر آ کآزاد کردت فضل حق
&&
زانک رحمت داشت بر خشمش سبق
\\
اندر آ اکنون که رستی از خطر
&&
سنگ بودی کیمیا کردت گهر
\\
رسته‌ای از کفر و خارستان او
&&
چون گلی بشکف به سروستان هو
\\
تو منی و من توم ای محتشم
&&
تو علی بودی علی را چون کشم
\\
معصیت کردی به از هر طاعتی
&&
آسمان پیموده‌ای در ساعتی
\\
بس خجسته معصیت کان کرد مرد
&&
نه ز خاری بر دمد اوراق ورد
\\
نه گناه عمر و قصد رسول
&&
می‌کشیدش تا بدرگاه قبول
\\
نه بسحر ساحران فرعونشان
&&
می‌کشید و گشت دولت عونشان
\\
گر نبودی سحرشان و آن جحود
&&
کی کشیدیشان به فرعون عنود
\\
کی بدیدندی عصا و معجزات
&&
معصیت طاعت شد ای قوم عصات
\\
ناامیدی را خدا گردن زدست
&&
چون گنه مانند طاعت آمدست
\\
چون مبدل می‌کند او سیئات
&&
طاعتی‌اش می‌کند رغم وشات
\\
زین شود مرجوم شیطان رجیم
&&
وز حسد او بطرقد گردد دو نیم
\\
او بکوشد تا گناهی پرورد
&&
زان گنه ما را به چاهی آورد
\\
چون ببیند کان گنه شد طاعتی
&&
گردد او را نامبارک ساعتی
\\
اندر آ من در گشادم مر ترا
&&
تف زدی و تحفه دادم مر ترا
\\
مر جفاگر را چنینها می‌دهم
&&
پیش پای چپ چه سان سر می‌نهم
\\
پس وفاگر را چه بخشم تو بدان
&&
گنجها و ملکهای جاودان
\\
\end{longtable}
\end{center}
