\begin{center}
\section*{بخش ۱۰۷ - بقیهٔ قصهٔ مطرب و پیغام رسانیدن امیرالممنین عمر رضی الله عنه باو آنچ هاتف آواز داد}
\label{sec:sh107}
\addcontentsline{toc}{section}{\nameref{sec:sh107}}
\begin{longtable}{l p{0.5cm} r}
باز گرد و حال مطرب گوش‌دار
&&
زانک عاجز گشت مطرب ز انتظار
\\
بانگ آمد مر عمر را کای عمر
&&
بندهٔ ما را ز حاجت باز خر
\\
بنده‌ای داریم خاص و محترم
&&
سوی گورستان تو رنجه کن قدم
\\
ای عمر بر جه ز بیت المال عام
&&
هفتصد دینار در کف نه تمام
\\
پیش او بر کای تو ما را اختیار
&&
این قدر بستان کنون معذور دار
\\
این قدر از بهر ابریشم‌بها
&&
خرج کن چون خرج شد اینجا بیا
\\
پس عمر زان هیبت آواز جست
&&
تا میان را بهر این خدمت ببست
\\
سوی گورستان عمر بنهاد رو
&&
در بغل همیان دوان در جست و جو
\\
گرد گورستان دوانه شد بسی
&&
غیر آن پیر او ندید آنجا کسی
\\
گفت این نبود دگر باره دوید
&&
مانده گشت و غیر آن پیر او ندید
\\
گفت حق فرمود ما را بنده‌ایست
&&
صافی و شایسته و فرخنده‌ایست
\\
پیر چنگی کی بود خاص خدا
&&
حبذا ای سر پنهان حبذا
\\
بار دیگر گرد گورستان بگشت
&&
همچو آن شیر شکاری گرد دشت
\\
چون یقین گشتش که غیر پیر نیست
&&
گفت در ظلمت دل روشن بسیست
\\
آمد او با صد ادب آنجا نشست
&&
بر عمر عطسه فتاد و پیر جست
\\
مر عمر را دید ماند اندر شگفت
&&
عزم رفتن کرد و لرزیدن گرفت
\\
گفت در باطن خدایا از تو داد
&&
محتسب بر پیرکی چنگی فتاد
\\
چون نظر اندر رخ آن پیر کرد
&&
دید او را شرمسار و روی‌زرد
\\
پس عمر گفتش مترس از من مرم
&&
کت بشارتها ز حق آورده‌ام
\\
چند یزدان مدحت خوی تو کرد
&&
تا عمر را عاشق روی تو کرد
\\
پیش من بنشین و مهجوری مساز
&&
تا بگوشت گویم از اقبال راز
\\
حق سلامت می‌کند می‌پرسدت
&&
چونی از رنج و غمان بی‌حدت
\\
نک قراضهٔ چند ابریشم‌بها
&&
خرج کن این را و باز اینجا بیا
\\
پیر لرزان گشت چون این را شنید
&&
دست می‌خایید و بر خود می‌طپید
\\
بانگ می‌زد کای خدای بی‌نظیر
&&
بس که از شرم آب شد بیچاره پیر
\\
چون بسی بگریست و از حد رفت درد
&&
چنگ را زد بر زمین و خرد کرد
\\
گفت ای بوده حجابم از اله
&&
ای مرا تو راه‌زن از شاه‌راه
\\
ای بخورده خون من هفتاد سال
&&
ای ز تو رویم سیه پیش کمال
\\
ای خدای با عطای با وفا
&&
رحم کن بر عمر رفته در جفا
\\
داد حق عمری که هر روزی از آن
&&
کس نداند قیمت آن در جهان
\\
خرج کردم عمر خود را دم بدم
&&
در دمیدم جمله را در زیر و بم
\\
آه کز یاد ره و پردهٔ عراق
&&
رفت از یادم دم تلخ فراق
\\
وای کز تری زیر افکند خرد
&&
خشک شد کشت دل من دل بمرد
\\
وای کز آواز این بیست و چهار
&&
کاروان بگذشت و بیگه شد نهار
\\
ای خدا فریاد زین فریادخواه
&&
داد خواهم نه ز کس زین دادخواه
\\
داد خود از کس نیابم جز مگر
&&
زانک او از من بمن نزدیکتر
\\
کین منی از وی رسد دم دم مرا
&&
پس ورا بینم چو این شد کم مرا
\\
همچو آن کو با تو باشد زرشمر
&&
سوی او داری نه سوی خود نظر
\\
\end{longtable}
\end{center}
