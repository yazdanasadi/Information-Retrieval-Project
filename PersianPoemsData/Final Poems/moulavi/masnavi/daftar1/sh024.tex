\begin{center}
\section*{بخش ۲۴ - بیان خسارت وزیر درین مکر}
\label{sec:sh024}
\addcontentsline{toc}{section}{\nameref{sec:sh024}}
\begin{longtable}{l p{0.5cm} r}
همچو شه نادان و غافل بد وزیر
&&
پنجه می‌زد با قدیم ناگزیر
\\
با چنان قادر خدایی کز عدم
&&
صد چو عالم هست گرداند بدم
\\
صد چو عالم در نظر پیدا کند
&&
چونک چشمت را به خود بینا کند
\\
گر جهان پیشت بزرگ و بی‌بنیست
&&
پیش قدرت ذره‌ای می‌دان که نیست
\\
این جهان خود حبس جانهای شماست
&&
هین روید آن سو که صحرای شماست
\\
این جهان محدود و آن خود بی‌حدست
&&
نقش و صورت پیش آن معنی سدست
\\
صد هزاران نیزهٔ فرعون را
&&
در شکست از موسی با یک عصا
\\
صد هزاران طب جالینوس بود
&&
پیش عیسی و دمش افسوس بود
\\
صد هزاران دفتر اشعار بود
&&
پیش حرف امیی‌اش عار بود
\\
با چنین غالب خداوندی کسی
&&
چون نمیرد گر نباشد او خسی
\\
بس دل چون کوه را انگیخت او
&&
مرغ زیرک با دو پا آویخت او
\\
فهم و خاطر تیز کردن نیست راه
&&
جز شکسته می‌نگیرد فضل شاه
\\
ای بسا گنج آگنان کنج‌کاو
&&
کان خیال‌اندیش را شد ریش گاو
\\
گاو که بود تا تو ریش او شوی
&&
خاک چه بود تا حشیش او شوی
\\
چون زنی از کار بد شد روی زرد
&&
مسخ کرد او را خدا و زهره کرد
\\
عورتی را زهره کردن مسخ بود
&&
خاک و گل گشتن نه مسخست ای عنود
\\
روح می‌بردت سوی چرخ برین
&&
سوی آب و گل شدی در اسفلین
\\
خویشتن را مسخ کردی زین سفول
&&
زان وجودی که بد آن رشک عقول
\\
پس ببین کین مسخ کردن چون بود
&&
پیش آن مسخ این به غایت دون بود
\\
اسپ همت سوی اختر تاختی
&&
آدم مسجود را نشناختی
\\
آخر آدم‌زاده‌ای ای ناخلف
&&
چند پنداری تو پستی را شرف
\\
چند گویی من بگیرم عالمی
&&
این جهان را پر کنم از خود همی
\\
گر جهان پر برف گردد سربسر
&&
تاب خور بگدازدش با یک نظر
\\
وزر او و صد وزیر و صدهزار
&&
نیست گرداند خدا از یک شرار
\\
عین آن تخییل را حکمت کند
&&
عین آن زهراب را شربت کند
\\
آن گمان‌انگیز را سازد یقین
&&
مهرها رویاند از اسباب کین
\\
پرورد در آتش ابراهیم را
&&
ایمنی روح سازد بیم را
\\
از سبب سوزیش من سوداییم
&&
در خیالاتش چو سوفسطاییم
\\
\end{longtable}
\end{center}
