\begin{center}
\section*{بخش ۱۶ - متابعت نصاری وزیر را}
\label{sec:sh016}
\addcontentsline{toc}{section}{\nameref{sec:sh016}}
\begin{longtable}{l p{0.5cm} r}
دل بدو دادند ترسایان تمام
&&
خود چه باشد قوت تقلید عام
\\
در درون سینه مهرش کاشتند
&&
نایب عیسیش می‌پنداشتند
\\
او بسر دجال یک چشم لعین
&&
ای خدا فریاد رس نعم المعین
\\
صد هزاران دام و دانه‌ست ای خدا
&&
ما چو مرغان حریص بی‌نوا
\\
دم بدم ما بستهٔ دام نویم
&&
هر یکی گر باز و سیمرغی شویم
\\
می‌رهانی هر دمی ما را و باز
&&
سوی دامی می‌رویم ای بی‌نیاز
\\
ما درین انبار گندم می‌کنیم
&&
گندم جمع آمده گم می‌کنیم
\\
می‌نیندیشیم آخر ما بهوش
&&
کین خلل در گندمست از مکر موش
\\
موش تا انبار ما حفره زدست
&&
و از فنش انبار ما ویران شدست
\\
اول ای جان دفع شر موش کن
&&
وانگهان در جمع گندم جوش کن
\\
بشنو از اخبار آن صدر الصدور
&&
لا صلوة تم الا بالحضور
\\
گر نه موشی دزد در انبار ماست
&&
گندم اعمال چل ساله کجاست
\\
ریزه‌ریزه صدق هر روزه چرا
&&
جمع می‌ناید درین انبار ما
\\
بس ستارهٔ آتش از آهن جهید
&&
وان دل سوزیده پذرفت و کشید
\\
لیک در ظلمت یکی دزدی نهان
&&
می‌نهد انگشت بر استارگان
\\
می‌کشد استارگان را یک به یک
&&
تا که نفروزد چراغی از فلک
\\
گر هزاران دام باشد در قدم
&&
چون تو با مایی نباشد هیچ غم
\\
چون عنایاتت بود با ما مقیم
&&
کی بود بیمی از آن دزد لئیم
\\
هر شبی از دام تن ارواح را
&&
می‌رهانی می‌کنی الواح را
\\
می‌رهند ارواح هر شب زین قفس
&&
فارغان نه حاکم و محکوم کس
\\
شب ز زندان بی‌خبر زندانیان
&&
شب ز دولت بی‌خبر سلطانیان
\\
نه غم و اندیشهٔ سود و زیان
&&
نه خیال این فلان و آن فلان
\\
حال عارف این بود بی‌خواب هم
&&
گفت ایزد هم رقود زین مرم
\\
خفته از احوال دنیا روز و شب
&&
چون قلم در پنجهٔ تقلیب رب
\\
آنک او پنجه نبیند در رقم
&&
فعل پندارد بجنبش از قلم
\\
شمه‌ای زین حال عارف وا نمود
&&
عقل را هم خواب حسی در ربود
\\
رفته در صحرای بی‌چون جانشان
&&
روحشان آسوده و ابدانشان
\\
وز صفیری باز دام اندر کشی
&&
جمله را در داد و در داور کشی
\\
چونک نور صبحدم سر بر زند
&&
کرکس زرین گردون پر زند
\\
فالق الاصباح اسرافیل‌وار
&&
جمله را در صورت آرد زان دیار
\\
روحهای منبسط را تن کند
&&
هر تنی را باز آبستن کند
\\
اسپ جانها را کند عاری ز زین
&&
سر النوم اخ الموتست این
\\
لیک بهر آنک روز آیند باز
&&
بر نهد بر پایشان بند دراز
\\
تا که روزش واکشد زان مرغزار
&&
وز چراگاه آردش در زیر بار
\\
کاش چون اصحاب کهف این روح را
&&
حفظ کردی یا چو کشتی نوح را
\\
تا ازین طوفان بیداری و هوش
&&
وا رهیدی این ضمیر و چشم و گوش
\\
ای بسی اصحاب کهف اندر جهان
&&
پهلوی تو پیش تو هست این زمان
\\
یار با او غار با او در سرود
&&
مهر بر چشمست و بر گوشت چه سود
\\
\end{longtable}
\end{center}
