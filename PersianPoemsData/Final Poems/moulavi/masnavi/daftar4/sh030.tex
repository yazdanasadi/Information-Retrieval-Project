\begin{center}
\section*{بخش ۳۰ - سبب هجرت ابراهیم ادهم قدس الله سره و ترک ملک خراسان}
\label{sec:sh030}
\addcontentsline{toc}{section}{\nameref{sec:sh030}}
\begin{longtable}{l p{0.5cm} r}
ملک برهم زن تو ادهم‌وار زود
&&
تا بیابی هم‌چو او ملک خلود
\\
خفته بود آن شه شبانه بر سریر
&&
حارسان بر بام اندر دار و گیر
\\
قصد شه از حارسان آن هم نبود
&&
که کند زان دفع دزدان و رنود
\\
او همی دانست که آن کو عادلست
&&
فارغست از واقعه آمن دلست
\\
عدل باشد پاسبان گامها
&&
نه به شب چوبک‌زنان بر بامها
\\
لیک بد مقصودش از بانگ رباب
&&
هم‌چو مشتاقان خیال آن خطاب
\\
نالهٔ سرنا و تهدید دهل
&&
چیزکی ماند بدان ناقور کل
\\
پس حکیمان گفته‌اند این لحنها
&&
از دوار چرخ بگرفتیم ما
\\
بانگ گردشهای چرخست این که خلق
&&
می‌سرایندش به طنبور و به حلق
\\
مؤمنان گویند که آثار بهشت
&&
نغز گردانید هر آواز زشت
\\
ما همه اجزای آدم بوده‌ایم
&&
در بهشت آن لحنها بشنوده‌ایم
\\
گرچه بر ما ریخت آب و گل شکی
&&
یادمان آمد از آنها چیزکی
\\
لیک چون آمیخت با خاک کرب
&&
کی دهند این زیر و آن بم آن طرب
\\
آب چون آمیخت با بول و کمیز
&&
گشت ز آمیزش مزاجش تلخ و تیز
\\
چیزکی از آب هستش در جسد
&&
بول گیرش آتشی را می‌کشد
\\
گر نجس شد آب این طبعش بماند
&&
که آتش غم را به طبع خود نشاند
\\
پس غدای عاشقان آمد سماع
&&
که درو باشد خیال اجتماع
\\
قوتی گیرد خیالات ضمیر
&&
بلک صورت گردد از بانگ و صفیر
\\
آتش عشق از نواها گشت تیز
&&
آن چنان که آتش آن جوزریز
\\
\end{longtable}
\end{center}
