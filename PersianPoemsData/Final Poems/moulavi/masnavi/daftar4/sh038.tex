\begin{center}
\section*{بخش ۳۸ - قصهٔ یاری خواستن حلیمه از بتان چون عقیب فطام مصطفی را علیه‌السلام گم کرد و لرزیدن و سجدهٔ بتان و گواهی دادن ایشان بر عظمت کار  مصطفی صلی‌الله علیه و سلم}
\label{sec:sh038}
\addcontentsline{toc}{section}{\nameref{sec:sh038}}
\begin{longtable}{l p{0.5cm} r}
قصهٔ راز حلیمه گویمت
&&
تا زداید داستان او غمت
\\
مصطفی را چون ز شیر او باز کرد
&&
بر کفش برداشت چون ریحان و ورد
\\
می‌گریزانیدش از هر نیک و بد
&&
تا سپارد آن شهنشه را به جد
\\
چون همی آورد امانت را ز بیم
&&
شد به کعبه و آمد او اندر حطیم
\\
از هوا بشنید بانگی کای حطیم
&&
تافت بر تو آفتابی بس عظیم
\\
ای حطیم امروز آید بر تو زود
&&
صد هزاران نور از خورشید جود
\\
ای حطیم امروز آرد در تو رخت
&&
محتشم شاهی که پیک اوست بخت
\\
ای حطیم امروز بی‌شک از نوی
&&
منزل جانهای بالایی شوی
\\
جان پاکان طلب طلب و جوق جوق
&&
آیدت از هر نواحی مست شوق
\\
گشت حیران آن حلیمه زان صدا
&&
نه کسی در پیش نه سوی قفا
\\
شش جهت خالی ز صورت وین ندا
&&
شد پیاپی آن ندا را جان فدا
\\
مصطفی را بر زمین بنهاد او
&&
تا کند آن بانگ خوش را جست و جو
\\
چشم می‌انداخت آن دم سو به سو
&&
که کجا است این شه اسرارگو
\\
کین چنین بانگ بلند از چپ و راست
&&
می‌رسد یا رب رساننده کجاست
\\
چون ندید او خیره و نومید شد
&&
جسم لرزان هم‌چو شاخ بید شد
\\
باز آمد سوی آن طفل رشید
&&
مصطفی را بر مکان خود ندید
\\
حیرت اندر حیرت آمد بر دلش
&&
گشت بس تاریک از غم منزلش
\\
سوی منزلها دوید و بانگ داشت
&&
که کی بر دردانه‌ام غارت گماشت
\\
مکیان گفتند ما را علم نیست
&&
ما ندانستیم که آنجا کودکیست
\\
ریخت چندان اشک و کرد او بس فغان
&&
که ازو گریان شدند آن دیگران
\\
سینه کوبان آن چنان بگریست خوش
&&
که اختران گریان شدند از گریه‌اش
\\
\end{longtable}
\end{center}
