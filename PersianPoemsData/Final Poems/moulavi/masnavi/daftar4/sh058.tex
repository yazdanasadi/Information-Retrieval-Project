\begin{center}
\section*{بخش ۵۸ - چالیش عقل با نفس هم چون تنازع  مجنون با ناقه میل مجنون  سوی حره میل ناقه واپس سوی کره  چنانک گفت مجنون هوا ناقتی خلفی و قدامی الهوی  و انی و ایاها لمختلفان}
\label{sec:sh058}
\addcontentsline{toc}{section}{\nameref{sec:sh058}}
\begin{longtable}{l p{0.5cm} r}
هم‌چو مجنون‌اند و چون ناقه‌ش یقین
&&
می‌کشد آن پیش و این واپس به کین
\\
میل مجنون پیش آن لیلی روان
&&
میل ناقه پس پی کره دوان
\\
یک دم ار مجنون ز خود غافل بدی
&&
ناقه گردیدی و واپس آمدی
\\
عشق و سودا چونک پر بودش بدن
&&
می‌نبودش چاره از بی‌خود شدن
\\
آنک او باشد مراقب عقل بود
&&
عقل را سودای لیلی در ربود
\\
لیک ناقه بس مراقب بود و چست
&&
چون بدیدی او مهار خویش سست
\\
فهم کردی زو که غافل گشت و دنگ
&&
رو سپس کردی به کره بی‌درنگ
\\
چون به خود باز آمدی دیدی ز جا
&&
کو سپس رفتست بس فرسنگها
\\
در سه روزه ره بدین احوالها
&&
ماند مجنون در تردد سالها
\\
گفت ای ناقه چو هر دو عاشقیم
&&
ما دو ضد پس همره نالایقیم
\\
نیستت بر وفق من مهر و مهار
&&
کرد باید از تو صحبت اختیار
\\
این دو همره یکدگر را راه‌زن
&&
گمره آن جان کو فرو ناید ز تن
\\
جان ز هجر عرش اندر فاقه‌ای
&&
تن ز عشق خاربن چون ناقه‌ای
\\
جان گشاید سوی بالا بالها
&&
در زده تن در زمین چنگالها
\\
تا تو با من باشی ای مردهٔ وطن
&&
پس ز لیلی دور ماند جان من
\\
روزگارم رفت زین گون حالها
&&
هم‌چو تیه و قوم موسی سالها
\\
خطوتینی بود این ره تا وصال
&&
مانده‌ام در ره ز شستت شصت سال
\\
راه نزدیک و بماندم سخت دیر
&&
سیر گشتم زین سواری سیرسیر
\\
سرنگون خود را از اشتر در فکند
&&
گفت سوزیدم ز غم تا چندچند
\\
تنگ شد بر وی بیابان فراخ
&&
خویشتن افکند اندر سنگلاخ
\\
آنچنان افکند خود را سخت زیر
&&
که مخلخل گشت جسم آن دلیر
\\
چون چنان افکند خود را سوی پست
&&
از قضا آن لحظه پایش هم شکست
\\
پای را بر بست و گفتا گو شوم
&&
در خم چوگانش غلطان می‌روم
\\
زین کند نفرین حکیم خوش‌دهن
&&
بر سواری کو فرو ناید ز تن
\\
عشق مولی کی کم از لیلی بود
&&
گوی گشتن بهر او اولی بود
\\
گوی شو می‌گرد بر پهلوی صدق
&&
غلط غلطان در خم چوگان عشق
\\
کین سفر زین پس بود جذب خدا
&&
وان سفر بر ناقه باشد سیر ما
\\
این چنین سیریست مستثنی ز جنس
&&
کان فزود از اجتهاد جن و انس
\\
این چنین جذبیست نی هر جذب عام
&&
که نهادش فضل احمد والسلام
\\
\end{longtable}
\end{center}
