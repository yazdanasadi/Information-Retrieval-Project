\begin{center}
\section*{بخش ۹۷ - شرح کردن موسی علیه‌السلام آن  چهار فضیلت را جهت  پای مزد ایمان فرعون}
\label{sec:sh097}
\addcontentsline{toc}{section}{\nameref{sec:sh097}}
\begin{longtable}{l p{0.5cm} r}
گفت موسی که اولین آن چهار
&&
صحتی باشد تنت را پایدار
\\
این علل‌هایی که در طب گفته‌اند
&&
دور باشد از تنت ای ارجمند
\\
ثانیا باشد ترا عمر دراز
&&
که اجل دارد ز عمرت احتراز
\\
وین نباشد بعد عمر مستوی
&&
که بناکام از جهان بیرون روی
\\
بلک خواهان اجل چون طفل شیر
&&
نه ز رنجی که ترا دارد اسیر
\\
مرگ‌جو باشی ولی نه از عجز رنج
&&
بلک بینی در خراب خانه گنج
\\
پس به دست خویش گیری تیشه‌ای
&&
می‌زنی بر خانه بی‌اندیشه‌ای
\\
که حجاب گنج بینی خانه را
&&
مانع صد خرمن این یک دانه را
\\
پس در آتش افکنی این دانه را
&&
پیش گیری پیشهٔ مردانه را
\\
ای به یک برگی ز باغی مانده
&&
هم‌چو کرمی برگش از رز رانده
\\
چون کرم این کرم را بیدار کرد
&&
اژدهای جهل را این کرم خورد
\\
کرم کرمی شد پر از میوه و درخت
&&
این چنین تبدیل گردد نیکبخت
\\
\end{longtable}
\end{center}
