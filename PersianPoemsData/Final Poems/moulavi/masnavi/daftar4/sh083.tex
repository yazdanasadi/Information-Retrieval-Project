\begin{center}
\section*{بخش ۸۳ - قصهٔ آن آبگیر و صیادان و آن سه ماهی یکی عاقل و یکی نیم عاقل وان دگر مغرور و ابله مغفل لاشی و عاقبت هر سه}
\label{sec:sh083}
\addcontentsline{toc}{section}{\nameref{sec:sh083}}
\begin{longtable}{l p{0.5cm} r}
قصهٔ آن آبگیرست ای عنود
&&
که درو سه ماهی اشگرف بود
\\
در کلیله خوانده باشی لیک آن
&&
قشر قصه باشد و این مغز جان
\\
چند صیادی سوی آن آبگیر
&&
برگذشتند و بدیدند آن ضمیر
\\
پس شتابیدند تا دام آورند
&&
ماهیان واقف شدند و هوشمند
\\
آنک عاقل بود عزم راه کرد
&&
عزم راه مشکل ناخواه کرد
\\
گفت با اینها ندارم مشورت
&&
که یقین سستم کنند از مقدرت
\\
مهر زاد و بوم بر جانشان تند
&&
کاهلی و جهلشان بر من زند
\\
مشورت را زنده‌ای باید نکو
&&
که ترا زنده کند وان زنده کو
\\
ای مسافر با مسافر رای زن
&&
زانک پایت لنگ دارد رای زن
\\
از دم حب الوطن بگذر مه‌ایست
&&
که وطن آن سوست جان این سوی نیست
\\
گر وطن خواهی گذر آن سوی شط
&&
این حدیث راست را کم خوان غلط
\\
\end{longtable}
\end{center}
