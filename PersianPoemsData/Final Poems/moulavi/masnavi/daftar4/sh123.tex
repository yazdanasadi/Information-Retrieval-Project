\begin{center}
\section*{بخش ۱۲۳ - حکایت آن زاهد کی در سال قحط شاد و خندان بود با مفلسی و بسیاری عیان و خلق می‌مردند از گرسنگی گفتندش چه هنگام  شادیست کی هنگام صد تعزیت است گفت مرا باری نیست}
\label{sec:sh123}
\addcontentsline{toc}{section}{\nameref{sec:sh123}}
\begin{longtable}{l p{0.5cm} r}
هم‌چنان کن زاهد اندر سال قحط
&&
بود او خندان و گریان جمله رهط
\\
پس بگفتندش چه جای خنده است
&&
قحط بیخ مؤمنان بر کنده است
\\
رحمت از ما چشم خود بر دوختست
&&
ز آفتاب تیز صحرا سوختست
\\
کشت و باغ و رز سیه استاده است
&&
در زمین نم نیست نه بالا نه پست
\\
خل می‌میرند زین قحط و عذاب
&&
ده ده و صد صد چو ماهی دور از آب
\\
بر مسلمانان نمی‌آری تو رحم
&&
مؤمنان خویشند و یک تن شحم و لحم
\\
رنج یک جزوی ز تن رنج همه‌ست
&&
گر دم صلحست یا خود ملحمه‌ست
\\
گفت در چشم شما قحطست این
&&
پیش چشمم چون بهشتست این زمین
\\
من همی‌بینم بهر دشت و مکان
&&
خوشه‌ها انبه رسیده تا میان
\\
خوشه‌ها در موج از باد صبا
&&
پر بیابان سبزتر از گندنا
\\
ز آزمون من دست بر وی می‌زنم
&&
دست و چشم خویش را چون بر کنم
\\
یار فرعون تنید ای قوم دون
&&
زان نماید مر شما را نیل خون
\\
یار موسی خرد گردید زود
&&
تا نماند خون بینید آب رود
\\
با پدر از تو جفایی می‌رود
&&
آن پدر در چشم تو سگ می‌شود
\\
آن پدر سگ نیست تاثیر جفاست
&&
که چنان حرمت نظر را سگ نماست
\\
گرگ می‌دیدند یوسف را به چشم
&&
چونک اخوان را حسودی بود و خشم
\\
با پدر چون صلح کردی خشم رفت
&&
آن سگی شد گشت بابا یار تفت
\\
\end{longtable}
\end{center}
