\begin{center}
\section*{بخش ۴۶ - باز آمدن آن شاعر بعد چند سال به امید همان صله و هزار دینار فرمودن بر قاعدهٔ خویش و گفتن وزیر نو هم حسن نام شاه را کی این سخت بسیارست و ما را خرجهاست و خزینه خالیست و من او را بده یک آن خشنود کنم}
\label{sec:sh046}
\addcontentsline{toc}{section}{\nameref{sec:sh046}}
\begin{longtable}{l p{0.5cm} r}
بعد سالی چند بهر رزق و کشت
&&
شاعر از فقر و عوز محتاج گشت
\\
گفت وقت فقر و تنگی دو دست
&&
جست و جوی آزموده بهترست
\\
درگهی را که آزمودم در کرم
&&
حاجت نو را بدان جانب برم
\\
معنی الله گفت آن سیبویه
&&
یولهون فی الحوائج هم لدیه
\\
گفت الهنا فی حوائجنا الیک
&&
والتمسناها وجدناها لدیک
\\
صد هزاران عاقل اندر وقت درد
&&
جمله نالان پیش آن دیان فرد
\\
هیچ دیوانهٔ فلیوی این کند
&&
بر بخیلی عاجزی کدیه تند
\\
گر ندیدندی هزاران بار بیش
&&
عاقلان کی جان کشیدندیش پیش
\\
بلک جملهٔ ماهیان در موجها
&&
جملهٔ پرندگان بر اوجها
\\
پیل و گرگ و حیدر اشکار نیز
&&
اژدهای زفت و مور و مار نیز
\\
بلک خاک و باد و آب و هر شرار
&&
مایه زو یابند هم دی هم بهار
\\
هر دمش لابه کند این آسمان
&&
که فرو مگذارم ای حق یک زمان
\\
استن من عصمت و حفظ تو است
&&
جمله مطوی یمین آن دو دست
\\
وین زمین گوید که دارم بر قرار
&&
ای که بر آبم تو کردستی سوار
\\
جملگان کیسه ازو بر دوختند
&&
دادن حاجت ازو آموختند
\\
هر نبیی زو برآورده برات
&&
استعینوا منه صبرا او صلات
\\
هین ازو خواهید نه از غیر او
&&
آب در یم جو مجو در خشک جو
\\
ور بخواهی از دگر هم او دهد
&&
بر کف میلش سخا هم او نهد
\\
آنک معرض را ز زر قارون کند
&&
رو بدو آری به طاعت چون کند
\\
بار دیگر شاعر از سودای داد
&&
روی سوی آن شه محسن نهاد
\\
هدیهٔ شاعر چه باشد شعر نو
&&
پیش محسن آرد و بنهد گرو
\\
محسنان با صد عطا و جود و بر
&&
زر نهاده شاعران را منتظر
\\
پیششان شعری به از صدتنگ شعر
&&
خاصه شاعر کو گهر آرد ز قعر
\\
آدمی اول حریص نان بود
&&
زانک قوت و نان ستون جان بود
\\
سوی کسب و سوی غصب و صد حیل
&&
جان نهاده بر کف از حرص و امل
\\
چون بنادر گشت مستغنی ز نان
&&
عاشق نامست و مدح شاعران
\\
تا که اصل و فصل او را بر دهند
&&
در بیان فضل او منبر نهند
\\
تا که کر و فر و زر بخشی او
&&
هم‌چو عنبر بو دهد در گفت و گو
\\
خلق ما بر صورت خود کرد حق
&&
وصف ما از وصف او گیرد سبق
\\
چونک آن خلاق شکر و حمدجوست
&&
آدمی را مدح‌جویی نیز خوست
\\
خاصه مرد حق که در فضلست چست
&&
پر شود زان باد چون خیک درست
\\
ور نباشد اهل زان باد دروغ
&&
خیک بدریدست کی گیرد فروغ
\\
این مثل از خود نگفتم ای رفیق
&&
سرسری مشنو چو اهلی و مفیق
\\
این پیمبر گفت چون بشنید قدح
&&
که چرا فربه شود احمد به مدح
\\
رفت شاعر پیش آن شاه و ببرد
&&
شعر اندر شکر احسان کان نمرد
\\
محسنان مردند و احسانها بماند
&&
ای خنک آن را که این مرکب براند
\\
ظالمان مردند و ماند آن ظلمها
&&
وای جانی کو کند مکر و دها
\\
گفت پیغامبر خنک آن را که او
&&
شد ز دنیا ماند ازو فعل نکو
\\
مرد محسن لیک احسانش نمرد
&&
نزد یزدان دین و احسان نیست خرد
\\
وای آنکو مرد و عصیانش نمود
&&
تا نپنداری به مرگ او جان ببرد
\\
این رها کن زانک شاعر بر گذر
&&
وام‌دارست و قوی محتاج زر
\\
برد شاعر شعر سوی شهریار
&&
بر امید بخشش و احسان پار
\\
نازنین شعری پر از در درست
&&
بر امید و بوی اکرام نخست
\\
شاه هم بر خوی خود گفتش هزار
&&
چون چنین بد عادت آن شهریار
\\
لیک این بار آن وزیر پر ز جود
&&
بر براق عز ز دنیا رفته بود
\\
بر مقام او وزیر نو رئیس
&&
گشته لیکن سخت بی‌رحم و خسیس
\\
گفت ای شه خرجها داریم ما
&&
شاعری را نبود این بخشش جزا
\\
من به ربع عشر این ای مغتنم
&&
مرد شاعر را خوش و راضی کنم
\\
خلق گفتندش که او از پیش‌دست
&&
ده هزاران زین دلاور برده است
\\
بعد شکر کلک خایی چون کند
&&
بعد سلطانی گدایی چون کند
\\
گفت بفشارم ورا اندر فشار
&&
تا شود زار و نزار از انتظار
\\
آنگه ار خاکش دهم از راه من
&&
در رباید هم‌چو گلبرگ از چمن
\\
این به من بگذار که استادم درین
&&
گر تقاضاگر بود هر آتشین
\\
از ثریا گر بپرد تا ثری
&&
نرم گردد چون ببیند او مرا
\\
گفت سلطانش برو فرمان تراست
&&
لیک شادش کن که نیکوگوی ماست
\\
گفت او را و دو صد اومیدلیس
&&
تو به من بگذار این بر من نویس
\\
پس فکندش صاحب اندر انتظار
&&
شد زمستان و دی و آمد بهار
\\
شاعر اندر انتظارش پیر شد
&&
پس زبون این غم و تدبیر شد
\\
گفت اگر زر نه که دشنامم دهی
&&
تا رهد جانم ترا باشم رهی
\\
انتظارم کشت باری گو برو
&&
تا رهد این جان مسکین از گرو
\\
بعد از آنش داد ربع عشر آن
&&
ماند شاعر اندر اندیشهٔ گران
\\
کانچنان نقد و چنان بسیار بود
&&
این که دیر اشکفت دستهٔ خار بود
\\
پس بگفتندش که آن دستور راد
&&
رفت از دنیا خدا مزدت دهاد
\\
که مضاعف زو همی‌شد آن عطا
&&
کم همی‌افتاد بخشش را خطا
\\
این زمان او رفت و احسان را ببرد
&&
او نمرد الحق بلی احسان بمرد
\\
رفت از ما صاحب راد و رشید
&&
صاحب سلاخ درویشان رسید
\\
رو بگیر این را و زینجا شب گریز
&&
تا نگیرد با تو این صاحب‌ستیز
\\
ما به صد حیلت ازو این هدیه را
&&
بستدیم ای بی‌خبر از جهد ما
\\
رو بایشان کرد و گفت ای مشفقان
&&
از کجا آمد بگویید این عوان
\\
چیست نام این وزیر جامه‌کن
&&
قوم گفتندش که نامش هم حسن
\\
گفت یا رب نام آن و نام این
&&
چون یکی آمد دریغ ای رب دین
\\
آن حسن نامی که از یک کلک او
&&
صد وزیر و صاحب آید جودخو
\\
این حسن کز ریش زشت این حسن
&&
می‌توان بافید ای جان صد رسن
\\
بر چنین صاحب چو شه اصغا کند
&&
شاه و ملکش را ابد رسوا کند
\\
\end{longtable}
\end{center}
