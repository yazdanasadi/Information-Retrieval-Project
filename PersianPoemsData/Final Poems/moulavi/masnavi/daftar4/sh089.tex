\begin{center}
\section*{بخش ۸۹ - در بیان آنک وهم قلب عقلست و ستیزهٔ  اوست بدو ماند و او نیست و قصهٔ مجاوبات موسی علیه‌السلام کی صاحب عقل بود با فرعون کی صاحب وهم بود}
\label{sec:sh089}
\addcontentsline{toc}{section}{\nameref{sec:sh089}}
\begin{longtable}{l p{0.5cm} r}
عقل ضد شهوتست ای پهلوان
&&
آنک شهوت می‌تند عقلش مخوان
\\
وهم خوانش آنک شهوت را گداست
&&
وهم قلب نقد زر عقلهاست
\\
بی‌محک پیدا نگردد وهم و عقل
&&
هر دو را سوی محک کن زود نقل
\\
این محک قرآن و حال انبیا
&&
چون منحک مر قلب را گوید بیا
\\
تا ببینی خویش را ز آسیب من
&&
که نه‌ای اهل فراز و شیب من
\\
عقل را گر اره‌ای سازد دو نیم
&&
هم‌چو زر باشد در آتش او بسیم
\\
وهم مر فرعون عالم‌سوز را
&&
عقل مر موسی به جان افروز را
\\
رفت موسی بر طریق نیستی
&&
گفت فرعونش بگو تو کیستی
\\
گفت من عقلم رسول ذوالجلال
&&
حجةالله‌ام امانم از ضلال
\\
گفت نی خامش رها کن های هو
&&
نسبت و نام قدیمت را بگو
\\
گفت که نسبت مر از خاکدانش
&&
نام اصلم کمترین بندگانش
\\
بنده‌زادهٔ آن خداوند وحید
&&
زاده از پشت جواری و عبید
\\
نسبت اصلم ز خاک و آب و گل
&&
آب و گل را داد یزدان جان و دل
\\
مرجع این جسم خاکم هم به خاک
&&
مرجع تو هم به خاک ای سهمناک
\\
اصل ما و اصل جمله سرکشان
&&
هست از خاکی و آن را صد نشان
\\
که مدد از خاک می‌گیرد تنت
&&
از غذایی خاک پیچد گردنت
\\
چون رود جان می‌شود او باز خاک
&&
اندر آن گور مخوف سهمناک
\\
هم تو و هم ما و هم اشباه تو
&&
خاک گردند و نماند جاه تو
\\
گفت غیر این نسب نامیت هست
&&
مر ترا آن نام خود اولیترست
\\
بندهٔ فرعون و بندهٔ بندگانش
&&
که ازو پرورد اول جسم و جانش
\\
بندهٔ یاغی طاغی ظلوم
&&
زین وطن بگریخته از فعل شوم
\\
خونی و غداری و حق‌ناشناس
&&
هم برین اوصاف خود می‌کن قیاس
\\
در غریبی خوار و درویش و خلق
&&
که ندانستی سپاس ما و حق
\\
گفت حاشا که بود با آن ملیک
&&
در خداوندی کسی دیگر شریک
\\
واحد اندر ملک او را یار نی
&&
بندگانش را جز او سالار نی
\\
نیست خلقش را دگر کس مالکی
&&
شرکتش دعوی کند جز هالکی
\\
نقش او کردست و نقاش من اوست
&&
غیر اگر دعوی کند او ظلم‌جوست
\\
تو نتوانی ابروی من ساختن
&&
چون توانی جان من بشناختن
\\
بلک آن غدار و آن طاغی توی
&&
که کنی با حق دعوی دوی
\\
گر بکشتم من عوانی را به سهو
&&
نه برای نفس کشتم نه به لهو
\\
من زدم مشتی و ناگاه اوفتاد
&&
آنک جانش خود نبد جانی بداد
\\
من سگی کشتم تو مرسل‌زادگان
&&
صدهزاران طفل بی‌جرم و زیان
\\
کشته‌ای و خونشان در گردنت
&&
تا چه آید بر تو زین خون خوردنت
\\
کشته‌ای ذریت یعقوب را
&&
بر امید قتل من مطلوب را
\\
کوری تو حق مرا خود برگزید
&&
سرنگون شد آنچ نفست می‌پزید
\\
گفت اینها را بهل بی‌هیچ شک
&&
این بود حق من و نان و نمک
\\
که مرا پیش حشر خواری کنی
&&
روز روشن بر دلم تاری کنی
\\
گفت خواری قیامت صعب‌تر
&&
گر نداری پاس من در خیر و شر
\\
زخم کیکی را نمی‌توانی کشید
&&
زخم ماری را تو چون خواهی چشید
\\
ظاهرا کار تو ویران می‌کنم
&&
لیک خاری را گلستان می‌کنم
\\
\end{longtable}
\end{center}
