\begin{center}
\section*{بخش ۳۶ - آزاد شدن بلقیس از ملک و مست  شدن او از شوق ایمان و التفات همت او از همهٔ ملک منقطع شدن وقت هجرت الا از تخت}
\label{sec:sh036}
\addcontentsline{toc}{section}{\nameref{sec:sh036}}
\begin{longtable}{l p{0.5cm} r}
چون سلیمان سوی مرغان سبا
&&
یک صفیری کرد بست آن جمله را
\\
جز مگر مرغی که بد بی‌جان و پر
&&
یا چو ماهی گنگ بود از اصل کر
\\
نی غلط گفتم که کر گر سر نهد
&&
پیش وحی کبریا سمعش دهد
\\
چونک بلقیس از دل و جان عزم کرد
&&
بر زمان رفته هم افسوس خورد
\\
ترک مال و ملک کرد او آن چنان
&&
که بترک نام و ننگ آن عاشقان
\\
آن غلامان و کنیزان بناز
&&
پیش چشمش هم‌چو پوسیده پیاز
\\
باغها و قصرها و آب رود
&&
پیش چشم از عشق گلحن می‌نمود
\\
عشق در هنگام استیلا و خشم
&&
زشت گرداند لطیفان را به چشم
\\
هر زمرد را نماید گندنا
&&
غیرت عشق این بود معنی لا
\\
لااله الا هو اینست ای پناه
&&
که نماید مه ترا دیگ سیاه
\\
هیچ مال و هیچ مخزن هیچ رخت
&&
می دریغش نامد الا جز که تخت
\\
پس سلیمان از دلش آگاه شد
&&
کز دل او تا دل او راه شد
\\
آن کسی که بانگ موران بشنود
&&
هم فغان سر دوران بشنود
\\
آنک گوید راز قالت نملة
&&
هم بداند راز این طاق کهن
\\
دید از دورش که آن تسلیم کیش
&&
تلخش آمد فرقت آن تخت خویش
\\
گر بگویم آن سبب گردد دراز
&&
که چرا بودش به تخت آن عشق و ساز
\\
گرچه این کلک قلم خود بی‌حسیست
&&
نیست جنس کاتب او را مونسیست
\\
هم‌چنین هر آلت پیشه‌وری
&&
هست بی‌جان مونس جانوری
\\
این سبب را من معین گفتمی
&&
گر نبودی چشم فهمت را نمی
\\
از بزرگی تخت کز حد می‌فزود
&&
نقل کردن تخت را امکان نبود
\\
خرده کاری بود و تفریقش خطر
&&
هم‌چو اوصال بدن با همدگر
\\
پس سلیمان گفت گر چه فی‌الاخیر
&&
سرد خواهد شد برو تاج و سریر
\\
چون ز وحدت جان برون آرد سری
&&
جسم را با فر او نبود فری
\\
چون برآید گوهر از قعر بحار
&&
بنگری اندر کف و خاشاک خوار
\\
سر بر آرد آفتاب با شرر
&&
دم عقرب را کی سازد مستقر
\\
لیک خود با این همه بر نقد حال
&&
جست باید تخت او را انتقال
\\
تا نگردد خسته هنگام لقا
&&
کودکانه حاجتش گردد روا
\\
هست بر ما سهل و او را بس عزیز
&&
تا بود بر خوان حوران دیو نیز
\\
عبرت جانش شود آن تخت ناز
&&
هم‌چو دلق و چارقی پیش ایاز
\\
تا بداند در چه بود آن مبتلا
&&
از کجاها در رسید او تا کجا
\\
خاک را و نطفه را و مضغه را
&&
پیش چشم ما همی‌دارد خدا
\\
کز کجا آوردمت ای بدنیت
&&
که از آن آید همی خفریقیت
\\
تو بر آن عاشق بدی در دور آن
&&
منکر این فضل بودی آن زمان
\\
این کرم چون دفع آن انکار تست
&&
که میان خاک می‌کردی نخست
\\
حجت انکار شد انشار تو
&&
از دوا بدتر شد این بیمار تو
\\
خاک را تصویر این کار از کجا
&&
نطفه را خصمی و انکار از کجا
\\
چون در آن دم بی‌دل و بی‌سر بدی
&&
فکرت و انکار را منکر بدی
\\
از جمادی چونک انکارت برست
&&
هم ازین انکار حشرت شد درست
\\
پس مثال تو چو آن حلقه‌زنیست
&&
کز درونش خواجه گوید خواجه نیست
\\
حلقه‌زن زین نیست دریابد که هست
&&
پس ز حلقه بر ندارد هیچ دست
\\
پس هم انکارت مبین می‌کند
&&
کز جماد او حشر صد فن می‌کند
\\
چند صنعت رفت ای انکار تا
&&
آب و گل انکار زاد از هل اتی
\\
آب وگل می‌گفت خود انکار نیست
&&
بانگ می‌زد بی‌خبر که اخبار نیست
\\
من بگویم شرح این از صد طریق
&&
لیک خاطر لغزد از گفت دقیق
\\
\end{longtable}
\end{center}
