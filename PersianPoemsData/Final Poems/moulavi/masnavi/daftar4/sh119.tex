\begin{center}
\section*{بخش ۱۱۹ - عروس آوردن پادشاه فرزند خود را از خوف انقطاع نسل}
\label{sec:sh119}
\addcontentsline{toc}{section}{\nameref{sec:sh119}}
\begin{longtable}{l p{0.5cm} r}
پس عروسی خواست باید بهر او
&&
تا نماید زین تزوج نسل رو
\\
گر رود سوی فنا این باز باز
&&
فرخ او گردد ز بعد باز باز
\\
صورت او باز گر زینجا رود
&&
معنی او در ولد باقی بود
\\
بهر این فرمود آن شاه نبیه
&&
مصطفی که الولد سر ابیه
\\
بهر این معنی همه خلق از شغف
&&
می‌بیاموزند طفلان را حرف
\\
تا بماند آن معانی در جهان
&&
چون شود آن قالب ایشان نهان
\\
حق به حکمت حرصشان دادست جد
&&
بهر رشد هر صغیر مستعد
\\
من هم از بهر دوام نسل خویش
&&
جفت خواهم پور خود را خوب کیش
\\
دختری خواهم ز نسل صالحی
&&
نی ز نسل پادشاهی کالحی
\\
شاه خود این صالحست آزاد اوست
&&
نی اسیر حرص فرجست و گلوست
\\
مر اسیران را لقب کردند شاه
&&
عکس چون کافور نام آن سیاه
\\
شد مفازه بادیهٔ خون‌خوار نام
&&
نیکبخت آن پیس را کردند عام
\\
بر اسیر شهوت و حرص و امل
&&
بر نوشته میر یا صدر اجل
\\
آن اسیران اجل را عام داد
&&
نام امیران اجل اندر بلاد
\\
صدر خوانندش که در صف نعال
&&
جان او پستست یعنی جاه و مال
\\
شاه چون با زاهدی خویشی گزید
&&
این خبر در گوش خاتونان رسید
\\
\end{longtable}
\end{center}
