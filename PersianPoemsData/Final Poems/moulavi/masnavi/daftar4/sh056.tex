\begin{center}
\section*{بخش ۵۶ - در تفسیر این حدیث مصطفی علیه‌السلام کی ان الله تعالی خلق الملائکة و رکب  فیهم العقل و خلق البهائم و رکب فیها الشهوة و خلق بنی آدم و رکب فیهم العقل و الشهوة فمن غلب عقله شهوته فهو اعلی من الملائکة و من غلب شهوته عقله فهو ادنی من البهائم}
\label{sec:sh056}
\addcontentsline{toc}{section}{\nameref{sec:sh056}}
\begin{longtable}{l p{0.5cm} r}
در حدیث آمد که یزدان مجید
&&
خلق عالم را سه گونه آفرید
\\
یک گره را جمله عقل و علم و جود
&&
آن فرشته‌ست او نداند جز سجود
\\
نیست اندر عنصرش حرص و هوا
&&
نور مطلق زنده از عشق خدا
\\
یک گروه دیگر از دانش تهی
&&
هم‌چو حیوان از علف در فربهی
\\
او نبیند جز که اصطبل و علف
&&
از شقاوت غافلست و از شرف
\\
این سوم هست آدمی‌زاد و بشر
&&
نیم او ز افرشته و نیمیش خر
\\
نیم خر خود مایل سفلی بود
&&
نیم دیگر مایل عقلی بود
\\
آن دو قوم آسوده از جنگ و حراب
&&
وین بشر با دو مخالف در عذاب
\\
وین بشر هم ز امتحان قسمت شدند
&&
آدمی شکلند و سه امت شدند
\\
یک گره مستغرق مطلق شدست
&&
هم‌چو عیسی با ملک ملحق شدست
\\
نقش آدم لیک معنی جبرئیل
&&
رسته از خشم و هوا و قال و قیل
\\
از ریاضت رسته وز زهد و جهاد
&&
گوییا از آدمی او خود نزاد
\\
قسم دیگر با خران ملحق شدند
&&
خشم محض و شهوت مطلق شدند
\\
وصف جبریلی دریشان بود رفت
&&
تنگ بود آن خانه و آن وصف زفت
\\
مرده گردد شخص کو بی‌جان شود
&&
خر شود چون جان او بی‌آن شود
\\
زانک جانی کان ندارد هست پست
&&
این سخن حقست و صوفی گفته است
\\
او ز حیوانها فزون‌تر جان کند
&&
در جهان باریک کاریها کند
\\
مکر و تلبیسی که او داند تنید
&&
آن ز حیوان دیگر ناید پدید
\\
جامه‌های زرکشی را بافتن
&&
درها از قعر دریا یافتن
\\
خرده‌کاریهای علم هندسه
&&
یا نجوم و علم طب و فلسفه
\\
که تعلق با همین دنیاستش
&&
ره به هفتم آسمان بر نیستش
\\
این همه علم بنای آخرست
&&
که عماد بود گاو و اشترست
\\
بهر استبقای حیوان چند روز
&&
نام آن کردند این گیجان رموز
\\
علم راه حق و علم منزلش
&&
صاحب دل داند آن را با دلش
\\
پس درین ترکیب حیوان لطیف
&&
آفرید و کرد با دانش الیف
\\
نام کالانعام کرد آن قوم را
&&
زانک نسبت کو بیقظه نوم را
\\
روح حیوانی ندارد غیر نوم
&&
حسهای منعکس دارند قوم
\\
یقظه آمد نوم حیوانی نماند
&&
انعکاس حس خود از لوح خواند
\\
هم‌چو حس آنک خواب او را ربود
&&
چون شد او بیدار عکسیت نمود
\\
لاجرم اسفل بود از سافلین
&&
ترک او کن لا احب الافلین
\\
\end{longtable}
\end{center}
