\begin{center}
\section*{بخش ۲۵ - قصهٔ عطاری کی سنگ ترازوی او گل  سرشوی بود و دزدیدن  مشتری گل خوار از آن گل هنگام سنجیدن شکر دزدیده و پنهان}
\label{sec:sh025}
\addcontentsline{toc}{section}{\nameref{sec:sh025}}
\begin{longtable}{l p{0.5cm} r}
پیش عطاری یکی گل‌خوار رفت
&&
تا خرد ابلوج قند خاص زفت
\\
پس بر عطار طرار دودل
&&
موضع سنگ ترازو بود گل
\\
گفت گل سنگ ترازوی منست
&&
گر ترا میل شکر بخریدنست
\\
گفت هستم در مهمی قندجو
&&
سنگ میزان هر چه خواهی باش گو
\\
گفت با خود پیش آنک گل‌خورست
&&
سنگ چه بود گل نکوتر از زرست
\\
هم‌چو آن دلاله که گفت ای پسر
&&
نو عروسی یافتم بس خوب‌فر
\\
سخت زیبا لیک هم یک چیز هست
&&
که آن ستیره دختر حلواگرست
\\
گفت بهتر این چنین خود گر بود
&&
دختر او چرب و شیرین‌تر بود
\\
گر نداری سنگ و سنگت از گلست
&&
این به و به گل مرا میوهٔ دلست
\\
اندر آن کفهٔ ترازو ز اعتداد
&&
او به جای سنگ آن گل را نهاد
\\
پس برای کفهٔ دیگر به دست
&&
هم به قدر آن شکر را می‌شکست
\\
چون نبودش تیشه‌ای او دیر ماند
&&
مشتری را منتظر آنجا نشاند
\\
رویش آن سو بود گل‌خور ناشکفت
&&
گل ازو پوشیده دزدیدن گرفت
\\
ترس ترسان که نباید ناگهان
&&
چشم او بر من فتد از امتحان
\\
دید عطار آن و خود مشغول کرد
&&
که فزون‌تر دزد هین ای روی‌زرد
\\
گر بدزدی وز گل من می‌بری
&&
رو که هم از پهلوی خود می‌خوری
\\
تو همی ترسی ز من لیک از خری
&&
من همی‌ترسم که تو کمتر خوری
\\
گرچه مشغولم چنان احمق نیم
&&
که شکر افزون کشی تو از نیم
\\
چون ببینی مر شکر را ز آزمود
&&
پس بدانی احمق و غافل کی بود
\\
مرغ زان دانه نظر خوش می‌کند
&&
دانه هم از دور راهش می‌زند
\\
کز زنای چشم حظی می‌بری
&&
نه کباب از پهلوی خود می‌خوری
\\
این نظر از دور چون تیرست و سم
&&
عشقت افزون می‌شود صبر تو کم
\\
مال دنیا دام مرغان ضعیف
&&
ملک عقبی دام مرغان شریف
\\
تا بدین ملکی که او دامست ژرف
&&
در شکار آرند مرغان شگرف
\\
من سلیمان می‌نخواهم ملکتان
&&
بلک من برهانم از هر هلکتان
\\
کین زمان هستید خود مملوک ملک
&&
مالک ملک آنک بجهید او ز هلک
\\
بازگونه ای اسیر این جهان
&&
نام خود کردی امیر این جهان
\\
ای تو بندهٔ این جهان محبوس جان
&&
چند گویی خویش را خواجهٔ جهان
\\
\end{longtable}
\end{center}
