\begin{center}
\section*{بخش ۲۳ - کرامات و نور شیخ عبدالله مغربی قدس الله سره}
\label{sec:sh023}
\addcontentsline{toc}{section}{\nameref{sec:sh023}}
\begin{longtable}{l p{0.5cm} r}
گفت عبدالله شیخ مغربی
&&
شصت سال از شب ندیدم من شبی
\\
من ندیدم ظلمتی در شصت سال
&&
نه به روز و نه به شب نه ز اعتلال
\\
صوفیان گفتند صدق قال او
&&
شب همی‌رفتیم در دنبال او
\\
در بیابانهای پر از خار و گو
&&
او چو ماه بدر ما را پیش‌رو
\\
روی پس ناکرده می‌گفتی به شب
&&
هین گو آمد میل کن در سوی چپ
\\
باز گفتی بعد یک دم سوی راست
&&
میل کن زیرا که خاری پیش پاست
\\
روز گشتی پاش را ما پای‌بوس
&&
گشته و پایش چو پاهای عروس
\\
نه ز خاک و نه ز گل بر وی اثر
&&
نه از خراش خار و آسیب حجر
\\
مغربی را مشرقی کرده خدای
&&
کرده مغرب را چو مشرق نورزای
\\
نور این شمس شموسی فارس است
&&
روز خاص و عام را او حارس است
\\
چون نباشد حارس آن نور مجید
&&
که هزاران آفتاب آرد پدید
\\
تو به نور او همی رو در امان
&&
در میان اژدها و کزدمان
\\
پیش پیشت می‌رود آن نور پاک
&&
می‌کند هر ره‌زنی را چاک‌چاک
\\
یوم لا یخزی النبی راست دان
&&
نور یسعی بین ایدیهم بخوان
\\
گرچه گردد در قیامت آن فزون
&&
از خدا اینجا بخواهید آزمون
\\
کو ببخشد هم به میغ و هم به ماغ
&&
نور جان والله اعلم بالبلاغ
\\
\end{longtable}
\end{center}
