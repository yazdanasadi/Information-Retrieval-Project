\begin{center}
\section*{بخش ۴۹ - درآمدن سلیمان علیه‌السلام هر روز در مسجد اقصی بعد از تمام شدن جهت عبادت و ارشاد عابدان و معتکفان و  رستن عقاقیر در مسجد}
\label{sec:sh049}
\addcontentsline{toc}{section}{\nameref{sec:sh049}}
\begin{longtable}{l p{0.5cm} r}
هر صباحی چون سلیمان آمدی
&&
خاضع اندر مسجد اقصی شدی
\\
نوگیاهی رسته دیدی اندرو
&&
پس بگفتی نام و نفع خود بگو
\\
تو چه دارویی چیی نامت چیست
&&
تو زیان کی و نفعت بر کیست
\\
پس بگفتی هر گیاهی فعل و نام
&&
که من آن را جانم و این را حمام
\\
من مرین را زهرم و او را شکر
&&
نام من اینست بر لوح از قدر
\\
پس طبیبان از سلیمان زان گیا
&&
عالم و دانا شدندی مقتدی
\\
تا کتبهای طبیبی ساختند
&&
جسم را از رنج می‌پرداختند
\\
این نجوم و طب وحی انبیاست
&&
عقل و حس را سوی بی‌سو ره کجاست
\\
عقل جزوی عقل استخراج نیست
&&
جز پذیرای فن و محتاج نیست
\\
قابل تعلیم و فهمست این خرد
&&
لیک صاحب وحی تعلیمش دهد
\\
جمله حرفتها یقین از وحی بود
&&
اول او لیک عقل آن را فزود
\\
هیچ حرفت را ببین کین عقل ما
&&
تاند او آموختن بی‌اوستا
\\
گرچه اندر مکر موی‌اشکاف بد
&&
هیچ پیشه رام بی‌استا نشد
\\
دانش پیشه ازین عقل ار بدی
&&
پیشهٔ بی‌اوستا حاصل شدی
\\
\end{longtable}
\end{center}
