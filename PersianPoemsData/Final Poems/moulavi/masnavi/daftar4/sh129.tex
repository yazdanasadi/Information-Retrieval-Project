\begin{center}
\section*{بخش ۱۲۹ - قصهٔ شکایت استر با شتر کی من بسیار در رو می‌افتم در راه رفتن تو کم در روی می‌آیی این چراست و جواب گفتن شتر او را}
\label{sec:sh129}
\addcontentsline{toc}{section}{\nameref{sec:sh129}}
\begin{longtable}{l p{0.5cm} r}
اشتری را دید روزی استری
&&
چونک با او جمع شد در آخری
\\
گفت من بسیار می‌افتم برو
&&
در گریوه و راه و در بازار و کو
\\
خاصه از بالای که تا زیر کوه
&&
در سر آیم هر زمانی از شکوه
\\
کم همی‌افتی تو در رو بهر چیست
&&
یا مگر خود جان پاکت دولتیست
\\
در سر آیم هر دم و زانو زنم
&&
پوز و زانو زان خطا پر خون کنم
\\
کژ شود پالان و رختم بر سرم
&&
وز مکاری هر زمان زخمی خورم
\\
هم‌چو کم عقلی که از عقل تباه
&&
بشکند توبه بهر دم در گناه
\\
مسخرهٔ ابلیس گردد در زمن
&&
از ضعیفی رای آن توبه‌شکن
\\
در سر آید هر زمان چون اسپ لنگ
&&
که بود بارش گران و راه سنگ
\\
می‌خورد از غیب بر سر زخم او
&&
از شکست توبه آن ادبارخو
\\
باز توبه می‌کند با رای سست
&&
دیو یک تف کرد و توبه‌ش را سکست
\\
ضعف اندر ضعف و کبرش آنچنان
&&
که به خواری بنگرد در واصلان
\\
ای شتر که تو مثال مؤمنی
&&
کم فتی در رو و کم بینی زنی
\\
تو چه داری که چنین بی‌آفتی
&&
بی‌عثاری و کم اندر رو فتی
\\
گفت گر چه هر سعادت از خداست
&&
در میان ما و تو بس فرقهاست
\\
سر بلندم من دو چشم من بلند
&&
بینش عالی امانست از گزند
\\
از سر که من ببینم پای کوه
&&
هر گو و هموار را من توه توه
\\
هم‌چنانک دید آن صدر اجل
&&
پیش کار خویش تا روز اجل
\\
آنچ خواهد بود بعد بیست سال
&&
داند اندر حال آن نیکو خصال
\\
حال خود تنها ندید آن متقی
&&
بلک حال مغربی و مشرقی
\\
نور در چشم و دلش سازد سکن
&&
بهر چه سازد پی حب الوطن
\\
هم‌چو یوسف کو بدید اول به خواب
&&
که سجودش کرد ماه و آفتاب
\\
از پس ده سال بلک بیشتر
&&
آنچ یوسف دید بد بر کرد سر
\\
نیست آن ینظر به نور الله گزاف
&&
نور ربانی بود گردون شکاف
\\
نیست اندر چشم تو آن نور رو
&&
هستی اندر حس حیوانی گرو
\\
تو ز ضعف چشم بینی پیش پا
&&
تو ضعیف و هم ضعیفت پیشوا
\\
پیشوا چشمست دست و پای را
&&
کو ببیند جای را ناجای را
\\
دیگر آنک چشم من روشن‌ترست
&&
دیگر آنک خلقت من اطهرست
\\
زانک هستم من ز اولاد حلال
&&
نه ز اولاد زنا و اهل ضلال
\\
تو ز اولاد زنایی بی‌گمان
&&
تیر کژ پرد چو بد باشد کمان
\\
\end{longtable}
\end{center}
