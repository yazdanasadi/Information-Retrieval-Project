\begin{center}
\section*{بخش ۱۸ - بقیهٔ قصهٔ بنای مسجد اقصی}
\label{sec:sh018}
\addcontentsline{toc}{section}{\nameref{sec:sh018}}
\begin{longtable}{l p{0.5cm} r}
چون سلیمان کرد آغاز بنا
&&
پاک چون کعبه همایون چون منی
\\
در بنااش دیده می‌شد کر و فر
&&
نی فسرده چون بناهای دگر
\\
در بنا هر سنگ کز که می‌سکست
&&
فاش سیروا بی‌همی گفت از نخست
\\
هم‌چو از آب و گل آدم‌کده
&&
نور ز آهک پاره‌ها تابان شده
\\
سنگ بی‌حمال آینده شده
&&
وان در و دیوارها زنده شده
\\
حق همی‌گوید که دیوار بهشت
&&
نیست چون دیوارها بی‌جان و زشت
\\
چون در و دیوار تن با آگهیست
&&
زنده باشد خانه چون شاهنشهیست
\\
هم درخت و میوه هم آب زلال
&&
با بهشتی در حدیث و در مقال
\\
زانک جنت را نه ز آلت بسته‌اند
&&
بلک از اعمال و نیت بسته‌اند
\\
این بنا ز آب و گل مرده بدست
&&
وان بنا از طاعت زنده شدست
\\
این به اصل خویش ماند پرخلل
&&
وان به اصل خود که علمست و عمل
\\
هم سریر و قصر و هم تاج و ثیاب
&&
با بهشتی در سؤال و در جواب
\\
فرش بی‌فراش پیچیده شود
&&
خانه بی‌مکناس روبیده شود
\\
خانهٔ دل بین ز غم ژولیده شد
&&
بی‌کناس از توبه‌ای روبیده شد
\\
تخت او سیار بی‌حمال شد
&&
حلقه و در مطرب و قوال شد
\\
هست در دل زندگی دارالخلود
&&
در زبانم چون نمی‌آید چه سود
\\
چون سلیمان در شدی هر بامداد
&&
مسجد اندر بهر ارشاد عباد
\\
پند دادی گه بگفت و لحن و ساز
&&
گه به فعل اعنی رکوعی یا نماز
\\
پند فعلی خلق را جذاب‌تر
&&
که رسد در جان هر باگوش و کر
\\
اندر آن وهم امیری کم بود
&&
در حشم تاثیر آن محکم بود
\\
\end{longtable}
\end{center}
