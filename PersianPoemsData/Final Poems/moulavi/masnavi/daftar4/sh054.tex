\begin{center}
\section*{بخش ۵۴ - تفسیر یا ایها المزمل}
\label{sec:sh054}
\addcontentsline{toc}{section}{\nameref{sec:sh054}}
\begin{longtable}{l p{0.5cm} r}
خواند مزمل نبی را زین سبب
&&
که برون آ از گلیم ای بوالهرب
\\
سر مکش اندر گلیم و رو مپوش
&&
که جهان جسمیست سرگردان تو هوش
\\
هین مشو پنهان ز ننگ مدعی
&&
که تو داری شمع وحی شعشعی
\\
هین قم اللیل که شمعی ای همام
&&
شمع اندر شب بود اندر قیام
\\
بی‌فروغت روز روشن هم شبست
&&
بی‌پناهت شیر اسیر ارنبست
\\
باش کشتیبان درین بحر صفا
&&
که تو نوح ثانیی ای مصطفی
\\
ره شناسی می‌بباید با لباب
&&
هر رهی را خاصه اندر راه آب
\\
خیز بنگر کاروان ره‌زده
&&
هر طرف غولیست کشتیبان شده
\\
خضر وقتی غوث هر کشتی توی
&&
هم‌چو روح‌الله مکن تنها روی
\\
پیش این جمعی چو شمع آسمان
&&
انقطاع و خلوت آری را بمان
\\
وقت خلوت نیست اندر جمع آی
&&
ای هدی چون کوه قاف و تو همای
\\
بدر بر صدر فلک شد شب روان
&&
سیر را نگذارد از بانگ سگان
\\
طاعنان هم‌چون سگان بر بدر تو
&&
بانگ می‌دارند سوی صدر تو
\\
این سگان کرند از امر انصتوا
&&
از سفه و عوع کنان بر بدر تو
\\
هین بمگذار ای شفا رنجور را
&&
تو ز خشم کر عصای کور را
\\
نه تو گفتی قاید اعمی به راه
&&
صد ثواب و اجر یابد از اله
\\
هر که او چل گام کوری را کشد
&&
گشت آمرزیده و یابد رشد
\\
پس بکش تو زین جهان بی‌قرار
&&
جوق کوران را قطار اندر قطار
\\
کار هادی این بود تو هادیی
&&
ماتم آخر زمان را شادیی
\\
هین روان کن ای امام المتقین
&&
این خیال‌اندیشگان را تا یقین
\\
هر که در مکر تو دارد دل گرو
&&
گردنش را من زنم تو شاد رو
\\
بر سر کوریش کوریها نهم
&&
او شکر پندارد و زهرش دهم
\\
عقلها از نور من افروختند
&&
مکرها از مکر من آموختند
\\
چیست خود آلاجق آن ترکمان
&&
پیش پای نره پیلان جهان
\\
آن چراغ او به پیش صرصرم
&&
خود چه باشد ای مهین پیغامبرم
\\
خیز در دم تو بصور سهمناک
&&
تا هزاران مرده بر روید ز خاک
\\
چون تو اسرافیل وقتی راست‌خیز
&&
رستخیزی ساز پیش از رستخیز
\\
هر که گوید کو قیامت ای صنم
&&
خویش بنما که قیامت نک منم
\\
در نگر ای سایل محنت‌زده
&&
زین قیامت صد جهان افزون شده
\\
ور نباشد اهل این ذکر و قنوت
&&
پس جواب الاحمق ای سلطان سکوت
\\
ز آسمان حق سکوت آید جواب
&&
چون بود جانا دعا نامستجاب
\\
ای دریغا وقت خرمنگاه شد
&&
لیک روز از بخت ما بیگاه شد
\\
وقت تنگست و فراخی این کلام
&&
تنگ می‌آید برو عمر دوام
\\
نیزه‌بازی اندرین کوه‌های تنگ
&&
نیزه‌بازان را همی آرد به تنگ
\\
وقت تنگ و خاطر و فهم عوام
&&
تنگ‌تر صد ره ز وقت است ای غلام
\\
چون جواب احمق آمد خامشی
&&
این درازی در سخن چون می‌کشی
\\
از کمال رحمت و موج کرم
&&
می‌دهد هر شوره را باران و نم
\\
\end{longtable}
\end{center}
