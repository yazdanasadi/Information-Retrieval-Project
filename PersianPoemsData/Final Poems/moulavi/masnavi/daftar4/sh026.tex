\begin{center}
\section*{بخش ۲۶ - دلداری کردن و نواختن سلیمان علیه‌السلام مر آن رسولان را  و دفع وحشت و آزار از دل ایشان و عذر قبول ناکردن هدیه شرح  کردن با ایشان}
\label{sec:sh026}
\addcontentsline{toc}{section}{\nameref{sec:sh026}}
\begin{longtable}{l p{0.5cm} r}
ای رسولان می‌فرستمتان رسول
&&
رد من بهتر شما را از قبول
\\
پیش بلقیس آنچ دیدیت از عجب
&&
باز گویید از بیابان ذهب
\\
تا بداند که به زر طامع نه‌ایم
&&
ما زر از زرآفرین آورده‌ایم
\\
آنک گر خواهد همه خاک زمین
&&
سر به سر زر گردد و در ثمین
\\
حق برای آن کند ای زرگزین
&&
روز محشر این زمین را نقره گین
\\
فارغیم از زر که ما بس پر فنیم
&&
خاکیان را سر به سر زرین کنیم
\\
از شما کی کدیهٔ زر می‌کنیم
&&
ما شما را کیمیاگر می‌کنیم
\\
ترک آن گیرید گر ملک سباست
&&
که برون آب و گل بس ملکهاست
\\
تخته‌بندست آن که تختش خوانده‌ای
&&
صدر پنداری و بر در مانده‌ای
\\
پادشاهی نیستت بر ریش خود
&&
پادشاهی چون کنی بر نیک و بد
\\
بی‌مراد تو شود ریشت سپید
&&
شرم دار از ریش خود ای کژ امید
\\
مالک الملک است هر کش سر نهد
&&
بی‌جهان خاک صد ملکش دهد
\\
لیک ذوق سجده‌ای پیش خدا
&&
خوشتر آید از دو صد دولت ترا
\\
پس بنالی که نخواهم ملکها
&&
ملک آن سجده مسلم کن مرا
\\
پادشاهان جهان از بدرگی
&&
بو نبردند از شراب بندگی
\\
ورنه ادهم‌وار سرگردان و دنگ
&&
ملک را برهم زدندی بی‌درنگ
\\
لیک حق بهر ثبات این جهان
&&
مهرشان بنهاد بر چشم و دهان
\\
تا شود شیرین بریشان تخت و تاج
&&
که ستانیم از جهانداران خراج
\\
از خراج ار جمع آری زر چو ریگ
&&
آخر آن از تو بماند مردریگ
\\
همره جانت نگردد ملک و زر
&&
زر بده سرمه ستان بهر نظر
\\
تا ببینی کین جهان چاهیست تنگ
&&
یوسفانه آن رسن آری به چنگ
\\
تا بگوید چون ز چاه آیی به بام
&&
جان که یا بشرای هذا لی غلام
\\
هست در چاه انعکاسات نظر
&&
کمترین آنک نماید سنگ زر
\\
وقت بازی کودکان را ز اختلال
&&
می‌نماید آن خزفها زر و مال
\\
عارفانش کیمیاگر گشته‌اند
&&
تا که شد کانها بر ایشان نژند
\\
\end{longtable}
\end{center}
