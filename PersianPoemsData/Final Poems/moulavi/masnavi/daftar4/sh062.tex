\begin{center}
\section*{بخش ۶۲ - بیان آنک عارف را غذاییست از نور حق کی ابیت عند ربی یطعمنی و یسقینی و قوله الجوع طعام الله یحیی به ابدان الصدیقین ای فی الجوع یصل طعام‌الله}
\label{sec:sh062}
\addcontentsline{toc}{section}{\nameref{sec:sh062}}
\begin{longtable}{l p{0.5cm} r}
زانک هر کره پی مادر رود
&&
تا بدان جنسیتش پیدا شود
\\
آدمی را شیر از سینه رسد
&&
شیر خر از نیم زیرینه رسد
\\
عدل قسامست و قسمت کردنیست
&&
این عجب که جبر نی و ظلم نیست
\\
جبر بودی کی پشیمانی بدی
&&
ظلم بودی کی نگهبانی بدی
\\
روز آخر شد سبق فردا بود
&&
راز ما را روز کی گنجا بود
\\
ای بکرده اعتماد واثقی
&&
بر دم و بر چاپلوس فاسقی
\\
قبه‌ای بر ساختستی از حباب
&&
آخر آن خیمه‌ست بس واهی‌طناب
\\
زرق چون برقست و اندر نور آن
&&
راه نتوانند دیدن ره‌روان
\\
این جهان و اهل او بی‌حاصل‌اند
&&
هر دو اندر بی‌وفایی یکدل‌اند
\\
زادهٔ دنیا چو دنیا بی‌وفاست
&&
گرچه رو آرد به تو آن رو قفاست
\\
اهل آن عالم چو آن عالم ز بر
&&
تا ابد در عهد و پیمان مستمر
\\
خود دو پیغمبر به هم کی ضد شدند
&&
معجزات از همدگر کی بستدند
\\
کی شود پژمرده میوهٔ آن جهان
&&
شادی عقلی نگردد اندهان
\\
نفس بی‌عهدست زان رو کشتنیست
&&
او دنی و قبله‌گاه او دنیست
\\
نفسها را لایقست این انجمن
&&
مرده را درخور بود گور و کفن
\\
نفس اگر چه زیرکست و خرده‌دان
&&
قبله‌اش دنیاست او را مرده دان
\\
آب وحی حق بدین مرده رسید
&&
شد ز خاک مرده‌ای زنده پدید
\\
تا نیاید وحش تو غره مباش
&&
تو بدان گلگونهٔ طال بقاش
\\
بانگ و صیتی جو که آن خامل نشد
&&
تاب خورشیدی که آن آفل نشد
\\
آن هنرهای دقیق و قال و قیل
&&
قوم فرعون‌اند اجل چون آب نیل
\\
رونق و طاق و طرنب و سحرشان
&&
گرچه خلقان را کشد گردن کشان
\\
سحرهای ساحران دان جمله را
&&
مرگ چوبی دان که آن گشت اژدها
\\
جادویها را همه یک لقمه کرد
&&
یک جهان پر شب بد آن را صبح خورد
\\
نور از آن خوردن نشد افزون و بیش
&&
بل همان سانست کو بودست پیش
\\
در اثر افزون شد و در ذات نی
&&
ذات را افزونی و آفات نی
\\
حق ز ایجاد جهان افزون نشد
&&
آنچ اول آن نبود اکنون نشد
\\
لیک افزون گشت اثر ز ایجاد خلق
&&
در میان این دو افزونیست فرق
\\
هست افزونی اثر اظهار او
&&
تا پدید آید صفات و کار او
\\
هست افزونی هر ذاتی دلیل
&&
کو بود حادث به علتها علیل
\\
\end{longtable}
\end{center}
