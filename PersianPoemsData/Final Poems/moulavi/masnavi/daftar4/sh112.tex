\begin{center}
\section*{بخش ۱۱۲ - وحی کردن حق به موسی علیه‌السلام کی ای موسی من کی  خالقم تعالی ترا دوست می‌دارم}
\label{sec:sh112}
\addcontentsline{toc}{section}{\nameref{sec:sh112}}
\begin{longtable}{l p{0.5cm} r}
گفت موسی را به وحی دل خدا
&&
کای گزیده دوست می‌دارم ترا
\\
گفت چه خصلت بود ای ذوالکرم
&&
موجب آن تا من آن افزون کنم
\\
گفت چون طفلی به پیش والده
&&
وقت قهرش دست هم در وی زده
\\
خود نداند که جز او دیار هست
&&
هم ازو مخمور هم از اوست مست
\\
مادرش گر سیلیی بر وی زند
&&
هم به مادر آید و بر وی تند
\\
از کسی یاری نخواهد غیر او
&&
اوست جمله شر او و خیر او
\\
خاطر تو هم ز ما در خیر و شر
&&
التفاتش نیست جاهای دگر
\\
غیر من پیشت چون سنگست و کلوخ
&&
گر صبی و گر جوان و گر شیوخ
\\
هم‌چنانک ایاک نعبد در حنین
&&
در بلا از غیر تو لانستعین
\\
هست این ایاک نعبد حصر را
&&
در لغت و آن از پی نفی ریا
\\
هست ایاک نستعین هم بهر حصر
&&
حصر کرده استعانت را و قصر
\\
که عبادت مر ترا آریم و بس
&&
طمع یاری هم ز تو داریم و بس
\\
\end{longtable}
\end{center}
