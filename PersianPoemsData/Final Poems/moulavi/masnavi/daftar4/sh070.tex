\begin{center}
\section*{بخش ۷۰ - نقصان اجرای جان و دل صوفی از طعام الله}
\label{sec:sh070}
\addcontentsline{toc}{section}{\nameref{sec:sh070}}
\begin{longtable}{l p{0.5cm} r}
صوفیی از فقر چون در غم شود
&&
عین فقرش دایه و مطعم شود
\\
زانک جنت از مکاره رسته است
&&
رحم قسم عاجزی اشکسته است
\\
آنک سرها بشکند او از علو
&&
رحم حق و خلق ناید سوی او
\\
این سخن آخر ندارد وان جوان
&&
از کمی اجرای نان شد ناتوان
\\
شاد آن صوفی که رزقش کم شود
&&
آن شبه‌ش در گردد و اویم شود
\\
زان جرای خاص هر که آگاه شد
&&
او سزای قرب و اجری‌گاه شد
\\
زان جرای روح چون نقصان شود
&&
جانش از نقصان آن لرزان شود
\\
پس بداند که خطایی رفته است
&&
که سمن‌زار رضا آشفته است
\\
هم‌چنانک آن شخص از نقصان کشت
&&
رقعه سوی صاحب خرمن نبشت
\\
رقعه‌اش بردند پیش میر داد
&&
خواند او رقعه جوابی وا نداد
\\
گفت او را نیست الا درد لوت
&&
پس جواب احمق اولیتر سکوت
\\
نیستش درد فراق و وصل هیچ
&&
بند فرعست او نجوید اصل هیچ
\\
احمقست و مردهٔ ما و منی
&&
کز غم فرعش فراغ اصل نی
\\
آسمانها و زمین یک سیب دان
&&
کز درخت قدرت حق شد عیان
\\
تو چه کرمی در میان سیب در
&&
وز درخت و باغبانی بی‌خبر
\\
آن یکی کرمی دگر در سیب هم
&&
لیک جانش از برون صاحب‌علم
\\
جنبش او وا شکافد سیب را
&&
بر نتابد سیب آن آسیب را
\\
بر دریده جنبش او پرده‌ها
&&
صورتش کرمست و معنی اژدها
\\
آتش که اول ز آهن می‌جهد
&&
او قدم بس سست بیرون می‌نهد
\\
دایه‌اش پنبه‌ست اول لیک اخیر
&&
می‌رساند شعله‌ها او تا اثیر
\\
مرد اول بستهٔ خواب و خورست
&&
آخر الامر از ملایک برترست
\\
در پناه پنبه و کبریتها
&&
شعله و نورش برآیدت بر سها
\\
عالم تاریک روشن می‌کند
&&
کندهٔ آهن به سوزن می‌کند
\\
گرچه آتش نیز هم جسمانی است
&&
نه ز روحست و نه از روحانی است
\\
جسم را نبود از آن عز بهره‌ای
&&
جسم پیش بحر جان چون قطره‌ای
\\
جسم از جان روزافزون می‌شود
&&
چون رود جان جسم بین چون می‌شود
\\
حد جسمت یک دو گز خود بیش نیست
&&
جان تو تا آسمان جولان‌کنیست
\\
تا به بغداد و سمرقند ای همام
&&
روح را اندر تصور نیم گام
\\
دو درم سنگست پیه چشمتان
&&
نور روحش تا عنان آسمان
\\
نور بی این چشم می‌بیند به خواب
&&
چشم بی‌این نور چه بود جز خراب
\\
جان ز ریش و سبلت تن فارغست
&&
لیک تن بی‌جان بود مردار و پست
\\
بارنامهٔ روح حیوانیست این
&&
پیشتر رو روح انسانی ببین
\\
بگذر از انسان هم و از قال و قیل
&&
تا لب دریای جان جبرئیل
\\
بعد از آنت جان احمد لب گزد
&&
جبرئیل از بیم تو واپس خزد
\\
گوید ار آیم به قدر یک کمان
&&
من به سوی تو بسوزم در زمان
\\
\end{longtable}
\end{center}
