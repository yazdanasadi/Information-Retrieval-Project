\begin{center}
\section*{بخش ۱۱۵ - مطالبه کردن موسی علیه‌السلام حضرت را کی خلقت خلقا اهلکتهم و جواب آمدن}
\label{sec:sh115}
\addcontentsline{toc}{section}{\nameref{sec:sh115}}
\begin{longtable}{l p{0.5cm} r}
گفت موسی ای خداوند حساب
&&
نقش کردی باز چون کردی خراب
\\
نر و ماده نقش کردی جان‌فزا
&&
وانگهان ویران کنی این را چرا
\\
گفت حق دانم که این پرسش ترا
&&
نیست از انکار و غفلت وز هوا
\\
ورنه تادیب و عتابت کردمی
&&
بهر این پرسش ترا آزردمی
\\
لیک می‌خواهی که در افعال ما
&&
باز جویی حکمت و سر بقا
\\
تا از آن واقف کنی مر عام را
&&
پخته گردانی بدین هر خام را
\\
قاصدا سایل شدی در کاشفی
&&
بر عوام ار چه که تو زان واقفی
\\
زآنک نیم علم آمد این سؤال
&&
هر برونی را نباشد آن مجال
\\
هم سؤال از علم خیزد هم جواب
&&
هم‌چنانک خار و گل از خاک و آب
\\
هم ضلال از علم خیزد هم هدی
&&
هم‌چنانک تلخ و شیرین از ندا
\\
ز آشنایی خیزد این بغض و ولا
&&
وز غذای خویش بود سقم و قوی
\\
مستفید اعجمی شد آن کلیم
&&
تا عجمیان را کند زین سر علیم
\\
ما هم از وی اعجمی سازیم خویش
&&
پاسخش آریم چون بیگانه پیش
\\
خرفروشان خصم یکدیگر شدند
&&
تا کلید قفل آن عقد آمدند
\\
پس بفرمودش خدا ای ذولباب
&&
چون بپرسیدی بیا بشنو جواب
\\
موسیا تخمی بکار اندر زمین
&&
تا تو خود هم وا دهی انصاف این
\\
چونک موسی کشت و شد کشتش تمام
&&
خوشه‌هااش یافت خوبی و نظام
\\
داس بگرفت و مر آن را می‌برید
&&
پس ندا از غیب در گوشش رسید
\\
که چرا کشتی کنی و پروری
&&
چون کمالی یافت آن را می‌بری
\\
گفت یا رب زان کنم ویران و پست
&&
که درینجا دانه هست و کاه هست
\\
دانه لایق نیست درانبار کاه
&&
کاه در انبار گندم هم تباه
\\
نیست حکمت این دو را آمیختن
&&
فرق واجب می‌کند در بیختن
\\
گفت این دانش تو از کی یافتی
&&
که به دانش بیدری بر ساختی
\\
گفت تمییزم تو دادی ای خدا
&&
گفت پس تمییز چون نبود مرا
\\
در خلایق روحهای پاک هست
&&
روحهای تیرهٔ گلناک هست
\\
این صدفها نیست در یک مرتبه
&&
در یکی درست و در دیگر شبه
\\
واجبست اظهار این نیک و تباه
&&
هم‌چنانک اظهار گندمها ز کاه
\\
بهر اظهارست این خلق جهان
&&
تا نماند گنج حکمتها نهان
\\
کنت کنزا کنت مخفیا شنو
&&
جوهر خود گم مکن اظهار شو
\\
\end{longtable}
\end{center}
