\begin{center}
\section*{بخش ۱۰ - مثال دنیا چون گولخن و تقوی چون حمام}
\label{sec:sh010}
\addcontentsline{toc}{section}{\nameref{sec:sh010}}
\begin{longtable}{l p{0.5cm} r}
شهوت دنیا مثال گلخنست
&&
که ازو حمام تقوی روشنست
\\
لیک قسم متقی زین تون صفاست
&&
زانک در گرمابه است و در نقاست
\\
اغنیا مانندهٔ سرگین‌کشان
&&
بهر آتش کردن گرمابه‌بان
\\
اندریشان حرص بنهاده خدا
&&
تا بود گرمابه گرم و با نوا
\\
ترک این تون گوی و در گرمابه ران
&&
ترک تون را عین آن گرمابه دان
\\
هر که در تونست او چون خادمست
&&
مر ورا که صابرست و حازمست
\\
هر که در حمام شد سیمای او
&&
هست پیدا بر رخ زیبای او
\\
تونیان را نیز سیما آشکار
&&
از لباس و از دخان و از غبار
\\
ور نبینی روش بویش را بگیر
&&
بو عصا آمد برای هر ضریر
\\
ور نداری بو در آرش در سخن
&&
از حدیث نو بدان راز کهن
\\
پس بگوید تونیی صاحب ذهب
&&
بیست سله چرک بردم تا به شب
\\
حرص تو چون آتشست اندر جهان
&&
باز کرده هر زبانه صد دهان
\\
پیش عقل این زر چو سرگین ناخوشست
&&
گرچه چون سرگین فروغ آتشست
\\
آفتابی که دم از آتش زند
&&
چرک تر را لایق آتش کند
\\
آفتاب آن سنگ را هم کرد زر
&&
تا بتون حرص افتد صد شرر
\\
آنک گوید مال گرد آورده‌ام
&&
چیست یعنی چرک چندین برده‌ام
\\
این سخن گرچه که رسوایی‌فزاست
&&
در میان تونیان زین فخرهاست
\\
که تو شش سله کشیدی تا به شب
&&
من کشیدم بیست سله بی کرب
\\
آنک در تون زاد و پاکی را ندید
&&
بوی مشک آرد برو رنجی پدید
\\
\end{longtable}
\end{center}
