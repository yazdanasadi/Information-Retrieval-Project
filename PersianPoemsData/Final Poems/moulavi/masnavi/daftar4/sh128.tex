\begin{center}
\section*{بخش ۱۲۸ - بیان آنک یا ایها الذین آمنوا لا تقدموا بین یدی الله و رسوله  چون نبی نیستی ز امت باش  چونک سلطان نه‌ای رعیت باش پس رو خاموش باش از خود زحمتی و رایی متراش}
\label{sec:sh128}
\addcontentsline{toc}{section}{\nameref{sec:sh128}}
\begin{longtable}{l p{0.5cm} r}
پس برو خاموش باش از انقیاد
&&
زیر ظل امر شیخ و اوستاد
\\
ورنه گر چه مستعد و قابلی
&&
مسخ گردی تو ز لاف کاملی
\\
هم ز استعداد وا مانی اگر
&&
سر کشی ز استاد راز و با خبر
\\
صبر کن در موزه دوزی تو هنوز
&&
ور بوی بی‌صبر گردی پاره‌دوز
\\
کهنه‌دوزان گر بدیشان صبر و حلم
&&
جمله نودوزان شدندی هم به علم
\\
بس بکوشی و بخر از کلال
&&
هم تو گویی خویش کالعقل عقال
\\
هم‌چو آن مرد مفلسف روز مرگ
&&
عقل را می‌دید بس بی‌بال و برگ
\\
بی‌غرض می‌کرد آن دم اعتراف
&&
کز ذکاوت راندیم اسپ از گزاف
\\
از غروری سر کشیدیم از رجال
&&
آشنا کردیم در بحر خیال
\\
آشنا هیچست اندر بحر روح
&&
نیست اینجا چاره جز کشتی نوح
\\
این چنین فرمود این شاه رسل
&&
که منم کشتی درین دریای کل
\\
یا کسی کو در بصیرتهای من
&&
شد خلیفهٔ راستی بر جای من
\\
کشتی نوحیم در دریا که تا
&&
رو نگردانی ز کشتی ای فتی
\\
هم‌چو کنعان سوی هر کوهی مرو
&&
از نبی لا عاصم الیوم شنو
\\
می‌نماید پست این کشتی ز بند
&&
می‌نماید کوه فکرت بس بلند
\\
پست منگر هان و هان این پست را
&&
بنگر آن فضل حق پیوست را
\\
در علو کوه فکرت کم نگر
&&
که یکی موجش کند زیر و زبر
\\
گر تو کنعانی نداری باورم
&&
گر دو صد چندین نصیحت پرورم
\\
گوش کنعان کی پذیرد این کلام
&&
که برو مهر خدایست و ختام
\\
کی گذارد موعظه بر مهر حق
&&
کی بگرداند حدث حکم سبق
\\
لیک می‌گویم حدیث خوش‌پیی
&&
بر امید آنک تو کنعان نه‌ای
\\
آخر این اقرار خواهی کرد هین
&&
هم ز اول روز آخر را ببین
\\
می‌توانی دید آخر را مکن
&&
چشم آخربینت را کور کهن
\\
هر که آخربین بود مسعودوار
&&
نبودش هر دم ز ره رفتن عثار
\\
گر نخواهی هر دمی این خفت‌خیز
&&
کن ز خاک پایی مردی چشم تیز
\\
کحل دیده ساز خاک پاش را
&&
تا بیندازی سر اوباش را
\\
که ازین شاگردی و زین افتقار
&&
سوزنی باشی شوی تو ذوالفقار
\\
سرمه کن تو خاک هر بگزیده را
&&
هم بسوزد هم بسازد دیده را
\\
چشم اشتر زان بود بس نوربار
&&
کو خورد از بهر نور چشم خار
\\
\end{longtable}
\end{center}
