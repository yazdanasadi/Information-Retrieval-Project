\begin{center}
\section*{بخش ۶۸ - مژده دادن ابویزید از زادن ابوالحسن خرقانی قدس الله روحهما پیش از سالها  و نشان صورت او سیرت او یک به یک و  نوشتن تاریخ‌نویسان آن در جهت رصد}
\label{sec:sh068}
\addcontentsline{toc}{section}{\nameref{sec:sh068}}
\begin{longtable}{l p{0.5cm} r}
آن شنیدی داستان بایزید
&&
که ز حال بوالحسن پیشین چه دید
\\
روزی آن سلطان تقوی می‌گذشت
&&
با مریدان جانب صحرا و دشت
\\
بوی خوش آمد مر او را ناگهان
&&
در سواد ری ز سوی خارقان
\\
هم بدانجا نالهٔ مشتاق کرد
&&
بوی را از باد استنشاق کرد
\\
بوی خوش را عاشقانه می‌کشید
&&
جان او از باد باده می‌چشید
\\
کوزه‌ای کو از یخابه پر بود
&&
چون عرق بر ظاهرش پیدا شود
\\
آن ز سردی هوا آبی شدست
&&
از درون کوزه نم بیرون نجست
\\
باد بوی‌آور مر او را آب گشت
&&
آب هم او را شراب ناب گشت
\\
چون درو آثار مستی شد پدید
&&
یک مرید او را از آن دم بر رسید
\\
پس بپرسیدش که این احوال خوش
&&
که برونست از حجاب پنج و شش
\\
گاه سرخ و گاه زرد و گه سپید
&&
می‌شود رویت چه حالست و نوید
\\
می‌کشی بوی و به ظاهر نیست گل
&&
بی‌شک از غیبست و از گلزار کل
\\
ای تو کام جان هر خودکامه‌ای
&&
هر دم از غیبت پیام و نامه‌ای
\\
هر دمی یعقوب‌وار از یوسفی
&&
می‌رسد اندر مشام تو شفا
\\
قطره‌ای بر ریز بر ما زان سبو
&&
شمه‌ای زان گلستان با ما بگو
\\
خو نداریم ای جمال مهتری
&&
که لب ما خشک و تو تنها خوری
\\
ای فلک‌پیمای چست چست‌خیز
&&
زانچ خوردی جرعه‌ای بر ما بریز
\\
میر مجلس نیست در دوران دگر
&&
جز تو ای شه در حریفان در نگر
\\
کی توان نوشید این می زیردست
&&
می یقین مر مرد را رسواگرست
\\
بوی را پوشیده و مکنون کند
&&
چشم مست خویشتن را چون کند
\\
خود نه آن بویست این که اندر جهان
&&
صد هزاران پرده‌اش دارد نهان
\\
پر شد از تیزی او صحرا و دشت
&&
دشت چه کز نه فلک هم در گذشت
\\
این سر خم را به کهگل در مگیر
&&
کین برهنه نیست خود پوشش‌پذیر
\\
لطف کن ای رازدان رازگو
&&
آنچ بازت صید کردش بازگو
\\
گفت بوی بوالعجب آمد به من
&&
هم‌چنانک مر نبی را از یمن
\\
که محمد گفت بر دست صبا
&&
از یمن می‌آیدم بوی خدا
\\
بوی رامین می‌رسد از جان ویس
&&
بوی یزدان می‌رسد هم از اویس
\\
از اویس و از قرن بوی عجب
&&
مر نبی را مست کرد و پر طرب
\\
چون اویس از خویش فانی گشته بود
&&
آن زمینی آسمانی گشته بود
\\
آن هلیلهٔ پروریده در شکر
&&
چاشنی تلخیش نبود دگر
\\
آن هلیلهٔ رسته از ما و منی
&&
نقش دارد از هلیله طعم نی
\\
این سخن پایان ندارد باز گرد
&&
تا چه گفت از وحی غیب آن شیرمرد
\\
\end{longtable}
\end{center}
