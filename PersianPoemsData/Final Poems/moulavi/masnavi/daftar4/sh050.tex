\begin{center}
\section*{بخش ۵۰ - آموختن پیشه گورکنی قابیل از زاغ پیش از آنک در عالم علم گورکنی و گور بود}
\label{sec:sh050}
\addcontentsline{toc}{section}{\nameref{sec:sh050}}
\begin{longtable}{l p{0.5cm} r}
کندن گوری که کمتر پیشه بود
&&
کی ز فکر و حیله و اندیشه بود
\\
گر بدی این فهم مر قابیل را
&&
کی نهادی بر سر او هابیل را
\\
که کجا غایب کنم این کشته را
&&
این به خون و خاک در آغشته را
\\
دید زاغی زاغ مرده در دهان
&&
بر گرفته تیز می‌آمد چنان
\\
از هوا زیر آمد و شد او به فن
&&
از پی تعلیم او را گورکن
\\
پس به چنگال از زمین انگیخت گرد
&&
زود زاغ مرده را در گور کرد
\\
دفن کردش پس بپوشیدش به خاک
&&
زاغ از الهام حق بد علم‌ناک
\\
گفت قابیل آه شه بر عقل من
&&
که بود زاغی ز من افزون به فن
\\
عقل کل را گفت مازاغ البصر
&&
عقل جزوی می‌کند هر سو نظر
\\
عقل مازاغ است نور خاصگان
&&
عقل زاغ استاد گور مردگان
\\
جان که او دنبالهٔ زاغان پرد
&&
زاغ او را سوی گورستان برد
\\
هین مدو اندر پی نفس چو زاغ
&&
کو به گورستان برد نه سوی باغ
\\
گر روی رو در پی عنقای دل
&&
سوی قاف و مسجد اقصای دل
\\
نوگیاهی هر دم ز سودای تو
&&
می‌دمد در مسجد اقصای تو
\\
تو سلیمان‌وار داد او بده
&&
پی بر از وی پای رد بر وی منه
\\
زانک حال این زمین با ثبات
&&
باز گوید با تو انواع نبات
\\
در زمین گر نیشکر ور خود نیست
&&
ترجمان هر زمین نبت ویست
\\
پس زمین دل که نبتش فکر بود
&&
فکرها اسرار دل را وا نمود
\\
گر سخن‌کش یابم اندر انجمن
&&
صد هزاران گل برویم چون چمن
\\
ور سخن‌کش یابم آن دم زن به مزد
&&
می‌گریزد نکته‌ها از دل چو دزد
\\
جنبش هر کس به سوی جاذبست
&&
جذب صدق نه چو جذب کاذبست
\\
می‌روی گه گمره و گه در رشد
&&
رشته پیدا نه و آنکت می‌کشد
\\
اشتر کوری مهار تو رهین
&&
تو کشش می‌بین مهارت را مبین
\\
گر شدی محسوس جذاب و مهار
&&
پس نماندی این جهان دارالغرار
\\
گبر دیدی کو پی سگ می‌رود
&&
سخرهٔ دیو ستنبه می‌شود
\\
در پی او کی شدی مانند حیز
&&
پی خود را واکشیدی گبر نیز
\\
گاو گر واقف ز قصابان بدی
&&
کی پی ایشان بدان دکان شدی
\\
یا بخوردی از کف ایشان سبوس
&&
یا بدادی شیرشان از چاپلوس
\\
ور بخوردی کی علف هضمش شدی
&&
گر ز مقصود علف واقف بدی
\\
پس ستون این جهان خود غفلتست
&&
چیست دولت کین دوادو با لتست
\\
اولش دو دو به آخر لت بخور
&&
جز درین ویرانه نبود مرگ خر
\\
تو به جد کاری که بگرفتی به دست
&&
عیبش این دم بر تو پوشیده شدست
\\
زان همی تانی بدادن تن به کار
&&
که بپوشید از تو عیبش کردگار
\\
همچنین هر فکر که گرمی در آن
&&
عیب آن فکرت شدست از تو نهان
\\
بر تو گر پیدا شدی زو عیب و شین
&&
زو رمیدی جانت بعد المشرقین
\\
حال که آخر زو پشیمان می‌شوی
&&
گر بود این حال اول کی دوی
\\
پس بپوشید اول آن بر جان ما
&&
تا کنیم آن کار بر وفق قضا
\\
چون قضا آورد حکم خود پدید
&&
چشم وا شد تا پشیمانی رسید
\\
این پشیمانی قضای دیگرست
&&
این پشیمانی بهل حق را پرست
\\
ور کنی عادت پشیمان خور شوی
&&
زین پشیمانی پشیمان‌تر شوی
\\
نیم عمرت در پریشانی رود
&&
نیم دیگر در پشیمانی رود
\\
ترک این فکر و پریشانی بگو
&&
حال و یار و کار نیکوتر بجو
\\
ور نداری کار نیکوتر به دست
&&
پس پشیمانیت بر فوت چه است
\\
گر همی دانی ره نیکو پرست
&&
ور ندانی چون بدانی کین به دست
\\
بد ندانی تا ندانی نیک را
&&
ضد را از ضد توان دید ای فتی
\\
چون ز ترک فکر این عاجز شدی
&&
از گناه آنگاه هم عاجز بدی
\\
چون بدی عاجز پشیمانی ز چیست
&&
عاجزی را باز جو کز جذب کیست
\\
عاجزی بی‌قادری اندر جهان
&&
کس ندیدست و نباشد این بدان
\\
همچنین هر آرزو که می‌بری
&&
تو ز عیب آن حجابی اندری
\\
ور نمودی علت آن آرزو
&&
خود رمیدی جان تو زان جست و جو
\\
گر نمودی عیب آن کار او ترا
&&
کس نبردی کش کشان آن سو ترا
\\
وان دگر کار کز آن هستی نفور
&&
زان بود که عیبش آمد در ظهور
\\
ای خدای رازدان خوش‌سخن
&&
عیب کار بد ز ما پنهان مکن
\\
عیب کار نیک را منما به ما
&&
تا نگردیم از روش سرد و هبا
\\
هم بر آن عادت سلیمان سنی
&&
رفت در مسجد میان روشنی
\\
قاعدهٔ هر روز را می‌جست شاه
&&
که ببیند مسجد اندر نو گیاه
\\
دل ببیند سر بدان چشم صفی
&&
آن حشایش که شد از عامه خفی
\\
\end{longtable}
\end{center}
