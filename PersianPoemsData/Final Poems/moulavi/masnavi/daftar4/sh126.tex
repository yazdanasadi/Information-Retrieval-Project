\begin{center}
\section*{بخش ۱۲۶ - تفسیر این حدیث کی ائنی لاستغفر الله فی کل یوم سبعین مرة}
\label{sec:sh126}
\addcontentsline{toc}{section}{\nameref{sec:sh126}}
\begin{longtable}{l p{0.5cm} r}
هم‌چو پیغامبر ز گفتن وز نثار
&&
توبه آرم روز من هفتاد بار
\\
لیک آن مستی شود توبه‌شکن
&&
منسی است این مستی تن جامه کن
\\
حکمت اظهار تاریخ دراز
&&
مستیی انداخت در دانای راز
\\
راز پنهان با چنین طبل و علم
&&
آب جوشان گشته از جف القلم
\\
رحمت بی‌حد روانه هر زمان
&&
خفته‌اید از درک آن ای مردمان
\\
جامهٔ خفته خورد از جوی آب
&&
خفته اندر خواب جویای سراب
\\
می‌رود آنجا که بوی آب هست
&&
زین تفکر راه را بر خویش بست
\\
زانک آنجا گفت زینجا دور شد
&&
بر خیالی از حقی مهجور شد
\\
دوربینانند و بس خفته‌روان
&&
رحمتی آریدشان ای ره‌روان
\\
من ندیدم تشنگی خواب آورد
&&
خواب آرد تشنگی بی‌خرد
\\
خود خرد آنست کو از حق چرید
&&
نه خرد کان را عطارد آورید
\\
\end{longtable}
\end{center}
