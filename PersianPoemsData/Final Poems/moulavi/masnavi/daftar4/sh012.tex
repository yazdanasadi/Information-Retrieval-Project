\begin{center}
\section*{بخش ۱۲ - معالجه کردن برادر دباغ دباغ را به خفیه به بوی سرگین}
\label{sec:sh012}
\addcontentsline{toc}{section}{\nameref{sec:sh012}}
\begin{longtable}{l p{0.5cm} r}
خلق را می‌راند از وی آن جوان
&&
تا علاجش را نبینند آن کسان
\\
سر به گوشش برد هم‌چون رازگو
&&
پس نهاد آن چیز بر بینی او
\\
کو به کف سرگین سگ ساییده بود
&&
داروی مغز پلید آن دیده بود
\\
ساعتی شد مرد جنبیدن گرفت
&&
خلق گفتند این فسونی بد شگفت
\\
کین بخواند افسون به گوش او دمید
&&
مرده بود افسون به فریادش رسید
\\
جنبش اهل فساد آن سو بود
&&
که زنا و غمزه و ابرو بود
\\
هر کرا مشک نصیحت سود نیست
&&
لاجرم با بوی بد خو کرد نیست
\\
مشرکان را زان نجس خواندست حق
&&
کاندرون پشک زادند از سبق
\\
کرم کو زادست در سرگین ابد
&&
می‌نگرداند به عنبر خوی خود
\\
چون نزد بر وی نثار رش نور
&&
او همه جسمست بی‌دل چون قشور
\\
ور ز رش نور حق قسمیش داد
&&
هم‌چو رسم مصر سرگین مرغ‌زاد
\\
لیک نه مرغ خسیس خانگی
&&
بلک مرغ دانش و فرزانگی
\\
تو بدان مانی کز آن نوری تهی
&&
زآنک بینی بر پلیدی می‌نهی
\\
از فراقت زرد شد رخسار و رو
&&
برگ زردی میوهٔ ناپخته تو
\\
دیگ ز آتش شد سیاه و دودفام
&&
گوشت از سختی چنین ماندست خام
\\
هشت سالت جوش دادم در فراق
&&
کم نشد یک ذره خامیت و نفاق
\\
غورهٔ تو سنگ بسته کز سقام
&&
غوره‌ها اکنون مویزند و تو خام
\\
\end{longtable}
\end{center}
