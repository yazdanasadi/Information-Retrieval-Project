\begin{center}
\section*{بخش ۱۱۷ - مثال دیگر هم درین معنی}
\label{sec:sh117}
\addcontentsline{toc}{section}{\nameref{sec:sh117}}
\begin{longtable}{l p{0.5cm} r}
هست بازیهای آن شیر علم
&&
مخبری از بادهای مکتتم
\\
گر نبودی جنبش آن بادها
&&
شیر مرده کی بجستی در هوا
\\
زان شناسی باد را گر آن صباست
&&
یا دبورست این بیان آن خفاست
\\
این بدن مانند آن شیر علم
&&
فکر می‌جنباند او را دم به دم
\\
فکر کان از مشرق آید آن صباست
&&
وآنک از مغرب دبور با وباست
\\
مشرق این باد فکرت دیگرست
&&
مغرب این باد فکرت زان سرست
\\
مه جمادست و بود شرقش جماد
&&
جان جان جان بود شرق فؤاد
\\
شرق خورشیدی که شد باطن‌فروز
&&
قشر و عکس آن بود خورشید روز
\\
زآنک چون مرده بود تن بی‌لهب
&&
پیش او نه روز بنماید نه شب
\\
ور نباشد آن چو این باشد تمام
&&
بی‌شب و بی روز دارد انتظام
\\
هم‌چنانک چشم می‌بیند به خواب
&&
بی‌مه و خورشید ماه و آفتاب
\\
نوم ما چون شد اخ الموت ای فلان
&&
زین برادر آن برادر را بدان
\\
ور بگویندت که هست آن فرع این
&&
مشنو آن را ای مقلد بی‌یقین
\\
می‌بیند خواب جانت وصف حال
&&
که به بیداری نبینی بیست سال
\\
در پی تعبیر آن تو عمرها
&&
می‌دوی سوی شهان با دها
\\
که بگو آن خواب را تعبیر چیست
&&
فرع گفتن این چنین سر را سگیست
\\
خواب عامست این و خود خواب خواص
&&
باشد اصل اجتبا و اختصاص
\\
پیل باید تا چو خسپد او ستان
&&
خواب بیند خطهٔ هندوستان
\\
خر نبیند هیچ هندستان به خواب
&&
خر ز هندستان نکردست اغتراب
\\
جان هم‌چون پیل باید نیک زفت
&&
تا به خواب او هند داند رفت تفت
\\
ذکر هندستان کند پیل از طلب
&&
پس مصور گردد آن ذکرش به شب
\\
اذکروا الله کار هر اوباش نیست
&&
ارجعی بر پای هر قلاش نیست
\\
لیک تو آیس مشو هم پیل باش
&&
ور نه پیلی در پی تبدیل باش
\\
کیمیاسازان گردون را ببین
&&
بشنو از میناگران هر دم طنین
\\
نقش‌بندانند در جو فلک
&&
کارسازانند بهر لی و لک
\\
گر نبینی خلق مشکین جیب را
&&
بنگر ای شب‌کور این آسیب را
\\
هر دم آسیبست بر ادراک تو
&&
نبت نو نو رسته بین از خاک تو
\\
زین بد ابراهیم ادهم دیده خواب
&&
بسط هندستان دل را بی‌حجاب
\\
لاجرم زنجیرها را بر درید
&&
مملکت بر هم زد و شد ناپدید
\\
آن نشان دید هندستان بود
&&
که جهد از خواب و دیوانه شود
\\
می‌فشاند خاک بر تدبیرها
&&
می‌دراند حلقهٔ زنجیرها
\\
آنچنان که گفت پیغامبر ز نور
&&
که نشانش آن بود اندر صدور
\\
که تجافی آرد از دار الغرور
&&
هم انابت آرد از دار السرور
\\
بهر شرح این حدیث مصطفی
&&
داستانی بشنو ای یار صفا
\\
\end{longtable}
\end{center}
