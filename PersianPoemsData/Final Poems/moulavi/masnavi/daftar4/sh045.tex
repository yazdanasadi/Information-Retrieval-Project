\begin{center}
\section*{بخش ۴۵ - قصهٔ شاعر و صله دادن شاه و مضاعف کردن آن وزیر بوالحسن نام}
\label{sec:sh045}
\addcontentsline{toc}{section}{\nameref{sec:sh045}}
\begin{longtable}{l p{0.5cm} r}
شاعری آورد شعری پیش شاه
&&
بر امید خلعت و اکرام و جاه
\\
شاه مکرم بود فرمودش هزار
&&
از زر سرخ و کرامات و نثار
\\
پس وزیرش گفت کین اندک بود
&&
ده هزارش هدیه وا ده تا رود
\\
از چنو شاعر نس از تو بحردست
&&
ده هزاری که بگفتم اندکست
\\
فقه گفت آن شاه را و فلسفه
&&
تا برآمد عشر خرمن از کفه
\\
ده هزارش داد و خلعت درخورش
&&
خانهٔ شکر و ثنا گشت آن سرش
\\
پس تفحص کرد کین سعی کی بود
&&
شاه را اهلیت من کی نمود
\\
پس بگفتندش فلان‌الدین وزیر
&&
آن حسن نام و حسن خلق و ضمیر
\\
در ثنای او یکی شعری دراز
&&
بر نبشت و سوی خانه رفت باز
\\
بی‌زبان و لب همان نعمای شاه
&&
مدح شه می‌کرد و خلعتهای شاه
\\
\end{longtable}
\end{center}
