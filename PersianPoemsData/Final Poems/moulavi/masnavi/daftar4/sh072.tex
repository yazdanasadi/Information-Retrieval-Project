\begin{center}
\section*{بخش ۷۲ - کژ وزیدن باد بر سلیمان علیه‌السلام به سبب زلت او}
\label{sec:sh072}
\addcontentsline{toc}{section}{\nameref{sec:sh072}}
\begin{longtable}{l p{0.5cm} r}
باد بر تخت سلیمان رفت کژ
&&
پس سلیمان گفت بادا کژ مغژ
\\
باد هم گفت ای سیلمان کژ مرو
&&
ور روی کژ از کژم خشمین مشو
\\
این ترازو بهر این بنهاد حق
&&
تا رود انصاف ما را در سبق
\\
از ترازو کم کنی من کم کنم
&&
تا تو با من روشنی من روشنم
\\
هم‌چنین تاج سلیمان میل کرد
&&
روز روشن را برو چون لیل کرد
\\
گفت تا جا کژ مشو بر فرق من
&&
آفتابا کم مشو از شرق من
\\
راست می‌کرد او به دست آن تاج را
&&
باز کژ می‌شد برو تاج ای فتی
\\
هشت بارش راست کرد و گشت کژ
&&
گفت تاجا چیست آخر کژ مغژ
\\
گفت اگر صد ره کنی تو راست من
&&
کژ شوم چون کژ روی ای مؤتمن
\\
پس سلیمان اندرونه راست کرد
&&
دل بر آن شهوت که بودش کرد سرد
\\
بعد از آن تاجش همان دم راست شد
&&
آنچنان که تاج را می‌خواست شد
\\
بعد از آنش کژ همی کرد او به قصد
&&
تاج او می‌گشت تارک‌جو به قصد
\\
هشت کرت کژ بکرد آن مهترش
&&
راست می‌شد تاج بر فرق سرش
\\
تاج ناطق گشت کای شه ناز کن
&&
چون فشاندی پر ز گل پرواز کن
\\
نیست دستوری کزین من بگذرم
&&
پرده‌های غیب این برهم درم
\\
بر دهانم نه تو دست خود ببند
&&
مر دهانم را ز گفت ناپسند
\\
پس ترا هر غم که پیش آید ز درد
&&
بر کسی تهمت منه بر خویش گرد
\\
ظن مبر بر دیگری ای دوستکام
&&
آن مکن که می‌سگالید آن غلام
\\
گاه جنگش با رسول و مطبخی
&&
گاه خشمش با شهنشاه سخی
\\
هم‌چو فرعونی که موسی هشته بود
&&
طفلکان خلق را سر می‌ربود
\\
آن عدو در خانهٔ آن کور دل
&&
او شده اطفال را گردن گسل
\\
تو هم از بیرون بدی با دیگران
&&
واندرون خوش گشته با نفس گران
\\
خود عدوت اوست قندش می‌دهی
&&
وز برون تهمت به هر کس می‌نهی
\\
هم‌چو فرعونی تو کور و کوردل
&&
با عدو خوش بی‌گناهان را مذل
\\
چند فرعونا کشی بی‌جرم را
&&
می‌نوازی مر تن پر غرم را
\\
عقل او بر عقل شاهان می‌فزود
&&
حکم حق بی‌عقل و کورش کرده بود
\\
مهر حق بر چشم و بر گوش خرد
&&
گر فلاطونست حیوانش کند
\\
حکم حق بر لوح می‌آید پدید
&&
آنچنان که حکم غیب بایزید
\\
\end{longtable}
\end{center}
