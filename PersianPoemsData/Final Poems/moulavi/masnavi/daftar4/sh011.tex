\begin{center}
\section*{بخش ۱۱ - قصهٔ آن دباغ کی در بازار عطاران از بوی عطر و مشک بیهوش و رنجور شد}
\label{sec:sh011}
\addcontentsline{toc}{section}{\nameref{sec:sh011}}
\begin{longtable}{l p{0.5cm} r}
آن یکی افتاد بیهوش و خمید
&&
چونک در بازار عطاران رسید
\\
بوی عطرش زد ز عطاران راد
&&
تا بگردیدش سر و بر جا فتاد
\\
هم‌چو مردار اوفتاد او بی‌خبر
&&
نیم روز اندر میان ره‌گذر
\\
جمع آمد خلق بر وی آن زمان
&&
جملگان لاحول‌گو درمان کنان
\\
آن یکی کف بر دل او می براند
&&
وز گلاب آن دیگری بر وی فشاند
\\
او نمی‌دانست کاندر مرتعه
&&
از گلاب آمد ورا آن واقعه
\\
آن یکی دستش همی‌مالید و سر
&&
وآن دگر کهگل همی آورد تر
\\
آن بخور عود و شکر زد به هم
&&
وآن دگر از پوششش می‌کرد کم
\\
وآن دگر نبضش که تا چون می‌جهد
&&
وان دگر بوی از دهانش می‌ستد
\\
تا که می خوردست و یا بنگ و حشیش
&&
خلق درماندند اندر بیهشیش
\\
پس خبر بردند خویشان را شتاب
&&
که فلان افتاده است آن‌جا خراب
\\
کس نمی‌داند که چون مصروع گشت
&&
یا چه شد کور افتاد از بام طشت
\\
یک برادر داشت آن دباغ زفت
&&
گربز و دانا بیامد زود تفت
\\
اندکی سرگین سگ در آستین
&&
خلق را بشکافت و آمد با حنین
\\
گفت من رنجش همی دانم ز چیست
&&
چون سبب دانی دوا کردن جلیست
\\
چون سبب معلوم نبود مشکلست
&&
داروی رنج و در آن صد محملست
\\
چون بدانستی سبب را سهل شد
&&
دانش اسباب دفع جهل شد
\\
گفت با خود هستش اندر مغز و رگ
&&
توی بر تو بوی آن سرگین سگ
\\
تا میان اندر حدث او تا به شب
&&
غرق دباغیست او روزی‌طلب
\\
پس چنین گفتست جالینوس مه
&&
آنچ عادت داشت بیمار آنش ده
\\
کز خلاف عادتست آن رنج او
&&
پس دوای رنجش از معتاد جو
\\
چون جعل گشتست از سرگین‌کشی
&&
از گلاب آید جعل را بیهشی
\\
هم از آن سرگین سگ داروی اوست
&&
که بدان او را همی معتاد و خوست
\\
الخبیثات الخبیثین را بخوان
&&
رو و پشت این سخن را باز دان
\\
ناصحان او را به عنبر یا گلاب
&&
می دوا سازند بهر فتح باب
\\
مر خبیثان را نسازد طیبات
&&
درخور و لایق نباشد ای ثقات
\\
چون ز عطر وحی کر گشتند و گم
&&
بد فغانشان که تطیرنا بکم
\\
رنج و بیماریست ما را این مقال
&&
نیست نیکو وعظتان ما را به فال
\\
گر بیاغازید نصحی آشکار
&&
ما کنیم آن دم شما را سنگسار
\\
ما بلغو و لهو فربه گشته‌ایم
&&
در نصیحت خویش را نسرشته‌ایم
\\
هست قوت ما دروغ و لاف و لاغ
&&
شورش معده‌ست ما را زین بلاغ
\\
رنج را صدتو و افزون می‌کنید
&&
عقل را دارو به افیون می‌کنید
\\
\end{longtable}
\end{center}
