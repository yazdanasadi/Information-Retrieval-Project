\begin{center}
\section*{بخش ۱۰۵ - مشورت کردن فرعون با وزیرش هامان در ایمان آوردن به موسی علیه‌السلام}
\label{sec:sh105}
\addcontentsline{toc}{section}{\nameref{sec:sh105}}
\begin{longtable}{l p{0.5cm} r}
گفت با هامان چون تنهااش بدید
&&
جست هامان و گریبان را درید
\\
بانگها زد گریه‌ها کرد آن لعین
&&
کوفت دستار و کله را بر زمین
\\
که چگونه گفت اندر روی شاه
&&
این چنین گستاخ آن حرف تباه
\\
جمله عالم را مسخر کرده تو
&&
کار را با بخت چون زر کرده تو
\\
از مشارق وز مغارب بی‌لجاج
&&
سوی تو آرند سلطانان خراج
\\
پادشاهان لب همی مالند شاد
&&
بر ستانهٔ خاک تو این کیقباد
\\
اسپ یاغی چون ببیند اسپ ما
&&
رو بگرداند گریزد بی عصا
\\
تاکنون معبود و مسجود جهان
&&
بوده‌ای گردی کمینهٔ بندگان
\\
در هزار آتش شدن زین خوشترست
&&
که خداوندی شود بنده‌پرست
\\
نه بکش اول مرا ای شاه چین
&&
تا نبیند چشم من بر شاه این
\\
خسروا اول مرا گردن بزن
&&
تا نبیند این مذلت چشم من
\\
خود نبودست و مبادا این چنین
&&
که زمین گردون شود گردون زمین
\\
بندگان‌مان خواجه‌تاش ما شوند
&&
بی‌دلان‌مان دلخراش ما شوند
\\
چشم‌روشن دشمنان و دوست کور
&&
گشت ما را پس گلستان قعر گور
\\
\end{longtable}
\end{center}
