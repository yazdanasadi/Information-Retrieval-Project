\begin{center}
\section*{بخش ۶۹ - قول رسول صلی الله علیه و سلم انی لاجد نفس الرحمن من قبل الیمن}
\label{sec:sh069}
\addcontentsline{toc}{section}{\nameref{sec:sh069}}
\begin{longtable}{l p{0.5cm} r}
گفت زین سو بوی یاری می‌رسد
&&
کاندرین ده شهریاری می‌رسد
\\
بعد چندین سال می‌زاید شهی
&&
می‌زند بر آسمانها خرگهی
\\
رویش از گلزار حق گلگون بود
&&
از من او اندر مقام افزون بود
\\
چیست نامش گفت نامش بوالحسن
&&
حلیه‌اش وا گفت ز ابرو و ذقن
\\
قد او و رنگ او و شکل او
&&
یک به یک واگفت از گیسو و رو
\\
حلیه‌های روح او را هم نمود
&&
از صفات و از طریقه و جا و بود
\\
حلیهٔ تن هم‌چو تن عاریتیست
&&
دل بر آن کم نه که آن یک ساعتیست
\\
حلیهٔ روح طبیعی هم فناست
&&
حلیهٔ آن جان طلب کان بر سماست
\\
جسم او هم‌چون چراغی بر زمین
&&
نور او بالای سقف هفتمین
\\
آن شعاع آفتاب اندر وثاق
&&
قرص او اندر چهارم چارطاق
\\
نقش گل در زیربینی بهر لاغ
&&
بوی گل بر سقف و ایوان دماغ
\\
مرد خفته در عدن دیده فرق
&&
عکس آن بر جسم افتاده عرق
\\
پیرهن در مصر رهن یک حریص
&&
پر شده کنعان ز بوی آن قمیص
\\
بر نبشتند آن زمان تاریخ را
&&
از کباب آراستند آن سیخ را
\\
چون رسید آن وقت و آن تاریخ راست
&&
زاده شد آن شاه و نرد ملک باخت
\\
از پس آن سالها آمد پدید
&&
بوالحسن بعد وفات بایزید
\\
جملهٔ خوهای او ز امساک وجود
&&
آن‌چنان آمد که آن شه گفته بود
\\
لوح محفوظ است او را پیشوا
&&
از چه محفوظست محفوظ از خطا
\\
نه نجومست و نه رملست و نه خواب
&&
وحی حق والله اعلم بالصواب
\\
از پی روپوش عامه در بیان
&&
وحی دل گویند آن را صوفیان
\\
وحی دل گیرش که منظرگاه اوست
&&
چون خطا باشد چو دل آگاه اوست
\\
مؤمنا ینظر به نور الله شدی
&&
از خطا و سهو آمن آمدی
\\
\end{longtable}
\end{center}
