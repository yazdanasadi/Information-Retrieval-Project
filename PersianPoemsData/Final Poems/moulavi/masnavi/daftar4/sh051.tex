\begin{center}
\section*{بخش ۵۱ - قصهٔ صوفی کی در میان گلستان سر به زانو مراقب بود یارانش گفتند سر برآور تفرج کن بر گلستان و ریاحین و مرغان و آثار رحمةالله تعالی}
\label{sec:sh051}
\addcontentsline{toc}{section}{\nameref{sec:sh051}}
\begin{longtable}{l p{0.5cm} r}
صوفیی در باغ از بهر گشاد
&&
صوفیانه روی بر زانو نهاد
\\
پس فرو رفت او به خود اندر نغول
&&
شد ملول از صورت خوابش فضول
\\
که چه خسپی آخر اندر رز نگر
&&
این درختان بین و آثار و خضر
\\
امر حق بشنو که گفتست انظروا
&&
سوی این آثار رحمت آر رو
\\
گفت آثارش دلست ای بوالهوس
&&
آن برون آثار آثارست و بس
\\
باغها و سبزه‌ها در عین جان
&&
بر برون عکسش چو در آب روان
\\
آن خیال باغ باشد اندر آب
&&
که کند از لطف آب آن اضطراب
\\
باغها و میوه‌ها اندر دلست
&&
عکس لطف آن برین آب و گلست
\\
گر نبودی عکس آن سرو سرور
&&
پس نخواندی ایزدش دار الغرور
\\
این غرور آنست یعنی این خیال
&&
هست از عکس دل و جان رجال
\\
جمله مغروران برین عکس آمده
&&
بر گمانی کین بود جنت‌کده
\\
می‌گریزند از اصول باغها
&&
بر خیالی می‌کنند آن لاغها
\\
چونک خواب غفلت آیدشان به سر
&&
راست بینند و چه سودست آن نظر
\\
بس به گورستان غریو افتاد و آه
&&
تا قیامت زین غلط وا حسرتاه
\\
ای خنک آن را که پیش از مرگ مرد
&&
یعنی او از اصل این رز بوی برد
\\
\end{longtable}
\end{center}
