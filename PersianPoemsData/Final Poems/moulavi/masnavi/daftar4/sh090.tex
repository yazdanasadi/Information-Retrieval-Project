\begin{center}
\section*{بخش ۹۰ - بیان آنک عمارت در ویرانیست و جمعیت در پراکندگیست و  درستی در شکست‌گیست و مراد در بی‌مرادیست و وجود در عدم  است و علی هذا بقیة الاضداد والازواج}
\label{sec:sh090}
\addcontentsline{toc}{section}{\nameref{sec:sh090}}
\begin{longtable}{l p{0.5cm} r}
آن یکی آمد زمین را می‌شکافت
&&
ابلهی فریاد کرد و بر نتافت
\\
کین زمین را از چه ویران می‌کنی
&&
می‌شکافی و پریشان می‌کنی
\\
گفت ای ابله برو و بر من مران
&&
تو عمارت از خرابی باز دان
\\
کی شود گلزار و گندم‌زار این
&&
تا نگردد زشت و ویران این زمین
\\
کی شود بستان و کشت و برگ و بر
&&
تا نگردد نظم او زیر و زبر
\\
تا بنشکافی به نشتر ریش چغز
&&
کی شود نیکو و کی گردید نغز
\\
تا نشوید خلطهاات از دوا
&&
کی رود شورش کجا آید شفا
\\
پاره پاره کرده درزی جامه را
&&
کس زند آن درزی علامه را
\\
که چرا این اطلس بگزیده را
&&
بردریدی چه کنم بدریده را
\\
هر بنای کهنه که آبادان کنند
&&
نه که اول کهنه را ویران کنند
\\
هم‌چنین نجار و حداد و قصاب
&&
هستشان پیش از عمارتها خراب
\\
آن هلیله و آن بلیله کوفتن
&&
زان تلف گردند معموری تن
\\
تا نکوبی گندم اندر آسیا
&&
کی شود آراسته زان خوان ما
\\
آن تقاضا کرد آن نان و نمک
&&
که ز شستت وا رهانم ای سمک
\\
گر پذیری پند موسی وا رهی
&&
از چنین شست بد نامنتهی
\\
بس که خود را کرده‌ای بندهٔ هوا
&&
کرمکی را کرده‌ای تو اژدها
\\
اژدها را اژدها آورده‌ام
&&
تا با صلاح آورم من دم به دم
\\
تا دم آن از دم این بشکند
&&
مار من آن اژدها را بر کند
\\
گر رضا دادی رهیدی از دو مار
&&
ورنه از جانت برآرد آن دمار
\\
گفت الحق سخت استا جادوی
&&
که در افکندی به مکر اینجا دوی
\\
خلق یک‌دل را تو کردی دو گروه
&&
جادوی رخنه کند در سنگ و کوه
\\
گفت هستم غرق پیغام خدا
&&
جادوی کی دید با نام خدا
\\
غفلت و کفرست مایهٔ جادوی
&&
مشعلهٔ دینست جان موسوی
\\
من به جادویان چه مانم ای وقیح
&&
کز دمم پر رشک می‌گردد مسیح
\\
من به جادویان چه مانم ای جنب
&&
که ز جانم نور می‌گیرد کتب
\\
چون تو با پر هوا بر می‌پری
&&
لاجرم بر من گمان آن می‌بری
\\
هر کرا افعال دام و دد بود
&&
بر کریمانش گمان بد بود
\\
چون تو جزو عالمی هر چون بوی
&&
کل را بر وصف خود بینی سوی
\\
گر تو برگردی و بر گردد سرت
&&
خانه را گردنده بیند منظرت
\\
ور تو در کشتی روی بر یم روان
&&
ساحل یم را همی بینی دوان
\\
گر تو باشی تنگ‌دل از ملحمه
&&
تنگ بینی جمله دنیا را همه
\\
ور تو خوش باشی به کام دوستان
&&
این جهان بنمایدت چون گلستان
\\
ای بسا کس رفته تا شام و عراق
&&
او ندیده هیچ جز کفر و نفاق
\\
وی بسا کس رفته تا هند و هری
&&
او ندیده جز مگر بیع و شری
\\
وی بسا کس رفته ترکستان و چین
&&
او ندیده هیچ جز مکر و کمین
\\
چون ندارد مدرکی جز رنگ و بو
&&
جملهٔ اقلیمها را گو بجو
\\
گاو در بغداد آید ناگهان
&&
بگذرد او زین سران تا آن سران
\\
از همه عیش و خوشیها و مزه
&&
او نبیند جز که قشر خربزه
\\
که بود افتاده بر ره یا حشیش
&&
لایق سیران گاوی یا خریش
\\
خشک بر میخ طبیعت چون قدید
&&
بستهٔ اسباب جانش لا یزید
\\
وان فضای خرق اسباب و علل
&&
هست ارض الله ای صدر اجل
\\
هر زمان مبدل شود چون نقش جان
&&
نو به نو بیند جهانی در عیان
\\
گر بود فردوس و انهار بهشت
&&
چون فسردهٔ یک صفت شد گشت زشت
\\
\end{longtable}
\end{center}
