\begin{center}
\section*{بخش ۸۶ - قصهٔ آن مرغ گرفته کی وصیت کرد کی بر گذشته پشیمانی مخور تدارک وقت اندیش و روزگار مبر در پشیمانی}
\label{sec:sh086}
\addcontentsline{toc}{section}{\nameref{sec:sh086}}
\begin{longtable}{l p{0.5cm} r}
آن یکی مرغی گرفت از مکر و دام
&&
مرغ او را گفت ای خواجهٔ همام
\\
به تو بسی گاوان و میشان خورده‌ای
&&
تو بسی اشتر به قربان کرده‌ای
\\
تو نگشتی سیر زانها در زمن
&&
هم نگردی سیر از اجزای من
\\
هل مرا تا که سه پندت بر دهم
&&
تا بدانی زیرکم یا ابلهم
\\
اول آن پند هم در دست تو
&&
ثانیش بر بام کهگل بست تو
\\
وآن سوم پند دهم من بر درخت
&&
که ازین سه پند گردی نیکبخت
\\
آنچ بر دستست اینست آن سخن
&&
که محالی را ز کس باور مکن
\\
بر کفش چون گفت اول پند زفت
&&
گشت آزاد و بر آن دیوار رفت
\\
گفت دیگر بر گذشته غم مخور
&&
چون ز تو بگذشت زان حسرت مبر
\\
بعد از آن گفتش که در جسمم کتیم
&&
ده درمسنگست یک در یتیم
\\
دولت تو بخت فرزندان تو
&&
بود آن گوهر به حق جان تو
\\
فوت کردی در که روزی‌ات نبود
&&
که نباشد مثل آن در در وجود
\\
آنچنان که وقت زادن حامله
&&
ناله دارد خواجه شد در غلغله
\\
مرغ گفتش نی نصیحت کردمت
&&
که مبادا بر گذشتهٔ دی غمت
\\
چون گذشت و رفت غم چون می‌خوری
&&
یا نکردی فهم پندم یا کری
\\
وان دوم پندت بگفتم کز ضلال
&&
هیچ تو باور مکن قول محال
\\
من نیم خود سه درمسنگ ای اسد
&&
ده درمسنگ اندرونم چون بود
\\
خواجه باز آمد به خود گفتا که هین
&&
باز گو آن پند خوب سیومین
\\
گفت آری خوش عمل کردی بدان
&&
تا بگویم پند ثالث رایگان
\\
پند گفتن با جهول خوابناک
&&
تخت افکندن بود در شوره خاک
\\
چاک حمق و جهل نپذیرد رفو
&&
تخم حکمت کم دهش ای پندگو
\\
\end{longtable}
\end{center}
