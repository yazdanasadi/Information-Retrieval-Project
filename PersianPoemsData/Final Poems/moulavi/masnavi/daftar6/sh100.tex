\begin{center}
\section*{بخش ۱۰۰ - ماخذهٔ یوسف صدیق صلوات‌الله علیه به حبس بضع سنین به سبب یاری خواستن از غیر حق و گفتن اذکرنی عند ربک مع  تقریره}
\label{sec:sh100}
\addcontentsline{toc}{section}{\nameref{sec:sh100}}
\begin{longtable}{l p{0.5cm} r}
آنچنان که یوسف از زندانیی
&&
با نیازی خاضعی سعدانیی
\\
خواست یاری گفت چون بیرون روی
&&
پیش شه گردد امورت مستوی
\\
یاد من کن پیش تخت آن عزیز
&&
تا مرا هم وا خرد زین حبس نیز
\\
کی دهد زندانیی در اقتناص
&&
مرد زندانی دیگر را خلاص
\\
اهل دنیا جملگان زندانیند
&&
انتظار مرگ دار فانیند
\\
جز مگر نادر یکی فردانیی
&&
تن بزندان جان او کیوانیی
\\
پس جزای آنک دید او را معین
&&
ماند یوسف حبس در بضع سنین
\\
یاد یوسف دیو از عقلش سترد
&&
وز دلش دیو آن سخن از یاد برد
\\
زین گنه کامد از آن نیکوخصال
&&
ماند در زندان ز داور چند سال
\\
که چه تقصیر آمد از خورشید داد
&&
تا تو چون خفاش افتی در سواد
\\
هین چه تقصیر آمد از بحر و سحاب
&&
تا تو یاری خواهی از ریگ و سراب
\\
عام اگر خفاش طبعند و مجاز
&&
یوسفا داری تو آخر چشم باز
\\
گر خفاشی رفت در کور و کبود
&&
باز سلطان دیده را باری چه بود
\\
پس ادب کردش بدین جرم اوستاد
&&
که مساز از چوب پوسیده عماد
\\
لیک یوسف را به خود مشغول کرد
&&
تا نیاید در دلش زان حبس درد
\\
آن‌چنانش انس و مستی داد حق
&&
که نه زندان ماند پیشش نه غسق
\\
نیست زندانی وحش‌تر از رحم
&&
ناخوش و تاریک و پرخون و وخم
\\
چون گشادت حق دریچه سوی خویش
&&
در رحم هر دم فزاید تنت بیش
\\
اندر آن زندان ز ذوق بی‌قیاس
&&
خوش شکفت از غرس جسم تو حواس
\\
زان رحم بیرون شدن بر تو درشت
&&
می‌گریزی از زهارش سوی پشت
\\
راه لذت از درون دان نه از برون
&&
ابلهی دان جستن قصر و حصون
\\
آن یکی در کنج مسجد مست و شاد
&&
وآن دگر در باغ ترش و بی‌مراد
\\
قصر چیزی نیست ویران کن بدن
&&
گنج در ویرانیست ای میر من
\\
این نمی‌بینی که در بزم شراب
&&
مست آنگه خوش شود کو شد خراب
\\
گرچه پر نقش است خانه بر کنش
&&
گنج جو و از گنج آبادان کنش
\\
خانهٔ پر نقش تصویر و خیال
&&
وین صور چون پرده بر گنج وصال
\\
پرتو گنجست و تابش‌های زر
&&
که درین سینه همی‌جوشد صور
\\
هم ز لطف و عکس آب با شرف
&&
پرده شد بر روی آب اجزای کف
\\
هم ز لطف و جوش جان با ثمن
&&
پرده‌ای بر روی جان شد شخص تن
\\
پس مثل بشنو که در افواه خاست
&&
که اینچ بر ماست ای برادر هم ز ماست
\\
زین حجاب این تشنگان کف‌پرست
&&
ز آب صافی اوفتاده دوردست
\\
آفتابا با چو تو قبله و امام
&&
شب‌پرستی و خفاشی می‌کنیم
\\
سوی خود کن این خفاشان را مطار
&&
زین خفاشیشان بخر ای مستجار
\\
این جوان زین جرم ضالست و مغیر
&&
که بمن آمد ولی او را مگیر
\\
در عماد الملک این اندیشه‌ها
&&
گشته جوشان چون اسد در بیشه‌ها
\\
ایستاده پیش سلطان ظاهرش
&&
در ریاض غیب جان طایرش
\\
چون ملایک او به اقلیم الست
&&
هر دمی می‌شد به شرب تازه مست
\\
اندرون سور و برون چون پر غمی
&&
در تن هم‌چون لحد خوش عالمی
\\
او درین حیرت بد و در انتظار
&&
تا چه پیدا آید از غیب و سرار
\\
اسپ را اندر کشیدند آن زمان
&&
پیش خوارمشاه سرهنگان کشان
\\
الحق اندر زیر این چرخ کبود
&&
آن‌چنان کره به قد و تگ نبود
\\
می‌ربودی رنگ او هر دیده را
&&
مرحب آن از برق و مه زاییده را
\\
هم‌چو مه هم‌چون عطارد تیزرو
&&
گوییی صرصر علف بودش نه جو
\\
ماه عرصهٔ آسمان را در شبی
&&
می‌برد اندر مسیر و مذهبی
\\
چون به یک شب مه برید ابراج را
&&
از چه منکر می‌شوی معراج را
\\
صد چو ماهست آن عجب در یتیم
&&
که به یک ایماء او شد مه دو نیم
\\
آن عجب کو در شکاف مه نمود
&&
هم به قدر ضعف حس خلق بود
\\
کار و بار انبیا و مرسلون
&&
هست از افلاک و اخترها برون
\\
تو برون رو هم ز افلاک و دوار
&&
وانگهان نظاره کن آن کار و بار
\\
در میان بیضه‌ای چون فرخ‌ها
&&
نشنوی تسبیح مرغان هوا
\\
معجزات این‌جا نخواهد شرح گشت
&&
ز اسپ و خوارمشاه گو و سرگذشت
\\
آفتاب لطف حق بر هر چه تافت
&&
از سگ و از اسپ فر کهف یافت
\\
تاب لطفش را تو یکسان هم مدان
&&
سنگ را و لعل را داد او نشان
\\
لعل را زان هست گنج مقتبس
&&
سنگ را گرمی و تابانی و بس
\\
آنک بر دیوار افتد آفتاب
&&
آن‌چنان نبود کز آب و اضطراب
\\
چون دمی حیران شد از وی شاه فرد
&&
روی خود سوی عماد الملک کرد
\\
کای اچی بس خوب اسپی نیست این
&&
از بهشتست این مگر نه از زمین
\\
پس عماد الملک گفتش ای خدیو
&&
چون فرشته گردد از میل تو دیو
\\
در نظر آنچ آوری گردید نیک
&&
بس گش و رعناست این مرکب ولیک
\\
هست ناقص آن سر اندر پیکرش
&&
چون سر گاوست گویی آن سرش
\\
در دل خوارمشه این دم کار کرد
&&
اسپ را در منظر شه خوار کرد
\\
چون غرض دلاله گشت و واصفی
&&
از سه گز کرباس یابی یوسفی
\\
چونک هنگام فراق جان شود
&&
دیو دلال در ایمان شود
\\
پس فروشد ابله ایمان را شتاب
&&
اندر آن تنگی به یک ابریق آب
\\
وان خیالی باشد و ابریق نی
&&
قصد آن دلال جز تخریق نی
\\
این زمان که تو صحیح و فربهی
&&
صدق را بهر خیالی می‌دهی
\\
می‌فروشی هر زمانی در کان
&&
هم‌چو طفلی می‌ستانی گردگان
\\
پس در آن رنجوری روز اجل
&&
نیست نادر گر بود اینت عمل
\\
در خیالت صورتی جوشیده‌ای
&&
هم‌چو جوزی وقت دق پوسیده‌ای
\\
هست از آغاز چون بدر آن خیال
&&
لیک آخر می‌شود هم‌چون هلال
\\
گر تو اول بنگری چون آخرش
&&
فارغ آیی از فریب فاترش
\\
جوز پوسیده‌ست دنیا ای امین
&&
امتحانش کم کن از دورش ببین
\\
شاه دید آن اسپ را با چشم حال
&&
وآن عمادالملک با چشم مل
\\
چشم شه دو گز همی دید از لغز
&&
چشم آن پایان‌نگر پنجاه گز
\\
آن چه سرمه‌ست آنک یزدان می‌کشد
&&
کز پس صد پرده بیند جان رشد
\\
چشم مهتر چون به آخر بود جفت
&&
پس بدان دیده جهان را جیفه گفت
\\
زین یکی ذمش که بشنود او وحسپ
&&
پس فسرد اندر دل شه مهر اسپ
\\
چشم خود بگذاشت و چشم او گزید
&&
هوش خود بگذاشت و قول او شنید
\\
این بهانه بود و آن دیان فرد
&&
از نیاز آن در دل شه سرد کرد
\\
در ببست از حسن او پیش بصر
&&
آن سخن بد در میان چون بانگ در
\\
پرده کرد آن نکته را بر چشم شه
&&
که از آن پرده نماید مه سیه
\\
پاک بنایی که بر سازد حصون
&&
در جهان غیب از گفت و فسون
\\
بانگ در دان گفت را از قصر راز
&&
تا که بانگ وا شدست این یا فراز
\\
بانگ در محسوس و در از حس برون
&&
تبصرون این بانگ و در لا تبصرون
\\
چنگ حکمت چونک خوش‌آواز شد
&&
تا چه در از روض جنت باز شد
\\
بانگ گفت بد چو دروا می‌شود
&&
از سقر تا خود چه در وا می‌شود
\\
بانگ در بشنو چو دوری از درش
&&
ای خنک او را که وا شد منظرش
\\
چون تو می‌بینی که نیکی می‌کنی
&&
بر حیات و راحتی بر می‌زنی
\\
چونک تقصیر و فسادی می‌رود
&&
آن حیات و ذوق پنهان می‌شود
\\
دید خود مگذار از دید خسان
&&
که به مردارت کشند این کرکسان
\\
چشم چون نرگس فروبندی که چی
&&
هین عصاام کش که کورم ای اچی
\\
وان عصاکش که گزیدی در سفر
&&
خود ببینی باشد از تو کورتر
\\
دست کورانه به حبل الله زن
&&
جز بر امر و نهی یزدانی متن
\\
چیست حبل‌الله رها کردن هوا
&&
کین هوا شد صرصری مر عاد را
\\
خلق در زندان نشسته از هواست
&&
مرغ را پرها ببسته از هواست
\\
ماهی اندر تابهٔ گرم از هواست
&&
رفته از مستوریان شرم از هواست
\\
خشم شحنه شعلهٔ نار از هواست
&&
چارمیخ و هیبت دار از هواست
\\
شحنهٔ اجسام دیدی بر زمین
&&
شحنهٔ احکام جان را هم ببین
\\
روح را در غیب خود اشکنجه‌هاست
&&
لیک تا نجهی شکنجه در خفاست
\\
چون رهیدی بینی اشکنجه و دمار
&&
زانک ضد از ضد گردد آشکار
\\
آنک در چه زاد و در آب سیاه
&&
او چه داند لطف دشت و رنج چاه
\\
چون رها کردی هوا از بیم حق
&&
در رسد سغراق از تسنیم حق
\\
لا تطرق فی هواک سل سبیل
&&
من جناب الله نحو السلسبیل
\\
لا تکن طوع الهوی مثل الحشیش
&&
ان ظل العرش اولی من عریش
\\
گفت سلطان اسپ را وا پس برید
&&
زودتر زین مظلمه بازم خرید
\\
با دل خود شه نفرمود این قدر
&&
شیر را مفریب زین راس البقر
\\
پای گاو اندر میان آری ز داو
&&
رو ندوزد حق بر اسپی شاخ گاو
\\
بس مناسب صنعتست این شهره زاو
&&
کی نهد بر جسم اسپ او عضو گاو
\\
زاو ابدان را مناسب ساخته
&&
قصرهای منتقل پرداخته
\\
در میان قصرها تخریج‌ها
&&
از سوی این سوی آن صهریج‌ها
\\
وز درونشان عالمی بی‌منتها
&&
در میان خرگهی چندین فضا
\\
گه چو کابوسی نماید ماه را
&&
گه نماید روضه قعر چاه را
\\
قبض و بسط چشم دل از ذوالجلال
&&
دم به دم چون می‌کند سحر حلال
\\
زین سبب درخواست از حق مصطفی
&&
زشت را هم زشت و حق را حق‌نما
\\
تا به آخر چون بگردانی ورق
&&
از پشیمانی نه افتم در قلق
\\
مکر که کرد آن عماد الملک فرد
&&
مالک الملکش بدان ارشاد کرد
\\
مکر حق سرچشمهٔ این مکرهاست
&&
قلب بین اصبعین کبریاست
\\
آنک سازد در دلت مکر و قیاس
&&
آتشی داند زدن اندر پلاس
\\
\end{longtable}
\end{center}
