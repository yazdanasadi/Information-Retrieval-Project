\begin{center}
\section*{بخش ۵۵ - دعوی کردن ترک و گرو بستن او کی درزی از من چیزی نتواند بردن}
\label{sec:sh055}
\addcontentsline{toc}{section}{\nameref{sec:sh055}}
\begin{longtable}{l p{0.5cm} r}
گفت خیاطیست نامش پور شش
&&
اندرین چستی و دزدی خلق‌کش
\\
گفت من ضامن که با صد اضطراب
&&
او نیارد برد پیشم رشته‌تاب
\\
پس بگفتندش که از تو چست‌تر
&&
مات او گشتند در دعوی مپر
\\
رو به عقل خود چنین غره مباش
&&
که شوی یاوه تو در تزویرهاش
\\
گرم‌تر شد ترک و بست آنجا گرو
&&
که نیارد برد نی کهنه نی نو
\\
مطمعانش گرم‌تر کردند زود
&&
او گرو بست و رهان را بر گشود
\\
که گرو این مرکب تازی من
&&
بدهم ار دزدد قماشم او به فن
\\
ور نتواند برد اسپی از شما
&&
وا ستانم بهر رهن مبتدا
\\
ترک را آن شب نبرد از غصه خواب
&&
با خیال دزد می‌کرد او حراب
\\
بامدادان اطلسی زد در بغل
&&
شد به بازار و دکان آن دغل
\\
پس سلامش کرد گرم و اوستاد
&&
جست از جا لب به ترحیبش گشاد
\\
گرم پرسیدش ز حد ترک بیش
&&
تا فکند اندر دل او مهر خویش
\\
چون بدید از وی نوای بلبلی
&&
پیشش افکند اطلس استنبلی
\\
که ببر این را قبای روز جنگ
&&
زیر نافم واسع و بالاش تنگ
\\
تنگ بالا بهر جسم‌آرای را
&&
زیر واسع تا نگیرد پای را
\\
گفت صد خدمت کنم ای ذو وداد
&&
در قبولش دست بر دیده نهاد
\\
پس بپیمود و بدید او روی کار
&&
بعد از آن بگشاد لب را در فشار
\\
از حکایتهای میران دگر
&&
وز کرمها و عطاء آن نفر
\\
وز بخیلان و ز تحشیراتشان
&&
از برای خنده هم داد او نشان
\\
هم‌چو آتش کرد مقراضی برون
&&
می‌برید و لب پر افسانه و فسون
\\
\end{longtable}
\end{center}
