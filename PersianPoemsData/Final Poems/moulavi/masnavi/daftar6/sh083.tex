\begin{center}
\section*{بخش ۸۳ - جواب گفتن مسلمان آنچ دید به یارانش جهود و ترسا و حسرت خوردن ایشان}
\label{sec:sh083}
\addcontentsline{toc}{section}{\nameref{sec:sh083}}
\begin{longtable}{l p{0.5cm} r}
پس مسلمان گفت ای یاران من
&&
پیشم آمد مصطفی سلطان من
\\
پس مرا گفت آن یکی بر طور تاخت
&&
با کلیم حق و نرد عشق باخت
\\
وان دگر را عیسی صاحب‌قران
&&
برد بر اوج چهارم آسمان
\\
خیز ای پس ماندهٔ دیده ضرر
&&
باری آن حلوا و یخنی را بخور
\\
آن هنرمندان پر فن راندند
&&
نامهٔ اقبال و منصب خواندند
\\
آن دو فاضل فضل خود در یافتند
&&
با ملایک از هنر در بافتند
\\
ای سلیم گول واپس مانده هین
&&
بر جه و بر کاسهٔ حلوا نشین
\\
پس بگفتندش که آنگه تو حریص
&&
ای عجیب خوردی ز حلوا و خبیص
\\
گفت چون فرمود آن شاه مطاع
&&
من کی بودم تا کنم زان امتناع
\\
تو جهود از امر موسی سر کشی
&&
گر بخواند در خوشی یا ناخوشی
\\
تو مسیحی هیچ از امر مسیح
&&
سر توانی تافت در خیر و قبیح
\\
من ز فخر انبیا سر چون کشم
&&
خورده‌ام حلوا و این دم سرخوشم
\\
پس بگفتندش که والله خواب راست
&&
تو بدیدی وین به از صد خواب ماست
\\
خواب تو بیداریست ای بو بطر
&&
که به بیداری عیانستش اثر
\\
در گذر از فضل و از جهدی و فن
&&
کار خدمت دارد و خلق حسن
\\
بهر این آوردمان یزدان برون
&&
ما خلقت الانس الا یعبدون
\\
سامری را آن هنر چه سود کرد
&&
کان فن از باب اللهش مردود کرد
\\
چه کشید از کیمیا قارون ببین
&&
که فرو بردش به قعر خود زمین
\\
بوالحکم آخر چه بر بست از هنر
&&
سرنگون رفت او ز کفران در سقر
\\
خود هنر آن داد که دید آتش عیان
&&
نه کپ دل علی النار الدخان
\\
ای دلیلت گنده‌تر پیش لبیب
&&
در حقیقت از دلیل آن طبیب
\\
چون دلیلت نیست جز این ای پسر
&&
گوه می‌خور در کمیزی می‌نگر
\\
ای دلیل تو مثال آن عصا
&&
در کفت دل علی عیب العمی
\\
غلغل و طاق و طرنب و گیر و دار
&&
که نمی‌بینم مرا معذور دار
\\
\end{longtable}
\end{center}
