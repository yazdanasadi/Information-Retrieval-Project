\begin{center}
\section*{بخش ۱۳۴ - وسوسه‌ای کی پادشاه‌زاده را پیدا شد از سبب استغنایی و کشفی کی از شاه دل او را حاصل شده بود و قصد ناشکری و سرکشی می‌کرد شاه را از راه الهام و سر شاه را خبر شد دلش درد کرد روح او را زخمی زد چنانک صورت شاه را خبر نبود الی آخره}
\label{sec:sh134}
\addcontentsline{toc}{section}{\nameref{sec:sh134}}
\begin{longtable}{l p{0.5cm} r}
چون مسلم گشت بی‌بیع و شری
&&
از درون شاه در جانش جری
\\
قوت می‌خوردی ز نور جان شاه
&&
ماه جانش هم‌چو از خورشید ماه
\\
راتبهٔ جانی ز شاه بی‌ندید
&&
دم به دم در جان مستش می‌رسید
\\
آن نه که ترسا و مشرک می‌خورند
&&
زان غذایی که ملایک می‌خورند
\\
اندرون خویش استغنا بدید
&&
گشت طغیانی ز استغنا پدید
\\
که نه من هم شاه و هم شه‌زاده‌ام
&&
چون عنان خود بدین شه داده‌ام
\\
چون مرا ماهی بر آمد با لمع
&&
من چرا باشم غباری را تبع
\\
آب در جوی منست و وقت ناز
&&
ناز غیر از چه کشم من بی‌نیاز
\\
سر چرا بندم چو درد سر نماند
&&
وقت روی زرد و چشم تر نماند
\\
چون شکرلب گشته‌ام عارض قمر
&&
باز باید کرد دکان دگر
\\
زین منی چون نفس زاییدن گرفت
&&
صد هزاران ژاژ خاییدن گرفت
\\
صد بیابان زان سوی حرص و حسد
&&
تا بدان‌جا چشم بد هم می‌رسد
\\
بحر شه که مرجع هر آب اوست
&&
چون نداند آنچ اندر سیل و جوست
\\
شاه را دل درد کرد از فکر او
&&
ناسپاسی عطای بکر او
\\
گفت آخر ای خس واهی‌ادب
&&
این سزای داد من بود ای عجب
\\
من چه کردم با تو زین گنج نفیس
&&
تو چه کردی با من از خوی خسیس
\\
من ترا ماهی نهادم در کنار
&&
که غروبش نیست تا روز شمار
\\
در جزای آن عطای نور پاک
&&
تو زدی در دیدهٔ من خار و خاک
\\
من ترا بر چرخ گشته نردبان
&&
تو شده در حرب من تیر و کمان
\\
درد غیرت آمد اندر شه پدید
&&
عکس درد شاه اندر وی رسید
\\
مرغ دولت در عتابش بر طپید
&&
پردهٔ آن گوشه گشته بر درید
\\
چون درون خود بدید آن خوش‌پسر
&&
از سیه‌کاری خود گرد و اثر
\\
از وظیفهٔ لطف و نعمت کم شده
&&
خانهٔ شادی او پر غم شده
\\
با خود آمد او ز مستی عقار
&&
زان گنه گشته سرش خانهٔ خمار
\\
خورده گندم حله زو بیرون شده
&&
خلد بر وی بادیه و هامون شده
\\
دید کان شربت ورا بیمار کرد
&&
زهر آن ما و منیها کار کرد
\\
جان چون طاوس در گل‌زار ناز
&&
هم‌چو چغدی شد به ویرانهٔ مجاز
\\
هم‌چو آدم دور ماند او از بهشت
&&
در زمین می‌راند گاوی بهر کشت
\\
اشک می‌راند او کای هندوی زاو
&&
شیر را کردی اسیر دم گاو
\\
کردی ای نفس بد بارد نفس
&&
بی‌حفاظی با شه فریادرس
\\
دام بگزیدی ز حرص گندمی
&&
بر تو شد هر گندم او کزدمی
\\
در سرت آمد هوای ما و من
&&
قید بین بر پای خود پنجاه من
\\
نوحه می‌کرد این نمط بر جان خویش
&&
که چرا گشتم ضد سلطان خویش
\\
آمد او با خویش و استغفار کرد
&&
با انابت چیز دیگر یار کرد
\\
درد کان از وحشت ایمان بود
&&
رحم کن کان درد بی‌درمان بود
\\
مر بشر را خود مبا جامهٔ درست
&&
چون رهید از صبر در حین صدر جست
\\
مر بشر را پنجه و ناخن مباد
&&
که نه دین اندیشد آنگه نه سداد
\\
آدمی اندر بلا کشته بهست
&&
نفس کافر نعمتست و گمرهست
\\
\end{longtable}
\end{center}
