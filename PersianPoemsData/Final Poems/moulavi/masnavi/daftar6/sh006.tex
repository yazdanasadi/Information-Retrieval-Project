\begin{center}
\section*{بخش ۶ - صبر فرمودن خواجه مادر دختر را کی غلام را زجر مکن من او را بی‌زجر ازین طمع باز آرم کی نه سیخ سوزد نه کباب خام ماند}
\label{sec:sh006}
\addcontentsline{toc}{section}{\nameref{sec:sh006}}
\begin{longtable}{l p{0.5cm} r}
گفت خواجه صبر کن با او بگو
&&
که ازو ببریم و بدهیمش به تو
\\
تا مگر این از دلش بیرون کنم
&&
تو تماشا کن که دفعش چون کنم
\\
تو دلش خوش کن بگو می‌دان درست
&&
که حقیقت دختر ما جفت تست
\\
ما ندانستیم ای خوش مشتری
&&
چونک دانستیم تو اولیتری
\\
آتش ما هم درین کانون ما
&&
لیلی آن ما و تو مجنون ما
\\
تا خیال و فکر خوش بر وی زند
&&
فکر شیرین مرد را فربه کند
\\
جانور فربه شود لیک از علف
&&
آدمی فربه ز عزست و شرف
\\
آدمی فربه شود از راه گوش
&&
جانور فربه شود از حلق و نوش
\\
گفت آن خاتون ازین ننگ مهین
&&
خود دهانم کی بجنبد اندرین
\\
این چنین ژاژی چه خایم بهر او
&&
گو بمیر آن خاین ابلیس‌خو
\\
گفت خواجه نی مترس و دم دهش
&&
تا رود علت ازو زین لطف خوش
\\
دفع او را دلبرا بر من نویس
&&
هل که صحت یابد آن باریک‌ریس
\\
چون بگفت آن خسته را خاتون چنین
&&
می‌نگنجید از تبختر بر زمین
\\
زفت گشت و فربه و سرخ و شکفت
&&
چون گل سرخ هزاران شکر گفت
\\
که گهی می‌گفت ای خاتون من
&&
که مبادا باشد این دستان و فن
\\
خواجه جمعیت بکرد و دعوتی
&&
که همی‌سازم فرج را وصلتی
\\
تا جماعت عشوه می‌دادند و گان
&&
که ای فرج بادت مبارک اتصال
\\
تا یقین‌تر شد فرج را آن سخن
&&
علت از وی رفت کل از بیخ و بن
\\
بعد از آن اندر شب گردک به فن
&&
امردی را بست حنی هم‌چو زن
\\
پر نگارش کرد ساعد چون عروس
&&
پس نمودش ماکیان دادش خروس
\\
مقنعه و حلهٔ عروسان نکو
&&
کنگ امرد را بپوشانید او
\\
شمع را هنگام خلوت زود کشت
&&
ماند هندو با چنان کنگ درشت
\\
هندوک فریاد می‌کرد و فغان
&&
از برون نشنید کس از دف‌زنان
\\
ضرب دف و کف و نعرهٔ مرد و زن
&&
کرد پنهان نعرهٔ آن نعره‌زن
\\
تا به روز آن هندوک را می‌فشارد
&&
چون بود در پیش سگ انبان آرد
\\
زود آوردند طاس و بوغ زفت
&&
رسم دامادان فرج حمام رفت
\\
رفت در حمام او رنجور جان
&&
کون دریده هم‌چو دلق تونیان
\\
آمد از حمام در گردک فسوس
&&
پیش او بنشست دختر چون عروس
\\
مادرش آنجا نشسته پاسبان
&&
که نباید کو کند روز امتحان
\\
ساعتی در وی نظر کرد از عناد
&&
آنگهان با هر دو دستش ده بداد
\\
گفت کس را خود مبادا اتصال
&&
با چو تو ناخوش عروس بدفعال
\\
روز رویت روی خاتونان تر
&&
کیر زشتت شب بتر از کیر خر
\\
هم‌چنان جمله نعیم این جهان
&&
بس خوشست از دور پیش از امتحان
\\
می‌نماید در نظر از دور آب
&&
چون روی نزدیک باشد آن سراب
\\
گنده پیرست او و از بس چاپلوس
&&
خویش را جلوه کند چون نو عروس
\\
هین مشو مغرور آن گلگونه‌اش
&&
نوش نیش‌آلودهٔ او را مچش
\\
صبر کن کالصبر مفتاح الفرج
&&
تا نیفتی چون فرج در صد حرج
\\
آشکارا دانه پنهان دام او
&&
خوش نماید ز اولت انعام او
\\
\end{longtable}
\end{center}
