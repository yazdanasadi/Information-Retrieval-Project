\begin{center}
\section*{بخش ۳۱ - معاتبهٔ مصطفی علیه‌السلام با صدیق رضی الله عنه کی ترا وصیت کردم کی به شرکت من بخر تو چرا بهر خود تنها خریدی و عذر او}
\label{sec:sh031}
\addcontentsline{toc}{section}{\nameref{sec:sh031}}
\begin{longtable}{l p{0.5cm} r}
گفت ای صدیق آخر گفتمت
&&
که مرا انباز کن در مکرمت
\\
گفت ما دو بندگان کوی تو
&&
کردمش آزاد من بر روی تو
\\
تو مرا می‌دار بنده و یار غار
&&
هیچ آزادی نخواهم زینهار
\\
که مرا از بندگیت آزادیست
&&
بی‌تو بر من محنت و بیدادیست
\\
ای جهان را زنده کرده ز اصطفا
&&
خاص کرده عام را خاصه مرا
\\
خوابها می‌دید جانم در شباب
&&
که سلامم کرد قرص آفتاب
\\
از زمینم بر کشید او بر سما
&&
همره او گشته بودم ز ارتقا
\\
گفتم این ماخولیا بود و محال
&&
هیچ گردد مستحیلی وصف حال
\\
چون ترا دیدم بدیدم خویش را
&&
آفرین آن آینهٔ خوش کیش را
\\
چون ترا دیدم محالم حال شد
&&
جان من مستغرق اجلال شد
\\
چون ترا دیدم خود ای روح البلاد
&&
مهر این خورشید از چشمم فتاد
\\
گشت عالی‌همت از نو چشم من
&&
جز به خواری نگردد اندر چمن
\\
نور جستم خود بدیدم نور نور
&&
حور جستم خود بدیدم رشک حور
\\
یوسفی جستم لطیف و سیم تن
&&
یوسفستانی بدیدم در تو من
\\
در پی جنت بدم در جست و جو
&&
جنتی بنمود از هر جزو تو
\\
هست این نسبت به من مدح و ثنا
&&
هست این نسبت به تو قدح و هجا
\\
هم‌چو مدح مرد چوپان سلیم
&&
مر خدا را پیش موسی کلیم
\\
که بجویم اشپشت شیرت دهم
&&
چارقت دوم من و پیشت نهم
\\
قدح او را حق به مدحی برگرفت
&&
گر تو هم رحمت کنی نبود شگفت
\\
رحم فرما بر قصور فهمها
&&
ای ورای عقلها و وهمها
\\
ایها العشاق اقبالی جدید
&&
از جهان کهنهٔ نوگر رسید
\\
زان جهان کو چارهٔ بیچاره‌جوست
&&
صد هزاران نادره دنیا دروست
\\
ابشروا یا قوم اذ جاء الفرج
&&
افرحوا یا قوم قد زال الحرج
\\
آفتابی رفت در کازهٔ هلال
&&
در تقاضا که ارحنا یا بلال
\\
زیر لب می‌گفتی از بیم عدو
&&
کوری او بر مناره رو بگو
\\
می‌دمد در گوش هر غمگین بشیر
&&
خیز ای مدبر ره اقبال گیر
\\
ای درین حبس و درین گند و شپش
&&
هین که تا کس نشنود رستی خمش
\\
چون کنی خامش کنون ای یار من
&&
کز بن هر مو بر آمد طبل‌زن
\\
آن‌چنان کر شد عدو رشک‌خو
&&
گوید این چندین دهل را بانگ کو
\\
می‌زند بر روش ریحان که طریست
&&
او ز کوری گوید این آسیب چیست
\\
می‌شکنجد حور دستش می‌کشد
&&
کور حیران کز چه دردم می‌کند
\\
این کشاکش چیست بر دست و تنم
&&
خفته‌ام بگذار تا خوابی کنم
\\
آنک در خوابش همی‌جویی ویست
&&
چشم بگشا کان مه نیکو پیست
\\
زان بلاها بر عزیزان بیش بود
&&
کان تجمش یار با خوبان فزود
\\
لاغ با خوبان کند بر هر رهی
&&
نیز کوران را بشوراند گهی
\\
خویش را یک‌دم برین کوران دهد
&&
تا غریو از کوی کوران بر جهد
\\
\end{longtable}
\end{center}
