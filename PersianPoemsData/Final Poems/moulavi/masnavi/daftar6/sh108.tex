\begin{center}
\section*{بخش ۱۰۸ - حکایت صدر جهان بخارا کی هر سایلی کی به زبان بخواستی از صدقهٔ عام بی‌دریغ او محروم شدی و آن دانشمند درویش به فراموشی و فرط حرص و تعجیل به زبان بخواست در موکب صدر جهان از وی رو بگردانید و او هر روز حیلهٔ نو ساختی و خود را گاه زن کردی زیر چادر وگاه نابینا کردی و چشم و  روی خود بسته به فراستش بشناختی  الی آخره}
\label{sec:sh108}
\addcontentsline{toc}{section}{\nameref{sec:sh108}}
\begin{longtable}{l p{0.5cm} r}
در بخارا خوی آن خواجیم اجل
&&
بود با خواهندگان حسن عمل
\\
داد بسیار و عطای بی‌شمار
&&
تا به شب بودی ز جودش زر نثار
\\
زر به کاغذپاره‌ها پیچیده بود
&&
تا وجودش بود می‌افشاند جود
\\
هم‌چو خورشید و چو ماه پاک‌باز
&&
آنچ گیرند از ضیا بدهند باز
\\
خاک را زربخش کی بود آفتاب
&&
زر ازو در کان و گنج اندر خراب
\\
هر صباحی یک گره را راتبه
&&
تا نماند امتی زو خایبه
\\
مبتلایان را بدی روزی عطا
&&
روز دیگر بیوگان را آن سخا
\\
روز دیگر بر علویان مقل
&&
با فقیهان فقیر مشتغل
\\
روز دیگر بر تهی‌دستان عام
&&
روز دیگر بر گرفتاران وام
\\
شرط او آن بود که کس با زبان
&&
زر نخواهد هیچ نگشاید لبان
\\
لیک خامش بر حوالی رهش
&&
ایستاده مفلسان دیواروش
\\
هر که کردی ناگهان با لب سؤال
&&
زو نبردی زین گنه یک حبه مال
\\
من صمت منکم نجا بد یاسه‌اش
&&
خامشان را بود کیسه و کاسه‌اش
\\
نادرا روزی یکی پیری بگفت
&&
ده زکاتم که منم با جوع جفت
\\
منع کرد از پیر و پیرش جد گرفت
&&
مانده خلق از جد پیر اندر شگفت
\\
گفت بس بی‌شرم پیری ای پدر
&&
پیر گفت از من توی بی‌شرم‌تر
\\
کین جهان خوردی و خواهی تو ز طمع
&&
کان جهان با این جهان گیری به جمع
\\
خنده‌اش آمد مال داد آن پیر را
&&
پیر تنها برد آن توفیر را
\\
غیر آن پیر ایچ خواهنده ازو
&&
نیم حبه زر ندید و نه تسو
\\
نوبت روز فقیهان ناگهان
&&
یک فقیه از حرص آمد در فغان
\\
کرد زاری‌ها بسی چاره نبود
&&
گفت هر نوعی نبودش هیچ سود
\\
روز دیگر با رگو پیچید پا
&&
ناکس اندر صف قوم مبتلا
\\
تخته‌ها بر ساق بست از چپ و راست
&&
تا گمان آید که او اشکسته‌پاست
\\
دیدش و بشناختش چیزی نداد
&&
روز دیگر رو بپوشید از لباد
\\
هم بدانستش ندادش آن عزیز
&&
از گناه و جرم گفتن هیچ چیز
\\
چونک عاجز شد ز صد گونه مکید
&&
چون زنان او چادری بر سر کشید
\\
در میان بیوگان رفت و نشست
&&
سر فرو افکند و پنهان کرد دست
\\
هم شناسیدش ندادش صدقه‌ای
&&
در دلش آمد ز حرمان حرقه‌ای
\\
رفت او پیش کفن‌خواهی پگاه
&&
که بپیچم در نمد نه پیش راه
\\
هیچ مگشا لب نشین و می‌نگر
&&
تا کند صدر جهان اینجا گذر
\\
بوک بیند مرده پندار به ظن
&&
زر در اندازد پی وجه کفن
\\
هر چه بدهد نیم آن بدهم به تو
&&
هم‌چنان کرد آن فقیر صله‌جو
\\
در نمد پیچید و بر راهش نهاد
&&
معبر صدر جهان آنجا فتاد
\\
زر در اندازید بر روی نمد
&&
دست بیرون کرد از تعجیل خود
\\
تا نگیرد آن کفن‌خواه آن صله
&&
تا نهان نکند ازو آن ده‌دله
\\
مرده از زیر نمد بر کرد دست
&&
سر برون آمد پی دستش ز پست
\\
گفت با صدر جهان چون بستدم
&&
ای ببسته بر من ابواب کرم
\\
گفت لیکن تا نمردی ای عنود
&&
از جناب من نبردی هیچ جود
\\
سر موتوا قبل موت این بود
&&
کز پس مردن غنیمت‌ها رسد
\\
غیر مردن هیچ فرهنگی دگر
&&
در نگیرد با خدای ای حیله‌گر
\\
یک عنایت به ز صد گون اجتهاد
&&
جهد را خوفست از صد گون فساد
\\
وآن عنایت هست موقوف ممات
&&
تجربه کردند این ره را ثقات
\\
بلک مرگش بی‌عنایت نیز نیست
&&
بی‌عنایت هان و هان جایی مه‌ایست
\\
آن زمرد باشد این افعی پیر
&&
بی زمرد کی شود افعی ضریر
\\
\end{longtable}
\end{center}
