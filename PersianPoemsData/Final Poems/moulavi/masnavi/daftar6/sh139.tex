\begin{center}
\section*{بخش ۱۳۹ - وصیت کردن آن شخص کی بعد از من او برد مال مرا از سه فرزند من کی کاهل‌ترست}
\label{sec:sh139}
\addcontentsline{toc}{section}{\nameref{sec:sh139}}
\begin{longtable}{l p{0.5cm} r}
آن یکی شخص به وقت مرگ خویش
&&
گفت بود اندر وصیت پیش‌پیش
\\
سه پسر بودش چو سه سرو روان
&&
وقف ایشان کرده او جان و روان
\\
گفت هرچه در کفم کاله و زرست
&&
او برد زین هر سه کو کاهل‌ترست
\\
گفت با قاضی و پس اندرز کرد
&&
بعد از آن جام شراب مرگ خورد
\\
گفته فرزندان به قاضی کای کریم
&&
نگذریم از حکم او ما سه یتیم
\\
ما چو اسمعیل ز ابراهیم خود
&&
سرنپیچیم ارچه قربان می‌کند
\\
گفت قاضی هر یکی با عاقلیش
&&
تا بگوید قصه‌ای از کاهلیش
\\
تا ببینم کاهلی هر یکی
&&
تا بدانم حال هر یک بی‌شکی
\\
عارفان از دو جهان کاهل‌ترند
&&
زانک بی شد یار خرمن می‌برند
\\
کاهلی را کرده‌اند ایشان سند
&&
کار ایشان را چو یزدان می‌کند
\\
کار یزدان را نمی‌بینند عام
&&
می‌نیاسایند از کد صبح و شام
\\
هین ز حد کاهلی گویید باز
&&
تا بدانم حد آن از کشف راز
\\
بی‌گمان که هر زبان پردهٔ دلست
&&
چون بجنبد پرده سرها واصلست
\\
پردهٔ کوچک چو یک شرحه کباب
&&
می‌بپوشد صورت صد آفتاب
\\
گر بیان نطق کاذب نیز هست
&&
لیک بوی از صدق و کذبش مخبرست
\\
آن نسیمی که بیایدت از چمن
&&
هست پیدا از سموم گولخن
\\
بوی صدق و بوی کذب گول‌گیر
&&
هست پیدا در نفس چون مشک و سیر
\\
گر ندانی یار را از ده‌دله
&&
از مشام فاسد خود کن گله
\\
بانگ حیزان و شجاعان دلیر
&&
هست پیدا چون فن روباه و شیر
\\
یا زبان هم‌چون سر دیگست راست
&&
چون بجنبد تو بدانی چه اباست
\\
از بخار آن بداند تیزهش
&&
دیگ شیرینی ز سکباج ترش
\\
دست بر دیگ نوی چون زد فتی
&&
وقت بخریدن بدید اشکسته را
\\
گفت دانم مرد را در حین ز پوز
&&
ور نگوید دانمش اندر سه روز
\\
وآن دگر گفت ار بگوید دانمش
&&
ور نگوید در سخن پیچانمش
\\
گفت اگر این مکر بشنیده بود
&&
لب ببندد در خموشی در رود
\\
\end{longtable}
\end{center}
