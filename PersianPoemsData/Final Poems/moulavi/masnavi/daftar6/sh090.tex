\begin{center}
\section*{بخش ۹۰ - قصهٔ آنک گاو بحری گوهر کاویان از قعر دریا بر آورد شب بر ساحل دریا نهد در درخش و تاب آن می‌چرد بازرگان از کمین برون آید چون گاو از گوهر دورتر رفته باشد بازرگان به لجم و گل تیره گوهر را بپوشاند و بر درخت گریزد الی آخر القصه و التقریب}
\label{sec:sh090}
\addcontentsline{toc}{section}{\nameref{sec:sh090}}
\begin{longtable}{l p{0.5cm} r}
گاو آبی گوهر از بحر آورد
&&
بنهد اندر مرج و گردش می‌چرد
\\
در شعاع نور گوهر گاو آب
&&
می‌چرد از سنبل و سوسن شتاب
\\
زان فکندهٔ گاو آبی عنبرست
&&
که غذااش نرگس و نیلوفرست
\\
هرکه باشد قوت او نور جلال
&&
چون نزاید از لبش سحر حلال
\\
هرکه چون زنبور وحیستش نفل
&&
چون نباشد خانهٔ او پر عسل
\\
می‌چرد در نور گوهر آن بقر
&&
ناگهان گردد ز گوهر دورتر
\\
تاجری بر در نهد لجم سیاه
&&
تا شود تاریک مرج و سبزه‌گاه
\\
پس گریزد مرد تاجر بر درخت
&&
گاوجویان مرد را با شاخ سخت
\\
بیست بار آن گاو تازد گرد مرج
&&
تا کند آن خصم را در شاخ درج
\\
چون ازو نومید گردد گاو نر
&&
آید آنجا که نهاده بد گهر
\\
لجم بیند فوق در شاه‌وار
&&
پس ز طین بگریزد او ابلیس‌وار
\\
کان بلیس از متن طین کور و کرست
&&
گاو کی داند که در گل گوهرست
\\
اهبطوا افکند جان را در حضیض
&&
از نمازش کرد محروم این محیض
\\
ای رفیقان زین مقیل و زان مقال
&&
اتقوا ان الهوی حیض الرجال
\\
اهبطوا افکند جان را در بدن
&&
تا به گل پنهان بود در عدن
\\
تاجرش داند ولیکن گاو نی
&&
اهل دل دانند و هر گل‌کاو نی
\\
هر گلی که اندر دل او گوهریست
&&
گوهرش غماز طین دیگریست
\\
وان گلی کز رش حق نوری نیافت
&&
صحبت گلهای پر در بر نتافت
\\
این سخن پایان ندارد موش ما
&&
هست بر لبهای جو بر گوش ما
\\
\end{longtable}
\end{center}
