\begin{center}
\section*{بخش ۸۰ - حکایت آن سه مسافر مسلمان و ترسا و جهود و آن کی به منزل قوتی یافتند و ترسا و جهود سیر بودند گفتند این قوت را فردا خوریم مسلمان صایم بود گرسنه ماند از آنک مغلوب بود}
\label{sec:sh080}
\addcontentsline{toc}{section}{\nameref{sec:sh080}}
\begin{longtable}{l p{0.5cm} r}
یک حکایت بشنو اینجا ای پسر
&&
تا نگردی ممتحن اندر هنر
\\
آن جهود و مؤمن و ترسا مگر
&&
همرهی کردند با هم در سفر
\\
با دو گمره همره آمد مؤمنی
&&
چون خرد با نفس و با آهرمنی
\\
مرغزی و رازی افتند از سفر
&&
همره و هم‌سفره پیش هم‌دگر
\\
در قفس افتند زاغ و جغد و باز
&&
جفت شد در حبس پاک و بی‌نماز
\\
کرده منزل شب به یک کاروانسرا
&&
اهل شرق و اهل غرب و ما ورا
\\
مانده در کاروانسرا خرد و شگرف
&&
روزها با هم ز سرما و ز برف
\\
چون گشاده شد ره و بگشاد بند
&&
بسکلند و هر یکی جایی روند
\\
چون قفس را بشکند شاه خرد
&&
جمع مرغان هر یکی سویی پرد
\\
پر گشاید پیش ازین بر شوق و یاد
&&
در هوای جنس خود سوی معاد
\\
پر گشاید هر دمی با اشک و آه
&&
لیک پریدن ندارد روی و راه
\\
راه شد هر یک پرد مانند باد
&&
سوی آن کز یاد آن پر می‌گشاد
\\
آن طرف که بود اشک و آه او
&&
چونک فرصت یافت باشد راه او
\\
در تن خود بنگر این اجزای تن
&&
از کجاها گرد آمد در بدن
\\
آبی و خاکی و بادی و آتشی
&&
عرشی و فرشی و رومی و گشی
\\
از امید عود هر یک بسته طرف
&&
اندرین کاروانسرا از بیم برف
\\
برف گوناگون جمود هر جماد
&&
در شتای بعد آن خورشید داد
\\
چون بتابد تف آن خورشید جشم
&&
کوه گردد گاه ریگ و گاه پشم
\\
در گداز آید جمادات گران
&&
چون گداز تن به وقت نقل جان
\\
چون رسیدند این سه همره منزلی
&&
هدیه‌شان آورد حلوا مقبلی
\\
برد حلوا پیش آن هر سه غریب
&&
محسنی از مطبخ انی قریب
\\
نان گرم و صحن حلوای عسل
&&
برد آنک در ثوابش بود امل
\\
الکیاسه والادب لاهل المدر
&&
الضیافه والقری لاهل الوبر
\\
الضیافة للغریب والقری
&&
اودع الرحمن فی اهل القری
\\
کل یوم فی القری ضیف حدیث
&&
ما له غیر الاله من مغیث
\\
کل لیل فی القری وفد جدید
&&
ما لهم ثم سوی الله محید
\\
تخمه بودند آن دو بیگانه ز خور
&&
بود صایم روز آن مؤمن مگر
\\
چون نماز شام آن حلوا رسید
&&
بود مؤمن مانده در جوع شدید
\\
آن دو کس گفتند ما از خور پریم
&&
امشبش بنهیم و فردایش خوریم
\\
صبر گیریم امشب از خور تن زنیم
&&
بهر فردا لوت را پنهان کنیم
\\
گفت مؤمن امشب این خورده شود
&&
صبر را بنهیم تا فردا بود
\\
پس بدو گفتند زین حکمت‌گری
&&
قصد تو آن است تا تنها خوری
\\
گفت ای یاران نه که ما سه تنیم
&&
چون خلاف افتاد تا قسمت کنیم
\\
هرکه خواهد قسم خود بر جان زند
&&
هرکه خواهد قسم خود پنهان کند
\\
آن دو گفتندش ز قسمت در گذر
&&
گوش کن قسام فی‌النار از خبر
\\
گفت قسام آن بود کو خویش را
&&
کرد قسمت بر هوا و بر خدا
\\
ملک حق و جمله قسم اوستی
&&
قسم دیگر را دهی دوگوستی
\\
این اسد غالب شدی هم بر سگان
&&
گر نبودی نوبت آن بدرگان
\\
قصدشان آن کان مسلمان غم خورد
&&
شب برو در بی‌نوایی بگذرد
\\
بود مغلوب او به تسلیم و رضا
&&
گفت سمعا طاعة اصحابنا
\\
پس بخفتند آن شب و برخاستند
&&
بامدادان خویش را آراستند
\\
روی شستند و دهان و هر یکی
&&
داشت اندر ورد راه و مسلکی
\\
یک زمانی هر کسی آورد رو
&&
سوی ورد خویش از حق فضل‌جو
\\
مؤمن و ترسا جهود و گبر و مغ
&&
جمله را رو سوی آن سلطان الغ
\\
بلک سنگ و خاک و کوه و آب را
&&
هست واگشت نهانی با خدا
\\
این سخن پایان ندارد هر سه یار
&&
رو به هم کردند آن دم یاروار
\\
آن یکی گفتا که هر یک خواب خویش
&&
آنچ دید او دوش گو آور به پیش
\\
هرکه خوابش بهتر این را او خورد
&&
قسم هر مفضول را افضل برد
\\
آنک اندر عقل بالاتر رود
&&
خوردن او خوردن جمله بود
\\
فوق آمد جان پر انوار او
&&
باقیان را بس بود تیمار او
\\
عاقلان را چون بقا آمد ابد
&&
پس به معنی این جهان باقی بود
\\
پس جهود آورد آنچ دیده بود
&&
تا کجا شب روح او گردیده بود
\\
گفت در ره موسی‌ام آمد به پیش
&&
گربه بیند دنبه اندر خواب خویش
\\
در پی موسی شدم تا کوه طور
&&
هر سه‌مان گشتیم ناپیدا ز نور
\\
هر سه سایه محو شد زان آفتاب
&&
بعد از آن زان نور شد یک فتح باب
\\
نور دیگر از دل آن نور رست
&&
پس ترقی جست آن ثانیش چست
\\
هم من و هم موسی و هم کوه طور
&&
هر سه گم گشتیم زان اشراق نور
\\
بعد از آن دیدم که که سه شاخ شد
&&
چونک نور حق درو نفاخ شد
\\
وصف هیبت چون تجلی زد برو
&&
می‌سکست از هم همی‌شد سو به سو
\\
آن یکی شاخ که آمد سوی یم
&&
گشت شیرین آب تلخ هم‌چو سم
\\
آن یکی شاخش فرو شد در زمین
&&
چشمهٔ دارو برون آمد معین
\\
که شفای جمله رنجوران شد آب
&&
از همایونی وحی مستطاب
\\
آن یکی شاخ دگر پرید زود
&&
تا جوار کعبه که عرفات بود
\\
باز از آن صعقه چو با خود آمدم
&&
طور بر جا بد نه افزون و نه کم
\\
لیک زیر پای موسی هم‌چو یخ
&&
می‌گدازید او نماندش شاخ و شخ
\\
با زمین هموار شد که از نهیب
&&
گشت بالایش از آن هیبت نشیب
\\
باز با خود آمدم زان انتشار
&&
باز دیدم طور و موسی برقرار
\\
وآن بیابان سر به سر در ذیل کوه
&&
پر خلایق شکل موسی در وجوه
\\
چون عصا و خرقهٔ او خرقه‌شان
&&
جمله سوی طور خوش دامن کشان
\\
جمله کفها در دعا افراخته
&&
نغمهٔ ارنی به هم در ساخته
\\
باز آن غشیان چو از من رفت زود
&&
صورت هر یک دگرگونم نمود
\\
انبیا بودند ایشان اهل ود
&&
اتحاد انبیاام فهم شد
\\
باز املاکی همی دیدم شگرف
&&
صورت ایشان بد از اجرام برف
\\
حلقهٔ دیگر ملایک مستعین
&&
صورت ایشان به جمله آتشین
\\
زین نسق می‌گفت آن شخص جهود
&&
بس جهودی که آخرش محمود بود
\\
هیچ کافر را به خواری منگرید
&&
که مسلمان مردنش باشد امید
\\
چه خبر داری ز ختم عمر او
&&
تا بگردانی ازو یک‌باره رو
\\
بعد از ان ترسا در آمد در کلام
&&
که مسیحم رو نمود اندر منام
\\
من شدم با او به چارم آسمان
&&
مرکز و مثوای خورشید جهان
\\
خود عجب‌های قلاع آسمان
&&
نسبتش نبود به آیات جهان
\\
هر کسی دانند ای فخر البنین
&&
که فزون باشد فن چرخ از زمین
\\
\end{longtable}
\end{center}
