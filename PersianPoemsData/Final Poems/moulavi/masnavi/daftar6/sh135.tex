\begin{center}
\section*{بخش ۱۳۵ - خطاب حق تعالی به عزرائیل علیه‌السلام کی ترا رحم بر کی بیشتر آمد ازین خلایق کی جانشان قبض کردی و جواب دادن عزرائیل حضرت را}
\label{sec:sh135}
\addcontentsline{toc}{section}{\nameref{sec:sh135}}
\begin{longtable}{l p{0.5cm} r}
حق به عزرائیل می‌گفت ای نقیب
&&
بر کی رحم آمد ترا از هر کئیب
\\
گفت بر جمله دلم سوزد به درد
&&
لیک ترسم امر را اهمال کرد
\\
تا بگویم کاشکی یزدان مرا
&&
در عوض قربان کند بهر فتی
\\
گفت بر کی بیشتر رحم آمدت
&&
از کی دل پر سوز و بریان‌تر شدت
\\
گفت روزی کشتیی بر موج تیز
&&
من شکستم ز امر تا شد ریز ریز
\\
پس بگفتی قبض کن جان همه
&&
جز زنی و غیر طفلی زان رمه
\\
هر دو بر یک تخته‌ای در ماندند
&&
تخته را آن موج‌ها می‌راندند
\\
باز گفتی جان مادر قبض کن
&&
طفل را بگذار تنها ز امر کن
\\
چون ز مادر بسکلیدم طفل را
&&
خود تو می‌دانی چه تلخ آمد مرا
\\
بس بدیدم دود ماتم‌های زفت
&&
تلخی آن طفل از فکرم نرفت
\\
گفت حق آن طفل را از فضل خویش
&&
موج را گفتم فکن در بیشه‌ایش
\\
بیشه‌ای پر سوسن و ریحان و گل
&&
پر درخت میوه‌دار خوش‌اکل
\\
چشمه‌های آب شیرین زلال
&&
پروریدم طفل را با صد دلال
\\
صد هزاران مرغ مطرب خوش‌صدا
&&
اندر آن روضه فکنده صد نوا
\\
پسترش کردم ز برگ نسترن
&&
کرده او را آمن از صدمهٔ فتن
\\
گفته من خورشید را کو را مگز
&&
باد را گفته برو آهسته وز
\\
ابر را گفته برو باران مریز
&&
برق را گفته برو مگرای تیز
\\
زین چمن ای دی مبران اعتدال
&&
پنجه ای بهمن برین روضه ممال
\\
\end{longtable}
\end{center}
