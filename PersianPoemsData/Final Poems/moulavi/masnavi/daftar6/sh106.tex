\begin{center}
\section*{بخش ۱۰۶ - رفتن پسران سلطان به حکم آنک الانسان حریص علی ما منع  ما بندگی خویش نمودیم ولیکن  خوی بد تو بنده ندانست خریدن  به سوی آن قلعهٔ ممنوع عنه آن همه وصیت‌ها و اندرزهای پدر را زیر پا  نهادند تا در چاه بلا افتادند و می‌گفتند ایشان را نفوس لوامه الم یاتکم  نذیر ایشان می‌گفتند گریان و پشیمان لوکنا نسمع او نعقل ماکنا فی اصحاب السعیر}
\label{sec:sh106}
\addcontentsline{toc}{section}{\nameref{sec:sh106}}
\begin{longtable}{l p{0.5cm} r}
این سخن پایان ندارد آن فریق
&&
بر گرفتند از پی آن دز طریق
\\
بر درخت گندم منهی زدند
&&
از طویلهٔ مخلصان بیرون شدند
\\
چون شدند از منع و نهیش گرم‌تر
&&
سوی آن قلعه بر آوردند سر
\\
بر ستیز قول شاه مجتبی
&&
تا به قلعهٔ صبرسوز هش‌ربا
\\
آمدند از رغم عقل پندتوز
&&
در شب تاریک بر گشته ز روز
\\
اندر آن قلعهٔ خوش ذات الصور
&&
پنج در در بحر و پنجی سوی بر
\\
پنج از آن چون حس به سوی رنگ و بو
&&
پنج از آن چون حس باطن رازجو
\\
زان هزاران صورت و نقش و نگار
&&
می‌شدند از سو به سو خوش بی‌قرار
\\
زین قدح‌های صور کم‌باش مست
&&
تا نگردی بت‌تراش و بت‌پرست
\\
از قدح‌های صور بگذر مه‌ایست
&&
باده در جامست لیک از جام نیست
\\
سوی باده‌بخش بگشا پهن فم
&&
چون رسد باده نیاید جام کم
\\
آدما معنی دلبندم بجوی
&&
ترک قشر و صورت گندم بگوی
\\
چونک ریگی آرد شد بهر خلیل
&&
دانک معزولست گندم ای نبیل
\\
صورت از بی‌صورت آید در وجود
&&
هم‌چنانک از آتشی زادست دود
\\
کمترین عیب مصور در خصال
&&
چون پیاپی بینیش آید ملال
\\
حیرت محض آردت بی‌صورتی
&&
زاده صد گون آلت از بی‌آلتی
\\
بی ز دستی دست‌ها بافد همی
&&
جان جان سازد مصور آدمی
\\
آنچنان که اندر دل از هجر و وصال
&&
می‌شود بافیده گوناگون خیال
\\
هیچ ماند این مؤثر با اثر
&&
هیچ ماند بانگ و نوحه با ضرر
\\
نوحه را صورت ضرر بی‌صورتست
&&
دست خایند از ضرر کش نیست دست
\\
این مثل نالایقست ای مستدل
&&
حیلهٔ تفهیم را جهد المقل
\\
صنع بی‌صورت بکارد صورتی
&&
تن بروید با حواس و آلتی
\\
تا چه صورت باشد آن بر وفق خود
&&
اندر آرد جسم را در نیک و بد
\\
صورت نعمت بود شاکر شود
&&
صورت مهلت بود صابر شود
\\
صورت رحمی بود بالان شود
&&
صورت زخمی بود نالان شود
\\
صورت شهری بود گیرد سفر
&&
صورت تیری بود گیرد سپر
\\
صورت خوبان بود عشرت کند
&&
صورت غیبی بود خلوت کند
\\
صورت محتاجی آرد سوی کسب
&&
صورت بازو وری آرد به غصب
\\
این ز حد و اندازه‌ها باشد برون
&&
داعی فعل از خیال گونه‌گون
\\
بی‌نهایت کیش‌ها و پیشه‌ها
&&
جمله ظل صورت اندیشه‌ها
\\
بر لب بام ایستاده قوم خوش
&&
هر یکی را بر زمین بین سایه‌اش
\\
صورت فکرست بر بام مشید
&&
وآن عمل چون سایه بر ارکان پدید
\\
فعل بر ارکان و فکرت مکتتم
&&
لیک در تاثیر و وصلت دو به هم
\\
آن صور در بزم کز جام خوشیست
&&
فایدهٔ او بی‌خودی و بیهشیست
\\
صورت مرد و زن و لعب و جماع
&&
فایده‌ش بی‌هوشی وقت وقاع
\\
صورت نان و نمک کان نعمتست
&&
فایده‌ش آن قوت بی‌صورتست
\\
در مصاف آن صورت تیغ و سپر
&&
فایده‌ش بی‌صورتی یعنی ظفر
\\
مدرسه و تعلیق و صورت‌های وی
&&
چون به دانش متصل شد گشت طی
\\
این صور چون بندهٔ بی‌صورتند
&&
پس چرا در نفی صاحب‌نعمتند
\\
این صور دارد ز بی‌صورت وجود
&&
چیست پس بر موجد خویشش جحود
\\
خود ازو یابد ظهور انکار او
&&
نیست غیر عکس خود این کار او
\\
صورت دیوار و سقف هر مکان
&&
سایهٔ اندیشهٔ معمار دان
\\
گرچه خود اندر محل افتکار
&&
نیست سنگ و چوب و خشتی آشکار
\\
فاعل مطلق یقین بی‌صورتست
&&
صورت اندر دست او چون آلتست
\\
گه گه آن بی‌صورت از کتم عدم
&&
مر صور را رو نماید از کرم
\\
تا مدد گیرد ازو هر صورتی
&&
از کمال و از جمال و قدرتی
\\
باز بی‌صورت چو پنهان کرد رو
&&
آمدند از بهر کد در رنگ و بو
\\
صورتی از صورت دیگر کمال
&&
گر بجوید باشد آن عین ضلال
\\
پس چه عرضه می‌کنی ای بی‌گهر
&&
احتیاج خود به محتاجی دگر
\\
چون صور بنده‌ست بر یزدان مگو
&&
ظن مبر صورت به تشبیهش مجو
\\
در تضرع جوی و در افنای خویش
&&
کز تفکر جز صور ناید به پیش
\\
ور ز غیر صورتت نبود فره
&&
صورتی کان بی‌تو زاید در تو به
\\
صورت شهری که آنجا می‌روی
&&
ذوق بی‌صورت کشیدت ای روی
\\
پس به معنی می‌روی تا لامکان
&&
که خوشی غیر مکانست و زمان
\\
صورت یاری که سوی او شوی
&&
از برای مونسی‌اش می‌روی
\\
پس بمعنی سوی بی‌صورت شدی
&&
گرچه زان مقصود غافل آمدی
\\
پس حقیقت حق بود معبود کل
&&
کز پی ذوقست سیران سبل
\\
لیک بعضی رو سوی دم کرده‌اند
&&
گرچه سر اصلست سر گم کرده‌اند
\\
لیک آن سر پیش این ضالان گم
&&
می‌دهد داد سری از راه دم
\\
آن ز سر می‌یابد آن داد این ز دم
&&
قوم دیگر پا و سر کردند گم
\\
چونک گم شد جمله جمله یافتند
&&
از کم آمد سوی کل بشتافتند
\\
\end{longtable}
\end{center}
