\begin{center}
\section*{بخش ۸۸ - لابه کردن موش مر چغز را کی بهانه میندیش و در نسیه مینداز انجاح این حاجت مرا کی فی التاخیر آفات و الصوفی ابن الوقت و ابن دست از دامن پدر باز ندارد و اب مشفق صوفی کی وقتست او را بنگرش به فردا محتاج نگرداند چندانش مستغرق دارد در گلزار سریع الحسابی خویش نه چون عوام منتظر مستقبل نباشد نهری باشد نه دهری کی لا صباح عند الله و لا مساء ماضی و مستقبل و ازل و ابد آنجا نباشد آدم سابق و دجال مسبوق نباشد کی این رسوم در خطهٔ عقل جز وی است و روح حیوانی در عالم لا مکان و لا زمان این رسوم نباشد پس او ابن وقتیست کی لا یفهم منه الا نفی تفرقة الا زمنة چنانک از الله واحد فهم شود نفی دوی نی حقیقت واحدی}
\label{sec:sh088}
\addcontentsline{toc}{section}{\nameref{sec:sh088}}
\begin{longtable}{l p{0.5cm} r}
صوفیی را گفت خواجهٔ سیم‌پاش
&&
ای قدمهای ترا جانم فراش
\\
یک درم خواهی تو امروز ای شهم
&&
یا که فردا چاشتگاهی سه درم
\\
گفت دی نیم درم راضی‌ترم
&&
زانک امروز این و فردا صد درم
\\
سیلی نقد از عطاء نسیه به
&&
نک قفا پیشت کشیدم نقد ده
\\
خاصه آن سیلی که از دست توست
&&
که قفا و سیلیش مست توست
\\
هین بیا ای جان جان و صد جهان
&&
خوش غنیمت دار نقد این زمان
\\
در مدزد آن روی مه از شب روان
&&
سرمکش زین جوی ای آب روان
\\
تا لب جو خندد از آب معین
&&
لب لب جو سر برآرد یاسمین
\\
چون ببینی بر لب جو سبزه مست
&&
پس بدان از دور که آنجا آب هست
\\
گفت سیماهم وجوه کردگار
&&
که بود غماز باران سبزه‌زار
\\
گر ببارد شب نبیند هیچ کس
&&
که بود در خواب هر نفس و نفس
\\
تازگی هر گلستان جمیل
&&
هست بر باران پنهانی دلیل
\\
ای اخی من خاکیم تو آبیی
&&
لیک شاه رحمت و وهابیی
\\
آن‌چنان کن از عطا و از قسم
&&
که گه و بی‌گه به خدمت می‌رسم
\\
بر لب جو من به جان می‌خوانمت
&&
می‌نبینم از اجابت مرحمت
\\
آمدن در آب بر من بسته شد
&&
زانک ترکیبم ز خاکی رسته شد
\\
یا رسولی یا نشانی کن مدد
&&
تا ترا از بانگ من آگه کند
\\
بحث کردند اندرین کار آن دو یار
&&
آخر آن بحث آن آمد قرار
\\
که به دست آرند یک رشتهٔ دراز
&&
تا ز جذب رشته گردد کشف راز
\\
یک سری بر پای این بندهٔ دوتو
&&
بست باید دیگرش بر پای تو
\\
تا به هم آییم زین فن ما دو تن
&&
اندر آمیزیم چون جان با بدن
\\
هست تن چون ریسمان بر پای جان
&&
می‌کشاند بر زمینش ز آسمان
\\
چغز جان در آب خواب بیهشی
&&
رسته از موش تن آید در خوشی
\\
موش تن زان ریسمان بازش کشد
&&
چند تلخی زین کشش جان می‌چشد
\\
گر نبودی جذب موش گنده‌مغز
&&
عیش‌ها کردی درون آب چغز
\\
باقیش چون روز برخیزی ز خواب
&&
بشنوی از نوربخش آفتاب
\\
یک سر رشته گره بر پای من
&&
زان سر دیگر تو پا بر عقده زن
\\
تا توانم من درین خشکی کشید
&&
مر ترا نک شد سر رشته پدید
\\
تلخ آمد بر دل چغز این حدیث
&&
که مرا در عقده آرد این خبیث
\\
هر کراهت در دل مرد بهی
&&
چون در آید از فنی نبود تهی
\\
وصف حق دان آن فراست را نه وهم
&&
نور دل از لوح کل کردست فهم
\\
امتناع پیل از سیران ببیت
&&
با جد آن پیلبان و بانگ هیت
\\
جانب کعبه نرفتی پای پیل
&&
با همه لت نه کثیر و نه قلیل
\\
گفتیی خود خشک شد پاهای او
&&
یا بمرد آن جان صول‌افزای او
\\
چونک کردندی سرش سوی یمن
&&
پیل نر صد اسپه گشتی گام‌زن
\\
حس پیل از زخم غیب آگاه بود
&&
چون بود حس ولی با ورود
\\
نه که یعقوب نبی آن پاک‌خو
&&
بهر یوسف با همه اخوان او
\\
از پدر چون خواستندش دادران
&&
تا برندش سوی صحرا یک زمان
\\
جمله گفتندش میندیش از ضرر
&&
یک دو روزش مهلتی ده ای پدر
\\
تا به هم در مرجها بازی کنیم
&&
ما درین دعوت امین و محسنیم
\\
گفت این دانم که نقلش از برم
&&
می‌فروزد در دلم درد و سقم
\\
این دلم هرگز نمی‌گوید دروغ
&&
که ز نور عرش دارد دل فروغ
\\
آن دلیل قاطعی بد بر فساد
&&
وز قضا آن را نکرد او اعتداد
\\
در گذشت از وی نشانی آن‌چنان
&&
که قضا در فلسفه بود آن زمان
\\
این عجب نبود که کور افتد به چاه
&&
بوالعجب افتادن بینای راه
\\
این قضا را گونه گون تصریفهاست
&&
چشم‌بندش یفعل‌الله ما یشاست
\\
هم بداند هم نداند دل فنش
&&
موم گردد بهر آن مهر آهنش
\\
گوییی دل گویدی که میل او
&&
چون درین شد هرچه افتد باش گو
\\
خویش را زین هم مغفل می‌کند
&&
در عقالش جان معقل می‌کند
\\
گر شود مات اندرین آن بوالعلا
&&
آن نباشد مات باشد ابتلا
\\
یک بلا از صد بلااش وا خرد
&&
یک هبوطش بر معارجها برد
\\
خام شوخی که رهانیدش مدام
&&
از خمار صد هزاران زشت خام
\\
عاقبت او پخته و استاد شد
&&
جست از رق جهان و آزاد شد
\\
از شراب لایزالی گشت مست
&&
شد ممیز از خلایق باز رست
\\
ز اعتقاد سست پر تقلیدشان
&&
وز خیال دیدهٔ بی‌دیدشان
\\
ای عجب چه فن زند ادراکشان
&&
پیش جزر و مد بحر بی‌نشان
\\
زان بیابان این عمارت‌ها رسید
&&
ملک و شاهی و وزارتها رسید
\\
زان بیابان عدم مشتاق شوق
&&
می‌رسند اندر شهادت جوق جوق
\\
کاروان بر کاروان زین بادیه
&&
می‌رسد در هر مسا و غادیه
\\
آید و گیرد وثاق ما گرو
&&
که رسیدم نوبت ما شد تو رو
\\
چون پسر چشم خرد را بر گشاد
&&
زود بابا رخت بر گردون نهاد
\\
جادهٔ شاهست آن زین سو روان
&&
وآن از آن سو صادران و واردان
\\
نیک بنگر ما نشسته می‌رویم
&&
می‌نبینی قاصد جای نویم
\\
بهر حالی می‌نگیری راس مال
&&
بلک از بهر غرض‌ها در مل
\\
پس مسافر این بود ای ره‌پرست
&&
که مسیر و روش در مستقبلست
\\
هم‌چنانک از پردهٔ دل بی‌کلال
&&
دم به دم در می‌رسد خیل خیال
\\
گر نه تصویرات از یک مغرس‌اند
&&
در پی هم سوی دل چون می‌رسند
\\
جوق جوق اسپاه تصویرات ما
&&
سوی چشمهٔ دل شتابان از ظما
\\
جره‌ها پر می‌کنند و می‌روند
&&
دایما پیدا و پنهان می‌شوند
\\
فکرها را اختران چرخ دان
&&
دایر اندر چرخ دیگر آسمان
\\
سعد دیدی شکر کن ایثار کن
&&
نحس دیدی صدقه و استغفار کن
\\
ما کییم این را بیا ای شاه من
&&
طالعم مقبل کن و چرخی بزن
\\
روح را تابان کن از انوار ماه
&&
که ز آسیب ذنب جان شد سیاه
\\
از خیال و وهم و ظن بازش رهان
&&
از چه و جور رسن بازش رهان
\\
تا ز دلداری خوب تو دلی
&&
پر بر آرد بر پرد ز آب و گلی
\\
ای عزیز مصر و در پیمان درست
&&
یوسف مظلوم در زندان تست
\\
در خلاص او یکی خوابی ببین
&&
زود که الله یحب المحسنین
\\
هفت گاو لاغری پر گزند
&&
هفت گاو فربهش را می‌خورند
\\
هفت خوشهٔ خشک زشت ناپسند
&&
سنبلات تازه‌اش را می‌چرند
\\
قحط از مصرش بر آمد ای عزیز
&&
هین مباش ای شاه این را مستجیز
\\
یوسفم در حبس تو ای شه نشان
&&
هین ز دستان زنانم وا رهان
\\
از سوی عرشی که بودم مربط او
&&
شهوت مادر فکندم که اهبطوا
\\
پس فتادم زان کمال مستتم
&&
از فن زالی به زندان رحم
\\
روح را از عرش آرد در حطیم
&&
لاجرم کید زنان باشد عظیم
\\
اول و آخر هبوط من ز زن
&&
چونک بودم روح و چون گشتم بدن
\\
بشنو این زاری یوسف در عثار
&&
یا بر آن یعقوب بی‌دل رحم آر
\\
ناله از اخوان کنم یا از زنان
&&
که فکندندم چو آدم از جنان
\\
زان مثال برگ دی پژمرده‌ام
&&
کز بهشت وصل گندم خورده‌ام
\\
چون بدیدم لطف و اکرام ترا
&&
وآن سلام سلم و پیغام ترا
\\
من سپند از چشم بد کردم پدید
&&
در سپندم نیز چشم بد رسید
\\
دافع هر چشم بد از پیش و پس
&&
چشم‌های پر خمار تست و بس
\\
چشم بد را چشم نیکویت شها
&&
مات و مستاصل کند نعم الدوا
\\
بل ز چشمت کیمیاها می‌رسد
&&
چشم بد را چشم نیکو می‌کند
\\
چشم شه بر چشم باز دل زدست
&&
چشم بازش سخت با همت شدست
\\
تا ز بس همت که یابید از نظر
&&
می‌نگیرد باز شه جز شیر نر
\\
شیر چه کان شاه‌باز معنوی
&&
هم شکار تست و هم صیدش توی
\\
شد صفیر باز جان در مرج دین
&&
نعره‌های لا احب الافلین
\\
باز دل را که پی تو می‌پرید
&&
از عطای بی‌حدت چشمی رسید
\\
یافت بینی بوی و گوش از تو سماع
&&
هر حسی را قسمتی آمد مشاع
\\
هر حسی را چون دهی ره سوی غیب
&&
نبود آن حس را فتور مرگ و شیب
\\
مالک الملکی به حس چیزی دهی
&&
تا که بر حس‌ها کند آن حس شهی
\\
\end{longtable}
\end{center}
