\begin{center}
\section*{بخش ۱۸ - استدعاء امیر ترک مخمور مطرب را بوقت صبوح و تفسیر این حدیث کی ان لله تعالی شرابا اعده لاولیائه اذا شربوا سکروا و اذا سکروا طابوا الی آخر الحدیث می در خم اسرار بدان می‌جوشد تا هر که مجردست از آن می نوشد قال الله تعالی ان الابرار یشربون این می که تو می‌خوری حرامست  ما می نخوریم جز حلالی «جهد کن تا ز نیست هست شوی  وز شراب خدای مست شوی»}
\label{sec:sh018}
\addcontentsline{toc}{section}{\nameref{sec:sh018}}
\begin{longtable}{l p{0.5cm} r}
اعجمی ترکی سحر آگاه شد
&&
وز خمار خمر مطرب‌خواه شد
\\
مطرب جان مونس مستان بود
&&
نقل و قوت و قوت مست آن بود
\\
مطرب ایشان را سوی مستی کشید
&&
باز مستی از دم مطرب چشید
\\
آن شراب حق بدان مطرب برد
&&
وین شراب تن ازین مطرب چرد
\\
هر دو گر یک نام دارد در سخن
&&
لیک شتان این حسن تا آن حسن
\\
اشتباهی هست لفظی در بیان
&&
لیک خود کو آسمان تا ریسمان
\\
اشتراک لفظ دایم ره‌زنست
&&
اشتراک گبر و مؤمن در تنست
\\
جسمها چون کوزه‌های بسته‌سر
&&
تا که در هر کوزه چه بود آن نگر
\\
کوزهٔ آن تن پر از آب حیات
&&
کوزهٔ این تن پر از زهر ممات
\\
گر به مظروفش نظر داری شهی
&&
ور به ظرفش بنگری تو گم‌رهی
\\
لفظ را مانندهٔ این جسم دان
&&
معنیش را در درون مانند جان
\\
دیدهٔ تن دایما تن‌بین بود
&&
دیدهٔ جان جان پر فن بین بود
\\
پس ز نقش لفظهای مثنوی
&&
صورتی ضالست و هادی معنوی
\\
در نبی فرمود کین قرآن ز دل
&&
هادی بعضی و بعضی را مضل
\\
الله الله چونک عارف گفت می
&&
پیش عارف کی بود معدوم شی
\\
فهم تو چون بادهٔ شیطان بود
&&
کی ترا وهم می رحمان بود
\\
این دو انبازند مطرب با شراب
&&
این بدان و آن بدین آرد شتاب
\\
پر خماران از دم مطرب چرند
&&
مطربانشان سوی میخانه برند
\\
آن سر میدان و این پایان اوست
&&
دل شده چون گوی در چوگان اوست
\\
در سر آنچ هست گوش آنجا رود
&&
در سر ار صفراست آن سودا شود
\\
بعد از آن این دو به بیهوشی روند
&&
والد و مولود آن‌جا یک شوند
\\
چونک کردند آشتی شادی و درد
&&
مطربان را ترک ما بیدار کرد
\\
مطرب آغازید بیتی خوابناک
&&
که انلنی الکاس یا من لا اراک
\\
انت وجهی لا عجب ان لا اراه
&&
غایة القرب حجاب الاشتباه
\\
انت عقلی لا عجب ان لم ارک
&&
من وفور الالتباس المشتبک
\\
جئت اقرب انت من حبل الورید
&&
کم اقل یا یا نداء للبعید
\\
بل اغالطهم انادی فی القفار
&&
کی اکتم من معیمؤمناغار
\\
\end{longtable}
\end{center}
