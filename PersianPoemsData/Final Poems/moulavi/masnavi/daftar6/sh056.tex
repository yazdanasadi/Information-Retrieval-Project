\begin{center}
\section*{بخش ۵۶ - مضاحک گفتن درزی و ترک را از قوت خنده بسته شدن دو چشم تنگ او و فرصت یافتن درزی}
\label{sec:sh056}
\addcontentsline{toc}{section}{\nameref{sec:sh056}}
\begin{longtable}{l p{0.5cm} r}
ترک خندیدن گرفت از داستان
&&
چشم تنگش گشت بسته آن زمان
\\
پاره‌ای دزدید و کردش زیر ران
&&
از جز حق از همه احیا نهان
\\
حق همی‌دید آن ولی ستارخوست
&&
لیک چون از حد بری غماز اوست
\\
ترک را از لذت افسانه‌اش
&&
رفت از دل دعوی پیشانه‌اش
\\
اطلس چه دعوی چه رهن چه
&&
ترک سرمستست در لاغ اچی
\\
لابه کردش ترک کز بهر خدا
&&
لاغ می‌گو که مرا شد مغتذا
\\
گفت لاغی خندمینی آن دغا
&&
که فتاد از قهقهه او بر قفا
\\
پاره‌ای اطلس سبک بر نیفه زد
&&
ترک غافل خوش مضاحک می‌مزد
\\
هم‌چنین بار سوم ترک خطا
&&
گفت لاغی گوی از بهر خدا
\\
گفت لاغی خندمین‌تر زان دو بار
&&
کرد او این ترک را کلی شکار
\\
چشم بسته عقل جسته مولهه
&&
مست ترک مدعی از قهقهه
\\
پس سوم بار از قبا دزدید شاخ
&&
که ز خنده‌ش یافت میدان فراخ
\\
چون چهارم بار آن ترک خطا
&&
لاغ از آن استا همی‌کرد اقتضا
\\
رحم آمد بر وی آن استاد را
&&
کرد در باقی فن و بیداد را
\\
گفت مولع گشت این مفتون درین
&&
بی‌خبر کین چه خسارست و غبین
\\
بوسه‌افشان کرد بر استاد او
&&
که بمن بهر خدا افسانه گو
\\
ای فسانه گشته و محو از وجود
&&
چند افسانه بخواهی آزمود
\\
خندمین‌تر از تو هیچ افسانه نیست
&&
بر لب گور خراب خویش ایست
\\
ای فرو رفته به گور جهل و شک
&&
چند جویی لاغ و دستان فلک
\\
تا بکی نوشی تو عشوهٔ این جهان
&&
که نه عقلت ماند بر قانون نه جان
\\
لاغ این چرخ ندیم کرد و مرد
&&
آب روی صد هزاران چون تو برد
\\
می‌درد می‌دوزد این درزی عام
&&
جامهٔ صدسالگان طفل خام
\\
لاغ او گر باغها را داد داد
&&
چون دی آمد داده را بر باد داد
\\
پیره‌طفلان شسته پیشش بهر کد
&&
تا به سعد و نحس او لاغی کند
\\
\end{longtable}
\end{center}
