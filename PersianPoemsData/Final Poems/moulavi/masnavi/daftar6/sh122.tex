\begin{center}
\section*{بخش ۱۲۲ - بیان این خبر کی الکذب ریبة والصدق طمانینة}
\label{sec:sh122}
\addcontentsline{toc}{section}{\nameref{sec:sh122}}
\begin{longtable}{l p{0.5cm} r}
قصهٔ آن خواب و گنج زر بگفت
&&
پس ز صدق او دل آن کس شکفت
\\
بوی صدقش آمد از سوگند او
&&
سوز او پیدا شد و اسپند او
\\
دل بیارامد به گفتار صواب
&&
آنچنان که تشنه آرامد به آب
\\
جز دل محجوب کو را علتیست
&&
از نبیش تا غبی تمییز نیست
\\
ورنه آن پیغام کز موضع بود
&&
بر زند بر مه شکافیده شود
\\
مه شکافد وان دل محجوب نی
&&
زانک مردودست او محبوب نی
\\
چشمه شد چشم عسس ز اشک مبل
&&
نی ز گفت خشک بل از بوی دل
\\
یک سخن از دوزخ آید سوی لب
&&
یک سخن از شهر جان در کوی لب
\\
بحر جان‌افزا و بحر پر حرج
&&
در میان هر دو بحر این لب مرج
\\
چون یپنلو در میان شهرها
&&
از نواحی آید آن‌جا بهرها
\\
کالهٔ معیوب قلب کیسه‌بر
&&
کالهٔ پر سود مستشرف چو در
\\
زین یپنلو هر که بازرگان‌ترست
&&
بر سره و بر قلب‌ها دیده‌ورست
\\
شد یپنلو مر ورا دار الرباح
&&
وآن گر را از عمی دار الجناح
\\
هر یکی ز اجزای عالم یک به یک
&&
بر غبی بندست و بر استاد فک
\\
بر یکی قندست و بر دیگر چو زهر
&&
بر یکی لطفست و بر دیگر چو قهر
\\
هر جمادی با نبی افسانه‌گو
&&
کعبه با حاجی گواه و نطق‌خو
\\
بر مصلی مسجد آمد هم گواه
&&
کو همی‌آمد به من از دور راه
\\
با خلیل آتش گل و ریحان و ورد
&&
باز بر نمرودیان مرگست و درد
\\
بارها گفتیم این را ای حسن
&&
می‌نگردم از بیانش سیر من
\\
بارها خوردی تو نان دفع ذبول
&&
این همان نانست چون نبوی ملول
\\
در تو جوعی می‌رسد تو ز اعتلال
&&
که همی‌سوزد ازو تخمه و ملال
\\
هرکه را درد مجاعت نقد شد
&&
نو شدن با جزو جزوش عقد شد
\\
لذت از جوعست نه از نقل نو
&&
با مجاعت از شکر به نان جو
\\
پس ز بی‌جوعیست وز تخمهٔ تمام
&&
آن ملالت نه ز تکرار کلام
\\
چون ز دکان و مکاس و قیل و قال
&&
در فریب مردمت ناید ملال
\\
چون ز غیبت و اکل لحم مردمان
&&
شصت سالت سیریی نامد از آن
\\
عشوه‌ها در صید شلهٔ کفته تو
&&
بی ملولی بارها خوش گفته تو
\\
بار آخر گوییش سوزان و چست
&&
گرم‌تر صد بار از بار نخست
\\
درد داروی کهن را نو کند
&&
درد هر شاخ ملولی خو کند
\\
کیمیای نو کننده دردهاست
&&
کو ملولی آن طرف که درد خاست
\\
هین مزن تو از ملولی آه سرد
&&
درد جو و درد جو و درد درد
\\
خادع دردند درمان‌های ژاژ
&&
ره‌زنند و زرستانان رسم باژ
\\
آب شوری نیست در مان عطش
&&
وقت خوردن گر نماید سرد و خوش
\\
لیک خادع گشته و مانع شد ز جست
&&
ز آب شیرینی کزو صد سبزه رست
\\
هم‌چنین هر زر قلبی مانعست
&&
از شناس زر خوش هرجا که هست
\\
پا و پرت را به تزویری برید
&&
که مراد تو منم گیر ای مرید
\\
گفت دردت چینم او خود درد بود
&&
مات بود ار چه به ظاهر برد بود
\\
رو ز درمان دروغین می‌گریز
&&
تا شود دردت مصیب و مشک‌بیز
\\
گفت نه دزدی تو و نه فاسقی
&&
مرد نیکی لیک گول و احمقی
\\
بر خیال و خواب چندین ره کنی
&&
نیست عقلت را تسوی روشنی
\\
بارها من خواب دیدم مستمر
&&
که به بغدادست گنجی مستتر
\\
در فلان سوی و فلان کویی دفین
&&
بود آن خود نام کوی این حزین
\\
هست در خانهٔ فلانی رو بجو
&&
نام خانه و نام او گفت آن عدو
\\
دیده‌ام خود بارها این خواب من
&&
که به بغدادست گنجی در وطن
\\
هیچ من از جا نرفتم زین خیال
&&
تو به یک خوابی بیایی بی‌ملال
\\
خواب احمق لایق عقل ویست
&&
هم‌چو او بی‌قیمتست و لاشیست
\\
خواب زن کمتر ز خواب مرد دان
&&
از پی نقصان عقل و ضعف جان
\\
خواب ناقص‌عقل و گول آید کساد
&&
پس ز بی‌عقلی چه باشد خواب باد
\\
گفت با خود گنج در خانهٔ منست
&&
پس مرا آن‌جا چه فقر و شیونست
\\
بر سر گنج از گدایی مرده‌ام
&&
زانک اندر غفلت و در پرده‌ام
\\
زین بشارت مست شد دردش نماند
&&
صد هزار الحمد بی لب او بخواند
\\
گفت بد موقوف این لت لوت من
&&
آب حیوان بود در حانوت من
\\
رو که بر لوت شگرفی بر زدم
&&
کوری آن وهم که مفلس بدم
\\
خواه احمق‌دان مرا خواهی فرو
&&
آن من شد هرچه می‌خواهی بگو
\\
من مراد خویش دیدم بی‌گمان
&&
هرچه خواهی گو مرا ای بددهان
\\
تو مرا پر درد گو ای محتشم
&&
پیش تو پر درد و پیش خود خوشم
\\
وای اگر بر عکس بودی این مطار
&&
پیش تو گلزار و پیش خویش راز
\\
\end{longtable}
\end{center}
