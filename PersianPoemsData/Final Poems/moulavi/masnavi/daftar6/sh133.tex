\begin{center}
\section*{بخش ۱۳۳ - متوفی شدن بزرگین از شه‌زادگان و آمدن برادر میانین به جنازهٔ برادر کی آن کوچکین صاحب‌فراش بود از رنجوری و نواختن پادشاه میانین را تا او هم لنگ احسان شد ماند پیش پادشاه صد هزار از غنایم غیبی و غنی بدو رسید از دولت و نظر آن شاه مع تقریر بعضه}
\label{sec:sh133}
\addcontentsline{toc}{section}{\nameref{sec:sh133}}
\begin{longtable}{l p{0.5cm} r}
کوچکین رنجور بود و آن وسط
&&
بر جنازهٔ آن بزرگ آمد فقط
\\
شاه دیدش گفت قاصد کین کیست
&&
که از آن بحرست و این هم ماهیست
\\
پس معرف گفت پور آن پدر
&&
این برادر زان برادر خردتر
\\
شه نوازیدش که هستی یادگار
&&
کرد او را هم بدان پرسش شکار
\\
از نواز شاه آن زار حنیذ
&&
در تن خود غیر جان جانی بدیذ
\\
در دل خود دید عالی غلغله
&&
که نیابد صوفی آن در صد چله
\\
عرصه و دیوار و کوه سنگ‌بافت
&&
پیش او چون نار خندان می‌شکافت
\\
ذره ذره پیش او هم‌چون قباب
&&
دم به دم می‌کرد صدگون فتح باب
\\
باب گه روزن شدی گاه شعاع
&&
خاک گه گندم شدی و گاه صاع
\\
در نظرها چرخ بس کهنه و قدید
&&
پیش چشمش هر دمی خلق جدید
\\
روح زیبا چونک وا رست از جسد
&&
از قضا بی شک چنین چشمش رسد
\\
صد هزاران غیب پیشش شد پدید
&&
آنچ چشم محرمان بیند بدید
\\
آنچ او اندر کتب بر خوانده بود
&&
چشم را در صورت آن بر گشود
\\
از غبار مرکب آن شاه نر
&&
یافت او کحل عزیزی در بصر
\\
برچنین گلزار دامن می‌کشید
&&
جزو جزوش نعره زن هل من مزید
\\
گلشنی کز بقل روید یک دمست
&&
گلشنی کز عقل روید خرمست
\\
گلشنی کز گل دمد گردد تباه
&&
گلشنی کز دل دمد وافر حتاه
\\
علم‌های با مزهٔ دانسته‌مان
&&
زان گلستان یک دو سه گل‌دسته دان
\\
زان زبون این دو سه گل دسته‌ایم
&&
که در گلزار بر خود بسته‌ایم
\\
آن‌چنان مفتاح‌ها هر دم بنان
&&
می‌فتد ای جان دریغا از بنان
\\
ور دمی هم فارغ آرندت ز نان
&&
گرد چارد گردی و عشق زنان
\\
باز استسقات چون شد موج‌زن
&&
ملک شهری بایدت پر نان و زن
\\
مار بودی اژدها گشتی مگر
&&
یک سرت بود این زمانی هفت‌سر
\\
اژدهای هفت‌سر دوزخ بود
&&
حرص تو دانه‌ست و دوزخ فخ بود
\\
دام را بدران بسوزان دانه را
&&
باز کن درهای نو این خانه را
\\
چون تو عاشق نیستی ای نرگدا
&&
هم‌چو کوهی بی‌خبر داری صدا
\\
کوه را گفتار کی باشد ز خود
&&
عکس غیرست آن صدا ای معتمد
\\
گفت تو زان سان که عکس دیگریست
&&
جمله احوالت به جز هم عکس نیست
\\
خشم و ذوقت هر دو عکس دیگران
&&
شادی قواده و خشم عوان
\\
آن عوان را آن ضعیف آخر چه کرد
&&
که دهد او را به کینه زجر و درد
\\
تا بکی عکس خیال لامعه
&&
جهد کن تا گرددت این واقعه
\\
تا که گفتارت ز حال تو بود
&&
سیر تو با پر و بال تو بود
\\
صید گیرد تیر هم با پر غیر
&&
لاجرم بی‌بهره است از لحم طیر
\\
باز صید آرد به خود از کوهسار
&&
لاجرم شاهش خوراند کبک و سار
\\
منطقی کز وحی نبود از هواست
&&
هم‌چو خاکی در هوا و در هباست
\\
گر نماید خواجه را این دم غلط
&&
ز اول والنجم بر خوان چند خط
\\
تا که ما ینطق محمد عن هوی
&&
ان هو الا بوحی احتوی
\\
احمدا چون نیستت از وحی یاس
&&
جسمیان را ده تحری و قیاس
\\
کز ضرورت هست مرداری حلال
&&
که تحری نیست در کعبهٔ وصال
\\
بی‌تحری و اجتهادات هدی
&&
هر که بدعت پیشه گیرد از هوی
\\
هم‌چو عادش بر برد باد و کشد
&&
نه سلیمانست تا تختش کشد
\\
عاد را با دست حمال خذول
&&
هم‌چو بره در کف مردی اکول
\\
هم‌چو فرزندش نهاده بر کنار
&&
می‌برد تا بکشدش قصاب‌وار
\\
عاد را آن باد ز استکبار بود
&&
یار خود پنداشتند اغیار بود
\\
چون بگردانید ناگه پوستین
&&
خردشان بشکست آن بئس القرین
\\
باد را بشکن که بس فتنه‌ست باد
&&
پیش از آن کت بشکند او هم‌چو عاد
\\
هود دادی پند که ای پر کبر خیل
&&
بر کند از دستتان این باد ذیل
\\
لشکر حق است باد و از نفاق
&&
چند روزی با شما کرد اعتناق
\\
او به سر با خالق خود راستست
&&
چون اجل آید بر آرد باد دست
\\
باد را اندر دهن بین ره‌گذر
&&
هر نفس آیان روان در کر و فر
\\
حلق و دندان‌ها ازو آمن بود
&&
حق چو فرماید به دندان در فتد
\\
کوه گردد ذره‌ای باد و ثقیل
&&
درد دندان داردش زار و علیل
\\
این همان بادست که امن می‌گذشت
&&
بود جان کشت و گشت او مرگ کشت
\\
دست آن کس که بکردت دست‌بوس
&&
وقت خشم آن دست می‌گردد دبوس
\\
یا رب و یا رب بر آرد او ز جان
&&
که ببر این باد را ای مستعان
\\
ای دهان غافل بدی زین باد رو
&&
از بن دندان در استغفار شو
\\
چشم سختش اشک‌ها باران کند
&&
منکران را درد الله‌خوان کند
\\
چون دم مردان نپذرفتی ز مرد
&&
وحی حق را هین پذیرا شو ز درد
\\
باد گوید پیکم از شاه بشر
&&
گه خبر خیر آورم گه شوم و شر
\\
ز آنک مامورم امیر خود نیم
&&
من چو تو غافل ز شاه خود کیم
\\
گر سلیمان‌وار بودی حال تو
&&
چون سلیمان گشتمی حمال تو
\\
عاریه‌ستم گشتمی ملک کفت
&&
کردمی بر راز خود من واقفت
\\
لیک چون تو یاغیی من مستعار
&&
می‌کنم خدمت ترا روزی سه چار
\\
پس چو عادت سرنگونی‌ها دهم
&&
ز اسپه تو یاغیانه بر جهم
\\
تا به غیب ایمان تو محکم شود
&&
آن زمان که ایمانت مایهٔ غم شود
\\
آن زمان خود جملگان مؤمن شوند
&&
آن زمان خود سرکشان بر سر دوند
\\
آن زمان زاری کنند و افتقار
&&
هم‌چو دزد و راه‌زن در زیر دار
\\
لیک گر در غیب گردی مستوی
&&
مالک دارین و شحنهٔ خود توی
\\
شحنگی و پادشاهی مقیم
&&
نه دو روزه و مستعارست و سقیم
\\
رستی از بیگار و کار خود کنی
&&
هم تو شاه و هم تو طبل خود زنی
\\
چون گلو تنگ آورد بر ما جهان
&&
خاک خوردی کاشکی حلق و دهان
\\
این دهان خود خاک‌خواری آمدست
&&
لیک خاکی را که آن رنگین شدست
\\
این کباب و این شراب و این شکر
&&
خاک رنگینست و نقشین ای پسر
\\
چونک خوردی و شد آن لحم و پوست
&&
رنگ لحمش داد و این هم خاک کوست
\\
هم ز خاکی بخیه بر گل می‌زند
&&
جمله را هم باز خاکی می‌کند
\\
هندو و قفچاق و رومی و حبش
&&
جمله یک رنگ‌اند اندر گور خوش
\\
تا بدانی کان همه رنگ و نگار
&&
جمله روپوشست و مکر و مستعار
\\
رنگ باقی صبغة الله است و بس
&&
غیر آن بر بسته دان هم‌چون جرس
\\
رنگ صدق و رنگ تقوی و یقین
&&
تا ابد باقی بود بر عابدین
\\
رنگ شک و رنگ کفران و نفاق
&&
تا ابد باقی بود بر جان عاق
\\
چون سیه‌رویی فرعون دغا
&&
رنگ آن باقی و جسم او فنا
\\
برق و فر روی خوب صادقین
&&
تن فنا شد وان به جا تو یومن دین
\\
زشت آن زشتست و خوب آن خوب و بس
&&
دایم آن ضحاک و این اندر عبس
\\
خاک را رنگ و فن و سنگی دهد
&&
طفل‌خویان را بر آن جنگی دهد
\\
از خمیری اشتر وشیری پزند
&&
کودکان از حرص آن کف می‌گزند
\\
شیر و اشتر نان شود اندر دهان
&&
در نگیرد این سخن با کودکان
\\
کودک اندر جهل و پندار و شکیست
&&
شکر باری قوت او اندکیست
\\
طفل را استیزه و صد آفتست
&&
شکر این که بی‌فن و بی‌قوتست
\\
وای ازین پیران طفل ناادیب
&&
گشته از قوت بلای هر رقیب
\\
چون سلاح و جهل جمع آید به هم
&&
گشت فرعونی جهان‌سوز از ستم
\\
شکر کن ای مرد درویش از قصور
&&
که ز فرعونی رهیدی وز کفور
\\
شکر که مظلومی و ظالم نه‌ای
&&
آمن از فرعونی و هر فتنه‌ای
\\
اشکم تی لاف اللهی نزد
&&
که آتشش را نیست از هیزم مدد
\\
اشکم خالی بود زندان دیو
&&
کش غم نان مانعست از مکر و ریو
\\
اشکم پر لوت دان بازار دیو
&&
تاجران دیو را در وی غریو
\\
تاجران ساحر لاشی‌فروش
&&
عقل‌ها را تیره کرده از خروش
\\
خم روان کرده ز سحری چون فرس
&&
کرده کرباسی ز مهتاب و غلس
\\
چون بریشم خاک را برمی‌تنند
&&
خاک در چشم ممیز می‌زنند
\\
چندلی را رنگ عودی می‌دهند
&&
بر کلوخیمان حسودی می‌دهند
\\
پاک آنک خاک را رنگی دهد
&&
هم‌چو کودکمان بر آن جنگی دهد
\\
دامنی پر خاک ما چون طفلکان
&&
در نظرمان خاک هم‌چون زر کان
\\
طفل را با بالغان نبود مجال
&&
طفل را حق کی نشاند با رجال
\\
میوه گر کهنه شود تا هست خام
&&
پخته نبود غوره گویندش به نام
\\
گر شود صدساله آن خام ترش
&&
طفل و غوره‌ست او بر هر تیزهش
\\
گرچه باشد مو و ریش او سپید
&&
هم در آن طفلی خوفست و امید
\\
که رسم یا نارسیده مانده‌ام
&&
ای عجب با من کند کرم آن کرم
\\
با چنین ناقابلی و دوریی
&&
بخشد این غورهٔ مرا انگوریی
\\
نیستم اومیدوار از هیچ سو
&&
وان کرم می‌گویدم لا تیاسوا
\\
دایما خاقان ما کردست طو
&&
گوشمان را می‌کشد لا تقنطوا
\\
گرچه ما زین ناامیدی در گویم
&&
چون صلا زد دست اندازان رویم
\\
دست اندازیم چون اسپان سیس
&&
در دویدن سوی مرعای انیس
\\
گام اندازیم و آن‌جا گام نی
&&
جام پردازیم و آن‌جا جام نی
\\
زانک آن‌جا جمله اشیا جانیست
&&
معنی اندر معنی اندر معنیست
\\
هست صورت سایه معنی آفتاب
&&
نور بی‌سایه بود اندر خراب
\\
چونک آنجا خشت بر خشتی نماند
&&
نور مه را سایهٔ زشتی نماند
\\
خشت اگر زرین بود بر کندنیست
&&
چون بهای خشت وحی و روشنیست
\\
کوه بهر دفع سایه مندکست
&&
پاره گشتن بهر این نور اندکست
\\
بر برون که چو زد نور صمد
&&
پاره شد تا در درونش هم زند
\\
گرسنه چون بر کفش زد قرص نان
&&
وا شکافد از هوس چشم و دهان
\\
صد هزاران پاره گشتن ارزد این
&&
از میان چرخ برخیز ای زمین
\\
تا که نور چرخ گردد سایه‌سوز
&&
شب ز سایهٔ تست ای یاغی روز
\\
این زمین چون گاهوارهٔ طفلکان
&&
بالغان را تنگ می‌دارد مکان
\\
بهر طفلان حق زمین را مهد خواند
&&
شیر در گهواره بر طفلان فشاند
\\
خانه تنگ آمد ازین گهواره‌ها
&&
طفلکان را زود بالغ کن شها
\\
ای گواره خانه را ضیق مدار
&&
تا تواند کرد بالغ انتشار
\\
\end{longtable}
\end{center}
