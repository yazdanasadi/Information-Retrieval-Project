\begin{center}
\section*{بخش ۱۲۹ - در تفسیر این خبر کی مصطفی صلوات‌الله علیه فرمود من کنت مولاه فعلی مولاه تا منافقان طعنه زدند کی بس نبودش کی ما مطیعی و چاکری نمودیم او را چاکری کودکی خلم آلودمان هم می‌فرماید الی آخره}
\label{sec:sh129}
\addcontentsline{toc}{section}{\nameref{sec:sh129}}
\begin{longtable}{l p{0.5cm} r}
زین سبب پیغامبر با اجتهاد
&&
نام خود وان علی مولا نهاد
\\
گفت هر کو را منم مولا و دوست
&&
ابن عم من علی مولای اوست
\\
کیست مولا آنک آزادت کند
&&
بند رقیت ز پایت بر کند
\\
چون به آزادی نبوت هادیست
&&
مؤمنان را ز انبیا آزادیست
\\
ای گروه مؤمنان شادی کنید
&&
هم‌چو سرو و سوسن آزادی کنید
\\
لیک می‌گویید هر دم شکر آب
&&
بی‌زبان چون گلستان خوش‌خضاب
\\
بی‌زبان گویند سرو و سبزه‌زار
&&
شکر آب و شکر عدل نوبهار
\\
حله‌ها پوشیده و دامن‌کشان
&&
مست و رقاص و خوش و عنبرفشان
\\
جزو جزو آبستن از شاه بهار
&&
جسمشان چون درج پر در ثمار
\\
مریمان بی شوی آبست از مسیح
&&
خامشان بی لاف و گفتاری فصیح
\\
ماه ما بی‌نطق خوش بر تافتست
&&
هر زبان نطق از فر ما یافتست
\\
نطق عیسی از فر مریم بود
&&
نطق آدم پرتو آن دم بود
\\
تا زیادت گردد از شکر ای ثقات
&&
پس نبات دیگرست اندر نبات
\\
عکس آن اینجاست ذل من قنع
&&
اندرین طورست عز من طمع
\\
در جوال نفس خود چندین مرو
&&
از خریداران خود غافل مشو
\\
\end{longtable}
\end{center}
