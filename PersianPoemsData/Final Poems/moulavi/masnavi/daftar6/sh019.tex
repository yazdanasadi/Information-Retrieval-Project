\begin{center}
\section*{بخش ۱۹ - در آمدن ضریر در خانهٔ مصطفی علیه‌السلام و گریختن عایشه رضی الله عنها از پیش ضریر و گفتن رسول علیه‌السلام کی چه می‌گریزی او ترا نمی‌بیند و جواب دادن عایشه رضی الله عنها رسول را صلی الله علیه و سلم}
\label{sec:sh019}
\addcontentsline{toc}{section}{\nameref{sec:sh019}}
\begin{longtable}{l p{0.5cm} r}
اندر آمد پیش پیغامبر ضریر
&&
کای نوابخش تنور هر خمیر
\\
ای تو میر آب و من مستسقیم
&&
مستغاث المستغاث ای ساقیم
\\
چون در آمد آن ضریر از در شتاب
&&
عایشه بگریخت بهر احتجاب
\\
زانک واقف بود آن خاتون پاک
&&
از غیوری رسول رشکناک
\\
هر که زیباتر بود رشکش فزون
&&
زانک رشک از ناز خیزد یا بنون
\\
گنده‌پیران شوی را قما دهند
&&
چونک از زشتی و پیری آگهند
\\
چون جمال احمدی در هر دو کون
&&
کی بدست ای فر یزدانیش عون
\\
نازهای هر دو کون او را رسد
&&
غیرت آن خورشید صدتو را رسد
\\
که در افکندم به کیوان گوی را
&&
در کشید ای اختران هم روی را
\\
در شعاع بی‌نظیرم لا شوید
&&
ورنه پیش نور نم رسوا شوید
\\
از کرم من هر شبی غایب شوم
&&
کی روم الا نمایم که روم
\\
تا شما بی من شبی خفاش‌وار
&&
پر زنان پرید گرد این مطار
\\
هم‌چو طاووسان پری عرضه کنید
&&
باز مست و سرکش و معجب شوید
\\
ننگرید آن پای خود را زشت‌ساز
&&
هم‌چو چارق کو بود شمع ایاز
\\
رو نمایم صبح بهر گوشمال
&&
تا نگردید از منی ز اهل شمال
\\
ترک آن کن که درازست آن سخن
&&
نهی کردست از درازی امر کن
\\
\end{longtable}
\end{center}
