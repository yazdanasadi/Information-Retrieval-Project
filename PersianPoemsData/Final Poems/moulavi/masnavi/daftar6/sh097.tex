\begin{center}
\section*{بخش ۹۷ - مثل دوبین هم‌چو آن غریب شهر کاش عمر نام کی از یک دکانش به سبب این به آن دکان دیگر حواله کرد و او فهم نکرد کی همه  دکان یکیست درین معنی کی به عمر نان نفروشند هم اینجا تدارک کنم  من غلط کردم نامم عمر نیست چون بدین دکان توبه و تدارک کنم نان  یابم از همه دکان‌های این شهر و اگر بی‌تدارک هم‌چنین عمر نام باشم  ازین دکان در گذرم محرومم و احولم و این دکان‌ها را از هم جدا دانسته‌ام}
\label{sec:sh097}
\addcontentsline{toc}{section}{\nameref{sec:sh097}}
\begin{longtable}{l p{0.5cm} r}
گر عمر نامی تو اندر شهر کاش
&&
کس بنفروشد به صد دانگت لواش
\\
چون به یک دکان بگفتی عمرم
&&
این عمر را نان فروشید از کرم
\\
او بگوید رو بدان دیگر دکان
&&
زان یکی نان به کزین پنجاه نان
\\
گر نبودی احول او اندر نظر
&&
او بگفتی نیست دکانی دگر
\\
پس ردی اشراق آن نااحولی
&&
بر دل کاشی شدی عمر علی
\\
این ازینجا گوید آن خباز را
&&
این عمر را نان فروش ای نانبا
\\
چون شنید او هم عمر نان در کشید
&&
پس فرستادت به دکان بعید
\\
کین عمر را نان ده ای انباز من
&&
راز یعنی فهم کن ز آواز من
\\
او همت زان سو حواله می‌کند
&&
هین عمر آمد که تا بر نان زند
\\
چون به یک دکان عمر بودی برو
&&
در همه کاشان ز نان محروم شو
\\
ور به یک دکان علی گفتی بگیر
&&
نان ازینجا بی‌حواله و بی‌زحیر
\\
احول دو بین چو بی‌بر شد ز نوش
&&
احول ده بینی ای مادر فروش
\\
اندرین کاشان خاک از احولی
&&
چون عمر می‌گرد چو نبوی علی
\\
هست احول را درین ویرانه دیر
&&
گوشه گوشه نقل نو ای ثم خیر
\\
ور دو چشم حق‌شناس آمد ترا
&&
دوست پر بین عرصهٔ هر دو سرا
\\
وا رهیدی از حوالهٔ جا به جا
&&
اندرین کاشان پر خوف و رجا
\\
اندرین جو غنچه دیدی یا شجر
&&
هم‌چو هر جو تو خیالش ظن مبر
\\
که ترا از عین این عکس نقوش
&&
حق حقیقت گردد و میوه‌فروش
\\
چشم ازین آب از حول حر می‌شود
&&
عکس می‌بیند سد پر می‌شود
\\
پس به معنی باغ باشد این نه آب
&&
پس مشو عریان چو بلقیس از حباب
\\
بار گوناگونست بر پشت خران
&&
هین به یک چون این خران را تو مران
\\
بر یکی خر بار لعل و گوهرست
&&
بر یکی خر بار سنگ و مرمرست
\\
بر همه جوها تو این حکمت مران
&&
اندرین جو ماه بین عکسش مخوان
\\
آب خضرست این نه آب دام و دد
&&
هر چه اندر روی نماید حق بود
\\
زین تگ جو ماه گوید من مهم
&&
من نه عکسم هم‌حدیث و هم‌رهم
\\
اندرین جو آنچ بر بالاست هست
&&
خواه بالا خواه در وی دار دست
\\
از دگر جوها مگیر این جوی را
&&
ماه دان این پرتو مه‌روی را
\\
این سخن پایان ندارد آن غریب
&&
بس گریست از درد خواجه شد کئیب
\\
\end{longtable}
\end{center}
