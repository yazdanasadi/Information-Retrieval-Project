\begin{center}
\section*{بخش ۷۴ - یافتن مرید مراد را و ملاقات او با شیخ نزدیک آن بیشه}
\label{sec:sh074}
\addcontentsline{toc}{section}{\nameref{sec:sh074}}
\begin{longtable}{l p{0.5cm} r}
اندرین بود او که شیخ نامدار
&&
زود پیش افتاد بر شیری سوار
\\
شیر غران هیزمش را می‌کشید
&&
بر سر هیزم نشسته آن سعید
\\
تازیانه‌ش مار نر بود از شرف
&&
مار را بگرفته چون خرزن به کف
\\
تو یقین می‌دان که هر شیخی که هست
&&
هم سواری می‌کند بر شیر مست
\\
گرچه آن محسوس و این محسوس نیست
&&
لیک آن بر چشم جان ملبوس نیست
\\
صد هزاران شیر زیر را نشان
&&
پیش دیدهٔ غیب‌دان هیزم‌کشان
\\
لیک یک یک را خدا محسوس کرد
&&
تا که بیند نیز او که نیست مرد
\\
دیدش از دور و بخندید آن خدیو
&&
گفت آن را مشنو ای مفتون دیو
\\
از ضمیر او بدانست آن جلیل
&&
هم ز نور دل بلی نعم الدلیل
\\
خواند بر وی یک به یک آن ذوفنون
&&
آنچ در ره رفت بر وی تا کنون
\\
بعد از آن در مشکل انکار زن
&&
بر گشاد آن خوش‌سراینده دهن
\\
کان تحمل از هوای نفس نیست
&&
آن خیال نفس تست آنجا مه‌ایست
\\
گرنه صبرم می‌کشیدی بار زن
&&
کی کشیدی شیر نر بیگار من
\\
اشتران بختییم اندر سبق
&&
مست و بی‌خود زیر محملهای حق
\\
من نیم در امر و فرمان نیم‌خام
&&
تا بیندیشم من از تشنیع عام
\\
عام ما و خاص ما فرمان اوست
&&
جان ما بر رو دوان جویان اوست
\\
فردی ما جفتی ما نه از هواست
&&
جان ما چون مهره در دست خداست
\\
ناز آن ابله کشیم و صد چو او
&&
نه ز عشق رنگ و نه سودای بو
\\
این قدر خود درس شاگردان ماست
&&
کر و فر ملحمهٔ ما تا کجاست
\\
تا کجا آنجا که جا را راه نیست
&&
جز سنابرق مه الله نیست
\\
از همه اوهام و تصویرات دور
&&
نور نور نور نور نور نور
\\
بهر تو ار پست کردم گفت و گو
&&
تا بسازی با رفیق زشت‌خو
\\
تا کشی خندان و خوش بار حرج
&&
از پی الصبر مفتاح الفرج
\\
چون بسازی با خسی این خسان
&&
گردی اندر نور سنتها رسان
\\
که انبیا رنج خسان بس دیده‌اند
&&
از چنین ماران بسی پیچیده‌اند
\\
چون مراد و حکم یزدان غفور
&&
بود در قدمت تجلی و ظهور
\\
بی ز ضدی ضد را نتوان نمود
&&
وان شه بی‌مثل را ضدی نبود
\\
\end{longtable}
\end{center}
