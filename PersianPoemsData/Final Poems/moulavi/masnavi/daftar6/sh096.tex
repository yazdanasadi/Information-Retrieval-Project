\begin{center}
\section*{بخش ۹۶ - باخبر شدن آن غریب از وفات آن محتسب و استغفار او از اعتماد بر مخلوق و تعویل بر عطای مخلوق و یاد نعمتهای حق کردنش و انابت به حق از جرم خود ثم الذین کفروا بربهم یعدلون}
\label{sec:sh096}
\addcontentsline{toc}{section}{\nameref{sec:sh096}}
\begin{longtable}{l p{0.5cm} r}
چون به هوش آمد بگفت ای کردگار
&&
مجرمم بودم به خلق اومیدوار
\\
گرچه خواجه بس سخاوت کرده بود
&&
هیچ آن کفو عطای تو نبود
\\
او کله بخشید و تو سر پر خرد
&&
او قبا بخشید و تو بالا و قد
\\
او زرم داد و تو دست زرشمار
&&
او ستورم داد و تو عقل سوار
\\
خواجه شمعم دادو تو چشم قریر
&&
خواجه نقلم داد و تو طعمه‌پذیر
\\
او وظیفه داد و تو عمر و حیات
&&
وعده‌اش زر وعدهٔ تو طیبات
\\
او وثاقم داد و تو چرخ و زمین
&&
در وثاقت او و صد چون او سمین
\\
زر از آن تست زر او نافرید
&&
نان از آن تست نان از تش رسید
\\
آن سخا و رحم هم تو دادیش
&&
کز سخاوت می‌فزودی شادیش
\\
من مرورا قبلهٔ خود ساختم
&&
قبله‌ساز اصل را انداختم
\\
ما کجا بودیم کان دیان دین
&&
عقل می‌کارید اندر آب و طین
\\
چون همی کرد از عدم گردون پدید
&&
وین بساط خاک را می‌گسترید
\\
ز اختران می‌ساخت او مصباح‌ها
&&
وز طبایع قفل با مفتاح‌ها
\\
ای بسا بنیادها پنهان و فاش
&&
مضمر این سقف کرد و این فراش
\\
آدم اصطرلاب اوصاف علوست
&&
وصف آدم مظهر آیات اوست
\\
هرچه در وی می‌نماید عکس اوست
&&
هم‌چو عکس ماه اندر آب جوست
\\
بر صطرلابش نقوش عنکبوت
&&
بهر اوصاف ازل دارد ثبوت
\\
تا ز چرخ غیب وز خورشید روح
&&
عنکبوتش درس گوید از شروح
\\
عنکبوت و این صطرلاب رشاد
&&
بی‌منجم در کف عام اوفتاد
\\
انبیا را داد حق تنجیم این
&&
غیب را چشمی بباید غیب‌بین
\\
در چه دنیا فتادند این قرون
&&
عکس خود را دید هر یک چه درون
\\
از برون دان آنچ در چاهت نمود
&&
ورنه آن شیری که در چه شد فرود
\\
برد خرگوشیش از ره کای فلان
&&
در تگ چاهست آن شیر ژیان
\\
در رو اندر چاه کین از وی بکش
&&
چون ازو غالب‌تری سر بر کنش
\\
آن مقلد سخرهٔ خرگوش شد
&&
از خیال خویشتن پر جوش شد
\\
او نگفت این نقش داد آب نیست
&&
این به جز تقلیب آن قلاب نیست
\\
تو هم از دشمن چو کینی می‌کشی
&&
ای زبون شش غلط در هر ششی
\\
آن عداوت اندرو عکس حقست
&&
کز صفات قهر آنجا مشتقست
\\
وآن گنه در وی ز جنس جرم تست
&&
باید آن خو را ز طبع خویش شست
\\
خلق زشتت اندرو رویت نمود
&&
که ترا او صفحهٔ آیینه بود
\\
چونک قبح خویش دیدی ای حسن
&&
اندر آیینه بر آیینه مزن
\\
می‌زند بر آب استارهٔ سنی
&&
خاک تو بر عکس اختر می‌زنی
\\
کین ستارهٔ نحس در آب آمدست
&&
تا کند او سعد ما را زیردست
\\
خاک استیلا بریزی بر سرش
&&
چونک پنداری ز شبهه اخترش
\\
عکس پنهان گشت و اندر غیب راند
&&
تو گمان بردی که آن اختر نماند
\\
آن ستارهٔ نحس هست اندر سما
&&
هم بدان سو بایدش کردن دوا
\\
بلک باید دل سوی بی‌سوی بست
&&
نحس این سو عکس نحس بی‌سو است
\\
داد داد حق شناس و بخششش
&&
عکس آن دادست اندر پنج و شش
\\
گر بود داد خسان افزون ز ریگ
&&
تو بمیری وآن بماند مردریگ
\\
عکس آخر چند پاید در نظر
&&
اصل بینی پیشه کن ای کژنگر
\\
حق چو بخشش کرد بر اهل نیاز
&&
با عطا بخشیدشان عمر دراز
\\
خالدین شد نعمت و منعم علیه
&&
محیی الموتاست فاجتازوا الیه
\\
داد حق با تو در آمیزد چو جان
&&
آنچنان که آن تو باشی و تو آن
\\
گر نماند اشتهای نان و آب
&&
بدهدت بی این دو قوت مستطاب
\\
فربهی گر رفت حق در لاغری
&&
فربهی پنهانت بخشد آن سری
\\
چون پری را قوت از بو می‌دهد
&&
هر ملک را قوت جان او می‌دهد
\\
جان چه باشد که تو سازی زو سند
&&
حق به عشق خویش زنده‌ت می‌کند
\\
زو حیات عشق خواه و جان مخواه
&&
تو ازو آن رزق خواه و نان مخواه
\\
خلق را چون آب دان صاف و زلال
&&
اندر آن تابان صفات ذوالجلال
\\
علمشان و عدلشان و لطفشان
&&
چون ستارهٔ چرخ در آب روان
\\
پادشاهان مظهر شاهی حق
&&
فاضلان مرآت آگاهی حق
\\
قرنها بگذشت و این قرن نویست
&&
ماه آن ماهست آب آن آب نیست
\\
عدل آن عدلست و فضل آن فضل هم
&&
لیک مستبدل شد آن قرن و امم
\\
قرنها بر قرنها رفت ای همام
&&
وین معانی بر قرار و بر دوام
\\
آن مبدل شد درین جو چند بار
&&
عکس ماه و عکس اختر بر قرار
\\
پس بنااش نیست بر آب روان
&&
بلک بر اقطار عرض آسمان
\\
این صفتها چون نجوم معنویست
&&
دانک بر چرخ معانی مستویست
\\
خوب‌رویان آینهٔ خوبی او
&&
عشق ایشان عکس مطلوبی او
\\
هم به اصل خود رود این خد و خال
&&
دایما در آب کی ماند خیال
\\
جمله تصویرات عکس آب جوست
&&
چون بمالی چشم خود خود جمله اوست
\\
باز عقلش گفت بگذار این حول
&&
خل دوشابست و دوشابست خل
\\
خواجه را چون غیر گفتی از قصور
&&
شرم‌دار ای احول از شاه غیور
\\
خواجه را که در گذشتست از اثیر
&&
جنس این موشان تاریکی مگیر
\\
خواجهٔ جان بین مبین جسم گران
&&
مغز بین او را مبینش استخوان
\\
خواجه را از چشم ابلیس لعین
&&
منگر و نسبت مکن او را به طین
\\
همره خورشید را شب‌پر مخوان
&&
آنک او مسجود شد ساجد مدان
\\
عکس‌ها را ماند این و عکس نیست
&&
در مثال عکس حق بنمودنیست
\\
آفتابی دید او جامد نماند
&&
روغن گل روغن کنجد نماند
\\
چون مبدل گشته‌اند ابدال حق
&&
نیستند از خلق بر گردان ورق
\\
قبلهٔ وحدانیت دو چون بود
&&
خاک مسجود ملایک چون شود
\\
چون درین جو دیدعکس سیب مرد
&&
دامنش را دید آن پر سیب کرد
\\
آنچ در جو دید کی باشد خیال
&&
چونک شد از دیدنش پر صد جوال
\\
تن مبین و آن مکن کان بکم و صم
&&
کذبوا بالحق لما جائهم
\\
ما رمیت اذ رمیت احمد بدست
&&
دیدن او دیدن خالق شدست
\\
خدمت او خدمت حق کردنست
&&
روز دیدن دیدن این روزنست
\\
خاصه این روزن درخشان از خودست
&&
نی ودیعهٔ آفتاب و فرقدست
\\
هم از آن خورشید زد بر روزنی
&&
لیک از راه و سوی معهود نی
\\
در میان شمس و این روزن رهی
&&
هست روزنها نشد زو آگهی
\\
تا اگر ابری بر آید چرخ‌پوش
&&
اندرین روزن بود نورش به جوش
\\
غیر راه این هوا و شش جهت
&&
در میان روزن و خور مالفت
\\
مدحت و تسبیح او تسبیح حق
&&
میوه می‌روید ز عین این طبق
\\
سیب روید زین سبد خوش لخت لخت
&&
عیب نبود گر نهی نامش درخت
\\
این سبد را تو درخت سیب خوان
&&
که میان هر دو راه آمد نهان
\\
آنچ روید از درخت بارور
&&
زین سبد روید همان نوع از ثمر
\\
پس سبد را تو درخت بخت بین
&&
زیر سایهٔ این سبد خوش می‌نشین
\\
نان چو اطلاق آورد ای مهربان
&&
نان چرا می‌گوییش محموده خوان
\\
خاک ره چون چشم روشن کرد و جان
&&
خاک او را سرمه بین و سرمه دان
\\
چون ز روی این زمین تابد شروق
&&
من چرا بالا کنم رو در عیوق
\\
شد فنا هستش مخوان ای چشم‌شوخ
&&
در چنین جو خشک کی ماند کلوخ
\\
پیش این خورشید کی تابد هلال
&&
با چنان رستم چه باشد زور زال
\\
طالبست و غالبست آن کردگار
&&
تا ز هستی‌ها بر آرد او دمار
\\
دو مگو و دو مدان و دو مخوان
&&
بنده را در خواجهٔ خود محو دان
\\
خواجه هم در نور خواجه‌آفرین
&&
فانیست و مرده و مات و دفین
\\
چون جدا بینی ز حق این خواجه را
&&
گم کنی هم متن و هم دیباجه را
\\
چشم و دل را هین گذاره کن ز طین
&&
این یکی قبله‌ست دو قبله مبین
\\
چون دو دیدی ماندی از هر دو طرف
&&
آتشی در خف فتاد و رفت خف
\\
\end{longtable}
\end{center}
