\begin{center}
\section*{بخش ۱۵ - حکایت پاسبان کی خاموش کرد تا دزدان رخت تاجران بردند به کلی بعد از آن هیهای و پاسبانی می‌کرد}
\label{sec:sh015}
\addcontentsline{toc}{section}{\nameref{sec:sh015}}
\begin{longtable}{l p{0.5cm} r}
پاسبانی خفت و دزد اسباب برد
&&
رختها را زیر هر خاکی فشرد
\\
روز شد بیدار شد آن کاروان
&&
دید رفته رخت و سیم و اشتران
\\
پس بدو گفتند ای حارس بگو
&&
که چه شد این رخت و این اسباب کو
\\
گفت دزدان آمدند اندر نقاب
&&
رختها بردند از پیشم شتاب
\\
قوم گفتندش که ای چو تل ریگ
&&
پس چه می‌کردی کیی ای مردریگ
\\
گفت من یک کس بدم ایشان گروه
&&
با سلاح و با شجاعت با شکوه
\\
گفت اگر در جنگ کم بودت امید
&&
نعره‌ای زن کای کریمان برجهید
\\
گفت آن دم کارد بنمودند و تیغ
&&
که خمش ورنه کشیمت بی‌دریغ
\\
آن زمان از ترس بستم من دهان
&&
این زمان هیهای و فریاد و فغان
\\
آن زمان بست آن دمم که دم زنم
&&
این زمان چندانک خواهی هی کنم
\\
چونک عمرت برد دیو فاضحه
&&
بی‌نمک باشد اعوذ و فاتحه
\\
گرچه باشد بی‌نمک اکنون حنین
&&
هست غفلت بی‌نمک‌تر زان یقین
\\
هم‌چنین هم بی‌نمک می‌نال نیز
&&
که ذلیلان را نظر کن ای عزیز
\\
قادری بی‌گاه باشد یا به گاه
&&
از تو چیزی فوت کی شد ای اله
\\
شاه لا تاسوا علی ما فاتکم
&&
کی شود از قدرتش مطلوب گم
\\
\end{longtable}
\end{center}
