\begin{center}
\section*{بخش ۹۳ - داستان آن مرد کی وظیفه داشت از محتسب تبریز و وامها کرده بود بر امید آن وظیفه و او را خبر نه از وفات او  حاصل از هیچ زنده‌ای وام او گزارده نشد الا از محتسب متوفی گزارده شد چنانک گفته‌اند لیس من مات فاستراح بمیت  انما المیت میت الاحیاء}
\label{sec:sh093}
\addcontentsline{toc}{section}{\nameref{sec:sh093}}
\begin{longtable}{l p{0.5cm} r}
آن یکی درویش ز اطراف دیار
&&
جانب تبریز آمد وامدار
\\
نه هزارش وام بد از زر مگر
&&
بود در تبریز بدرالدین عمر
\\
محتسب بد او به دل بحر آمده
&&
هر سر مویش یکی حاتم‌کده
\\
حاتم ار بودی گدای او شدی
&&
سر نهادی خاک پای او شدی
\\
گر بدادی تشنه را بحری زلال
&&
در کرم شرمنده بودی زان نوال
\\
ور بکردی ذره‌ای را مشرقی
&&
بودی آن در همتش نالایقی
\\
بر امید او بیامد آن غریب
&&
کو غریبان را بدی خویش و نسیب
\\
با درش بود آن غریب آموخته
&&
وام بی‌حد از عطایش توخته
\\
هم به پشت آن کریم او وام کرد
&&
که ببخششهاش واثق بود مرد
\\
لا ابالی گشته زو و وام‌جو
&&
بر امید قلزم اکرام‌خو
\\
وام‌داران روترش او شادکام
&&
هم‌چو گل خندان از آن روض الکرام
\\
گرم شد پشتش ز خورشید عرب
&&
چه غمستش از سبال بولهب
\\
چونک دارد عهد و پیوند سحاب
&&
کی دریغ آید ز سقایانش آب
\\
ساحران واقف از دست خدا
&&
کی نهند این دست و پا را دست و پا
\\
روبهی که هست زان شیرانش پشت
&&
بشکند کلهٔ پلنگان را به مشت
\\
\end{longtable}
\end{center}
