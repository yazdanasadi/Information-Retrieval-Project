\begin{center}
\section*{بخش ۱۱ - مدافعهٔ امرا آن حجت را به شبههٔ جبریانه و جواب دادن شاه ایشان را}
\label{sec:sh011}
\addcontentsline{toc}{section}{\nameref{sec:sh011}}
\begin{longtable}{l p{0.5cm} r}
پس بگفتند آن امیران کین فنیست
&&
از عنایتهاش کار جهد نیست
\\
قسمت حقست مه را روی نغز
&&
دادهٔ بختست گل را بوی نغز
\\
گفت سلطان بلک آنچ از نفس زاد
&&
ریع تقصیرست و دخل اجتهاد
\\
ورنه آدم کی بگفتی با خدا
&&
ربنا انا ظلمنا نفسنا
\\
خود بگفتی کین گناه از نفس بود
&&
چون قضا این بود حزم ما چه سود
\\
هم‌چو ابلیسی که گفت اغویتنی
&&
تو شکستی جام و ما را می‌زنی
\\
بل قضا حقست و جهد بنده حق
&&
هین مباش اعور چو ابلیس خلق
\\
در تردد مانده‌ایم اندر دو کار
&&
این تردد کی بود بی‌اختیار
\\
این کنم یا آن کنم او کی گود
&&
که دو دست و پای او بسته بود
\\
هیچ باشد این تردد بر سرم
&&
که روم در بحر یا بالا پرم
\\
این تردد هست که موصل روم
&&
یا برای سحر تا بابل روم
\\
پس تردد را بباید قدرتی
&&
ورنه آن خنده بود بر سبلتی
\\
بر قضا کم نه بهانه ای جوان
&&
جرم خود را چون نهی بر دیگران
\\
خون کند زید و قصاص او به عمر
&&
می خورد عمرو و بر احمد حد خمر
\\
گرد خود برگرد و جرم خود ببین
&&
جنبش از خود بین و از سایه مبین
\\
که نخواهد شد غلط پاداش میر
&&
خصم را می‌داند آن میر بصیر
\\
چون عسل خوردی نیامد تب به غیر
&&
مزد روز تو نیامد شب به غیر
\\
در چه کردی جهد کان وا تو نگشت
&&
تو چه کاریدی که نامد ریع کشت
\\
فعل تو که زاید از جان و تنت
&&
هم‌چو فرزندت بگیرد دامنت
\\
فعل را در غیب صورت می‌کنند
&&
فعل دزدی را نه داری می‌زنند
\\
دار کی ماند به دزدی لیک آن
&&
هست تصویر خدای غیب‌دان
\\
در دل شحنه چو حق الهام داد
&&
که چنین صورت بساز از بهر داد
\\
تا تو عالم باشی و عادل قضا
&&
نامناسب چون دهد داد و سزا
\\
چونک حاکم این کند اندر گزین
&&
چون کند حکم احکم این حاکمین
\\
چون بکاری جو نروید غیر جو
&&
قرض تو کردی ز که خواهد گرو
\\
جرم خود را بر کسی دیگر منه
&&
هوش و گوش خود بدین پاداش ده
\\
جرم بر خود نه که تو خود کاشتی
&&
با جزا و عدل حق کن آشتی
\\
رنج را باشد سبب بد کردنی
&&
بد ز فعل خود شناس از بخت نی
\\
آن نظر در بخت چشم احوال کند
&&
کلب را کهدانی و کاهل کند
\\
متهم کن نفس خود را ای فتی
&&
متهم کم کن جزای عدل را
\\
توبه کن مردانه سر آور به ره
&&
که فمن یعمل بمثقال یره
\\
در فسون نفس کم شو غره‌ای
&&
که آفتاب حق نپوشد ذره‌ای
\\
هست این ذرات جسمی ای مفید
&&
پیش این خورشید جسمانی پدید
\\
هست ذرات خواطر و افتکار
&&
پیش خورشید حقایق آشکار
\\
\end{longtable}
\end{center}
