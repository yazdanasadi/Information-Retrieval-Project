\begin{center}
\section*{بخش ۸۵ - حکایت تعلق موش با چغز و بستن پای هر دو به رشته‌ای دراز و بر کشیدن زاغ موش را و معلق شدن چغز و نالیدن و پشیمانی او از تعلق با غیر جنس و با جنس خود ناساختن}
\label{sec:sh085}
\addcontentsline{toc}{section}{\nameref{sec:sh085}}
\begin{longtable}{l p{0.5cm} r}
از قضا موشی و چغزی با وفا
&&
بر لب جو گشته بودند آشنا
\\
هر دو تن مربوط میقاتی شدند
&&
هر صباحی گوشه‌ای می‌آمدند
\\
نرد دل با هم‌دگر می‌باختند
&&
از وساوس سینه می‌پرداختند
\\
هر دو را دل از تلاقی متسع
&&
هم‌دگر را قصه‌خوان و مستمع
\\
رازگویان با زبان و بی‌زبان
&&
الجماعه رحمه را تاویل دان
\\
آن اشر چون جفت آن شاد آمدی
&&
پنج ساله قصه‌اش یاد آمدی
\\
جوش نطق از دل نشان دوستیست
&&
بستگی نطق از بی‌الفتیست
\\
دل که دلبر دید کی ماند ترش
&&
بلبلی گل دید کی ماند خمش
\\
ماهی بریان ز آسیب خضر
&&
زنده شد در بحر گشت او مستقر
\\
یار را با یار چون بنشسته شد
&&
صد هزاران لوح سر دانسته شد
\\
لوح محفوظ است پیشانی یار
&&
راز کونینش نماید آشکار
\\
هادی راهست یار اندر قدوم
&&
مصطفی زین گفت اصحابی نجوم
\\
نجم اندر ریگ و دریا رهنماست
&&
چشم اندر نجم نه کو مقتداست
\\
چشم را با روی او می‌دار جفت
&&
گرد منگیزان ز راه بحث و گفت
\\
زانک گردد نجم پنهان زان غبار
&&
چشم بهتر از زبان با عثار
\\
تا بگوید او که وحیستش شعار
&&
کان نشاند گرد و ننگیزد غبار
\\
چون شد آدم مظهر وحی و وداد
&&
ناطقهٔ او علم الاسما گشاد
\\
نام هر چیزی چنانک هست آن
&&
از صحیفهٔ دل روی گشتش زبان
\\
فاش می‌گفتی زبان از ریتش
&&
جمله را خاصیت و ماهیتش
\\
آنچنان نامی که اشیا را سزد
&&
نه چنانک حیز را خواند اسد
\\
نوح نهصد سال در راه سوی
&&
بود هر روزیش تذکیر نوی
\\
لعل او گویا ز یاقوت القلوب
&&
نه رساله خوانده نه قوت القلوب
\\
وعظ را ناموخته هیچ از شروح
&&
بلک ینبوع کشوف و شرح روح
\\
زان میی کان می چو نوشیده شود
&&
آب نطق از گنگ جوشیده شود
\\
طفل نوزاده شود حبر فصیح
&&
حکمت بالغ بخواند چون مسیح
\\
از کهی که یافت زان می خوش‌لبی
&&
صد غزل آموخت داود نبی
\\
جمله مرغان ترک کرده چیک چیک
&&
هم‌زبان و یار داود ملیک
\\
چه عجب که مرغ گردد مست او
&&
هم شنود آهن ندای دست او
\\
صرصری بر عاد قتالی شده
&&
مر سلیمان را چو حمالی شده
\\
صرصری می‌برد بر سر تخت شاه
&&
هر صباح و هر مسا یک ماهه راه
\\
هم شده حمال و هم جاسوس او
&&
گفت غایب را کنان محسوس او
\\
باد دم که گفت غایب یافتی
&&
سوی گوش آن ملک بشتافتی
\\
که فلانی این چنین گفت این زمان
&&
ای سلیمان مه صاحب‌قران
\\
\end{longtable}
\end{center}
