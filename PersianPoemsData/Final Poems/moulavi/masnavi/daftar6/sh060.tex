\begin{center}
\section*{بخش ۶۰ - باز مکرر کردن صوفی سال را}
\label{sec:sh060}
\addcontentsline{toc}{section}{\nameref{sec:sh060}}
\begin{longtable}{l p{0.5cm} r}
گفت صوفی قادرست آن مستعان
&&
که کند سودای ما را بی زیان
\\
آنک آتش را کند ورد و شجر
&&
هم تواند کرد این را بی‌ضرر
\\
آنک گل آرد برون از عین خار
&&
هم تواند کرد این دی را بهار
\\
آنک زو هر سرو آزادی کند
&&
قادرست ار غصه را شادی کند
\\
آنک شد موجود از وی هر عدم
&&
گر بدارد باقیش او را چه کم
\\
آنک تن را جان دهد تا حی شود
&&
گر نمیراند زیانش کی شود
\\
خود چه باشد گر ببخشد آن جواد
&&
بنده را مقصود جان بی‌اجتهاد
\\
دور دارد از ضعیفان در کمین
&&
مکر نفس و فتنهٔ دیو لعین
\\
\end{longtable}
\end{center}
