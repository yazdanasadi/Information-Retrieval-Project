\begin{center}
\section*{بخش ۳ - نکوهیدن ناموسهای پوسیده را کی مانع ذوق ایمان و دلیل ضعف صدق‌اند و راه‌زن صد هزار ابله چنانک راه‌زن آن مخنث شده بودند گوسفندان و نمی‌یارست گذشتن و پرسیدن مخنث از چوپان کی این گوسفندان تو مرا عجب گزند گفت ای مردی و در تو رگ مردی هست همه فدای تو اند و اگر مخنثی هر یکی ترا اژدرهاست مخنثی دیگر هست کی چون گوسفندان را بیند در حال از راه باز گردد نیارد پرسیدن ترسد کی اگر بپرسم گوسفندان در من افتند و مرا بگزند}
\label{sec:sh003}
\addcontentsline{toc}{section}{\nameref{sec:sh003}}
\begin{longtable}{l p{0.5cm} r}
ای ضیاء الحق حسام‌الدین بیا
&&
ای صقال روح و سلطان الهدی
\\
مثنوی را مسرح مشروح ده
&&
صورت امثال او را روح ده
\\
تا حروفش جمله عقل و جان شوند
&&
سوی خلدستان جان پران شوند
\\
هم به سعی تو ز ارواح آمدند
&&
سوی دام حرف و مستحقن شدند
\\
باد عمرت در جهان هم‌چون خضر
&&
جان‌فزا و دستگیر و مستمر
\\
چون خضر و الیاس مانی در جهان
&&
تا زمین گردد ز لطفت آسمان
\\
گفتمی از لطف تو جزوی ز صد
&&
گر نبودی طمطراق چشم بد
\\
لیک از چشم بد زهراب دم
&&
زخمهای روح‌فرسا خورده‌ام
\\
جز به رمز ذکر حال دیگران
&&
شرح حالت می‌نیارم در بیان
\\
این بهانه هم ز دستان دلیست
&&
که ازو پاهای دل اندر گلیست
\\
صد دل و جان عاشق صانع شده
&&
چشم بد یا گوش بد مانع شده
\\
خود یکی بوطالب آن عم رسول
&&
می‌نمودش شنعهٔ عربان مهول
\\
که چه گویندم عرب کز طفل خود
&&
او بگردانید دیدن معتمد
\\
گفتش ای عم یک شهادت تو بگو
&&
تا کنم با حق خصومت بهر تو
\\
گفت لیکن فاش گردد ازسماع
&&
کل سر جاوز الاثنین شاع
\\
من بمانم در زبان این عرب
&&
پش ایشان خوار گردم زین سبب
\\
لیک گر بودیش لطف ما سبق
&&
کی بدی این بددلی با جذب حق
\\
الغیاث ای تو غیاث المستغیث
&&
زین دو شاخهٔ اختیارات خبیث
\\
من ز دستان و ز مکر دل چنان
&&
مات گشتم که بماندم از فغان
\\
من که باشم چرخ با صد کار و بار
&&
زین کمین فریاد کرد از اختیار
\\
که ای خداوند کریم و بردبار
&&
ده امانم زین دو شاخهٔ اختیار
\\
جذب یک راههٔ صراط المستقیم
&&
به ز دو راه تردد ای کریم
\\
زین دو ره گرچه همه مقصد توی
&&
لیک خود جان کندن آمد این دوی
\\
زین دو ره گرچه به جز تو عزم نیست
&&
لیک هرگز رزم هم‌چون بزم نیست
\\
در نبی بشنو بیانش از خدا
&&
آیت اشفقن ان یحملنها
\\
این تردد هست در دل چون وغا
&&
کین بود به یا که آن حال مرا
\\
در تردد می‌زند بر همدگر
&&
خوف و اومید بهی در کر و فر
\\
\end{longtable}
\end{center}
