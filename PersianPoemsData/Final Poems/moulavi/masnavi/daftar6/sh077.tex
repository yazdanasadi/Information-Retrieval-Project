\begin{center}
\section*{بخش ۷۷ - رجوع کردن به قصهٔ قبه و گنج}
\label{sec:sh077}
\addcontentsline{toc}{section}{\nameref{sec:sh077}}
\begin{longtable}{l p{0.5cm} r}
نک خیال آن فقیرم بی‌ریا
&&
عاجز آورد از بیا و از بیا
\\
بانگ او تو نشنوی من بشنوم
&&
زانک در اسرار همراز ویم
\\
طالب گنجش مبین خود گنج اوست
&&
دوست کی باشد به معنی غیر دوست
\\
سجده خود را می‌کند هر لحظه او
&&
سجده پیش آینه‌ست از بهر رو
\\
گر بدیدی ز آینه او یک پشیز
&&
بی‌خیالی زو نماندی هیچ چیز
\\
هم خیالاتش هم او فانی شدی
&&
دانش او محو نادانی شدی
\\
دانشی دیگر ز نادانی ما
&&
سر برآوردی عیان که انی انا
\\
اسجدوا لادم ندا آمد همی
&&
که آدمید و خویش بینیدش دمی
\\
احولی از چشم ایشان دور کرد
&&
تا زمین شد عین چرخ لاژورد
\\
لا اله گفت و الا الله گفت
&&
گشت لا الا الله و وحدت شکفت
\\
آن حبیب و آن خلیل با رشد
&&
وقت آن آمد که گوش ما کشد
\\
سوی چشمه که دهان زینها بشو
&&
آنچ پوشیدیم از خلقان مگو
\\
ور بگویی خود نگردد آشکار
&&
تو به قصد کشف گردی جرم‌دار
\\
لیک من اینک بریشان می‌تنم
&&
قایل این سامع این هم منم
\\
صورت درویش و نقش گنج گو
&&
رنج کیش‌اند این گروه از رنج گو
\\
چشمهٔ راحت بریشان شد حرام
&&
می‌خورند از زهر قاتل جام‌جام
\\
خاکها پر کرده دامن می‌کشند
&&
تا کنند این چشمه‌ها را خشک‌بند
\\
کی شود این چشمهٔ دریامدد
&&
مکتنس زین مشت خاک نیک و بد
\\
لیک گوید با شما من بسته‌ام
&&
بی‌شما من تا ابد پیوسته‌ام
\\
قوم معکوس‌اند اندر مشتها
&&
خاک‌خوار و آب را کرده رها
\\
ضد طبع انبیا دارند خلق
&&
اژدها را متکا دارند خلق
\\
چشم‌بند ختم چون دانسته‌ای
&&
هیچ دانی از چه دیده بسته‌ای
\\
بر چه بگشادی بدل این دیده‌ها
&&
یک به یک بئس البدل دان آن ترا
\\
لیک خورشید عنایت تافته‌ست
&&
آیسان را از کرم در یافته‌ست
\\
نرد بس نادر ز رحمت باخته
&&
عین کفران را انابت ساخته
\\
هم ازین بدبختی خلق آن جواد
&&
منفجر کرده دو صد چشمهٔ وداد
\\
غنچه را از خار سرمایه دهد
&&
مهره را از مار پیرایه دهد
\\
از سواد شب برون آرد نهار
&&
وز کف معسر برویاند یسار
\\
آرد سازد ریگ را بهر خلیل
&&
کو با داود گردد هم رسیل
\\
کوه با وحشت در آن ابر ظلم
&&
بر گشاید بانگ چنگ و زیر و بم
\\
خیز ای داود از خلقان نفیر
&&
ترک آن کردی عوض از ما بگیر
\\
\end{longtable}
\end{center}
