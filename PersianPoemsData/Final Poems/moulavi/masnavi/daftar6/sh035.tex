\begin{center}
\section*{بخش ۳۵ - رنجور شدن این هلال و بی‌خبری خواجهٔ او از رنجوری او از تحقیر و ناشناخت و واقف شدن دل مصطفی علیه‌السلام از رنجوری و حال او و افتقاد و عیادت رسول علیه‌السلام این هلال را}
\label{sec:sh035}
\addcontentsline{toc}{section}{\nameref{sec:sh035}}
\begin{longtable}{l p{0.5cm} r}
از قضا رنجور و ناخوش شد هلال
&&
مصطفی را وحی شد غماز حال
\\
بد ز رنجوریش خواجه‌ش بی‌خبر
&&
که بر او بد کساد و بی‌خطر
\\
خفته نه روز اندر آخر محسنی
&&
هیچ کس از حال او آگاه نی
\\
آنک کس بود و شهنشاه کسان
&&
عقل صد چون قلزمش هر جا رسان
\\
وحیش آمد رحم حق غم‌خوار شد
&&
که فلان مشتاق تو بیمار شد
\\
مصطفی بهر هلال با شرف
&&
رفت از بهر عیادت آن طرف
\\
در پی خورشید وحی آن مه دوان
&&
وآن صحابه در پیش چون اختران
\\
ماه می‌گوید که اصحابی نجوم
&&
للسری قدوه و للطاغی رجوم
\\
میر را گفتند که آن سلطان رسید
&&
او ز شادی بی‌دل و جان برجهید
\\
برگمان آن ز شادی زد دو دست
&&
کان شهنشه بهر او میر آمدست
\\
چون فرو آمد ز غرفه آن امیر
&&
جان همی‌افشاند پامزد بشیر
\\
پس زمین‌بوس و سلام آورد او
&&
کرد رخ را از طرب چون ورد او
\\
گفت بسم‌الله مشرف کن وطن
&&
تا که فردوسی شود این انجمن
\\
تا فزاید قصر من بر آسمان
&&
که بدیدم قطب دوران زمان
\\
گفتش از بهر عتاب آن محترم
&&
من برای دیدن تو نامدم
\\
گفت روحم آن تو خود روح چیست
&&
هین بفرما کین تجشم بهر کیست
\\
تا شوم من خاک پای آن کسی
&&
که به باغ لطف تستش مغرسی
\\
پس بگفتش کان هلال عرش کو
&&
هم‌چو مهتاب از تواضع فرش کو
\\
آن شهی در بندگی پنهان شده
&&
بهر جاسوسی به دنیا آمده
\\
تو مگو کو بنده و آخرجی ماست
&&
این بدان که گنج در ویرانه‌هاست
\\
ای عجب چونست از سقم آن هلال
&&
که هزاران بدر هستش پای‌مال
\\
گفت از رنجش مرا آگاه نیست
&&
لیک روزی چند بر درگاه نیست
\\
صحبت او با ستور و استرست
&&
سایس است و منزلش این آخرست
\\
\end{longtable}
\end{center}
