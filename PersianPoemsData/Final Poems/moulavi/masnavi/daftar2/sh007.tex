\begin{center}
\section*{بخش ۷ - بسته شدن تقریر معنی حکایت به سبب میل مستمع به استماع ظاهر صورت حکایت}
\label{sec:sh007}
\addcontentsline{toc}{section}{\nameref{sec:sh007}}
\begin{longtable}{l p{0.5cm} r}
کی گذارد آنک رشک روشنیست
&&
تا بگویم آنچ فرض و گفتنیست
\\
بحر کف پیش آرد و سدی کند
&&
جر کند وز بعد جر مدی کند
\\
این زمان بشنو چه مانع شد مگر
&&
مستمع را رفت دل جای دگر
\\
خاطرش شد سوی صوفی قنق
&&
اندر آن سودا فرو شد تا عنق
\\
لازم آمد باز رفتن زین مقال
&&
سوی آن افسانه بهر وصف حال
\\
صوفی آن صورت مپندار ای عزیز
&&
همچو طفلان تا کی از جوز و مویز
\\
جسم ما جوز و مویزست ای پسر
&&
گر تو مردی زین دو چیز اندر گذر
\\
ور تو اندر نگذری اکرام حق
&&
بگذراند مر ترا از نه طبق
\\
بشنو اکنون صورت افسانه را
&&
لیک هین از که جداکن دانه را
\\
\end{longtable}
\end{center}
