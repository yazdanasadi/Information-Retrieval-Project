\begin{center}
\section*{بخش ۶۴ - باز جواب گفتن ابلیس معاویه را}
\label{sec:sh064}
\addcontentsline{toc}{section}{\nameref{sec:sh064}}
\begin{longtable}{l p{0.5cm} r}
گفت ما اول فرشته بوده‌ایم
&&
راه طاعت را بجان پیموده‌ایم
\\
سالکان راه را محرم بدیم
&&
ساکنان عرش را همدم بدیم
\\
پیشهٔ اول کجا از دل رود
&&
مهر اول کی ز دل بیرون شود
\\
در سفر گر روم بینی یا ختن
&&
از دل تو کی رود حب الوطن
\\
ما هم از مستان این می بوده‌ایم
&&
عاشقان درگه وی بوده‌ایم
\\
ناف ما بر مهر او ببریده‌اند
&&
عشق او در جان ما کاریده‌اند
\\
روز نیکو دیده‌ایم از روزگار
&&
آب رحمت خورده‌ایم اندر بهار
\\
نی که ما را دست فضلش کاشتست
&&
از عدم ما را نه او بر داشتست
\\
ای بسا کز وی نوازش دیده‌ایم
&&
در گلستان رضا گردیده‌ایم
\\
بر سر ما دست رحمت می‌نهاد
&&
چشمه‌های لطف از ما می‌گشاد
\\
وقت طفلی‌ام که بودم شیرجو
&&
گاهوارم را کی جنبانید او
\\
از کی خوردم شیر غیر شیر او
&&
کی مرا پرورد جز تدبیر او
\\
خوی کان با شیر رفت اندر وجود
&&
کی توان آن را ز مردم واگشود
\\
گر عتابی کرد دریای کرم
&&
بسته کی گردند درهای کرم
\\
اصل نقدش داد و لطف و بخششست
&&
قهر بر وی چون غباری از غشست
\\
از برای لطف عالم را بساخت
&&
ذره‌ها را آفتاب او نواخت
\\
فرقت از قهرش اگر آبستنست
&&
بهر قدر وصل او دانستنست
\\
تا دهد جان را فراقش گوشمال
&&
جان بداند قدر ایام وصال
\\
گفت پیغامبر که حق فرموده است
&&
قصد من از خلق احسان بوده است
\\
آفریدم تا ز من سودی کنند
&&
تا ز شهدم دست‌آلودی کنند
\\
نه برای آنک تا سودی کنم
&&
وز برهنه من قبایی بر کنم
\\
چند روزی که ز پیشم رانده‌ست
&&
چشم من در روی خوبش مانده‌ست
\\
کز چنان رویی چنین قهر ای عجب
&&
هر کسی مشغول گشته در سبب
\\
من سبب را ننگرم کان حادثست
&&
زانک حادث حادثی را باعثست
\\
لطف سابق را نظاره می‌کنم
&&
هرچه آن حادث دو پاره می‌کنم
\\
ترک سجده از حسد گیرم که بود
&&
آن حسد از عشق خیزد نه از جحود
\\
هر حسد از دوستی خیزد یقین
&&
که شود با دوست غیری همنشین
\\
هست شرط دوستی غیرت‌پزی
&&
همچو شرط عطسه گفتن دیر زی
\\
چونک بر نطعش جز این بازی نبود
&&
گفت بازی کن چه دانم در فزود
\\
آن یکی بازی که بد من باختم
&&
خویشتن را در بلا انداختم
\\
در بلا هم می‌چشم لذات او
&&
مات اویم مات اویم مات او
\\
چون رهاند خویشتن را ای سره
&&
هیچ کس در شش جهت از ششدره
\\
جزو شش از کل شش چون وا رهد
&&
خاصه که بی چون مرورا کژ نهد
\\
هر که در شش او درون آتشست
&&
اوش برهاند که خلاق ششست
\\
خود اگر کفرست و گر ایمان او
&&
دست‌باف حضرتست و آن او
\\
\end{longtable}
\end{center}
