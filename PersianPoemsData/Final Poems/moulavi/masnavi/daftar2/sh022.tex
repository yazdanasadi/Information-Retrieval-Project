\begin{center}
\section*{بخش ۲۲ - براه کردن شاه یکی را از آن دو غلام و ازین دیگر پرسیدن}
\label{sec:sh022}
\addcontentsline{toc}{section}{\nameref{sec:sh022}}
\begin{longtable}{l p{0.5cm} r}
آن غلامک را چو دید اهل ذکا
&&
آن دگر را کرد اشارت که بیا
\\
کاف رحمت گفتمش تصغیر نیست
&&
جد گود فرزندکم تحقیر نیست
\\
چون بیامد آن دوم در پیش شاه
&&
بود او گنده‌دهان دندان سیاه
\\
گرچه شه ناخوش شد از گفتار او
&&
جست و جویی کرد هم ز اسرار او
\\
گفت با این شکل و این گند دهان
&&
دور بنشین لیک آن سوتر مران
\\
که تو اهل نامه و رقعه بدی
&&
نه جلیس و یار و هم‌بقعه بدی
\\
تا علاج آن دهان تو کنیم
&&
تو حبیب و ما طبیب پر فنیم
\\
بهر کیکی نو گلیمی سوختن
&&
نیست لایق از تو دیده دوختن
\\
با همه بنشین دو سه دستان بگو
&&
تا ببینم صورت عقلت نکو
\\
آن ذکی را پس فرستاد او به کار
&&
سوی حمامی که رو خود را بخار
\\
وین دگر را گفت خه تو زیرکی
&&
صد غلامی در حقیقت نه یکی
\\
آن نه‌ای که خواجه‌تاش تو نمود
&&
از تو ما را سرد می‌کرد آن حسود
\\
گفت او دزد و کژست و کژنشین
&&
حیز و نامرد و چنینست و چنین
\\
گفت پیوسته بدست او راست‌گو
&&
راست‌گویی من ندیدستم چو او
\\
راست‌گویی در نهادش خلقتیست
&&
هرچه گوید من نگویم آن تهیست
\\
کژ ندانم آن نکواندیش را
&&
متهم دارم وجود خویش را
\\
باشد او در من ببیند عیبها
&&
من نبینم در وجود خود شها
\\
هر کسی گر عیب خود دیدی ز پیش
&&
کی بدی فارغ وی از اصلاح خویش
\\
غافل‌اند این خلق از خود ای پدر
&&
لاجرم گویند عیب همدگر
\\
من نبینم روی خود را ای شمن
&&
من ببینم روی تو تو روی من
\\
آنکسی که او ببیند روی خویش
&&
نور او از نور خلقانست بیش
\\
گر بیمرد دید او باقی بود
&&
زانک دیدش دید خلاقی بود
\\
نور حسی نبود آن نوری که او
&&
روی خود محسوس بیند پیش رو
\\
گفت اکنون عیبهای او بگو
&&
آنچنان که گفت او از عیب تو
\\
تا بدانم که تو غمخوار منی
&&
کدخدای ملکت و کار منی
\\
گفت ای شه من بگویم عیبهاش
&&
گرچه هست او مر مرا خوش خواجه‌تاش
\\
عیب او مهر و وفا و مردمی
&&
عیب او صدق و ذکا و همدمی
\\
کمترین عیبش جوامردی و داد
&&
آن جوامردی که جان را هم بداد
\\
صد هزاران جان خدا کرده پدید
&&
چه جوامردی بود کان را ندید
\\
ور بدیدی کی بجان بخلش بدی
&&
بهر یک جان کی چنین غمگین شدی
\\
بر لب جو بخل آب آن را بود
&&
کو ز جوی آب نابینا بود
\\
گفت پیغامبر که هر که از یقین
&&
داند او پاداش خود در یوم دین
\\
که یکی را ده عوض می‌آیدش
&&
هر زمان جودی دگرگون زایدش
\\
جود جمله از عوضها دیدنست
&&
پس عوض دیدن ضد ترسیدنست
\\
بخل نادیدن بود اعواض را
&&
شاد دارد دید در خواض را
\\
پس بعالم هیچ کس نبود بخیل
&&
زانک کس چیزی نبازد بی بدیل
\\
پس سخا از چشم آمد نه ز دست
&&
دید دارد کار جز بینا نرست
\\
عیب دیگر این که خودبین نیست او
&&
هست او در هستی خود عیب‌جو
\\
عیب‌گوی و عیب‌جوی خود بدست
&&
با همه نیکو و با خود بد بدست
\\
گفت شه جلدی مکن در مدح یار
&&
مدح خود در ضمن مدح او میار
\\
زانک من در امتحان آرم ورا
&&
شرمساری آیدت در ما ورا
\\
\end{longtable}
\end{center}
