\begin{center}
\section*{بخش ۹ - گمان بردن کاروانیان که بهیمهٔ صوفی رنجورست}
\label{sec:sh009}
\addcontentsline{toc}{section}{\nameref{sec:sh009}}
\begin{longtable}{l p{0.5cm} r}
چونک صوفی بر نشست و شد روان
&&
رو در افتادن گرفت او هر زمان
\\
هر زمانش خلق بر می‌داشتند
&&
جمله رنجورش همی‌پنداشتند
\\
آن یکی گوشش همی‌پیچید سخت
&&
وان دگر در زیر کامش جست لخت
\\
وان دگر در نعل او می‌جست سنگ
&&
وان دگر در چشم او می‌دید زنگ
\\
باز می‌گفتند ای شیخ این ز چیست
&&
دی نمی‌گفتی که شکر این خر قویست
\\
گفت آن خر کو بشب لا حول خورد
&&
جز بدین شیوه نداند راه کرد
\\
چونک قوت خر بشب لا حول بود
&&
شب مسبح بود و روز اندر سجود
\\
آدمی خوارند اغلب مردمان
&&
از سلام علیکشان کم جو امان
\\
خانهٔ دیوست دلهای همه
&&
کم پذیر از دیومردم دمدمه
\\
از دم دیو آنک او لا حول خورد
&&
همچو آن خر در سر آید در نبرد
\\
هر که در دنیا خورد تلبیس دیو
&&
وز عدو دوست‌رو تعظیم و ریو
\\
در ره اسلام و بر پول صراط
&&
در سر آید همچو آن خر از خباط
\\
عشوه‌های یار بد منیوش هین
&&
دام بین ایمن مرو تو بر زمین
\\
صد هزار ابلیس لا حول آر بین
&&
آدما ابلیس را در مار بین
\\
دم دهد گوید ترا ای جان و دوست
&&
تا چو قصابی کشد از دوست پوست
\\
دم دهد تا پوستت بیرون کشد
&&
وای او کز دشمنان افیون چشد
\\
سر نهد بر پای تو قصاب‌وار
&&
دم دهد تا خونت ریزد زار زار
\\
همچو شیری صید خود را خویش کن
&&
ترک عشوهٔ اجنبی و خویش کن
\\
همچو خادم دان مراعات خسان
&&
بی‌کسی بهتر ز عشوهٔ ناکسان
\\
در زمین مردمان خانه مکن
&&
کار خود کن کار بیگانه مکن
\\
کیست بیگانه تن خاکی تو
&&
کز برای اوست غمناکی تو
\\
تا تو تن را چرب و شیرین می‌دهی
&&
جوهر خود را نبینی فربهی
\\
گر میان مشک تن را جا شود
&&
روز مردن گند او پیدا شود
\\
مشک را بر تن مزن بر دل بمال
&&
مشک چه بود نام پاک ذوالجلال
\\
آن منافق مشک بر تن می‌نهد
&&
روح را در قعر گلخن می‌نهد
\\
بر زبان نام حق و در جان او
&&
گندها از فکر بی ایمان او
\\
ذکر با او همچو سبزهٔ گلخنست
&&
بر سر مبرز گلست و سوسنست
\\
آن نبات آنجا یقین عاریتست
&&
جای آن گل مجلسست و عشرتست
\\
طیبات آید به سوی طیبین
&&
للخبیثین الخبیثات است هین
\\
کین مدار آنها که از کین گمرهند
&&
گورشان پهلوی کین‌داران نهند
\\
اصل کینه دوزخست و کین تو
&&
جزو آن کلست و خصم دین تو
\\
چون تو جزو دوزخی پس هوش دار
&&
جزو سوی کل خود گیرد قرار
\\
ور تو جزو جنتی ای نامدار
&&
عیش تو باشد ز جنت پایدار
\\
تلخ با تلخان یقین ملحق شود
&&
کی دم باطل قرین حق شود
\\
ای برادر تو همان اندیشه‌ای
&&
ما بقی تو استخوان و ریشه‌ای
\\
گر گلست اندیشهٔ تو گلشنی
&&
ور بود خاری تو هیمهٔ گلخنی
\\
گر گلابی بر سر جیبت زنند
&&
ور تو چون بولی برونت افکنند
\\
طبله‌ها در پیش عطاران ببین
&&
جنس را با جنس خود کرده قرین
\\
جنسها با جنسها آمیخته
&&
زین تجانس زینتی انگیخته
\\
گر در آمیزند عود و شکرش
&&
بر گزیند یک یک از یک‌دیگرش
\\
طبله‌ها بشکست و جانها ریختند
&&
نیک و بد درهمدگر آمیختند
\\
حق فرستاد انبیا را با ورق
&&
تا گزید این دانه‌ها را بر طبق
\\
پیش ازیشان ما همه یکسان بدیم
&&
کس ندانستی که ما نیک و بدیم
\\
قلب و نیکو در جهان بودی روان
&&
چون همه شب بود و ما چون شب‌روان
\\
تا بر آمد آفتاب انبیا
&&
گفت ای غش دور شو صافی بیا
\\
چشم داند فرق کردن رنگ را
&&
چشم داند لعل را و سنگ را
\\
چشم داند گوهر و خاشاک را
&&
چشم را زان می‌خلد خاشاکها
\\
دشمن روزند این قلابکان
&&
عاشق روزند آن زرهای کان
\\
زانک روزست آینهٔ تعریف او
&&
تا ببیند اشرفی تشریف او
\\
حق قیامت را لقب زان روز کرد
&&
روز بنماید جمال سرخ و زرد
\\
پس حقیقت روز سر اولیاست
&&
روز پیش ماهشان چون سایه‌هاست
\\
عکس راز مرد حق دانید روز
&&
عکس ستاریش شام چشم‌دوز
\\
زان سبب فرمود یزدان والضحی
&&
والضحی نور ضمیر مصطفی
\\
قول دیگر کین ضحی را خواست دوست
&&
هم برای آنک این هم عکس اوست
\\
ورنه بر فانی قسم گفتن خطاست
&&
خود فنا چه لایق گفت خداست
\\
از خلیلی لا احب افلین
&&
پس فنا چون خواست رب العالمین
\\
لا احب افلین گفت آن خلیل
&&
کی فنا خواهد ازین رب جلیل
\\
باز واللیل است ستاری او
&&
وان تن خاکی زنگاری او
\\
آفتابش چون برآمد زان فلک
&&
با شب تن گفت هین ما ودعک
\\
وصل پیدا گشت از عین بلا
&&
زان حلاوت شد عبارت ما قلی
\\
هر عبارت خود نشان حالتیست
&&
حال چون دست و عبارت آلتیست
\\
آلت زرگر به دست کفشگر
&&
همچو دانهٔ کشت کرده ریگ در
\\
و آلت اسکاف پیش برزگر
&&
پیش سگ که استخوان در پیش خر
\\
بود انا الحق در لب منصور نور
&&
بود انا الله در لب فرعون زور
\\
شد عصا اندر کف موسی گوا
&&
شد عصا اندر کف ساحر هبا
\\
زین سبب عیسی بدان همراه خود
&&
در نیاموزید آن اسم صمد
\\
کو نداند نقص بر آلت نهد
&&
سنگ بر گل زن تو آتش کی جهد
\\
دست و آلت همچو سنگ و آهنست
&&
جفت باید جفت شرط زادنست
\\
آنک بی جفتست و بی آلت یکیست
&&
در عدد شکست و آن یک بی‌شکیست
\\
آنک دو گفت و سه گفت و بیش ازین
&&
متفق باشند در واحد یقین
\\
احولی چون دفع شد یکسان شوند
&&
دو سه گویان هم یکی گویان شوند
\\
گر یکی گویی تو در میدان او
&&
گرد بر می‌گرد از چوگان او
\\
گوی آنگه راست و بی نقصان شود
&&
کو ز زخم دست شه رقصان شود
\\
گوش دار ای احول اینها را بهوش
&&
داروی دیده بکش از راه گوش
\\
پس کلام پاک در دلهای کور
&&
می‌نپاید می‌رود تا اصل نور
\\
وان فسون دیو در دلهای کژ
&&
می‌رود چون کفش کژ در پای کژ
\\
گرچه حکمت را به تکرار آوری
&&
چون تو نااهلی شود از تو بری
\\
ورچه بنویسی نشانش می‌کنی
&&
ورچه می‌لافی بیانش می‌کنی
\\
او ز تو رو در کشد ای پر ستیز
&&
بندها را بگسلد وز تو گریز
\\
ور نخوانی و ببیند سوز تو
&&
علم باشد مرغ دست‌آموز تو
\\
او نپاید پیش هر نااوستا
&&
همچو طاووسی به خانهٔ روستا
\\
\end{longtable}
\end{center}
