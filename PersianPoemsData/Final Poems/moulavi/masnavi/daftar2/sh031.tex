\begin{center}
\section*{بخش ۳۱ - ظاهر شدن فضل و زیرکی لقمان پیش امتحان کنندگان}
\label{sec:sh031}
\addcontentsline{toc}{section}{\nameref{sec:sh031}}
\begin{longtable}{l p{0.5cm} r}
هر طعامی کوریدندی بوی
&&
کس سوی لقمان فرستادی ز پی
\\
تا که لقمان دست سوی آن برد
&&
قاصدا تا خواجه پس‌خوردش خورد
\\
سؤر او خوردی و شور انگیختی
&&
هر طعامی کو نخوردی ریختی
\\
ور بخوردی بی دل و بی اشتها
&&
این بود پیوندی بی انتها
\\
خربزه آورده بودند ارمغان
&&
گفت رو فرزند لقمان را بخوان
\\
چون برید و داد او را یک برین
&&
همچو شکر خوردش و چون انگبین
\\
از خوشی که خورد داد او را دوم
&&
تا رسید آن گرچها تا هفدهم
\\
ماند گرچی گفت این را من خورم
&&
تا چه شیرین خربزه‌ست این بنگرم
\\
او چنین خوش می‌خورد کز ذوق او
&&
طبعها شد مشتهی و لقمه‌جو
\\
چون بخورد از تلخیش آتش فروخت
&&
هم زبان کرد آبله هم حلق سوخت
\\
ساعتی بی‌خود شد از تلخی آن
&&
بعد از آن گفتش که ای جان و جهان
\\
نوش چون کردی تو چندین زهر را
&&
لطف چون انگاشتی این قهر را
\\
این چه صبرست این صبوری ازچه روست
&&
یا مگر پیش تو این جانت عدوست
\\
چون نیاوردی به حیلت حجتی
&&
که مرا عذریست بس کن ساعتی
\\
گفت من از دست نعمت‌بخش تو
&&
خورده‌ام چندان که از شرمم دوتو
\\
شرمم آمد که یکی تلخ از کفت
&&
من ننوشم ای تو صاحب‌معرفت
\\
چون همه اجزام از انعام تو
&&
رسته‌اند و غرق دانه و دام تو
\\
گر ز یک تلخی کنم فریاد و داد
&&
خاک صد ره بر سر اجزام باد
\\
لذت دست شکربخشت بداشت
&&
اندرین بطیخ تلخی کی گذاشت
\\
از محبت تلخها شیرین شود
&&
از محبت مسها زرین شود
\\
از محبت دردها صافی شود
&&
از محبت دردها شافی شود
\\
از محبت مرده زنده می‌کنند
&&
از محبت شاه بنده می‌کنند
\\
این محبت هم نتیجهٔ دانشست
&&
کی گزافه بر چنین تختی نشست
\\
دانش ناقص کجا این عشق زاد
&&
عشق زاید ناقص اما بر جماد
\\
بر جمادی رنگ مطلوبی چو دید
&&
از صفیری بانگ محبوبی شنید
\\
دانش ناقص نداند فرق را
&&
لاجرم خورشید داند برق را
\\
چونک ملعون خواند ناقص را رسول
&&
بود در تاویل نقصان عقول
\\
زانک ناقص‌تن بود مرحوم رحم
&&
نیست بر مرحوم لایق لعن و زخم
\\
نقص عقلست آن که بد رنجوریست
&&
موجب لعنت سزای دوریست
\\
زانک تکمیل خردها دور نیست
&&
لیک تکمیل بدن مقدور نیست
\\
کفر و فرعونی هر گبر بعید
&&
جمله از نقصان عقل آمد پدید
\\
بهر نقصان بدن آمد فرج
&&
در نبی که ما علی الاعمی حرج
\\
برق آفل باشد و بس بی وفا
&&
آفل از باقی ندانی بی صفا
\\
برق خندد بر کی می‌خندد بگو
&&
بر کسی که دل نهد بر نور او
\\
نورهای چرخ ببریده‌پیست
&&
آن چو لا شرقی و لا غربی کیست
\\
برق را خو یخطف الابصار دان
&&
نور باقی را همه انصار دان
\\
بر کف دریا فرس را راندن
&&
نامه‌ای در نور برقی خواندن
\\
از حریصی عاقبت نادیدنست
&&
بر دل و بر عقل خود خندیدنست
\\
عاقبت بینست عقل از خاصیت
&&
نفس باشد کو نبیند عاقبت
\\
عقل کو مغلوب نفس او نفس شد
&&
مشتری مات زحل شد نحس شد
\\
هم درین نحسی بگردان این نظر
&&
در کسی که کرد نحست در نگر
\\
آن نظر که بنگرد این جر و مد
&&
او ز نحسی سوی سعدی نقب زد
\\
زان همی‌گرداندت حالی به حال
&&
ضد به ضد پیداکنان در انتقال
\\
تا که خوفت زاید از ذات الشمال
&&
لذت ذات الیمین یرجی الرجال
\\
تا دو پر باشی که مرغ یک پره
&&
عاجز آید از پریدن ای سره
\\
یا رها کن تا نیایم در کلام
&&
یا بده دستور تا گویم تمام
\\
ورنه این خواهی نه آن فرمان تراست
&&
کس چه داند مر ترا مقصد کجاست
\\
جان ابراهیم باید تا به نور
&&
بیند اندر نار فردوس و قصور
\\
پایه پایه بر رود بر ماه و خور
&&
تا نماند همچو حلقه بند در
\\
چون خلیل از آسمان هفتمین
&&
بگذرد که لا احب الافلین
\\
این جهان تن غلط‌انداز شد
&&
جز مر آن را کو ز شهوت باز شد
\\
\end{longtable}
\end{center}
