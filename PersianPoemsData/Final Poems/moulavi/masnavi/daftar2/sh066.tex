\begin{center}
\section*{بخش ۶۶ - باز جواب گفتن ابلیس معاویه را}
\label{sec:sh066}
\addcontentsline{toc}{section}{\nameref{sec:sh066}}
\begin{longtable}{l p{0.5cm} r}
گفت ابلیسش گشای این عقده‌ها
&&
من محکم قلب را و نقد را
\\
امتحان شیر و کلبم کرد حق
&&
امتحان نقد و قلبم کرد حق
\\
قلب را من کی سیه‌رو کرده‌ام
&&
صیرفی‌ام قیمت او کرده‌ام
\\
نیکوان را رهنمایی می‌کنم
&&
شاخه‌های خشک را بر می‌کنم
\\
این علفها می‌نهم از بهر چیست
&&
تا پدید آید که حیوان جنس کیست
\\
گرگ از آهو چو زاید کودکی
&&
هست در گرگیش و آهویی شکی
\\
تو گیاه و استخوان پیشش بریز
&&
تا کدامین سو کند او گام تیز
\\
گر به سوی استخوان آید سگست
&&
ور گیا خواهد یقین آهو رگست
\\
قهر و لطفی جفت شد با همدگر
&&
زاد ازین هر دو جهانی خیر و شر
\\
تو گیاه و استخوان را عرضه کن
&&
قوت نفس و قوت جان را عرضه کن
\\
گر غذای نفس جوید ابترست
&&
ور غذای روح خواهد سرورست
\\
گر کند او خدمت تن هست خر
&&
ور رود در بحر جان یابد گهر
\\
گرچه این دو مختلف خیر و شرند
&&
لیک این هر دو به یک کار اندرند
\\
انبیا طاعات عرضه می‌کنند
&&
دشمنان شهوات عرضه می‌کنند
\\
نیک را چون بد کنم یزدان نیم
&&
داعیم من خالق ایشان نیم
\\
خوب را من زشت سازم رب نه‌ام
&&
زشت را و خوب را آیینه‌ام
\\
سوخت هندو آینه از درد را
&&
کین سیه‌رو می‌نماید مرد را
\\
گفت آیینه گناه از من نبود
&&
جرم او را نه که روی من زدود
\\
او مرا غماز کرد و راست‌گو
&&
تا بگویم زشت کو و خوب کو
\\
من گواهم بر گوا زندان کجاست
&&
اهل زندان نیستم ایزد گواست
\\
هر کجا بینم نهال میوه‌دار
&&
تربیتها می‌کنم من دایه‌وار
\\
هر کجا بینم درخت تلخ و خشک
&&
می‌برم من تا رهد از پشک مشک
\\
خشک گوید باغبان را کای فتی
&&
مر مرا چه می‌بری سر بی خطا
\\
باغبان گوید خمش ای زشت‌خو
&&
بس نباشد خشکی تو جرم تو
\\
خشک گوید راستم من کژ نیم
&&
تو چرا بی‌جرم می‌بری پیم
\\
باغبان گوید اگر مسعودیی
&&
کاشکی کژ بودیی تر بودیی
\\
جاذب آب حیاتی گشتیی
&&
اندر آب زندگی آغشتیی
\\
تخم تو بد بوده است و اصل تو
&&
با درخت خوش نبوده وصل تو
\\
شاخ تلخ ار با خوشی وصلت کند
&&
آن خوشی اندر نهادش بر زند
\\
\end{longtable}
\end{center}
