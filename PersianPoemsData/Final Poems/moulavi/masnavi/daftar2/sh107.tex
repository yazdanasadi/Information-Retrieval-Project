\begin{center}
\section*{بخش ۱۰۷ - جواب اشکال}
\label{sec:sh107}
\addcontentsline{toc}{section}{\nameref{sec:sh107}}
\begin{longtable}{l p{0.5cm} r}
این بداند کانک اهل خاطرست
&&
غایب آفاق او را حاضرست
\\
پیش مریم حاضر آید در نظر
&&
مادر یحیی که دورست از بصر
\\
دیده‌ها بسته ببیند دوست را
&&
چون مشبک کرده باشد پوست را
\\
ور ندیدش نه از برون نه از اندرون
&&
از حکایت گیر معنی ای زبون
\\
نی چنان کافسانه‌ها بشنیده بود
&&
همچو شین بر نقش آن چفسیده بود
\\
تا همی‌گفت آن کلیله بی‌زبان
&&
چون سخن نوشد ز دمنه بی بیان
\\
ور بدانستند لحن همدگر
&&
فهم آن چون مرد بی نطقی بشر
\\
در میان شیر و گاو آن دمنه چون
&&
شد رسول و خواند بر هر دو فسون
\\
چون وزیر شیر شد گاو نبیل
&&
چون ز عکس ماه ترسان گشت پیل
\\
این کلیله و دمنه جمله افتراست
&&
ورنه کی با زاغ لک‌لک را مریست
\\
ای برادر قصه چون پیمانه‌ایست
&&
معنی اندر وی مثال دانه‌ایست
\\
دانهٔ معنی بگیرد مرد عقل
&&
ننگرد پیمانه را گر گشت نقل
\\
ماجرای بلبل و گل گوش دار
&&
گر چه گفتی نیست آنجا آشکار
\\
\end{longtable}
\end{center}
