\begin{center}
\section*{بخش ۲۰ - ملامت کردن مردم شخصی را کی مادرش را کشت به تهمت}
\label{sec:sh020}
\addcontentsline{toc}{section}{\nameref{sec:sh020}}
\begin{longtable}{l p{0.5cm} r}
آن یکی از خشم مادر را بکشت
&&
هم به زخم خنجر و هم زخم مشت
\\
آن یکی گفتش که از بد گوهری
&&
یاد ناوردی تو حق مادری
\\
هی تو مادر را چرا کشتی بگو
&&
او چه کرد آخر بگو ای زشت‌خو
\\
گفت کاری کرد کان عار ویست
&&
کشتمش کان خاک ستار ویست
\\
گفت آن کس را بکش ای محتشم
&&
گفت پس هر روز مردی را کشم
\\
کشتم او را رستم از خونهای خلق
&&
نای او برم بهست از نای خلق
\\
نفس تست آن مادر بد خاصیت
&&
که فساد اوست در هر ناحیت
\\
هین بکش او را که بهر آن دنی
&&
هر دمی قصد عزیزی می‌کنی
\\
از وی این دنیای خوش بر تست تنگ
&&
از پی او با حق و با خلق جنگ
\\
نفس کشتی باز رستی ز اعتذار
&&
کس ترا دشمن نماند در دیار
\\
گر شکال آرد کسی بر گفت ما
&&
از برای انبیا و اولیا
\\
کانبیا را نی که نفس کشته بود
&&
پس چراشان دشمنان بود و حسود
\\
گوش نه تو ای طلب‌کار صواب
&&
بشنو این اشکال و شبهت را جواب
\\
دشمن خود بوده‌اند آن منکران
&&
زخم بر خود می‌زدند ایشان چنان
\\
دشمن آن باشد که قصد جان کند
&&
دشمن آن نبود که خود جان می‌کند
\\
نیست خفاشک عدو آفتاب
&&
او عدو خویش آمد در حجاب
\\
تابش خورشید او را می‌کشد
&&
رنج او خورشید هرگز کی کشد
\\
دشمن آن باشد کزو آید عذاب
&&
مانع آید لعل را از آفتاب
\\
مانع خویشند جملهٔ کافران
&&
از شعاع جوهر پیغامبران
\\
کی حجاب چشم آن فردند خلق
&&
چشم خود را کور و کژ کردند خلق
\\
چون غلام هندوی کو کین کشد
&&
از ستیزهٔ خواجه خود را می‌کشد
\\
سرنگون می‌افتد از بام سرا
&&
تا زیانی کرده باشد خواجه را
\\
گر شود بیمار دشمن با طبیب
&&
ور کند کودک عداوت با ادیب
\\
در حقیقت ره‌زن جان خودند
&&
راه عقل و جان خود را خود زدند
\\
گازری گر خشم گیرد ز آفتاب
&&
ماهیی گر خشم می‌گیرد ز آب
\\
تو یکی بنگر کرا دارد زیان
&&
عاقبت که بود سیاه‌اختر از آن
\\
گر ترا حق آفریند زشت‌رو
&&
هان مشو هم زشت‌رو هم زشت‌خو
\\
ور برد کفشت مرو در سنگ‌لاخ
&&
ور دو شاخستت مشو تو چار شاخ
\\
تو حسودی کز فلان من کمترم
&&
می‌فزاید کمتری در اخترم
\\
خود حسد نقصان و عیبی دیگرست
&&
بلک از جمله کمیها بترست
\\
آن بلیس از ننگ و عار کمتری
&&
خویش را افکند در صد ابتری
\\
از حسد می‌خواست تا بالا بود
&&
خود چه بالا بلک خون‌پالا بود
\\
آن ابوجهل از محمد ننگ داشت
&&
وز حسد خود را به بالا می‌فراشت
\\
بوالحکم نامش بد و بوجهل شد
&&
ای بسا اهل از حسد نااهل شد
\\
من ندیدم در جهان جست و جو
&&
هیچ اهلیت به از خوی نکو
\\
انبیا را واسطه زان کرد حق
&&
تا پدید آید حسدها در قلق
\\
زانک کس را از خدا عاری نبود
&&
حاسد حق هیچ دیاری نبود
\\
آن کسی کش مثل خود پنداشتی
&&
زان سبب با او حسد برداشتی
\\
چون مقرر شد بزرگی رسول
&&
پس حسد ناید کسی را از قبول
\\
پس بهر دوری ولیی قایمست
&&
تا قیامت آزمایش دایمست
\\
هر که را خوی نکو باشد برست
&&
هر کسی کو شیشه‌دل باشد شکست
\\
پس امام حی قایم آن ولیست
&&
خواه از نسل عمر خواه از علیست
\\
مهدی و هادی ویست ای راه‌جو
&&
هم نهان و هم نشسته پیش رو
\\
او چو نورست و خرد جبریل اوست
&&
وان ولی کم ازو قندیل اوست
\\
وانک زین قندیل کم مشکات ماست
&&
نور را در مرتبه ترتیبهاست
\\
زانک هفصد پرده دارد نور حق
&&
پرده‌های نور دان چندین طبق
\\
از پس هر پرده قومی را مقام
&&
صف صف‌اند این پرده‌هاشان تا امام
\\
اهل صف آخرین از ضعف خویش
&&
چشمشان طاقت ندارد نور بیش
\\
وان صف پیش از ضعیفی بصر
&&
تاب نارد روشنایی بیشتر
\\
روشنایی کو حیات اولست
&&
رنج جان و فتنهٔ این احولست
\\
احولیها اندک اندک کم شود
&&
چون ز هفصد بگذرد او یم شود
\\
آتشی که اصلاح آهن یا زرست
&&
کی صلاح آبی و سیب ترست
\\
سیب و آبی خامیی دارد خفیف
&&
نی چو آهن تابشی خواهد لطیف
\\
لیک آهن را لطیف آن شعله‌هاست
&&
کو جذوب تابش آن اژدهاست
\\
هست آن آهن فقیر سخت‌کش
&&
زیر پتک و آتش است او سرخ و خوش
\\
حاجب آتش بود بی واسطه
&&
در دل آتش رود بی رابطه
\\
بی‌حجاب آب و فرزندان آب
&&
پختگی ز آتش نیابند و خطاب
\\
واسطه دیگی بود یا تابه‌ای
&&
همچو پا را در روش پاتابه‌ای
\\
یا مکانی در میان تا آن هوا
&&
می‌شود سوزان و می‌آرد بما
\\
پس فقیر آنست کو بی واسطه‌ست
&&
شعله‌ها را با وجودش رابطه‌ست
\\
پس دل عالم ویست ایرا که تن
&&
می‌رسد از واسطهٔ این دل بفن
\\
دل نباشد تن چه داند گفت و گو
&&
دل نجوید تن چه داند جست و جو
\\
پس نظرگاه شعاع آن آهنست
&&
پس نظرگاه خدا دل نه تنست
\\
بس مثال و شرح خواهد این کلام
&&
لیک ترسم تا نلغزد وهم عام
\\
تا نگردد نیکوی ما بدی
&&
اینک گفتم هم نبد جز بی‌خودی
\\
پای کژ را کفش کژ بهتر بود
&&
مر گدا را دستگه بر در بود
\\
\end{longtable}
\end{center}
