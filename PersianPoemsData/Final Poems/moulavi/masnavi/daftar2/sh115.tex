\begin{center}
\section*{بخش ۱۱۵ - حیران شدن حاجیان در کرامات آن زاهد  کی در بادیه تنهاش یافتند}
\label{sec:sh115}
\addcontentsline{toc}{section}{\nameref{sec:sh115}}
\begin{longtable}{l p{0.5cm} r}
زاهدی بد در میان بادیه
&&
در عبادت غرق چون عبادیه
\\
حاجیان آنجا رسیدند از بلاد
&&
دیده‌شان بر زاهد خشک اوفتاد
\\
جای زاهد خشک بود او ترمزاج
&&
از سموم بادیه بودش علاج
\\
حاجیان حیران شدند از وحدتش
&&
و آن سلامت در میان آفتش
\\
در نماز استاده بد بر روی ریگ
&&
ریگ کز تفش بجوشد آب دیگ
\\
گفتیی سرمست در سبزه و گلست
&&
یا سواره بر براق و دلدلست
\\
یا که پایش بر حریر و حله‌هاست
&&
یا سموم او را به از باد صباست
\\
پس بماندند آن جماعت با نیاز
&&
تا شود درویش فارغ از نماز
\\
چون ز استغراق باز آمد فقیر
&&
زان جماعت زندهٔ روشن‌ضمیر
\\
دید کآبش می‌چکید از دست و رو
&&
جامه‌اش تر بود از آثار وضو
\\
پس بپرسیدش که آبت از کجاست
&&
دست را بر داشت کز سوی سماست
\\
گفت هر گاهی که خواهی می‌رسد
&&
بی ز چاه و بی ز حبل من مسد
\\
مشکل ما حل کن ای سلطان دین
&&
تا ببخشد حال تو ما را یقین
\\
وا نما سری ز اسرارت بما
&&
تا ببریم از میان زنارها
\\
چشم را بگشود سوی آسمان
&&
که اجابت کن دعای حاجیان
\\
رزق‌جویی را ز بالا خوگرم
&&
تو ز بالا بر گشودستی درم
\\
ای نموده تو مکان از لامکان
&&
فی السماء رزقکم کرده عیان
\\
در میان این مناجات ابر خوش
&&
زود پیدا شد چو پیل آب‌کش
\\
همچو آب از مشک باریدن گرفت
&&
در گو و در غارها مسکن گرفت
\\
ابر می‌بارید چون مشک اشکها
&&
حاجیان جمله گشاده مشکها
\\
یک جماعت زان عجایب کارها
&&
می‌بریدند از میان زنارها
\\
قوم دیگر را یقین در ازدیاد
&&
زین عجب والله اعلم بالرشاد
\\
قوم دیگر ناپذیرا ترش و خام
&&
ناقصان سرمدی تم الکلام
\\
\end{longtable}
\end{center}
