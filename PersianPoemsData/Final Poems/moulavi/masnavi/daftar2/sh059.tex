\begin{center}
\section*{بخش ۵۹ - دوم بار در سخن کشیدن سایل آن بزرگ را تا حال او معلوم‌تر گردد}
\label{sec:sh059}
\addcontentsline{toc}{section}{\nameref{sec:sh059}}
\begin{longtable}{l p{0.5cm} r}
گفت آن طالب که آخر یک نفس
&&
ای سواره بر نی این سو ران فرس
\\
راند سوی او که هین زوتر بگو
&&
کاسپ من بس توسنست و تندخو
\\
تا لگد بر تو نکوبد زود باش
&&
از چه می‌پرسی بیانش کن تو فاش
\\
او مجال راز دل گفتن ندید
&&
زو برون شو کرد و در لاغش کشید
\\
گفت می‌خواهم درین کوچه زنی
&&
کیست لایق از برای چون منی
\\
گفت سه گونه زن‌اند اندر جهان
&&
آن دو رنج و این یکی گنج روان
\\
آن یکی را چون بخواهی کل تراست
&&
وآن دگر نیمی ترا نیمی جداست
\\
وآن سیم هیچ او ترا نبود بدان
&&
این شنودی دور شو رفتم روان
\\
تا ترا اسپم نپراند لگد
&&
که بیفتی بر نخیزی تا ابد
\\
شیخ راند اندر میان کودکان
&&
بانگ زد بار دگر او را جوان
\\
که بیا آخر بگو تفسیر این
&&
این زنان سه نوع گفتی بر گزین
\\
راند سوی او و گفتش بکر خاص
&&
کل ترا باشد ز غم یابی خلاص
\\
وانک نیمی آن تو بیوه بود
&&
وانک هیچست آن عیال با ولد
\\
چون ز شوی اولش کودک بود
&&
مهر و کل خاطرش آن سو رود
\\
دور شو تا اسپ نندازد لگد
&&
سم اسپ توسنم بر تو رسد
\\
های هویی کرد شیخ باز راند
&&
کودکان را باز سوی خویش خواند
\\
باز بانگش کرد آن سایل بیا
&&
یک سؤالم ماند ای شاه کیا
\\
باز راند این سو بگو زوتر چه بود
&&
که ز میدان آن بچه گویم ربود
\\
گفت ای شه با چنین عقل و ادب
&&
این چه شیدست این چه فعلست ای عجب
\\
تو ورای عقل کلی در بیان
&&
آفتابی در جنون چونی نهان
\\
گفت این اوباش رایی می‌زنند
&&
تا درین شهر خودم قاضی کنند
\\
دفع می‌گفتم مرا گفتند نی
&&
نیست چون تو عالمی صاحب فنی
\\
با وجود تو حرامست و خبیث
&&
که کم از تو در قضا گوید حدیث
\\
در شریعت نیست دستوری که ما
&&
کمتر از تو شه کنیم و پیشوا
\\
زین ضرورت گیج و دیوانه شدم
&&
لیک در باطن همانم که بدم
\\
عقل من گنجست و من ویرانه‌ام
&&
گنج اگر پیدا کنم دیوانه‌ام
\\
اوست دیوانه که دیوانه نشد
&&
این عسس را دید و در خانه نشد
\\
دانش من جوهر آمد نه عرض
&&
این بهایی نیست بهر هر غرض
\\
کان قندم نیستان شکرم
&&
هم زمن می‌روید و من می‌خورم
\\
علم تقلیدی و تعلیمیست آن
&&
کز نفور مستمع دارد فغان
\\
چون پی دانه نه بهر روشنیست
&&
همچو طالب‌علم دنیای دنیست
\\
طالب علمست بهر عام و خاص
&&
نه که تا یابد ازین عالم خلاص
\\
همچو موشی هر طرف سوراخ کرد
&&
چونک نورش راند از در گفت برد
\\
چونک سوی دشت و نورش ره نبود
&&
هم در آن ظلمات جهدی می‌نمود
\\
گر خدایش پر دهد پر خرد
&&
برهد از موشی و چون مرغان پرد
\\
ور نجوید پر بماند زیر خاک
&&
ناامید از رفتن راه سماک
\\
علم گفتاری که آن بی جان بود
&&
عاشق روی خریداران بود
\\
گرچه باشد وقت بحث علم زفت
&&
چون خریدارش نباشد مرد و رفت
\\
مشتری من خدایست او مرا
&&
می‌کشد بالا که الله اشتری
\\
خونبهای من جمال ذوالجلال
&&
خونبهای خود خورم کسب حلال
\\
این خریداران مفلس را بهل
&&
چه خریداری کند یک مشت گل
\\
گل مخور گل را مخر گل را مجو
&&
زانک گل خوارست دایم زردرو
\\
دل بخور تا دایما باشی جوان
&&
از تجلی چهره‌ات چون ارغوان
\\
یا رب این بخشش نه حد کار ماست
&&
لطف تو لطف خفی را خود سزاست
\\
دست گیر از دست ما ما را بخر
&&
پرده را بر دار و پردهٔ ما مدر
\\
باز خر ما را ازین نفس پلید
&&
کاردش تا استخوان ما رسید
\\
از چو ما بیچارگان این بند سخت
&&
کی گشاید ای شه بی‌تاج و تخت
\\
این چنین قفل گران را ای ودود
&&
کی تواند جز که فضل تو گشود
\\
ما ز خود سوی تو گردانیم سر
&&
چون توی از ما به ما نزدیکتر
\\
این دعا هم بخشش و تعلیم تست
&&
گرنه در گلخن گلستان از چه رست
\\
در میان خون و روده فهم و عقل
&&
جز ز اکرام تو نتوان کرد نقل
\\
از دو پاره پیه این نور روان
&&
موج نورش می‌زند بر آسمان
\\
گوشت‌پاره که زبان آمد ازو
&&
می‌رود سیلاب حکمت همچو جو
\\
سوی سوراخی که نامش گوشهاست
&&
تا بباغ جان که میوه‌ش هوشهاست
\\
شاه‌راه باغ جانها شرع اوست
&&
باغ و بستانهای عالم فرع اوست
\\
اصل و سرچشمهٔ خوشی آنست آن
&&
زود تجری تحتها الانهار خوان
\\
\end{longtable}
\end{center}
