\begin{center}
\section*{بخش ۵۶ - به حیلت در سخن آوردن سایل آن بزرگ را کی خود را دیوانه ساخته بود}
\label{sec:sh056}
\addcontentsline{toc}{section}{\nameref{sec:sh056}}
\begin{longtable}{l p{0.5cm} r}
آن یکی می‌گفت خواهم عاقلی
&&
مشورت آرم بدو در مشکلی
\\
آن یکی گفتش که اندر شهر ما
&&
نیست عاقل جز که آن مجنون‌نما
\\
بر نیی گشته سواره نک فلان
&&
می‌دواند در میان کودکان
\\
صاحب رایست و آتش‌پاره‌ای
&&
آسمان قدرست و اخترباره‌ای
\\
فر او کروبیان را جان شدست
&&
او درین دیوانگی پنهان شدست
\\
لیک هر دیوانه را جان نشمری
&&
سر منه گوساله را چون سامری
\\
چون ولیی آشکارا با تو گفت
&&
صد هزاران غیب و اسرار نهفت
\\
مر ترا آن فهم و آن دانش نبود
&&
وا ندانستی تو سرگین را ز عود
\\
از جنون خود را ولی چون پرده ساخت
&&
مر ورا ای کور کی خواهی شناخت
\\
گر ترا بازست آن دیدهٔ یقین
&&
زیر هر سنگی یکی سرهنگ بین
\\
پیش آن چشمی که باز و رهبرست
&&
هر گلیمی را کلیمی در برست
\\
مر ولی را هم ولی شهره کند
&&
هر که را او خواست با بهره کند
\\
کس نداند از خرد او را شناخت
&&
چونک او مر خویش را دیوانه ساخت
\\
چون بدزدد دزد بینایی ز کور
&&
هیچ یابد دزد را او در عبور
\\
کور نشناسد که دزد او که بود
&&
گرچه خود بر وی زند دزد عنود
\\
چون گزد سگ کور صاحب‌ژنده را
&&
کی شناسد آن سگ درنده را
\\
\end{longtable}
\end{center}
