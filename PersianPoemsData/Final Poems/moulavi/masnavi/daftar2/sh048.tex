\begin{center}
\section*{بخش ۴۸ - رفتن مصطفی علیه السلام به عیادت صحابی و بیان فایدهٔ عیادت}
\label{sec:sh048}
\addcontentsline{toc}{section}{\nameref{sec:sh048}}
\begin{longtable}{l p{0.5cm} r}
از صحابه خواجه‌ای بیمار شد
&&
واندر آن بیماریش چون تار شد
\\
مصطفی آمد عیادت سوی او
&&
چون همه لطف و کرم بد خوی او
\\
در عیادت رفتن تو فایده‌ست
&&
فایدهٔ آن باز با تو عایده‌ست
\\
فایدهٔ اول که آن شخص علیل
&&
بوک قطبی باشد و شاه جلیل
\\
چون دو چشم دل نداری ای عنود
&&
که نمی‌دانی تو هیزم را ز عود
\\
چونک گنجی هست در عالم مرنج
&&
هیچ ویران را مدان خالی ز گنج
\\
قصد هر درویش می‌کن از گزاف
&&
چون نشان یابی بجد می‌کن طواف
\\
چون ترا آن چشم باطن‌بین نبود
&&
گنج می‌پندار اندر هر وجود
\\
ور نباشد قطب یار ره بود
&&
شه نباشد فارس اسپه بود
\\
پس صلهٔ یاران ره لازم شمار
&&
هر که باشد گر پیاده گر سوار
\\
ور عدو باشد همین احسان نکوست
&&
که باحسان بس عدو گشتست دوست
\\
ور نگردد دوست کینش کم شود
&&
زانک احسان کینه را مرهم شود
\\
بس فواید هست غیر این ولیک
&&
از درازی خایفم ای یار نیک
\\
حاصل این آمد که یار جمع باش
&&
همچو بتگر از حجر یاری تراش
\\
زانک انبوهی و جمع کاروان
&&
ره‌زنان را بشکند پشت و سنان
\\
\end{longtable}
\end{center}
