\begin{center}
\section*{بخش ۶۱ - وصیت کردن پیغامبر علیه السلام مر آن بیمار را و دعا آموزانیدنش}
\label{sec:sh061}
\addcontentsline{toc}{section}{\nameref{sec:sh061}}
\begin{longtable}{l p{0.5cm} r}
گفت پیغامبر مر آن بیمار را
&&
این بگو کای سهل‌کن دشوار را
\\
آتنا فی دار دنیانا حسن
&&
آتنا فی دار عقبانا حسن
\\
راه را بر ما چو بستان کن لطیف
&&
منزل ما خود تو باشی ای شریف
\\
مؤمنان در حشر گویند ای ملک
&&
نی که دوزخ بود راه مشترک
\\
مؤمن و کافر برو یابد گذار
&&
ما ندیدیم اندرین ره دود و نار
\\
نک بهشت و بارگاه آمنی
&&
پس کجا بود آن گذرگاه دنی
\\
پس ملک گوید که آن روضهٔ خضر
&&
که فلان جا دیده‌اید اندر گذر
\\
دوزخ آن بود و سیاستگاه سخت
&&
بر شما شد باغ و بستان و درخت
\\
چون شما این نفس دوزخ‌خوی را
&&
آتشی گبر فتنه‌جوی را
\\
جهدها کردید و او شد پر صفا
&&
نار را کشتید از بهر خدا
\\
آتش شهوت که شعله می‌زدی
&&
سبزهٔ تقوی شد و نور هدی
\\
آتش خشم از شما هم حلم شد
&&
ظلمت جهل از شما هم علم شد
\\
آتش حرص از شما ایثار شد
&&
و آن حسد چون خار بد گلزار شد
\\
چون شما این جمله آتشهای خویش
&&
بهر حق کشتید جمله پیش پیش
\\
نفس ناری را چو باغی ساختید
&&
اندرو تخم وفا انداختید
\\
بلبلان ذکر و تسبیح اندرو
&&
خوش سرایان در چمن بر طرف جو
\\
داعی حق را اجابت کرده‌اید
&&
در جحیم نفس آب آورده‌اید
\\
دوزخ ما نیز در حق شما
&&
سبزه گشت و گلشن و برگ و نوا
\\
چیست احسان را مکافات ای پسر
&&
لطف و احسان و ثواب معتبر
\\
نی شما گفتید ما قربانییم
&&
پیش اوصاف بقا ما فانییم
\\
ما اگر قلاش و گر دیوانه‌ایم
&&
مست آن ساقی و آن پیمانه‌ایم
\\
بر خط و فرمان او سر می‌نهیم
&&
جان شیرین را گروگان می‌دهیم
\\
تا خیال دوست در اسرار ماست
&&
چاکری و جانسپاری کار ماست
\\
هر کجا شمع بلا افروختند
&&
صد هزاران جان عاشق سوختند
\\
عاشقانی کز درون خانه‌اند
&&
شمع روی یار را پروانه‌اند
\\
ای دل آنجا رو که با تو روشنند
&&
وز بلاها مر ترا چون جوشنند
\\
بر جنایاتت مواسا می‌کنند
&&
در میان جان ترا جا می‌کنند
\\
زان میان جان ترا جا می‌کنند
&&
تا ترا پر باده چون جا می‌کنند
\\
در میان جان ایشان خانه گیر
&&
در فلک خانه کن ای بدر منیر
\\
چون عطارد دفتر دل وا کنند
&&
تا که بر تو سرها پیدا کنند
\\
پیش خویشان باش چون آواره‌ای
&&
بر مه کامل زن ار مه پاره‌ای
\\
جزو را از کل خود پرهیز چیست
&&
با مخالف این همه آمیز چیست
\\
جنس را بین نوع گشته در روش
&&
غیبها بین عین گشته در رهش
\\
تا چو زن عشوه خری ای بی‌خرد
&&
از دروغ و عشوه کی یابی مدد
\\
چاپلوس و لفظ شیرین و فریب
&&
می‌ستانی می‌نهی چون زن به جیب
\\
مر ترا دشنام و سیلی شهان
&&
بهتر آید از ثنای گمرهان
\\
صفع شاهان خور مخور شهد خسان
&&
تا کسی گردی ز اقبال کسان
\\
زانک ازیشان خلعت و دولت رسد
&&
در پناه روح جان گردد جسد
\\
هر کجا بینی برهنه و بی‌نوا
&&
دان که او بگریختست از اوستا
\\
تا چنان گردد که می‌خواهد دلش
&&
آن دل کور بد بی‌حاصلش
\\
گر چنان گشتی که استا خواستی
&&
خویش را و خویش را آراستی
\\
هر که از استا گریزد در جهان
&&
او ز دولت می‌گریزد این بدان
\\
پیشه‌ای آموختی در کسب تن
&&
چنگ اندر پیشهٔ دینی بزن
\\
در جهان پوشیده گشتی و غنی
&&
چون برون آیی ازینجا چون کنی
\\
پیشه‌ای آموز کاندر آخرت
&&
اندر آید دخل کسب مغفرت
\\
آن جهان شهریست پر بازار و کسب
&&
تا نپنداری که کسب اینجاست حسب
\\
حق تعالی گفت کین کسب جهان
&&
پیش آن کسبست لعب کودکان
\\
همچو آن طفلی که بر طفلی تند
&&
شکل صحبت‌کن مساسی می‌کند
\\
کودکان سازند در بازی دکان
&&
سود نبود جز که تعبیر زمان
\\
شب شود در خانه آید گرسنه
&&
کودکان رفته بمانده یک تنه
\\
این جهان بازی‌گهست و مرگ شب
&&
باز گردی کیسه خالی پر تعب
\\
کسب دین عشقست و جذب اندرون
&&
قابلیت نور حق را ای حرون
\\
کسب فانی خواهدت این نفس خس
&&
چند کسب خس کنی بگذار بس
\\
نفس خس گر جویدت کسب شریف
&&
حیله و مکری بود آن را ردیف
\\
\end{longtable}
\end{center}
