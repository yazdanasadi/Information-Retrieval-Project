\begin{center}
\section*{بخش ۴۲ - تتمهٔ حکایت خرس و آن ابله کی بر وفای او اعتماد کرده بود}
\label{sec:sh042}
\addcontentsline{toc}{section}{\nameref{sec:sh042}}
\begin{longtable}{l p{0.5cm} r}
خرس هم از اژدها چون وا رهید
&&
وآن کرم زان مرد مردانه بدید
\\
چون سگ اصحاب کهف آن خرس زار
&&
شد ملازم در پی آن بردبار
\\
آن مسلمان سر نهاد از خستگی
&&
خرس حارس گشت از دل‌بستگی
\\
آن یکی بگذشت و گفتش حال چیست
&&
ای برادر مر ترا این خرس کیست
\\
قصه وا گفت و حدیث اژدها
&&
گفت بر خرسی منه دل ابلها
\\
دوستی ابله بتر از دشمنیست
&&
او بهر حیله که دانی راندنیست
\\
گفت والله از حسودی گفت این
&&
ورنه خرسی چه نگری این مهر بین
\\
گفت مهر ابلهان عشوه‌ده است
&&
این حسودی من از مهرش به است
\\
هی بیا با من بران این خرس را
&&
خرس را مگزین مهل هم‌جنس را
\\
گفت رو رو کار خود کن ای حسود
&&
گفت کارم این بد و رزقت نبود
\\
من کم از خرسی نباشم ای شریف
&&
ترک او کن تا منت باشم حریف
\\
بر تو دل می‌لرزدم ز اندیشه‌ای
&&
با چنین خرسی مرو در بیشه‌ای
\\
این دلم هرگز نلرزید از گزاف
&&
نور حقست این نه دعوی و نه لاف
\\
مؤمنم ینظر بنور الله شده
&&
هان و هان بگریز ازین آتشکده
\\
این همه گفت و به گوشش در نرفت
&&
بدگمانی مرد سدیست زفت
\\
دست او بگرفت و دست از وی کشید
&&
گفت رفتم چون نه‌ای یار رشید
\\
گفت رو بر من تو غمخواره مباش
&&
بوالفضولا معرفت کمتر تراش
\\
باز گفتش من عدوی تو نیم
&&
لطف باشد گر بیابی در پیم
\\
گفت خوابستم مرا بگذار و رو
&&
گفت آخر یار را منقاد شو
\\
تا بخسپی در پناه عاقلی
&&
در جوار دوستی صاحب‌دلی
\\
در خیال افتاد مرد از جد او
&&
خشمگین شد زود گردانید رو
\\
کین مگر قصد من آمد خونیست
&&
یا طمع دارد گدا و تونیست
\\
یا گرو بستست با یاران بدین
&&
که بترساند مرا زین همنشین
\\
خود نیامد هیچ از خبث سرش
&&
یک گمان نیک اندر خاطرش
\\
ظن نیکش جملگی بر خرس بود
&&
او مگر مر خرس را هم‌جنس بود
\\
عاقلی را از سگی تهمت نهاد
&&
خرس را دانست اهل مهر و داد
\\
\end{longtable}
\end{center}
