\begin{center}
\section*{بخش ۹۹ - گفتن عایشه رضی الله عنها مصطفی را علیه السلام کی تو بی مصلی بهر جا نماز می‌کنی چونست}
\label{sec:sh099}
\addcontentsline{toc}{section}{\nameref{sec:sh099}}
\begin{longtable}{l p{0.5cm} r}
عایشه روزی به پیغامبر بگفت
&&
یا رسول الله تو پیدا و نهفت
\\
هر کجا یابی نمازی می‌کنی
&&
می‌دود در خانه ناپاک و دنی
\\
گرچه می‌دانی که هر طفل پلید
&&
کرد مستعمل بهر جا که رسید
\\
گفت پیغامبر که از بهر مهان
&&
حق نجس را پاک گرداند بدان
\\
سجده‌گاهم را از آن رو لطف حق
&&
پاک گردانید تا هفتم طبق
\\
هان و هان ترک حسد کن با شهان
&&
ور نه ابلیسی شوی اندر جهان
\\
کو اگر زهری خورد شهدی شود
&&
تو اگر شهدی خوری زهری بود
\\
کو بدل گشت و بدل شد کار او
&&
لطف گشت و نور شد هر نار او
\\
قوت حق بود مر بابیل را
&&
ور نه مرغی چون کشد مر پیل را
\\
لشکری را مرغکی چندی شکست
&&
تا بدانی کان صلابت از حقست
\\
گر ترا وسواس آید زین قبیل
&&
رو بخوان تو سورهٔ اصحاب فیل
\\
ور کنی با او مری و همسری
&&
کافرم دان گر تو زیشان سر بری
\\
\end{longtable}
\end{center}
