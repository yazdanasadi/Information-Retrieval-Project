\begin{center}
\section*{بخش ۴۰ - اعتماد کردن بر تملق و وفای خرس}
\label{sec:sh040}
\addcontentsline{toc}{section}{\nameref{sec:sh040}}
\begin{longtable}{l p{0.5cm} r}
اژدهایی خرس را در می‌کشید
&&
شیر مردی رفت و فریادش رسید
\\
شیر مردانند در عالم مدد
&&
آن زمان کافغان مظلومان رسد
\\
بانگ مظلومان ز هر جا بشنوند
&&
آن طرف چون رحمت حق می‌دوند
\\
آن ستونهای خللهای جهان
&&
آن طبیبان مرضهای نهان
\\
محض مهر و داوری و رحمتند
&&
همچو حق بی علت و بی رشوتند
\\
این چه یاری می‌کنی یبکارگیش
&&
گوید از بهر غم و بیچارگیش
\\
مهربانی شد شکار شیرمرد
&&
در جهان دارو نجوید غیر درد
\\
هر کجا دردی دوا آنجا رود
&&
هر کجا پستیست آب آنجا دود
\\
آب رحمت بایدت رو پست شو
&&
وانگهان خور خمر رحمت مست شو
\\
رحمت اندر رحمت آمد تا به سر
&&
بر یکی رحمت فرو مای ای پسر
\\
چرخ را در زیر پا آر ای شجاع
&&
بشنو از فوق فلک بانگ سماع
\\
پنبهٔ وسواس بیرون کن ز گوش
&&
تا به گوشت آید از گردون خروش
\\
پاک کن دو چشم را از موی عیب
&&
تا ببینی باغ و سروستان غیب
\\
دفع کن از مغز و از بینی زکام
&&
تا که ریح الله در آید در مشام
\\
هیچ مگذار از تب و صفرا اثر
&&
تا بیابی از جهان طعم شکر
\\
داروی مردی کن و عنین مپوی
&&
تا برون آیند صد گون خوب‌روی
\\
کندهٔ تن را ز پای جان بکن
&&
تا کند جولان به گردت انجمن
\\
غل بخل از دست و گردن دور کن
&&
بخت نو در یاب در چرخ کهن
\\
ور نمی‌توانی به کعبهٔ لطف پر
&&
عرضه کن بیچارگی بر چاره‌گر
\\
زاری و گریه قوی سرمایه‌ایست
&&
رحمت کلی قوی‌تر دایه‌ایست
\\
دایه و مادر بهانه‌جو بود
&&
تا که کی آن طفل او گریان شود
\\
طفل حاجات شما را آفرید
&&
تا بنالید و شود شیرش پدید
\\
گفت ادعوا الله بی زاری مباش
&&
تا بجوشد شیرهای مهرهاش
\\
هوی هوی باد و شیرافشان ابر
&&
در غم ما اند یک ساعت تو صبر
\\
فی السماء رزقکم بشنیده‌ای
&&
اندرین پستی چه بر چفسیده‌ای
\\
ترس و نومیدیت دان آواز غول
&&
می‌کشد گوش تو تا قعر سفول
\\
هر ندایی که ترا بالا کشید
&&
آن ندا می‌دان که از بالا رسید
\\
هر ندایی که ترا حرص آورد
&&
بانگ گرگی دان که او مردم درد
\\
این بلندی نیست از روی مکان
&&
این بلندیهاست سوی عقل و جان
\\
هر سبب بالاتر آمد از اثر
&&
سنگ و آهن فایق آمد بر شرر
\\
آن فلانی فوق آن سرکش نشست
&&
گرچه در صورت به پهلویش نشست
\\
فوقی آنجاست از روی شرف
&&
جای دور از صدر باشد مستخف
\\
سنگ و آهن زین جهت که سابق است
&&
در عمل فوقی این دو لایق است
\\
وآن شرر از روی مقصودی خویش
&&
ز آهن و سنگست زین رو پیش و پیش
\\
سنگ و آهن اول و پایان شرر
&&
لیک این هر دو تنند و جان شرر
\\
در زمان شاخ از ثمر سابق‌ترست
&&
در هنر از شاخ او فایق‌ترست
\\
چونک مقصود از شجر آمد ثمر
&&
پس ثمر اول بود و آخر شجر
\\
خرس چون فریاد کرد از اژدها
&&
شیرمردی کرد از چنگش جدا
\\
حیلت و مردی به هم دادند پشت
&&
اژدها را او بدین قوت بکشت
\\
اژدها را هست قوت حیله نیست
&&
نیز فوق حیلهٔ تو حیله‌ایست
\\
حیلهٔ خود را چو دیدی باز رو
&&
کز کجا آمد سوی آغاز رو
\\
هر چه در پستیست آمد از علا
&&
چشم را سوی بلندی نه هلا
\\
روشنی بخشد نظر اندر علی
&&
گرچه اول خیرگی آرد بلی
\\
چشم را در روشنایی خوی کن
&&
گر نه خفاشی نظر آن سوی کن
\\
عاقبت‌بینی نشان نور تست
&&
شهوت حالی حقیقت گور تست
\\
عاقبت‌بینی که صد بازی بدید
&&
مثل آن نبود که یک بازی شنید
\\
زان یکی بازی چنان مغرور شد
&&
کز تکبر ز اوستادان دور شد
\\
سامری‌وار آن هنر در خود چو دید
&&
او ز موسی از تکبر سر کشید
\\
او ز موسی آن هنر آموخته
&&
وز معلم چشم را بر دوخته
\\
لاجرم موسی دگر بازی نمود
&&
تا که آن بازی و جانش را ربود
\\
ای بسا دانش که اندر سر دود
&&
تا شود سرور بدان خود سر رود
\\
سر نخواهی که رود تو پای باش
&&
در پناه قطب صاحب‌رای باش
\\
گرچه شاهی خویش فوق او مبین
&&
گرچه شهدی جز نبات او مچین
\\
فکر تو نقش است و فکر اوست جان
&&
نقد تو قلبست و نقد اوست کان
\\
او توی خود را بجو در اوی او
&&
کو و کو گو فاخته شو سوی او
\\
ور نخواهی خدمت ابناء جنس
&&
در دهان اژدهایی همچو خرس
\\
بوک استادی رهاند مر ترا
&&
وز خطر بیرون کشاند مر ترا
\\
زاریی می‌کن چو زورت نیست هین
&&
چونک کوری سر مکش از راه‌بین
\\
تو کم از خرسی نمی‌نالی ز درد
&&
خرس رست از درد چون فریاد کرد
\\
ای خدا این سنگ دل را موم کن
&&
ناله‌اش را تو خوش و مرحوم کن
\\
\end{longtable}
\end{center}
