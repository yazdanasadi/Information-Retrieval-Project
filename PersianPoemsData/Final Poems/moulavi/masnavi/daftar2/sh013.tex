\begin{center}
\section*{بخش ۱۳ - تمامی قصهٔ زنده شدن استخوانها به دعای عیسی علیه السلام}
\label{sec:sh013}
\addcontentsline{toc}{section}{\nameref{sec:sh013}}
\begin{longtable}{l p{0.5cm} r}
خواند عیسی نام حق بر استخوان
&&
از برای التماس آن جوان
\\
حکم یزدان از پی آن خام مرد
&&
صورت آن استخوان را زنده کرد
\\
از میان بر جست یک شیر سیاه
&&
پنجه‌ای زد کرد نقشش را تباه
\\
کله‌اش بر کند مغزش ریخت زود
&&
مغز جوزی کاندرو مغزی نبود
\\
گر ورا مغزی بدی اشکستنش
&&
خود نبودی نقص الا بر تنش
\\
گفت عیسی چون شتابش کوفتی
&&
گفت زان رو که تو زو آشوفتی
\\
گفت عیسی چون نخوردی خون مرد
&&
گفت در قسمت نبودم رزق خورد
\\
ای بسا کس همچو آن شیر ژیان
&&
صید خود ناخورده رفته از جهان
\\
قسمتش کاهی نه و حرصش چو کوه
&&
وجه نه و کرده تحصیل وجوه
\\
ای میسر کرده بر ما در جهان
&&
سخره و بیگار ما را وا رهان
\\
طعمه بنموده بما وان بوده شست
&&
آنچنان بنما بما آن را که هست
\\
گفت آن شیر ای مسیحا این شکار
&&
بود خالص از برای اعتبار
\\
گر مرا روزی بدی اندر جهان
&&
خود چه کارستی مرا با مردگان
\\
این سزای آنک یابد آب صاف
&&
همچو خر در جو بمیزد از گزاف
\\
گر بداند قیمت آن جوی خر
&&
او به جای پا نهد در جوی سر
\\
او بیابد آنچنان پیغامبری
&&
میر آبی زندگانی‌پروری
\\
چون نمیرد پیش او کز امر کن
&&
ای امیر آب ما را زنده کن
\\
هین سگ نفس ترا زنده مخواه
&&
کو عدو جان تست از دیرگاه
\\
خاک بر سر استخوانی را که آن
&&
مانع این سگ بود از صید جان
\\
سگ نه‌ای بر استخوان چون عاشقی
&&
دیوچه‌وار از چه بر خون عاشقی
\\
آن چه چشمست آن که بیناییش نیست
&&
ز امتحانها جز که رسواییش نیست
\\
سهو باشد ظنها را گاه گاه
&&
این چه ظنست این که کور آمد ز راه
\\
دیده آ بر دیگران نوحه‌گری
&&
مدتی بنشین و بر خود می‌گری
\\
ز ابر گریان شاخ سبز و تر شود
&&
زانک شمع از گریه روشن‌تر شود
\\
هر کجا نوحه کنند آنجا نشین
&&
زانک تو اولیتری اندر حنین
\\
زانک ایشان در فراق فانی‌اند
&&
غافل از لعل بقای کانی‌اند
\\
زانک بر دل نقش تقلیدست بند
&&
رو به آب چشم بندش را برند
\\
زانک تقلید آفت هر نیکویست
&&
که بود تقلید اگر کوه قویست
\\
گر ضریری لمترست و تیز خشم
&&
گوشت‌پاره‌ش دان چو او را نیست چشم
\\
گر سخن گوید ز مو باریکتر
&&
آن سرش را زان سخن نبود خبر
\\
مستیی دارد ز گفت خود ولیک
&&
از بر وی تا بمی راهیست نیک
\\
همچو جویست او نه او آبی خورد
&&
آب ازو بر آب‌خوران بگذرد
\\
آب در جو زان نمی‌گیرد قرار
&&
زانک آن جو نیست تشنه و آب‌خوار
\\
همچو نایی نالهٔ زاری کند
&&
لیک بیگار خریداری کند
\\
نوحه‌گر باشد مقلد در حدیث
&&
جز طمع نبود مراد آن خبیث
\\
نوحه‌گر گوید حدیث سوزناک
&&
لیک کو سوز دل و دامان چاک
\\
از محقق تا مقلد فرقهاست
&&
کین چو داوودست و آن دیگر صداست
\\
منبع گفتار این سوزی بود
&&
وان مقلد کهنه‌آموزی بود
\\
هین مشو غره بدان گفت حزین
&&
بار بر گاوست و بر گردون حنین
\\
هم مقلد نیست محروم از ثواب
&&
نوحه‌گر را مزد باشد در حساب
\\
کافر و مؤمن خدا گویند لیک
&&
درمیان هر دو فرقی هست نیک
\\
آن گدا گوید خدا از بهر نان
&&
متقی گوید خدا از عین جان
\\
گر بدانستی گدا از گفت خویش
&&
پیش چشم او نه کم ماندی نه بیش
\\
سالها گوید خدا آن نان‌خواه
&&
همچو خر مصحف کشد از بهر کاه
\\
گر بدل در تافتی گفت لبش
&&
ذره ذره گشته بودی قالبش
\\
نام دیوی ره برد در ساحری
&&
تو بنام حق پشیزی می‌بری
\\
\end{longtable}
\end{center}
