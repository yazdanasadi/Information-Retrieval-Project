\begin{center}
\section*{بخش ۷۴ - کرامات شیخ اقطع و زنبیل بافتن او بدو دست}
\label{sec:sh074}
\addcontentsline{toc}{section}{\nameref{sec:sh074}}
\begin{longtable}{l p{0.5cm} r}
در عریش او را یکی زایر بیافت
&&
کو بهر دو دست می زنبیل بافت
\\
گفت او را ای عدو جان خویش
&&
در عریشم آمده سر کرده پیش
\\
این چراکردی شتاب اندر سباق
&&
گفت از افراط مهر و اشتیاق
\\
پس تبسم کرد و گفت اکنون بیا
&&
لیک مخفی دار این را ای کیا
\\
تا نمیرم من مگو این با کسی
&&
نه قرینی نه حبیبی نه خسی
\\
بعد از آن قومی دگر از روزنش
&&
مطلع گشتند بر بافیدنش
\\
گفت حکمت را تو دانی کردگار
&&
من کنم پنهان تو کردی آشکار
\\
آمد الهامش که یکچندی بدند
&&
که درین غم بر تو منکر می‌شدند
\\
که مگر سالوس بود او در طریق
&&
که خدا رسواش کرد اندر فریق
\\
من نخواهم کان رمه کافر شوند
&&
در ضلالت در گمان بد روند
\\
این کرامت را بکردیم آشکار
&&
که دهیمت دست اندر وقت کار
\\
تا که آن بیچارگان بد گمان
&&
رد نگردند از جناب آسمان
\\
من ترا بی این کرامتها ز پیش
&&
خود تسلی دادمی از ذات خویش
\\
این کرامت بهر ایشان دادمت
&&
وین چراغ از بهر آن بنهادمت
\\
تو از آن بگذشته‌ای کز مرگ تن
&&
ترسی وز تفریق اجزای بدن
\\
وهم تفریق سر و پا از تو رفت
&&
دفع وهم اسپر رسیدت نیک زفت
\\
\end{longtable}
\end{center}
