\begin{center}
\section*{بخش ۱۰۴ - باز شرح کردن حکایت آن طالب روزی حلال بی کسب و رنج در عهد داود علیه السلام  و مستجاب شدن دعای او}
\label{sec:sh104}
\addcontentsline{toc}{section}{\nameref{sec:sh104}}
\begin{longtable}{l p{0.5cm} r}
یادم آمد آن حکایت کان فقیر
&&
روز و شب می‌کرد افغان و نفیر
\\
وز خدا می‌خواست روزی حلال
&&
بی شکار و رنج و کسب و انتقال
\\
پیش ازین گفتیم بعضی حال او
&&
لیک تعویق آمد و شد پنج‌تو
\\
هم بگوییمش کجا خواهد گریخت
&&
چون ز ابر فضل حق حکمت بریخت
\\
صاحب گاوش بدید و گفت هین
&&
ای بظلمت گاو من گشته رهین
\\
هین چراکشتی بگو گاو مرا
&&
ابله طرار انصاف اندر آ
\\
گفت من روزی ز حق می‌خواستم
&&
قبله را از لابه می‌آراستم
\\
آن دعای کهنه‌ام شد مستجاب
&&
روزی من بود کشتم نک جواب
\\
او ز خشم آمد گریبانش گرفت
&&
چند مشتی زد به رویش ناشکفت
\\
\end{longtable}
\end{center}
