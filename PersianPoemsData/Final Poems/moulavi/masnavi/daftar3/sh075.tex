\begin{center}
\section*{بخش ۷۵ - سبب جرات ساحران فرعون بر قطع دست و پا}
\label{sec:sh075}
\addcontentsline{toc}{section}{\nameref{sec:sh075}}
\begin{longtable}{l p{0.5cm} r}
ساحران را نه که فرعون لعین
&&
کرد تهدید سیاست بر زمین
\\
که ببرم دست و پاتان از خلاف
&&
پس در آویزم ندارمتان معاف
\\
او همی‌پنداشت کایشان در همان
&&
وهم و تخویفند و وسواس و گمان
\\
که بودشان لرزه و تخویف و ترس
&&
از توهمها و تهدیدات نفس
\\
او نمی‌داست کایشان رسته‌اند
&&
بر دریچهٔ نور دل بنشسته‌اند
\\
این جهان خوابست اندر ظن مه‌ایست
&&
گر رود درخواب دستی باک نیست
\\
گر بخواب اندر سرت ببرید گاز
&&
هم سرت بر جاست و هم عمرت دراز
\\
گر ببینی خواب در خود را دو نیم
&&
تن‌درستی چون بخیزی نی سقیم
\\
حاصل اندر خواب نقصان بدن
&&
نیست باک و نه دوصد پاره شدن
\\
این جهان را که بصورت قایمست
&&
گفت پیغامبر که حلم نایمست
\\
از ره تقلید تو کردی قبول
&&
سالکان این دیده پیدا بی رسول
\\
روز در خوابی مگو کین خواب نیست
&&
سایه فرعست اصل جز مهتاب نیست
\\
خواب و بیداریت آن دان ای عضد
&&
که ببیند خفته کو در خواب شد
\\
او گمان برده که این دم خفته‌ام
&&
بی‌خبر زان کوست درخواب دوم
\\
هاون گردون اگر صد بارشان
&&
خرد کوبد اندرین گلزارشان
\\
اصل این ترکیب را چون دیده‌اند
&&
از فروع وهم کم ترسیده‌اند
\\
سایهٔ خود را ز خود دانسته‌اند
&&
چابک و چست و گش و بر جسته‌اند
\\
کوزه‌گر گر کوزه‌ای را بشکند
&&
چون بخواهد باز خود قایم کند
\\
کور را هر گام باشد ترس چاه
&&
با هزاران ترس می‌آید براه
\\
مرد بینا دید عرض راه را
&&
پس بداند او مغاک و چاه را
\\
پا و زانواش نلرزد هر دمی
&&
رو ترش کی دارد او از هر غمی
\\
خیز فرعونا که ما آن نیستیم
&&
که بهر بانگی و غولی بیستیم
\\
خرقهٔ ما را بدر دوزنده هست
&&
ورنه ما را خود برهنه‌تر به است
\\
بی لباس این خوب را اندر کنار
&&
خوش در آریم ای عدو نابکار
\\
خوشتر از تجرید از تن وز مزاج
&&
نیست ای فرعون بی الهام گیج
\\
\end{longtable}
\end{center}
