\begin{center}
\section*{بخش ۱۷۷ - مسلهٔ فنا و بقای درویش}
\label{sec:sh177}
\addcontentsline{toc}{section}{\nameref{sec:sh177}}
\begin{longtable}{l p{0.5cm} r}
گفت قایل در جهان درویش نیست
&&
ور بود درویش آن درویش نیست
\\
هست از روی بقای ذات او
&&
نیست گشته وصف او در وصف هو
\\
چون زبانهٔ شمع پیش آفتاب
&&
نیست باشد هست باشد در حساب
\\
هست باشد ذات او تا تو اگر
&&
بر نهی پنبه بسوزد زان شرر
\\
نیست باشد روشنی ندهد ترا
&&
کرده باشد آفتاب او را فنا
\\
در دو صد من شهد یک اوقیه خل
&&
چون در افکندی و در وی گشت حل
\\
نیست باشد طعم خل چون می‌چشی
&&
هست اوقیه فزون چون برکشی
\\
پیش شیری آهوی بیهوش شد
&&
هستی‌اش در هست او روپوش شد
\\
این قیاس ناقصان بر کار رب
&&
جوشش عشقست نه از ترک ادب
\\
نبض عاشق بی ادب بر می‌جهد
&&
خویش را در کفهٔ شه می‌نهد
\\
بی‌ادب‌تر نیست کس زو در جهان
&&
با ادب‌تر نیست کس زو در نهان
\\
هم بنسبت دان وفاق ای منتجب
&&
این دو ضد با ادب با بی‌ادب
\\
بی‌ادب باشد چو ظاهر بنگری
&&
که بود دعوی عشقش هم‌سری
\\
چون به باطن بنگری دعوی کجاست
&&
او و دعوی پیش آن سلطان فناست
\\
مات زید زید اگر فاعل بود
&&
لیک فاعل نیست کو عاطل بود
\\
او ز روی لفظ نحوی فاعلست
&&
ورنه او مفعول و موتش قاتلست
\\
فاعل چه کو چنان مقهور شد
&&
فاعلیها جمله از وی دور شد
\\
\end{longtable}
\end{center}
