\begin{center}
\section*{بخش ۲۰۷ - جواب طعنه‌زننده در مثنوی از قصور فهم خود}
\label{sec:sh207}
\addcontentsline{toc}{section}{\nameref{sec:sh207}}
\begin{longtable}{l p{0.5cm} r}
ای سگ طاعن تو عو عو می‌کنی
&&
طعن قرآن را برون‌شو می‌کنی
\\
این نه آن شیرست کز وی جان بری
&&
یا ز پنجهٔ قهر او ایمان بری
\\
تا قیامت می‌زند قرآن ندی
&&
ای گروهی جهل را گشته فدی
\\
که مرا افسانه می‌پنداشتید
&&
تخم طعن و کافری می‌کاشتید
\\
خود بدیدیت آنک طعنه می‌زدیت
&&
که شما فانی و افسانه بدیت
\\
من کلام حقم و قایم به ذات
&&
قوت جان جان و یاقوت زکات
\\
نور خورشیدم فتاده بر شما
&&
لیک از خورشید ناگشته جدا
\\
نک منم ینبوع آن آب حیات
&&
تا رهانم عاشقان را از ممات
\\
گر چنان گند آزتان ننگیختی
&&
جرعه‌ای بر گورتان حق ریختی
\\
نه بگیرم گفت و پند آن حکیم
&&
دل نگردانم بهر طعنی سقیم
\\
\end{longtable}
\end{center}
