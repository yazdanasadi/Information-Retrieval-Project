\begin{center}
\section*{بخش ۱۷۸ - قصه وکیل صدر جهان کی متهم شد و از بخارا گریخت از بیم جان باز عشقش کشید رو کشان کی کار جان سهل باشد عاشقان را}
\label{sec:sh178}
\addcontentsline{toc}{section}{\nameref{sec:sh178}}
\begin{longtable}{l p{0.5cm} r}
در بخارا بندهٔ صدر جهان
&&
متهم شد گشت از صدرش نهان
\\
مدت ده سال سرگردان بگشت
&&
گه خراسان گه کهستان گاه دشت
\\
از پس ده سال او از اشتیاق
&&
گشت بی‌طاقت ز ایام فراق
\\
گفت تاب فرقتم زین پس نماند
&&
صبر کی داند خلاعت را نشاند
\\
از فراق این خاکها شوره بود
&&
آب زرد و گنده و تیره شود
\\
باد جان‌افزا وخم گردد وبا
&&
آتشی خاکستری گردد هبا
\\
باغ چون جنت شود دار المرض
&&
زرد و ریزان برگ او اندر حرض
\\
عقل دراک از فراق دوستان
&&
همچو تیرانداز اشکسته کمان
\\
دوزخ از فرقت چنان سوزان شدست
&&
پیر از فرقت چنان لرزان شدست
\\
گر بگویم از فراق چون شرار
&&
تا قیامت یک بود از صد هزار
\\
پس ز شرح سوز او کم زن نفس
&&
رب سلم رب سلم گوی و بس
\\
هرچه از وی شاد گردی در جهان
&&
از فراق او بیندیش آن زمان
\\
زانچ گشتی شاد بس کس شاد شد
&&
آخر از وی جست و همچون باد شد
\\
از تو هم بجهد تو دل بر وی منه
&&
پیش از آن کو بجهد از وی تو بجه
\\
\end{longtable}
\end{center}
