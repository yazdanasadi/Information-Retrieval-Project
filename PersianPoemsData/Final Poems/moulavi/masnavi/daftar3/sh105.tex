\begin{center}
\section*{بخش ۱۰۵ - رفتن هر دو خصم نزد داود علیه السلام}
\label{sec:sh105}
\addcontentsline{toc}{section}{\nameref{sec:sh105}}
\begin{longtable}{l p{0.5cm} r}
می‌کشیدش تا به داود نبی
&&
که بیا ای ظالم گیج غبی
\\
حجت بارد رها کن ای دغا
&&
عقل در تن آور و با خویش آ
\\
این چه می‌گویی دعا چه بود مخند
&&
بر سر و و ریش من و خویش ای لوند
\\
گفت من با حق دعاها کرده‌ام
&&
اندرین لابه بسی خون خورده‌ام
\\
من یقین دارم دعا شد مستجاب
&&
سر بزن بر سنگ ای منکرخطاب
\\
گفت گرد آیید هین یا مسلمین
&&
ژاژ بینید و فشار این مهین
\\
ای مسلمانان دعا مال مرا
&&
چون از آن او کند بهر خدا
\\
گر چنین بودی همه عالم بدین
&&
یک دعا املاک بردندی بکین
\\
گر چنین بودی گدایان ضریر
&&
محتشم گشته بدندی و امیر
\\
روز و شب اندر دعااند و ثنا
&&
لابه‌گویان که تو ده‌مان ای خدا
\\
تا تو ندهی هیچ کس ندهد یقین
&&
ای گشاینده تو بگشا بند این
\\
مکسب کوران بود لابه و دعا
&&
جز لب نانی نیابند از عطا
\\
خلق گفتند این مسلمان راست‌گوست
&&
وین فروشندهٔ دعاها ظلم‌جوست
\\
این دعا کی باشد از اسباب ملک
&&
کی کشید این را شریعت خود بسلک
\\
بیع و بخشش یا وصیت یا عطا
&&
یا ز جنس این شود ملکی ترا
\\
در کدامین دفترست این شرع نو
&&
گاو را تو باز ده یا حبس رو
\\
او به سوی آسمان می‌کرد رو
&&
واقعهٔ ما را نداند غیر تو
\\
در دل من آن دعا انداختی
&&
صد امید اندر دلم افراختی
\\
من نمی‌کردم گزافه آن دعا
&&
همچو یوسف دیده بودم خوابها
\\
دید یوسف آفتاب و اختران
&&
پیش او سجده‌کنان چون چاکران
\\
اعتمادش بود بر خواب درست
&&
در چه و زندان جز آن را می‌نجست
\\
ز اعتماد او نبودش هیچ غم
&&
از غلامی وز ملام و بیش و کم
\\
اعتمادی داشت او بر خواب خویش
&&
که چو شمعی می‌فروزیدش ز پیش
\\
چون در افکندند یوسف را به چاه
&&
بانگ آمد سمع او را از اله
\\
که تو روزی شه شوی ای پهلوان
&&
تا بمالی این جفا در رویشان
\\
قایل این بانگ ناید در نظر
&&
لیک دل بشناخت قایل را ز اثر
\\
قوتی و راحتی و مسندی
&&
در میان جان فتادش زان ندا
\\
چاه شد بر وی بدان بانگ جلیل
&&
گلشن و بزمی چو آتش بر خلیل
\\
هر جفا که بعد از آنش می‌رسید
&&
او بدان قوت بشادی می‌کشید
\\
همچنانک ذوق آن بانگ الست
&&
در دل هر مؤمنی تا حشر هست
\\
تا نباشد در بلاشان اعتراض
&&
نه ز امر و نهی حقشان انقباض
\\
لقمهٔ حکمی که تلخی می‌نهد
&&
گلشکر آن را گوارش می‌دهد
\\
گلشکر آن را که نبود مستند
&&
لقمه را ز انکار او قی می‌کند
\\
هر که خوابی دید از روز الست
&&
مست باشد در ره طاعات مست
\\
می‌کشد چون اشتر مست این جوال
&&
بی فتور و بی گمان و بی ملال
\\
کفک تصدیقش بگرد پوز او
&&
شد گواه مستی و دلسوز او
\\
اشتر از قوت چو شیر نر شده
&&
زیر ثقل بار اندک‌خور شده
\\
ز آرزوی ناقه صد فاقه برو
&&
می‌نماید کوه پیشش تار مو
\\
در الست آنکو چنین خوابی ندید
&&
اندرین دنیا نشد بنده و مرید
\\
ور بشد اندر تردد صد دله
&&
یک زمان شکرستش و سالی گله
\\
پای پیش و پای پس در راه دین
&&
می‌نهد با صد تردد بی یقین
\\
وام‌دار شرح اینم نک گرو
&&
ور شتابستت ز الم نشرح شنو
\\
چون ندارد شرح این معنی کران
&&
خر به سوی مدعی گاو ران
\\
گفت کورم خواند زین جرم آن دغا
&&
بس بلیسانه قیاسست ای خدا
\\
من دعا کورانه کی می‌کرده‌ام
&&
جز به خالق کدیه کی آورده‌ام
\\
کور از خلقان طمع دارد ز جهل
&&
من ز تو کز تست هر دشوار سهل
\\
آن یکی کورم ز کوران بشمرید
&&
او نیاز جان و اخلاصم ندید
\\
کوری عشقست این کوری من
&&
حب یعمی و یصمست ای حسن
\\
کورم از غیر خدا بینا بدو
&&
مقتضای عشق این باشد نکو
\\
تو که بینایی ز کورانم مدار
&&
دایرم برگرد لطفت ای مدار
\\
آنچنانک یوسف صدیق را
&&
خواب بنمودی و گشتش متکا
\\
مر مرا لطف تو هم خوابی نمود
&&
آن دعای بی‌حدم بازی نبود
\\
می‌نداند خلق اسرار مرا
&&
ژاژ می‌دانند گفتار مرا
\\
حقشان است و کی داند راز غیب
&&
غیر علام سر و ستار عیب
\\
خصم گفتش رو به من کن حق بگو
&&
رو چه سوی آسمان کردی عمو
\\
شید می‌آری غلط می‌افکنی
&&
لاف عشق و لاف قربت می‌زنی
\\
با کدامین روی چون دل‌مرده‌ای
&&
روی سوی آسمانها کرده‌ای
\\
غلغلی در شهر افتاده ازین
&&
آن مسلمان می‌نهد رو بر زمین
\\
کای خدا این بنده را رسوا مکن
&&
گر بدم هم سر من پیدا مکن
\\
تو همی‌دانی و شبهای دراز
&&
که همی‌خواندم ترا با صد نیاز
\\
پیش خلق این را اگر خود قدر نیست
&&
پیش تو همچون چراغ روشنیست
\\
\end{longtable}
\end{center}
