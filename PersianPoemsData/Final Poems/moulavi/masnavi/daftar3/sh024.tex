\begin{center}
\section*{بخش ۲۴ - تفسیر ولتعرفنهم فی لحن القول}
\label{sec:sh024}
\addcontentsline{toc}{section}{\nameref{sec:sh024}}
\begin{longtable}{l p{0.5cm} r}
گفت یزدان مر نبی را در مساق
&&
یک نشانی سهل‌تر ز اهل نفاق
\\
گر منافق زفت باشد نغز و هول
&&
وا شناسی مر ورا در لحن و قول
\\
چون سفالین کوزه‌ها را می‌خری
&&
امتحانی می‌کنی ای مشتری
\\
می‌زنی دستی بر آن کوزه چرا
&&
تا شناسی از طنین اشکسته را
\\
بانگ اشکسته دگرگون می‌بود
&&
بانگ چاووشست پیشش می‌رود
\\
بانگ می‌آید که تعریفش کند
&&
همچو مصدر فعل تصریفش کند
\\
چون حدیث امتحان رویی نمود
&&
یادم آمد قصهٔ هاروت زود
\\
\end{longtable}
\end{center}
