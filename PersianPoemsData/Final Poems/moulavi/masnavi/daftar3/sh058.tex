\begin{center}
\section*{بخش ۵۸ - مثال رنجور شدن آدمی بوهم تعظیم خلق  و رغبت مشتریان بوی و حکایت معلم}
\label{sec:sh058}
\addcontentsline{toc}{section}{\nameref{sec:sh058}}
\begin{longtable}{l p{0.5cm} r}
کودکان مکتبی از اوستاد
&&
رنج دیدند از ملال و اجتهاد
\\
مشورت کردند در تعویق کار
&&
تا معلم در فتد در اضطرار
\\
چون نمی‌آید ورا رنجوریی
&&
که بگیرد چند روز او دوریی
\\
تا رهیم از حبس و تنگی و ز کار
&&
هست او چون سنگ خارا بر قرار
\\
آن یکی زیرکتر این تدبیر کرد
&&
که بگوید اوستا چونی تو زرد
\\
خیر باشد رنگ تو بر جای نیست
&&
این اثر یا از هوا یا از تبیست
\\
اندکی اندر خیال افتد ازین
&&
تو برادر هم مدد کن این‌چنین
\\
چون درآیی از در مکتب بگو
&&
خیر باشد اوستا احوال تو
\\
آن خیالش اندکی افزون شود
&&
کز خیالی عاقلی مجنون شود
\\
آن سوم و آن چارم و پنجم چنین
&&
در پی ما غم نمایند و حنین
\\
تا چو سی کودک تواتر این خبر
&&
متفق گویند یابد مستقر
\\
هر یکی گفتش که شاباش ای ذکی
&&
باد بختت بر عنایت متکی
\\
متفق گشتند در عهد وثیق
&&
که نگرداند سخن را یک رفیق
\\
بعد از آن سوگند داد او جمله را
&&
تا که غمازی نگوید ماجرا
\\
رای آن کودک بچربید از همه
&&
عقل او در پیش می‌رفت از رمه
\\
آن تفاوت هست در عقل بشر
&&
که میان شاهدان اندر صور
\\
زین قبل فرمود احمد در مقال
&&
در زبان پنهان بود حسن رجال
\\
\end{longtable}
\end{center}
