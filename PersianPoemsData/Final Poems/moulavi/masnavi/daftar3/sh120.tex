\begin{center}
\section*{بخش ۱۲۰ - صفت خرمی شهر اهل سبا و ناشکری ایشان}
\label{sec:sh120}
\addcontentsline{toc}{section}{\nameref{sec:sh120}}
\begin{longtable}{l p{0.5cm} r}
اصلشان بد بود آن اهل سبا
&&
می‌رمیدندی ز اسباب لقا
\\
دادشان چندان ضیاع و باغ و راغ
&&
از چپ و از راست از بهر فراغ
\\
بس که می‌افتاد از پری ثمار
&&
تنگ می‌شد معبر ره بر گذار
\\
آن نثار میوه ره را می‌گرفت
&&
از پری میوه ره‌رو در شگفت
\\
سله بر سر در درختستانشان
&&
پر شدی ناخواست از میوه‌فشان
\\
باد آن میوه فشاندی نه کسی
&&
پر شدی زان میوه دامنها بسی
\\
خوشه‌های زفت تا زیر آمده
&&
بر سر و روی رونده می‌زده
\\
مرد گلخن‌تاب از پری زر
&&
بسته بودی در میان زرین کمر
\\
سگ کلیچه کوفتی در زیر پا
&&
تخمه بودی گرگ صحرا از نوا
\\
گشته آمن شهر و ده از دزد و گرگ
&&
بز نترسیدی هم از گرگ سترگ
\\
گر بگویم شرح نعمتهای قوم
&&
که زیادت می‌شد آن یوما بیوم
\\
مانع آید از سخنهای مهم
&&
انبیا بردند امر فاستقم
\\
\end{longtable}
\end{center}
