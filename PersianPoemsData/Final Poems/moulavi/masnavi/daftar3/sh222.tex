\begin{center}
\section*{بخش ۲۲۲ - جذب معشوق عاشق را من حیث لا یعمله العاشق و لا یرجوه و لا یخطر بباله و لا یظهر من ذلک الجذب اثر فی العاشق الا الخوف الممزوج بالیاس مع دوام الطلب}
\label{sec:sh222}
\addcontentsline{toc}{section}{\nameref{sec:sh222}}
\begin{longtable}{l p{0.5cm} r}
آمدیم اینجا که در صدر جهان
&&
گر نبودی جذب آن عاشق نهان
\\
ناشکیباکی بدی او از فراق
&&
کی دوان باز آمدی سوی وثاق
\\
میل معشوقان نهانست و ستیر
&&
میل عاشق با دو صد طبل و نفیر
\\
یک حکایت هست اینجا ز اعتبار
&&
لیک عاجز شد بخاری ز انتظار
\\
ترک آن کردیم کو در جست و جوست
&&
تاکه پیش از مرگ بیند روی دوست
\\
تا رهد از مرگ تا یابد نجات
&&
زانک دید دوستست آب حیات
\\
هر که دید او نباشد دفع مرگ
&&
دوست نبود که نه میوه‌ستش نه برگ
\\
کار آن کارست ای مشتاق مست
&&
کاندر آن کار ار رسد مرگت خوشست
\\
شد نشان صدق ایمان ای جوان
&&
آنک آید خوش ترا مرگ اندر آن
\\
گر نشد ایمان تو ای جان چنین
&&
نیست کامل رو بجو اکمال دین
\\
هر که اندر کار تو شد مرگ‌دوست
&&
بر دل تو بی کراهت دوست اوست
\\
چون کراهت رفت آن خود مرگ نیست
&&
صورت مرگست و نقلان کردنیست
\\
چون کراهت رفت مردن نفع شد
&&
پس درست آید که مردن دفع شد
\\
دوست حقست و کسی کش گفت او
&&
که توی آن من و من آن تو
\\
گوش دار اکنون که عاشق می‌رسد
&&
بسته عشق او را به حبل من مسد
\\
چون بدید او چهرهٔ صدر جهان
&&
گوییا پریدش از تن مرغ جان
\\
همچو چوب خشک افتاد آن تنش
&&
سرد شد از فرق جان تا ناخنش
\\
هرچه کردند از بخور و از گلاب
&&
نه بجنبید و نه آمد در خطاب
\\
شاه چون دید آن مزعفر روی او
&&
پس فرود آمد ز مرکب سوی او
\\
گفت عاشق دوست می‌جوید بتفت
&&
چونک معشوق آمد آن عاشق برفت
\\
عاشق حقی و حق آنست کو
&&
چون بیاید نبود از تو تای مو
\\
صد چو تو فانیست پیش آن نظر
&&
عاشقی بر نفی خود خواجه مگر
\\
سایه‌ای و عاشقی بر آفتاب
&&
شمس آید سایه لا گردد شتاب
\\
\end{longtable}
\end{center}
