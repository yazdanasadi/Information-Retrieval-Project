\begin{center}
\section*{بخش ۱۶۸ - وفات یافتن بلال رضی الله عنه با شادی}
\label{sec:sh168}
\addcontentsline{toc}{section}{\nameref{sec:sh168}}
\begin{longtable}{l p{0.5cm} r}
چون بلال از ضعف شد همچون هلال
&&
رنگ مرگ افتاد بر روی بلال
\\
جفت او دیدش بگفتا وا حرب
&&
پس بلالش گفت نه نه وا طرب
\\
تا کنون اندر حرب بودم ز زیست
&&
تو چه دانی مرگ چون عیشست و چیست
\\
این همی گفت و رخش در عین گفت
&&
نرگس و گلبرگ و لاله می‌شکفت
\\
تاب رو و چشم پر انوار او
&&
می گواهی داد بر گفتار او
\\
هر سیه دل می سیه دیدی ورا
&&
مردم دیده سیاه آمد چرا
\\
مردم نادیده باشد رو سیاه
&&
مردم دیده بود مرآت ماه
\\
خود کی بیند مردم دیدهٔ ترا
&&
در جهان جز مردم دیده‌فزا
\\
چون به غیر مردم دیده‌ش ندید
&&
پس به غیر او کی در رنگش رسید
\\
پس جز او جمله مقلد آمدند
&&
در صفات مردم دیده بلند
\\
گفت جفتش الفراق ای خوش‌خصال
&&
گفت نه نه الوصالست الوصال
\\
گفت جفت امشب غریبی می‌روی
&&
از تبار و خویش غایب می‌شوی
\\
گفت نه نه بلک امشب جان من
&&
می‌رسد خود از غریبی در وطن
\\
گفت رویت را کجا بینیم ما
&&
گفت اندر حلقهٔ خاص خدا
\\
حلقهٔ خاصش به تو پیوسته است
&&
گر نظر بالا کنی نه سوی پست
\\
اندر آن حلقه ز رب العالمین
&&
نور می‌تابد چو در حلقه نگین
\\
گفت ویران گشت این خانه دریغ
&&
گفت اندر مه نگر منگر به میغ
\\
کرد ویران تا کند معمورتر
&&
قومم انبه بود و خانه مختصر
\\
\end{longtable}
\end{center}
