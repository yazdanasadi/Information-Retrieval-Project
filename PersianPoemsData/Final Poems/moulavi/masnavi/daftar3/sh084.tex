\begin{center}
\section*{بخش ۸۴ - سال کردن بهلول آن درویش را}
\label{sec:sh084}
\addcontentsline{toc}{section}{\nameref{sec:sh084}}
\begin{longtable}{l p{0.5cm} r}
گفت بهلول آن یکی درویش را
&&
چونی ای درویش واقف کن مرا
\\
گفت چون باشد کسی که جاودان
&&
بر مراد او رود کار جهان
\\
سیل و جوها بر مراد او روند
&&
اختران زان سان که خواهد آن شوند
\\
زندگی و مرگ سرهنگان او
&&
بر مراد او روانه کو بکو
\\
هر کجا خواهد فرستد تعزیت
&&
هر کجا خواهد ببخشد تهنیت
\\
سالکان راه هم بر گام او
&&
ماندگان از راه هم در دام او
\\
هیچ دندانی نخندد در جهان
&&
بی رضا و امر آن فرمان‌روان
\\
گفت ای شه راست گفتی همچنین
&&
در فر و سیمای تو پیداست این
\\
این و صد چندینی ای صادق ولیک
&&
شرح کن این را بیان کن نیک نیک
\\
آنچنانک فاضل و مرد فضول
&&
چون به گوش او رسد آرد قبول
\\
آنچنانش شرح کن اندر کلام
&&
که از آن هم بهره یابد عقل عام
\\
ناطق کامل چو خوان‌پاشی بود
&&
خوانش بر هر گونهٔ آشی بود
\\
که نماند هیچ مهمان بی نوا
&&
هر کسی یابد غذای خود جدا
\\
همچو قرآن که بمعنی هفت توست
&&
خاص را و عام را مطعم دروست
\\
گفت این باری یقین شد پیش عام
&&
که جهان در امر یزدانست رام
\\
هیچ برگی در نیفتد از درخت
&&
بی قضا و حکم آن سلطان بخت
\\
از دهان لقمه نشد سوی گلو
&&
تا نگوید لقمه را حق که ادخلوا
\\
میل و رغبت کان زمام آدمیست
&&
جنبش آن رام امر آن غنیست
\\
در زمینها و آسمانها ذره‌ای
&&
پر نجنباند نگردد پره‌ای
\\
جز به فرمان قدیم نافذش
&&
شرح نتوان کرد و جلدی نیست خوش
\\
کی شمرد برگ درختان را تمام
&&
بی‌نهایت کی شود در نطق رام
\\
این قدر بشنو که چون کلی کار
&&
می‌نگردد جز بامر کردگار
\\
چون قضای حق رضای بنده شد
&&
حکم او را بندهٔ خواهنده شد
\\
بی تکلف نه پی مزد و ثواب
&&
بلک طبع او چنین شد مستطاب
\\
زندگی خود نخواهد بهر خوذ
&&
نه پی ذوقی حیات مستلذ
\\
هرکجا امر قدم را مسلکیست
&&
زندگی و مردگی پیشش یکیست
\\
بهر یزدان می‌زید نه بهر گنج
&&
بهر یزدان می‌مرد نه از خوف رنج
\\
هست ایمانش برای خواست او
&&
نه برای جنت و اشجار و جو
\\
ترک کفرش هم برای حق بود
&&
نه ز بیم آنک در آتش رود
\\
این چنین آمد ز اصل آن خوی او
&&
نه ریاضت نه بجست و جوی او
\\
آنگهان خندد که او بیند رضا
&&
همچو حلوای شکر او را قضا
\\
بنده‌ای کش خوی و خلقت این بود
&&
نه جهان بر امر و فرمانش رود
\\
پس چرا لابه کند او یا دعا
&&
که بگردان ای خداوند این قضا
\\
مرگ او و مرگ فرزندان او
&&
بهر حق پیشش چو حلوا در گلو
\\
نزع فرزندان بر آن باوفا
&&
چون قطایف پیش شیخ بی‌نوا
\\
پس چراگوید دعا الا مگر
&&
در دعا بیند رضای دادگر
\\
آن شفاعت و آن دعا نه از رحم خود
&&
می‌کند آن بندهٔ صاحب رشد
\\
رحم خود را او همان دم سوختست
&&
که چراغ عشق حق افروختست
\\
دوزخ اوصاف او عشقست و او
&&
سوخت مر اوصاف خود را مو بمو
\\
هر طروقی این فروقی کی شناخت
&&
جز دقوقی تا درین دولت بتاخت
\\
\end{longtable}
\end{center}
