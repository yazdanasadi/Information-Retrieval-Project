\begin{center}
\section*{بخش ۱۱۶ - بیان آنک نفس آدمی بجای آن خونیست کی مدعی گاو گشته بود و آن گاو کشنده عقلست و داود حقست یا شیخ  کی نایب حق است کی بقوت و یاری او تواند ظالم را کشتن و توانگر شدن  به روزی بی‌کسب و بی‌حساب}
\label{sec:sh116}
\addcontentsline{toc}{section}{\nameref{sec:sh116}}
\begin{longtable}{l p{0.5cm} r}
نفس خود را کش جهانی را زنده کن
&&
خواجه را کشتست او را بنده کن
\\
مدعی گاو نفس تست هین
&&
خویشتن را خواجه کردست و مهین
\\
آن کشندهٔ گاو عقل تست رو
&&
بر کشنده گاو تن منکر مشو
\\
عقل اسیرست و همی خواهد ز حق
&&
روزیی بی رنج و نعمت بر طبق
\\
روزی بی رنج او موقوف چیست
&&
آنک بکشد گاو را کاصل بدیست
\\
نفس گوید چون کشی تو گاو من
&&
زانک گاو نفس باشد نقش تن
\\
خواجه‌زادهٔ عقل مانده بی‌نوا
&&
نفس خونی خواجه گشت و پیشوا
\\
روزی بی‌رنج می‌دانی که چیست
&&
قوت ارواحست و ارزاق نبیست
\\
لیک موقوفست بر قربان گاو
&&
گنج اندر گاو دان ای کنج‌کاو
\\
دوش چیزی خورده‌ام ور نه تمام
&&
دادمی در دست فهم تو زمام
\\
دوش چیزی خورده‌ام افسانه است
&&
هرچه می‌آید ز پنهان خانه است
\\
چشم بر اسباب از چه دوختیم
&&
گر ز خوش‌چشمان کرشم آموختیم
\\
هست بر اسباب اسبابی دگر
&&
در سبب منگر در آن افکن نظر
\\
انبیا در قطع اسباب آمدند
&&
معجزات خویش بر کیوان زدند
\\
بی‌سبب مر بحر را بشکافتند
&&
بی زراعت چاش گندم یافتند
\\
ریگها هم آرد شد از سعیشان
&&
پشم بز ابریشم آمد کش‌کشان
\\
جمله قرآن هست در قطع سبب
&&
عز درویش و هلاک بولهب
\\
مرغ بابیلی دو سه سنگ افکند
&&
لشکر زفت حبش را بشکند
\\
پیل را سوراخ سوراخ افکند
&&
سنگ مرغی کو به بالا پر زند
\\
دم گاو کشته بر مقتول زن
&&
تا شود زنده همان دم در کفن
\\
حلق‌ببریده جهد از جای خویش
&&
خون خود جوید ز خون‌پالای خویش
\\
همچنین ز آغاز قرآن تا تمام
&&
رفض اسبابست و علت والسلام
\\
کشف این نه از عقل کارافزا شود
&&
بندگی کن تا ترا پیداشود
\\
بند معقولات آمد فلسفی
&&
شهسوار عقل عقل آمد صفی
\\
عقل عقلت مغز و عقل تست پوست
&&
معدهٔ حیوان همیشه پوست‌جوست
\\
مغزجوی از پوست دارد صد ملال
&&
مغز نغزان را حلال آمد حلال
\\
چونک قشر عقل صد برهان دهد
&&
عقل کل کی گام بی ایقان نهد
\\
عقل دفترها کند یکسر سیاه
&&
عقل عقل آفاق دارد پر ز ماه
\\
از سیاهی و سپیدی فارغست
&&
نور ماهش بر دل و جان بازغست
\\
این سیاه و این سپید ار قدر یافت
&&
زان شب قدرست کاختروار تافت
\\
قیمت همیان و کیسه از زرست
&&
بی ز زر همیان و کیسه ابترست
\\
همچنانک قدر تن از جان بود
&&
قدر جان از پرتو جانان بود
\\
گر بدی جان زنده بی پرتو کنون
&&
هیچ گفتی کافران را میتون
\\
هین بگو که ناطقه جو می‌کند
&&
تا به قرنی بعد ما آبی رسد
\\
گرچه هر قرنی سخن‌آری بود
&&
لیک گفت سالفان یاری بود
\\
نه که هم توریت و انجیل و زبور
&&
شد گواه صدق قرآن ای شکور
\\
روزی بی‌رنج جو و بی‌حساب
&&
کز بهشتت آورد جبریل سیب
\\
بلک رزقی از خداوند بهشت
&&
بی‌صداع باغبان بی رنج کشت
\\
زانک نفع نان در آن نان داد اوست
&&
بدهدت آن نفع بی توسیط پوست
\\
ذوق پنهان نقش نان چون سفره‌ایست
&&
نان بی سفره ولی را بهره‌ایست
\\
رزق جانی کی بری با سعی و جست
&&
جز به عدل شیخ کو داود تست
\\
نفس چون با شیخ بیند کام تو
&&
از بن دندان شود او رام تو
\\
صاحب آن گاو رام آنگاه شد
&&
کز دم داود او آگاه شد
\\
عقل گاهی غالب آید در شکار
&&
برسگ نفست که باشد شیخ یار
\\
نفس اژدرهاست با صد زور و فن
&&
روی شیخ او را زمرد دیده کن
\\
گر تو صاحب گاو را خواهی زبون
&&
چون خران سیخش کن آن سو ای حرون
\\
چون به نزدیک ولی الله شود
&&
آن زبان صد گزش کوته شود
\\
صد زبان و هر زبانش صد لغت
&&
زرق و دستانش نیاید در صفت
\\
مدعی گاو نفس آمد فصیح
&&
صد هزاران حجت آرد ناصحیح
\\
شهر را بفریبد الا شاه را
&&
ره نتاند زد شه آگاه را
\\
نفس را تسبیح و مصحف در یمین
&&
خنجر و شمشیر اندر آستین
\\
مصحف و سالوس او باور مکن
&&
خویش با او هم‌سر و هم‌سر مکن
\\
سوی حوضت آورد بهر وضو
&&
واندر اندازد ترا در قعر او
\\
عقل نورانی و نیکو طالبست
&&
نفس ظلمانی برو چون غالبست
\\
زانک او در خانه عقل تو غریب
&&
بر در خود سگ بود شیر مهیب
\\
باش تا شیران سوی بیشه روند
&&
وین سگان کور آنجا بگروند
\\
مکر نفس و تن نداند عام شهر
&&
او نگردد جز بوحی القلب قهر
\\
هر که جنس اوست یار او شود
&&
جز مگر داود کان شیخت بود
\\
کو مبدل گشت و جنس تن نماند
&&
هر که را حق در مقام دل نشاند
\\
خلق جمله علتی‌اند از کمین
&&
یار علت می‌شود علت یقین
\\
هر خسی دعوی داودی کند
&&
هر که بی تمییز کف در وی زند
\\
از صیادی بشنود آواز طیر
&&
مرغ ابله می‌کند آن سوی سیر
\\
نقد را از نقل نشناسد غویست
&&
هین ازو بگریز اگر چه معنویست
\\
رسته و بر بسته پیش او یکیست
&&
گر یقین دعوی کند او در شکیست
\\
این چنین کس گر ذکی مطلقست
&&
چونش این تمییز نبود احمقست
\\
هین ازو بگریز چون آهو ز شیر
&&
سوی او مشتاق ای دانا دلیر
\\
\end{longtable}
\end{center}
