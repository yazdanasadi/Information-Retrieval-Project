\begin{center}
\section*{بخش ۳۶ - وحی آمدن به مادر موسی کی موسی را در آب افکن}
\label{sec:sh036}
\addcontentsline{toc}{section}{\nameref{sec:sh036}}
\begin{longtable}{l p{0.5cm} r}
باز وحی آمد که در آبش فکن
&&
روی در اومید دار و مو مکن
\\
در فکن در نیلش و کن اعتماد
&&
من ترا با وی رسانم رو سپید
\\
این سخن پایان ندارد مکرهاش
&&
جمله می‌پیچید هم در ساق و پاش
\\
صد هزاران طفل می‌کشت او برون
&&
موسی اندر صدر خانه در درون
\\
از جنون می‌کشت هر جا بد جنین
&&
از حیل آن کورچشم دوربین
\\
اژدها بد مکر فرعون عنود
&&
مکر شاهان جهان را خورده بود
\\
لیک ازو فرعون‌تر آمد پدید
&&
هم ورا هم مکر او را در کشید
\\
اژدها بود و عصا شد اژدها
&&
این بخورد آن را به توفیق خدا
\\
دست شد بالای دست این تا کجا
&&
تا بیزدان که الیه المنتهی
\\
کان یکی دریاست بی غور و کران
&&
جمله دریاها چو سیلی پیش آن
\\
حیله‌ها و چاره‌ها گر اژدهاست
&&
پیش الا الله آنها جمله لاست
\\
چون رسید اینجا بیانم سر نهاد
&&
محو شد والله اعلم بالرشاد
\\
آنچ در فرعون بود اندر تو هست
&&
لیک اژدرهات محبوس چهست
\\
ای دریغ این جمله احوال توست
&&
تو بر آن فرعون بر خواهیش بست
\\
گر ز تو گویند وحشت زایدت
&&
ور ز دیگر آفسان بنمایدت
\\
چه خرابت می‌کند نفس لعین
&&
دور می‌اندازدت سخت این قرین
\\
آتشت را هیزم فرعون نیست
&&
ورنه چون فرعون او شعله‌زنیست
\\
\end{longtable}
\end{center}
