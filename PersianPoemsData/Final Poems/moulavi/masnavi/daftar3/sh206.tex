\begin{center}
\section*{بخش ۲۰۶ - تفسیر یا جبال اوبی معه والطیر}
\label{sec:sh206}
\addcontentsline{toc}{section}{\nameref{sec:sh206}}
\begin{longtable}{l p{0.5cm} r}
روی داود از فرش تابان شده
&&
کوهها اندر پیش نالان شده
\\
کوه با داود گشته همرهی
&&
هردو مطرب مست در عشق شهی
\\
یا جبال اوبی امر آمده
&&
هر دو هم‌آواز و هم‌پرده شده
\\
گفت داودا تو هجرت دیده‌ای
&&
بهر من از همدمان ببریده‌ای
\\
ای غریب فرد بی مونس شده
&&
آتش شوق از دلت شعله زده
\\
مطربان خواهی و قوال و ندیم
&&
کوهها را پیشت آرد آن قدیم
\\
مطرب و قوال و سرنایی کند
&&
که به پیشت بادپیمایی کند
\\
تا بدانی ناله چون که را رواست
&&
بی لب و دندان ولی را ناله‌هاست
\\
نغمهٔ اجزای آن صافی‌جسد
&&
هر دمی در گوش حسش می‌رسد
\\
همنشینان نشنوند او بشنود
&&
ای خنک جان کو به غیبش بگرود
\\
بنگرد در نفس خود صد گفت و گو
&&
همنشین او نبرده هیچ بو
\\
صد سؤال و صد جواب اندر دلت
&&
می‌رسد از لامکان تا منزلت
\\
بشنوی تو نشنود زان گوشها
&&
گر به نزدیک تو آرد گوش را
\\
گیرم ای کر خود تو آن را نشنوی
&&
چون مثالش دیده‌ای چون نگروی
\\
\end{longtable}
\end{center}
