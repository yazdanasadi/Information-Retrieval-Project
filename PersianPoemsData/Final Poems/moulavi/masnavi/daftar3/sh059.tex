\begin{center}
\section*{بخش ۵۹ - عقول خلق متفاوتست در اصل فطرت و نزد  معتزله متساویست تفاوت عقول از تحصیل  علم است}
\label{sec:sh059}
\addcontentsline{toc}{section}{\nameref{sec:sh059}}
\begin{longtable}{l p{0.5cm} r}
اختلاف عقلها در اصل بود
&&
بر وفاق سنیان باید شنود
\\
بر خلاف قول اهل اعتزال
&&
که عقول از اصل دارند اعتدال
\\
تجربه و تعلیم بیش و کم کند
&&
تا یکی را از یکی اعلم کند
\\
باطلست این زانک رای کودکی
&&
که ندارد تجربه در مسلکی
\\
بر دمید اندیشه‌ای زان طفل خرد
&&
پیر با صد تجربه بویی نبرد
\\
خود فزون آن به که آن از فطرتست
&&
تا ز افزونی که جهد و فکرتست
\\
تو بگو دادهٔ خدا بهتر بود
&&
یاکه لنگی راهوارانه رود
\\
\end{longtable}
\end{center}
