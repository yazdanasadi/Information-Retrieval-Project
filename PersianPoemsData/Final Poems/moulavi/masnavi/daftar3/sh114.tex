\begin{center}
\section*{بخش ۱۱۴ - برون رفتن به سوی آن درخت}
\label{sec:sh114}
\addcontentsline{toc}{section}{\nameref{sec:sh114}}
\begin{longtable}{l p{0.5cm} r}
چون برون رفتند سوی آن درخت
&&
گفت دستش را سپس بندید سخت
\\
تا گناه و جرم او پیدا کنم
&&
تا لوای عدل بر صحرا زنم
\\
گفت ای سگ جد او را کشته‌ای
&&
تو غلامی خواجه زین رو گشته‌ای
\\
خواجه را کشتی و بردی مال او
&&
کرد یزدان آشکارا حال او
\\
آن زنت او را کنیزک بوده است
&&
با همین خواجه جفا بنموده است
\\
هر چه زو زایید ماده یا که نر
&&
ملک وارث باشد آنها سر بسر
\\
تو غلامی کسب و کارت ملک اوست
&&
شرع جستی شرع بستان رو نکوست
\\
خواجه را کشتی باستم زار زار
&&
هم برینجا خواجه گویان زینهار
\\
کارد از اشتاب کردی زیر خاک
&&
از خیالی که بدیدی سهمناک
\\
نک سرش با کارد در زیر زمین
&&
باز کاوید این زمین را همچنین
\\
نام این سگ هم نبشته کارد بر
&&
کرد با خواجه چنین مکر و ضرر
\\
همچنان کردند چون بشکافتند
&&
در زمین آن کارد و سر را یافتند
\\
ولوله در خلق افتاد آن زمان
&&
هر یکی زنار ببرید از میان
\\
بعد از آن گفتش بیا ای دادخواه
&&
داد خود بستان بدان روی سیاه
\\
\end{longtable}
\end{center}
