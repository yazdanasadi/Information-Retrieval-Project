\begin{center}
\section*{بخش ۲۰۱ - باقی قصهٔ مهمان آن مسجد مهمان کش و ثبات و صدق او}
\label{sec:sh201}
\addcontentsline{toc}{section}{\nameref{sec:sh201}}
\begin{longtable}{l p{0.5cm} r}
آن غریب شهر سربالا طلب
&&
گفت می‌خسپم درین مسجد بشب
\\
مسجدا گر کربلای من شوی
&&
کعبهٔ حاجت‌روای من شوی
\\
هین مرا بگذار ای بگزیده دار
&&
تا رسن‌بازی کنم منصوروار
\\
گر شدیت اندر نصیحت جبرئیل
&&
می‌نخواهد غوث در آتش خلیل
\\
جبرئیلا رو که من افروخته
&&
بهترم چون عود و عنبر سوخته
\\
جبرئیلا گر چه یاری می‌کنی
&&
چون برادر پاس داری می‌کنی
\\
ای برادر من بر آذر چابکم
&&
من نه آن جانم که گردم بیش و کم
\\
جان حیوانی فزاید از علف
&&
آتشی بود و چو هیزم شد تلف
\\
گر نگشتی هیزم او مثمر بدی
&&
تا ابد معمور و هم عامر بدی
\\
باد سوزانت این آتش بدان
&&
پرتو آتش بود نه عین آن
\\
عین آتش در اثیر آمد یقین
&&
پرتو و سایهٔ ویست اندر زمین
\\
لاجرم پرتو نپاید ز اضطراب
&&
سوی معدن باز می‌گردد شتاب
\\
قامت تو بر قرار آمد بساز
&&
سایه‌ات کوته دمی یکدم دراز
\\
زانک در پرتو نیابد کس ثبات
&&
عکسها وا گشت سوی امهات
\\
هین دهان بر بند فتنه لب گشاد
&&
خشک آر الله اعلم بالرشاد
\\
\end{longtable}
\end{center}
