\begin{center}
\section*{بخش ۲ - قصهٔ خورندگان پیل‌بچه از حرص و  ترک نصیحت ناصح}
\label{sec:sh002}
\addcontentsline{toc}{section}{\nameref{sec:sh002}}
\begin{longtable}{l p{0.5cm} r}
آن شنیدی تو که در هندوستان
&&
دید دانایی گروهی دوستان
\\
گرسنه مانده شده بی‌برگ و عور
&&
می‌رسیدند از سفر از راه دور
\\
مهر داناییش جوشید و بگفت
&&
خوش سلامیشان و چون گلبن شکفت
\\
گفت دانم کز تجوع وز خلا
&&
جمع آمد رنجتان زین کربلا
\\
لیک الله الله ای قوم جلیل
&&
تا نباشد خوردتان فرزند پیل
\\
پیل هست این سو که اکنون می‌روید
&&
پیل‌زاده مشکرید و بشنوید
\\
پیل‌بچگانند اندر راهتان
&&
صید ایشان هست بس دلخواهتان
\\
بس ضعیف‌اند و لطیف و بس سمین
&&
لیک مادر هست طالب در کمین
\\
از پی فرزند صد فرسنگ راه
&&
او بگردد در حنین و آه آه
\\
آتش و دود آید از خرطوم او
&&
الحذر زان کودک مرحوم او
\\
اولیا اطفال حق‌اند ای پسر
&&
غایبی و حاضری بس با خبر
\\
غایبی مندیش از نقصانشان
&&
کو کشد کین از برای جانشان
\\
گفت اطفال من‌اند این اولیا
&&
در غریبی فرد از کار و کیا
\\
از برای امتحان خوار و یتیم
&&
لیک اندر سر منم یار و ندیم
\\
پشت‌دار جمله عصمتهای من
&&
گوییا هستند خود اجزای من
\\
هان و هان این دلق‌پوشان من‌اند
&&
صد هزار اندر هزار و یک تن‌اند
\\
ورنه کی کردی به یک چوبی هنر
&&
موسیی فرعون را زیر و زبر
\\
ورنه کی کردی به یک نفرین بد
&&
نوح شرق و غرب را غرقاب خود
\\
بر نکندی یک دعای لوط راد
&&
جمله شهرستانشان را بی مراد
\\
گشت شهرستان چون فردوسشان
&&
دجلهٔ آب سیه رو بین نشان
\\
سوی شامست این نشان و این خبر
&&
در ره قدسش ببینی در گذر
\\
صد هزاران ز انبیای حق‌پرست
&&
خود بهر قرنی سیاستها بدست
\\
گر بگویم وین بیان افزون شود
&&
خود جگر چه بود که کهها خون شود
\\
خون شود کهها و باز آن بفسرد
&&
تو نبینی خون شدن کوری و رد
\\
طرفه کوری دوربین تیزچشم
&&
لیک از اشتر نبیند غیر پشم
\\
مو بمو بیند ز صرفه حرص انس
&&
رقص بی مقصود دارد همچو خرس
\\
رقص آنجا کن که خود را بشکنی
&&
پنبه را از ریش شهوت بر کنی
\\
رقص و جولان بر سر میدان کنند
&&
رقص اندر خون خود مردان کنند
\\
چون رهند از دست خود دستی زنند
&&
چون جهند از نقص خود رقصی کنند
\\
مطربانشان از درون دف می‌زنند
&&
بحرها در شورشان کف می‌زنند
\\
تو نبینی لیک بهر گوششان
&&
برگها بر شاخها هم کف‌زنان
\\
تو نبینی برگها را کف زدن
&&
گوش دل باید نه این گوش بدن
\\
گوش سر بر بند از هزل و دروغ
&&
تا ببینی شهر جان با فروغ
\\
سر کشد گوش محمد در سخن
&&
کش بگوید در نبی حق هو اذن
\\
سر به سر گوشست و چشم است این نبی
&&
تازه زو ما مرضعست او ما صبی
\\
این سخن پایان ندارد باز ران
&&
سوی اهل پیل و بر آغاز ران
\\
\end{longtable}
\end{center}
