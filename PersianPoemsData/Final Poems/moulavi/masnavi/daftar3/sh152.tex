\begin{center}
\section*{بخش ۱۵۲ - آمدن آن زن کافر با طفل شیرخواره به نزدیک مصطفی علیه السلام و ناطق شدن عیسی‌وار به معجزات رسول صلی الله علیه و سلم}
\label{sec:sh152}
\addcontentsline{toc}{section}{\nameref{sec:sh152}}
\begin{longtable}{l p{0.5cm} r}
هم از آن ده یک زنی از کافران
&&
سوی پیغامبر دوان شد ز امتحان
\\
پیش پیغامبر در آمد با خمار
&&
کودکی دو ماه زن را بر کنار
\\
گفت کودک سلم الله علیک
&&
یا رسول الله قد جئنا الیک
\\
مادرش از خشم گفتش هی خموش
&&
کیت افکند این شهادت را بگوش
\\
این کیت آموخت ای طفل صغیر
&&
که زبانت گشت در طفلی جریر
\\
گفت حق آموخت آنگه جبرئیل
&&
در بیان با جبرئیلم من رسیل
\\
گفت کو گفتا که بالای سرت
&&
می‌نبینی کن به بالا منظرت
\\
ایستاده بر سر تو جبرئیل
&&
مر مرا گشته به صد گونه دلیل
\\
گفت می‌بینی تو گفتا که بلی
&&
بر سرت تابان چو بدری کاملی
\\
می‌بیاموزد مرا وصف رسول
&&
زان علوم می‌رهاند زین سفول
\\
پس رسولش گفت ای طفل رضیع
&&
چیست نامت باز گو و شو مطیع
\\
گفت نامم پیش حق عبدالعزیز
&&
عبد عزی پیش این یک مشت حیز
\\
من ز عزی پاک و بیزار و بری
&&
حق آنک دادت این پیغامبری
\\
کودک دو ماهه همچون ماه بدر
&&
درس بالغ گفته چون اصحاب صدر
\\
پس حنوط آن دم ز جنت در رسید
&&
تا دماغ طفل و مادر بو کشید
\\
هر دو می‌گفتند کز خوف سقوط
&&
جان سپردن به برین بوی حنوط
\\
آن کسی را کش معرف حق بود
&&
جامد و نامیش صد صدق زند
\\
آنکسی را کش خدا حافظ بود
&&
مرغ و ماهی مر ورا حارس شود
\\
\end{longtable}
\end{center}
