\begin{center}
\section*{بخش ۱۸ - رسیدن خواجه و قومش به ده و نادیده و ناشناخته آوردن روستایی ایشان را}
\label{sec:sh018}
\addcontentsline{toc}{section}{\nameref{sec:sh018}}
\begin{longtable}{l p{0.5cm} r}
بعد ماهی چون رسیدند آن طرف
&&
بی‌نوا ایشان ستوران بی علف
\\
روستایی بین که از بدنیتی
&&
می‌کند بعد اللتیا والتی
\\
روی پنهان می‌کند زیشان بروز
&&
تا سوی باغش بنگشایند پوز
\\
آنچنان رو که همه رزق و شرست
&&
از مسلمانان نهان اولیترست
\\
رویها باشد که دیوان چون مگس
&&
بر سرش بنشسته باشند چون حرس
\\
چون ببینی روی او در تو فتند
&&
یا مبین آن رو چو دیدی خوش مخند
\\
در چنان روی خبیث عاصیه
&&
گفت یزدان نسفعن بالناصیه
\\
چون بپرسیدند و خانه‌ش یافتند
&&
همچو خویشان سوی در بشتافتند
\\
در فرو بستند اهل خانه‌اش
&&
خواجه شد زین کژروی دیوانه‌وش
\\
لیک هنگام درشتی هم نبود
&&
چون در افتادی بچه تیزی چه سود
\\
بر درش ماندند ایشان پنج روز
&&
شب بسرما روز خود خورشیدسوز
\\
نه ز غفلت بود ماندن نه خری
&&
بلک بود از اضطرار و بی‌خری
\\
با لئیمان بسته نیکان ز اضطرار
&&
شیر مرداری خورد از جوع زار
\\
او همی‌دیدش همی‌کردش سلام
&&
که فلانم من مرا اینست نام
\\
گفت باشد من چه دانم تو کیی
&&
یا پلیدی یا قرین پاکیی
\\
گفت این دم با قیامت شد شبیه
&&
تا برادر شد یفر من اخیه
\\
شرح می‌کردش که من آنم که تو
&&
لوتها خوردی ز خوان من دوتو
\\
آن فلان روزت خریدم آن متاع
&&
کل سر جاوز الاثنین شاع
\\
سر مهر ما شنیدستند خلق
&&
شرم دارد رو چو نعمت خورد حلق
\\
او همی‌گفتش چه گویی ترهات
&&
نه ترا دانم نه نام تو نه جات
\\
پنجمین شب ابر و بارانی گرفت
&&
کاسمان از بارشش دارد شگفت
\\
چون رسید آن کارد اندر استخوان
&&
حلقه زد خواجه که مهتر را بخوان
\\
چون بصد الحاح آمد سوی در
&&
گفت آخر چیست ای جان پدر
\\
گفت من آن حقها بگذاشتم
&&
ترک کردم آنچ می‌پنداشتم
\\
پنج‌ساله رنج دیدم پنج روز
&&
جان مسکینم درین گرما و سوز
\\
یک جفا از خویش و از یار و تبار
&&
در گرانی هست چون سیصد هزار
\\
زانک دل ننهاد بر جور و جفاش
&&
جانش خوگر بود با لطف و وفاش
\\
هرچه بر مردم بلا و شدتست
&&
این یقین دان کز خلاف عادتست
\\
گفت ای خورشید مهرت در زوال
&&
گر تو خونم ریختی کردم حلال
\\
امشب باران به ما ده گوشه‌ای
&&
تا بیابی در قیامت توشه‌ای
\\
گفت یک گوشه‌ست آن باغبان
&&
هست اینجا گرگ را او پاسبان
\\
در کفش تیر و کمان از بهر گرگ
&&
تا زند گر آید آن گرگ سترگ
\\
گر تو آن خدمت کنی جا آن تست
&&
ورنه جای دیگری فرمای جست
\\
گفت صد خدمت کنم تو جای ده
&&
آن کمان و تیر در کفم بنه
\\
من نخسپم حارسی رز کنم
&&
گر بر آرد گرگ سر تیرش زنم
\\
بهر حق مگذارم امشب ای دودل
&&
آب باران بر سر و در زیر گل
\\
گوشه‌ای خالی شد و او با عیال
&&
رفت آنجا جای تنگ و بی مجال
\\
چون ملخ بر همدگر گشته سوار
&&
از نهیب سیل اندر کنج غار
\\
شب همه شب جمله گویان ای خدا
&&
این سزای ما سزای ما سزا
\\
این سزای آنک شد یار خسان
&&
یا کسی کرداز برای ناکسان
\\
این سزای آنک اندر طمع خام
&&
ترک گوید خدمت خاک کرام
\\
خاک پاکان لیسی و دیوارشان
&&
بهتر از عام و رز و گلزارشان
\\
بندهٔ یک مرد روشن‌دل شوی
&&
به که بر فرق سر شاهان روی
\\
از ملوک خاک جز بانگ دهل
&&
تو نخواهی یافت ای پیک سبل
\\
شهریان خود ره‌زنان نسبت بروح
&&
روستایی کیست گیج و بی فتوح
\\
این سزای آنک بی تدبیر عقل
&&
بانگ غولی آمدش بگزید نقل
\\
چون پشیمانی ز دل شد تا شغاف
&&
زان سپس سودی ندارد اعتراف
\\
آن کمان و تیر اندر دست او
&&
گرگ را جویان همه شب سو بسو
\\
گرگ بر وی خود مسلط چون شرر
&&
گرگ جویان و ز گرگ او بی‌خبر
\\
هر پشه هر کیک چون گرگی شده
&&
اندر آن ویرانه‌شان زخمی زده
\\
فرصت آن پشه راندن هم نبود
&&
از نهیب حملهٔ گرگ عنود
\\
تا نباید گرگ آسیبی زند
&&
روستایی ریش خواجه بر کند
\\
این چنین دندان‌کنان تا نیمشب
&&
جانشان از ناف می‌آمد به لب
\\
ناگهان تمثال گرگ هشته‌ای
&&
سر بر آورد از فراز پشته‌ای
\\
تیر را بگشاد آن خواجه ز شست
&&
زد بر آن حیوان که تا افتاد پست
\\
اندر افتادن ز حیوان باد جست
&&
روستایی های کرد و کوفت دست
\\
ناجوامردا که خرکرهٔ منست
&&
گفت نه این گرگ چون آهرمنست
\\
اندرو اشکال گرگی ظاهرست
&&
شکل او از گرگی او مخبرست
\\
گفت نه بادی که جست از فرج وی
&&
می‌شناسم همچنانک آبی ز می
\\
کشته‌ای خرکره‌ام را در ریاض
&&
که مبادت بسط هرگز ز انقباض
\\
گفت نیکوتر تفحص کن شبست
&&
شخصها در شب ز ناظر محجبست
\\
شب غلط بنماید و مبدل بسی
&&
دید صایب شب ندارد هر کسی
\\
هم شب و هم ابر و هم باران ژرف
&&
این سه تاریکی غلط آرد شگرف
\\
گفت آن بر من چو روز روشنست
&&
می‌شناسم باد خرکرهٔ منست
\\
در میان بیست باد آن باد را
&&
می‌شناسم چون مسافر زاد را
\\
خواجه بر جست و بیامد ناشکفت
&&
روستایی را گریبانش گرفت
\\
کابله طرار شید آورده‌ای
&&
بنگ و افیون هر دو با هم خورده‌ای
\\
در سه تاریکی شناسی باد خر
&&
چون ندانی مر مرا ای خیره‌سر
\\
آنک داند نیمشب گوساله را
&&
چون نداند همره ده‌ساله را
\\
خویشتن را عارف و واله کنی
&&
خاک در چشم مروت می‌زنی
\\
که مرا از خویش هم آگاه نیست
&&
در دلم گنجای جز الله نیست
\\
آنچ دی خوردم از آنم یاد نیست
&&
این دل از غیر تحیر شاد نیست
\\
عاقل و مجنون حقم یاد آر
&&
در چنین بی‌خویشیم معذور دار
\\
آنک مرداری خورد یعنی نبید
&&
شرع او را سوی معذوران کشید
\\
مست و بنگی را طلاق و بیع نیست
&&
همچو طفلست او معاف و معتقیست
\\
مستیی کید ز بوی شاه فرد
&&
صد خم می در سر و مغز آن نکرد
\\
پس برو تکلیف چون باشد روا
&&
اسب ساقط گشت و شد بی دست و پا
\\
بار کی نهد در جهان خرکره را
&&
درس کی دهد پارسی بومره را
\\
بار بر گیرند چون آمد عرج
&&
گفت حق لیس علی الاعمی حرج
\\
سوی خود اعمی شدم از حق بصیر
&&
پس معافم از قلیل و از کثیر
\\
لاف درویشی زنی و بی‌خودی
&&
های هوی مستیان ایزدی
\\
که زمین را من ندانم ز آسمان
&&
امتحانت کرد غیرت امتحان
\\
باد خرکرهٔ چنین رسوات کرد
&&
هستی نفی ترا اثبات کرد
\\
این چنین رسوا کند حق شید را
&&
این چنین گیرد رمیده‌صید را
\\
صد هزاران امتحانست ای پسر
&&
هر که گوید من شدم سرهنگ در
\\
گر نداند عامه او را ز امتحان
&&
پختگان راه جویندش نشان
\\
چون کند دعوی خیاطی خسی
&&
افکند در پیش او شه اطلسی
\\
که ببر این را بغلطاق فراخ
&&
ز امتحان پیدا شود او را دو شاخ
\\
گر نبودی امتحان هر بدی
&&
هر مخنث در وغا رستم بدی
\\
خود مخنث را زره پوشیده گیر
&&
چون ببیند زخم گردد چون اسیر
\\
مست حق هشیار چون شد از دبور
&&
مست حق ناید به خود تا نفخ صور
\\
بادهٔ حق راست باشد بی دروغ
&&
دوغ خوردی دوغ خوردی دوغ دوغ
\\
ساختی خود را جنید و بایزید
&&
رو که نشناسم تبر را از کلید
\\
بدرگی و منبلی و حرص و آز
&&
چون کنی پنهان بشید ای مکرساز
\\
خویش را منصور حلاجی کنی
&&
آتشی در پنبهٔ یاران زنی
\\
که بنشناسم عمر از بولهب
&&
باد کرهٔ خود شناسم نیمشب
\\
ای خری کین از تو خر باور کند
&&
خویش را بهر تو کور و کر کند
\\
خویش را از ره‌روان کمتر شمر
&&
تو حریف ره‌ریانی گه مخور
\\
باز پر از شید سوی عقل تاز
&&
کی پرد بر آسمان پر مجاز
\\
خویشتن را عاشق حق ساختی
&&
عشق با دیو سیاهی باختی
\\
عاشق و معشوق را در رستخیز
&&
دو بدو بندند و پیش آرند تیز
\\
تو چه خود را گیج و بی‌خود کرده‌ای
&&
خون رز کو خون ما را خورده‌ای
\\
رو که نشناسم ترا از من بجه
&&
عارف بی‌خویشم و بهلول ده
\\
تو توهم می‌کنی از قرب حق
&&
که طبق‌گر دور نبود از طبق
\\
این نمی‌بینی که قرب اولیا
&&
صد کرامت دارد و کار و کیا
\\
آهن از داوود مومی می‌شود
&&
موم در دستت چو آهن می‌بود
\\
قرب خلق و رزق بر جمله‌ست عام
&&
قرب وحی عشق دارند این کرام
\\
قرب بر انواع باشد ای پدر
&&
می‌زند خورشید بر کهسار و زر
\\
لیک قربی هست با زر شید را
&&
که از آن آگه نباشد بید را
\\
شاخ خشک و تر قریب آفتاب
&&
آفتاب از هر دو کی دارد حجاب
\\
لیک کو آن قربت شاخ طری
&&
که ثمار پخته از وی می‌خوری
\\
شاخ خشک از قربت آن آفتاب
&&
غیر زوتر خشک گشتن گو بیاب
\\
آنچنان مستی مباش ای بی‌خرد
&&
که به عقل آید پشیمانی خورد
\\
بلک از آن مستان که چون می می‌خورند
&&
عقلهای پخته حسرت می‌برند
\\
ای گرفته همچو گربه موش پیر
&&
گر از آن می شیرگیری شیر گیر
\\
ای بخورده از خیالی جام هیچ
&&
همچو مستان حقایق بر مپیچ
\\
می‌فتی این سو و آن سو مست‌وار
&&
ای تو این سو نیستت زان سو گذار
\\
گر بدان سو راه یابی بعد از آن
&&
گه بدین سو گه بدان سو سر فشان
\\
جمله این سویی از آن سو کپ مزن
&&
چون نداری مرگ هرزه جان مکن
\\
آن خضرجان کز اجل نهراسد او
&&
شاید ار مخلوق را نشناسد او
\\
کام از ذوق توهم خوش کنی
&&
در دمی در خیک خود پرش کنی
\\
پس به یک سوزن تهی گردی ز باد
&&
این چنین فربه تن عاقل مباد
\\
کوزه‌ها سازی ز برف اندر شتا
&&
کی کند چون آب بیند آن وفا
\\
\end{longtable}
\end{center}
