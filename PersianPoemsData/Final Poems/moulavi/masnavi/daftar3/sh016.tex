\begin{center}
\section*{بخش ۱۶ - رفتن خواجه و قومش به سوی ده}
\label{sec:sh016}
\addcontentsline{toc}{section}{\nameref{sec:sh016}}
\begin{longtable}{l p{0.5cm} r}
خواجه و بچگان جهازی ساختند
&&
بر ستوران جانب ده تاختند
\\
شادمانه سوی صحرا راندند
&&
سافروا کی تغنموا بر خواندند
\\
کز سفرها ماه کیخسرو شود
&&
بی سفرها ماه کی خسرو شود
\\
از سفر بیدق شود فرزین راد
&&
وز سفر یابید یوسف صد مراد
\\
روز روی از آفتابی سوختند
&&
شب ز اختر راه می‌آموختند
\\
خوب گشته پیش ایشان راه زشت
&&
از نشاط ده شده ره چون بهشت
\\
تلخ از شیرین‌لبان خوش می‌شود
&&
خار از گلزار دلکش می‌شود
\\
حنظل از معشوق خرما می‌شود
&&
خانه از همخانه صحرا می‌شود
\\
ای بسا از نازنینان خارکش
&&
بر امید گل‌عذار ماه‌وش
\\
ای بسا حمال گشته پشت‌ریش
&&
از برای دلبر مه‌روی خویش
\\
کرده آهنگر جمال خود سیاه
&&
تا که شب آید ببوسد روی ماه
\\
خواجه تا شب بر دکانی چار میخ
&&
زانک سروی در دلش کردست بیخ
\\
تاجری دریا و خشکی می‌رود
&&
آن بمهر خانه‌شینی می‌دود
\\
هر که را با مرده سودایی بود
&&
بر امید زنده‌سیمایی بود
\\
آن دروگر روی آورده به چوب
&&
بر امید خدمت مه‌روی خوب
\\
بر امید زنده‌ای کن اجتهاد
&&
کو نگردد بعد روزی دو جماد
\\
مونسی مگزین خسی را از خسی
&&
عاریت باشد درو آن مونسی
\\
انس تو با مادر و بابا کجاست
&&
گر به جز حق مونسانت را وفاست
\\
انس تو با دایه و لالا چه شد
&&
گر کسی شاید بغیر حق عضد
\\
انس تو با شیر و با پستان نماند
&&
نفرت تو از دبیرستان نماند
\\
آن شعاعی بود بر دیوارشان
&&
جانب خورشید وا رفت آن نشان
\\
بر هر آن چیزی که افتد آن شعاع
&&
تو بر آن هم عاشق آیی ای شجاع
\\
عشق تو بر هر چه آن موجود بود
&&
آن ز وصف حق زر اندود بود
\\
چون زری با اصل رفت و مس بماند
&&
طبع سیر آمد طلاق او براند
\\
از زر اندود صفاتش پا بکش
&&
از جهالت قلب را کم گوی خوش
\\
کان خوشی در قلبها عاریتست
&&
زیر زینت مایهٔ بی زینتست
\\
زر ز روی قلب در کان می‌رود
&&
سوی آن کان رو تو هم کان می‌رود
\\
نور از دیوار تا خور می‌رود
&&
تو بدان خور رو که در خور می‌رود
\\
زین سپس پستان تو آب از آسمان
&&
چون ندیدی تو وفا در ناودان
\\
معدن دنبه نباشد دام گرگ
&&
کی شناسد معدن آن گرگ سترگ
\\
زر گمان بردند بسته در گره
&&
می‌شتابیدند مغروران به ده
\\
همچنین خندان و رقصان می‌شدند
&&
سوی آن دولاب چرخی می‌زدند
\\
چون همی‌دیدند مرغی می‌پرید
&&
جانب ده صبر جامه می‌درید
\\
هر که می‌آمد ز ده از سوی او
&&
بوسه می‌دادند خوش بر روی او
\\
گر تو روی یار ما را دیده‌ای
&&
پس تو جان را جان و ما را دیده‌ای
\\
\end{longtable}
\end{center}
