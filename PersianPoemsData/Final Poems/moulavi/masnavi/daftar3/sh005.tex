\begin{center}
\section*{بخش ۵ - بیان آنک خطای محبان بهترست از  صواب بیگانگان بر محبوب}
\label{sec:sh005}
\addcontentsline{toc}{section}{\nameref{sec:sh005}}
\begin{longtable}{l p{0.5cm} r}
آن بلال صدق در بانگ نماز
&&
حی را هی همی‌خواند از نیاز
\\
تا بگفتند ای پیمبر نیست راست
&&
این خطا اکنون که آغاز بناست
\\
ای نبی و ای رسول کردگار
&&
یک مؤذن کو بود افصح بیار
\\
عیب باشد اول دین و صلاح
&&
لحن خواندن لفظ حی عل فلاح
\\
خشم پیغمبر بجوشید و بگفت
&&
یک دو رمزی از عنایات نهفت
\\
کای خسان نزد خدا هی بلال
&&
بهتر از صد حی و خی و قیل و قال
\\
وا مشورانید تا من رازتان
&&
وا نگویم آخر و آغازتان
\\
گر نداری تو دم خوش در دعا
&&
رو دعا می‌خواه ز اخوان صفا
\\
\end{longtable}
\end{center}
