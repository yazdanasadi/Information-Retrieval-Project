\begin{center}
\section*{بخش ۲۰۵ - تشبیه صورت اولیا و صورت کلام اولیا به صورت عصای موسی و صورت افسون عیسی علیهما السلام}
\label{sec:sh205}
\addcontentsline{toc}{section}{\nameref{sec:sh205}}
\begin{longtable}{l p{0.5cm} r}
آدمی همچون عصای موسی‌است
&&
آدمی همچون فسون عیسی‌است
\\
در کف حق بهر داد و بهر زین
&&
قلب مومن هست بین اصبعین
\\
ظاهرش چوبی ولیکن پیش او
&&
کون یک لقمه چو بگشاید گلو
\\
تو مبین ز افسون عیسی حرف و صوت
&&
آن ببین کز وی گریزان گشت موت
\\
تو مبین ز افسونش آن لهجات پست
&&
آن نگر که مرده بر جست و نشست
\\
تو مبین مر آن عصا را سهل یافت
&&
آن ببین که بحر خضرا را شکافت
\\
تو ز دوری دیده‌ای چتر سیاه
&&
یک قدم فا پیش نه بنگر سپاه
\\
تو ز دوری می‌نبینی جز که گرد
&&
اندکی پیش آ ببین در گرد مرد
\\
دیده‌ها را گرد او روشن کند
&&
کوهها را مردی او بر کند
\\
چون بر آمد موسی از اقصای دشت
&&
کوه طور از مقدمش رقاص گشت
\\
\end{longtable}
\end{center}
