\begin{center}
\section*{بخش ۱۲۲ - معجزه خواستن قوم از پیغامبران}
\label{sec:sh122}
\addcontentsline{toc}{section}{\nameref{sec:sh122}}
\begin{longtable}{l p{0.5cm} r}
قوم گفتند ای گروه مدعی
&&
کو گواه علم طب و نافعی
\\
چون شما بسته همین خواب و خورید
&&
همچو ما باشید در ده می‌چرید
\\
چون شما در دام این آب و گلید
&&
کی شما صیاد سیمرغ دلید
\\
حب جاه و سروری دارد بر آن
&&
که شمارد خویش از پیغامبران
\\
ما نخواهیم این چنین لاف و دروغ
&&
کردن اندر گوش و افتادن بدوغ
\\
انبیا گفتند کین زان علتست
&&
مایهٔ کوری حجاب ریتست
\\
دعوی ما را شنیدیت و شما
&&
می‌نبینید این گهر در دست ما
\\
امتحانست این گهر مر خلق را
&&
ماش گردانیم گرد چشمها
\\
هر که گوید کو گوا گفتش گواست
&&
کو نمی‌بیند گهر حبس عماست
\\
آفتابی در سخن آمد که خیز
&&
که بر آمد روز بر جه کم ستیز
\\
تو بگویی آفتابا کو گواه
&&
گویدت ای کور از حق دیده خواه
\\
روز روشن هر که او جوید چراغ
&&
عین جستن کوریش دارد بلاغ
\\
ور نمی‌بینی گمانی برده‌ای
&&
که صباحست و تو اندر پرده‌ای
\\
کوری خود را مکن زین گفت فاش
&&
خامش و در انتظار فضل باش
\\
در میان روز گفتن روز کو
&&
خویش رسوا کردنست ای روزجو
\\
صبر و خاموشی جذوب رحمتست
&&
وین نشان جستن نشان علتست
\\
انصتوا بپذیر تا بر جان تو
&&
آید از جانان جزای انصتوا
\\
گر نخواهی نکس پیش این طبیب
&&
بر زمین زن زر و سر را ای لبیب
\\
گفت افزون را تو بفروش و بخر
&&
بذل جان و بذل جاه و بذل زر
\\
تا ثنای تو بگوید فضل هو
&&
که حسد آرد فلک بر جاه تو
\\
چون طبیبان را نگه دارید دل
&&
خود ببینید و شوید ازخود خجل
\\
دفع این کوری بدست خلق نیست
&&
لیک اکرام طبیبان از هدیست
\\
این طبیبان را به جان بنده شوید
&&
تا به مشک و عنبر آکنده شوید
\\
\end{longtable}
\end{center}
