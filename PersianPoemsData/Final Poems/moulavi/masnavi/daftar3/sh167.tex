\begin{center}
\section*{بخش ۱۶۷ - حیله دفع مغبون شدن در بیع و شرا}
\label{sec:sh167}
\addcontentsline{toc}{section}{\nameref{sec:sh167}}
\begin{longtable}{l p{0.5cm} r}
آن یکی یاری پیمبر را بگفت
&&
که منم در بیعها با غبن جفت
\\
مکر هر کس کو فروشد یا خرد
&&
همچو سحرست و ز راهم می‌برد
\\
گفت در بیعی که ترسی از غرار
&&
شرط کن سه روز خود را اختیار
\\
که تانی هست از رحمان یقین
&&
هست تعجیلت ز شیطان لعین
\\
پیش سگ چون لقمه نان افکنی
&&
بو کند آنگه خورد ای معتنی
\\
او ببینی بو کند ما با خرد
&&
هم ببوییمش به عقل منتقد
\\
با تانی گشت موجود از خدا
&&
تابه شش روز این زمین و چرخها
\\
ورنه قادر بود کو کن فیکون
&&
صد زمین و چرخ آوردی برون
\\
آدمی را اندک اندک آن همام
&&
تا چهل سالش کند مرد تمام
\\
گرچه قادر بود کاندر یک نفس
&&
از عدم پران کند پنجاه کس
\\
عیسی قادر بود کو از یک دعا
&&
بی توقف بر جهاند مرده را
\\
خالق عیسی بنتواند که او
&&
بی توقف مردم آرد تو بتو
\\
این تانی از پی تعلیم تست
&&
که طلب آهسته باید بی سکست
\\
جو یکی کوچک که دایم می‌رود
&&
نه نجس گردد نه گنده می‌شود
\\
زین تانی زاید اقبال و سرور
&&
این تانی بیضه دولت چون طیور
\\
مرغ کی ماند به بیضه‌ای عنید
&&
گرچه از بیضه همی آید پدید
\\
باش تا اجزای تو چون بیضه‌ها
&&
مرغها زایند اندر انتها
\\
بیضهٔ مار ارچه ماند در شبه
&&
بیضه گنجشک را دورست ره
\\
دانهٔ آبی به دانه سیب نیز
&&
گرچه ماند فرقها دان ای عزیز
\\
برگها هم‌رنگ باشد در نظر
&&
میوه‌ها هر یک بود نوعی دگر
\\
برگهای جسمها ماننده‌اند
&&
لیک هر جانی بریعی زنده‌اند
\\
خلق در بازار یکسان می‌روند
&&
آن یکی در ذوق و دیگر دردمند
\\
همچنان در مرگ یکسان می‌رویم
&&
نیم در خسران و نیمی خسرویم
\\
\end{longtable}
\end{center}
