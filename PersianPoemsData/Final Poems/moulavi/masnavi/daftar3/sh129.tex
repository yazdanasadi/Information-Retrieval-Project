\begin{center}
\section*{بخش ۱۲۹ - جواب آن مثل کی منکران گفتند از رسالت خرگوش پیغام به پیل از ماه آسمان}
\label{sec:sh129}
\addcontentsline{toc}{section}{\nameref{sec:sh129}}
\begin{longtable}{l p{0.5cm} r}
سر آن خرگوش دان دیو فضول
&&
که به پیش نفس تو آمد رسول
\\
تا که نفس گول را محروم کرد
&&
ز آب حیوانی که از وی خضر خورد
\\
بازگونه کرده‌ای معنیش را
&&
کفر گفتی مستعد شو نیش را
\\
اضطراب ماه گفتی در زلال
&&
که بترسانید پیلان را شغال
\\
قصهٔ خرگوش و پیل آری و آب
&&
خشیت پیلان ز مه در اضطراب
\\
این چه ماند آخر ای کوران خام
&&
با مهی که شد زبونش خاص و عام
\\
چه مه و چه آفتاب و چه فلک
&&
چه عقول و چه نفوس و چه ملک
\\
آفتاب آفتاب آفتاب
&&
این چه می‌گویم مگر هستم بخواب
\\
صد هزاران شهر را خشم شهان
&&
سرنگون کردست ای بد گم‌رهان
\\
کوه بر خود می‌شکافد صد شکاف
&&
آفتابی از کسوفش در شغاف
\\
خشم مردان خشک گرداند سحاب
&&
خشم دلها کرد عالمها خراب
\\
بنگرید ای مردگان بی حنوط
&&
در سیاستگاه شهرستان لوط
\\
پیل خود چه بود که سه مرغ پران
&&
کوفتند آن پیلکان را استخوان
\\
اضعف مرغان ابابیلست و او
&&
پیل را بدرید و نپذیرد رفو
\\
کیست کو نشنید آن طوفان نوح
&&
یا مصاف لشکر فرعون و روح
\\
روحشان بشکست و اندر آب ریخت
&&
ذره ذره آبشان بر می‌گسیخت
\\
کیست کو نشنید احوال ثمود
&&
و آنک صرصر عادیان را می‌ربود
\\
چشم باری در چنان پیلان گشا
&&
که بدندی پیل‌کش اندر وغا
\\
آنچنان پیلان و شاهان ظلوم
&&
زیر خشم دل همیشه در رجوم
\\
تا ابد از ظلمتی در ظلمتی
&&
می‌روند و نیست غوثی رحمتی
\\
نام نیک و بد مگر نشنیده‌اید
&&
جمله دیدند و شما نادیده‌اید
\\
دیده را نادیده می‌آرید لیک
&&
چشمتان را وا گشاید مرگ نیک
\\
گیر عالم پر بود خورشید و نور
&&
چون روی در ظلمتی مانند گور
\\
بی نصیب آیی از آن نور عظیم
&&
بسته‌روزن باشی از ماه کریم
\\
تو درون چاه رفتستی ز کاخ
&&
چه گنه دارد جهانهای فراخ
\\
جان که اندر وصف گرگی ماند او
&&
چون ببیند روی یوسف را بگو
\\
لحن داودی به سنگ و که رسید
&&
گوش آن سنگین دلانش کم شنید
\\
آفرین بر عقل و بر انصاف باد
&&
هر زمان والله اعلم بالرشاد
\\
صدقوا رسلا کراما یا سبا
&&
صدقوا روحا سباها من سبا
\\
صدقوهم هم شموس طالعه
&&
یومنوکم من مخازی القارعه
\\
صدقوهم هم بدور زاهره
&&
قبل ان یلقوکم بالساهره
\\
صدقوهم هم مصابیح الدجی
&&
اکرموهم هم مفاتیح الرجا
\\
صدقوا من لیس یرجو خیرکم
&&
لا تضلوا لا تصدوا غیرکم
\\
پارسی گوییم هین تازی بهل
&&
هندوی آن ترک باش ای آب و گل
\\
هین گواهیهای شاهان بشنوید
&&
بگرویدند آسمانها بگروید
\\
\end{longtable}
\end{center}
