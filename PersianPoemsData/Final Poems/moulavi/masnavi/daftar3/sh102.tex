\begin{center}
\section*{بخش ۱۰۲ - دعا و شفاعت دقوقی در خلاص کشتی}
\label{sec:sh102}
\addcontentsline{toc}{section}{\nameref{sec:sh102}}
\begin{longtable}{l p{0.5cm} r}
چون دقوقی آن قیامت را بدید
&&
رحم او جوشید و اشک او دوید
\\
گفت یا رب منگر اندر فعلشان
&&
دستشان گیر ای شه نیکو نشان
\\
خوش سلامتشان به ساحل با زبر
&&
ای رسیده دست تو در بحر و بر
\\
ای کریم و ای رحیم سرمدی
&&
در گذار از بدسگالان این بدی
\\
ای بداده رایگان صد چشم و گوش
&&
بی ز رشوت بخش کرده عقل و هوش
\\
پیش از استحقاق بخشیده عطا
&&
دیده از ما جمله کفران و خطا
\\
ای عظیم از ما گناهان عظیم
&&
تو توانی عفو کردن در حریم
\\
ما ز آز و حرص خود را سوختیم
&&
وین دعا را هم ز تو آموختیم
\\
حرمت آن که دعا آموختی
&&
در چنین ظلمت چراغ افروختی
\\
همچنین می‌رفت بر لفظش دعا
&&
آن زمان چون مادران با وفا
\\
اشک می‌رفت از دو چشمش و آن دعا
&&
بی خود از وی می بر آمد بر سما
\\
آن دعای بی خودان خود دیگرست
&&
آن دعا زو نیست گفت داورست
\\
آن دعا حق می‌کند چون او فناست
&&
آن دعا و آن اجابت از خداست
\\
واسطهٔ مخلوق نه اندر میان
&&
بی‌خبر زان لابه کردن جسم و جان
\\
بندگان حق رحیم و بردبار
&&
خوی حق دارند در اصلاح کار
\\
مهربان بی‌رشوتان یاری‌گران
&&
در مقام سخت و در روز گران
\\
هین بجو این قوم را ای مبتلا
&&
هین غنیمت دارشان پیش از بلا
\\
رست کشتی از دم آن پهلوان
&&
واهل کشتی را بجهد خود گمان
\\
که مگر بازوی ایشان در حذر
&&
بر هدف انداخت تیری از هنر
\\
پا رهاند روبهان را در شکار
&&
و آن زدم دانند روباهان غرار
\\
عشقها با دم خود بازند کین
&&
می‌رهاند جان ما را در کمین
\\
روبها پا را نگه دار از کلوخ
&&
پا چو نبود دم چه سود ای چشم‌شوخ
\\
ما چو روباهان و پای ما کرام
&&
می‌رهاندمان ز صدگون انتقام
\\
حیلهٔ باریک ما چون دم ماست
&&
عشقها بازیم با دم چپ و راست
\\
دم بجنبانیم ز استدلال و مکر
&&
تا که حیران ماند از ما زید و بکر
\\
طالب حیرانی خلقان شدیم
&&
دست طمع اندر الوهیت زدیم
\\
تا بافسون مالک دلها شویم
&&
این نمی‌بینیم ما کاندر گویم
\\
در گوی و در چهی ای قلتبان
&&
دست وا دار از سبال دیگران
\\
چون به بستانی رسی زیبا و خوش
&&
بعد از آن دامان خلقان گیر و کش
\\
ای مقیم حبس چار و پنج و شش
&&
نغز جایی دیگران را هم بکش
\\
ای چو خربنده حریف کون خر
&&
بوسه گاهی یافتی ما را ببر
\\
چون ندادت بندگی دوست دست
&&
میل شاهی از کجاات خاستست
\\
در هوای آنک گویندت زهی
&&
بسته‌ای در گردن جانت زهی
\\
روبها این دم حیلت را بهل
&&
وقف کن دل بر خداوندان دل
\\
در پناه شیر کم ناید کباب
&&
روبها تو سوی جیفه کم شتاب
\\
تو دلا منظور حق آنگه شوی
&&
که چو جزوی سوی کل خود روی
\\
حق همی‌گوید نظرمان در دلست
&&
نیست بر صورت که آن آب و گلست
\\
تو همی‌گویی مرا دل نیز هست
&&
دل فراز عرش باشد نه به پست
\\
در گل تیره یقین هم آب هست
&&
لیک زان آبت نشاید آب‌دست
\\
زانک گر آبست مغلوب گلست
&&
پس دل خود را مگو کین هم دلست
\\
آن دلی کز آسمانها برترست
&&
آن دل ابدال یا پیغامبرست
\\
پاک گشته آن ز گل صافی شده
&&
در فزونی آمده وافی شده
\\
ترک گل کرده سوی بحر آمده
&&
رسته از زندان گل بحری شده
\\
آب ما محبوس گل ماندست هین
&&
بحر رحمت جذب کن ما را ز طین
\\
بحر گوید من ترا در خود کشم
&&
لیک می‌لافی که من آب خوشم
\\
لاف تو محروم می‌دارد ترا
&&
ترک آن پنداشت کن در من درآ
\\
آب گل خواهد که در دریا رود
&&
گل گرفته پای آب و می‌کشد
\\
گر رهاند پای خود از دست گل
&&
گل بماند خشک و او شد مستقل
\\
آن کشیدن چیست از گل آب را
&&
جذب تو نقل و شراب ناب را
\\
همچنین هر شهوتی اندر جهان
&&
خواه مال و خواه جاه و خواه نان
\\
هر یکی زینها ترا مستی کند
&&
چون نیابی آن خمارت می‌زند
\\
این خمار غم دلیل آن شدست
&&
که بدان مفقود مستی‌ات بدست
\\
جز به اندازهٔ ضرورت زین مگیر
&&
تا نگردد غالب و بر تو امیر
\\
سر کشیدی تو که من صاحب‌دلم
&&
حاجت غیری ندارم واصلم
\\
آنچنانک آب در گل سر کشد
&&
که منم آب و چرا جویم مدد
\\
دل تو این آلوده را پنداشتی
&&
لاجرم دل ز اهل دل برداشتی
\\
خود روا داری که آن دل باشد این
&&
کو بود در عشق شیر و انگبین
\\
لطف شیر و انگبین عکس دلست
&&
هر خوشی را آن خوش از دل حاصلست
\\
پس بود دل جوهر و عالم عرض
&&
سایهٔ دل چون بود دل را غرض
\\
آن دلی کو عاشق مالست و جاه
&&
یا زبون این گل و آب سیاه
\\
یا خیالاتی که در ظلمات او
&&
می‌پرستدشان برای گفت و گو
\\
دل نباشد غیر آن دریای نور
&&
دل نظرگاه خدا وانگاه کور
\\
نه دل اندر صد هزاران خاص و عام
&&
در یکی باشد کدامست آن کدام
\\
ریزهٔ دل را بهل دل را بجو
&&
تا شود آن ریزه چون کوهی ازو
\\
دل محیطست اندرین خطهٔ وجود
&&
زر همی‌افشاند از احسان و جود
\\
از سلام حق سلامیها نثار
&&
می‌کند بر اهل عالم اختیار
\\
هر که را دامن درستست و معد
&&
آن نثار دل بر آنکس می‌رسد
\\
دامن تو آن نیازست و حضور
&&
هین منه در دامن آن سنگ فجور
\\
تا ندرد دامنت زان سنگها
&&
تا بدانی نقد را از رنگها
\\
سنگ پر کردی تو دامن از جهان
&&
هم ز سنگ سیم و زر چون کودکان
\\
از خیال سیم و زر چون زر نبود
&&
دامن صدقت درید و غم فزود
\\
کی نماید کودکان را سنگ سنگ
&&
تا نگیرد عقل دامنشان به چنگ
\\
پیر عقل آمد نه آن موی سپید
&&
مو نمی‌گنجد درین بخت و امید
\\
\end{longtable}
\end{center}
