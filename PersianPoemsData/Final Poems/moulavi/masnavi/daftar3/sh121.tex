\begin{center}
\section*{بخش ۱۲۱ - آمدن پیغامبران حق به نصیحت اهل سبا}
\label{sec:sh121}
\addcontentsline{toc}{section}{\nameref{sec:sh121}}
\begin{longtable}{l p{0.5cm} r}
سیزده پیغامبر آنجا آمدند
&&
گم‌رهان را جمله رهبر می‌شدند
\\
که هله نعمت فزون شد شکر کو
&&
مرکب شکر ار بخسپد حرکوا
\\
شکر منعم واجب آید در خرد
&&
ورنه بگشاید در خشم ابد
\\
هین کرم بینید وین خود کس کند
&&
کز چنین نعمت به شکری بس کند
\\
سر ببخشد شکر خواهد سجده‌ای
&&
پا ببخشد شکر خواهد قعده‌ای
\\
قوم گفته شکر ما را برد غول
&&
ما شدیم از شکر و از نعمت ملول
\\
ما چنان پژمرده گشتیم ازعطا
&&
که نه طاعتمان خوش آید نه خطا
\\
ما نمی‌خواهیم نعمتها و باغ
&&
ما نمی‌خواهیم اسباب و فراغ
\\
انبیا گفتند در دل علتیست
&&
که از آن در حق‌شناسی آفتیست
\\
نعمت از وی جملگی علت شود
&&
طعمه در بیمار کی قوت شود
\\
چند خوش پیش تو آمد ای مصر
&&
جمله ناخوش گشت و صاف او کدر
\\
تو عدو این خوشیها آمدی
&&
گشت ناخوش هر چه بر وی کف زدی
\\
هر که اوشد آشنا و یار تو
&&
شد حقیر و خوار در دیدار تو
\\
هر که او بیگانه باشد با تو هم
&&
پیش تو او بس مه‌است و محترم
\\
این هم از تاثیر آن بیماریست
&&
زهر او در جمله جفتان ساریست
\\
دفع آن علت بباید کرد زود
&&
که شکر با آن حدث خواهد نمود
\\
هر خوشی کاید به تو ناخوش شود
&&
آب حیوان گر رسد آتش شود
\\
کیمیای مرگ و جسکست آن صفت
&&
مرگ گردد زان حیاتت عاقبت
\\
بس غدایی که ز وی دل زنده شد
&&
چون بیامد در تن تو گنده شد
\\
بس عزیزی که بناز اشکار شد
&&
چون شکارت شد بر تو خوار شد
\\
آشنایی عقل با عقل از صفا
&&
چون شود هر دم فزون باشد ولا
\\
آشنایی نفس با هر نفس پست
&&
تو یقین می‌دان که دم دم کمترست
\\
زانک نفسش گرد علت می‌تند
&&
معرفت را زود فاسد می‌کند
\\
گر نخواهی دوست را فردا نفیر
&&
دوستی با عاقل و با عقل گیر
\\
از سموم نفس چون با علتی
&&
هر چه گیری تو مرض را آلتی
\\
گر بگیری گوهری سنگی شود
&&
ور بگیری مهر دل جنگی شود
\\
ور بگیری نکتهٔ بکری لطیف
&&
بعد درکت گشت بی‌ذوق و کثیف
\\
که من این را بس شنیدم کهنه شد
&&
چیز دیگر گو به جز آن ای عضد
\\
چیز دیگر تازه و نو گفته گیر
&&
باز فردا زان شوی سیر و نفیر
\\
دفع علت کن چو علت خو شود
&&
هرحدیثی کهنه پیشت نو شود
\\
تا که از کهنه برآرد برگ نو
&&
بشکفاند کهنه صد خوشه ز گو
\\
ما طبیبانیم شاگردان حق
&&
بحر قلزم دید ما را فانفلق
\\
آن طبیبان طبیعت دیگرند
&&
که به دل از راه نبضی بنگرند
\\
ما به دل بی واسطه خوش بنگریم
&&
کز فراست ما به عالی منظریم
\\
آن طبیبان غذااند و ثمار
&&
جان حیوانی بدیشان استوار
\\
ما طبیبان فعالیم و مقال
&&
ملهم ما پرتو نور جلال
\\
کین چنین فعلی ترا نافع بود
&&
و آنچنان فعلی ز ره قاطع بود
\\
اینچنین قولی ترا پیش آورد
&&
و آنچنان قولی ترا نیش آورد
\\
آن طبیبان را بود بولی دلیل
&&
وین دلیل ما بود وحی جلیل
\\
دست‌مزدی می نخواهیم از کسی
&&
دست‌مزد ما رسد از حق بسی
\\
هین صلا بیماری ناسور را
&&
داروی ما یک بیک رنجور را
\\
\end{longtable}
\end{center}
