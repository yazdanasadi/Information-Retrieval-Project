\begin{center}
\section*{بخش ۶۶ - رفتن مادران کودکان به عیادت اوستاد}
\label{sec:sh066}
\addcontentsline{toc}{section}{\nameref{sec:sh066}}
\begin{longtable}{l p{0.5cm} r}
بامدادان آمدند آن مادران
&&
خفته استا همچو بیمار گران
\\
هم عرق کرده ز بسیاری لحاف
&&
سر ببسته رو کشیده در سجاف
\\
آه آهی می‌کند آهسته او
&&
جملگان گشتند هم لا حول‌گو
\\
خیر باشد اوستاد این درد سر
&&
جان تو ما را نبودست زین خبر
\\
گفت من هم بی‌خبر بودم ازین
&&
آگهم مادر غران کردند هین
\\
من بدم غافل بشغل قال و قیل
&&
بود در باطن چنین رنجی ثقیل
\\
چون بجد مشغول باشد آدمی
&&
او ز دید رنج خود باشد عمی
\\
از زنان مصر یوسف شد سمر
&&
که ز مشغولی بشد زیشان خبر
\\
پاره پاره کرده ساعدهای خویش
&&
روح واله که نه پس بیند نه پیش
\\
ای بسا مرد شجاع اندر حراب
&&
که ببرد دست یا پایش ضراب
\\
او همان دست آورد در گیر و دار
&&
بر گمان آنک هست او بر قرار
\\
خود ببیند دست رفته در ضرر
&&
خون ازو بسیار رفته بی‌خبر
\\
\end{longtable}
\end{center}
