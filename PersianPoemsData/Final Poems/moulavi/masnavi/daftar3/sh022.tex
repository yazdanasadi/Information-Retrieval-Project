\begin{center}
\section*{بخش ۲۲ - دعوی طاوسی کردن آن شغال کی در  خم صباغ افتاده بود}
\label{sec:sh022}
\addcontentsline{toc}{section}{\nameref{sec:sh022}}
\begin{longtable}{l p{0.5cm} r}
و آن شغال رنگ‌رنگ آمد نهفت
&&
بر بناگوش ملامت‌گر بکفت
\\
بنگر آخر در من و در رنگ من
&&
یک صنم چون من ندارد خود شمن
\\
چون گلستان گشته‌ام صد رنگ و خوش
&&
مر مرا سجده کن از من سر مکش
\\
کر و فر و آب و تاب و رنگ بین
&&
فخر دنیا خوان مرا و رکن دین
\\
مظهر لطف خدایی گشته‌ام
&&
لوح شرح کبریایی گشته‌ام
\\
ای شغالان هین مخوانیدم شغال
&&
کی شغالی را بود چندین جمال
\\
آن شغالان آمدند آنجا بجمع
&&
همچو پروانه به گرداگرد شمع
\\
پس چه خوانیمت بگو ای جوهری
&&
گفت طاوس نر چون مشتری
\\
پس بگفتندش که طاوسان جان
&&
جلوه‌ها دارند اندر گلستان
\\
تو چنان جلوه کنی گفتا که نی
&&
بادیه نارفته چون کوبم منی
\\
بانگ طاووسان کنی گفتا که لا
&&
پس نه‌ای طاووس خواجه بوالعلا
\\
خلعت طاووس آید ز آسمان
&&
کی رسی از رنگ و دعویها بدان
\\
\end{longtable}
\end{center}
