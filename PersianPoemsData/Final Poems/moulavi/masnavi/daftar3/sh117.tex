\begin{center}
\section*{بخش ۱۱۷ - گریختن عیسی علیه السلام فراز کوه از احمقان}
\label{sec:sh117}
\addcontentsline{toc}{section}{\nameref{sec:sh117}}
\begin{longtable}{l p{0.5cm} r}
عیسی مریم به کوهی می‌گریخت
&&
شیرگویی خون او می‌خواست ریخت
\\
آن یکی در پی دوید و گفت خیر
&&
در پیت کس نیست چه گریزی چو طیر
\\
با شتاب او آنچنان می‌تاخت جفت
&&
کز شتاب خود جواب او نگفت
\\
یک دو میدان در پی عیسی براند
&&
پس بجد جد عیسی را بخواند
\\
کز پی مرضات حق یک لحظه بیست
&&
که مرا اندر گریزت مشکلیست
\\
از کی این سو می‌گریزی ای کریم
&&
نه پیت شیر و نه خصم و خوف و بیم
\\
گفت از احمق گریزانم برو
&&
می‌رهانم خویش را بندم مشو
\\
گفت آخر آن مسیحا نه توی
&&
که شود کور و کر از تو مستوی
\\
گفت آری گفت آن شه نیستی
&&
که فسون غیب را ماویستی
\\
چون بخوانی آن فسون بر مرده‌ای
&&
برجهد چون شیر صید آورده‌ای
\\
گفت آری آن منم گفتا که تو
&&
نه ز گل مرغان کنی ای خوب‌رو
\\
گفت آری گفت پس ای روح پاک
&&
هرچه خواهی می‌کنی از کیست باک
\\
با چنین برهان که باشد در جهان
&&
که نباشد مر ترا از بندگان
\\
گفت عیسی که به ذات پاک حق
&&
مبدع تن خالق جان در سبق
\\
حرمت ذات و صفات پاک او
&&
که بود گردون گریبان‌چاک او
\\
کان فسون و اسم اعظم را که من
&&
بر کر و بر کور خواندم شد حسن
\\
بر که سنگین بخواندم شد شکاف
&&
خرقه را بدرید بر خود تا بناف
\\
برتن مرده بخواندم گشت حی
&&
بر سر لاشی بخواندم گشت شی
\\
خواندم آن را بر دل احمق بود
&&
صد هزاران بار و درمانی نشد
\\
سنگ خارا گشت و زان خو بر نگشت
&&
ریگ شد کز وی نروید هیچ کشت
\\
گفت حکمت چیست کآنجا اسم حق
&&
سود کرد اینجا نبود آن را سبق
\\
آن همان رنجست و این رنجی چرا
&&
او نشد این را و آن را شد دوا
\\
گفت رنج احمقی قهر خداست
&&
رنج و کوری نیست قهر آن ابتلاست
\\
ابتلا رنجیست کان رحم آورد
&&
احمقی رنجیست کان زخم آورد
\\
آنچ داغ اوست مهر او کرده است
&&
چاره‌ای بر وی نیارد برد دست
\\
ز احمقان بگریز چون عیسی گریخت
&&
صحبت احمق بسی خونها که ریخت
\\
اندک اندک آب را دزدد هوا
&&
دین چنین دزدد هم احمق از شما
\\
گرمیت را دزدد و سردی دهد
&&
همچو آن کو زیر کون سنگی نهد
\\
آن گریز عیسی نه از بیم بود
&&
آمنست او آن پی تعلیم بود
\\
زمهریر ار پر کند آفاق را
&&
چه غم آن خورشید با اشراق را
\\
\end{longtable}
\end{center}
