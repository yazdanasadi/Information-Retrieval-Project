\begin{center}
\section*{بخش ۸۶ - بازگشتن به قصهٔ دقوقی}
\label{sec:sh086}
\addcontentsline{toc}{section}{\nameref{sec:sh086}}
\begin{longtable}{l p{0.5cm} r}
مر علی را در مثالی شیر خواند
&&
شیر مثل او نباشد گرچه راند
\\
از مثال و مثل و فرق آن بران
&&
جانب قصهٔ دقوقی ای جوان
\\
آنک در فتوی امام خلق بود
&&
گوی تقوی از فرشته می‌ربود
\\
آنک اندر سیر مه را مات کرد
&&
هم ز دین‌داری او دین رشک خورد
\\
با چنین تقوی و اوراد و قیام
&&
طالب خاصان حق بودی مدام
\\
در سفر معظم مرادش آن بدی
&&
که دمی بر بندهٔ خاصی زدی
\\
این همی‌گفتی چو می‌رفتی براه
&&
کن قرین خاصگانم ای اله
\\
یا رب آنها راکه بشناسد دلم
&&
بنده و بسته‌میان ومجملم
\\
و آنک نشناسم تو ای یزدان جان
&&
بر من محجوبشان کن مهربان
\\
حضرتش گفتی که ای صدر مهین
&&
این چه عشقست و چه استسقاست این
\\
مهر من داری چه می‌جویی دگر
&&
چون خدا با تست چون جویی بشر
\\
او بگفتی یا رب ای دانای راز
&&
تو گشودی در دلم راه نیاز
\\
درمیان بحر اگر بنشسته‌ام
&&
طمع در آب سبو هم بسته‌ام
\\
همچو داودم نود نعجه مراست
&&
طمع در نعجهٔ حریفم هم بخاست
\\
حرص اندر عشق تو فخرست و جاه
&&
حرص اندر غیر تو ننگ و تباه
\\
شهوت و حرص نران بیشی بود
&&
و آن حیزان ننگ و بدکیشی بود
\\
حرص مردان از ره پیشی بود
&&
در مخنث حرص سوی پس رود
\\
آن یکی حرص از کمال مردی است
&&
و آن دگر حرص افتضاح و سردی است
\\
آه سری هست اینجا بس نهان
&&
که سوی خضری شود موسی روان
\\
همچو مستسقی کز آبش سیر نیست
&&
بر هر آنچ یافتی بالله مه‌ایست
\\
بی نهایت حضرتست این بارگاه
&&
صدر را بگذار صدر تست راه
\\
\end{longtable}
\end{center}
