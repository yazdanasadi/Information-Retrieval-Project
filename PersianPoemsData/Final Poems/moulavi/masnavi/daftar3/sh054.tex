\begin{center}
\section*{بخش ۵۴ - حکایت آن شخص کی در عهد داود شب و روز دعا می‌کرد کی مرا روزی حلال ده بی رنج}
\label{sec:sh054}
\addcontentsline{toc}{section}{\nameref{sec:sh054}}
\begin{longtable}{l p{0.5cm} r}
آن یکی در عهد داوود نبی
&&
نزد هر دانا و پیش هر غبی
\\
این دعا می‌کرد دایم کای خدا
&&
ثروتی بی رنج روزی کن مرا
\\
چون مرا تو آفریدی کاهلی
&&
زخم‌خواری سست‌جنبی منبلی
\\
بر خران پشت‌ریش بی‌مراد
&&
بار اسپان و استران نتوان نهاد
\\
کاهلم چون آفریدی ای ملی
&&
روزیم ده هم ز راه کاهلی
\\
کاهلم من سایهٔ خسپم در وجود
&&
خفتم اندر سایهٔ این فضل و جود
\\
کاهلان و سایه‌خسپان را مگر
&&
روزیی بنوشته‌ای نوعی دگر
\\
هر که را پایست جوید روزیی
&&
هر که را پا نیست کن دلسوزیی
\\
رزق را می‌ران به سوی آن حزین
&&
ابر را باران به سوی هر زمین
\\
چون زمین را پا نباشد جود تو
&&
ابر را راند به سوی او دوتو
\\
طفل را چون پا نباشد مادرش
&&
آید و ریزد وظیفه بر سرش
\\
روزیی خواهم بناگه بی تعب
&&
که ندارم من ز کوشش جز طلب
\\
مدت بسیار می‌کرد این دعا
&&
روز تا شب شب همه شب تا ضحی
\\
خلق می‌خندید بر گفتار او
&&
بر طمع‌خامی و بر بیگار او
\\
که چه می‌گوید عجب این سست‌ریش
&&
یا کسی دادست بنگ بیهشیش
\\
راه روزی کسب و رنجست و تعب
&&
هر کسی را پیشه‌ای داد و طلب
\\
اطلبوا الارزاق فی اسبابها
&&
ادخلو الاوطان من ابوابها
\\
شاه و سلطان و رسول حق کنون
&&
هست داود نبی ذو فنون
\\
با چنان عزی و نازی کاندروست
&&
که گزیدستش عنایتهای دوست
\\
معجزاتش بی شمار و بی عدد
&&
موج بخشایش مدد اندر مدد
\\
هیچ کس را خود ز آدم تا کنون
&&
کی بدست آواز صد چون ارغنون
\\
که بهر وعظی بمیراند دویست
&&
آدمی را صوت خوبش کرد نیست
\\
شیر و آهو جمع گردد آن زمان
&&
سوی تذکیرش مغفل این از آن
\\
کوه و مرغان هم‌رسایل با دمش
&&
هردو اندر وقت دعوت محرمش
\\
این و صد چندین مرورا معجزات
&&
نور رویش بی جهان و در جهات
\\
با همه تمکین خدا روزی او
&&
کرده باشد بسته اندر جست و جو
\\
بی زره‌بافی و رنجی روزیش
&&
می‌نیاید با همه پیروزیش
\\
این چنین مخذول واپس مانده‌ای
&&
خانه کنده دون و گردون‌رانده‌ای
\\
این چنین مدبر همی خواهد که زود
&&
بی تجارت پر کند دامن ز سود
\\
این چنین گیجی بیامد در میان
&&
که بر آیم بر فلک بی نردبان
\\
این همی‌گفتش بتسخر رو بگیر
&&
که رسیدت روزی و آمد بشیر
\\
و آن همی خندید ما را هم بده
&&
زانچ یابی هدیه‌ای سالار ده
\\
او ازین تشنیع مردم وین فسوس
&&
کم نمی‌کرد از دعا و چاپلوس
\\
تا که شد در شهر معروف و شهیر
&&
کو ز انبان تهی جوید پنیر
\\
شد مثل در خام‌طبعی آن گدا
&&
او ازین خواهش نمی‌آمد جدا
\\
\end{longtable}
\end{center}
