\begin{center}
\section*{بخش ۱۶۶ - جواب حمزه مر خلق را}
\label{sec:sh166}
\addcontentsline{toc}{section}{\nameref{sec:sh166}}
\begin{longtable}{l p{0.5cm} r}
گفت حمزه چونک بودم من جوان
&&
مرگ می‌دیدم وداع این جهان
\\
سوی مردن کس برغبت کی رود
&&
پیش اژدرها برهنه کی شود
\\
لیک از نور محمد من کنون
&&
نیستم این شهر فانی را زبون
\\
از برون حس لشکرگاه شاه
&&
پر همی‌بینم ز نور حق سپاه
\\
خیمه در خیمه طناب اندر طناب
&&
شکر آنک کرد بیدارم ز خواب
\\
آنک مردن پیش چشمش تهلکه‌ست
&&
امر لا تلقوا بگیرد او به دست
\\
و آنک مردن پیش او شد فتح باب
&&
سارعوا آید مرورا در خطاب
\\
الحذر ای مرگ‌بینان بارعوا
&&
العجل ای حشربینان سارعوا
\\
الصلا ای لطف‌بینان افرحوا
&&
البلا ای قهربینان اترحوا
\\
هر که یوسف دید جان کردش فدی
&&
هر که گرگش دید برگشت از هدی
\\
مرگ هر یک ای پسر همرنگ اوست
&&
پیش دشمن دشمن و بر دوست دوست
\\
پیش ترک آیینه را خوش رنگیست
&&
پیش زنگی آینه هم زنگیست
\\
آنک می‌ترسی ز مرگ اندر فرار
&&
آن ز خود ترسانی ای جان هوش دار
\\
روی زشت تست نه رخسار مرگ
&&
جان تو همچون درخت و مرگ برگ
\\
از تو رستست ار نکویست ار بدست
&&
ناخوش و خوش هر ضمیرت از خودست
\\
گر بخاری خسته‌ای خود کشته‌ای
&&
ور حریر و قزدری خود رشته‌ای
\\
دانک نبود فعل همرنگ جزا
&&
هیچ خدمت نیست همرنگ عطا
\\
مزد مزدوران نمی‌ماند بکار
&&
کان عرض وین جوهرست و پایدار
\\
آن همه سختی و زورست و عرق
&&
وین همه سیمست و زرست و طبق
\\
گر ترا آید ز جایی تهمتی
&&
کرد مظلومت دعا در محنتی
\\
تو همی‌گویی که من آزاده‌ام
&&
بر کسی من تهمتی ننهاده‌ام
\\
تو گناهی کرده‌ای شکل دگر
&&
دانه کشتی دانه کی ماند به بر
\\
او زنا کرد و جزا صد چوب بود
&&
گوید او من کی زدم کس را بعود
\\
نه جزای آن زنا بود این بلا
&&
چوب کی ماند زنا را در خلا
\\
مار کی ماند عصا را ای کلیم
&&
درد کی ماند دوا را ای حکیم
\\
تو به جای آن عصا آب منی
&&
چون بیفکندی شد آن شخص سنی
\\
یار شد یا مار شد آن آب تو
&&
زان عصا چونست این اعجاب تو
\\
هیچ ماند آب آن فرزند را
&&
هیچ ماند نیشکر مر قند را
\\
چون سجودی یا رکوعی مرد کشت
&&
شد در آن عالم سجود او بهشت
\\
چونک پرید از دهانش حمد حق
&&
مرغ جنت ساختش رب الفلق
\\
حمد و تسبیحت نماند مرغ را
&&
گرچه نطفهٔ مرغ بادست و هوا
\\
چون ز دستت رست ایثار و زکات
&&
گشت این دست آن طرف نخل و نبات
\\
آب صبرت جوی آب خلد شد
&&
جوی شیر خلد مهر تست و ود
\\
ذوق طاعت گشت جوی انگبین
&&
مستی و شوق تو جوی خمر بین
\\
این سببها آن اثرها را نماند
&&
کس نداند چونش جای آن نشاند
\\
این سببها چون به فرمان تو بود
&&
چار جو هم مر ترا فرمان نمود
\\
هر طرف خواهی روانش می‌کنی
&&
آن صفت چون بد چنانش می‌کنی
\\
چون منی تو که در فرمان تست
&&
نسل آن در امر تو آیند چست
\\
می‌دود بر امر تو فرزند نو
&&
که منم جزوت که کردی‌اش گرو
\\
آن صفت در امر تو بود این جهان
&&
هم در امر تست آن جوها روان
\\
آن درختان مر ترا فرمان‌برند
&&
کان درختان از صفاتت با برند
\\
چون به امر تست اینجا این صفات
&&
پس در امر تست آنجا آن جزات
\\
چون ز دستت زخم بر مظلوم رست
&&
آن درختی گشت ازو زقوم رست
\\
چون ز خشم آتش تو در دلها زدی
&&
مایهٔ نار جهنم آمدی
\\
آتشت اینجا چو آدم سوز بود
&&
آنچ از وی زاد مرد افروز بود
\\
آتش تو قصد مردم می‌کند
&&
نار کز وی زاد بر مردم زند
\\
آن سخنهای چو مار و کزدمت
&&
مار و کزدم گشت و می‌گیرد دمت
\\
اولیا را داشتی در انتظار
&&
انتظار رستخیزت گشت یار
\\
وعدهٔ فردا و پس‌فردای تو
&&
انتظار حشرت آمد وای تو
\\
منتظر مانی در آن روز دراز
&&
در حساب و آفتاب جان‌گداز
\\
کآسمان را منتظر می‌داشتی
&&
تخم فردا ره روم می‌کاشتی
\\
خشم تو تخم سعیر دوزخست
&&
هین بکش این دوزخت را کین فخست
\\
کشتن این نار نبود جز به نور
&&
نورک اطفا نارنا نحن الشکور
\\
گر تو بی نوری کنی حلمی بدست
&&
آتشت زنده‌ست و در خاکسترست
\\
آن تکلف باشد و روپوش هین
&&
نار را نکشد به غیر نور دین
\\
تا نبینی نور دین آمن مباش
&&
کاتش پنهان شود یک روز فاش
\\
نور آبی دان و هم در آب چفس
&&
چونک داری آب از آتش مترس
\\
آب آتش را کشد کآتش به خو
&&
می‌بسوزد نسل و فرزندان او
\\
سوی آن مرغابیان رو روز چند
&&
تا ترا در آب حیوانی کشند
\\
مرغ خاکی مرغ آبی هم‌تنند
&&
لیک ضدانند آب و روغنند
\\
هر یکی مر اصل خود را بنده‌اند
&&
احتیاطی کن بهم ماننده‌اند
\\
همچنانک وسوسه و وحی الست
&&
هر دو معقولند لیکن فرق هست
\\
هر دو دلالان بازار ضمیر
&&
رختها را می‌ستایند ای امیر
\\
گر تو صراف دلی فکرت شناس
&&
فرق کن سر دو فکر چون نخاس
\\
ور ندانی این دو فکرت از گمان
&&
لا خلابه گوی و مشتاب و مران
\\
\end{longtable}
\end{center}
