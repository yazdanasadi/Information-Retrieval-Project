\begin{center}
\section*{بخش ۱۷۴ - شناختن هر حیوانی بوی عدو خود را و حذر کردن و بطالت و خسارت آنکس کی عدو کسی بود کی ازو حذر ممکن نیست و فرار ممکن نی و مقابله ممکن نی}
\label{sec:sh174}
\addcontentsline{toc}{section}{\nameref{sec:sh174}}
\begin{longtable}{l p{0.5cm} r}
اسپ داند بانگ و بوی شیر را
&&
گر چه حیوانست الا نادرا
\\
بل عدو خویش را هر جانور
&&
خود بداند از نشان و از اثر
\\
روز خفاشک نیارد بر پرید
&&
شب برون آمد چو دزدان و چرید
\\
از همه محروم‌تر خفاش بود
&&
که عدو آفتاب فاش بود
\\
نه تواند در مصافش زخم خورد
&&
نه بنفرین تاندش مهجور کرد
\\
آفتابی که بگرداند قفاش
&&
از برای غصه و قهر خفاش
\\
غایت لطف و کمال او بود
&&
گرنه خفاشش کجا مانع شود
\\
دشمنی گیری بحد خویش گیر
&&
تا بود ممکن که گردانی اسیر
\\
قطره با قلزم چو استیزه کند
&&
ابلهست او ریش خود بر می‌کند
\\
حیلت او از سبالش نگذرد
&&
چنبرهٔ حجرهٔ قمر چون بر درد
\\
با عدو آفتاب این بد عتاب
&&
ای عدو آفتاب آفتاب
\\
ای عدو آفتابی کز فرش
&&
می‌بلرزد آفتاب و اخترش
\\
تو عدو او نه‌ای خصم خودی
&&
چه غم آتش را که تو هیزم شدی
\\
ای عجب از سوزشت او کم شود
&&
یا ز درد سوزشت پر غم شود
\\
رحمتش نه رحمت آدم بود
&&
که مزاج رحم آدم غم بود
\\
رحمت مخلوق باشد غصه‌ناک
&&
رحمت حق از غم و غصه‌ست پاک
\\
رحمت بی‌چون چنین دان ای پدر
&&
ناید اندر وهم از وی جز اثر
\\
\end{longtable}
\end{center}
