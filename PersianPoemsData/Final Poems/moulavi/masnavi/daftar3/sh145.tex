\begin{center}
\section*{بخش ۱۴۵ - بیان آنک ایمان مقلد خوفست و رجا}
\label{sec:sh145}
\addcontentsline{toc}{section}{\nameref{sec:sh145}}
\begin{longtable}{l p{0.5cm} r}
داعی هر پیشه اومیدست و بوک
&&
گرچه گردنشان ز کوشش شد چو دوک
\\
بامدادان چون سوی دکان رود
&&
بر امید و بوک روزی می‌دود
\\
بوک روزی نبودت چون می‌روی
&&
خوف حرمان هست تو چونی قوی
\\
خوف حرمان ازل در کسب لوت
&&
چون نکردت سست اندر جست و جوت
\\
گویی گرچه خوف حرمان هست پیش
&&
هست اندر کاهلی این خوف بیش
\\
هست در کوشش امیدم بیشتر
&&
دارم اندر کاهلی افزون خطر
\\
پس چرا در کار دین ای بدگمان
&&
دامنت می‌گیرد این خوف زیان
\\
یا ندیدی کاهل این بازار ما
&&
در چه سودند انبیا و اولیا
\\
زین دکان رفتن چه کانشان رو نمود
&&
اندرین بازار چون بستند سود
\\
آتش آن را رام چون خلخال شد
&&
بحر آن را رام شد حمال شد
\\
آهن آن را رام شد چون موم شد
&&
باد آن را بنده و محکوم شد
\\
\end{longtable}
\end{center}
