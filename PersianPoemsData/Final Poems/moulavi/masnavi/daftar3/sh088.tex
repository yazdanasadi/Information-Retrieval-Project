\begin{center}
\section*{بخش ۸۸ - بازگشتن به قصهٔ دقوقی}
\label{sec:sh088}
\addcontentsline{toc}{section}{\nameref{sec:sh088}}
\begin{longtable}{l p{0.5cm} r}
آن دقوقی رحمة الله علیه
&&
گفت سافرت مدی فی خافقیه
\\
سال و مه رفتم سفر از عشق ماه
&&
بی‌خبر از راه حیران در اله
\\
پا برهنه می‌روی بر خار و سنگ
&&
گفت من حیرانم و بی خویش و دنگ
\\
تو مبین این پایها را بر زمین
&&
زانک بر دل می‌رود عاشق یقین
\\
از ره و منزل ز کوتاه و دراز
&&
دل چه داند کوست مست دل‌نواز
\\
آن دراز و کوته اوصاف تنست
&&
رفتن ارواح دیگر رفتنست
\\
تو سفرکردی ز نطفه تا بعقل
&&
نه بگامی بود نه منزل نه نقل
\\
سیر جان بی چون بود در دور و دیر
&&
جسم ما از جان بیاموزید سیر
\\
سیر جسمانه رها کرد او کنون
&&
می‌رود بی‌چون نهان در شکل چون
\\
گفت روزی می‌شدم مشتاق‌وار
&&
تا ببینم در بشر انوار یار
\\
تا ببینم قلزمی در قطره‌ای
&&
آفتابی درج اندر ذره‌ای
\\
چون رسیدم سوی یک ساحل بگام
&&
بود بیگه گشته روز و وقت شام
\\
\end{longtable}
\end{center}
