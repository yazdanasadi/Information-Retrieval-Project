\begin{center}
\section*{بخش ۶۲ - رنجور شدن اوستاد به وهم}
\label{sec:sh062}
\addcontentsline{toc}{section}{\nameref{sec:sh062}}
\begin{longtable}{l p{0.5cm} r}
گشت استا سست از وهم و ز بیم
&&
بر جهید و می‌کشانید او گلیم
\\
خشمگین با زن که مهر اوست سست
&&
من بدین حالم نپرسید و نجست
\\
خود مرا آگه نکرد از رنگ من
&&
قصد دارد تا رهد از ننگ من
\\
او به حسن و جلوهٔ خود مست گشت
&&
بی‌خبر کز بام افتادم چو طشت
\\
آمد و در را بتندی وا گشاد
&&
کودکان اندر پی آن اوستاد
\\
گفت زن خیرست چون زود آمدی
&&
که مبادا ذات نیکت را بدی
\\
گفت کوری رنگ و حال من ببین
&&
از غمم بیگانگان اندر حنین
\\
تو درون خانه از بغض و نفاق
&&
می‌نبینی حال من در احتراق
\\
گفت زن ای خواجه عیبی نیستت
&&
وهم و ظن لاش بی معنیستت
\\
گفتش ای غر تو هنوزی در لجاج
&&
می‌نبینی این تغیر و ارتجاج
\\
گر تو کور و کر شدی ما را چه جرم
&&
ما درین رنجیم و در اندوه و گرم
\\
گفت ای خواجه بیارم آینه
&&
تا بدانی که ندارم من گنه
\\
گفت رو مه تو رهی مه آینت
&&
دایما در بغض و کینی و عنت
\\
جامهٔ خواب مرا زو گستران
&&
تا بخسپم که سر من شد گران
\\
زن توقف کرد مردش بانگ زد
&&
کای عدو زوتر ترا این می‌سزد
\\
\end{longtable}
\end{center}
