\begin{center}
\section*{بخش ۷۹ - عذر گفتن شیخ بهر ناگریستن بر فرزندان}
\label{sec:sh079}
\addcontentsline{toc}{section}{\nameref{sec:sh079}}
\begin{longtable}{l p{0.5cm} r}
شیخ گفت او را مپندار ای رفیق
&&
که ندارم رحم و مهر و دل شفیق
\\
بر همه کفار ما را رحمتست
&&
گرچه جان جمله کافر نعمتست
\\
بر سگانم رحمت و بخشایش است
&&
که چرا از سنگهاشان مالش است
\\
آن سگی که می‌گزد گویم دعا
&&
که ازین خو وا رهانش ای خدا
\\
این سگان را هم در آن اندیشه دار
&&
که نباشند از خلایق سنگسار
\\
زان بیاورد اولیا را بر زمین
&&
تا کندشان رحمة للعالمین
\\
خلق را خواند سوی درگاه خاص
&&
حق را خواند که وافر کن خلاص
\\
جهد بنماید ازین سو بهر پند
&&
چون نشد گوید خدایا در مبند
\\
رحمت جزوی بود مر عام را
&&
رحمت کلی بود همام را
\\
رحمت جزوش قرین گشته بکل
&&
رحمت دریا بود هادی سبل
\\
رحمت جزوی بکل پیوسته شو
&&
رحمت کل را تو هادی بین و رو
\\
تا که جزوست او نداند راه بحر
&&
هر غدیری را کند ز اشباه بحر
\\
چون نداند راه یم کی ره برد
&&
سوی دریا خلق را چون آورد
\\
متصل گردد به بحر آنگاه او
&&
ره برد تا بحر همچون سیل و جو
\\
ور کند دعوت به تقلیدی بود
&&
نه از عیان و وحی تاییدی بود
\\
گفت پس چون رحم داری بر همه
&&
همچو چوپانی به گرد این رمه
\\
چون نداری نوحه بر فرزند خویش
&&
چونک فصاد اجلشان زد بنیش
\\
چون گواه رحم اشک دیده‌هاست
&&
دیدهٔ تو بی نم و گریه چراست
\\
رو به زن کرد و بگفتش ای عجوز
&&
خود نباشد فصل دی همچون تموز
\\
جمله گر مردند ایشان گر حی‌اند
&&
غایب و پنهان ز چشم دل کی‌اند
\\
من چو بینمشان معین پیش خویش
&&
از چه رو رو را کنم همچون تو ریش
\\
گرچه بیرون‌اند از دور زمان
&&
با من‌اند و گرد من بازی‌کنان
\\
گریه از هجران بود یا از فراق
&&
با عزیزانم وصالست و عناق
\\
خلق اندر خواب می‌بینندشان
&&
من به بیداری همی‌بینم عیان
\\
زین جهان خود را دمی پنهان کنم
&&
برگ حس را از درخت افشان کنم
\\
حس اسیر عقل باشد ای فلان
&&
عقل اسیر روح باشد هم بدان
\\
دست بستهٔ عقل را جان باز کرد
&&
کارهای بسته را هم ساز کرد
\\
حسها و اندیشه بر آب صفا
&&
همچو خس بگرفته روی آب را
\\
دست عقل آن خس به یکسو می‌برد
&&
آب پیدا می‌شود پیش خرد
\\
خس بس انبه بود بر جو چون حباب
&&
خس چو یکسو رفت پیدا گشت آب
\\
چونک دست عقل نگشاید خدا
&&
خس فزاید از هوا بر آب ما
\\
آب را هر دم کند پوشیده او
&&
آن هوا خندان و گریان عقل تو
\\
چونک تقوی بست دو دست هوا
&&
حق گشاید هر دو دست عقل را
\\
پس حواس چیره محکوم تو شد
&&
چون خرد سالار و مخدوم تو شد
\\
حس را بی‌خواب خواب اندر کند
&&
تا که غیبیها ز جان سر بر زند
\\
هم به بیداری ببینی خوابها
&&
هم ز گردون بر گشاید بابها
\\
\end{longtable}
\end{center}
