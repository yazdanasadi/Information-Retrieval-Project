\begin{center}
\section*{بخش ۱۶۱ - دویدن آن شخص به سوی موسی به زنهار چون از خروس خبر مرگ خود شنید}
\label{sec:sh161}
\addcontentsline{toc}{section}{\nameref{sec:sh161}}
\begin{longtable}{l p{0.5cm} r}
چون شنید اینها دوان شد تیز و تفت
&&
بر در موسی کلیم الله رفت
\\
رو همی‌مالید در خاک او ز بیم
&&
که مرا فریاد رس زین ای کلیم
\\
گفت رو بفروش خود را و بره
&&
چونک استا گشته‌ای بر جه ز چه
\\
بر مسلمانان زیان انداز تو
&&
کیسه و همیانها را کن دوتو
\\
من درون خشت دیدم این قضا
&&
که در آیینه عیان شد مر ترا
\\
عاقل اول بیند آخر را بدل
&&
اندر آخر بیند از دانش مقل
\\
باز زاری کرد کای نیکوخصال
&&
مر مرا در سر مزن در رو ممال
\\
از من آن آمد که بودم ناسزا
&&
ناسزایم را تو ده حسن الجزا
\\
گفت تیری جست از شست ای پسر
&&
نیست سنت کید آن واپس به سر
\\
لیک در خواهم ز نیکوداوری
&&
تا که ایمان آن زمان با خود بری
\\
چونک ایمان برده باشی زنده‌ای
&&
چونک با ایمان روی پاینده‌ای
\\
هم در آن دم حال بر خواجه بگشت
&&
تا دلش شوریده و آوردند طشت
\\
شورش مرگست نه هیضهٔ طعام
&&
قی چه سودت دارد ای بدبخت خام
\\
چار کس بردند تا سوی وثاق
&&
ساق می‌مالید او بر پشت ساق
\\
پند موسی نشنوی شوخی کنی
&&
خویشتن بر تیغ پولادی زنی
\\
شرم ناید تیغ را از جان تو
&&
آن تست این ای برادر آن تو
\\
\end{longtable}
\end{center}
