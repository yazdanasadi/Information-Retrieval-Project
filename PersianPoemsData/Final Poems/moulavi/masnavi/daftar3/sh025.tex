\begin{center}
\section*{بخش ۲۵ - قصهٔ هاروت و ماروت و دلیری  ایشان بر امتحانات حق تعالی}
\label{sec:sh025}
\addcontentsline{toc}{section}{\nameref{sec:sh025}}
\begin{longtable}{l p{0.5cm} r}
پیش ازین زان گفته بودیم اندکی
&&
خود چه گوییم از هزارانش یکی
\\
خواستم گفتن در آن تحقیقها
&&
تا کنون وا ماند از تعویقها
\\
حملهٔ دیگر ز بسیارش قلیل
&&
گفته آید شرح یک عضوی ز پیل
\\
گوش کن هاروت را ماروت را
&&
ای غلام و چاکران ما روت را
\\
مست بودند از تماشای اله
&&
وز عجایبهای استدراج شاه
\\
این چنین مستیست ز استدراج حق
&&
تا چه مستیها کند معراج حق
\\
دانهٔ دامش چنین مستی نمود
&&
خوان انعامش چه‌ها داند گشود
\\
مست بودند و رهیده از کمند
&&
های هوی عاشقانه می‌زدند
\\
یک کمین و امتحان در راه بود
&&
صرصرش چون کاه که را می‌ربود
\\
امتحان می‌کردشان زیر و زبر
&&
کی بود سرمست را زینها خبر
\\
خندق و میدان بپیش او یکیست
&&
چاه و خندق پیش او خوش مسلکیست
\\
آن بز کوهی بر آن کوه بلند
&&
بر دود از بهر خوردی بی‌گزند
\\
تا علف چیند ببیند ناگهان
&&
بازیی دیگر ز حکم آسمان
\\
بر کهی دیگر بر اندازد نظر
&&
ماده بز بیند بر آن کوه دگر
\\
چشم او تاریک گردد در زمان
&&
بر جهد سرمست زین که تا بدان
\\
آنچنان نزدیک بنماید ورا
&&
که دویدن گرد بالوعهٔ سرا
\\
آن هزاران گز دو گز بنمایدش
&&
تا ز مستی میل جستن آیدش
\\
چونک بجهد در فتد اندر میان
&&
در میان هر دو کوه بی امان
\\
او ز صیادان به که بگریخته
&&
خود پناهش خون او را ریخته
\\
شسته صیادان میان آن دو کوه
&&
انتظار این قضای با شکوه
\\
باشد اغلب صید این بز همچنین
&&
ورنه چالاکست و چست و خصم‌بین
\\
رستم ارچه با سر و سبلت بود
&&
دام پاگیرش یقین شهوت بود
\\
همچو من از مستی شهوت ببر
&&
مستی شهوت ببین اندر شتر
\\
باز این مستی شهوت در جهان
&&
پیش مستی ملک دان مستهان
\\
مستی آن مستی این بشکند
&&
او به شهوت التفاتی کی کند
\\
آب شیرین تا نخوردی آب شور
&&
خوش بود خوش چون درون دیده نور
\\
قطره‌ای از باده‌های آسمان
&&
بر کند جان را ز می وز ساقیان
\\
تا چه مستیها بود املاک را
&&
وز جلالت روحهای پاک را
\\
که به بوی دل در آن می بسته‌اند
&&
خم بادهٔ این جهان بشکسته‌اند
\\
جز مگر آنها که نومیدند و دور
&&
همچو کفاری نهفته در قبور
\\
ناامید از هر دو عالم گشته‌اند
&&
خارهای بی‌نهایت کشته‌اند
\\
پس ز مستیها بگفتند ای دریغ
&&
بر زمین باران بدادیمی چو میغ
\\
گستریدیمی درین بی‌داد جا
&&
عدل و انصاف و عبادات و وفا
\\
این بگفتند و قضا می‌گفت بیست
&&
پیش پاتان دام ناپیدا بسیست
\\
هین مدو گستاخ در دشت بلا
&&
هین مران کورانه اندر کربلا
\\
که ز موی و استخوان هالکان
&&
می‌نیابد راه پای سالکان
\\
جملهٔ راه استخوان و موی و پی
&&
بس که تیغ قهر لاشی کرد شی
\\
گفت حق که بندگان جفت عون
&&
بر زمین آهسته می‌رانند و هون
\\
پا برهنه چون رود در خارزار
&&
جز بوقفه و فکرت و پرهیزگار
\\
این قضا می‌گفت لیکن گوششان
&&
بسته بود اندر حجاب جوششان
\\
چشمها و گوشها را بسته‌اند
&&
جز مر آنها را که از خود رسته‌اند
\\
جز عنایت که گشاید چشم را
&&
جز محبت که نشاند خشم را
\\
جهد بی توفیق خود کس را مباد
&&
در جهان والله اعلم بالسداد
\\
\end{longtable}
\end{center}
