\begin{center}
\section*{بخش ۱۵۷ - قانع شدن آن طالب به تعلیم زبان مرغ خانگی و سگ و اجابت موسی علیه السلام}
\label{sec:sh157}
\addcontentsline{toc}{section}{\nameref{sec:sh157}}
\begin{longtable}{l p{0.5cm} r}
گفت باری نطق سگ کو بر درست
&&
نطق مرغ خانگی کاهل پرست
\\
گفت موسی هین تو دانی رو رسید
&&
نطق این هر دو شود بر تو پدید
\\
بامدادان از برای امتحان
&&
ایستاد او منتظر بر آستان
\\
خادمه سفره بیفشاند و فتاد
&&
پاره‌ای نان بیات آثار زاد
\\
در ربود آن را خروسی چون گرو
&&
گفت سگ کردی تو بر ما ظلم رو
\\
دانهٔ گندم توانی خورد و من
&&
عاجزم در دانه خوردن در وطن
\\
گندم و جو را و باقی حبوب
&&
می‌توانی خورد و من نه ای طروب
\\
این لب نانی که قسم ماست نان
&&
می‌ربایی این قدر را از سگان
\\
\end{longtable}
\end{center}
