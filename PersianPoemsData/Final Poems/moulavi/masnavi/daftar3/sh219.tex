\begin{center}
\section*{بخش ۲۱۹ - تفسیر این خبر کی مصطفی علیه السلام فرمود لا تفضلونی علی یونس بن متی}
\label{sec:sh219}
\addcontentsline{toc}{section}{\nameref{sec:sh219}}
\begin{longtable}{l p{0.5cm} r}
گفت پیغامبر که معراج مرا
&&
نیست بر معراج یونس اجتبا
\\
آن من بر چرخ و آن او نشیب
&&
زانک قرب حق برونست از حساب
\\
قرب نه بالا نه پستی رفتنست
&&
قرب حق از حبس هستی رستنست
\\
نیست را چه جای بالا است و زیر
&&
نیست را نه زود و نه دورست و دیر
\\
کارگاه و گنج حق در نیستیست
&&
غرهٔ هستی چه دانی نیست چیست
\\
حاصل این اشکست ایشان ای کیا
&&
می‌نماند هیچ با اشکست ما
\\
آنچنان شادند در ذل و تلف
&&
همچو ما در وقت اقبال و شرف
\\
برگ بی‌برگی همه اقطاع اوست
&&
فقر و خواریش افتخارست و علوست
\\
آن یکی گفت ار چنانست آن ندید
&&
چون بخندید او که ما را بسته دید
\\
چونک او مبدل شدست و شادیش
&&
نیست زین زندان و زین آزادیش
\\
پس به قهر دشمنان چون شاد شد
&&
چون ازین فتح و ظفر پر باد شد
\\
شاد شد جانش که بر شیران نر
&&
یافت آسان نصرت و دست و ظفر
\\
پس بدانستیم کو آزاد نیست
&&
جز به دنیا دلخوش و دلشاد نیست
\\
ورنه چون خندد که اهل آن جهان
&&
بر بد و نیک‌اند مشفق مهربان
\\
این بمنگیدند در زیر زبان
&&
آن اسیران با هم اندر بحث آن
\\
تا موکل نشنود بر ما جهد
&&
خود سخن در گوش آن سلطان برد
\\
\end{longtable}
\end{center}
