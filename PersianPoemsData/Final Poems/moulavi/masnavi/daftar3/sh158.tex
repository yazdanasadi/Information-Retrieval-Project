\begin{center}
\section*{بخش ۱۵۸ - جواب خروس سگ را}
\label{sec:sh158}
\addcontentsline{toc}{section}{\nameref{sec:sh158}}
\begin{longtable}{l p{0.5cm} r}
پس خروسش گفت تن زن غم مخور
&&
که خدا بدهد عوض زینت دگر
\\
اسپ این خواجه سقط خواهد شدن
&&
روز فردا سیر خور کم کن حزن
\\
مر سگان را عید باشد مرگ اسپ
&&
روزی وافر بود بی جهد و کسپ
\\
اسپ را بفروخت چون بشنید مرد
&&
پیش سگ شد آن خروسش روی‌زرد
\\
روز دیگر همچنان نان را ربود
&&
آن خروس و سگ برو لب بر گشود
\\
کای خروس عشوه‌ده چند این دروغ
&&
ظالمی و کاذبی و بی فروغ
\\
اسپ کش گفتی سقط گردد کجاست
&&
کور اخترگوی و محرومی ز راست
\\
گفت او را آن خروس با خبر
&&
که سقط شد اسپ او جای دگر
\\
اسپ را بفروخت و جست او از زیان
&&
آن زیان انداخت او بر دیگران
\\
لیک فردا استرش گردد سقط
&&
مر سگان را باشد آن نعمت فقط
\\
زود استر را فروشید آن حریص
&&
یافت از غم وز زیان آن دم محیص
\\
روز ثالث گفت سگ با آن خروس
&&
ای امیر کاذبان با طبل و کوس
\\
گفت او بفروخت استر را شتاب
&&
گفت فردایش غلام آید مصاب
\\
چون غلام او بمیرد نانها
&&
بر سگ و خواهنده ریزند اقربا
\\
این شنید و آن غلامش را فروخت
&&
رست از خسران و رخ را بر فروخت
\\
شکرها می‌کرد و شادیها که من
&&
رستم از سه واقعه اندر زمن
\\
تا زبان مرغ و سگ آموختم
&&
دیدهٔ س القضا را دوختم
\\
روز دیگر آن سگ محروم گفت
&&
کای خروس ژاژخا کو طاق و جفت
\\
\end{longtable}
\end{center}
