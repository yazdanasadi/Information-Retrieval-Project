\begin{center}
\section*{بخش ۹۵ - هفت مرد شدن آن هفت درخت}
\label{sec:sh095}
\addcontentsline{toc}{section}{\nameref{sec:sh095}}
\begin{longtable}{l p{0.5cm} r}
بعد دیری گشت آنها هفت مرد
&&
جمله در قعده پی یزدان فرد
\\
چشم می‌مالم که آن هفت ارسلان
&&
تا کیانند و چه دارند از جهان
\\
چون به نزدیکی رسیدم من ز راه
&&
کردم ایشان را سلام از انتباه
\\
قوم گفتندم جواب آن سلام
&&
ای دقوقی مفخر و تاج کرام
\\
گفتم آخر چون مرا بشناختند
&&
پیش ازین بر من نظر ننداختند
\\
از ضمیر من بدانستند زود
&&
یکدگر را بنگریدند از فرود
\\
پاسخم دادند خندان کای عزیز
&&
این بپوشیدست اکنون بر تو نیز
\\
بر دلی کو در تحیر با خداست
&&
کی شود پوشیده راز چپ و راست
\\
گفتم ار سوی حقایق بشکفند
&&
چون ز اسم حرف رسمی واقفند
\\
گفت اگر اسمی شود غیب از ولی
&&
آن ز استغراق دان نه از جاهلی
\\
بعد از آن گفتند ما را آرزوست
&&
اقتدا کردن به تو ای پاک دوست
\\
گفتم آری لیک یک ساعت که من
&&
مشکلاتی دارم از دور زمن
\\
تا شود آن حل به صحبتهای پاک
&&
که به صحبت روید انگوری ز خاک
\\
دانهٔ پرمغز با خاک دژم
&&
خلوتی و صحبتی کرد از کرم
\\
خویشتن در خاک کلی محو کرد
&&
تا نماندش رنگ و بو و سرخ و زرد
\\
از پس آن محو قبض او نماند
&&
پرگشاد و بسط شد مرکب براند
\\
پیش اصل خویش چون بی‌خویش شد
&&
رفت صورت جلوهٔ معنیش شد
\\
سر چنین کردند هین فرمان تراست
&&
تف دل از سر چنین کردن بخاست
\\
ساعتی با آن گروه مجتبی
&&
چون مراقب گشتم و از خود جدا
\\
هم در آن ساعت ز ساعت رست جان
&&
زانک ساعت پیر گرداند جوان
\\
جمله تلوینها ز ساعت خاستست
&&
رست از تلوین که از ساعت برست
\\
چون ز ساعت ساعتی بیرون شوی
&&
چون نماند محرم بی‌چون شوی
\\
ساعت از بی‌ساعتی آگاه نیست
&&
زانکش آن سو جز تحیر راه نیست
\\
هر نفر را بر طویله خاص او
&&
بسته‌اند اندر جهان جست و جو
\\
منتصب بر هر طویله رایضی
&&
جز بدستوری نیاید رافضی
\\
از هوس گر از طویله بسکلد
&&
در طویله دیگران سر در کند
\\
در زمان آخرجیان چست خوش
&&
گوشهٔ افسار او گیرند و کش
\\
حافظان را گر نبینی ای عیار
&&
اختیارت را ببین بی اختیار
\\
اختیاری می‌کنی و دست و پا
&&
بر گشادستت چرا حسبی چرا
\\
روی در انکار حافظ برده‌ای
&&
نام تهدیدات نفسش کرده‌ای
\\
\end{longtable}
\end{center}
