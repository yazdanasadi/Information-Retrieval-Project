\begin{center}
\section*{بخش ۷۶ - حکایت استر پیش شتر کی من بسیار در رو می‌افتم و تو نمی‌افتی الا به نادر}
\label{sec:sh076}
\addcontentsline{toc}{section}{\nameref{sec:sh076}}
\begin{longtable}{l p{0.5cm} r}
گفت استر با شتر کای خوش رفیق
&&
در فراز و شیب و در راه دقیق
\\
تو نه آیی در سر و خوش می‌روی
&&
من همی‌آیم بسر در چون غوی
\\
من همی‌افتم برو در هر دمی
&&
خواه در خشکی و خواه اندر نمی
\\
این سبب را باز گو با من که چیست
&&
تا بدانم من که چون باید بزیست
\\
گفت چشم من ز تو روشن‌ترست
&&
بعد از آن هم از بلندی ناظرست
\\
چون برآیم بر سرکوه بلند
&&
آخر عقبه ببینم هوشمند
\\
پس همه پستی و بالایی راه
&&
دیده‌ام را وا نماید هم اله
\\
هر قدم من از سر بینش نهم
&&
از عثار و اوفتادن وا رهم
\\
تو ببینی پیش خود یک دو سه گام
&&
دانه بینی و نبینی رنج دام
\\
یستوی الاعمی لدیکم والبصیر
&&
فی المقام و النزول والمسیر
\\
چون جنین را در شکم حق جان دهد
&&
جذب اجزا در مزاج او نهد
\\
از خورش او جذب اجزا می‌کند
&&
تار و پود جسم خود را می‌تند
\\
تا چهل سالش بجذب جزوها
&&
حق حریصش کرده باشد در نما
\\
جذب اجزا روح را تعلیم کرد
&&
چون نداند جذب اجزا شاه فرد
\\
جامع این ذره‌ها خورشید بود
&&
بی غذا اجزات را داند ربود
\\
آن زمانی که در آیی تو ز خواب
&&
هوش و حس رفته را خواند شتاب
\\
تا بدانی کان ازو غایب نشد
&&
باز آید چون بفرماید که عد
\\
\end{longtable}
\end{center}
