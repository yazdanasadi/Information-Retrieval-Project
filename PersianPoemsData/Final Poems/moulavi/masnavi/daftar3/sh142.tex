\begin{center}
\section*{بخش ۱۴۲ - مخصوص بودن یعقوب علیه السلام به چشیدن جام حق از روی یوسف و کشیدن بوی حق از بوی یوسف و حرمان برادران و غیر هم ازین هر دو}
\label{sec:sh142}
\addcontentsline{toc}{section}{\nameref{sec:sh142}}
\begin{longtable}{l p{0.5cm} r}
آنچ یعقوب از رخ یوسف بدید
&&
خاص او بد آن به اخوان کی رسید
\\
این ز عشقش خویش در چه می‌کند
&&
و آن بکین از بهر او چه می‌کند
\\
سفرهٔ او پیش این از نان تهیست
&&
پیش یعقوبست پر کو مشتهیست
\\
روی ناشسته نبیند روی حور
&&
لا صلوة گفت الا بالطهور
\\
عشق باشد لوت و پوت جانها
&&
جوع ازین رویست قوت جانها
\\
جوع یوسف بود آن یعقوب را
&&
بوی نانش می‌رسید از دور جا
\\
آنک بستد پیرهن را می‌شتافت
&&
بوی پیراهان یوسف می‌نیافت
\\
و آنک صد فرسنگ زان سو بود او
&&
چونک بد یعقوب می‌بویید بو
\\
ای بسا عالم ز دانش بی‌نصیب
&&
حافظ علمست آنکس نه حبیب
\\
مستمع از وی همی‌یابد مشام
&&
گرچه باشد مستمع از جنس عام
\\
زانک پیراهان بدستش عاریه‌ست
&&
چون بدست آن نخاسی جاریه‌ست
\\
جاریه پیش نخاسی سرسریست
&&
در کف او از برای مشتریست
\\
قسمت حقست روزی دادنی
&&
هر یکی را سوی دیگر راه نی
\\
یک خیال نیک باغ آن شده
&&
یک خیال زشت راه این زده
\\
آن خدایی کز خیالی باغ ساخت
&&
وز خیالی دوزخ و جای گداخت
\\
پس کی داند راه گلشنهای او
&&
پس کی داند جای گلخنهای او
\\
دیدبان دل نبیند در مجال
&&
کز کدامین رکن جان آید خیال
\\
گر بدیدی مطلعش را ز احتیال
&&
بند کردی راه هر ناخوش خیال
\\
کی رسد جاسوس را آنجا قدم
&&
که بود مرصاد و در بند عدم
\\
دامن فضلش بکف کن کوروار
&&
قبض اعمی این بود ای شهرهٔار
\\
دامن او امر و فرمان ویست
&&
نیکبختی که تقی جان ویست
\\
آن یکی در مرغزار و جوی آب
&&
و آن یکی پهلوی او اندر عذاب
\\
او عجب مانده که ذوق این ز چیست
&&
و آن عجب مانده که این در حبس کیست
\\
هین چرا خشکی که اینجا چشمه هاست
&&
هین چرا زردی که اینجا صد دواست
\\
همنشینا هین در آ اندر چمن
&&
گوید ای جان من نیارم آمدن
\\
\end{longtable}
\end{center}
