\begin{center}
\section*{بخش ۱۴۰ - حکایت جوحی کی چادر پوشید و در وعظ میان زنان نشست  و حرکتی کرد زنی او را بشناخت کی مردست نعره‌ای زد}
\label{sec:sh140}
\addcontentsline{toc}{section}{\nameref{sec:sh140}}
\begin{longtable}{l p{0.5cm} r}
واعظی بد بس گزیده در بیان
&&
زیر منبر جمع مردان و زنان
\\
رفت جوحی چادر و روبند ساخت
&&
در میان آن زنان شد ناشناخت
\\
سایلی پرسید واعظ را به راز
&&
موی عانه هست نقصان نماز
\\
گفت واعظ چون شود عانه دراز
&&
پس کراهت باشد از وی در نماز
\\
یا به آهک یا ستره بسترش
&&
تا نمازت کامل آید خوب و خوش
\\
گفت سایل آن درازی تا چه حد
&&
شرط باشد تا نمازم کم بود
\\
گفت چون قدر جوی گردد به طول
&&
پس ستردن فرض باشد ای سئول
\\
گفت جوحی زود ای خوهر ببین
&&
عانهٔ من گشته باشد این چنین
\\
بهر خشنودی حق پیش آر دست
&&
که آن به مقدار کراهت آمدست
\\
دست زن در کرد در شلوار مرد
&&
کیر او بر دست زن آسیب کرد
\\
نعره‌ای زد سخت اندر حال زن
&&
گفت واعظ بر دلش زد گفت من
\\
گفت نه بر دل نزد بر دست زد
&&
وای اگر بر دل زدی ای پر خرد
\\
بر دل آن ساحران زد اندکی
&&
شد عصا و دست ایشان را یکی
\\
گر عصا بستانی از پیری شها
&&
بیش رنجد که آن گروه از دست و پا
\\
نعرهٔ لاضیر بر گردون رسید
&&
هین ببر که جان ز جان کندن رهید
\\
ما بدانستیم ما این تن نه‌ایم
&&
از ورای تن به یزدان می‌زییم
\\
ای خنک آن را که ذات خود شناخت
&&
اندر امن سرمدی قصری بساخت
\\
کودکی گرید پی جوز و مویز
&&
پیش عاقل باشد آن بس سهل چیز
\\
پیش دل جوز و مویز آمد جسد
&&
طفل کی در دانش مردان رسد
\\
هر که محجوبست او خود کودکست
&&
مرد آن باشد که بیرون از شکست
\\
گر بریش و خایه مردستی کسی
&&
هر بزی را ریش و مو باشد بسی
\\
پیشوای بد بود آن بز شتاب
&&
می‌برد اصحاب را پیش قصاب
\\
ریش شانه کرده که من سابقم
&&
سابقی لیکن به سوی مرگ و غم
\\
هین روش بگزین و ترک ریش کن
&&
ترک این ما و من و تشویش کن
\\
تا شوی چون بوی گل با عاشقان
&&
پیشوا و رهنمای گلستان
\\
کیست بوی گل دم عقل و خرد
&&
خوش قلاووز ره ملک ابد
\\
\end{longtable}
\end{center}
