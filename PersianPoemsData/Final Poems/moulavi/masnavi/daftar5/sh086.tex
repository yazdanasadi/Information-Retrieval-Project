\begin{center}
\section*{بخش ۸۶ - حکایت در تقریر این سخن کی چندین گاه گفت ذکر را آزمودیم مدتی صبر و خاموشی را بیازماییم}
\label{sec:sh086}
\addcontentsline{toc}{section}{\nameref{sec:sh086}}
\begin{longtable}{l p{0.5cm} r}
چند پختی تلخ و تیز و شورگز
&&
این یکی بار امتحان شیرین بپز
\\
آن یکی را در قیامت ز انتباه
&&
در کف آید نامهٔ عصیان سیاه
\\
سرسیه چون نامه‌های تعزیه
&&
پر معاصی متن نامه و حاشیه
\\
جمله فسق و معصیت بد یک سری
&&
هم‌چو دارالحرب پر از کافری
\\
آنچنان نامهٔ پلید پر وبال
&&
در یمین ناید درآید در شمال
\\
خود همین‌جا نامهٔ خود را ببین
&&
دست چپ را شاید آن یا در یمین
\\
موزهٔ چپ کفش چپ هم در دکان
&&
آن چپ دانیش پیش از امتحان
\\
چون نباشی راست می‌دان که چپی
&&
هست پیدا نعرهٔ شیر و کپی
\\
آنک گل را شاهد و خوش‌بو کند
&&
هر چپی را راست فضل او کند
\\
هر شمالی را یمینی او دهد
&&
بحر را ماء معینی او دهد
\\
گر چپی با حضرت او راست باش
&&
تا ببینی دست‌برد لطفهاش
\\
تو روا داری که این نامهٔ مهین
&&
بگذرد از چپ در آید در یمین
\\
این چنین نامه که پرظلم و جفاست
&&
کی بود خود درخور اندر دست راست
\\
\end{longtable}
\end{center}
