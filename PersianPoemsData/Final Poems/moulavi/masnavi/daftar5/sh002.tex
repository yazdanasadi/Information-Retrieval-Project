\begin{center}
\section*{بخش ۲ - تفسیر خذ اربعة من الطیر فصرهن الیک}
\label{sec:sh002}
\addcontentsline{toc}{section}{\nameref{sec:sh002}}
\begin{longtable}{l p{0.5cm} r}
تو خلیل وقتی ای خورشیدهش
&&
این چهار اطیار ره‌زن را بکش
\\
زانک هر مرغی ازینها زاغ‌وش
&&
هست عقل عاقلان را دیده‌کش
\\
چار وصف تن چو مرغان خلیل
&&
بسمل ایشان دهد جان را سبیل
\\
ای خلیل اندر خلاص نیک و بد
&&
سر ببرشان تا رهد پاها ز سد
\\
کل توی و جملگان اجزای تو
&&
بر گشا که هست پاشان پای تو
\\
از تو عالم روح زاری می‌شود
&&
پشت صد لشکر سواری می‌شود
\\
زانک این تن شد مقام چار خو
&&
نامشان شد چار مرغ فتنه‌جو
\\
خلق را گر زندگی خواهی ابد
&&
سر ببر زین چار مرغ شوم بد
\\
بازشان زنده کن از نوعی دگر
&&
که نباشد بعد از آن زیشان ضرر
\\
چار مرغ معنوی راه‌زن
&&
کرده‌اند اندر دل خلقان وطن
\\
چون امیر جمله دلهای سوی
&&
اندرین دور ای خلیفهٔ حق توی
\\
سر ببر این چار مرغ زنده را
&&
سر مدی کن خلق ناپاینده را
\\
بط و طاوسست و زاغست و خروس
&&
این مثال چار خلق اندر نفوس
\\
بط حرصست و خروس آن شهوتست
&&
جاه چون طاوس و زاغ امنیتست
\\
منیتش آن که بود اومیدساز
&&
طامع تابید یا عمر دراز
\\
بط حرص آمد که نولش در زمین
&&
در تر و در خشک می‌جوید دفین
\\
یک زمان نبود معطل آن گلو
&&
نشنود از حکم جز امر کلوا
\\
هم‌چو یغماجیست خانه می‌کند
&&
زود زود انبان خود پر می‌کند
\\
اندر انبان می‌فشارد نیک و بد
&&
دانه‌های در و حبات نخود
\\
تا مبادا یاغیی آید دگر
&&
می‌فشارد در جوال او خشک و تر
\\
وقت تنگ و فرصت اندک او مخوف
&&
در بغل زد هر چه زودتر بی‌وقوف
\\
لیک مؤمن ز اعتماد آن حیات
&&
می‌کند غارت به مهل و با انات
\\
آمنست از فوت و از یاغی که او
&&
می‌شناسد قهر شه را بر عدو
\\
آمنست از خواجه‌تاشان دگر
&&
که بیایندش مزاحم صرفه‌بر
\\
عدل شه را دید در ضبط حشم
&&
که نیارد کرد کس بر کس ستم
\\
لاجرم نشتابد و ساکن بود
&&
از فوات حظ خود آمن بود
\\
بس تانی دارد و صبر و شکیب
&&
چشم‌سیر و مثرست و پاک‌جیب
\\
کین تانی پرتو رحمان بود
&&
وان شتاب از هزهٔ شیطان بود
\\
زانک شیطانش بترساند ز فقر
&&
بارگیر صبر را بکشد به عقر
\\
از نبی بشنو که شیطان در وعید
&&
می‌کند تهدیدت از فقر شدید
\\
تا خوری زشت و بری زشت و شتاب
&&
نی مروت نی‌تانی نی ثواب
\\
لاجرم کافر خورد در هفت بطن
&&
دین و دل باریک و لاغر زفت بطن
\\
\end{longtable}
\end{center}
