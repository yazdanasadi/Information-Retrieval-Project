\begin{center}
\section*{بخش ۵۵ - پرسیدن آن پادشاه از آن مدعی نبوت کی آنک رسول راستین باشد و ثابت شود با او چه باشد کی کسی را بخشد یا به صحبت و خدمت او چه بخشش یابند غیر نصیحت به زبان کی می‌گوید}
\label{sec:sh055}
\addcontentsline{toc}{section}{\nameref{sec:sh055}}
\begin{longtable}{l p{0.5cm} r}
شاه پرسیدش که باری وحی چیست
&&
یا چه حاصل دارد آن کس کو نبیست
\\
گفت خود آن چیست کش حاصل نشد
&&
یا چه دولت ماند کو واصل نشد
\\
گیرم این وحی نبی گنجور نیست
&&
هم کم از وحی دل زنبور نیست
\\
چونک او حی الرب الی النحل آمدست
&&
خانهٔ وحیش پر از حلوا شدست
\\
او به نور وحی حق عزوجل
&&
کرد عالم را پر از شمع و عسل
\\
این که کرمناست و بالا می‌رود
&&
وحیش از زنبور کمتر کی بود
\\
نه تو اعطیناک کوثر خوانده‌ای
&&
پس چرا خشکی و تشنه مانده‌ای
\\
یا مگر فرعونی و کوثر چو نیل
&&
بر تو خون گشتست و ناخوش ای علیل
\\
توبه کن بیزار شو از هر عدو
&&
کو ندارد آب کوثر در کدو
\\
هر کرا دیدی ز کوثر سرخ‌رو
&&
او محمدخوست با او گیر خو
\\
تا احب لله آیی در حساب
&&
کز درخت احمدی با اوست سیب
\\
هر کرا دیدی ز کوثر خشک لب
&&
دشمنش می‌دار هم‌چون مرگ و تب
\\
گر چه بابای توست و مام تو
&&
کو حقیقت هست خون‌آشام تو
\\
از خلیل حق بیاموز این سیر
&&
که شد او بیزار اول از پدر
\\
تا که ابغض لله آیی پیش حق
&&
تا نگیرد بر تو رشک عشق دق
\\
تا نخوانی لا و الا الله را
&&
در نیابی منهج این راه را
\\
\end{longtable}
\end{center}
