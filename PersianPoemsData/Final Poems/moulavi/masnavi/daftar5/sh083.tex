\begin{center}
\section*{بخش ۸۳ - حواله کردن پادشاه قبول و توبهٔ نمامان و حجره گشایان و سزا دادن ایشان با ایاز کی یعنی این جنایت بر عرض او رفته است}
\label{sec:sh083}
\addcontentsline{toc}{section}{\nameref{sec:sh083}}
\begin{longtable}{l p{0.5cm} r}
این جنایت بر تن و عرض ویست
&&
زخم بر رگهای آن نیکوپیست
\\
گرچه نفس واحدیم از روی جان
&&
ظاهرا دورم ازین سود و زیان
\\
تهمتی بر بنده شه را عار نیست
&&
جز مزید حلم و استظهار نیست
\\
متهم را شاه چون قارون کند
&&
بی‌گنه را تو نظر کن چون کند
\\
شاه را غافل مدان از کار کس
&&
مانع اظهار آن حلمست و بس
\\
من هنا یشفع به پیش علم او
&&
لا ابالی‌وار الا حلم او
\\
آن گنه اول ز حلمش می‌جهد
&&
ورنه هیبت آن مجالش کی دهد
\\
خونبهای جرم نفس قاتله
&&
هست بر حلمش دیت بر عاقله
\\
مست و بی‌خود نفس ما زان حلم بود
&&
دیو در مستی کلاه از وی ربود
\\
گرنه ساقی حلم بودی باده‌ریز
&&
دیو با آدم کجا کردی ستیز
\\
گاه علم آدم ملایک را کی بود
&&
اوستاد علم و نقاد نقود
\\
چونک در جنت شراب حلم خورد
&&
شد ز یک بازی شیطان روی زرد
\\
آن بلادرهای تعلیم ودود
&&
زیرک و دانا و چستش کرده بود
\\
باز آن افیون حلم سخت او
&&
دزد را آورد سوی رخت او
\\
عقل آید سوی حلمش مستجیر
&&
ساقیم تو بوده‌ای دستم بگیر
\\
\end{longtable}
\end{center}
