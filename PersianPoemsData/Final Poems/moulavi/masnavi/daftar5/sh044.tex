\begin{center}
\section*{بخش ۴۴ - تفسیر خلقنا الانسان فی احسن تقویم ثم رددناه اسفل سافلین و  تفسیر و من نعمره ننکسه فی الخلق}
\label{sec:sh044}
\addcontentsline{toc}{section}{\nameref{sec:sh044}}
\begin{longtable}{l p{0.5cm} r}
آدم حسن و ملک ساجد شده
&&
هم‌چو آدم باز معزول آمده
\\
گفت آوه بعد هستی نیستی
&&
گفت جرمت این که افزون زیستی
\\
جبرئیلش می‌کشاند مو کشان
&&
که برو زین خلد و از جوق خوشان
\\
گفت بعد از عز این اذلال چیست
&&
گفت آن دادست و اینت داوریست
\\
جبرئیلا سجده می‌کردی به جان
&&
چون کنون می‌رانیم تو از جنان
\\
حله می‌پرد ز من در امتحان
&&
هم‌چو برگ از نخ در فصل خزان
\\
آن رخی که تاب او بد ماه‌وار
&&
شد به پیری هم‌چو پشت سوسمار
\\
وان سر و فرق گش شعشع شده
&&
وقت پیری ناخوش و اصلع شده
\\
وان قد صف در نازان چون سنان
&&
گشته در پیری دو تا هم‌چون کمان
\\
رنگ لاله گشته رنگ زعفران
&&
زور شیرش گشته چون زهرهٔ زنان
\\
آنک مردی در بغل کردی به فن
&&
می‌بگیرندش بغل وقت شدن
\\
این خود آثار غم و پژمردگیست
&&
هر یکی زینها رسول مردگیست
\\
\end{longtable}
\end{center}
