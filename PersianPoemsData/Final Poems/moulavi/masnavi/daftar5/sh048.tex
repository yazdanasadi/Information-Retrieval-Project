\begin{center}
\section*{بخش ۴۸ - تفسیر و هو معکم}
\label{sec:sh048}
\addcontentsline{toc}{section}{\nameref{sec:sh048}}
\begin{longtable}{l p{0.5cm} r}
یک سپد پر نان ترا بی‌فرق سر
&&
تو همی خواهی لب نان در به در
\\
در سر خود پیچ هل خیره‌سری
&&
رو در دل زن چرا بر هر دری
\\
تا بزانویی میان آب‌جو
&&
غافل از خود زین و آن تو آب جو
\\
پیش آب و پس هم آب با مدد
&&
چشمها را پیش سد و خلف سد
\\
اسپ زیر ران و فارس اسپ‌جو
&&
چیست این گفت اسپ لیکن اسپ کو
\\
هی نه اسپست این به زیر تو پدید
&&
گفت آری لیک خود اسپی که دید
\\
مست آب و پیش روی اوست آن
&&
اندر آب و بی‌خبر ز آب روان
\\
چون گهر در بحر گوید بحر کو
&&
وآن خیال چون صدف دیوار او
\\
گفتن آن کو حجابش می‌شود
&&
ابر تاب آفتابش می‌شود
\\
بند چشم اوست هم چشم بدش
&&
عین رفع سد او گشته سدش
\\
بند گوش او شده هم هوش او
&&
هوش با حق دار ای مدهوش او
\\
\end{longtable}
\end{center}
