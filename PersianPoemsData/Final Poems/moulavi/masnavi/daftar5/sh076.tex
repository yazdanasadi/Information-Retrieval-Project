\begin{center}
\section*{بخش ۷۶ - حکمت نظر کردن در چارق و پوستین کی فلینظر الانسان مم خلق}
\label{sec:sh076}
\addcontentsline{toc}{section}{\nameref{sec:sh076}}
\begin{longtable}{l p{0.5cm} r}
بازگردان قصهٔ عشق ایاز
&&
که آن یکی گنجیست مالامال راز
\\
می‌رود هر روز در حجرهٔ برین
&&
تا ببیند چارقی با پوستین
\\
زانک هستی سخت مستی آورد
&&
عقل از سر شرم از دل می‌برد
\\
صد هزاران قرن پیشین را همین
&&
مستی هستی بزد ره زین کمین
\\
شد عزرائیلی ازین مستی بلیس
&&
که چرا آدم شود بر من رئیس
\\
خواجه‌ام من نیز و خواجه‌زاده‌ام
&&
صد هنر را قابل و آماده‌ام
\\
در هنر من از کسی کم نیستم
&&
تا به خدمت پیش دشمن بیستم
\\
من ز آتش زاده‌ام او از وحل
&&
پیش آتش مر وحل را چه محل
\\
او کجا بود اندر آن دوری که من
&&
صدر عالم بودم و فخر زمن
\\
\end{longtable}
\end{center}
