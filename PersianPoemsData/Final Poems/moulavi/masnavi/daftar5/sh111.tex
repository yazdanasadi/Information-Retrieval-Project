\begin{center}
\section*{بخش ۱۱۱ - جواب گفتن خر روباه را}
\label{sec:sh111}
\addcontentsline{toc}{section}{\nameref{sec:sh111}}
\begin{longtable}{l p{0.5cm} r}
گفت رو رو هین ز پیشم ای عدو
&&
تا نبینم روی تو ای زشت‌رو
\\
آن خدایی که ترا بدبخت کرد
&&
روی زشتت را کریه و سخت کرد
\\
با کدامین روی می‌آیی به من
&&
این چنین سغری ندارد کرگدن
\\
رفته‌ای در خون جانم آشکار
&&
که ترا من ره‌برم تا مرغزار
\\
تا بدیدم روی عزرائیل را
&&
باز آوردی فن و تسویل را
\\
گرچه من ننگ خرانم یا خرم
&&
جانورم جان دارم این را کی خرم
\\
آنچ من دیدم ز هول بی‌امان
&&
طفل دیدی پیر گشتی در زمان
\\
بی‌دل و جان از نهیب آن شکوه
&&
سرنگون خود را در افکندم ز کوه
\\
بسته شد پایم در آن دم از نهیب
&&
چون بدیدم آن عذاب بی‌حجاب
\\
عهد کردم با خدا کای ذوالمنن
&&
برگشا زین بستگی تو پای من
\\
تا ننوشم وسوسهٔ کس بعد ازین
&&
عهد کردم نذر کردم ای معین
\\
حق گشاده کرد آن دم پای من
&&
زان دعا و زاری و ایمای من
\\
ورنه اندر من رسیدی شیر نر
&&
چون بدی در زیر پنجهٔ شیر خر
\\
باز بفرستادت آن شیر عرین
&&
سوی من از مکر ای بئس القرین
\\
حق ذات پاک الله الصمد
&&
که بود به مار بد از یار بد
\\
مار بد جانی ستاند از سلیم
&&
یار بد آرد سوی نار مقیم
\\
از قرین بی‌قول و گفت و گوی او
&&
خو بدزدد دل نهان از خوی او
\\
چونک او افکند بر تو سایه را
&&
دزدد آن بی‌مایه از تو مایه را
\\
عقل تو گر اژدهایی گشت مست
&&
یار بد او را زمرد دان که هست
\\
دیدهٔ عقلت بدو بیرون جهد
&&
طعن اوت اندر کف طاعون نهد
\\
\end{longtable}
\end{center}
