\begin{center}
\section*{بخش ۱۰۲ - جواب گفتن خر روباه را کی توکل بهترین کسبهاست کی هر  کسبی محتاجست به توکل کی ای خدا این کار مرا راست آر و دعا  متضمن توکلست و توکل کسبی است کی به هیچ کسبی دیگر محتاج  نیست الی آخره}
\label{sec:sh102}
\addcontentsline{toc}{section}{\nameref{sec:sh102}}
\begin{longtable}{l p{0.5cm} r}
گفت من به از توکل بر ربی
&&
می‌ندانم در دو عالم مکسبی
\\
کسب شکرش را نمی‌دانم ندید
&&
تا کشد رزق خدا رزق و مزید
\\
بحثشان بسیار شد اندر خطاب
&&
مانده گشتند از سؤال و از جواب
\\
بعد از آن گفتش بدان در مملکه
&&
نهی لا تلقوا بایدی تهلکه
\\
صبر در صحرای خشک و سنگ‌لاخ
&&
احمقی باشد جهان حق فراخ
\\
نقل کن زینجا به سوی مرغزار
&&
می‌چر آنجا سبزه گرد جویبار
\\
مرغزاری سبز مانند جنان
&&
سبزه رسته اندر آنجا تا میان
\\
خرم آن حیوان که او آنجا شود
&&
اشتر اندر سبزه ناپیدا شود
\\
هر طرف در وی یکی چشمهٔ روان
&&
اندرو حیوان مرفه در امان
\\
از خری او را نمی‌گفت ای لعین
&&
تو از آن‌جایی چرا زاری چنین
\\
کو نشاط و فربهی و فر تو
&&
چیست این لاغر تن مضطر تو
\\
شرح روضه گر دروغ و زور نیست
&&
پس چرا چشمت ازو مخمور نیست
\\
این گدا چشمی و این نادیدگی
&&
از گدایی تست نه از بگلربگی
\\
چون ز چشمه آمدی چونی تو خشک
&&
ور تو ناف آهویی کو بوی مشک
\\
زانک می‌گویی و شرحش می‌کنی
&&
چون نشانی در تو نامد ای سنی
\\
\end{longtable}
\end{center}
