\begin{center}
\section*{بخش ۲۰ - صفت طاوس و طبع او و سبب کشتن ابراهیم علیه‌السلام او را}
\label{sec:sh020}
\addcontentsline{toc}{section}{\nameref{sec:sh020}}
\begin{longtable}{l p{0.5cm} r}
آمدیم اکنون به طاوس دورنگ
&&
کو کند جلوه برای نام و ننگ
\\
همت او صید خلق از خیر و شر
&&
وز نتیجه و فایدهٔ آن بی‌خبر
\\
بی‌خبر چون دام می‌گیرد شکار
&&
دام را چه علم از مقصود کار
\\
دام را چه ضر و چه نفع از گرفت
&&
زین گرفت بیهده‌ش دارم شگفت
\\
ای برادر دوستان افراشتی
&&
با دو صد دلداری و بگذاشتی
\\
کارت این بودست از وقت ولاد
&&
صید مردم کردن از دام وداد
\\
زان شکار و انبهی و باد و بود
&&
دست در کن هیچ یابی تار و پود
\\
بیشتر رفتست و بیگاهست روز
&&
تو به جد در صید خلقانی هنوز
\\
آن یکی می‌گیر و آن می‌هل ز دام
&&
وین دگر را صید می‌کن چون لام
\\
باز این را می‌هل و می‌جو دگر
&&
اینت لعب کودکان بی‌خبر
\\
شب شود در دام تو یک صید نی
&&
دام بر تو جز صداع و قید نی
\\
پس تو خود را صید می‌کردی به دام
&&
که شدی محبوس و محرومی ز کام
\\
در زمانه صاحب دامی بود
&&
هم‌چو ما احمق که صید خود کند
\\
چون شکار خوک آمد صید عام
&&
رنج بی‌حد لقمه خوردن زو حرام
\\
آنک ارزد صید را عشقست و بس
&&
لیک او کی گنجد اندر دام کس
\\
تو مگر آیی و صید او شوی
&&
دام بگذاری به دام او روی
\\
عشق می‌گوید به گوشم پست پست
&&
صید بودن خوش‌تر از صیادیست
\\
گول من کن خویش را و غره شو
&&
آفتابی را رها کن ذره شو
\\
بر درم ساکن شو و بی‌خانه باش
&&
دعوی شمعی مکن پروانه باش
\\
تا ببینی چاشنی زندگی
&&
سلطنت بینی نهان در بندگی
\\
نعل بینی بازگونه در جهان
&&
تخته‌بندان را لقب گشته شهان
\\
بس طناب اندر گلو و تاج دار
&&
بر وی انبوهی که اینک تاجدار
\\
هم‌چو گور کافران بیرون حلل
&&
اندرون قهر خدا عز و جل
\\
چون قبور آن را مجصص کرده‌اند
&&
پردهٔ پندار پیش آورده‌اند
\\
طبع مسکینت مجصص از هنر
&&
هم‌چو نخل موم بی‌برگ و ثمر
\\
\end{longtable}
\end{center}
