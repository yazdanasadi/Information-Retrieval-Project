\begin{center}
\section*{بخش ۱۰۵ - حکایت آن مخنث و پرسیدن لوطی ازو در حالت لواطه کی این خنجر از بهر چیست گفت از برای آنک هر کی با من بد اندیشد  اشکمش بشکافم لوطی بر سر او آمد شد می‌کرد و می‌گفت الحمدلله کی من بد نمی‌اندیشم با تو «بیت من بیت نیست اقلیمست  هزل من هزل نیست تعلیمست» ان الله یستحیی ان یضرب مثلا ما بعوضة فما فوقها ای فما فوقها فی تغییر النفوس بالانکار ان ما ذا ا راد الله بهذا مثلا و آنگه جواب می‌فرماید کی این خواستم یضل به کثیرا و یهدی به کثیرا کی هر فتنه‌ای هم‌چون  میزانست بسیاران ازو سرخ‌رو شوند و بسیاران بی‌مراد شوند و لو  تاملت فیه قلیلا وجدت من نتایجه الشریفة کثیرا}
\label{sec:sh105}
\addcontentsline{toc}{section}{\nameref{sec:sh105}}
\begin{longtable}{l p{0.5cm} r}
کنده‌ای را لوطیی در خانه برد
&&
سرنگون افکندش و در وی فشرد
\\
بر میانش خنجری دید آن لعین
&&
پس بگفتش بر میانت چیست این
\\
گفت آنک با من ار یک بدمنش
&&
بد بیندیشد بدرم اشکمش
\\
گفت لوطی حمد لله را که من
&&
بد نه اندیشیده‌ام با تو به فن
\\
چون که مردی نیست خنجرها چه سود
&&
چون نباشد دل ندارد سود خود
\\
از علی میراث داری ذوالفقار
&&
بازوی شیر خدا هستت بیار
\\
گر فسونی یاد داری از مسیح
&&
کو لب و دندان عیسی ای قبیح
\\
کشتیی سازی ز توزیع و فتوح
&&
کو یکی ملاح کشتی هم‌چو نوح
\\
بت شکستی گیرم ابراهیم‌وار
&&
کو بت تن را فدی کردن بنار
\\
گر دلیلت هست اندر فعل آر
&&
تیغ چوبین را بدان کن ذوالفقار
\\
آن دلیلی که ترا مانع شود
&&
از عمل آن نقمت صانع بود
\\
خایفان راه را کردی دلیر
&&
از همه لرزان‌تری تو زیر زیر
\\
بر همه درس توکل می‌کنی
&&
در هوا تو پشه را رگ می‌زنی
\\
ای مخنث پیش رفته از سپاه
&&
بر دروغ ریش تو کیرت گواه
\\
چون ز نامردی دل آکنده بود
&&
ریش و سبلت موجب خنده بود
\\
توبه‌ای کن اشک باران چون مطر
&&
ریش و سبلت را ز خنده باز خر
\\
داروی مردی بخور اندر عمل
&&
تا شوی خورشید گرم اندر حمل
\\
معده را بگذار و سوی دل خرام
&&
تا که بی‌پرده ز حق آید سلام
\\
یک دو گامی رو تکلف ساز خوش
&&
عشق گیرد گوش تو آنگاه کش
\\
\end{longtable}
\end{center}
