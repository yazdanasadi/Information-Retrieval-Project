\begin{center}
\section*{بخش ۲۷ - در بیان آنک صفا و سادگی نفس مطمنه از فکرتها مشوش شود چنانک بر روی آینه چیزی نویسی یا نقش کنی اگر چه پاک کنی داغی بماند و نقصانی}
\label{sec:sh027}
\addcontentsline{toc}{section}{\nameref{sec:sh027}}
\begin{longtable}{l p{0.5cm} r}
روی نفس مطمئنه در جسد
&&
زخم ناخنهای فکرت می‌کشد
\\
فکرت بد ناخن پر زهر دان
&&
می‌خراشد در تعمق روی جان
\\
تا گشاید عقدهٔ اشکال را
&&
در حدث کردست زرین بیل را
\\
عقده را بگشاده گیر ای منتهی
&&
عقدهٔ سختست بر کیسهٔ تهی
\\
دز گشاد عقده‌ها گشتی تو پیر
&&
عقدهٔ چندی دگر بگشاده گیر
\\
عقده‌ای که آن بر گلوی ماست سخت
&&
که بدانی که خسی یا نیک‌بخت
\\
حل این اشکال کن گر آدمی
&&
خرج این کن دم اگر آدم‌دمی
\\
حد اعیان و عرض دانسته گیر
&&
حد خود را دان که نبود زین گزیر
\\
چون بدانی حد خود زین حدگریز
&&
تا به بی‌حد در رسی ای خاک‌بیز
\\
عمر در محمول و در موضوع رفت
&&
بی‌بصیرت عمر در مسموع رفت
\\
هر دلیلی بی‌نتیجه و بی‌اثر
&&
باطل آمد در نتیجهٔ خود نگر
\\
جز به مصنوعی ندیدی صانعی
&&
بر قیاس اقترانی قانعی
\\
می‌فزاید در وسایط فلسفی
&&
از دلایل باز برعکسش صفی
\\
این گریزد از دلیل و از حجاب
&&
از پی مدلول سر برده به جیب
\\
گر دخان او را دلیل آتشست
&&
بی‌دخان ما را در آن آتش خوشست
\\
خاصه این آتش که از قرب ولا
&&
از دخان نزدیک‌تر آمد به ما
\\
پس سیه‌کاری بود رفتن ز جان
&&
بهر تخییلات جان سوی دخان
\\
\end{longtable}
\end{center}
