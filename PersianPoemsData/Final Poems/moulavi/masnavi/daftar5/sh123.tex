\begin{center}
\section*{بخش ۱۲۳ - مثل}
\label{sec:sh123}
\addcontentsline{toc}{section}{\nameref{sec:sh123}}
\begin{longtable}{l p{0.5cm} r}
آن یکی می‌خورد نان فخفره
&&
گفت سایل چون بدین استت شره
\\
گفت جوع از صبر چون دوتا شود
&&
نان جو در پیش من حلوا شود
\\
پس توانم که همه حلوا خورم
&&
چون کنم صبری صبورم لاجرم
\\
خود نباشد جوع هر کس را زبون
&&
کین علف‌زاریست ز اندازه برون
\\
جوع مر خاصان حق را داده‌اند
&&
تا شوند از جوع شیر زورمند
\\
جوع هر جلف گدا را کی دهند
&&
چون علف کم نیست پیش او نهند
\\
که بخور که هم بدین ارزانیی
&&
تو نه‌ای مرغاب مرغ نانیی
\\
\end{longtable}
\end{center}
