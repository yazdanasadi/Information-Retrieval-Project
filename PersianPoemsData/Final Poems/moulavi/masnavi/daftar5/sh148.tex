\begin{center}
\section*{بخش ۱۴۸ - حکایت مات کردن دلقک سید شاه ترمد را}
\label{sec:sh148}
\addcontentsline{toc}{section}{\nameref{sec:sh148}}
\begin{longtable}{l p{0.5cm} r}
شاه با دلقک همی شطرنج باخت
&&
مات کردش زود خشم شه بتاخت
\\
گفت شه شه و آن شه کبرآورش
&&
یک یک از شطرنج می‌زد بر سرش
\\
که بگیر اینک شهت ای قلتبان
&&
صبر کرد آن دلقک و گفت الامان
\\
دست دیگر باختن فرمود میر
&&
او چنان لرزان که عور از زمهریر
\\
باخت دست دیگر و شه مات شد
&&
وقت شه شه گفتن و میقات شد
\\
بر جهید آن دلقک و در کنج رفت
&&
شش نمد بر خود فکند از بیم تفت
\\
زیر بالشها و زیر شش نمد
&&
خفت پنهان تا ز زخم شه رهد
\\
گفت شه هی هی چه کردی چیست این
&&
گفت شه شه شه شه ای شاه گزین
\\
کی توان حق گفت جز زیر لحاف
&&
با تو ای خشم‌آور آتش‌سجاف
\\
ای تو مات و من ز زخم شاه مات
&&
می‌زنم شه شه به زیر رختهات
\\
چون محله پر شد از هیهای میر
&&
وز لگد بر در زدن وز دار و گیر
\\
خلق بیرون جست زود از چپ و راست
&&
کای مقدم وقت عفوست و رضاست
\\
مغز او خشکست و عقلش این زمان
&&
کمترست از عقل و فهم کودکان
\\
زهد و پیری ضعف بر ضعف آمده
&&
واندر آن زهدش گشادی ناشده
\\
رنج دیده گنج نادیده ز یار
&&
کارها کرده ندیده مزد کار
\\
یا نبود آن کار او را خود گهر
&&
یا نیامد وقت پاداش از قدر
\\
یا که بود آن سعی چون سعی جهود
&&
یا جزا وابستهٔ میقات بود
\\
مر ورا درد و مصیبت این بس است
&&
که درین وادی پر خون بی‌کس است
\\
چشم پر درد و نشسته او به کنج
&&
رو ترش کرده فرو افکنده لنج
\\
نه یکی کحال کو را غم خورد
&&
نیش عقلی که به کحلی پی برد
\\
اجتهادی می‌کند با حزر و ظن
&&
کار در بوکست تا نیکو شدن
\\
زان رهش دورست تا دیدار دوست
&&
کو نجوید سر رئیسیش آرزوست
\\
ساعتی او با خدا اندر عتاب
&&
که نصیبم رنج آمد زین حساب
\\
ساعتی با بخت خود اندر جدال
&&
که همه پران و ما ببریده بال
\\
هر که محبوس است اندر بو و رنگ
&&
گرچه در زهدست باشد خوش تنگ
\\
تا برون ناید ازین ننگین مناخ
&&
کی شود خویش خوش و صدرش فراخ
\\
زاهدان را در خلا پیش از گشاد
&&
کارد و استره نشاید هیچ داد
\\
کز ضجر خود را بدراند شکم
&&
غصهٔ آن بی‌مرادیها و غم
\\
\end{longtable}
\end{center}
