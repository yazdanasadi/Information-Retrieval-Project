\begin{center}
\section*{بخش ۹۵ - حکایت دیدن خر هیزم‌فروش با نوایی اسپان تازی را بر آخر  خاص و تمنا بردن آن دولت را در موعظهٔ آنک تمنا نباید بردن الا  مغفرت و عنایت و هدایت کی اگر در صد لون رنجی چون لذت  مغفرت بود همه شیرین شود باقی هر دولتی کی آن را ناآزموده تمنی  می‌بری با آن رنجی قرینست کی آن را نمی‌بینی چنانک از هر دامی  دانه پیدا بود و فخ پنهان تو درین یک دام مانده‌ای تمنی می‌بری کی  کاشکی با آن دانه‌ها رفتمی پنداری کی آن دانه‌ها بی‌دامست}
\label{sec:sh095}
\addcontentsline{toc}{section}{\nameref{sec:sh095}}
\begin{longtable}{l p{0.5cm} r}
بود سقایی مرورا یک خری
&&
گشته از محنت دو تا چون چنبری
\\
پشتش از بار گران صد جای ریش
&&
عاشق و جویان روز مرگ خویش
\\
جو کجا از کاه خشک او سیر نی
&&
در عقب زخمی و سیخی آهنی
\\
میر آخر دید او را رحم کرد
&&
که آشنای صاحب خر بود مرد
\\
پس سلامش کرد و پرسیدش ز حال
&&
کز چه این خر گشت دوتا هم‌چو دال
\\
گفت از درویشی و تقصیر من
&&
که نمی‌یابد خود این بسته‌دهن
\\
گفت بسپارش به من تو روز چند
&&
تا شود در آخر شه زورمند
\\
خر بدو بسپرد و آن رحمت‌پرست
&&
در میان آخر سلطانش بست
\\
خر ز هر سو مرکب تازی بدید
&&
با نوا و فربه و خوب و جدید
\\
زیر پاشان روفته آبی زده
&&
که به وقت وجو به هنگام آمده
\\
خارش و مالش مر اسپان را بدید
&&
پوز بالا کرد کای رب مجید
\\
نه که مخلوق توم گیرم خرم
&&
از چه زار و پشت ریش و لاغرم
\\
شب ز درد پشت و از جوع شکم
&&
آرزومندم به مردن دم به دم
\\
حال این اسپان چنین خوش با نوا
&&
من چه مخصوصم به تعذیب و بلا
\\
ناگهان آوازهٔ پیگار شد
&&
تازیان را وقت زین و کار شد
\\
زخمهای تیر خوردند از عدو
&&
رفت پیکانها دریشان سو به سو
\\
از غزا باز آمدند آن تازیان
&&
اندر آخر جمله افتاده ستان
\\
پایهاشان بسته محکم با نوار
&&
نعلبندان ایستاده بر قطار
\\
می‌شکافیدند تن‌هاشان بنیش
&&
تا برون آرند پیکانها ز ریش
\\
آن خر آن را دید و می‌گفت ای خدا
&&
من به فقر و عافیت دادم رضا
\\
زان نوا بیزارم و زان زخم زشت
&&
هرکه خواهد عافیت دنیا بهشت
\\
\end{longtable}
\end{center}
