\begin{center}
\section*{بخش ۹۰ - نوبت جستن رسیدن به نصوح و آواز آمدن که همه را جستیم نصوح را بجویید و بیهوش شدن نصوح از آن هیبت و گشاده شدن کار بعد از نهایت بستگی کماکان یقول رسول الله صلی الله علیه و سلم اذا اصابه مرض او هم اشتدی ازمة تنفرجی}
\label{sec:sh090}
\addcontentsline{toc}{section}{\nameref{sec:sh090}}
\begin{longtable}{l p{0.5cm} r}
جمله را جستیم پیش آی ای نصوح
&&
گشت بیهوش آن زمان پرید روح
\\
هم‌چو دیوار شکسته در فتاد
&&
هوش و عقلش رفت شد او چون جماد
\\
چونک هوشش رفت از تن بی‌امان
&&
سر او با حق بپیوست آن زمان
\\
چون تهی گشت و وجود او نماند
&&
باز جانش را خدا در پیش خواند
\\
چون شکست آن کشتی او بی‌مراد
&&
در کنار رحمت دریا فتاد
\\
جان به حق پیوست چون بی‌هوش شد
&&
موج رحمت آن زمان در جوش شد
\\
چون که جانش وا رهید از ننگ تن
&&
رفت شادان پیش اصل خویشتن
\\
جان چو باز و تن مرورا کنده‌ای
&&
پای بسته پر شکسته بنده‌ای
\\
چونک هوشش رفت و پایش بر گشاد
&&
می‌پرد آن باز سوی کیقباد
\\
چونک دریاهای رحمت جوش کرد
&&
سنگها هم آب حیوان نوش کرد
\\
ذرهٔ لاغر شگرف و زفت شد
&&
فرش خاکی اطلس و زربفت شد
\\
مردهٔ صدساله بیرون شد ز گور
&&
دیو ملعون شد به خوبی رشک حور
\\
این همه روی زمین سرسبز شد
&&
چوب خشک اشکوفه کرد و نغز شد
\\
گرگ با بره حریف می شده
&&
ناامیدان خوش‌رگ و خوش پی شده
\\
\end{longtable}
\end{center}
