\begin{center}
\section*{بخش ۱۳۴ - معنی ما شاء الله کان یعنی خواست خواست او و رضا رضای او جویید از خشم دیگران و رد دیگران دلتنگ مباشید آن کان اگر چه لفظ ماضیست لیکن در فعل خدا ماضی و مستقبل نباشد کی لیس  عند الله صباح و لا مساء}
\label{sec:sh134}
\addcontentsline{toc}{section}{\nameref{sec:sh134}}
\begin{longtable}{l p{0.5cm} r}
قول بنده ایش شاء الله کان
&&
بهر آن نبود که تنبل کن در آن
\\
بلک تحریضست بر اخلاص و جد
&&
که در آن خدمت فزون شو مستعد
\\
گر بگویند آنچ می‌خواهی تو راد
&&
کار کار تست برحسب مراد
\\
آنگهان تنبل کنی جایز بود
&&
کانچ خواهی و آنچ گویی آن شود
\\
چون بگویند ایش شاء الله کان
&&
حکم حکم اوست مطلق جاودان
\\
پس چرا صد مرده اندر ورد او
&&
بر نگردی بندگانه گرد او
\\
گر بگویند آنچ می‌خواهد وزیر
&&
خواست آن اوست اندر دار و گیر
\\
گرد او گردان شوی صد مرده زود
&&
تا بریزد بر سرت احسان و جود
\\
یا گریزی از وزیر و قصر او
&&
این نباشد جست و جوی نصر او
\\
بازگونه زین سخن کاهل شدی
&&
منعکس ادراک و خاطر آمدی
\\
امر امر آن فلان خواجه‌ست هین
&&
چیست یعنی با جز او کمتر نشین
\\
گرد خواجه گرد چون امر آن اوست
&&
کو کشد دشمن رهاند جان دوست
\\
هرچه او خواهد همان یابی یقین
&&
یاوه کم رو خدمت او برگزین
\\
نی چو حاکم اوست گرد او مگرد
&&
تا شوی نامه سیاه و روی زود
\\
حق بود تاویل که آن گرمت کند
&&
پر امید و چست و با شرمت کند
\\
ور کند سستت حقیقت این بدان
&&
هست تبدیل و نه تاویلست آن
\\
این برای گرم کردن آمدست
&&
تا بگیرد ناامیدان را دو دست
\\
معنی قرآن ز قرآن پرس و بس
&&
وز کسی که آتش زدست اندر هوس
\\
پیش قرآن گشت قربانی و پست
&&
تا که عین روح او قرآن شدست
\\
روغنی کو شد فدای گل به کل
&&
خواه روغن بوی کن خواهی تو گل
\\
\end{longtable}
\end{center}
