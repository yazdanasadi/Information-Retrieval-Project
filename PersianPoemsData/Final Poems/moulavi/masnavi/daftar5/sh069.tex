\begin{center}
\section*{بخش ۶۹ - بیان آنک مخلوقی کر ترا ازو ظلمی رسد به حقیقت او هم‌چون  آلتیست عارف آن بود کی بحق رجوع کند نه به آلت و اگر به آلت رجوع  کند به ظاهر نه از جهل کند بلک برای مصلحتی چنانک ابایزید قدس  الله سره گفت کی چندین سالست کی من با مخلوق سخن نگفته‌ام و از مخلوق  سخن نشنیده‌ام ولیکن خلق چنین پندارند کی با ایشان سخن می‌گویم و ازیشان می‌شنوم زیرا ایشان مخاطب اکبر را نمی‌بینند کی ایشان چون صدااند او را نسبت به حال من التفات مستمع عاقل به صدا نباشد چنانک مثل است معروف قال الجدار للوتد لم تشقنی قال الوتد انظر الی من یدقنی}
\label{sec:sh069}
\addcontentsline{toc}{section}{\nameref{sec:sh069}}
\begin{longtable}{l p{0.5cm} r}
احمقانه از سنان رحمت مجو
&&
زان شهی جو کان بود در دست او
\\
باسنان و تیغ لابه چون کنی
&&
کو اسیر آمد به دست آن سنی
\\
او به صنعت آزرست و من صنم
&&
آلتی کو سازدم من آن شوم
\\
گر مرا ساغر کند ساغر شوم
&&
ور مرا خنجر کند خنجر شوم
\\
گر مرا چشمه کند آبی هم
&&
ور مرا آتش کند تابی دهم
\\
گر مرا باران کند خرمن دهم
&&
ور مرا ناوک کند در تن جهم
\\
گر مرا ماری کند زهر افکنم
&&
ور مرا یاری کند خدمت کنم
\\
من چو کلکم در میان اصبعین
&&
نیستم در صف طاعت بین بین
\\
خاک را مشغول کرد او در سخن
&&
یک کفی بربود از آن خاک کهن
\\
ساحرانه در ربود از خاکدان
&&
خاک مشغول سخن چون بی‌خودان
\\
برد تا حق تربت بی‌رای را
&&
تا به مکتب آن گریزان پای را
\\
گفت یزدان که به علم روشنم
&&
که ترا جلاد این خلقان کنم
\\
گفت یا رب دشمنم گیرند خلق
&&
چون فشارم خلق را در مرگ حلق
\\
تو روا داری خداوند سنی
&&
که مرا مبغوض و دشمن‌رو کنی
\\
گفت اسبابی پدید آرم عیان
&&
از تب و قولنج و سرسام و سنان
\\
که بگردانم نظرشان را ز تو
&&
در مرضها و سببهای سه تو
\\
گفت یا رب بندگان هستند نیز
&&
که سببها را بدرند ای عزیز
\\
چشمشان باشد گذاره از سبب
&&
در گذشته از حجب از فضل رب
\\
سرمهٔ توحید از کحال حال
&&
یافته رسته ز علت و اعتلال
\\
ننگرند اندر تب و قولنج و سل
&&
راه ندهند این سببها را به دل
\\
زانک هر یک زین مرضها را دواست
&&
چون دوا نپذیرد آن فعل قضاست
\\
هر مرض دارد دوا می‌دان یقین
&&
چون دوای رنج سرما پوستین
\\
چون خدا خواهد که مردی بفسرد
&&
سردی از صد پوستین هم بگذرد
\\
در وجودش لرزه‌ای بنهد که آن
&&
نه به جامه به شود نه از آشیان
\\
چون قضا آید طبیب ابله شود
&&
وان دوا در نفع هم گمره شود
\\
کی شود محجوب ادراک بصیر
&&
زین سببهای حجاب گول‌گیر
\\
اصل بیند دیده چون اکمل بود
&&
فرع بیند چونک مرد احول بود
\\
\end{longtable}
\end{center}
