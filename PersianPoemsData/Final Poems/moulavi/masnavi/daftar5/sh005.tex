\begin{center}
\section*{بخش ۵ - سبب رجوع کردن آن مهمان به خانهٔ مصطفی علیه‌السلام در آن ساعت که مصطفی نهالین ملوث او را به دست خود می‌شست و خجل شدن او و جامه چاک کردن و نوحهٔ او بر خود و بر سعادت خود}
\label{sec:sh005}
\addcontentsline{toc}{section}{\nameref{sec:sh005}}
\begin{longtable}{l p{0.5cm} r}
کافرک را هیکلی بد یادگار
&&
یاوه دید آن را و گشت او بی‌قرار
\\
گفت آن حجره که شب جا داشتم
&&
هیکل آنجا بی‌خبر بگذاشتم
\\
گر چه شرمین بود شرمش حرص برد
&&
حرص اژدرهاست نه چیزیست خرد
\\
از پی هیکل شتاب اندر دوید
&&
در وثاق مصطفی و آن را بدید
\\
کان یدالله آن حدث را هم به خود
&&
خوش همی‌شوید که دورش چشم بد
\\
هیکلش از یاد رفت و شد پدید
&&
اندرو شوری گریبان را درید
\\
می‌زد او دو دست را بر رو و سر
&&
کله را می‌کوفت بر دیوار و در
\\
آنچنان که خون ز بینی و سرش
&&
شد روان و رحم کرد آن مهترش
\\
نعره‌ها زد خلق جمع آمد برو
&&
گبر گویان ایهاالناس احذروا
\\
می‌زد او بر سر کای بی‌عقل سر
&&
می‌زد او بر سینه کای بی‌نور بر
\\
سجده می‌کرد او کای کل زمین
&&
شرمسارست از تو این جزو مهین
\\
تو که کلی خاضع امر ویی
&&
من که جزوم ظالم و زشت و غوی
\\
تو که کلی خوار و لرزانی ز حق
&&
من که جزوم در خلاف و در سبق
\\
هر زمان می‌کرد رو بر آسمان
&&
که ندارم روی ای قبلهٔ جهان
\\
چون ز حد بیرون بلرزید و طپید
&&
مصطفی‌اش در کنار خود کشید
\\
ساکنش کرد و بسی بنواختش
&&
دیده‌اش بگشاد و داد اشناختش
\\
تا نگرید ابر کی خندد چمن
&&
تا نگرید طفل کی جوشد لبن
\\
طفل یک روزه همی‌داند طریق
&&
که بگریم تا رسد دایهٔ شفیق
\\
تو نمی‌دانی که دایهٔ دایگان
&&
کم دهد بی‌گریه شیر او رایگان
\\
گفت فلیبکوا کثیرا گوش دار
&&
تا بریزد شیر فضل کردگار
\\
گریهٔ ابرست و سوز آفتاب
&&
استن دنیا همین دو رشته تاب
\\
گر نبودی سوز مهر و اشک ابر
&&
کی شدی جسم و عرض زفت و سطبر
\\
کی بدی معمور این هر چار فصل
&&
گر نبودی این تف و این گریه اصل
\\
سوز مهر و گریهٔ ابر جهان
&&
چون همی دارد جهان را خوش‌دهان
\\
آفتاب عقل را در سوز دار
&&
چشم را چون ابر اشک‌افروز دار
\\
چشم گریان بایدت چون طفل خرد
&&
کم خور آن نان را که نان آب تو برد
\\
تن چو با برگست روز و شب از آن
&&
شاخ جان در برگ‌ریزست و خزان
\\
برگ تن بی‌برگی جانست زود
&&
این بباید کاستن آن را فزود
\\
اقرضوا الله قرض ده زین برگ تن
&&
تا بروید در عوض در دل چمن
\\
قرض ده کم کن ازین لقمهٔ تنت
&&
تا نماید وجه لا عین رات
\\
تن ز سرگین خویش چون خالی کند
&&
پر ز مشک و در اجلالی کند
\\
زین پلیدی بدهد و پاکی برد
&&
از یطهرکم تن او بر خورد
\\
دیو می‌ترساندت که هین و هین
&&
زین پشیمان گردی و گردی حزین
\\
گر گدازی زین هوسها تو بدن
&&
بس پشیمان و غمین خواهی شدن
\\
این بخور گرمست و داروی مزاج
&&
وآن بیاشام از پی نفع و علاج
\\
هم بدین نیت که این تن مرکبست
&&
آنچ خو کردست آنش اصوبست
\\
هین مگردان خو که پیش آید خلل
&&
در دماغ و دل بزاید صد علل
\\
این چنین تهدیدها آن دیو دون
&&
آرد و بر خلق خواند صد فسون
\\
خویش جالینوس سازد در دوا
&&
تا فریبد نفس بیمار ترا
\\
کین ترا سودست از درد و غمی
&&
گفت آدم را همین در گندمی
\\
پیش آرد هیهی و هیهات را
&&
وز لویشه پیچد او لبهات را
\\
هم‌چو لبهای فرس و در وقت نعل
&&
تا نماید سنگ کمتر را چو لعل
\\
گوشهاات گیرد او چون گوش اسب
&&
می‌کشاند سوی حرص و سوی کسب
\\
بر زند بر پات نعلی ز اشتباه
&&
که بمانی تو ز درد آن ز راه
\\
نعل او هست آن تردد در دو کار
&&
این کنم یا آن کنم هین هوش دار
\\
آن بکن که هست مختار نبی
&&
آن مکن که کرد مجنون و صبی
\\
حفت الجنه بچه محفوف گشت
&&
بالمکاره که ازو افزود کشت
\\
صد فسون دارد ز حیلت وز دغا
&&
که کند در سله گر هست اژدها
\\
گر بود آب روان بر بنددش
&&
ور بود حبر زمان برخنددش
\\
عقل را با عقل یاری یار کن
&&
امرهم شوری بخوان و کار کن
\\
\end{longtable}
\end{center}
