\begin{center}
\section*{بخش ۹۲ - باز خواندن شه‌زاده نصوح را از بهر دلاکی بعد از استحکام توبه و قبول توبه و بهانه کردن او و دفع گفتن}
\label{sec:sh092}
\addcontentsline{toc}{section}{\nameref{sec:sh092}}
\begin{longtable}{l p{0.5cm} r}
بعد از آن آمد کسی کز مرحمت
&&
دختر سلطان ما می‌خواندت
\\
دختر شاهت همی‌خواند بیا
&&
تا سرش شویی کنون ای پارسا
\\
جز تو دلاکی نمی‌خواهد دلش
&&
که بمالد یا بشوید با گلش
\\
گفت رو رو دست من بی‌کار شد
&&
وین نصوح تو کنون بیمار شد
\\
رو کسی دیگر بجو اشتاب و تفت
&&
که مرا والله دست از کار رفت
\\
با دل خود گفت کز حد رفت جرم
&&
از دل من کی رود آن ترس و گرم
\\
من بمردم یک ره و باز آمدم
&&
من چشیدم تلخی مرگ و عدم
\\
توبه‌ای کردم حقیقت با خدا
&&
نشکنم تا جان شدن از تن جدا
\\
بعد آن محنت کرا بار دگر
&&
پا رود سوی خطر الا که خر
\\
\end{longtable}
\end{center}
