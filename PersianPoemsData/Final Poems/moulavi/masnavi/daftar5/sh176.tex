\begin{center}
\section*{بخش ۱۷۶ - قصد شاه به کشتن امرا و شفاعت کردن  ایاز پیش تخت سلطان کی ای شاه عالم العفو اولی}
\label{sec:sh176}
\addcontentsline{toc}{section}{\nameref{sec:sh176}}
\begin{longtable}{l p{0.5cm} r}
پس ایاز مهرافزا بر جهید
&&
پیش تخت آن الغ سلطان دوید
\\
سجده‌ای کرد و گلوی خود گرفت
&&
کای قبادی کز تو چرخ آرد شگفت
\\
ای همایی که همایان فرخی
&&
از تو دارند و سخاوت هر سخی
\\
ای کریمی که کرمهای جهان
&&
محو گردد پیش ایثارت نهان
\\
ای لطیفی که گل سرخت بدید
&&
از خجالت پیرهن را بر درید
\\
از غفوری تو غفران چشم‌سیر
&&
روبهان بر شیر از عفو تو چیر
\\
جز که عفو تو کرا دارد سند
&&
هر که با امر تو بی‌باکی کند
\\
غفلت و گستاخی این مجرمان
&&
از وفور عفو تست ای عفولان
\\
دایما غفلت ز گستاخی دمد
&&
که برد تعظیم از دیده رمد
\\
غفلت و نسیان بد آموخته
&&
ز آتش تعظیم گردد سوخته
\\
هیبتش بیداری و فطنت دهد
&&
سهو نسیان از دلش بیرون جهد
\\
وقت غارت خواب ناید خلق را
&&
تا بنرباید کسی زو دلق را
\\
خواب چون در می‌رمد از بیم دلق
&&
خواب نسیان کی بود با بیم حلق
\\
لاتؤاخذ ان نسینا شد گواه
&&
که بود نسیان بوجهی هم گناه
\\
زانک استکمال تعظیم او نکرد
&&
ورنه نسیان در نیاوردی نبرد
\\
گرچه نسیان لابد و ناچار بود
&&
در سبب ورزیدن او مختار بود
\\
که تهاون کرد در تعظیمها
&&
تا که نسیان زاد یا سهو و خطا
\\
هم‌چو مستی کو جنایتها کند
&&
گوید او معذور بودم من ز خود
\\
گویدش لیکن سبب ای زشتکار
&&
از تو بد در رفتن آن اختیار
\\
بی‌خودی نامد بخود تش خواندی
&&
اختیارت خود نشد تش راندی
\\
گر رسیدی مستی بی‌جهد تو
&&
حفظ کردی ساقی جان عهد تو
\\
پشت‌دارت بودی او و عذرخواه
&&
من غلام زلت مست اله
\\
عفوهای جمله عالم ذره‌ای
&&
عکس عفوت ای ز تو هر بهره‌ای
\\
عفوها گفته ثنای عفو تو
&&
نیست کفوش ایها الناس اتقوا
\\
جانشان بخش و ز خودشان هم مران
&&
کام شیرین تو اند ای کامران
\\
رحم کن بر وی که روی تو بدید
&&
فرقت تلخ تو چون خواهد کشید
\\
از فراق و هجر می‌گویی سخن
&&
هر چه خواهی کن ولیکن این مکن
\\
صد هزاران مرگ تلخ شصت تو
&&
نیست مانند فراق روی تو
\\
تلخی هجر از ذکور و از اناث
&&
دور دار ای مجرمان را مستغاث
\\
بر امید وصل تو مردن خوشست
&&
تلخی هجر تو فوق آتشست
\\
گبر می‌گوید میان آن سقر
&&
چه غمم بودی گرم کردی نظر
\\
کان نظر شیرین کنندهٔ رنجهاست
&&
ساحران را خونبهای دست و پاست
\\
\end{longtable}
\end{center}
