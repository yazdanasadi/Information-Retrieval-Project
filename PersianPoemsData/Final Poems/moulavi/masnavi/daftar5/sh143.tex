\begin{center}
\section*{بخش ۱۴۳ - حکایت آن مذن زشت آواز کی در کافرستان بانگ نماز داد و مرد کافری او را هدیه داد}
\label{sec:sh143}
\addcontentsline{toc}{section}{\nameref{sec:sh143}}
\begin{longtable}{l p{0.5cm} r}
یک مؤذن داشت بس آواز بد
&&
در میان کافرستان بانگ زد
\\
چند گفتندش مگو بانگ نماز
&&
که شود جنگ و عداوتها دراز
\\
او ستیزه کرد و پس بی‌احتراز
&&
گفت در کافرستان بانگ نماز
\\
خلق خایف شد ز فتنهٔ عامه‌ای
&&
خود بیامد کافری با جامه‌ای
\\
شمع و حلوا با چنان جامهٔ لطیف
&&
هدیه آورد و بیامد چون الیف
\\
پرس پرسان کین مؤذن کو کجاست
&&
که صلا و بانگ او راحت‌فزاست
\\
هین چه راحت بود زان آواز زشت
&&
گفت که آوازش فتاد اندر کنشت
\\
دختری دارم لطیف و بس سنی
&&
آرزو می‌بود او رامؤمنی
\\
هیچ این سودا نمی‌رفت از سرش
&&
پندها می‌داد چندین کافرش
\\
در دل او مهر ایمان رسته بود
&&
هم‌چو مجمر بود این غم من چو عود
\\
در عذاب و درد و اشکنجه بدم
&&
که بجنبد سلسلهٔ او دم به دم
\\
هیچ چاره می‌ندانستم در آن
&&
تا فرو خواند این مؤذن آن اذان
\\
گفت دختر چیست این مکروه بانگ
&&
که بگوشم آمد این دو چار دانگ
\\
من همه عمر این چنین آواز زشت
&&
هیچ نشنیدم درین دیر و کنشت
\\
خوهرش گفتا که این بانگ اذان
&&
هست اعلام و شعار مؤمنان
\\
باورش نامد بپرسید از دگر
&&
آن دگر هم گفت آری ای پدر
\\
چون یقین گشتش رخ او زرد شد
&&
از مسلمانی دل او سرد شد
\\
باز رستم من ز تشویش و عذاب
&&
دوش خوش خفتم در آن بی‌خوف خواب
\\
راحتم این بود از آواز او
&&
هدیه آوردم به شکر آن مرد کو
\\
چون بدیدش گفت این هدیه پذیر
&&
که مرا گشتی مجیر و دستگیر
\\
آنچ کردی با من از احسان و بر
&&
بندهٔ تو گشته‌ام من مستمر
\\
گر به مال و ملک و ثروت فردمی
&&
من دهانت را پر از زر کردمی
\\
هست ایمان شما زرق و مجاز
&&
راه‌زن هم‌چون که آن بانگ نماز
\\
لیک از ایمان و صدق بایزید
&&
چند حسرت در دل و جانم رسید
\\
هم‌چو آن زن کو جماع خر بدید
&&
گفت آوه چیست این فحل فرید
\\
گر جماع اینست بردند این خران
&&
بر کس ما می‌ریند این شوهران
\\
داد جمله داد ایمان بایزید
&&
آفرینها بر چنین شیر فرید
\\
قطره‌ای ز ایمانش در بحر ار رود
&&
بحر اندر قطره‌اش غرقه شود
\\
هم‌چو ز آتش ذره‌ای در بیشه‌ها
&&
اندر آن ذره شود بیشه فنا
\\
چون خیالی در دل شه یا سپاه
&&
کرد اندر جنگ خصمان را تباه
\\
یک ستاره در محمد رخ نمود
&&
تا فنا شد گوهر گبر و جهود
\\
آنک ایمان یافت رفت اندر امان
&&
کفرهای باقیان شد دو گمان
\\
کفر صرف اولین باری نماند
&&
یا مسلمانی و یا بیمی نشاند
\\
این به حیله آب و روغن کردنیست
&&
این مثلها کفو ذرهٔ نور نیست
\\
ذره نبود جز حقیری منجسم
&&
ذره نبود شارق لا ینقسم
\\
گفتن ذره مرادی دان خفی
&&
محرم دریا نه‌ای این دم کفی
\\
آفتاب نیر ایمان شیخ
&&
گر نماید رخ ز شرق جان شیخ
\\
جمله پستی گنج گیرد تا ثری
&&
جمله بالا خلد گیرد اخضری
\\
او یکی جان دارد از نور منیر
&&
او یکی تن دارد از خاک حقیر
\\
ای عجب اینست او یا آن بگو
&&
که بماندم اندرین مشکل عمو
\\
گر وی اینست ای برادر چیست آن
&&
پر شده از نور او هفت آسمان
\\
ور وی آنست این بدن ای دوست چیست
&&
ای عجب زین دو کدامین است و کیست
\\
\end{longtable}
\end{center}
