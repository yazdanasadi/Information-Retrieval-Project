\begin{center}
\section*{بخش ۱۶۲ - حکایت عیاضی رحمه‌الله کی هفتاد غزو کرده بود سینه برهنه بر امید شهید شدن چون از آن نومید شد از جهاد اصغر رو به جهاد اکبر آورد و خلوت گزید ناگهان طبل غازیان شنید نفس از اندرون زنجیر می‌درانید سوی غزا و متهم داشتن او نفس خود را درین رغبت}
\label{sec:sh162}
\addcontentsline{toc}{section}{\nameref{sec:sh162}}
\begin{longtable}{l p{0.5cm} r}
گفت عیاضی نود بار آمدم
&&
تن برهنه بوک زخمی آیدم
\\
تن برهنه می‌شدم در پیش تیر
&&
تا یکی تیری خورم من جای‌گیر
\\
تیر خوردن بر گلو یا مقتلی
&&
در نیابد جز شهیدی مقبلی
\\
بر تنم یک جایگه بی‌زخم نیست
&&
این تنم از تیر چون پرویز نیست
\\
لیک بر مقتل نیامد تیرها
&&
کار بخت است این نه جلدی و دها
\\
چون شهیدی روزی جانم نبود
&&
رفتم اندر خلوت و در چله زود
\\
در جهاد اکبر افکندم بدن
&&
در ریاضت کردن و لاغر شدن
\\
بانگ طبل غازیان آمد به گوش
&&
که خرامیدند جیش غزوکوش
\\
نفس از باطن مرا آواز داد
&&
که به گوش حس شنیدم بامداد
\\
خیز هنگام غزا آمد برو
&&
خویش را در غزو کردن کن گرو
\\
گفتم ای نفس خبیث بی‌وفا
&&
از کجا میل غزا تو از کجا
\\
راست گوی ای نفس کین حیلت‌گریست
&&
ورنه نفس شهوت از طاعت بریست
\\
گر نگویی راست حمله آرمت
&&
در ریاضت سخت‌تر افشارمت
\\
نفس بانگ آورد آن دم از درون
&&
با فصاحت بی‌دهان اندر فسون
\\
که مرا هر روز اینجا می‌کشی
&&
جان من چون جان گبران می‌کشی
\\
هیچ کس را نیست از حالم خبر
&&
که مرا تو می‌کشی بی‌خواب و خور
\\
در غزا بجهم به یک زخم از بدن
&&
خلق بیند مردی و ایثار من
\\
گفتم ای نفسک منافق زیستی
&&
هم منافق می‌مری تو چیستی
\\
در دو عالم تو مرایی بوده‌ای
&&
در دو عالم تو چنین بیهوده‌ای
\\
نذر کردم که ز خلوت هیچ من
&&
سر برون نارم چو زنده‌ست این بدن
\\
زانک در خلوت هر آنچ تن کند
&&
نه از برای روی مرد و زن کند
\\
جنبش و آرامش اندر خلوتش
&&
جز برای حق نباشد نیتش
\\
این جهاد اکبرست آن اصغرست
&&
هر دو کار رستمست و حیدرست
\\
کار آن کس نیست کو را عقل و هوش
&&
پرد از تن چون بجنبد دنب موش
\\
آن چنان کس را بباید چون زنان
&&
دور بودن از مصاف و از سنان
\\
صوفیی آن صوفیی این اینت حیف
&&
آن ز سوزن کشته این را طعمه سیف
\\
نقش صوفی باشد او را نیست جان
&&
صوفیان بدنام هم زین صوفیان
\\
بر در و دیوار جسم گل‌سرشت
&&
حق ز غیرت نقش صد صوفی نبشت
\\
تا ز سحر آن نقشها جنبان شود
&&
تا عصای موسوی پنهان شود
\\
نقشها را میخورد صدق عصا
&&
چشم فرعونیست پر گرد و حصا
\\
صوفی دیگر میان صف حرب
&&
اندر آمد بیست بار از بهر ضرب
\\
با مسلمانان به کافر وقت کر
&&
وانگشت او با مسلمانان به فر
\\
زخم خورد و بست زخمی را که خورد
&&
بار دیگر حمله آورد و نبرد
\\
تا نمیرد تن به یک زخم از گزاف
&&
تا خورد او بیست زخم اندر مصاف
\\
حیفش آمد که به زخمی جان دهد
&&
جان ز دست صدق او آسان رهد
\\
\end{longtable}
\end{center}
