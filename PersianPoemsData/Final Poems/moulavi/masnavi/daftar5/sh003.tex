\begin{center}
\section*{بخش ۳ - در سبب ورود این حدیث مصطفی صلوات الله علیه که الکافر یاکل فی سبعة امعاء و الممن یاکل فی معا واحد}
\label{sec:sh003}
\addcontentsline{toc}{section}{\nameref{sec:sh003}}
\begin{longtable}{l p{0.5cm} r}
کافران مهمان پیغامبر شدند
&&
وقت شام ایشان به مسجد آمدند
\\
که آمدیم ای شاه ما اینجا قنق
&&
ای تو مهمان‌دار سکان افق
\\
بی‌نواییم و رسیده ما ز دور
&&
هین بیفشان بر سر ما فضل و نور
\\
گفت ای یاران من قسمت کنید
&&
که شما پر از من و خوی منید
\\
پر بود اجسام هر لشکر ز شاه
&&
زان زنندی تیغ بر اعدای جاه
\\
تو بخشم شه زنی آن تیغ را
&&
ورنه بر اخوان چه خشم آید ترا
\\
بر برادر بی‌گناهی می‌زنی
&&
عکس خشم شاه گرز ده‌منی
\\
شه یکی جانست و لشکر پر ازو
&&
روح چون آبست واین اجسام جو
\\
آب روح شاه اگر شیرین بود
&&
جمله جوها پر ز آب خوش شود
\\
که رعیت دین شه دارند و بس
&&
این چنین فرمود سلطان عبس
\\
هر یکی یاری یکی مهمان گزید
&&
در میان یک زفت بود و بی‌ندید
\\
جشم ضخمی داشت کس او را نبرد
&&
ماند در مسجد چو اندر جام درد
\\
مصطفی بردش چو وا ماند از همه
&&
هفت بز بد شیرده اندر رمه
\\
که مقیم خانه بودندی بزان
&&
بهر دوشیدن برای وقت خوان
\\
نان و آش و شیر آن هر هفت بز
&&
خورد آن بوقحط عوج ابن غز
\\
جمله اهل بیت خشم‌آلو شدند
&&
که همه در شیر بز طامع بدند
\\
معده طبلی‌خوار هم‌چون طبل کرد
&&
قسم هژده آدمی تنها بخورد
\\
وقت خفتن رفت و در حجره نشست
&&
پس کنیزک از غضب در را ببست
\\
از برون زنجیر در را در فکند
&&
که ازو بد خشمگین و دردمند
\\
گبر را در نیم‌شب یا صبحدم
&&
چون تقاضا آمد و درد شکم
\\
از فراش خویش سوی در شتافت
&&
دست بر در چون نهاد او بسته یافت
\\
در گشادن حیله کرد آن حیله‌ساز
&&
نوع نوع و خود نشد آن بند باز
\\
شد تقاضا بر تقاضا خانه تنگ
&&
ماند او حیران و بی‌درمان و دنگ
\\
حیله کرد او و به خواب اندر خزید
&&
خویشتن در خواب در ویرانه دید
\\
زانک ویرانه بد اندر خاطرش
&&
شد به خواب اندر همانجا منظرش
\\
خویش در ویرانهٔ خالی چو دید
&&
او چنان محتاج اندر دم برید
\\
گشت بیدار و بدید آن جامه خواب
&&
پر حدث دیوانه شد از اضطراب
\\
ز اندرون او برآمد صد خروش
&&
زین چنین رسواییی بی خاک‌پوش
\\
گفت خوابم بتر از بیداریم
&&
گه خورم این سو و آن سو می‌ریم
\\
بانگ می‌زد وا ثبورا وا ثبور
&&
هم‌چنانک کافر اندر قعر گور
\\
منتظر که کی شود این شب به سر
&&
یا برآید در گشادن بانگ در
\\
تا گریزد او چو تیری از کمان
&&
تا نبیند هیچ کس او را چنان
\\
قصه بسیارست کوته می‌کنم
&&
باز شد آن در رهید از درد و غم
\\
\end{longtable}
\end{center}
