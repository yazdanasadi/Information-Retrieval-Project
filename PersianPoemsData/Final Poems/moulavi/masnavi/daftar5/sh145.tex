\begin{center}
\section*{بخش ۱۴۵ - حکایت آن امیر کی غلام را گفت کی می بیار غلام رفت و سبوی می آورد در راه زاهدی بود امر معروف کرد زد سنگی و سبو را بشکست امیر بشنید و قصد گوشمال زاهد کرد و این قصد در عهد دین عیسی بود علیه‌السلام کی هنوز می حرام نشده بود ولیکن زاهد تقزیزی می‌کرد و از تنعم منع می‌کرد}
\label{sec:sh145}
\addcontentsline{toc}{section}{\nameref{sec:sh145}}
\begin{longtable}{l p{0.5cm} r}
بود امیری خوش دلی می‌باره‌ای
&&
کهف هر مخمور و هر بیچاره‌ای
\\
مشفقی مسکین‌نوازی عادلی
&&
جوهری زربخششی دریادلی
\\
شاه مردان و امیرالمؤمنین
&&
راه‌بان و رازدان و دوست‌بین
\\
دور عیسی بود و ایام مسیح
&&
خلق دلدار و کم‌آزار و ملیح
\\
آمدش مهمان بناگاهان شبی
&&
هم امیری جنس او خوش‌مذهبی
\\
باده می‌بایستشان در نظم حال
&&
باده بود آن وقت ماذون و حلال
\\
باده‌شان کم بود و گفتا ای غلام
&&
رو سبو پر کن به ما آور مدام
\\
از فلان راهب که دارد خمر خاص
&&
تا ز خاص و عام یابد جان خلاص
\\
جرعه‌ای زان جام راهب آن کند
&&
که هزاران جره و خمدان کند
\\
اندر آن می مایهٔ پنهانی است
&&
آنچنان که اندر عبا سلطانی است
\\
تو بدلق پاره‌پاره کم نگر
&&
که سیه کردند از بیرون زر
\\
از برای چشم بد مردود شد
&&
وز برون آن لعل دودآلود شد
\\
گنج و گوهر کی میان خانه‌هاست
&&
گنجها پیوسته در ویرانه‌هاست
\\
گنج آدم چون بویران بد دفین
&&
گشت طینش چشم‌بند آن لعین
\\
او نظر می‌کرد در طین سست سست
&&
جان همی‌گفتش که طینم سد تست
\\
دو سبو بستد غلام و خوش دوید
&&
در زمان در دیر رهبانان رسید
\\
زر بداد و بادهٔ چون زر خرید
&&
سنگ داد و در عوض گوهر خرید
\\
باده‌ای که آن بر سر شاهان جهد
&&
تاج زر بر تارک ساقی نهد
\\
فتنه‌ها و شورها انگیخته
&&
بندگان و خسروان آمیخته
\\
استخوانها رفته جمله جان شده
&&
تخت و تخته آن زمان یکسان شده
\\
وقت هشیاری چو آب و روغنند
&&
وقت مستی هم‌چو جان اندر تنند
\\
چون هریسه گشته آنجا فرق نیست
&&
نیست فرقی کاندر آنجا غرق نیست
\\
این چنین باده همی‌برد آن غلام
&&
سوی قصر آن امیر نیک‌نام
\\
پیشش آمد زاهدی غم دیده‌ای
&&
خشک مغزی در بلا پیچیده‌ای
\\
تن ز آتشهای دل بگداخته
&&
خانه از غیر خدا پرداخته
\\
گوشمال محنت بی‌زینهار
&&
داغها بر داغها چندین هزار
\\
دیده هر ساعت دلش در اجتهاد
&&
روز و شب چفسیده او بر اجتهاد
\\
سال و مه در خون و خاک آمیخته
&&
صبر و حلمش نیم‌شب بگریخته
\\
گفت زاهد در سبوها چیست آن
&&
گفت باده گفت آن کیست آن
\\
گفت آن آن فلان میر اجل
&&
گفت طالب را چنین باشد عمل
\\
طالب یزدان و آنگه عیش و نوش
&&
بادهٔ شیطان و آنگه نیم هوش
\\
هوش تو بی می چنین پژمرده است
&&
هوشها باید بر آن هوش تو بست
\\
تا چه باشد هوش تو هنگام سکر
&&
ای چو مرغی گشته صید دام سکر
\\
\end{longtable}
\end{center}
