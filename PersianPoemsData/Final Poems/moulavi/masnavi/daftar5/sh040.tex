\begin{center}
\section*{بخش ۴۰ - حکایت محمد خوارزمشاه کی شهر سبزوار کی همه رافضی باشند به جنگ بگرفت اما جان خواستند گفت آنگه امان دهم کی ازین شهر پیش من به هدیه ابوبکر نامی بیارید}
\label{sec:sh040}
\addcontentsline{toc}{section}{\nameref{sec:sh040}}
\begin{longtable}{l p{0.5cm} r}
شد محمد الپ الغ خوارزمشاه
&&
در قتال سبزوار پر پناه
\\
تنگشان آورد لشکرهای او
&&
اسپهش افتاد در قتل عدو
\\
سجده آوردند پیشش کالامان
&&
حلقه‌مان در گوش کن وا بخش جان
\\
هر خراج و صلتی که بایدت
&&
آن ز ما هر موسمی افزایدت
\\
جان ما آن توست ای شیرخو
&&
پیش ما چندی امانت باش گو
\\
گفت نرهانید از من جان خویش
&&
تا نیاریدم ابوبکری به پیش
\\
تا مرا بوبکر نام از شهرتان
&&
هدیه نارید ای رمیده امتان
\\
بدرومتان هم‌چو کشت ای قوم دون
&&
نه خراج استانم و نه هم فسون
\\
بس جوال زر کشیدندش به راه
&&
کز چنین شهری ابوبکری مخواه
\\
کی بود بوبکر اندر سبزوار
&&
یا کلوخ خشک اندر جویبار
\\
رو بتابید از زر و گفت ای مغان
&&
تا نیاریدم ابوبکر ارمغان
\\
هیچ سودی نیست کودک نیستم
&&
تا به زر و سیم حیران بیستم
\\
تا نیاری سجده نرهی ای زبون
&&
گر بپیمایی تو مسجد را به کون
\\
منهیان انگیختند از چپ و راست
&&
که اندرین ویرانه بوبکری کجاست
\\
بعد سه روز و سه شب که اشتافتند
&&
یک ابوبکری نزاری یافتند
\\
ره گذر بود و بمانده از مرض
&&
در یکی گوشهٔ خرابه پر حرض
\\
خفته بود او در یکی کنجی خراب
&&
چون بدیدندش بگفتندش شتاب
\\
خیز که سلطان ترا طالب شدست
&&
کز تو خواهد شهر ما از قتل رست
\\
گفت اگر پایم بدی یا مقدمی
&&
خود به راه خود به مقصد رفتمی
\\
اندرین دشمن‌کده کی ماندمی
&&
سوی شهر دوستان می‌راندمی
\\
تختهٔ مرده‌کشان بفراشتند
&&
وان ابوبکر مرا برداشتند
\\
سوی خوارمشاه حمالان کشان
&&
می‌کشیدندش که تا بیند نشان
\\
سبزوارست این جهان و مرد حق
&&
اندرین جا ضایعست و ممتحق
\\
هست خوارمشاه یزدان جلیل
&&
دل همی خواهد ازین قوم رذیل
\\
گفت لا ینظر الی تصویرکم
&&
فابتغوا ذا القلب فی‌تدبیر کم
\\
من ز صاحب‌دل کنم در تو نظر
&&
نه به نقش سجده و ایثار زر
\\
تو دل خود را چو دل پنداشتی
&&
جست و جوی اهل دل بگذاشتی
\\
دل که گر هفصد چو این هفت آسمان
&&
اندرو آید شود یاوه و نهان
\\
این چنین دل ریزه‌ها را دل مگو
&&
سبزوار اندر ابوبکری بجو
\\
صاحب دل آینهٔ شش‌رو شود
&&
حق ازو در شش جهت ناظر بود
\\
هر که اندر شش جهت دارد مقر
&&
نکندش بی‌واسطهٔ او حق نظر
\\
گر کند رد از برای او کند
&&
ور قبول آرد همو باشد سند
\\
بی‌ازو ندهد کسی را حق نوال
&&
شمه‌ای گفتم من از صاحب‌وصال
\\
موهبت را بر کف دستش نهد
&&
وز کفش آن را به مرحومان دهد
\\
با کفش دریای کل را اتصال
&&
هست بی‌چون و چگونه و بر کمال
\\
اتصالی که نگنجد در کلام
&&
گفتنش تکلیف باشد والسلام
\\
صد جوال زر بیاری ای غنی
&&
حق بگوید دل بیار ای منحنی
\\
گر ز تو راضیست دل من راضیم
&&
ور ز تو معرض بود اعراضیم
\\
ننگرم در تو در آن دل بنگرم
&&
تحفه او را آر ای جان بر درم
\\
با تو او چونست هستم من چنان
&&
زیر پای مادران باشد جنان
\\
مادر و بابا و اصل خلق اوست
&&
ای خنک آنکس که داند دل ز پوست
\\
تو بگویی نک دل آوردم به تو
&&
گویدت پرست ازین دلها قتو
\\
آن دلی آور که قطب عالم اوست
&&
جان جان جان جان آدم اوست
\\
از برای آن دل پر نور و بر
&&
هست آن سلطان دلها منتظر
\\
تو بگردی روزها در سبزوار
&&
آنچنان دل را نیابی ز اعتبار
\\
پس دل پژمردهٔ پوسیده‌جان
&&
بر سر تخته نهی آن سو کشان
\\
که دل آوردم ترا ای شهریار
&&
به ازین دل نبود اندر سبزوار
\\
گویدت این گورخانه‌ست ای جری
&&
که دل مرده بدینجا آوری
\\
رو بیاور آن دلی کو شاه‌خوست
&&
که امان سبزوار کون ازوست
\\
گویی آن دل زین جهان پنهان بود
&&
زانک ظلمت با ضیا ضدان بود
\\
دشمنی آن دل از روز الست
&&
سبزوار طبع را میراثی است
\\
زانک او بازست و دنیا شهر زاغ
&&
دیدن ناجنس بر ناجنس داغ
\\
ور کند نرمی نفاقی می‌کند
&&
ز استمالت ارتفاقی می‌کند
\\
می‌کند آری نه از بهر نیاز
&&
تا که ناصح کم کند نصح دراز
\\
زانک این زاغ خس مردارجو
&&
صد هزاران مکر دارد تو به تو
\\
گر پذیرند آن نفاقش را رهید
&&
شد نفاقش عین صدق مستفید
\\
زانک آن صاحب دل با کر و فر
&&
هست در بازار ما معیوب‌خر
\\
صاحب دل جو اگر بی‌جان نه‌ای
&&
جنس دل شو گر ضد سلطان نه‌ای
\\
آنک زرق او خوش آید مر ترا
&&
آن ولی تست نه خاص خدا
\\
هر که او بر خو و بر طبع تو زیست
&&
پیش طبع تو ولی است و نبیست
\\
رو هوا بگذار تا بویت شود
&&
وان مشام خوش عبرجویت شود
\\
از هوارانی دماغت فاسدست
&&
مشک و عنبر پیش مغزت کاسدست
\\
حد ندارد این سخن و آهوی ما
&&
می‌گریزد اندر آخر جابجا
\\
\end{longtable}
\end{center}
