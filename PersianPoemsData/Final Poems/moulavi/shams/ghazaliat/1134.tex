\begin{center}
\section*{غزل شماره ۱۱۳۴: چرا ز قافله یک کس نمی‌شود بیدار}
\label{sec:1134}
\addcontentsline{toc}{section}{\nameref{sec:1134}}
\begin{longtable}{l p{0.5cm} r}
چرا ز قافله یک کس نمی‌شود بیدار
&&
که رخت عمر ز کی باز می‌برد طرار
\\
چرا ز خواب و ز طرار می‌نیازاری
&&
چرا از او که خبر می‌کند کنی آزار
\\
تو را هر آنک بیازرد شیخ و واعظ توست
&&
که نیست مهر جهان را چو نقش آب قرار
\\
یکی همیشه همی‌گفت راز با خانه
&&
مشو خراب به ناگه مرا بکن اخبار
\\
شبی به ناگه خانه بر او فرود آمد
&&
چه گفت گفت کجا شد وصیت بسیار
\\
نگفتمت خبرم کن تو پیش از افتادن
&&
که چاره سازم من با عیال خود به فرار
\\
خبر نکردی ای خانه کو حق صحبت
&&
فروفتادی و کشتی مرا به زاری زار
\\
جواب گفت مر او را فصیح آن خانه
&&
که چند چند خبر کردمت به لیل و نهار
\\
بدان طرف که دهان را گشادمی بشکاف
&&
که قوتم برسیدست وقت شد هش دار
\\
همی‌زدی به دهانم ز حرص مشتی گل
&&
شکاف‌ها همی‌بستی سراسر دیوار
\\
ز هر کجا که گشادم دهان فروبستی
&&
نهشتیم که بگویم چه گویم ای معمار
\\
بدان که خانه تن توست و رنج‌ها چو شکاف
&&
شکاف رنج به دارو گرفتی ای بیمار
\\
مثال کاه و گلست آن مزوره و معجون
&&
هلا تو کاه گل اندر شکاف می‌افشار
\\
دهان گشاید تن تا بگویدت رفتم
&&
طبیب آید و بندد بر او ره گفتار
\\
خمار درد سرت از شراب مرگ شناس
&&
مده شراب بنفشه بهل شراب انار
\\
وگر دهی تو به عادت دهش که روپوشست
&&
چه روی پوشی زان کوست عالم الاسرار
\\
بخور شراب انابت بساز قرص ورع
&&
ز توبه ساز تو معجون غذا ز استغفار
\\
بگیر نبض دل و دین خود ببین چونی
&&
نگاه کن تو به قاروره عمل یک بار
\\
به حق گریز که آب حیات او دارد
&&
تو زینهار از او خواه هر نفس زنهار
\\
اگر کیست بگوید که خواست فایده نیست
&&
بگو که خواست از او خاست چون بود بی‌کار
\\
مرید چیست به تازی مرید خواهنده
&&
مرید از آن مرادست و صید از آن شکار
\\
اگر نخواست مرا پس چرام خواهان کرد
&&
که زرد کرد رخم را فراق آن رخسار
\\
وگر نه غمزه او زد به تیغ عشق مرا
&&
چراست این دل من خون و چشم من خونبار
\\
خزان مرید بهارست زرد و آه کنان
&&
نه عاقبت به سر او رسید شیخ بهار
\\
چو زنده گشت مرید بهار و مرده نماند
&&
مرید حق ز چه ماند میان ره مردار
\\
به سوی باغ بیا و جزای فعل ببین
&&
شکوفه لایق هر تخم پاک در اظهار
\\
چو واعظان خضرکسوه بهار ای جان
&&
زبان حال گشا و خموش باش ای یار
\\
\end{longtable}
\end{center}
