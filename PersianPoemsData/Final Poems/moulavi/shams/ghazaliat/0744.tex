\begin{center}
\section*{غزل شماره ۷۴۴: رو ترش کردی مگر دی باده‌ات گیرا نبود}
\label{sec:0744}
\addcontentsline{toc}{section}{\nameref{sec:0744}}
\begin{longtable}{l p{0.5cm} r}
رو ترش کردی مگر دی باده‌ات گیرا نبود
&&
ساقیت بیگانه بود و آن شه زیبا نبود
\\
یا به قاصد رو ترش کردی ز بیم چشم بد
&&
بر کدامین یوسف از چشم بدان غوغا نبود
\\
چشم بد خستش ولیکن عاقبت محمود بود
&&
چشم بد با حفظ حق جز باطل و سودا نبود
\\
هین مترس از چشم بد وان ماه را پنهان مکن
&&
آن مه نادر که او در خانه جوزا نبود
\\
در دل مردان شیرین جمله تلخی‌های عشق
&&
جز شراب و جز کباب و شکر و حلوا نبود
\\
این شراب و نقل و حلوا هم خیال احولست
&&
اندر آن دریای بی‌پایان به جز دریا نبود
\\
یک زمان گرمی به کاری یک زمان سردی در آن
&&
جز به فرمان حق این گرما و این سرما نبود
\\
هین خمش کن در خموشی نعره می‌زن روح وار
&&
تو کی دیدی زین خموشان کو به جان گویا نبود
\\
\end{longtable}
\end{center}
