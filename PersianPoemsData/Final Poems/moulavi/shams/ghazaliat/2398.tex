\begin{center}
\section*{غزل شماره ۲۳۹۸: آن دم که دررباید باد از رخ تو پرده}
\label{sec:2398}
\addcontentsline{toc}{section}{\nameref{sec:2398}}
\begin{longtable}{l p{0.5cm} r}
آن دم که دررباید باد از رخ تو پرده
&&
زنده شود بجنبد هر جا که هست مرده
\\
از جنگ سوی ساز آ وز ناز و خشم بازآ
&&
ای رخت‌های خود را از رخت ما نورده
\\
ای بخت و بامرادی کاندر صبوح شادی
&&
آن جام کیقبادی تو داده ما بخورده
\\
اندیشه کرد سیران در هجر و گشت سکران
&&
صافت چگونه باشد چون جان فزاست درده
\\
تو آفتاب مایی از کوه اگر برآیی
&&
چه جوش‌ها برآرد این عالم فسرده
\\
ای دوش لب گشاده داد نبات داده
&&
خوش وعده‌ای نهاده ما روزها شمرده
\\
بر باده و بر افیون عشق تو برفزوده
&&
و از آفتاب و از مه رویت گرو ببرده
\\
ای شیر هر شکاری آخر روا نداری
&&
دل را به خرده گیری سوزیش همچو خرده
\\
گر چه در این جهانم فتوی نداد جانم
&&
گرد و دراز گشتن بر طمع نیم گرده
\\
ای دوست چند گویی که از چه زردرویی
&&
صفراییم برآرم در شور خویش زرده
\\
کی رغم چشم بد را آری تو جعد خود را
&&
کاین را به تو سپردم ای دل به ما سپرده
\\
نی با تو اتفاقم نی صبر در فراقم
&&
ز آسیب این دو حالت جان می‌شود فشرده
\\
هم تو بگو که گفتت کالنقش فی الحجر شد
&&
گفتار ما ز دل‌ها زو می‌شود سترده
\\
\end{longtable}
\end{center}
