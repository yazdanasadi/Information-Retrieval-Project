\begin{center}
\section*{غزل شماره ۳۰۲۶: خیره چرا گشته‌ای خواجه مگر عاشقی}
\label{sec:3026}
\addcontentsline{toc}{section}{\nameref{sec:3026}}
\begin{longtable}{l p{0.5cm} r}
خیره چرا گشته‌ای خواجه مگر عاشقی
&&
کاسه بزن کوزه خور خواجه اگر عاشقی
\\
کاش بدانستیی بر چه در ایستاده‌ای
&&
کاش بدانستیی بر چه قمر عاشقی
\\
چشمه آن آفتاب خواب نبیند فلک
&&
چشمت از او روشنست تیزنظر عاشقی
\\
شیر فلک زین خطر خون شده استش جگر
&&
راست بگویم مرنج سخته جگر عاشقی
\\
ای گل تر راست گو بر چه دریدی قبا
&&
ای مه لاغرشده بر چه سحر عاشقی
\\
ای دل دریاصفت موج تو ز اندیشه‌هاست
&&
هر دم کف می‌کنی بر چه گهر عاشقی
\\
آنک از او گشت دنگ غم نخورد از خدنگ
&&
ور تو سپر بفکنی سسته سپر عاشقی
\\
جمله اجزای خاک هست چو ما عشقناک
&&
لیک تو ای روح پاک نادره‌تر عاشقی
\\
ای خرد ار بحریی دم مزن و دم بخور
&&
چون هنرت خامشیست بر چه هنر عاشقی
\\
\end{longtable}
\end{center}
