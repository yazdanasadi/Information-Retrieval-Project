\begin{center}
\section*{غزل شماره ۱۹۹۸: به خدا گل ز تو آموخت شکر خندیدن}
\label{sec:1998}
\addcontentsline{toc}{section}{\nameref{sec:1998}}
\begin{longtable}{l p{0.5cm} r}
به خدا گل ز تو آموخت شکر خندیدن
&&
به خدا که ز تو آموخت کمر بندیدن
\\
به خدا چرخ همان دید که من دیدستم
&&
ور نه دیدی ز چه بودیش به سر گردیدن
\\
گفتم ای نی تو چنین زار چرا می نالی
&&
گفت خوردم دم او شرط بود نالیدن
\\
گفتم ای ماه نو این جمله گداز تو ز چیست
&&
گفت کاهش دهدم فایده بالیدن
\\
فایده زفت شدن در کمی و کاستن است
&&
از پی خرج بود مکسبه‌ها ورزیدن
\\
پر پروانه پی درک تف شمع بود
&&
چونک آن یافت نخواهد پر و دریازیدن
\\
در فنا جلوه شود فایده هستی‌ها
&&
پس نباید ز بلا گریه و درچغزیدن
\\
پس خمش باش همی‌خور ز کمان‌هاش خدنگ
&&
چون هنر در کمیت خواهد افزاییدن
\\
\end{longtable}
\end{center}
