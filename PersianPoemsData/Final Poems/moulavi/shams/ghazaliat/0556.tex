\begin{center}
\section*{غزل شماره ۵۵۶: جور و جفا و دوریی کان کنکار می‌کند}
\label{sec:0556}
\addcontentsline{toc}{section}{\nameref{sec:0556}}
\begin{longtable}{l p{0.5cm} r}
جور و جفا و دوریی کان کنکار می‌کند
&&
بر دل و جان عاشقان چون کنه کار می‌کند
\\
هم تک یار یار کو راحت مطلقست او
&&
یار ز حکم و داوری با تو چه یار می‌کند
\\
یک صفتی قرین شود چرخ بدو زمین شود
&&
یک صفتی خریف را فصل بهار می‌کند
\\
از صفتی فرشته را دیو و بلیس می‌کند
&&
وز تبشی شب مرا رشک بهار می‌کند
\\
می زده را معالجه هم به می از چه می‌کند
&&
اشتر مست را ز می باز چه بار می‌کند
\\
از کف پیر میکده مجلسیان خرف شده
&&
دور ز حد گذشت کو آن که شمار می‌کند
\\
هست شد آن عدم که او دولت هست‌ها بود
&&
مست شد آن خرد که او یاد خمار می‌کند
\\
عشرت خشک لب شده آمد و تر همی‌زند
&&
آن تریی که اندر او آب غبار می‌کند
\\
ساقی جان بیا که دل بی‌تو شدست مشتغل
&&
تا که نبیند او تو را با کی قرار می‌کند
\\
جزو دوید تا به کل خار گرفت صدر گل
&&
جذبه خارخار بین کان دل خار می‌کند
\\
مطرب جان بیا بزن تنتن تن تنن تنن
&&
کاین دل مست از به گه یاد نگار می‌کند
\\
یاد نگار می‌کند قصد کنار می‌کند
&&
روح نثار می‌کند شیر شکار می‌کند
\\
تا که چه دید دوش او یا که چه کرد نوش او
&&
کز بن بامداد او ناله زار می‌کند
\\
گفت حبیب نادرست همچو الست و جنس او
&&
تا که به پاسخ بلی چرخ دوار می‌کند
\\
جمله مکونات را چرخ زنان چو چرخ دان
&&
جسم جهار می‌کند روح سرار می‌کند
\\
دور به گرد ساغرش هست نصیب اسعدی
&&
کو بحراک دست او دور سوار می‌کند
\\
ای همراه راه بین بر سر راه ماه بین
&&
لیک خمش سخن مگو گفت غبار می‌کند
\\
\end{longtable}
\end{center}
