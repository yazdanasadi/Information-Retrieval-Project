\begin{center}
\section*{غزل شماره ۱۹۸۰: یارکان رقصی کنید اندر غمم خوشتر از این}
\label{sec:1980}
\addcontentsline{toc}{section}{\nameref{sec:1980}}
\begin{longtable}{l p{0.5cm} r}
یارکان رقصی کنید اندر غمم خوشتر از این
&&
کره عشقم رمید و نی لگامستم نی زین
\\
پیش روی ماه ما مستانه یک رقصی کنید
&&
مطربا بهر خدا بر دف بزن ضرب حزین
\\
رقص کن در عشق جانم ای حریف مهربان
&&
مطربا دف را بکوب و نیست بختت غیر از این
\\
آن دف خوب تو این جا هست مقبول و صواب
&&
مطربا دف را بزن بس مر تو را طاعت همین
\\
مطربا این دف برای عشق شاه دلبر است
&&
مفخر تبریز جان جان جان‌ها شمس دین
\\
مطربا گفتی تو نام شمس دین و شمس دین
&&
درربودی از سرم یک بارگی تو عقل و دین
\\
چونک گفتی شمس دین زنهار تو فارغ مشو
&&
کفر باشد در طلب گر زانک گویی غیر این
\\
مطربا گشتی ملول از گفت من از گفت من
&&
همچنان خواهی مکن تو همچنین و همچنین
\\
\end{longtable}
\end{center}
