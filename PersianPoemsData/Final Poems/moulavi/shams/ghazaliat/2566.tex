\begin{center}
\section*{غزل شماره ۲۵۶۶: آورد طبیب جان یک طبله ره آوردی}
\label{sec:2566}
\addcontentsline{toc}{section}{\nameref{sec:2566}}
\begin{longtable}{l p{0.5cm} r}
آورد طبیب جان یک طبله ره آوردی
&&
گر پیر خرف باشی تو خوب و جوان گردی
\\
تن را بدهد هستی جان را بدهد مستی
&&
از دل ببرد سستی وز رخ ببرد زردی
\\
آن طبله عیسی بد میراث طبیبان شد
&&
تریاق در او یابی گر زهر اجل خوردی
\\
ای طالب آن طبله روی آر بدین قبله
&&
چون روی بدو آری مه روی جهان گردی
\\
حبیب است در او پنهان کان ناید در دندان
&&
نی تری و نی خشکی نی گرمی و نی سردی
\\
زان حب کم از حبه آیی بر آن قبه
&&
کان مسکن عیسی شد و آن حبه بدان خردی
\\
شد محرز و شد محرز از داد تو هر عاجز
&&
لاغر نشود هرگز آن را که تو پروردی
\\
گفتم به طبیب جان امروز هزاران سان
&&
صدق قدمی باشد چون تو قدم افشردی
\\
از جا نبرد چیزی آن را که تو جا دادی
&&
غم نسترد آن دل را کو را ز غم استردی
\\
خامش کن و دم درکش چون تجربه افتادت
&&
ترک گروان برگو تو زان گروان فردی
\\
\end{longtable}
\end{center}
