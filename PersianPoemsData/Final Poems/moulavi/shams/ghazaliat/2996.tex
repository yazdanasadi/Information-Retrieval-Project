\begin{center}
\section*{غزل شماره ۲۹۹۶: گر من ز دست بازی هر غم پژولمی}
\label{sec:2996}
\addcontentsline{toc}{section}{\nameref{sec:2996}}
\begin{longtable}{l p{0.5cm} r}
گر من ز دست بازی هر غم پژولمی
&&
زیرک نبودمی و خردمند گولمی
\\
گر آفتاب عشق نبودیم چون زحل
&&
گه در صعود انده و گه در نزولمی
\\
ور بوی مصر عشق قلاوز نیستی
&&
چون اهل تیه حرص گرفتار غولمی
\\
ور آفتاب جان‌ها خانه نشین بدی
&&
دربند فتح باب و خروج و دخولمی
\\
ور گلستان جان نبدی ممتحن نواز
&&
من چون صبا ز باغ وفا کی رسولمی
\\
عشق ار سماع باره و دف خواه نیستی
&&
من همچو نای و چنگ غزل کی شخولمی
\\
ساقیم گر ندادی داروی فربهی
&&
همچون لب زجاج و قدح در نحولمی
\\
گر سایه چمن نبدی و فروغ او
&&
من چون درخت بخت خسان بی‌اصولمی
\\
بر خاک من امانت حق گر نتافتی
&&
من چون مزاج خاک ظلوم و جهولمی
\\
از گور سوی جنت اگر راه نیستی
&&
در گور تن چرا خوش و باعرض و طولمی
\\
ور راه نیستی به یمین از سوی شمال
&&
کی چون چمن حریف جنوب و شمولمی
\\
گر گلشن کرم نبدی کی شکفتمی
&&
ور لطف و فضل حق نبدی من فضولمی
\\
بس کن ز آفتاب شنو مطلع قصص
&&
آن مطلع ار نبودی من در افولمی
\\
\end{longtable}
\end{center}
