\begin{center}
\section*{غزل شماره ۲۱۲۱: افندس مسین کاغا یومیندن}
\label{sec:2121}
\addcontentsline{toc}{section}{\nameref{sec:2121}}
\begin{longtable}{l p{0.5cm} r}
افندس مسین کاغا یومیندن
&&
کابیکینونین کالی زویمسن
\\
یتی بیرسس یتی قومسس
&&
بیمی تی پاتیس بیمی تی خسس
\\
هله دل من هله جان من
&&
هله این من هله آن من
\\
هله خان من هله مان من
&&
هله گنج من هله کان من
\\
هذا سیدی هذا سندی
&&
هذا سکنی هذا مددی
\\
هذا کنفی هذا عمدی
&&
هذا ازلی هذا ابدی
\\
یا من وجهه ضعف القمر
&&
یا من قده ضعف الشجر
\\
یا من زارنی وقت السحر
&&
یا من عشقه نور النظر
\\
گر تو بدوی ور تو بپری
&&
ز این دلیر جان خود جان نبری
\\
ور جان ببری از دست غمش
&&
از مرده خری والله بتری
\\
ایلا کالیمو ایلا شاهیمو
&&
خاراذی دیدش ذتمش انیمو
\\
یوذ پسه بنی پوپونی لالی
&&
میذن چاکوسش کالی تویالی
\\
از لیلی خود مجنون شده‌ام
&&
وز صد مجنون افزون شده‌ام
\\
وز خون جگر پرخون شده‌ام
&&
باری بنگر تا چون شده‌ام
\\
گر ز آنک مرا زین جان بکشی
&&
من غرقه شوم در عین خوشی
\\
دریا شود این دو چشم سرم
&&
گر گوش مرا زان سو بکشی
\\
یا منبسطا فی تربیتی
&&
یا مبتشرا فی تهنیتی
\\
ان کنت تری ان تقتلنی
&&
یا قاتلنا انت دیتی
\\
گر خویش تو بر مستی بزنی
&&
هستی تو بر هستی بزنی
\\
در حلقه ما بهر دل ما
&&
شکلی بکنی دستی بزنی
\\
صد گونه خوشی دیدم ز اشی
&&
گفتم که لبت گفتا نچشی
\\
بر گورم اگر آیی بنگر
&&
پرعشق بود چشمم ز کشی
\\
آن باغ بود نی نقش ثمر
&&
و آن گنج بود نی صورت زر
\\
شب عیش بود نی نقل و سمر
&&
لا تسالنی زان چیز دیگر
\\
\end{longtable}
\end{center}
