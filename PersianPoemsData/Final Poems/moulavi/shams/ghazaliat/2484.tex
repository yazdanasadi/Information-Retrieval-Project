\begin{center}
\section*{غزل شماره ۲۴۸۴: خواجه اگر تو همچو ما بیخود و شوخ و مستی}
\label{sec:2484}
\addcontentsline{toc}{section}{\nameref{sec:2484}}
\begin{longtable}{l p{0.5cm} r}
خواجه اگر تو همچو ما بیخود و شوخ و مستیی
&&
طوق قمر شکستیی فوق فلک نشستیی
\\
کی دم کس شنیدیی یا غم کس کشیدیی
&&
یا زر و سیم چیدیی گر تو فناپرستیی
\\
برجهیی به نیم شب با شه غیب خوش لقب
&&
ساغر باده طرب بر سر غم شکستیی
\\
ای تو مدد حیات را از جهت زکات را
&&
طره دلربات را بر دل من ببستیی
\\
عاشق مست از کجا شرم و شکست از کجا
&&
شنگ و وقیح بودیی گر گرو الستییی
\\
ور ز شراب دنگیی کی پی نام و ننگیی
&&
ور تو چو من نهنگیی کی به درون شستیی
\\
بازرسید مست ما داد قدح به دست ما
&&
گر دهدی به دست تو شاد و فراخ دستیی
\\
گر قدحش بدیدیی چون قدحش پریدیی
&&
وز کف جام بخش او از کف خود برستییی
\\
وز رخ یوسفانه‌اش عقل شدی ز خانه‌اش
&&
بخت شدی مساعدش ساعد خود نخستیی
\\
ور تو به گاه خاستی پس تو چه سست پاستی
&&
ور تو چو تیر راستی از پر کژ بجستیی
\\
خامش کن اگر تو را از خمشان خبر بدی
&&
وقت کلام لاییی وقت سکوت هستیی
\\
\end{longtable}
\end{center}
