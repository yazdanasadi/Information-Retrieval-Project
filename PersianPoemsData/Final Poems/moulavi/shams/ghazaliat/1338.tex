\begin{center}
\section*{غزل شماره ۱۳۳۸: بقا اندر بقا باشد طریق کم زنان ای دل}
\label{sec:1338}
\addcontentsline{toc}{section}{\nameref{sec:1338}}
\begin{longtable}{l p{0.5cm} r}
بقا اندر بقا باشد طریق کم زنان ای دل
&&
یقین اندر یقین آمد قلندر بی‌گمان ای دل
\\
به هر لحظه ز تدبیری به اقلیمی رود میری
&&
ز جاه و قوت پیری که باشد غیب دان ای دل
\\
کجا باشید صاحب دل دو روز اندر یکی منزل
&&
چو او را سیر شد حاصل از آن سوی جهان ای دل
\\
چو بگذشتی تو گردون را بدیدی بحر پرخون را
&&
ببین تو ماه بی‌چون را به شهر لامکان ای دل
\\
زبون آن کشش باشد کسی کان ره خوشش باشد
&&
روانش پرچشش باشد زهی جان و روان ای دل
\\
دهد نوری طبیعت را دهد دادی شریعت را
&&
چو بسپارد ودیعت را بدان سرحد جان ای دل
\\
شنودی شمس تبریزی گمان بردی از او چیزی
&&
یکی سری دل آمیزی تو را آمد عیان ای دل
\\
\end{longtable}
\end{center}
