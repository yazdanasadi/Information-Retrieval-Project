\begin{center}
\section*{غزل شماره ۱۸۹۳: صد گوش نوم باز شد از راز شنودن}
\label{sec:1893}
\addcontentsline{toc}{section}{\nameref{sec:1893}}
\begin{longtable}{l p{0.5cm} r}
صد گوش نوم باز شد از راز شنودن
&&
بی بوددهنده نتوان زادن و بودن
\\
استودن تو باد بهار آمد و من باغ
&&
خوش حامله می گردد اجزا ز ستودن
\\
بر همدگر افتادن مستان چه لطیف است
&&
وز همدگر آن جام وفا را بربودن
\\
ای آنک به عشق رخ تو واجب و حق است
&&
آیینه دل را ز خرافات زدودن
\\
آواز صفیر تو شنیدیم و فریضه است
&&
این هدهد جان را گره از پای گشودن
\\
تا چند در این ابر نهان باشد آن ماه
&&
جان‌ها به لب آمد هله وقت است نمودن
\\
ای گلشن روی تو ز دی ایمن و فارغ
&&
وی سنبل ابروی تو ایمن ز درودن
\\
ساقی چو تویی کفر بود بودن هشیار
&&
وان شب که تویی ماه حرام است غنودن
\\
چون آمد پیراهن خوش بوی تو یوسف
&&
بس بارد و سرد است کنون لخلخه سودن
\\
گفتم که ببوسم کف پای تو مرا گفت
&&
آن جسم بود کش بتوانند بسودن
\\
پس تا شه ما گوید کو راست مسلم
&&
پر کردن افهام و بر افهام فزودن
\\
\end{longtable}
\end{center}
