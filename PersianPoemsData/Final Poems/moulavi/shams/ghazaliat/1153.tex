\begin{center}
\section*{غزل شماره ۱۱۵۳: مه تو یار ندارد جز او تو یار مگیر}
\label{sec:1153}
\addcontentsline{toc}{section}{\nameref{sec:1153}}
\begin{longtable}{l p{0.5cm} r}
مه تو یار ندارد جز او تو یار مگیر
&&
رخش کنار ندارد از او کنار مگیر
\\
جهان شکارگهی دان ز هر طرف صیدی
&&
درآ چو شیر به جز شیر نر شکار مگیر
\\
هوای نفس مهارست و خلق چون شتران
&&
به غیر آن شتر مست را مهار مگیر
\\
وجود جمله غبارست تابش از مه ماست
&&
به ماه پشت میار و ره غبار مگیر
\\
بران ز پیش جهان را که مار گنج تواست
&&
تواش به حسن چو طاووس گیر و مار مگیر
\\
چو خلق بر کف دستت نهند چون سیماب
&&
ز عشق بر کف سیماب شو قرار مگیر
\\
به حس دست بدان ار چه چشم تو بستست
&&
ز گلشن ازلی گل بچین و خار مگیر
\\
به بوی آن گل بگشاد دیده یعقوب
&&
نسیم یوسف ما را ز کرته خوار مگیر
\\
کیست یوسف جان شاه شمس تبریزی
&&
به غیر حضرت او را تو اعتبار مگیر
\\
\end{longtable}
\end{center}
