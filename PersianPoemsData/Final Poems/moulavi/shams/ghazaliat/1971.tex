\begin{center}
\section*{غزل شماره ۱۹۷۱: از دخول هر غری افسرده‌ای در کار من}
\label{sec:1971}
\addcontentsline{toc}{section}{\nameref{sec:1971}}
\begin{longtable}{l p{0.5cm} r}
از دخول هر غری افسرده‌ای در کار من
&&
دور بادا وصف نفس آلودشان از یار من
\\
دررمید از ننگ ایشان و خبیثی‌ها و مکر
&&
از وظیفه مدح یارم این دل هشیار من
\\
خاک لعنت بر سر افسوس داری بدرگی
&&
کو کند از خاکساری درهم این هنجار من
\\
ای بریده دست دزدی کو بدزدد حکمتم
&&
و آنگهی دکان بگیرد بر سر بازار من
\\
شرم ناید مر ورا از روی من شرم از کجا
&&
ای حرامش باد هر تعلیم از اسرار من
\\
آن حرامی کز شقاوت تا رود گمره رود
&&
یا رب و ای ذوالجلال از حرمت دلدار من
\\
خاطرش از زیرکی یا آن ضمیرش از صفا
&&
بر فراز عرش رفتی یاد کردی یار من
\\
ای دل مسکین من از شرکت ناکس مرم
&&
زانک این سنت ز نااهلان بود ناچار من
\\
گر غران و ملحدان مر آب و نان را می خورند
&&
خوردن نان هیچ نگذارم پی این عار من
\\
صبر کن تا دررسد یک مژده‌ای زان مه لقا
&&
صبر کن تا رو نماید ابر گوهردار من
\\
صبر آن باشد دلا کز مدح آن بحر صفا
&&
رو نگردانی بلی و بشنوی گفتار من
\\
گیرم از لطف معانی رفت تمییز از جهان
&&
کی رود بوی دل و جان یم دربار من
\\
ور رود از دیگران بو از خدیوم کی رود
&&
از شهنشه شمس دین آن تا ابد تذکار من
\\
کز شراب جان من رویدهمی تبریز در
&&
لاله‌ها و گلبنان بر شیوه رخسار من
\\
ای خداوند این همه غیرت ز رشک سر توست
&&
ای هوای نازنین و شاه بی‌آزار من
\\
من قیاسی کرده‌ام رشک تو را در حق او
&&
لیک اندر رشک تو باطل بود پرگار من
\\
ای شهنشه شمس دین دانم که از چندین حجاب
&&
بشنود بیداریت این لابه‌های زار من
\\
بینش تو بیند این کز پرتو رشک خداست
&&
سنگ‌ها از هر طرف بر سینه سگسار من
\\
از کرم مپسند این را کاین سوار جان من
&&
جز به خرگاهت فرود آید از این رهوار من
\\
ور فروآید به جز خرگاه تو من از خدا
&&
من فنای محض خواهم ای خدایا یار من
\\
دوش دیدم کز هوس صد تخم مار اندر رگی
&&
درفکندم امتحان را تا چه گردد مار من
\\
دیدمش ماری شده او هر زمان در می فزود
&&
من پشیمان گشته‌ام زان صنعت و کردار من
\\
من پشیمان قصد او کردم و او از خشم خود
&&
بر زمین می زد همی دندان پرزهرار من
\\
کاین چنین شاگردکی بدفعل و بدرگ سر کشد
&&
ای خدا ضایع مکن این رنج و این ادرار من
\\
\end{longtable}
\end{center}
