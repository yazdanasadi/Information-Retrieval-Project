\begin{center}
\section*{غزل شماره ۹۶۳: دل من که باشد که تو را نباشد}
\label{sec:0963}
\addcontentsline{toc}{section}{\nameref{sec:0963}}
\begin{longtable}{l p{0.5cm} r}
دل من که باشد که تو را نباشد
&&
تن من کی باشد که فنا نباشد
\\
فلکش گرفتم چو مهش گرفتم
&&
چه زنند هر دو چو ضیا نباشد
\\
به درون جنت به میان نعمت
&&
چه شکنجه باشد چو لقا نباشد
\\
چو تو عذر خواهی گنه و جفا را
&&
چه کند جفاها که وفا نباشد
\\
چو خطا تو گیری به عتاب کردن
&&
چه کند دل و جان که خطا نباشد
\\
دو هزار دفتر چو به درس گویم
&&
نه فسرده باشم چو صفا نباشد
\\
سمنی نخندد شجری نرقصد
&&
چمنی نبوید چو صبا نباشد
\\
تو به فقر اگر چه که برهنه گردی
&&
چه غمست مه را که قبا نباشد
\\
چه عجب که جاهل ز دلست غافل
&&
ملکی و شاهی همه را نباشد
\\
همه مجرمان را کرمش بخواند
&&
چو به توبه آیند و دغا نباشد
\\
بگداز جان را مه آسمان را
&&
به خدا که چیزی چو خدا نباشد
\\
چه کنی سری را که فنا بکوبد
&&
چه کنی زری را که تو را نباشد
\\
همه روز گویی چو گلست یارم
&&
چه کنی گلی را که بقا نباشد
\\
مگریز ای جان ز بلای جانان
&&
که تو خام مانی چو بلا نباشد
\\
چه خوشست شب‌ها ز مهی که آن مه
&&
همه روی باشد که قفا نباشد
\\
چه خوشست شاهی که غلام او شد
&&
چه خوشست یاری که جدا نباشد
\\
تو خمش کن ای تن که دلم بگوید
&&
که حدیث دل را من و ما نباشد
\\
\end{longtable}
\end{center}
