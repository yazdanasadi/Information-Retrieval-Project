\begin{center}
\section*{غزل شماره ۸۷۸: صحرا خوشست لیک چو خورشید فر دهد}
\label{sec:0878}
\addcontentsline{toc}{section}{\nameref{sec:0878}}
\begin{longtable}{l p{0.5cm} r}
صحرا خوشست لیک چو خورشید فر دهد
&&
بستان خوشست لیک چو گلزار بر دهد
\\
خورشید دیگریست که فرمان و حکم او
&&
خورشید را برای مصالح سفر دهد
\\
بوسه به او رسد که رخش همچو زر بود
&&
او را نمی‌رسد که رود مال و زر دهد
\\
بنگر به طوطیان که پر و بال می‌زنند
&&
سوی شکرلبی که به ایشان شکر دهد
\\
هر کس شکرلبی بگزیده‌ست در جهان
&&
ما را شکرلبیست که چیزی دگر دهد
\\
ما را شکرلبیست شکرها گدای اوست
&&
ما را شهنشهیست که ملک و ظفر دهد
\\
همت بلند دار اگر شاه زاده‌ای
&&
قانع مشو ز شاه که تاج و کمر دهد
\\
برکن تو جامه‌ها و در آب حیات رو
&&
تا پاره‌های خاک تو لعل و گهر دهد
\\
بگریز سوی عشق و بپرهیز از آن بتی
&&
کو دلبری نماید و خون جگر دهد
\\
در چشم من نیاید خوبی هیچ خوب
&&
نقاش جسم جان را غیبی صور دهد
\\
کی آب شور نوشد با مرغ‌های کور
&&
آن مرغ را که عقل ز کوثر خبر دهد
\\
خود پر کند دو دیده ما را به حسن خویش
&&
گر ماه آن ببیند در حال سر دهد
\\
در دیده گدای تو آید نگار خاک
&&
حاشا ز دیده‌ای که خدایش نظر دهد
\\
خامش ز حرف گفتن تا بوک عقل کل
&&
ما را ز عقل جزوی راه و عبر دهد
\\
\end{longtable}
\end{center}
