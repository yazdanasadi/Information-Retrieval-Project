\begin{center}
\section*{غزل شماره ۷۵۹: دل من رای تو دارد سر سودای تو دارد}
\label{sec:0759}
\addcontentsline{toc}{section}{\nameref{sec:0759}}
\begin{longtable}{l p{0.5cm} r}
دل من رای تو دارد سر سودای تو دارد
&&
رخ فرسوده زردم غم صفرای تو دارد
\\
سر من مست جمالت دل من دام خیالت
&&
گهر دیده نثار کف دریای تو دارد
\\
ز تو هر هدیه که بردم به خیال تو سپردم
&&
که خیال شکرینت فر و سیمای تو دارد
\\
غلطم گر چه خیالت به خیالات نماند
&&
همه خوبی و ملاحت ز عطاهای تو دارد
\\
گل صدبرگ به پیش تو فروریخت ز خجلت
&&
که گمان برد که او هم رخ رعنای تو دارد
\\
سر خود پیش فکنده چو گنه کار تو عرعر
&&
که خطا کرد و گمان برد که بالای تو دارد
\\
جگر و جان عزیزان چو رخ زهره فروزان
&&
همه چون ماه گدازان که تمنای تو دارد
\\
دل من تابه حلوا ز بر آتش سودا
&&
اگر از شعله بسوزد نه که حلوای تو دارد
\\
هله چون دوست به دستی همه جا جای نشستی
&&
خنک آن بی‌خبری کو خبر از جای تو دارد
\\
اگرم در نگشایی ز ره بام درآیم
&&
که زهی جان لطیفی که تماشای تو دارد
\\
به دو صد بام برآیم به دو صد دام درآیم
&&
چه کنم آهوی جانم سر صحرای تو دارد
\\
خمش ای عاشق مجنون بمگو شعر و بخور خون
&&
که جهان ذره به ذره غم غوغای تو دارد
\\
سوی تبریز شو ای دل بر شمس الحق مفضل
&&
چو خیالش به تو آید که تقاضای تو دارد
\\
\end{longtable}
\end{center}
