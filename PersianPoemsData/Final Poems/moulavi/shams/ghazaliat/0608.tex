\begin{center}
\section*{غزل شماره ۶۰۸: ایمان بر کفر تو ای شاه چه کس باشد}
\label{sec:0608}
\addcontentsline{toc}{section}{\nameref{sec:0608}}
\begin{longtable}{l p{0.5cm} r}
ایمان بر کفر تو ای شاه چه کس باشد
&&
سیمرغ فلک پیما پیش تو مگس باشد
\\
آب حیوان ایمان خاک سیهی کفران
&&
بر آتش تو هر دو ماننده خس باشد
\\
جان را صفت ایمان شد وین جان به نفس جان شد
&&
دل غرقه عمان شد چه جای نفس باشد
\\
شب کفر و چراغ ایمان خورشید چو شد رخشان
&&
با کفر بگفت ایمان رفتیم که بس باشد
\\
ایمان فرسی دین را مر نفس چو فرزین را
&&
وان شاه نوآیین را چه جای فرس باشد
\\
ایمان گودت پیش آ وان کفر گود پس رو
&&
چون شمع تنت جان شد نی پیش و نی پس باشد
\\
شمس الحق تبریزی رانی تو چنان بالا
&&
تا جز من پابرجا خود دست مرس باشد
\\
\end{longtable}
\end{center}
