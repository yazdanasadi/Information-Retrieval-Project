\begin{center}
\section*{غزل شماره ۳۰۳: مجلس خوش کن از آن دو پاره چوب}
\label{sec:0303}
\addcontentsline{toc}{section}{\nameref{sec:0303}}
\begin{longtable}{l p{0.5cm} r}
مجلس خوش کن از آن دو پاره چوب
&&
عود را درسوز و بربط را بکوب
\\
این ننالد تا نکوبی بر رگش
&&
وان دگر در نفی و در سوزست خوب
\\
مجلسی پرگرد بر خاشاک فکر
&&
خیز ای فراش فرش جان بروب
\\
تا نسوزی بوی ندهد آن بخور
&&
تا نکوبی نفع ندهد این حبوب
\\
نیر اعظم بدان شد آفتاب
&&
کو در آتش خانه دارد بی‌لغوب
\\
ماه از آن پیک و محاسب می‌شود
&&
کو نیاساید ز سیران و رکوب
\\
عود خلقانند این پیغامبران
&&
تا رسدشان بوی علام الغیوب
\\
گر به بو قانع نه‌ای تو هم بسوز
&&
تا که معدن گردی ای کان عیوب
\\
چون بسوزی پر شود چرخ از بخور
&&
چون بسوزد دل رسد وحی القلوب
\\
حد ندارد این سخن کوتاه کن
&&
گر چه جان گلستان آمد جنوب
\\
صاحب العودین لا تهملهما
&&
حرقن ذا حرکن ذا للکروب
\\
من یلج بین السکاری لا یفق
&&
من یذق من راح روح لا یتوب
\\
اغتنم بالراح عجل و استعد
&&
من خمار دونه شق الجیوب
\\
این تنجو ان سلطان الهوی
&&
جاذب العشاق جبار طلوب
\\
\end{longtable}
\end{center}
