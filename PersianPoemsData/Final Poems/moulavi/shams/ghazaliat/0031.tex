\begin{center}
\section*{غزل شماره ۳۱: بادا مبارک در جهان سور و عروسی‌های ما}
\label{sec:0031}
\addcontentsline{toc}{section}{\nameref{sec:0031}}
\begin{longtable}{l p{0.5cm} r}
بادا مبارک در جهان سور و عروسی‌های ما
&&
سور و عروسی را خدا ببرید بر بالای ما
\\
زهره قرین شد با قمر طوطی قرین شد با شکر
&&
هر شب عروسیی دگر از شاه خوش سیمای ما
\\
ان القلوب فرجت ان النفوس زوجت
&&
ان الهموم اخرجت در دولت مولای ما
\\
بسم الله امشب بر نوی سوی عروسی می‌روی
&&
داماد خوبان می‌شوی ای خوب شهرآرای ما
\\
خوش می‌روی در کوی ما خوش می‌خرامی سوی ما
&&
خوش می‌جهی در جوی ما ای جوی و ای جویای ما
\\
خوش می‌روی بر رای ما خوش می‌گشایی پای ما
&&
خوش می‌بری کف‌های ما ای یوسف زیبای ما
\\
از تو جفا کردن روا وز ما وفا جستن خطا
&&
پای تصرف را بنه بر جان خون پالای ما
\\
ای جان جان جان را بکش تا حضرت جانان ما
&&
وین استخوان را هم بکش هدیه بر عنقای ما
\\
رقصی کنید ای عارفان چرخی زنید ای منصفان
&&
در دولت شاه جهان آن شاه جان افزای ما
\\
در گردن افکنده دهل در گردک نسرین و گل
&&
کامشب بود دف و دهل نیکوترین کالای ما
\\
خاموش کامشب زهره شد ساقی به پیمانه و به مد
&&
بگرفته ساغر می‌کشد حمرای ما حمرای ما
\\
والله که این دم صوفیان بستند از شادی میان
&&
در غیب پیش غیبدان از شوق استسقای ما
\\
قومی چو دریا کف زنان چون موج‌ها سجده کنان
&&
قومی مبارز چون سنان خون خوار چون اجزای ما
\\
خاموش کامشب مطبخی شاهست از فرخ رخی
&&
این نادره که می‌پزد حلوای ما حلوای ما
\\
\end{longtable}
\end{center}
