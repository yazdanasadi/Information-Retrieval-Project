\begin{center}
\section*{غزل شماره ۲۵۵۷: هر آن چشمی که گریان است در عشق دلارامی}
\label{sec:2557}
\addcontentsline{toc}{section}{\nameref{sec:2557}}
\begin{longtable}{l p{0.5cm} r}
هر آن چشمی که گریان است در عشق دلارامی
&&
بشارت آیدش روزی ز وصل او به پیغامی
\\
هر آن چشم سپیدی کو سیه کرده‌ست تن جامه
&&
سیاهش شد سپید آخر سپیدش شد سیه فامی
\\
چو گریان بود آن یعقوب کنعان از پی یوسف
&&
بشارت آمدش ناگه از آن خوش روی خوش نامی
\\
مثال نردبان باشد به نالیدن به عشق اندر
&&
چو او بر نردبان کوشد رسد ناگاه بر بامی
\\
حریف عشق پیش آید چو بیند مر تو را بیخود
&&
کبابی از جگر در کف ز خون دل یکی جامی
\\
که آب لطف آن دلبر گرفته قاف تا قاف است
&&
از آن است آتش هجران که تا پخته شود خامی
\\
برای امتحان مرغ جان عاشق وحشی
&&
بلا چون ضربت دامی و زلف یار چون دامی
\\
که تا زین دام و زین ضربت کشاکش یابد این وحشی
&&
نماند ناز و تندی او شود همراز و هم رامی
\\
چنان چون میوه‌های خام از آن پخته شود شیرین
&&
که گاهش تاب خورشید است و گاهش طره شامی
\\
ز رنج عام و لطف خاص حکمت‌ها شود پیدا
&&
که تا صافی شود دردی که تا خاصه شود عامی
\\
گهی از خوف محرومی و هجران ابد سوزی
&&
گهی اندر امید وصل یکتا زفت انعامی
\\
خصوصا درد این مسکین که عالم سوز طوفان است
&&
زهی تلخی و ناکامی که شیرین است از او کامی
\\
به هر گامی اگر صد تیر آید از هوای او
&&
نگردم از هوای او نگردانم یکی گامی
\\
منم در وام عشق شاه تا گردن بحمدالله
&&
مبارک صاحب وامی مبارک کردن وامی
\\
زهی دریای لطف حق زهی خورشید ربانی
&&
به هر صد قرن نبود این چه جای سال و ایامی
\\
ز مخدومی شمس الدین تبریزی بیابد جان
&&
خلاصه نور ایمانی صفای جان اسلامی
\\
چه جای نور اسلام است که نورانی و روحانی
&&
شود واله اگر پیدا شود از دفترش لامی
\\
\end{longtable}
\end{center}
