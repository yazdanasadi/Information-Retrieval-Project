\begin{center}
\section*{غزل شماره ۹۹: دلارام نهان گشته ز غوغا}
\label{sec:0099}
\addcontentsline{toc}{section}{\nameref{sec:0099}}
\begin{longtable}{l p{0.5cm} r}
دلارام نهان گشته ز غوغا
&&
همه رفتند و خلوت شد برون آ
\\
برآور بنده را از غرقه خون
&&
فرح ده روی زردم را ز صفرا
\\
کنار خویش دریا کردم از اشک
&&
تماشا چون نیایی سوی دریا
\\
چو تو در آینه دیدی رخ خود
&&
از آن خوشتر کجا باشد تماشا
\\
غلط کردم در آیینه نگنجی
&&
ز نورت می‌شود لا کل اشیاء
\\
رهید آن آینه از رنج صیقل
&&
ز رویت می‌شود پاک و مصفا
\\
تو پنهانی چو عقل و جمله از تست
&&
خرابی‌ها عمارت‌ها به هر جا
\\
هر آنک پهلوی تو خانه گیرد
&&
به پیشش پست شد بام ثریا
\\
چه باشد حال تن کز جان جدا شد
&&
چه عذر آورد کسی کز تست عذرا
\\
چه یاری یابد از یاران همدل
&&
کسی کز جان شیرین گشت تنها
\\
به از صبحی تو خلقان را به هر روز
&&
به از خوابی ضعیفان را به شب‌ها
\\
تو را در جان بدیدم بازرستم
&&
چو گمراهان نگویم زیر و بالا
\\
چو در عالم زدی تو آتش عشق
&&
جهان گشتست همچون دیگ حلوا
\\
همه حسن از تو باید ماه و خورشید
&&
همه مغز از تو باید جدی و جوزا
\\
بدان شد شب شفا و راحت خلق
&&
که سودای توش بخشید سودا
\\
چو پروانه‌ست خلق و روز چون شمع
&&
که از زیب خودش کردی تو زیبا
\\
هر آن پروانه که شمع تو را دید
&&
شبش خوشتر ز روز آمد به سیما
\\
همی‌پرد به گرد شمع حسنت
&&
به روز و شب ندارد هیچ پروا
\\
نمی‌یارم بیان کردن از این بیش
&&
بگفتم این قدر باقی تو فرما
\\
بگو باقی تو شمس الدین تبریز
&&
که به گوید حدیث قاف عنقا
\\
\end{longtable}
\end{center}
