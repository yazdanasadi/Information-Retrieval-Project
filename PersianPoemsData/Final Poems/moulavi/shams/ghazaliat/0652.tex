\begin{center}
\section*{غزل شماره ۶۵۲: تدبیر کند بنده و تقدیر نداند}
\label{sec:0652}
\addcontentsline{toc}{section}{\nameref{sec:0652}}
\begin{longtable}{l p{0.5cm} r}
تدبیر کند بنده و تقدیر نداند
&&
تدبیر به تقدیر خداوند نماند
\\
بنده چو بیندیشد پیداست چه بیند
&&
حیله بکند لیک خدایی نتواند
\\
گامی دو چنان آید کو راست نهادست
&&
وان گاه که داند که کجاهاش کشاند
\\
استیزه مکن مملکت عشق طلب کن
&&
کاین مملکتت از ملک الموت رهاند
\\
باری تو بهل کام خود و نور خرد گیر
&&
کاین کام تو را زود به ناکام رساند
\\
اشکاری شه باش و مجو هیچ شکاری
&&
کاشکار تو را باز اجل بازستاند
\\
چون باز شهی رو به سوی طبله بازش
&&
کان طبله تو را نوش دهد طبل نخواند
\\
از شاه وفادارتر امروز کسی نیست
&&
خر جانب او ران که تو را هیچ نراند
\\
زندانی مرگند همه خلق یقین دان
&&
محبوس تو را از تک زندان نرهاند
\\
دانی که در این کوی رضا بانگ سگان چیست
&&
تا هر که مخنث بود آتش برماند
\\
حاشا ز سواری که بود عاشق این راه
&&
که بانگ سگ کوی دلش را بطپاند
\\
\end{longtable}
\end{center}
