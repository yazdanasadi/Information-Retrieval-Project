\begin{center}
\section*{غزل شماره ۲۵۷۳: در پرده خاک ای جان عیشی است به پنهانی}
\label{sec:2573}
\addcontentsline{toc}{section}{\nameref{sec:2573}}
\begin{longtable}{l p{0.5cm} r}
در پرده خاک ای جان عیشی است به پنهانی
&&
و اندر تتق غیبی صد یوسف کنعانی
\\
این صورت تن رفته و آن صورت جا مانده
&&
ای صورت جان باقی وی صورت تن فانی
\\
گر چاشنیی خواهی هر شب بنگر خود را
&&
تن مرده و جان پران در روضه رضوانی
\\
ای عشق که آن داری یا رب چه جهان داری
&&
چندان صفتت کردم والله که دو چندانی
\\
المؤمن حلوی و العاش علوی
&&
با تو چه زبان گویم ای جان که نمی‌دانی
\\
چندان بدوان لنگان کاین پای فروماند
&&
وآنگه رسد از سلطان صد مرکب میدانی
\\
می مرد یکی عاشق می‌گفت یکی او را
&&
در حالت جان کندن چون است که خندانی
\\
گفتا چو بپردازم من جمله دهان گردم
&&
صدمرده همی‌خندم بی‌خنده دندانی
\\
زیرا که یکی نیمم نی بود شکر گشتم
&&
نیم دگرم دارد عزم شکرافشانی
\\
هر کو نمرد خندان تو شمع مخوان او را
&&
بو بیش دهد عنبر در وقت پریشانی
\\
ای شهره نوای تو جان است سزای تو
&&
تو مطرب جانانی چون در طمع نانی
\\
کس کیسه میفشان گو کس خرقه میفکن گو
&&
اومید کی ضایع شد از کیسه ربانی
\\
از کیسه حق گردون صد نور و ضیا ریزد
&&
دریا ز عطای حق دارد گهرافشانی
\\
نان ریزه سفره‌ست این کز چرخ همی‌ریزد
&&
بگذر ز فلک بررو گر درخور آن خوانی
\\
گر خسته شود کفت کفی دگرت بخشد
&&
ور خسته شود حلقت در حلقه سلطانی
\\
برگو غزلی برگو پامزد خود از حق جو
&&
بر سوخته زن آبی چون چشمه حیوانی
\\
\end{longtable}
\end{center}
