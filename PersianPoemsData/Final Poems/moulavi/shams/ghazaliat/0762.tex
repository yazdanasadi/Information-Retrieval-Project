\begin{center}
\section*{غزل شماره ۷۶۲: بدرد مرده کفن را به سر گور برآید}
\label{sec:0762}
\addcontentsline{toc}{section}{\nameref{sec:0762}}
\begin{longtable}{l p{0.5cm} r}
بدرد مرده کفن را به سر گور برآید
&&
اگر آن مرده ما را ز بت من خبر آید
\\
چه کند مرده و زنده چو از او یابد چیزی
&&
که اگر کوه ببیند بجهد پیشتر آید
\\
ز ملامت نگریزم که ملامت ز تو آید
&&
که ز تلخی تو جان را همه طعم شکر آید
\\
بخور آن را که رسیدت مهل از بهر ذخیره
&&
که تو بر جوی روانی چو بخوردی دگر آید
\\
بنگر صنعت خوبش بشنو وحی قلوبش
&&
همگی نور نظر شو همه ذوق از نظر آید
\\
مبر امید که عمرم بشد و یار نیامد
&&
بگه آید وی و بی‌گه نه همه در سحر آید
\\
تو مراقب شو و آگه گه و بی‌گاه که ناگه
&&
مثل کحل عزیزی شه ما در بصر آید
\\
چو در این چشم درآید شود این چشم چو دریا
&&
چو به دریا نگرد از همه آبش گهر آید
\\
نه چنان گوهر مرده که نداند گهر خود
&&
همه گویا همه جویا همگی جانور آید
\\
تو چه دانی تو چه دانی که چه کانی و چه جانی
&&
که خدا داند و بیند هنری کز بشر آید
\\
تو سخن گفتن بی‌لب هله خو کن چو ترازو
&&
که نماند لب و دندان چو ز دنیا گذر آید
\\
\end{longtable}
\end{center}
