\begin{center}
\section*{غزل شماره ۲۴۹۷: صبح چو آفتاب زد رایت روشناییی}
\label{sec:2497}
\addcontentsline{toc}{section}{\nameref{sec:2497}}
\begin{longtable}{l p{0.5cm} r}
صبح چو آفتاب زد رایت روشناییی
&&
لعل و عقیق می‌کند در دل کان گداییی
\\
گر ز فلک نهان بود در ظلمات کان بود
&&
گوهر سنگ را بود با فلک آشناییی
\\
نور ز شرق می‌زند کوه شکاف می‌کند
&&
در دل سنگ می‌نهد شعشعه عطاییی
\\
در پی هر منوری هست یقین منوری
&&
در پی هر زمینیی مرتقب سماییی
\\
صورت بت نمی‌شود بی‌دل و دست آزری
&&
آزر بتگری کجا باشد بی‌خداییی
\\
گفت پیمبر به حق کآدمی است کان زر
&&
فرق میان کان و کان هست به زرنماییی
\\
\end{longtable}
\end{center}
