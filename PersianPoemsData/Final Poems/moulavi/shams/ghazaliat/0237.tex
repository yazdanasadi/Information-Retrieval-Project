\begin{center}
\section*{غزل شماره ۲۳۷: یار ما دلدار ما عالم اسرار ما}
\label{sec:0237}
\addcontentsline{toc}{section}{\nameref{sec:0237}}
\begin{longtable}{l p{0.5cm} r}
یار ما دلدار ما عالم اسرار ما
&&
یوسف دیدار ما رونق بازار ما
\\
بر دم امسال ما عاشق آمد پار ما
&&
مفلسانیم و تویی گنج ما دینار ما
\\
کاهلانیم و تویی حج ما پیکار ما
&&
خفتگانیم و تویی دولت بیدار ما
\\
خستگانیم و تویی مرهم بیمار ما
&&
ما خرابیم و تویی از کرم معمار ما
\\
دوش گفتم عشق را ای شه عیار ما
&&
سر مکش منکر مشو برده‌ای دستار ما
\\
پس جوابم داد او کز توست این کار ما
&&
هر چه گویی وادهد چون صدا کهسار ما
\\
گفتمش خود ما کهیم این صدا گفتار ما
&&
زانک که را اختیار نبود ای مختار ما
\\
گفت بشنو اولا شمه‌ای ز اسرار ما
&&
هر ستوری لاغری کی کشاند بار ما
\\
گفتمش از ما ببر زحمت اخبار ما
&&
بلبلی مستی بکن هم ز بوتیمار ما
\\
هستی تو فخر ما هستی ما عار ما
&&
احمد و صدیق بین در دل چون غار ما
\\
می ننوشد هر میی مست دردی خوار ما
&&
خور ز دست شه خورد مرغ خوش منقار ما
\\
چون بخسپد در لحد قالب مردار ما
&&
رسته گردد زین قفس طوطی طیار ما
\\
خود شناسد جای خود مرغ زیرکسار ما
&&
بعد ما پیدا کنی در زمین آثار ما
\\
گر به بستان بی‌توایم خار شد گلزار ما
&&
ور به زندان با توایم گل بروید خار ما
\\
گر در آتش با توایم نور گردد نار ما
&&
ور به جنت بی‌توایم نار شد انوار ما
\\
از تو شد باز سپید زاغ ما و سار ما
&&
بس کن و دیگر مگو کاین بود گفتار ما
\\
\end{longtable}
\end{center}
