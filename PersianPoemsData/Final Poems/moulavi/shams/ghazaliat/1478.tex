\begin{center}
\section*{غزل شماره ۱۴۷۸: المنه لله که ز پیکار رهیدیم}
\label{sec:1478}
\addcontentsline{toc}{section}{\nameref{sec:1478}}
\begin{longtable}{l p{0.5cm} r}
المنه لله که ز پیکار رهیدیم
&&
زین وادی خم در خم پرخار رهیدیم
\\
زین جان پر از وهم کژاندیشه گذشتیم
&&
زین چرخ پر از مکر جگرخوار رهیدیم
\\
دکان حریصان به دغل رخت همه برد
&&
دکان بشکستیم و از آن کار رهیدیم
\\
در سایه آن گلشن اقبال بخفتیم
&&
وز غرقه آن قلزم زخار رهیدیم
\\
بی‌اسب همه فارس و بی‌می همه مستیم
&&
از ساغر و از منت خمار رهیدیم
\\
ما توبه شکستیم و ببستیم دو صد بار
&&
دیدیم مه توبه به یک بار رهیدیم
\\
زان عیسی عشاق و ز افسون مسیحش
&&
از علت و قاروره و بیمار رهیدیم
\\
چون شاهد مشهور بیاراست جهان را
&&
از شاهد و از برده بلغار رهیدیم
\\
ای سال چه سالی تو که از طالع خوبت
&&
ز افسانه پار و غم پیرار رهیدیم
\\
در عشق ز سه روزه وز چله گذشتیم
&&
مذکور چو پیش آمد از اذکار رهیدیم
\\
خاموش کز این عشق و از این علم لدنیش
&&
از مدرسه و کاغذ و تکرار رهیدیم
\\
خاموش کز این کان و از این گنج الهی
&&
از مکسبه و کیسه و بازار رهیدیم
\\
هین ختم بر این کن که چو خورشید برآمد
&&
از حارس و از دزد و شب تار رهیدیم
\\
\end{longtable}
\end{center}
