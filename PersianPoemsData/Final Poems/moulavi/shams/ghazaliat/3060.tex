\begin{center}
\section*{غزل شماره ۳۰۶۰: نهان شدند معانی ز یار بی‌معنی}
\label{sec:3060}
\addcontentsline{toc}{section}{\nameref{sec:3060}}
\begin{longtable}{l p{0.5cm} r}
نهان شدند معانی ز یار بی‌معنی
&&
کجا روم که نروید به پیش من دیوی
\\
کی دید خربزه زاری لطیف بی‌سرخر
&&
که من بجستم عمری ندیده‌ام باری
\\
بگو به نفس مصور مکن چنین صورت
&&
از این سپس متراش این چنین بت ای مانی
\\
اگر نقوش مصور همه از این جنس اند
&&
مخواه دیده بینا خنک تن اعمی
\\
دو گونه رنج و عذابست جان مجنون را
&&
بلای صحبت لولی و فرقت لیلی
\\
ورای پرده یکی دیو زشت سر برکرد
&&
بگفتمش که تویی مرگ و جسک گفت آری
\\
بگفتم او را صدق که من ندیدستم
&&
ز تو غلیظتر اندر سپاه بویحیی
\\
بگفتمش که دلم بارگاه لطف خداست
&&
چه کار دارد قهر خدا در این مأوی
\\
به روز حشر که عریان کنند زشتان را
&&
رمند جمله زشتان ز زشتی دنیی
\\
در این بدم که به ناگاه او مبدل شد
&&
مثال صورت حوری به قدرت مولی
\\
رخی لطیف و منزه ز رنگ و گلگونه
&&
کفی ظریف و مبرا ز حیله حنی
\\
چنانک خار سیه را بهارگه بینی
&&
کند میان سمن زار گلرخی دعوی
\\
زهی بدیع خدایی که کرد شب را روز
&&
ز دوزخی به درآورد جنت و طوبی
\\
کسی که دیده به صنع لطیف او خو داد
&&
نترسد ار چه فتد در دهان صد افعی
\\
به افعیی بنگر کو هزار افعی خورد
&&
شد او عصا و مطیعی به قبضه موسی
\\
از آن عصا نشود مر تو را که فرعونی
&&
چو مهره دزدی زان رو به افعیی اولی
\\
خمش که رنج برای کریم گنج شود
&&
برایمؤمنروضه‌ست نار در عقبی
\\
\end{longtable}
\end{center}
