\begin{center}
\section*{غزل شماره ۱۳۳۱: هر کی در او نیست از این عشق رنگ}
\label{sec:1331}
\addcontentsline{toc}{section}{\nameref{sec:1331}}
\begin{longtable}{l p{0.5cm} r}
هر کی در او نیست از این عشق رنگ
&&
نزد خدا نیست به جز چوب و سنگ
\\
عشق برآورد ز هر سنگ آب
&&
عشق تراشید ز آیینه زنگ
\\
کفر به جنگ آمد و ایمان به صلح
&&
عشق بزد آتش در صلح و جنگ
\\
عشق گشاید دهن از بحر دل
&&
هر دو جهان را بخورد چون نهنگ
\\
عشق چو شیرست نه مکر و نه ریو
&&
نیست گهی روبه و گاهی پلنگ
\\
چونک مدد بر مدد آید ز عشق
&&
جان برهد از تن تاریک و تنگ
\\
عشق ز آغاز همه حیرتست
&&
عقل در او خیره و جان گشته دنگ
\\
در تبریزست دلم ای صبا
&&
خدمت ما را برسان بی‌درنگ
\\
\end{longtable}
\end{center}
