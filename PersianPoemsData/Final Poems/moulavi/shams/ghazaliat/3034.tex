\begin{center}
\section*{غزل شماره ۳۰۳۴: می‌رسد ای جان باد بهاری}
\label{sec:3034}
\addcontentsline{toc}{section}{\nameref{sec:3034}}
\begin{longtable}{l p{0.5cm} r}
می‌رسد ای جان باد بهاری
&&
تا سوی گلشن دست برآری
\\
سبزه و سوسن لاله و سنبل
&&
گفت بروید هر چه بکاری
\\
غنچه و گل‌ها مغفرت آمد
&&
تا ننماید زشتی خاری
\\
رفعت آمد سرو سهی را
&&
یافت عزیزی از پس خواری
\\
روح درآید در همه گلشن
&&
کب نماید روح سپاری
\\
خوبی گلشن ز آب فزاید
&&
سخت مبارک آمد یاری
\\
کرد پیامی برگ به میوه
&&
زود بیایی گوش نخاری
\\
شاه ثمارست آن عنب خوش
&&
زانک درختش داشت نزاری
\\
در دی شهوت چند بماند
&&
باغ دل ما حبس و حصاری
\\
راه ز دل جو ماه ز جان جو
&&
خاک چه دارد غیر غباری
\\
خیز بشو رو لیک به آبی
&&
کرد گل را خوب عذاری
\\
گفت به ریحان شاخ شکوفه
&&
در ره ما نه هر چه که داری
\\
بلبل مرغان گفت به بستان
&&
دام شما راییم شکاری
\\
لابه کند گل رحمت حق را
&&
بر ما دی را برنگماری
\\
گوید یزدان شیره ز میوه
&&
کی به کف آید تا نفشاری
\\
غم مخور از دی وز غز و غارت
&&
وز در من بین کارگزاری
\\
شکر و ستایش ذوق و فزایش
&&
رو ننماید جز که به زاری
\\
عمر ببخشم بی‌ز شمارت
&&
گر بستانم عمر شماری
\\
باده ببخشم بی‌ز خمارت
&&
گر بستانم خمر خماری
\\
چند نگاران دارد دانش
&&
کاغذها را چند نگاری
\\
از تو سیه شد چهره کاغذ
&&
چونک بخوانی خط نهاری
\\
دود رها کن نور نگر تو
&&
از مه جانان در شب تاری
\\
بس کن و بس کن ز اسب فرود آ
&&
تا که کند او شاه سواری
\\
\end{longtable}
\end{center}
