\begin{center}
\section*{غزل شماره ۱۷۴۰: خوشی خوشی تو ولی من هزار چندانم}
\label{sec:1740}
\addcontentsline{toc}{section}{\nameref{sec:1740}}
\begin{longtable}{l p{0.5cm} r}
خوشی خوشی تو ولی من هزار چندانم
&&
به خواب دوش که را دیده‌ام نمی‌دانم
\\
ز خوشدلی و طرب در جهان نمی‌گنجم
&&
ولی ز چشم جهان همچو روح پنهانم
\\
درخت اگر نبدی پا به گل مرا جستی
&&
کز این شکوفه و گل حسرت گلستانم
\\
همیشه دامن شادی کشیدمی سوی خویش
&&
کشد کنون کف شادی به خویش دامانم
\\
ز بامداد کسی غلملیج می کندم
&&
گزاف نیست که من ناشتاب خندانم
\\
ترانه‌ها ز من آموزد این نفس زهره
&&
هزار زهره غلام دماغ سکرانم
\\
شکرلبی لب ما را به گاه شیرین کرد
&&
که غرقه گشت شکر اندر آب دندانم
\\
صلا که قامت چون سرو او صلا درداد
&&
که من نماز شما را لطیف ارکانم
\\
صلا که فاتحه قفل‌های بسته منم
&&
بدان چو فاتحه تان در نماز می خوانم
\\
به دار ملک ملاحت لبش چو غماز است
&&
که بنگرید نصیب مرا که دربانم
\\
چنانک پیش جنونم عقول حیرانند
&&
من از فسردگی این عقول حیرانم
\\
فسرده ماند یخی که به زیر سایه بود
&&
ندید شعشعه آفتاب رخشانم
\\
تبسم خوش خورشید هر یخی که بدید
&&
سبال مالد و گوید که آب حیوانم
\\
بیار ناطق کلی بگو تو باقی را
&&
ز گفتنم برهان من خموش برهانم
\\
\end{longtable}
\end{center}
