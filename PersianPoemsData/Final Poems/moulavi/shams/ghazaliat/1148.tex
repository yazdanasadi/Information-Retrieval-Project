\begin{center}
\section*{غزل شماره ۱۱۴۸: کسی بگفت ز ما یا از اوست نیکی و شر}
\label{sec:1148}
\addcontentsline{toc}{section}{\nameref{sec:1148}}
\begin{longtable}{l p{0.5cm} r}
کسی بگفت ز ما یا از اوست نیکی و شر
&&
هنوز خواجه در اینست ریش خواجه نگر
\\
عجب که خواجه به رنگی که طفل بود بماند
&&
که ریش خواجه سیه بود و گشت رنگ دگر
\\
بگویمت که چرا خواجه زیر و بالا گفت
&&
بدان سبب که نگشتست خواجه زیر و زبر
\\
به چار پا و دو پا خواجه گرد عالم گشت
&&
ولیک هیچ نرفتست قعر بحر به سر
\\
گمان خواجه چنانست که خواجه بهتر گشت
&&
ولیک هست چو بیمار دق واپستر
\\
به حجت و به لجاج و ستیزه افزون گشت
&&
ز جان و حجت ذوقش نبود هیچ خبر
\\
طریق بحث لجاجست و اعتراض و دلیل
&&
طریق دل همه دیده‌ست و ذوق و شهد و شکر
\\
\end{longtable}
\end{center}
