\begin{center}
\section*{غزل شماره ۹۲۱: سخن که خیزد از جان ز جان حجاب کند}
\label{sec:0921}
\addcontentsline{toc}{section}{\nameref{sec:0921}}
\begin{longtable}{l p{0.5cm} r}
سخن که خیزد از جان ز جان حجاب کند
&&
ز گوهر و لب دریا زبان حجاب کند
\\
بیان حکمت اگر چه شگرف مشعله ایست
&&
ز آفتاب حقایق بیان حجاب کند
\\
جهان کفست و صفات خداست چون دریا
&&
ز صاف بحر کف این جهان حجاب کند
\\
همی‌شکاف تو کف را که تا به آب رسی
&&
به کف بحر بمنگر که آن حجاب کند
\\
ز نقش‌های زمین و ز آسمان مندیش
&&
که نقش‌های زمین و زمان حجاب کند
\\
برای مغز سخن قشر حرف را بشکاف
&&
که زلف‌ها ز جمال بتان حجاب کند
\\
تو هر خیال که کشف حجاب پنداری
&&
بیفکنش که تو را خود همان حجاب کند
\\
نشان آیت حقست این جهان فنا
&&
ولی ز خوبی حق این نشان حجاب کند
\\
ز شمس تبریز ار چه قرضه ایست وجود
&&
قراضه ایست که جان را ز کان حجاب کند
\\
\end{longtable}
\end{center}
