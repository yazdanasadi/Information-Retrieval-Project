\begin{center}
\section*{غزل شماره ۵۵۱: جان و جهان چو روی تو در دو جهان کجا بود}
\label{sec:0551}
\addcontentsline{toc}{section}{\nameref{sec:0551}}
\begin{longtable}{l p{0.5cm} r}
جان و جهان چو روی تو در دو جهان کجا بود
&&
گر تو ستم کنی به جان از تو ستم روا بود
\\
چون همه سوی نور تست کیست دورو به عهد تو
&&
چون همه رو گرفته‌ای روی دگر کجا بود
\\
آنک بدید روی تو در نظرش چه سرد شد
&&
گنج که در زمین بود ماه که در سما بود
\\
با تو برهنه خوشترم جامه تن برون کنم
&&
تا که کنار لطف تو جان مرا قبا بود
\\
ذوق تو زاهدی برد جام تو عارفی کشد
&&
وصف تو عالمی کند ذات تو مر مرا بود
\\
هر که حدیث جان کند با رخ تو نمایمش
&&
عشق تو چون زمردی گر چه که اژدها بود
\\
هر که رخش چنین بود شاه غلام او شود
&&
گر چه که بنده‌ای بود خاصه که در هوا بود
\\
این دل پاره پاره را پیش خیال تو نهم
&&
گر سخن وفا کند گویم کاین وفا بود
\\
چون در ماجرا زنم خانه شرع وا شود
&&
شاهد من رخش بود نرگس او گوا بود
\\
از تبریز شمس دین چونک مرا نعم رسد
&&
جز تبریز و شمس دین جمله وجود لا بود
\\
\end{longtable}
\end{center}
