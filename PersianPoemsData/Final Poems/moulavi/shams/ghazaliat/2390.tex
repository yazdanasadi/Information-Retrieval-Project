\begin{center}
\section*{غزل شماره ۲۳۹۰: آن آتشی که داری در عشق صاف و ساده}
\label{sec:2390}
\addcontentsline{toc}{section}{\nameref{sec:2390}}
\begin{longtable}{l p{0.5cm} r}
آن آتشی که داری در عشق صاف و ساده
&&
فردا از او ببینی صد حور رو گشاده
\\
بنگر به شهوت خود ساده‌ست و صاف بی‌رنگ
&&
یک عالمی صنم بین از ساده ای بزاده
\\
زنبور شهد جانت هر چند ناپدید است
&&
شش خانه‌های او بین از شهد پر نهاده
\\
اندازه تن تو خود سه گز است و کمتر
&&
در خان خود تو بنگر از نه فلک زیاده
\\
تا چند کاسه لیسی این کوزه بر زمین زن
&&
برگیر کاه گل را از روی خنب باده
\\
سجاده آتشین کن تا سجده صاف گردد
&&
آتش رخی برآید از زیر این سجاده
\\
آید سوارگشته بر عشق شمس تبریز
&&
اندر رکاب آن شه خورشید و مه پیاده
\\
\end{longtable}
\end{center}
