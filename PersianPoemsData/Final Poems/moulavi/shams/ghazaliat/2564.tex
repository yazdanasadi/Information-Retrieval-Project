\begin{center}
\section*{غزل شماره ۲۵۶۴: گر عشق بزد راهم ور عقل شد از مستی}
\label{sec:2564}
\addcontentsline{toc}{section}{\nameref{sec:2564}}
\begin{longtable}{l p{0.5cm} r}
گر عشق بزد راهم ور عقل شد از مستی
&&
ای دولت و اقبالم آخر نه توام هستی
\\
رستن ز جهان شک هرگز نبود اندک
&&
خاک کف پای شه کی باشد سردستی
\\
ای طوطی جان پر زن بر خرمن شکر زن
&&
بر عمر موفر زن کز بند قفس رستی
\\
ای جان سوی جانان رو در حلقه مردان رو
&&
در روضه و بستان رو کز هستی خود جستی
\\
در حیرت تو ماندم از گریه و از خنده
&&
با رفعت تو رستم از رفعت و از پستی
\\
ای دل بزن انگشتک بی‌زحمت لی و لک
&&
در دولت پیوسته رفتی و بپیوستی
\\
آن باده فروش تو بس گفت به گوش تو
&&
جان‌ها بپرستندت گر جسم بنپرستی
\\
ای خواجه شنگولی ای فتنه صد لولی
&&
بشتاب چه می مولی آخر دل ما خستی
\\
گر خیر و شرت باشد ور کر و فرت باشد
&&
ور صد هنرت باشد آخر نه در آن شستی
\\
چالاک کسی یارا با آن دل چون خارا
&&
تا ره نزدی ما را از پای بننشستی
\\
درجست در این گفتن بنمودن و بنهفتن
&&
یک پرده برافکندی صد پرده نو بستی
\\
\end{longtable}
\end{center}
