\begin{center}
\section*{غزل شماره ۳۱۹۸: گهی پرده‌سوزی، گهی پرده‌داری}
\label{sec:3198}
\addcontentsline{toc}{section}{\nameref{sec:3198}}
\begin{longtable}{l p{0.5cm} r}
گهی پرده‌سوزی، گهی پرده‌داری
&&
تو سر خزانی، تو جان بهاری
\\
خزان و بهار از تو شد تلخ و شیرین
&&
توی قهر و لطفش، بیا تا چه داری
\\
بهاران بیاید، ببخشی سعادت
&&
خزان چون بیاید، سعادت بکاری
\\
ز گلها که روید بهارت ز دلها
&&
به پیش افکند گل سر، از شرمساری
\\
گرین گل ازان گل یکی لطف بردی
&&
نکردی یکی خار در باغ خاری
\\
همه پادشاهان، شکاری بجویند
&&
توی که به جانت بجوید شکاری
\\
شکاران به پیشت، گلوها کشیده
&&
که جان بخش ما را، سزد جان سپاری
\\
قراری گرفته، غم عشق در دل
&&
قرار غم الحق دهد بی‌قراری
\\
دلا معنی بی‌قراری بگویم
&&
بنه گوش، یارانه بشنو، که یاری
\\
فدیت لمولی به افتخاری
&&
بطی‌الاجابة، سریع‌الفرار
\\
و منذ سبانی هواه، ترانی
&&
اموت و احیی، بغیر اختیاری
\\
اموت بهجر، و احیی بوصل
&&
فهذاک سکری، وذاک خماری
\\
عجبت بانی اذرب بشمس
&&
اذا غاب عنی زمان‌التواری
\\
اذا غاب غبنا، و ان عاتعدنا
&&
کذا عادةالشمس فوق‌الذراری
\\
بمائین یحیی، بحس و عقل
&&
فذوا الحس راکد، وذوا العقل جاری
\\
فماالعقل، الا طلاب المواقب
&&
و ماالحس الاخداع العواری
\\
فذو العقل یبصر هداه و یخضع
&&
و ذوالحس یبصر هواه یماری
\\
گهی آفتابی ز بالا بتابی
&&
گهی ابرواری چو گوهر بباری
\\
زمین گوهرت را به جای چراغی
&&
نهد پیش مهمان به شبهای تاری
\\
ز من چون روی تو ز من رود هم
&&
برم چون بیایی، مرا هم بیاری
\\
\end{longtable}
\end{center}
