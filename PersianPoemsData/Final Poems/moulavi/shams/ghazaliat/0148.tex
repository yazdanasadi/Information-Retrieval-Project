\begin{center}
\section*{غزل شماره ۱۴۸: از پی شمس حق و دین دیده گریان ما}
\label{sec:0148}
\addcontentsline{toc}{section}{\nameref{sec:0148}}
\begin{longtable}{l p{0.5cm} r}
از پی شمس حق و دین دیده گریان ما
&&
از پی آن آفتابست اشک چون باران ما
\\
کشتی آن نوح کی بینیم هنگام وصال
&&
چونک هستی‌ها نماند از پی طوفان ما
\\
جسم ما پنهان شود در بحر باد اوصاف خویش
&&
رو نماید کشتی آن نوح بس پنهان ما
\\
بحر و هجران رو نهد در وصل و ساحل رو دهد
&&
پس بروید جمله عالم لاله و ریحان ما
\\
هر چه می‌بارید اکنون دیده گریان ما
&&
سر آن پیدا کند صد گلشن خندان ما
\\
شرق و غرب این زمین از گلستان یک سان شود
&&
خار و خس پیدا نباشد در گل یک سان ما
\\
زیر هر گلبن نشسته ماه رویی زهره رخ
&&
چنگ عشرت می‌نوازد از پی خاقان ما
\\
هر زمان شهره بتی بینی که از هر گوشه‌ای
&&
جام می را می‌دهد در دست بادستان ما
\\
دیده نادیده ما بوسه دیده زان بتان
&&
تا ز حیرانی گذشته دیده حیران ما
\\
جان سودا نعره زن‌ها این بتان سیمبر
&&
دل گود احسنت عیش خوب بی‌پایان ما
\\
خاک تبریزست اندر رغبت لطف و صفا
&&
چون صفای کوثر و چون چشمه حیوان ما
\\
\end{longtable}
\end{center}
