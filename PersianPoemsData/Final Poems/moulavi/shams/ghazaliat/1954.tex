\begin{center}
\section*{غزل شماره ۱۹۵۴: عیش‌هاتان نوش بادا هر زمان ای عاشقان}
\label{sec:1954}
\addcontentsline{toc}{section}{\nameref{sec:1954}}
\begin{longtable}{l p{0.5cm} r}
عیش‌هاتان نوش بادا هر زمان ای عاشقان
&&
وز شما کان شکر باد این جهان ای عاشقان
\\
نوش و جوش عاشقان تا عرش و تا کرسی رسید
&&
برگذشت از عرش و فرش این کاروان ای عاشقان
\\
از لب دریا چه گویم لب ندارد بحر جان
&&
برفزوده‌ست از مکان و لامکان ای عاشقان
\\
ما مثال موج‌ها اندر قیام و در سجود
&&
تا بدید آید نشان از بی‌نشان ای عاشقان
\\
گر کسی پرسد کیانید ای سراندازان شما
&&
هین بگوییدش که جان جان جان ای عاشقان
\\
گر کسی غواص نبود بحر جان بخشنده است
&&
کو همی‌بخشد گهرها رایگان ای عاشقان
\\
این چنین شد وان چنان شد خلق را در حقه کرد
&&
بازرستیم از چنین و از چنان ای عاشقان
\\
ما رمیت اذ رمیت از شکارستان غیب
&&
می جهاند تیرهای بی‌کمان ای عاشقان
\\
چون ز جست و جوی دل نومید گشتم آمدم
&&
خفته دیدم دل ستان با دلستان ای عاشقان
\\
گفتم ای دل خوش گزیدی دل بخندید و بگفت
&&
گل ستاند گل ستان از گلستان ای عاشقان
\\
زیر پای من گل است و زیر پاهاشان گل است
&&
چون بکوبم پا میان منکران ای عاشقان
\\
خرما آن دم که از مستی جانان جان ما
&&
می نداند آسمان از ریسمان ای عاشقان
\\
طرفه دریایی معلق آمد این دریای عشق
&&
نی به زیر و نی به بالا نی میان ای عاشقان
\\
تا بدید آمد شعاع شمس تبریزی ز شرق
&&
جان مطلق شد زمین و آسمان ای عاشقان
\\
\end{longtable}
\end{center}
