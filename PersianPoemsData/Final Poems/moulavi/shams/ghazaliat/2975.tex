\begin{center}
\section*{غزل شماره ۲۹۷۵: هر روز بامداد به آیین دلبری}
\label{sec:2975}
\addcontentsline{toc}{section}{\nameref{sec:2975}}
\begin{longtable}{l p{0.5cm} r}
هر روز بامداد به آیین دلبری
&&
ای جان جان جان به من آیی و دل بری
\\
ای کوی من گرفته ز بوی تو گلشنی
&&
وی روی من گرفته ز روی تو زرگری
\\
هر روز باغ دل را رنگی دگر دهی
&&
اکنون نماند دل را شکل صنوبری
\\
هر شب مقام دیگر و هر روز شهر نو
&&
چون لولیان گرفته دل من مسافری
\\
این شهسوار عشق قطاریق می‌رود
&&
حیران شدم ز جستن این اسب لاغری
\\
از برق و آب و باد گذشته‌ست سم او
&&
آن جا که سم او است نه خشکی است و نه تری
\\
راهی که فکر نیز نیارد در او شدن
&&
شیران شرزه را رود از دل دلاوری
\\
چه شیر کآسمان و زمین زین ره مهیب
&&
از سر به وقت عرض نهادند لمتری
\\
از هیبت قدر بنهادند رو به جبر
&&
وز بیم رهزنان نگزیدند رهبری
\\
آری جنون ساعه شرط شجاعت است
&&
با مایه خرد نکند هیچ کس نری
\\
تا باخودی کجا به صف بیخودان رسی
&&
تا بر دری چگونه صف هجر بردری
\\
ای دل خیال او را پیش آر و قبله ساز
&&
قانع مشو از او به مراعات سرسری
\\
قانع چرا شدی به یکی صورتت که داد
&&
پنداشتی مگر که همین یک مصوری
\\
خاموش باش طبل مزن وقت حمله شد
&&
در صف جنگ آی اگر مرد لشکری
\\
\end{longtable}
\end{center}
