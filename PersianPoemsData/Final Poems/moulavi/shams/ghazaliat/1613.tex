\begin{center}
\section*{غزل شماره ۱۶۱۳: به خدا کز غم عشقت نگریزم نگریزم}
\label{sec:1613}
\addcontentsline{toc}{section}{\nameref{sec:1613}}
\begin{longtable}{l p{0.5cm} r}
به خدا کز غم عشقت نگریزم نگریزم
&&
وگر از من طلبی جان نستیزم نستیزم
\\
قدحی دارم بر کف به خدا تا تو نیایی
&&
هله تا روز قیامت نه بنوشم نه بریزم
\\
سحرم روی چو ماهت شب من زلف سیاهت
&&
به خدا بی‌رخ و زلفت نه بخسبم نه بخیزم
\\
ز جلال تو جلیلم ز دلال تو دلیلم
&&
که من از نسل خلیلم که در این آتش تیزم
\\
بده آن آب ز کوزه که نه عشقی است دوروزه
&&
چو نماز است و چو روزه غم تو واجب و ملزم
\\
به خدا شاخ درختی که ندارد ز تو بختی
&&
اگرش آب دهد یم شود او کنده هیزم
\\
بپر ای دل سوی بالا به پر و قوت مولا
&&
که در آن صدر معلا چو تویی نیست ملازم
\\
همگان وقت بلاها بستایند خدا را
&&
تو شب و روز مهیا چو فلک جازم و حازم
\\
صفت مفخر تبریز نگویم به تمامت
&&
چه کنم رشک نخواهد که من آن غالیه بیزم
\\
\end{longtable}
\end{center}
