\begin{center}
\section*{غزل شماره ۲۰۰۲: تو سبب سازی و دانایی آن سلطان بین}
\label{sec:2002}
\addcontentsline{toc}{section}{\nameref{sec:2002}}
\begin{longtable}{l p{0.5cm} r}
تو سبب سازی و دانایی آن سلطان بین
&&
آنچ ممکن نبود در کف او امکان بین
\\
آهن اندر کف او نرمتر از مومی بین
&&
پیش نور رخ او اختر را پنهان بین
\\
نم اندیشه بیا قلزم اندیشه نگر
&&
صورت چرخ بدیدی هله اکنون جان بین
\\
جان بنفروختی ای خر به چنین مشتریی
&&
رو به بازار غمش جان چو علف ارزان بین
\\
هر کی بفسرد بر او سخت نماید حرکت
&&
اندکی گرم شو و جنبش را آسان بین
\\
خشک کردی تو دماغ از طلب بحث و دلیل
&&
بفشان خویش ز فکر و لمع برهان بین
\\
هست میزان معینت و بدان می‌سنجی
&&
هله میزان بگذار و زر بی‌میزان بین
\\
نفسی موضع تنگ و نفسی جای فراخ
&&
می جان نوش و از آن پس همه را میدان بین
\\
سحر کرده‌ست تو را دیو همی‌خوان قل اعوذ
&&
چونک سرسبز شدی جمله گل و ریحان بین
\\
چون تو سرسبز شدی سبز شود جمله جهان
&&
اتحادی عجبی در عرض و ابدان بین
\\
چون دمی چرخ زنی و سر تو برگردد
&&
چرخ را بنگر و همچون سر خود گردان بین
\\
ز آنک تو جزو جهانی مثل کل باشی
&&
چونک نو شد صفتت آن صفت از ارکان بین
\\
همه ارکان چو لباس آمد و صنعش چو بدن
&&
چند مغرور لباسی بدن انسان بین
\\
روی ایمان تو در آیینه اعمال ببین
&&
پرده بردار و درآ شعشعه ایمان بین
\\
گر تو عاشق شده‌ای حسن بجو احسان نی
&&
ور تو عباس زمانی بنشین احسان بین
\\
لابه کردم شه خود را پس از این او گوید
&&
چونک دریاش بجوشد در بی‌پایان بین
\\
\end{longtable}
\end{center}
