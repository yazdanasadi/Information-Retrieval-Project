\begin{center}
\section*{غزل شماره ۲۹۵۹: ای آنک جمله عالم از توست یک نشانی}
\label{sec:2959}
\addcontentsline{toc}{section}{\nameref{sec:2959}}
\begin{longtable}{l p{0.5cm} r}
ای آنک جمله عالم از توست یک نشانی
&&
زخمت بر این نشانه آمد کنون تو دانی
\\
زخمی بزن دگر تو مرهم نخواهم از تو
&&
گر یک جهان نماند چه غم تو صد جهانی
\\
در شرح درنیایی چون شرح سر حقی
&&
در جان چرا نیایی چون جان جان جانی
\\
ماییم چون درختان صنع تو باد گردان
&&
خود کار باد دارد هر چند شد نهانی
\\
زان باد سبز گردیم زان باد زرد گردیم
&&
گر برگ را بریزی از میوه کی ستانی
\\
در نقش باغ پیش است در اصل میوه پیش است
&&
تو اولین گهر را آخر همی‌رسانی
\\
خواهم که از تو گویم وز جز تو دست شویم
&&
پنهان شوی و ما را در صف همی‌کشانی
\\
\end{longtable}
\end{center}
