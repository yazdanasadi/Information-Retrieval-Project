\begin{center}
\section*{غزل شماره ۳۰۳۸: ببرد عقل و دلم را براق عشق معانی}
\label{sec:3038}
\addcontentsline{toc}{section}{\nameref{sec:3038}}
\begin{longtable}{l p{0.5cm} r}
ببرد عقل و دلم را براق عشق معانی
&&
مرا بپرس کجا برد آن طرف که ندانی
\\
بدان رواق رسیدم که ماه و چرخ ندیدم
&&
بدان جهان که جهان هم جدا شود ز جهانی
\\
یکی دمیم امان ده که عقل من به من آید
&&
بگویمت صفت جان تو گوش دار که جانی
\\
ولیک پیشتر آ خواجه گوش بر دهنم ده
&&
که گوش دارد دیوار و این سریست نهانی
\\
عنایتیست ز جانان چنین غریب کرامت
&&
ز راه گوش درآید چراغ‌های عیانی
\\
رفیق خضر خرد شو به سوی چشمه حیوان
&&
که تا چو چشمه خورشید روز نور فشانی
\\
چنانک گشت زلیخا جوان به همت یوسف
&&
جهان کهنه بیابد از این ستاره جوانی
\\
فروخورد مه و خورشید و قطب هفت فلک را
&&
سهیل جان چو برآید ز سوی رکن یمانی
\\
دمی قراضه دین را بگیر و زیر زبان نه
&&
که تا به نقد ببینی که در درونه چه کانی
\\
فتاده‌ای به دهان‌ها همی‌گزندت مردم
&&
لطیف و پخته چو نانی بدان همیشه چنانی
\\
چو ذره پای بکوبی چو نور دست تو گیرد
&&
ز سردیست و ز تری که همچو ریگ گرانی
\\
چو آفتاب برآمد به خاک تیره بگوید
&&
که چون قرین تو گشتم تو صاحب دو قرانی
\\
تو بز نه‌ای که برآیی چراغپایه به بازی
&&
که پیش گله شیران چو نره شیر شبانی
\\
چراغ پنج حست را به نور دل بفروزان
&&
حواس پنج نمازست و دل چو سبع مثانی
\\
همی‌رسد ز سموات هر صبوح ندایی
&&
که ره بری به نشانی چو گرد ره بنشانی
\\
سپس مکش چو مخنث عنان عزم که پیشت
&&
دو لشکرست که در وی تو پیش رو چو سنانی
\\
شکر به پیش تو آمد که برگشای دهان را
&&
چرا ز دعوت شکر چو پسته بسته دهانی
\\
بگیر طبله شکر بخور به طبل که نوشت
&&
مکوب طبل فسانه چرا حریف زبانی
\\
ز شمس مفخر تبریز آفتاب پرستی
&&
که اوست شمس معارف رئیس شمس مکانی
\\
\end{longtable}
\end{center}
