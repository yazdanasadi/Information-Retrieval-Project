\begin{center}
\section*{غزل شماره ۹۴۵: ندا رسید به جان‌ها که چند می‌پایید}
\label{sec:0945}
\addcontentsline{toc}{section}{\nameref{sec:0945}}
\begin{longtable}{l p{0.5cm} r}
ندا رسید به جان‌ها که چند می‌پایید
&&
به سوی خانه اصلی خویش بازآیید
\\
چو قاف قربت ما زاد و بود اصل شماست
&&
به کوه قاف بپرید خوش چو عنقایید
\\
ز آب و گل چو چنین کنده ایست بر پاتان
&&
بجهد کنده ز پا پاره پاره بگشایید
\\
سفر کنید از این غربت و به خانه روید
&&
از این فراق ملولیم عزم فرمایید
\\
به دوغ گنده و آب چه و بیابان‌ها
&&
حیات خویش به بیهوده چند فرسایید
\\
خدای پر شما را ز جهد ساخته است
&&
چو زنده‌اید بجنبید و جهد بنمایید
\\
به کاهلی پر و بال امید می‌پوسد
&&
چو پر و بال بریزد دگر چه را شایید
\\
از این خلاص ملولید و قعر این چه نی
&&
هلا مبارک در قعر چاه می‌پایید
\\
ندای فاعتبروا بشنوید اولوالابصار
&&
نه کودکیت سر آستین چه می‌خایید
\\
خود اعتبار چه باشد به جز ز جو جستن
&&
هلا ز جو بجهید آن طرف چو برنایید
\\
درون هاون شهوت چه آب می‌کوبید
&&
چو آبتان نبود باد لاف پیمایید
\\
حطام خواند خدا این حشیش دنیا را
&&
در این حشیش چو حیوان چه ژاژ می‌خایید
\\
هلا که باده بیامد ز خم برون آیید
&&
پی قطایف و پالوده تن بپالایید
\\
هلا که شاهد جان آینه همی‌جوید
&&
به صیقل آینه‌ها را ز زنگ بزدایید
\\
نمی‌هلند که مخلص بگویم این‌ها را
&&
ز اصل چشمه بجویید آن چو جویایید
\\
\end{longtable}
\end{center}
