\begin{center}
\section*{غزل شماره ۲۸۵۶: صنما چگونه گویم که تو نور جان مایی}
\label{sec:2856}
\addcontentsline{toc}{section}{\nameref{sec:2856}}
\begin{longtable}{l p{0.5cm} r}
صنما چگونه گویم که تو نور جان مایی
&&
که چه طاقت است جان را چو تو نور خود نمایی
\\
تو چنان همایی ای جان که به زیر سایه تو
&&
به کف آورند زاغان همه خلقت همایی
\\
کرم تو عذرخواه همه مجرمان عالم
&&
تو امان هر بلایی تو گشاد بندهایی
\\
تویی گوهری که محو است دو هزار بحر در تو
&&
تویی بحر بی‌کرانه ز صفات کبریایی
\\
به وصال می‌بنالم که چه بی‌وفا قرینی
&&
به فراق می‌بزارم که چه یار باوفایی
\\
به گه وصال آن مه چه بود خدای داند
&&
که گه فراق باری طرب است و جان فزایی
\\
دل اگر جنون آرد خردش تویی که رفتی
&&
رخ توست عذرخواهش به گهی که رخ گشایی
\\
\end{longtable}
\end{center}
