\begin{center}
\section*{غزل شماره ۱۸۵۴: چه دانی تو خراباتی که هست از شش جهت بیرون}
\label{sec:1854}
\addcontentsline{toc}{section}{\nameref{sec:1854}}
\begin{longtable}{l p{0.5cm} r}
چه دانی تو خراباتی که هست از شش جهت بیرون
&&
خرابات قدیم است آن و تو نو آمده اکنون
\\
نباشد مرغ خودبین را به باغ بیخودان پروا
&&
نشد مجنون آن لیلی به جز لیلی صد مجنون
\\
هزاران مجلس است آن سو و این مجلس از آن سوتر
&&
که این بی‌چونتر است اندر میان عالم بی‌چون
\\
ببین جان‌های آن شیران در آن بیشه ز اجل لرزان
&&
کز آن شیر اجل شیران نمی‌میزند الا خون
\\
بسی سیمرغ ربانی که تسبیحش اناالحق شد
&&
بسوزد پر و بال او اگر یک پر زند آن سون
\\
وزیر و حاجب و محمود ایازی را شده چاکر
&&
که آن جا کو قدم دارد بود سرهای مردان دون
\\
تو معذوری در انکارت که آن جا می شود حیران
&&
جنید و شیخ بسطامی شقیق و کرخی و ذاالنون
\\
ازیرا راه نتوان برد سوی آفتاب ای جان
&&
مگر کان آفتاب از خود برآید سوی این هامون
\\
مگر هم لطف شمس الدین تبریزیت برهاند
&&
وگر نی این غزل می خوان و بر خود می دم این افسون
\\
\end{longtable}
\end{center}
