\begin{center}
\section*{غزل شماره ۲۴۳: بگشا در بیا درآ که مبا عیش بی‌شما}
\label{sec:0243}
\addcontentsline{toc}{section}{\nameref{sec:0243}}
\begin{longtable}{l p{0.5cm} r}
بگشا در بیا درآ که مبا عیش بی‌شما
&&
به حق چشم مست تو که تویی چشمه وفا
\\
سخنم بسته می‌شود تو یکی زلف برگشا
&&
انا و الشمس و الضحی تلف الحب و الولا
\\
انا فی العشق آیه فاقرونی علی الملا
&&
امه العشق فاعرجوا دونکم سلم الهوی
\\
دیدمش مست می‌گذشت گفتم ای ماه تا کجا
&&
گفت نی همچنین مکن همچنین در پیم بیا
\\
در پیش چون روان شدم برگرفت تیز تیزپا
&&
در پی گام تیز او چه محل باد و برق را
\\
انا منذ رایتهم انا صرت بلا انا
&&
صوره فی زجاجه نور الارض و السما
\\
رکب القلب نوره فجلی القلب و اصطفی
&&
کل من رام نوره استضا مثله استضا
\\
کیف یلقاه غیره کل من غیر فنا
&&
تو بیا بی‌تو پیش من که تو نامحرمی تو را
\\
به ثنا لابه کردمش گفتم ای جان جان فزا
&&
گفت یک دم ثنا مگو که دوی هست در ثنا
\\
تو دو لب از دوی ببند بگشا دیده بقا
&&
ز لب بسته گر سخن بگشاید گشا گشا
\\
ان علینا بیانه تو میا در میان ما
&&
چو در خانه دید تنگ بکند مرد جامه‌ها
\\
نی که هر شب روان تو ز تنت می‌شود جدا
&&
به میان روان تو صفتی هست ناسزا
\\
که گر آن ریگ نیستی نامدی باز چون صبا
&&
شب نرفتی دوان دوان به لب قلزم صفا
\\
بازآمد و تا ویست بنده بنده‌ست خدا خدا
&&
ماند در کیسه بدن چو زر و سیم ناروا
\\
جان بنه بر کف طلب که طلب هست کیمیا
&&
تا تن از جان جدا شدن مشو از جان جان جدا
\\
گر چه نی را تهی کنند نگذارند بی‌نوا
&&
رو پی شیر و شیر گیر که علیی و مرتضی
\\
نیست بودی تو قرن‌ها بر تو خواندند هل اتی
&&
خط حقست نقش دل خط حق را مخوان خطا
\\
الفی لا شود و تو ز الف لام گشت لا
&&
هله دست و دهان بشو که لبش گفت الصلا
\\
چو به حق مشتغل شدی فارغ از آب و گل شدی
&&
چو که بی‌دست و دل شدی دست درزن در این ابا
\\
\end{longtable}
\end{center}
