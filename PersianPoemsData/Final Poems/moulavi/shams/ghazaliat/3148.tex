\begin{center}
\section*{غزل شماره ۳۱۴۸: تا شدستی امیر چوگانی}
\label{sec:3148}
\addcontentsline{toc}{section}{\nameref{sec:3148}}
\begin{longtable}{l p{0.5cm} r}
تا شدستی امیر چوگانی
&&
ما شدستیم گوی میدانی
\\
ما در این دور مست و بی‌خبریم
&&
سر این دور را تو می‌دانی
\\
چون به دور و تسلسل انجامد
&&
نکته ابتر بود به ربانی
\\
لیک دور و تسلسل اندر عشق
&&
شرط هر حجتست و برهانی
\\
گوش موشان خانه کی شنود
&&
نعره بلبل گلستانی
\\
چشم پیران کور کی بیند
&&
شیوه شاهدان روحانی
\\
هر کی کورست عشق می‌سازد
&&
بهر او سرمه سپاهانی
\\
هر کی پیرست هم جوان گردد
&&
چون دهد عشق آب حیوانی
\\
جمله یاران ز عشق زنده شدند
&&
تو چنین مانده‌ای چه می‌مانی
\\
خرسواری پیاده شو از خر
&&
خر به میدان نباشد ارزانی
\\
خرسواره چرا شدی شاها
&&
خسروی وز نژاد سلطانی
\\
لایق پشت خر نباشی تو
&&
تو معود به پشت اسپانی
\\
در جنود مجنده بودی
&&
ای که اکنون تو روح انسانی
\\
گفتنی‌ها بگفتمی ای جان
&&
گر نترسیدمی ز ویرانی
\\
\end{longtable}
\end{center}
