\begin{center}
\section*{غزل شماره ۱۱۳۹: مجوی شادی چون در غمست میل نگار}
\label{sec:1139}
\addcontentsline{toc}{section}{\nameref{sec:1139}}
\begin{longtable}{l p{0.5cm} r}
مجوی شادی چون در غمست میل نگار
&&
که در دو پنجه شیری تو ای عزیز شکار
\\
اگر چه دلبر ریزد گلابه بر سر تو
&&
قبول کن تو مر آن را به جای مشک تتار
\\
درون تو چو یکی دشمنیست پنهانی
&&
بجز جفا نبود هیچ دفع آن سگسار
\\
کسی که بر نمدی چوب زد نه بر نمدست
&&
ولی غرض همه تا آن برون شود ز غبار
\\
غبارهاست درون تو از حجاب منی
&&
همی‌برون نشود آن غبار از یک بار
\\
به هر جفا و به هر زخم اندک اندک آن
&&
رود ز چهره دل گه به خواب و گه بیدار
\\
اگر به خواب گریزی به خواب دربینی
&&
جفای یار و سقط‌های آن نکوکردار
\\
تراش چوب نه بهر هلاکت چوبست
&&
برای مصلحتی راست در دل نجار
\\
از این سبب همه شر طریق حق خیرست
&&
که عاقبت بنماید صفاش آخر کار
\\
نگر به پوست که دباغ در پلیدی‌ها
&&
همی‌بمالد آن را هزار بار هزار
\\
که تا برون رود از پوست علت پنهان
&&
اگر چه پوست نداند ز اندک و بسیار
\\
تو شمس مفخر تبریز چاره‌ها داری
&&
شتاب کن که تو را قدرتیست در اسرار
\\
\end{longtable}
\end{center}
