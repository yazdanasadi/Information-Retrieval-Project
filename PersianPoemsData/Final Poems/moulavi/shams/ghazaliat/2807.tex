\begin{center}
\section*{غزل شماره ۲۸۰۷: گشت جان از صدر شمس الدین یکی سوداییی}
\label{sec:2807}
\addcontentsline{toc}{section}{\nameref{sec:2807}}
\begin{longtable}{l p{0.5cm} r}
گشت جان از صدر شمس الدین یکی سوداییی
&&
در درون ظلمت سودا را داناییی
\\
یک بلندی یافت بختم در هوای شمس دین
&&
کز ورای آن نباشد وهم را گنجاییی
\\
مایه سودا در این عشقم چنان بالا گرفت
&&
کز سر سودا نداند پستی از بالاییی
\\
موج سودا و جنونی کز هوای او بخاست
&&
بر سر آن موج چون خاشاک من هرجاییی
\\
عقل پابرجای من چون دید شور بحر او
&&
با چنین شوری ندارد عقل کل تواناییی
\\
مصحف دیوانگی دیدم بخواندم آیتی
&&
گشت منسوخ از جنونم دانش و قراییی
\\
عشق یکتا دزد شب رو بود اندر سینه‌ها
&&
عقل را خفته بگیرد دزددش یکتاییی
\\
پیش از این سودا دل و جان عاقل رای خودند
&&
بعد از آن غرقاب کی باشد تو را خودراییی
\\
رو تو در بیمارخانه عاشقی تا بنگری
&&
هر طرف دیوانه جانی هر سوی شیداییی
\\
دوش دیدم عشق را می‌کرد از خون سرشک
&&
بر سر بام دلم از هجر خون انداییی
\\
هست مر سودای عاشق را دلا این خاصیت
&&
گر چه او پستی رود باشد بر آن بالاییی
\\
گرد دارایی جان مظلم ناپایدار
&&
گشت جان پایداری از چنان داراییی
\\
یک دمی مرده شو از جمله فضولی‌ها ببین
&&
هر نفس جان بخشیی هر دم مسیح آساییی
\\
یک نفس در پرده عشقش چو جانت غسل کرد
&&
همچو مریم از دمی بینی تو عیسی زاییی
\\
چون بزادی همچو مریم آن مسیح بی‌پدر
&&
گردد این رخسار سرخت زعفران سیماییی
\\
نام مخدومی شمس الدین همی‌گو هر دمی
&&
تا بگیرد شعر و نظمت رونق و رعناییی
\\
خون ببین در نظم شعرم شعر منگر بهر آنک
&&
دیده و دل را به عشقش هست خون پالاییی
\\
خون چو می‌جوشد منش از شعر رنگی می‌دهم
&&
تا نه خون آلود گردد جامه خون آلاییی
\\
من چو جانداری بدم در خدمت آن پادشاه
&&
اینک اکنون در فراقش می‌کنم جان ساییی
\\
در هوای سایه‌ای عنقای آن خورشید لطف
&&
دل به غربت برگرفته عادت عنقاییی
\\
چون به خوبی و ملاحت هست تنها در جهان
&&
داد جان را از زمانه شیوه تنهاییی
\\
چون شوم نومید از آن آهو که مشکش دم به دم
&&
در طلب می‌داردم از بوی و از بویاییی
\\
آه از آن رخسار مریخی خون ریزش مرا
&&
آه از آن ترکانه چشم کافر یغماییی
\\
عقل در دهلیز عشقش خاکروبی بی‌دلی
&&
ناطقه در لشکرش یا طبلیی یا ناییی
\\
او همه دیده‌ست اندر درد و اندر رنج من
&&
من نمی‌تانم که گویم نیستش بیناییی
\\
من نظر کردم دمی در جان سودارنگ خویش
&&
دیدم او را پیچ پیچ و شورش و درواییی
\\
گفتم آخر چیست گفتا دست را از من بشو
&&
من نیم در عشق او امروزی و فرداییی
\\
در هر آن شهری که نوشروان عشقش حاکم است
&&
شد به جان درباختن آن شهر حاتم طاییی
\\
و اندر آن جانی که گردان شد پیاله عشق او
&&
عقل را باشد از آن جان محو و ناپیداییی
\\
چون خیالش نیم شب در سینه آید می‌نگر
&&
هر نواحی یوسفی و هر طرف حوراییی
\\
در شکرریز لبش جان‌ها به هنگام وصال
&&
هر سر مویی تو را بوده‌ست شکرخاییی
\\
چون میی در عشق او تا کهنه‌تر تو مستتر
&&
کی جوانی یاد آرد جانت یا برناییی
\\
سلسله این عشق درجنبان و شورم بیش کن
&&
بحر سودا را بجوش و کن جنون افزاییی
\\
این عجب بحری که بهر نازکی خاک تو
&&
قطره‌ای گشته‌ست و ننماید همی‌دریاییی
\\
بهر ضعف این دماغ زخمگاه عشق خویش
&&
می‌کند آن زلف عنبر مشک و عنبرساییی
\\
چهره‌های یوسفان و فتنه انگیزان دهر
&&
از گدایی حسن او دارند هر زیباییی
\\
گر شود موسی بیاموزم جهودی را تمام
&&
ور بود عیسی بگیرم ملت ترساییی
\\
گر به جانش میل باشد جان شوم همچون هوا
&&
ور به دنیا رو بیارد من شوم دنیاییی
\\
جان من چون سفره خود را درکشد از سحر او
&&
گرده گرم از تنورت بخشدش پهناییی
\\
نفس و شیطان در غرور باغ لطفت می‌چرند
&&
ز اعتماد عفو تو دارند بدفرماییی
\\
نفس را نفسی نماند دیو را دیوی شود
&&
گر تو از رخسار یک دم پرده‌ها بگشاییی
\\
ای صبا جانم تو را چاکر شدی بر چشم و سر
&&
گر ز تبریزم کنی خاک کفش بخشاییی
\\
\end{longtable}
\end{center}
