\begin{center}
\section*{غزل شماره ۱۳۹۷: زین دو هزاران من و ما ای عجبا من چه منم}
\label{sec:1397}
\addcontentsline{toc}{section}{\nameref{sec:1397}}
\begin{longtable}{l p{0.5cm} r}
زین دو هزاران من و ما ای عجبا من چه منم
&&
گوش بنه عربده را دست منه بر دهنم
\\
چونک من از دست شدم در ره من شیشه منه
&&
ور بنهی پا بنهم هر چه بیابم شکنم
\\
زانک دلم هر نفسی دنگ خیال تو بود
&&
گر طربی در طربم گر حزنی در حزنم
\\
تلخ کنی تلخ شوم لطف کنی لطف شوم
&&
با تو خوش است ای صنم لب شکر خوش ذقنم
\\
اصل تویی من چه کسم آینه‌ای در کف تو
&&
هر چه نمایی بشوم آینه ممتحنم
\\
تو به صفت سرو چمن من به صفت سایه تو
&&
چونک شدم سایه گل پهلوی گل خیمه زنم
\\
بی‌تو اگر گل شکنم خار شود در کف من
&&
ور همه خارم ز تو من جمله گل و یاسمنم
\\
دم به دم از خون جگر ساغر خونابه کشم
&&
هر نفسی کوزه خود بر در ساقی شکنم
\\
دست برم هر نفسی سوی گریبان بتی
&&
تا بخراشد رخ من تا بدرد پیرهنم
\\
لطف صلاح دل و دین تافت میان دل من
&&
شمع دل است او به جهان من کیم او را لگنم
\\
\end{longtable}
\end{center}
