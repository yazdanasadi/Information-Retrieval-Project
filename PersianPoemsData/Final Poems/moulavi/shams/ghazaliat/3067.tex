\begin{center}
\section*{غزل شماره ۳۰۶۷: تو در عقیله ترتیب کفش و دستاری}
\label{sec:3067}
\addcontentsline{toc}{section}{\nameref{sec:3067}}
\begin{longtable}{l p{0.5cm} r}
تو در عقیله ترتیب کفش و دستاری
&&
چگونه رطل گران خوار را به دست آری
\\
به جان من به خرابات آی یک لحظه
&&
تو نیز آدمیی مردمی و جان داری
\\
بیا و خرقه گرو کن به می فروش الست
&&
که پیش از آب و گلست از الست خماری
\\
فقیر و عارف و درویش وانگهی هشیار
&&
مجاز بود چنین نام‌ها تو پنداری
\\
سماع و شرب سقاهم نه کار درویش‌ست
&&
زیان و سود کم و بیش کار بازاری
\\
بیا بگو که چه باشد الست عیش ابد
&&
ملنگ هین به تکلف که سخت رهواری
\\
سری که درد ندارد چراش می‌بندی
&&
چرا نهی تن بی‌رنج را به بیماری
\\
\end{longtable}
\end{center}
