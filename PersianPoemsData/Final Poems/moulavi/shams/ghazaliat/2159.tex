\begin{center}
\section*{غزل شماره ۲۱۵۹: باده چو هست ای صنم بازمگیر و نی مگو}
\label{sec:2159}
\addcontentsline{toc}{section}{\nameref{sec:2159}}
\begin{longtable}{l p{0.5cm} r}
باده چو هست ای صنم بازمگیر و نی مگو
&&
عرضه مکن دو دست تی پر کن زود آن سبو
\\
ای طربون غم شکن سنگ بر این سبو مزن
&&
از در حق به یک سبو کم نشده‌ست آب جو
\\
زان قدحی که ساحران جان به فدا شدند از آن
&&
چون کف موسی نبی بزم نهاد و کرد طو
\\
فاش بیا و فاش ده باده عشق فاش به
&&
عید شده‌ست و عام را گر رمضان است باش گو
\\
رغم سپید ماخ را رقص درآر شاخ را
&&
و آن کرم فراخ را بازگشای تو به تو
\\
مهره که درربوده‌ای بر کف دست نه دمی
&&
و آن گروی که برده‌ای بار دوم ز ما مجو
\\
مرده به مرگ پار من زنده شده ز یار من
&&
چند خزیده در کفن زنده از آن مسیح خو
\\
منکر حشر روز دین ژاژ مخا بیا ببین
&&
رسته چو سبزه از زمین سروقدان باغ هو
\\
خامش کرده جملگان ناطق غیب بی‌زبان
&&
خطبه بخوانده بر جهان بی‌نغمات و گفت و گو
\\
\end{longtable}
\end{center}
