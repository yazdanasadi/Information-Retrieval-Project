\begin{center}
\section*{غزل شماره ۲۲۳۱: می‌دوید از هر طرف در جست و جو}
\label{sec:2231}
\addcontentsline{toc}{section}{\nameref{sec:2231}}
\begin{longtable}{l p{0.5cm} r}
می‌دوید از هر طرف در جست و جو
&&
چشم پرخون تیغ در کف عشق او
\\
دوش خفته خلق اندر خواب خوش
&&
او به قصد جان عاشق سو به سو
\\
گاه چون مه تافته بر بام‌ها
&&
گاه چون باد صبا او کو به کو
\\
ناگهان افکند طشت ما ز بام
&&
پاسبانان درشده در گفت و گو
\\
در میان کوی بانگ دزد خاست
&&
او بزد زخمی و پنهان کرد رو
\\
گرد او را پاسبانی درنیافت
&&
کش زبون گشته‌ست چرخ تندخو
\\
بر سر زخم آمد افلاطون عقل
&&
کو نشان‌ها را بداند مو به مو
\\
گفت دانستم که زخم دست کیست
&&
کو است اصل فتنه‌های تو به تو
\\
چونک زخم او است نبود چاره‌ای
&&
آنچ او بشکافت نپذیرد رفو
\\
از پی این زخم جان نو رسید
&&
جان کهنه دست‌ها از خود بشو
\\
عشق شمس الدین تبریزی است این
&&
کو برون است از جهان رنگ و بو
\\
\end{longtable}
\end{center}
