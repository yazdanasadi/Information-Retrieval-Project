\begin{center}
\section*{غزل شماره ۱۹۹: ای خان و مان بمانده و از شهر خود جدا}
\label{sec:0199}
\addcontentsline{toc}{section}{\nameref{sec:0199}}
\begin{longtable}{l p{0.5cm} r}
ای خان و مان بمانده و از شهر خود جدا
&&
شاد آمدیت از سفر خانه خدا
\\
روز از سفر به فاقه و شب‌ها قرار نی
&&
در عشق حج کعبه و دیدار مصطفا
\\
مالیده رو و سینه در آن قبله گاه حق
&&
در خانه خدا شده قد کان آمنسا
\\
چونید و چون بدیت در این راه باخطر
&&
ایمن کند خدای در این راه جمله را
\\
در آسمان ز غلغل لبیک حاجیان
&&
تا عرش نعره‌ها و غریوست از صدا
\\
جان چشم تو ببوسد و بر پات سر نهد
&&
ای مروه را بدیده و بررفته بر صفا
\\
مهمان حق شدیت و خدا وعده کرده است
&&
مهمان عزیز باشد خاصه به پیش ما
\\
جان خاک اشتری که کشد بار حاجیان
&&
تا مشعرالحرام و تا منزل منا
\\
بازآمده ز حج و دل آن جا شده مقیم
&&
جان حلقه را گرفته و تن گشته مبتلا
\\
از شام ذات جحفه و از بصره ذات عرق
&&
باتیغ و باکفن شده این جا که ربنا
\\
کوه صفا برآ به سر کوه رخ به بیت
&&
تکبیر کن برادر و تهلیل و هم دعا
\\
اکنون که هفت بار طوافت قبول شد
&&
اندر مقام دو رکعت کن قدوم را
\\
وانگه برآ به مروه و مانند این بکن
&&
تا هفت بار و باز به خانه طواف‌ها
\\
تا روز ترویه بشنو خطبه بلیغ
&&
وانگه به جانب عرفات آی در صلا
\\
وانگه به موقف آی و به قرب جبل بایست
&&
پس بامداد بار دگر بیست هم به جا
\\
وان گاه روی سوی منی آر و بعد از آن
&&
تا هفت بار می‌زن و می‌گیر سنگ‌ها
\\
از ما سلام بادا بر رکن و بر حطیم
&&
ای شوق ما به زمزم و آن منزل وفا
\\
صبحی بود ز خواب بخیزیم گرد ما
&&
از اذخر و خلیل به ما بو دهد صبا
\\
\end{longtable}
\end{center}
