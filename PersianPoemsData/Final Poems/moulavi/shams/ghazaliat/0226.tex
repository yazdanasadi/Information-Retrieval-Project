\begin{center}
\section*{غزل شماره ۲۲۶: برفت یار من و یادگار ماند مرا}
\label{sec:0226}
\addcontentsline{toc}{section}{\nameref{sec:0226}}
\begin{longtable}{l p{0.5cm} r}
برفت یار من و یادگار ماند مرا
&&
رخ معصفر و چشم پرآب و وااسفا
\\
دو دیده باشد پرنم چو در ویست مقیم
&&
فرات و کوثر آب حیات جان افزا
\\
چرا رخم نکند زرگری چو متصلست
&&
به گنج بی‌حد و کان جمال و حسن و بها
\\
چراست وااسفاگوی زانک یعقوبست
&&
ز یوسف کش مه روی خویش گشته جدا
\\
ز ناز اگر برود تا ستاره بار شوم
&&
رسد چو می‌زندش آفتاب طال بقا
\\
اگر چیم ز چراگاه جان برون کردست
&&
کجاست زهره و یارا که گویمش که چرا
\\
الست عشق رسید و هر آن که گفت بلی
&&
گواه گفت بلی هست صد هزار بلا
\\
بلا درست و بلادر تو را کند زیرک
&&
خصوص در یتیمی که هست از آن دریا
\\
منم کبوتر او گر براندم سر نی
&&
کجا پرم نپرم جز که گرد بام و سرا
\\
منم ز سایه او آفتاب عالمگیر
&&
که سلطنت رسد آن را که یافت ظل هما
\\
بس است دعوت دعوت بهل دعا می‌گو
&&
مسیح رفت به چارم سما به پر دعا
\\
\end{longtable}
\end{center}
