\begin{center}
\section*{غزل شماره ۴۲۹: عاشقی و بی‌وفایی کار ماست}
\label{sec:0429}
\addcontentsline{toc}{section}{\nameref{sec:0429}}
\begin{longtable}{l p{0.5cm} r}
عاشقی و بی‌وفایی کار ماست
&&
کار کار ماست چون او یار ماست
\\
قصد جان جمله خویشان کنیم
&&
هر چه خویش ما کنون اغیار ماست
\\
عقل اگر سلطان این اقلیم شد
&&
همچو دزد آویخته بر دار ماست
\\
خویش و بی‌خویشی به یک جا کی بود
&&
هر گلی کز ما بروید خار ماست
\\
خودپرستی نامبارک حالتیست
&&
کاندر او ایمان ما انکار ماست
\\
آنک افلاطون و جالینوس توست
&&
از منی پرعلت و بیمار ماست
\\
نوبهاری کو نوی خود بدید
&&
جان گلزارست اما زار ماست
\\
این منی خاکست زر در وی بجو
&&
کاندر او گنجور یار غار ماست
\\
خاک بی‌آتش بننماید گهر
&&
عشق و هجران ابر آتشبار ماست
\\
طالبا بشنو که بانگ آتشست
&&
تا نپنداری که این گفتار ماست
\\
طالبا بگذر از این اسرار خود
&&
سر طالب پرده اسرار ماست
\\
نور و نار توست ذوق و رنج تو
&&
رو بدان جایی که نور و نار ماست
\\
گاه گویی شیرم و گه شیرگیر
&&
شیرگیر و شیر تو کفتار ماست
\\
طالب ره طالب شه کی بود
&&
گر چه دل دارد مگو دلدار ماست
\\
شهر از عاقل تهی خواهد شدن
&&
این چنین ساقی که این خمار ماست
\\
عاشق و مفلس کند این شهر را
&&
این چنین چابک که این طرار ماست
\\
مدرسه عشق و مدرس ذوالجلال
&&
ما چو طالب علم و این تکرار ماست
\\
شمس تبریزی که شاه دلبری‌ست
&&
با همه شاهنشهی جاندار ماست
\\
\end{longtable}
\end{center}
