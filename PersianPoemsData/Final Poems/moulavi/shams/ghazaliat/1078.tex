\begin{center}
\section*{غزل شماره ۱۰۷۸: شادیی کان از جهان اندر دلت آید مخر}
\label{sec:1078}
\addcontentsline{toc}{section}{\nameref{sec:1078}}
\begin{longtable}{l p{0.5cm} r}
شادیی کان از جهان اندر دلت آید مخر
&&
شادیی کان از دلت آید زهی کان شکر
\\
بازخر جان مرا زین هر دو فراش ای خدا
&&
پهلوی اصحاب کهفم خوش بخسبان بی‌خبر
\\
سایه شادیست غم غم در پی شادی دود
&&
ترک شادی کن که این دو نسکلد از همدگر
\\
در پی روزست شب و اندر پی شادیست غم
&&
چون بدیدی روز دان کز شب نتان کردن حذر
\\
تا پی غم می‌دوی شادی پی تو می‌دود
&&
چون پی شادی روی تو غم بود بر ره گذر
\\
یاد می‌کن آن نهنگی را که ما را درکشد
&&
تا نماند فهم و وهم و خوب و زشت و خشک و تر
\\
همچو شمع نخل بندان کآتشش در خود کشد
&&
کاغذ پرنقش و صورت درفتد در آب در
\\
\end{longtable}
\end{center}
