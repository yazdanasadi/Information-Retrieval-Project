\begin{center}
\section*{غزل شماره ۱۰۱۶: انا فتحنا عینکم فاستبصروا الغیب البصر}
\label{sec:1016}
\addcontentsline{toc}{section}{\nameref{sec:1016}}
\begin{longtable}{l p{0.5cm} r}
انا فتحنا عینکم فاستبصروا الغیب البصر
&&
انا قضینا بینکم فاستبشروا بالمنتصر
\\
باد صبا ای خوش خبر مژده بیاور دل ببر
&&
جانم فدات ای مژده ور بستان تو جانم ماحضر
\\
شمشیرها جوشن شود ویرانه‌ها گلشن شود
&&
چشم جهان روشن شود چون از تو آید یک نظر
\\
ای قهر بی‌دندان شده وی لطف صد چندان شده
&&
جان و جهان خندان شده چون داد جان‌ها را ظفر
\\
هر کس که دیدت ای ضیا وان حضرت باکبریا
&&
بادا ورا شرم از خدا گر او بلافد از هنر
\\
نگذاشت شیر بیشه‌ای از هست ما یک ریشه‌ای
&&
الا که نیم اندیشه‌ای در روز و شب هجران شمر
\\
ای آفرین بر روی شه کز وی خجل شد روی مه
&&
کوران به دیده گفته خه بشنوده لطفش گوش کر
\\
از عشق آن سلطان من وان دارو و درمان من
&&
کی سیر گردد جان من در جان من جوع البقر
\\
ان کان عیشا قد هجر و اختل عقلی من سهر
&&
والله روحی ما نفر والله روحی ما کفر
\\
من ابروش او ماه وش او روز و من همچو شبش
&&
او جان و من چون قالبش حیران از آن خوبی و فر
\\
آه از دعا بی‌سامعی جرم و گنه بی‌شافعی
&&
درد و الم بی‌نافعی رویم چو زر بی‌سیمبر
\\
کی باشد آن در سفته من الحمدلله گفته من
&&
مستطرب و خوش خفته من در سایه‌های آن شجر
\\
تا دیدمی جانان خود من جویمی درمان خود
&&
که گویمش هجران خود بنمایمش خون جگر
\\
ای گوهر بحر بقا چون حق تو بس پنهان لقا
&&
مخدوم شمس الدین را تبریز شهر و مشتهر
\\
\end{longtable}
\end{center}
