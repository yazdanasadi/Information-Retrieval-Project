\begin{center}
\section*{غزل شماره ۲۴۴۵: از بامدادان ساغری پر کرد خوش خماره‌ای}
\label{sec:2445}
\addcontentsline{toc}{section}{\nameref{sec:2445}}
\begin{longtable}{l p{0.5cm} r}
از بامدادان ساغری پر کرد خوش خماره‌ای
&&
چون فرقدی عرعرقدی شکرلبی مه پاره‌ای
\\
آن نرگس سرمست او و آن طره چون شست او
&&
و آن ساغری در دست او هر چاره بیچاره‌ای
\\
چنگ از شمال و از یمین اندر بر حوران عین
&&
در گلشنی پر یاسمین بر چشمه‌ای فواره‌ای
\\
ای ساقی شیرین صلا جان علی و بوالعلا
&&
بر کف بنه ساغر هلا بر رغم هر غم باره‌ای
\\
چون آفتاب آسمان می‌گرد و جوهر می‌فشان
&&
بر تشنگان و خاکیان در عالم غداره‌ای
\\
ای ساحر و ای ذوفنون ای مایه پنجه جنون
&&
هنگام کار آمد کنون ما هر یکی آن کاره‌ای
\\
چون ساغری پرداختم جامه حیا انداختم
&&
عشقی عجب می‌باختم با غره غراره‌ای
\\
افلاکیان بر آسمان زان بوی باده سرگران
&&
ماه مرا سجده کنان سرمست هر فراره‌ای
\\
انهار باده سو به سو در هر چمن پنجاه جو
&&
بر سنگ زن بشکن سبو بر رغم هر خشم آره‌ای
\\
رحمت به پستی می‌رسد اکسیر هستی می‌رسد
&&
سلطان مستی می‌رسد با لشکر جراره‌ای
\\
خیمه معیشت برکنی آتش به خیمه درزنی
&&
گر از سر بامی کنی در سابقان نظاره‌ای
\\
مستی چو کشتی و عمد هر لحظه کژمژ می‌شود
&&
بر موج‌ها بر می‌زند در قلزمی زخاره‌ای
\\
می‌گویم ای صاحب عمل و ای رسته جانت از علل
&&
چون رستی از حبس اجل بی‌روزن و درساره‌ای
\\
زین عالم تلخ و ترش زین چرخ پیر طفل کش
&&
هم قصه گو و هم خمش هم بنده هم اماره‌ای
\\
گفتا مرا شاه جهان درداد یک ساغر نهان
&&
خود را بدیدم ناگهان در شهر جان سیاره‌ای
\\
پنهان بود بر مرد و زن در رفتن و در آمدن
&&
راه جهان ممتحن از غیرت ستاره‌ای
\\
چون معبرم خیره نگر نی رخنه پیدا و نه در
&&
چون چشمه‌ای برکرده سر بی‌معدنی از خاره‌ای
\\
ای چاشنی شکران درده همان رطل گران
&&
شیرم بده چون مادران بیرون کش از گهواره‌ای
\\
ای ساز و ناز ناکسان حیرت فزای نرگسان
&&
ای خاک را روزی رسان مقصود هر آواره‌ای
\\
زان باده همچون عسس ایمن کن هر دزد و خس
&&
سجده کنانند این نفس هر فکر دل افشاره‌ای
\\
ای جام راح روح جو آسایش مجروح جو
&&
ای ساقی خورشیدرو خون ریز هر استاره‌ای
\\
ای روزی دل‌ها رسان جان کسان و ناکسان
&&
ترکاری و یاغی به سان هموار و ناهمواره‌ای
\\
چون نفخ صوری در صور شورنده حشر و حشر
&&
زنجیر تو چون طوق زر تشریف هر جباره‌ای
\\
بردی ز جان معقول را وین عقل چون معزول را
&&
کردی دماغ گول را از علم تو عیاره‌ای
\\
تا گردن شک می‌زند بر میر و بر بک می‌زند
&&
بر عقل خنبک می‌زند یا بر فن مکاره‌ای
\\
بس کن درآ در انجمن در انخلاق مرد و زن
&&
می‌ساز و صورت می‌شکن در خلوت فخاره‌ای
\\
چون گل سخن گوی و خمش هرگز نباشد روترش
&&
در صدر دل مانند هش بر اوج چون طیاره‌ای
\\
\end{longtable}
\end{center}
