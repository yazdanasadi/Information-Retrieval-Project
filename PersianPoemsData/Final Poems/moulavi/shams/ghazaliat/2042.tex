\begin{center}
\section*{غزل شماره ۲۰۴۲: ای سنگ دل تو جان را دریای پرگهر کن}
\label{sec:2042}
\addcontentsline{toc}{section}{\nameref{sec:2042}}
\begin{longtable}{l p{0.5cm} r}
ای سنگ دل تو جان را دریای پرگهر کن
&&
ای زلف شب مثالش در نیم شب سحر کن
\\
چنگی که زد دل و جان در عشق بانوا کن
&&
نی‌های بی‌زبان را زان شهد پرشکر کن
\\
چون صد هزار در در سمع و بصر تو داری
&&
یک دامنی از آن در در کار کور و کر کن
\\
از خون آن جگرها که بوی عشق دارد
&&
از بهر اهل دل را یک قلیه جگر کن
\\
بس شیوه‌ها که کردند جان‌ها و ره نبردند
&&
ای چاره ساز جان‌ها یک شیوه دگر کن
\\
مرغان آب و گل را پرها به گل فروشد
&&
ای تو همای دولت پر برفشان سفر کن
\\
چون دیو ره بپیما تا بینی آن پری را
&&
و اندر بر چو سیمش تو کار دل چو زر کن
\\
هر چت اشارت آید چون و چرا رها کن
&&
با خوی تند آن مه زنهار سر به سر کن
\\
پای ملخ که جان است چون مور پیش او بر
&&
در پیش آن سلیمان بر هر رهی حشر کن
\\
آبی است تلخ دریا در زیر گنج گوهر
&&
بگذار آب تلخش تو زیر او زبر کن
\\
ماری است مهره دارد زان سوی زهر در سر
&&
ور ز آنک مهره خواهی از زهر او گذر کن
\\
خواهی درخت طوبی نک شمس حق تبریز
&&
خواهی تو عیش باقی در ظل آن شجر کن
\\
\end{longtable}
\end{center}
