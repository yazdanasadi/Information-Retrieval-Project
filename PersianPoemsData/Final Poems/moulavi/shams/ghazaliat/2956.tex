\begin{center}
\section*{غزل شماره ۲۹۵۶: دل را تمام برکن ای جان ز نیک نامی}
\label{sec:2956}
\addcontentsline{toc}{section}{\nameref{sec:2956}}
\begin{longtable}{l p{0.5cm} r}
دل را تمام برکن ای جان ز نیک نامی
&&
تا یک به یک بدانی اسرار را تمامی
\\
ای عاشق الهی ناموس خلق خواهی
&&
ناموس و پادشاهی در عشق هست خامی
\\
عاشق چو قند باید بی‌چون و چند باید
&&
جانی بلند باید کان حضرتی است سامی
\\
هستی تو از سر و بن در چشم خویش ناخن
&&
زنار روم گم کن در عشق زلف شامی
\\
در عشق علم جهل است ناموس علم سهل است
&&
نادان علم اهل است دانای علم عامی
\\
از کوی بی‌نشانش زان سوی جهل و دانش
&&
وز جان جان جانش عشق آمدت سلامی
\\
بر بام عشق بی‌تن دیدم چو ماه روشن
&&
بر در بمانده‌ام من زان شیوه‌های بامی
\\
گر مست و گر میم من نی از دف و نیم من
&&
از شیوه ویم من مست شراب جامی
\\
آن چهره چو آتش در زیر زلف دلکش
&&
گردن ببسته جان خوش در حلقه‌های دامی
\\
گوید غمت ز تیزی وقتی که خون تو ریزی
&&
کای دل تو خود چه چیزی وی جان تو خود کدامی
\\
ای جان شبی که زادی آن شب سری نهادی
&&
دادی تو آنچ دادی وز جان مطیع و رامی
\\
ای روح برپریدی بر ساحلی چریدی
&&
دل دادی و خریدی آن را که تش غلامی
\\
گر رند و گر قلاشی ما را تو خواجه تاشی
&&
ای شمس هر طواشی تبریز را نظامی
\\
\end{longtable}
\end{center}
