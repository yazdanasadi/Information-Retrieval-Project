\begin{center}
\section*{غزل شماره ۲۹۹۰: جان خاک آن مهی که خداش است مشتری}
\label{sec:2990}
\addcontentsline{toc}{section}{\nameref{sec:2990}}
\begin{longtable}{l p{0.5cm} r}
جان خاک آن مهی که خداش است مشتری
&&
آن کس ملک ندید و نه انسان و نی پری
\\
چون از خودی برون شد او آدمی نماند
&&
او راست چشم روشن و گوش پیمبری
\\
تا آدمی است آدمی و تا ملک ملک
&&
بسته‌ست چشم هر دو از آن جان و دلبری
\\
عالم به حکم او است مر او را چه فخر از این
&&
چون آن او است خالق عالم به یک سوی
\\
بحری که کمترین شبه را گوهری کند
&&
حاشا از او که لاف برآرد ز گوهری
\\
آن ذره است لایق رقص چنان شعاع
&&
کو گشت از هزار چو خورشید و مه بری
\\
آن ذره‌ای که گر قدمش بوسد آفتاب
&&
خود ننگرد به تابش او جز که سرسری
\\
بنما مها به کوری خورشید تابشی
&&
تا زین سپس زنخ نزند از منوری
\\
درتاب شاه و مفخر تبریز شمس دین
&&
تا هر دو کون پر شود از نور داوری
\\
\end{longtable}
\end{center}
