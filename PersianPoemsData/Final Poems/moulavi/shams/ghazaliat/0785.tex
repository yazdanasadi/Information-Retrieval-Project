\begin{center}
\section*{غزل شماره ۷۸۵: ما نه زان محتشمانیم که ساغر گیرند}
\label{sec:0785}
\addcontentsline{toc}{section}{\nameref{sec:0785}}
\begin{longtable}{l p{0.5cm} r}
ما نه زان محتشمانیم که ساغر گیرند
&&
و نه زان مفلسکان که بز لاغر گیرند
\\
ما از آن سوختگانیم که از لذت سوز
&&
آب حیوان بهلند و پی آذر گیرند
\\
چو مه از روزن هر خانه که اندرتابیم
&&
از ضیا شب صفتان جمله ره در گیرند
\\
ناامیدان که فلک ساغر ایشان بشکست
&&
چو ببینند رخ ما طرب از سر گیرند
\\
آنک زین جرعه کشد جمله جهانش نکشد
&&
مگر او را به گلیم از بر ما برگیرند
\\
هر کی او گرم شد این جا نشود غره کس
&&
اگرش سردمزاجان همه در زر گیرند
\\
در فروبند و بده باده که آن وقت رسید
&&
زردرویان تو را که می احمر گیرند
\\
به یکی دست می خالص ایمان نوشند
&&
به یکی دست دگر پرچم کافر گیرند
\\
آب ماییم به هر جا که بگردد چرخی
&&
عود ماییم به هر سور که مجمر گیرند
\\
پس این پرده ازرق صنمی مه روییست
&&
که ز نور رخش انجم همه زیور گیرند
\\
ز احتراقات و ز تربیع و نحوست برهند
&&
اگر او را سحری گوشه چادر گیرند
\\
تو دورای و دودلی و دل صاف آن‌ها راست
&&
که دل خود بهلند و دل دلبر گیرند
\\
خمش ای عقل عطارد که در این مجلس عشق
&&
حلقه زهره بیانت همه تسخر گیرند
\\
\end{longtable}
\end{center}
