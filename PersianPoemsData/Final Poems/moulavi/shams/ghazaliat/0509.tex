\begin{center}
\section*{غزل شماره ۵۰۹: مرغ دلم باز پریدن گرفت}
\label{sec:0509}
\addcontentsline{toc}{section}{\nameref{sec:0509}}
\begin{longtable}{l p{0.5cm} r}
مرغ دلم باز پریدن گرفت
&&
طوطی جان قند چریدن گرفت
\\
اشتر دیوانه سرمست من
&&
سلسله عقل دریدن گرفت
\\
جرعه آن باده بی‌زینهار
&&
بر سر و بر دیده دویدن گرفت
\\
شیر نظر با سگ اصحاب کهف
&&
خون مرا باز خوریدن گرفت
\\
باز در این جوی روان گشت آب
&&
بر لب جو سبزه دمیدن گرفت
\\
باد صبا باز وزان شد به باغ
&&
بر گل و گلزار وزیدن گرفت
\\
عشق فروشید به عیبی مرا
&&
سوخت دلش بازخریدن گرفت
\\
راند مرا رحمتش آمد بخواند
&&
جانب ما خوش نگریدن گرفت
\\
دشمن من دید که با دوستم
&&
او ز حسد دست گزیدن گرفت
\\
دل برهید از دغل روزگار
&&
در بغل عشق خزیدن گرفت
\\
ابروی غماز اشارت کنان
&&
جانب آن چشم خمیدن گرفت
\\
عشق چو دل را به سوی خویش خواند
&&
دل ز همه خلق رمیدن گرفت
\\
خلق عصااند عصا را فکند
&&
قبضه هر کور که دیدن گرفت
\\
خلق چو شیرند رها کرد شیر
&&
طفل که او لوت کشیدن گرفت
\\
روح چو بازیست که پران شود
&&
کز سوی شه طبل شنیدن گرفت
\\
بس کن زیرا که حجاب سخن
&&
پرده به گرد تو تنیدن گرفت
\\
\end{longtable}
\end{center}
