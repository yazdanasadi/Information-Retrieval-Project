\begin{center}
\section*{غزل شماره ۱۲۴۳: اندرآ ای اصل اصل شادمانی شاد باش}
\label{sec:1243}
\addcontentsline{toc}{section}{\nameref{sec:1243}}
\begin{longtable}{l p{0.5cm} r}
اندرآ ای اصل اصل شادمانی شاد باش
&&
اندرآ ای آب آب زندگانی شاد باش
\\
گرت بیند زندگانی تا ابد باقی شود
&&
ورت بیند مرده هم داند که جانی شاد باش
\\
همچنین تو دم به دم آن جام باقی می‌رسان
&&
تا شویم از دست و آن باقی تو دانی شاد باش
\\
بر نشانه خاک ما اینک نشان زخم تو
&&
ای نشانه شاد زی و ای نشانی شاد باش
\\
ای هما کز سایه‌ات پر یافت کوه قاف نیز
&&
ای همای خوش لقای آن جهانی شاد باش
\\
هم ظریفی هم حریفی هم چراغی هم شراب
&&
هم جهانی هم نهانی هم عیانی شاد باش
\\
تحفه‌های آن جهانی می‌رسانی دم به دم
&&
می‌رسان و می‌رسان خوش می‌رسانی شاد باش
\\
رخت‌ها را می‌کشاند جان مستان سوی تو
&&
می‌چشان و می‌کشان خوش می‌کشانی شاد باش
\\
ای جهان را شاد کرده وی زمین را جمله گنج
&&
تا زمین گوید تو را کای آسمانی شاد باش
\\
گر سر خوبی بخارد دلبری در عهد تو
&&
پرچمش آرند پیشت ارمغانی شاد باش
\\
گوهر آدم به عالم شمس تبریزی تویی
&&
ای ز تو حیران شده بحر معانی شاد باش
\\
\end{longtable}
\end{center}
