\begin{center}
\section*{غزل شماره ۱۲۳۴: قضا آمد شنو طبل نفیرش}
\label{sec:1234}
\addcontentsline{toc}{section}{\nameref{sec:1234}}
\begin{longtable}{l p{0.5cm} r}
قضا آمد شنو طبل نفیرش
&&
نفیرش تلختر یا زخم تیرش
\\
چو دایه این جهان پستان سیه کرد
&&
گلوگیر آمدت چون شهد شیرش
\\
خنک طفلی که دندان خرد یافت
&&
رهد زین دایه و شیر و زحیرش
\\
بشارت‌های غیبی شد غذااش
&&
ز شیرش وارهانید از بشیرش
\\
چو هر دم می‌رسد تلقین عشقش
&&
چه غم دارد ز منکر یا نکیرش
\\
چو آن خورشید بر وی سایه انداخت
&&
ز دوزخ ایمنست و زمهریرش
\\
به اقبال جوان واگشت جانی
&&
که راه دین نزد این چرخ پیرش
\\
بدان دارالامان و اصل خود رفت
&&
رهید از دامگاه و دار و گیرش
\\
رهید از بند شحنه حرص و آزی
&&
که کرده بود بیچاره و حقیرش
\\
رو ای جان کز رباط کهنه جستی
&&
ز غصه آجر و حجره و حصیرش
\\
نثارش آید از رضوان جنت
&&
کنارش گیرد آن بدر منیرش
\\
تماشا یافت آن چشم عفیفش
&&
سعادت یافت آن نفس فقیرش
\\
خجسته باد باغستان خلدش
&&
مبارک باد آن نعم المصیرش
\\
\end{longtable}
\end{center}
