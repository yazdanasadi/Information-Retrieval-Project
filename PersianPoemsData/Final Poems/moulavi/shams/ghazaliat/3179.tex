\begin{center}
\section*{غزل شماره ۳۱۷۹: این طریق دارهم یا سندی و سیدی}
\label{sec:3179}
\addcontentsline{toc}{section}{\nameref{sec:3179}}
\begin{longtable}{l p{0.5cm} r}
این طریق دارهم یا سندی و سیدی
&&
اهد الی وصالهم، ذبت من‌التباعد
\\
ای که به قصد نیمشب بسته نقاب آمدی
&&
آن همه حسن و نیکوی نست مناسب بدی
\\
یافئتی فدیتکم فی امل اتیتکم
&&
قد قطعت وسایلی حیلة قول حاسد
\\
جان شهان و حاجبان! چشم و چراغ طالبان
&&
بی‌تو ز جان و جا شدم، تو ز برم کجا شدی؟
\\
یا ملک الا یا من، یا شرف الاماکن
&&
جتک کی تعیذنی، سطوة کل معتدی
\\
یار سرور و دولتم، خواجهٔ هر سعادتم
&&
لیک تو با همه جفا خوشتر ازین همه بدی
\\
رحمتکم محیطة، رافتکم بسیطة
&&
سادتنا، تقبلو توبة کل عابد
\\
مست میی نمی‌شوم، جز ز شراب اولین
&&
ده قدحی، چه کم شود از خم فضل ایزدی؟
\\
طلعتکم بدورنا، بهجتنا و نورنا
&&
ظل خیال طیفکم دولة کل ماجد
\\
ای دل خسته هان و هان، تا نرمی ز سرخوشان
&&
پا نکشی ز عاشقان، ورنه جهود و مرتدی
\\
قبلتنا خیالهم لذتنا دلالهم
&&
یا سندی، جمالهم فتنة کل زاهد
\\
قدر وصالشان بدان یاد کن، آنک پیش ازین
&&
همچو زنان تعزیت بر سر و رو همی زدی
\\
خادعنی و غرنی، هیجنی و جرنی
&&
نور هلال وصلکم من افق مشید
\\
ای دل مست جست‌وجو، صورت عشق را بگو
&&
«بر دو جهان خروج کن، هرچه کنی میدی »
\\
\end{longtable}
\end{center}
