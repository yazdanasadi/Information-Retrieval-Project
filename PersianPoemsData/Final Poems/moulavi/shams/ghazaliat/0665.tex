\begin{center}
\section*{غزل شماره ۶۶۵: خنک جانی که او یاری پسندد}
\label{sec:0665}
\addcontentsline{toc}{section}{\nameref{sec:0665}}
\begin{longtable}{l p{0.5cm} r}
خنک جانی که او یاری پسندد
&&
کز او دوریش خود صورت نبندد
\\
تو باشی خنده و یار تو شادی
&&
که بی‌شادی دهان کس نخندد
\\
تو باشی سجده و یار تو تعظیم
&&
که بی‌تعظیم هرگز سر نخنبد
\\
تو باشی چون صدا و یار غارت
&&
چو آوازی به نزد کوه و گنبد
\\
تو آدینه بوی او وقت خطبه
&&
نه ز آدینه جدا چون روز شنبد
\\
نگر آخر دمی در نحن اقرب
&&
نظر را تا نجنباند نجنبد
\\
خیالی خوش دهد دل زان بنازد
&&
خیالی زشت آرد دل بتندد
\\
بر او مسخره آمد دل و جان
&&
گه از صله گه از سیلیش رندد
\\
مزن سیلی چنانک گیج گردم
&&
ز گیجی دور افتم ز اصل و مسند
\\
خمش تا درس گوید آن زبانی
&&
که لا باشد به پیشش صد مهند
\\
اگر گویی تو نی را هی خمش کن
&&
بگوید با لبش گو ای مؤید
\\
\end{longtable}
\end{center}
