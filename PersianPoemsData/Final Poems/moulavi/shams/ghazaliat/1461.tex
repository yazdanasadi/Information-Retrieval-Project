\begin{center}
\section*{غزل شماره ۱۴۶۱: پایی به میان درنه تا عیش ز سر گیرم}
\label{sec:1461}
\addcontentsline{toc}{section}{\nameref{sec:1461}}
\begin{longtable}{l p{0.5cm} r}
پایی به میان درنه تا عیش ز سر گیرم
&&
تو تلخ مشو با من تا تنگ شکر گیرم
\\
بی‌رنگ فرورفتم در عشق تو ای دلبر
&&
برکش تو از این خنبم تا رنگ دگر گیرم
\\
دلتنگتر از میمم چون در طمع و بیمم
&&
من قرص به دو نیمم چون شکل قمر گیرم
\\
ای از رخ شاه جان صد بیذق را سلطان
&&
بر اسب نشین ای جان تا غاشیه برگیرم
\\
وز باد لجاج خود وز غصه نیک و بد
&&
هر چند بدم در خود والله که بتر گیرم
\\
امنی است مرا از تو امنم تویی ای مه رو
&&
یا امن دهم زین سو یا راه خطر گیرم
\\
چون سرو خمید از من گلزار چرید از من
&&
ایمان چو رمید از من ترسم که کفر گیرم
\\
تو غمزه غمازی از تیر سپر سازی
&&
چون تیر تو اندازی پس من چه سپر گیرم
\\
زیر و زبر عشقم شمس الحق تبریز است
&&
جان را ز پی عشقش من زیر و زبر گیرم
\\
\end{longtable}
\end{center}
