\begin{center}
\section*{غزل شماره ۱۳۹۳: مرده بدم زنده شدم گریه بدم خنده شدم}
\label{sec:1393}
\addcontentsline{toc}{section}{\nameref{sec:1393}}
\begin{longtable}{l p{0.5cm} r}
مرده بدم زنده شدم گریه بدم خنده شدم
&&
دولت عشق آمد و من دولت پاینده شدم
\\
دیده سیر است مرا جان دلیر است مرا
&&
زهره شیر است مرا زهره تابنده شدم
\\
گفت که دیوانه نه‌ای لایق این خانه نه‌ای
&&
رفتم دیوانه شدم سلسله بندنده شدم
\\
گفت که سرمست نه‌ای رو که از این دست نه‌ای
&&
رفتم و سرمست شدم وز طرب آکنده شدم
\\
گفت که تو کشته نه‌ای در طرب آغشته نه‌ای
&&
پیش رخ زنده کنش کشته و افکنده شدم
\\
گفت که تو زیرککی مست خیالی و شکی
&&
گول شدم هول شدم وز همه برکنده شدم
\\
گفت که تو شمع شدی قبله این جمع شدی
&&
جمع نیم شمع نیم دود پراکنده شدم
\\
گفت که شیخی و سری پیش رو و راهبری
&&
شیخ نیم پیش نیم امر تو را بنده شدم
\\
گفت که با بال و پری من پر و بالت ندهم
&&
در هوس بال و پرش بی‌پر و پرکنده شدم
\\
گفت مرا دولت نو راه مرو رنجه مشو
&&
زانک من از لطف و کرم سوی تو آینده شدم
\\
گفت مرا عشق کهن از بر ما نقل مکن
&&
گفتم آری نکنم ساکن و باشنده شدم
\\
چشمه خورشید تویی سایه گه بید منم
&&
چونک زدی بر سر من پست و گدازنده شدم
\\
تابش جان یافت دلم وا شد و بشکافت دلم
&&
اطلس نو بافت دلم دشمن این ژنده شدم
\\
صورت جان وقت سحر لاف همی‌زد ز بطر
&&
بنده و خربنده بدم شاه و خداونده شدم
\\
شکر کند کاغذ تو از شکر بی‌حد تو
&&
کآمد او در بر من با وی ماننده شدم
\\
شکر کند خاک دژم از فلک و چرخ به خم
&&
کز نظر وگردش او نورپذیرنده شدم
\\
شکر کند چرخ فلک از ملک و ملک و ملک
&&
کز کرم و بخشش او روشن بخشنده شدم
\\
شکر کند عارف حق کز همه بردیم سبق
&&
بر زبر هفت طبق اختر رخشنده شدم
\\
زهره بدم ماه شدم چرخ دو صد تاه شدم
&&
یوسف بودم ز کنون یوسف زاینده شدم
\\
از توام ای شهره قمر در من و در خود بنگر
&&
کز اثر خنده تو گلشن خندنده شدم
\\
باش چو شطرنج روان خامش و خود جمله زبان
&&
کز رخ آن شاه جهان فرخ و فرخنده شدم
\\
\end{longtable}
\end{center}
