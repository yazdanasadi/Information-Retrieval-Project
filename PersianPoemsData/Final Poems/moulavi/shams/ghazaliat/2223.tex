\begin{center}
\section*{غزل شماره ۲۲۲۳: ای همه سرگشتگان مهمان تو}
\label{sec:2223}
\addcontentsline{toc}{section}{\nameref{sec:2223}}
\begin{longtable}{l p{0.5cm} r}
ای همه سرگشتگان مهمان تو
&&
آفتاب از آسمان پرسان تو
\\
چشم بد از روی خوبت دور باد
&&
ای هزاران جان فدای جان تو
\\
چون فدا گردند جاویدان شوند
&&
ز آنک اکسیر است جان را کان تو
\\
گاو و بزغاله و بره گردون چرخ
&&
باد ای ماه بتان قربان تو
\\
ز آنک قربان‌ها همه باقی شوند
&&
در هوای عید بی‌پایان تو
\\
در سرای عصمت یزدان تویی
&&
بخت و دولت روز و شب دربان تو
\\
ای خدا این باغ را سرسبز دار
&&
در بهارستان بی‌نقصان تو
\\
تا ملایک میوه از وی می‌کشند
&&
می‌چرند از نخل و سیبستان تو
\\
این شکرخانه همیشه باز باد
&&
پرنبات و شکر پنهان تو
\\
آب این جو ای خدا تیره مباد
&&
تا به هر سو می‌رود ز احسان تو
\\
این دعا را یا رب آمین هم تو کن
&&
ای دعا آن تو آمین آن تو
\\
چنگ و قانون جهان را تارهاست
&&
ناله هر تار در فرمان تو
\\
من بخفتم تو مرا انگیختی
&&
تا چو گویم در خم چوگان تو
\\
ور نه خاکی از کجا عشق از کجا
&&
گر نبودی جذبه‌های جان تو
\\
خاک خشکی مست شد تر می‌زند
&&
آن توست این آن توست این آن تو
\\
دی مرا پرسید لطفش کیستی
&&
گفتم ای جان گربه در انبان تو
\\
گفت ای گربه بشارت مر تو را
&&
که تو را شیری کند سلطان تو
\\
من خمش کردم توام نگذاشتی
&&
همچو چنگم سخره افغان تو
\\
\end{longtable}
\end{center}
