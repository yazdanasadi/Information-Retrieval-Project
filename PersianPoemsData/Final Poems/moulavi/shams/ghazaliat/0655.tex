\begin{center}
\section*{غزل شماره ۶۵۵: از بهر خدا عشق دگر یار مدارید}
\label{sec:0655}
\addcontentsline{toc}{section}{\nameref{sec:0655}}
\begin{longtable}{l p{0.5cm} r}
از بهر خدا عشق دگر یار مدارید
&&
در مجلس جان فکر دگر کار مدارید
\\
یار دگر و کار دگر کفر و محالست
&&
در مجلس دین مذهب کفار مدارید
\\
در مجلس جان فکر چنانست که گفتار
&&
پنهان چو نمی‌ماند اضمار مدارید
\\
گر بانگ نیاید ز فسا بوی بیاید
&&
در دل نظر فاحشه آثار مدارید
\\
آن حارس دل مشرف جان سخت غیورست
&&
با غیرت او رو سوی اغیار مدارید
\\
هر وسوسه را بحث و تفکر بمخوانید
&&
هر گمشده را سرور و سالار مدارید
\\
یاقوت کرم قوت شما بازنگیرد
&&
خود را گرو نفس علف خوار مدارید
\\
العزه لله جمیعا چو شنیدیت
&&
خاطر به سوی سبلت و دستار مدارید
\\
چون اول خط نقطه بد و آخر نقطه
&&
خود را تبع گردش پرگار مدارید
\\
در مشهد اعظم به تشهد بنشینید
&&
هش را به سوی گنبد دوار مدارید
\\
انکار بسوزد چو شهادت بفروزد
&&
با شاهد حق نکرت انکار مدارید
\\
یک نیم جهان کرکس و نیمیش چو مردار
&&
هین چشم چو کرکس سوی مردار مدارید
\\
آن نفس فریبنده که غرست و غرورست
&&
هین عشق بر آن غره غرار مدارید
\\
گه زلف برافشاند و گه جیب گشاید
&&
گلگونه او را به جز از خار مدارید
\\
او یار وفا نبود و از یار ببرد
&&
آن ده دله را محرم اسرار مدارید
\\
او باده بریزد عوضش سرکه فروشد
&&
آن حامضه را ساقی و خمار مدارید
\\
ما حلقه مستان خوش ساقی خویشیم
&&
ما را سقط و بارد و هشیار مدارید
\\
گر ناف دهی پشک فروشد عوض مشک
&&
آن ناف ورا نافه تاتار مدارید
\\
چون روح برآمد به سر منبر تذکیر
&&
خود را سپس پرده گفتار مدارید
\\
\end{longtable}
\end{center}
