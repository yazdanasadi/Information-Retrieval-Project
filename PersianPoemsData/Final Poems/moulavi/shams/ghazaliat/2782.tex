\begin{center}
\section*{غزل شماره ۲۷۸۲: گر شراب عشق کار جان حیوانیستی}
\label{sec:2782}
\addcontentsline{toc}{section}{\nameref{sec:2782}}
\begin{longtable}{l p{0.5cm} r}
گر شراب عشق کار جان حیوانیستی
&&
عشق شمس الدین به عالم فاش و یک سانیستی
\\
گر نه در انوار غیرت غرق بودی عشق او
&&
حلقه گوش روان و جان انسانیستی
\\
گر نبودی بزم شمس الدین برون از هر دو کون
&&
جام او بر خاک همچون ابر نیسانیستی
\\
ابر نیسان خود چه باشد نزد بحر فضل او
&&
قاف تا قاف از میش خود موج طوفانیستی
\\
آفتاب و ماه را خود کی بدی زهره شعاع
&&
گر نه در رشک خدا سیماش پنهانیستی
\\
گر جمالش ماجرا کردی میان یوسفان
&&
یوسف مصری ابد پابند و زندانیستی
\\
گر نه از لطفش بپرهیزیدمی من گفتمی
&&
کز بهشت لطف او فردوس ریحانیستی
\\
نفس سگ دندان برآوردی گزیدی پای جان
&&
ساقیا گر نه می سرتیز دندانیستی
\\
جام همچون شمع را بر آتش می برفروز
&&
پس بسوز این عقل را گر بیت احزانیستی
\\
درکش آن معشوقه بدمست را در بزم ما
&&
کو ز مکر و عشوه‌ها گوییی که دستانیستی
\\
پس ز جام شمس تبریزی بده یک جرعه‌ای
&&
بعد از آن مر عاشقان را وقت حیرانیستی
\\
\end{longtable}
\end{center}
