\begin{center}
\section*{غزل شماره ۱۶۶۵: عاشقی بر من پریشانت کنم}
\label{sec:1665}
\addcontentsline{toc}{section}{\nameref{sec:1665}}
\begin{longtable}{l p{0.5cm} r}
عاشقی بر من پریشانت کنم
&&
کم عمارت کن که ویرانت کنم
\\
گر دو صد خانه کنی زنبوروار
&&
چون مگس بی‌خان و بی‌مانت کنم
\\
تو بر آنک خلق را حیران کنی
&&
من بر آنک مست و حیرانت کنم
\\
گر که قافی تو را چون آسیا
&&
آرم اندر چرخ و گردانت کنم
\\
ور تو افلاطون و لقمانی به علم
&&
من به یک دیدار نادانت کنم
\\
تو به دست من چو مرغی مرده‌ای
&&
من صیادم دام مرغانت کنم
\\
بر سر گنجی چو ماری خفته‌ای
&&
من چو مار خسته پیچانت کنم
\\
خواه دلیلی گو و خواهی خود مگو
&&
در دلالت عین برهانت کنم
\\
خواه گو لاحول خواهی خود مگو
&&
چون شهت لاحول شیطانت کنم
\\
چند می باشی اسیر این و آن
&&
گر برون آیی از این آنت کنم
\\
ای صدف چون آمدی در بحر ما
&&
چون صدف‌ها گوهرافشانت کنم
\\
بر گلویت تیغ‌ها را دست نیست
&&
گر چو اسماعیل قربانت کنم
\\
چون خلیلی هیچ از آتش مترس
&&
من ز آتش صد گلستانت کنم
\\
دامن ما گیر اگر تردامنی
&&
تا چو مه از نور دامانت کنم
\\
من همایم سایه کردم بر سرت
&&
تا که افریدون و سلطانت کنم
\\
هین قرائت کم کن و خاموش باش
&&
تا بخوانم عین قرآنت کنم
\\
\end{longtable}
\end{center}
