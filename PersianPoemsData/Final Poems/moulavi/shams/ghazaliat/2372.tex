\begin{center}
\section*{غزل شماره ۲۳۷۲: هله بحری شو و در رو مکن از دور نظاره}
\label{sec:2372}
\addcontentsline{toc}{section}{\nameref{sec:2372}}
\begin{longtable}{l p{0.5cm} r}
هله بحری شو و در رو مکن از دور نظاره
&&
که بود در تک دریا کف دریا به کناره
\\
چو رخ شاه بدیدی برو از خانه چو بیذق
&&
رخ خورشید چو دیدی هله گم شو چو ستاره
\\
چو بدان بنده نوازی شده‌ای پاک و نمازی
&&
همگان را تو صلا گو چو مؤذن ز مناره
\\
تو در این ماه نظر کن که دلت روشن از او شد
&&
تو در این شاه نگه کن که رسیده‌ست سواره
\\
نه بترسم نه بلرزم چو کشد خنجر عزت
&&
به خدا خنجر او را بدهم رشوت و پاره
\\
کی بود آب که دارد به لطافت صفت او
&&
که دو صد چشمه برآرد ز دل مرمر و خاره
\\
تو همه روز برقصی پی تتماج و حریره
&&
تو چه دانی هوس دل پی این بیت و حراره
\\
چو بدیدم بر سیمش ز زر و سیم نفورم
&&
که نفور است نسیمش ز کف سیم شماره
\\
تو از آن بار نداری که سبکسار چو بیدی
&&
تو از آن کار نداری که شدستی همه کاره
\\
همه حجاج برفته حرم و کعبه بدیده
&&
تو شتر هم نخریده که شکسته‌ست مهاره
\\
بنگر سوی حریفان که همه مست و خرابند
&&
تو خمش باش و چنان شو هله ای عربده باره
\\
\end{longtable}
\end{center}
