\begin{center}
\section*{غزل شماره ۱۰۷۷: از کنار خویش یابم هر دمی من بوی یار}
\label{sec:1077}
\addcontentsline{toc}{section}{\nameref{sec:1077}}
\begin{longtable}{l p{0.5cm} r}
از کنار خویش یابم هر دمی من بوی یار
&&
چون نگیرم خویش را من هر شبی اندر کنار
\\
دوش باغ عشق بودم آن هوس بر سر دوید
&&
مهر او از دیده برزد تا روان شد جویبار
\\
هر گل خندان که رویید از لب آن جوی مهر
&&
رسته بود از خار هستی جسته بود از ذوالفقار
\\
هر درخت و هر گیاهی در چمن رقصان شده
&&
لیک اندر چشم عامه بسته بود و برقرار
\\
ناگهان اندررسید از یک طرف آن سرو ما
&&
تا که بیخود گشت باغ و دست بر هم زد چنار
\\
رو چو آتش می‌چو آتش عشق آتش هر سه خوش
&&
جان ز آتش‌های درهم پرفغان این الفرار
\\
در جهان وحدت حق این عدد را گنج نیست
&&
وین عدد هست از ضرورت در جهان پنج و چار
\\
صد هزاران سیب شیرین بشمری در دست خویش
&&
گر یکی خواهی که گردد جمله را در هم فشار
\\
صد هزاران دانه انگور از حجاب پوست شد
&&
چون نماند پوست ماند باده‌های شهریار
\\
بی‌شمار حرف‌ها این نطق در دل بین که چیست
&&
ساده رنگی نیست شکلی آمده از اصل کار
\\
شمس تبریزی نشسته شاهوار و پیش او
&&
شعر من صف‌ها زده چون بندگان اختیار
\\
\end{longtable}
\end{center}
