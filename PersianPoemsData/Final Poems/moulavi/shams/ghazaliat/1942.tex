\begin{center}
\section*{غزل شماره ۱۹۴۲: من ز گوش او بدزدم حلقه دیگر نهان}
\label{sec:1942}
\addcontentsline{toc}{section}{\nameref{sec:1942}}
\begin{longtable}{l p{0.5cm} r}
من ز گوش او بدزدم حلقه دیگر نهان
&&
تا نداند چشم دشمن ور بداند گو بدان
\\
بر رخم خطی نبشت و من نهان می داشتم
&&
زین سپس پنهان ندارم هر کی خواند گو بخوان
\\
طوق زر عشق او هم لایق این گردن است
&&
بشکند از طوق عشقش گردن گردن کشان
\\
کوس محمودی همه بر اشتر محمود باد
&&
بار دل هم دل کشد محرم کجا باشد زبان
\\
آینه آهن دلی باید که تا زخمش کشد
&&
زخم آیینه نباشد درخور آیینه دان
\\
لیک روی دوست بینی بی‌خبر باشی ز زخم
&&
چون زنان مصر بیخود در جمال یوسفان
\\
صد هزاران حسن یوسف در جمال روی کیست
&&
شمس تبریزی ما آن خوش نشین خوش نشان
\\
\end{longtable}
\end{center}
