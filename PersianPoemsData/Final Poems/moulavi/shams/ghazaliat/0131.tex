\begin{center}
\section*{غزل شماره ۱۳۱: من چو موسی در زمان آتش شوق و لقا}
\label{sec:0131}
\addcontentsline{toc}{section}{\nameref{sec:0131}}
\begin{longtable}{l p{0.5cm} r}
من چو موسی در زمان آتش شوق و لقا
&&
سوی کوه طور رفتم حبذا لی حبذا
\\
دیدم آن جا پادشاهی خسروی جان پروری
&&
دلربایی جان فزایی بس لطیف و خوش لقا
\\
کوه طور و دشت و صحرا از فروغ نور او
&&
چون بهشت جاودانی گشته از فر و ضیا
\\
ساقیان سیمبر را جام زرین‌ها به کف
&&
رویشان چون ماه تابان پیش آن سلطان ما
\\
روی‌های زعفران را از جمالش تاب‌ها
&&
چشم‌های محرمان را از غبارش توتیا
\\
از نوای عشق او آن جا زمین در جوش بود
&&
وز هوای وصل او در چرخ دایم شد سما
\\
در فنا چون بنگرید آن شاه شاهان یک نظر
&&
پای همت را فنا بنهاد بر فرق بقا
\\
مطرب آن جا پرده‌ها بر هم زند خود نور او
&&
کی گذارد در دو عالم پرده‌ای را در هوا
\\
جمع گشته سایه الطاف با خورشید فضل
&&
جمع اضداد از کمال عشق او گشته روا
\\
چون نقاب از روی او باد صبا اندرربود
&&
محو گشت آن جا خیال جمله شان و شد هبا
\\
لیک اندر محو هستیشان یکی صد گشته بود
&&
هست محو و محو هست آن جا بدید آمد مرا
\\
تا بدیدم از ورای آن جهان جان صفت
&&
ذره‌ها اندر هوایش از وفا و از صفا
\\
بس خجل گشتم ز رویش آن زمان تا لاجرم
&&
هر زمان زنار می‌ببریدم از جور و جفا
\\
گفتم ای مه توبه کردم توبه‌ها را رد مکن
&&
گفت بس راهست پیشت تا ببینی توبه را
\\
صادق آمد گفت او وز ماه دور افتاده‌ام
&&
چون حجاج گمشده اندر مغیلان فنا
\\
نور آن مه چون سهیل و شهر تبریز آن یمن
&&
این یکی رمزی بود از شاه ما صدرالعلا
\\
\end{longtable}
\end{center}
