\begin{center}
\section*{غزل شماره ۱۸۷۱: دو چیز نخواهد بد در هر دو جهان می دان}
\label{sec:1871}
\addcontentsline{toc}{section}{\nameref{sec:1871}}
\begin{longtable}{l p{0.5cm} r}
دو چیز نخواهد بد در هر دو جهان می دان
&&
از عاشق حق توبه وز باد هوا انبان
\\
گر توبه شود دریا یک قطره نیابم من
&&
ور خاک درآیم من آن خاک شود سوزان
\\
در خاک تنم بنگر کز جان هواپیشه
&&
هر ذره در این سودا گشته‌ست چو دل گردان
\\
خاصیت من این است هر جا که روم اینم
&&
چه دوزد پالان گر هر جا که رود پالان
\\
گویند که هر کی هست در گور اسیر آید
&&
در حقه تنگ آن مشک نگذارد مشک افشان
\\
در سینه تاریکت دل را چه بود شادی
&&
زندان نبود سینه میدان بود آن میدان
\\
اندر رحم مادر چون طفل طرب یابد
&&
آن خون به از این باده وان جا به از این بستان
\\
گر شرح کنم این را ترسم که مقلد را
&&
آید به خیال اندر اندیشه سرگردان
\\
\end{longtable}
\end{center}
