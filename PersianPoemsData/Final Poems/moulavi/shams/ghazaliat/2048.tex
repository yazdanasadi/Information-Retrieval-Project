\begin{center}
\section*{غزل شماره ۲۰۴۸: آن کیست ای خدای کز این دام خامشان}
\label{sec:2048}
\addcontentsline{toc}{section}{\nameref{sec:2048}}
\begin{longtable}{l p{0.5cm} r}
آن کیست ای خدای کز این دام خامشان
&&
ما را همی‌کشد به سوی خود کشان کشان
\\
ای آنک می‌کشی تو گریبان جان ما
&&
از جمع سرکشان به سوی جمع سرخوشان
\\
بگرفته گوش ما و بسوزیده هوش ما
&&
ساقی باهشانی و آرام بی‌هشان
\\
بی‌دست می‌کشی تو و بی‌تیغ می‌کشی
&&
شاگرد چشم تو نظر بی‌گنه کشان
\\
آب حیات نزل شهیدان عشق توست
&&
این تشنه کشتگان را ز آن نزل می‌چشان
\\
دل را گره گشای نسیم وصال توست
&&
شاخ امید را به نسیمی همی‌فشان
\\
خود حسن ساکن است و مقیم اندر آن وجود
&&
زان ساکنند زیر و زبر این مفتشان
\\
مقصود ره روان همه دیدار ساکنان
&&
مقصود ناطقان همه اصغای خامشان
\\
آتش در آب گشته نهان وقت جوش آب
&&
چون آب آتش آمد الغوث ز آتشان
\\
در روح دررسی چو گذشتی ز نقش‌ها
&&
وز چرخ بگذری چو گذشتی ز مه وشان
\\
همیان چه می‌نهی به امانت به مفلسان
&&
پا را چه می‌نهی تو به دندان گربشان
\\
از نو چو میر گولان بستد کلاه و کفش
&&
خواهی تو روستایی خواهی ز اکدشان
\\
دانش سلاح توست و سلاح از نشان مرد
&&
مردی چو نیست به که نباشد تو را نشان
\\
دیگر مگو سخن که سخن ریگ آب توست
&&
خورشید را نگر چو نه‌ای جنس اعمشان
\\
\end{longtable}
\end{center}
