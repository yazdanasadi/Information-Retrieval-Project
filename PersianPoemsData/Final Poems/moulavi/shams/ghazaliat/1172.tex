\begin{center}
\section*{غزل شماره ۱۱۷۲: جاء الربیع و البطر زال الشتاء و الخطر}
\label{sec:1172}
\addcontentsline{toc}{section}{\nameref{sec:1172}}
\begin{longtable}{l p{0.5cm} r}
جاء الربیع و البطر زال الشتاء و الخطر
&&
من فضل رب عنده کل الخطایا تغتفر
\\
آمد ترش رویی دگر یا زمهریرست او مگر
&&
برریز جامی بر سرش ای ساقی همچون شکر
\\
اوحی الیکم ربکم انا غفرنا ذنبکم
&&
و ارضوا بما یقضی لکم ان الرضا خیر السیر
\\
یا می دهش از بلبله یا خود به راهش کن هله
&&
زیرا میان گلرخان خوش نیست عفریت ای پسر
\\
و قایل یقول لی انا علمنا بره
&&
فاحک لدینا سره لا تشتغل فیما اشتهر
\\
درده می بیغامبری تا خر نماند در خری
&&
خر را بروید در زمان از باده عیسی دو پر
\\
السر فیک یا فتی لا تلتمس فیما اتی
&&
من لیس سر عنده لم ینتفع مما ظهر
\\
در مجلس مستان دل هشیار اگر آید مهل
&&
دانی که مستان را بود در حال مستی خیر و شر
\\
انظر الی اهل الردی کم عاینوا نور الهدی
&&
لم ترتفع استارهم من بعد ما انشق القمر
\\
ای پاسبان بر در نشین در مجلس ما ره مده
&&
جز عاشقی آتش دلی کید از او بوی جگر
\\
یا ربنا رب المنن ان انت لم ترحم فمن
&&
منک الهدی منک الردی ما غیر ذا الا غرر
\\
جز عاشقی عاشق کنی مستی لطیفی روشنی
&&
نشناسد از مستی خود او سرکله را از کمر
\\
یا شوق این العافیه کی اضطفر بالقافیه
&&
عندی صفات صافیه فی جنبها نطقی کدر
\\
گر دست خواهی پا نهد ور پای خواهی سر نهد
&&
ور بیل خواهی عاریت بر جای بیل آرد تبر
\\
ان کان نطقی مدرسی قد ظل عشقی مخرسی
&&
و العشق قرن غالب فینا و سلطان الظفر
\\
ای خواجه من آغشته‌ام بی‌شرم و بی‌دل گشته‌ام
&&
اسپر سلامت نیستم در پیش تیغم چون سپر
\\
سر کتیم لفظه سیف حسیم لحظه
&&
شمس الضحی لا تختفی الا بسحار سحر
\\
خواهم یکی گوینده‌ای مستی خرابی زنده‌ای
&&
کآتش به خواب اندرزند وین پرده گوید تا سحر
\\
یا ساحراء ابصارنا بالغت فی اسحارنا
&&
فارفق بنا اودارنا انا حبسنا فی السفر
\\
اندر تن من گر رگی هشیار یابی بردرش
&&
چون شیرگیر او نشد او را در این ره سگ شمر
\\
یا قوم موسی اننا فی التیه تهنا مثلکم
&&
کیف اهتدیتم فاخبروا لا تکتموا عنا الخبر
\\
آن‌ها خراب و مست و خوش وین‌ها غلام پنج و شش
&&
آن‌ها جدا وین‌ها جدا آن‌ها دگر وین‌ها دگر
\\
ان عوقوا ترحالنا فالمن و السلوی لنا
&&
اصلحت ربی بالنا طاب السفر طاب الحضر
\\
گفتن همه جنگ آورد در بوی و در رنگ آورد
&&
چون رافضی جنگ افکند هر دم علی را با عمر
\\
اسکت و لا تکثر اخی ان طلت تکثر ترتخی
&&
الحیل فی ریح الهوی فاحفظه کلا لا وزر
\\
خامش کن و کوتاه کن نظاره آن ماه کن
&&
آن مه که چون بر ماه زد از نورش انشق القمر
\\
ان الهوی قد غرنا من بعد ما قد سرنا
&&
فاکشف به لطف ضرنا قال النبی لا ضرر
\\
ای میر مه روپوش کن ای جان عاشق جوش کن
&&
ما را چو خود بی‌هوش کن بی‌هوش خوش در ما نگر
\\
قالوا ندبر شأنکم نفتح لکم آذانکم
&&
نرفع لکم ارکانکم انتم مصابیح البشر
\\
ز اندازه بیرون خورده‌ام کاندازه را گم کرده‌ام
&&
شدوا یدی شدوا فمی هذا دواء من سکر
\\
هاکم معاریج اللقا فیها تداریج البقا
&&
انعم به من مستقی اکرم به من مستقر
\\
هین نیش ما را نوش کن افغان ما را گوش کن
&&
ما را چو خود بی‌هوش کن بی‌هوش سوی ما نگر
\\
العیش حقا عیشکم و الموت حقا موتکم
&&
و الدین و الدنیا لکم هذا جزاء من شکر
\\
\end{longtable}
\end{center}
