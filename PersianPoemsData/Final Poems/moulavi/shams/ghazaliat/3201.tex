\begin{center}
\section*{غزل شماره ۳۲۰۱: یا ساقی الحی اسمع سؤالی}
\label{sec:3201}
\addcontentsline{toc}{section}{\nameref{sec:3201}}
\begin{longtable}{l p{0.5cm} r}
یا ساقی الحی اسمع سؤالی
&&
انشد فادی، واخبر بحال
\\
قالو تسلی، حاشا و کلا
&&
عشق تجلی من ذی‌الجلال
\\
العشق فنی، والشوق دنی
&&
والخمر منی، والسکر حالی
\\
عشق وجیهی، بحر یلیه
&&
والحوت فیه روح‌الرجال
\\
انتم شفایی، انتم دوایی
&&
انتم رجایی، انتم کمالی
\\
الفخ کامن، والعشق آمن
&&
والرب ضامن، کی لاتبالی
\\
عشق موبد، فتلی تعمد
&&
و انا معود، بأس‌النزال
\\
گفتم که: « ما را هنگامه بنما »
&&
گفت: « اینک اما تو در جوالی
\\
بدران جوال و سر را برون کن
&&
تا خود ببینی کندر وصالی
\\
اندر ره جان پا را مرنجان
&&
زیرا همایی با پر و بالی »
\\
گفتم که: « عاشق بیند مرافق »
&&
گفتا که: « لالا ان کان سالی »
\\
گفتم که: « بکشی تو بی‌گنه را »
&&
گفتا: « کذا هوالوصل غالی »
\\
گفتم « چه نوشم زان شهد؟ » گفتا
&&
« مومت نباشد هان، تا نمالی »
\\
انعم صباحا، واطلب رباحا
&&
وابسط جناحا فالقصر عالی
\\
می‌نال چون نا، خوش همنشینا!
&&
حقست بینا، هر چون که نالی
\\
انا وجدنا درا، فقدنا
&&
لما ولجنا، موج‌اللیالی
\\
می گرد شبها، گرد طلبها
&&
تا پیشت آید نیکو سگالی
\\
می گرد شب در، مانند اختر
&&
ان‌اللیالی بحراللالی
\\
دارم رسولی، اما ملولی
&&
یارب خلص، عن ذی‌الملال
\\
عندی شراب لوذقت منه
&&
بس شیرگیری، گرچه شغالی
\\
درکش چو افیون، واره تو اکنون
&&
گه در جوابی، گه در سوالی
\\
من سخت مستم، به خود خوشستم
&&
یا من تلمنی، لم تدر حالی
\\
جانا فرود آ، از بام بالا
&&
وانعم بوصل، فالبیت خالی
\\
گفتم که: « بشنو، رمزی ز بنده »
&&
گفتا که: « اسکت یا ذاالمقال »
\\
گفتم: « خموشی صعبست » گفتا:
&&
یا ذاالمقال، صرذاالمعالی
\\
کس نیست محرم، کوتاه کن دم
&&
والله اعلم، والله تالی
\\
\end{longtable}
\end{center}
