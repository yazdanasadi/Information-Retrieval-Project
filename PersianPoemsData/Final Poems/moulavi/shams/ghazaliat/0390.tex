\begin{center}
\section*{غزل شماره ۳۹۰: ساربانا اشتران بین سر به سر قطار مست}
\label{sec:0390}
\addcontentsline{toc}{section}{\nameref{sec:0390}}
\begin{longtable}{l p{0.5cm} r}
ساربانا اشتران بین سر به سر قطار مست
&&
میر مست و خواجه مست و یار مست اغیار مست
\\
باغبانا رعد مطرب ابر ساقی گشت و شد
&&
باغ مست و راغ مست و غنچه مست و خار مست
\\
آسمانا چند گردی گردش عنصر ببین
&&
آب مست و باد مست و خاک مست و نار مست
\\
حال صورت این چنین و حال معنی خود مپرس
&&
روح مست و عقل مست و خاک مست اسرار مست
\\
رو تو جباری رها کن خاک شو تا بنگری
&&
ذره ذره خاک را از خالق جبار مست
\\
تا نگویی در زمستان باغ را مستی نماند
&&
مدتی پنهان شدست از دیده مکار مست
\\
بیخ‌های آن درختان می نهانی می‌خورند
&&
روزکی دو صبر می‌کن تا شود بیدار مست
\\
گر تو را کوبی رسد از رفتن مستان مرنج
&&
با چنان ساقی و مطرب کی رود هموار مست
\\
ساقیا باده یکی کن چند باشد عربده
&&
دوستان ز اقرار مست و دشمنان ز انکار مست
\\
باد را افزون بده تا برگشاید این گره
&&
باده تا در سر نیفتد کی دهد دستار مست
\\
بخل ساقی باشد آن جا یا فساد باده‌ها
&&
هر دو ناهموار باشد چون رود رهوار مست
\\
روی‌های زرد بین و باده گلگون بده
&&
زانک از این گلگون ندارد بر رخ و رخسار مست
\\
باده‌ای داری خدایی بس سبک خوار و لطیف
&&
زان اگر خواهد بنوشد روز صد خروار مست
\\
شمس تبریزی به دورت هیچ کس هشیار نیست
&&
کافر و مؤمن خراب و زاهد و خمار مست
\\
\end{longtable}
\end{center}
