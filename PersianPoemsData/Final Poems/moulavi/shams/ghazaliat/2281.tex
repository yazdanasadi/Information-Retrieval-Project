\begin{center}
\section*{غزل شماره ۲۲۸۱: ای از تو خاکی تن شده تن فکرت و گفتن شده}
\label{sec:2281}
\addcontentsline{toc}{section}{\nameref{sec:2281}}
\begin{longtable}{l p{0.5cm} r}
ای از تو خاکی تن شده تن فکرت و گفتن شده
&&
وز گفت و فکرت بس صور در غیب آبستن شده
\\
هر صورتی پرورده‌ای معنی است لیک افسرده‌ای
&&
صورت چو معنی شد کنون آغاز را روشن شده
\\
یخ را اگر بیند کسی و آن کس نداند اصل یخ
&&
چون دید کآخر آب شد در اصل یخ بی‌ظن شده
\\
اندیشه جز زیبا مکن کو تار و پود صورت است
&&
ز اندیشه‌ای احسن تند هر صورتی احسن شده
\\
زان سوی کاندازی نظر آن جنس می‌آید صور
&&
پس از نظر آید صور اشکال مرد و زن شده
\\
با آن نشین کو روشن است کز دل سوی دل روزن است
&&
خاک از چه ورد و سوسن است کش آب هم مسکن شده
\\
ور همنشین حق شوی جان خوش مطلق شوی
&&
یا رب چه بارونق شوی ای جان جان من شده
\\
از جا به بی‌جا آمده اه رفته هیهای آمده
&&
بی‌دست و بی‌پای آمده چون ماه خوش خرمن شده
\\
یا رب که چون می‌بینمش ای بنده جان و دینمش
&&
خود چیست این تمکینمش ای عقل از این امکن شده
\\
هر ذره‌ای را محرم او هر خوش دمی را همدم او
&&
نادیده زو زاهد شده زو دیده تردامن شده
\\
ای عشق حق سودای او آن او است او جویای او
&&
وی می‌دمد در وای او ای طالب معدن شده
\\
هم طالب و مطلوب او هم عاشق و معشوق او
&&
هم یوسف و یعقوب او هم طوق و هم گردن شده
\\
اوصافت ای کس کم چو تو پایان ندارد همچو تو
&&
چند آب و روغن می‌کنم ای آب من روغن شده
\\
\end{longtable}
\end{center}
