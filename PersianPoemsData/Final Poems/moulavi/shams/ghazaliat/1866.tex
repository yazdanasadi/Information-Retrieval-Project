\begin{center}
\section*{غزل شماره ۱۸۶۶: ای کار من از تو زر ای سیمبر مستان}
\label{sec:1866}
\addcontentsline{toc}{section}{\nameref{sec:1866}}
\begin{longtable}{l p{0.5cm} r}
ای کار من از تو زر ای سیمبر مستان
&&
هم سیم به یادم ده هم سیم و زرم بستان
\\
در عین زمستانی چون گرم کنی مرکب
&&
از گرمی میدانت برسوزد تابستان
\\
گر طفلک یک روزه شب‌های تو را بیند
&&
از شیر بری گردد وز مادر وز پستان
\\
ای وای از آن ساعت کاین خاطر چون پیلم
&&
سرمست شما گردد یاد آرد هندستان
\\
روزی که تب مرگم یک باره فروگیرد
&&
هر پاره ز من گردد از آتش تب سستان
\\
تو از پس پرده دل ناگاه سری درکن
&&
تا هر سر موی من گردند چو سرمستان
\\
هر خاطر من بکری بر بام و در از عشقت
&&
چندان بکند شیوه چندان بکند دستان
\\
تا تابش روی تو درپیچد در هر یک
&&
وز چون تو شهی گردد هر خاطرم آبستان
\\
شمس الحق تبریزی هر کس که ز تو پرسد
&&
می بینم و می گویم از رشک کدام است آن
\\
\end{longtable}
\end{center}
