\begin{center}
\section*{غزل شماره ۸۴۳: در عشق زنده باید کز مرده هیچ ناید}
\label{sec:0843}
\addcontentsline{toc}{section}{\nameref{sec:0843}}
\begin{longtable}{l p{0.5cm} r}
در عشق زنده باید کز مرده هیچ ناید
&&
دانی که کیست زنده آن کو ز عشق زاید
\\
گرمی شیر غران تیزی تیغ بران
&&
نری جمله نران با عشق کند آید
\\
در راه رهزنانند وین همرهان زنانند
&&
پای نگارکرده این راه را نشاید
\\
طبل غزا برآمد وز عشق لشکر آمد
&&
کو رستم سرآمد تا دست برگشاید
\\
رعدش بغرد از دل جانش ز ابر قالب
&&
چون برق بجهد از تن یک لحظه‌ای نپاید
\\
هرگز چنین سری را تیغ اجل نبرد
&&
کاین سر ز سربلندی بر ساق عرش ساید
\\
هرگز چنین دلی را غصه فرونگیرد
&&
غم‌های عالم او را شادی دل فزاید
\\
دریا پیش ترش رو او ابر نوبهارست
&&
عالم بدوست شیرین قاصد ترش نماید
\\
شیرش نخواهد آهو آهوی اوست یاهو
&&
منکر در این چراخور بسیار ژاژ خاید
\\
در عشق جوی ما را در ما بجوی او را
&&
گاهی منش ستایم گاه او مرا ستاید
\\
تا چون صدف ز دریا بگشاید او دهانی
&&
دریای ما و من را چون قطره دررباید
\\
\end{longtable}
\end{center}
