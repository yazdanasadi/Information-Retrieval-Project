\begin{center}
\section*{غزل شماره ۱۵۲: دوش آن جانان ما افتان و خیزان یک قبا}
\label{sec:0152}
\addcontentsline{toc}{section}{\nameref{sec:0152}}
\begin{longtable}{l p{0.5cm} r}
دوش آن جانان ما افتان و خیزان یک قبا
&&
مست آمد با یکی جامی پر از صرف صفا
\\
جام می می‌ریخت ره ره زانک مست مست بود
&&
خاک ره می‌گشت مست و پیش او می‌کوفت پا
\\
صد هزاران یوسف از حسنش چو من حیران شده
&&
ناله می‌کردند کی پیدای پنهان تا کجا
\\
جان به پیشش در سجود از خاک ره بد بیشتر
&&
عقل دیوانه شده نعره زنان که مرحبا
\\
جیب‌ها بشکافته آن خویشتن داران ز عشق
&&
دل سبک مانند کاه و روی‌ها چون کهربا
\\
عالمی کرده خرابه از برای یک کرشم
&&
وز خمار چشم نرگس عالمی دیگر هبا
\\
هوشیاران سر فکنده جمله خود از بیم و ترس
&&
پیش او صف‌ها کشیده بی‌دعا و بی‌ثنا
\\
و آنک مستان خمار جادوی اویند نیز
&&
چون ثنا گویند کز هستی فتادستند جدا
\\
من جفاگر بی‌وفا جستم که هم جامم شود
&&
پیش جام او بدیدم مست افتاده وفا
\\
ترک و هندو مست و بدمستی همی‌کردند دوش
&&
چون دو خصم خونی ملحد دل دوزخ سزا
\\
گه به پای همدگر چون مجرمان معترف
&&
می‌فتادندی به زاری جان سپار و تن فدا
\\
باز دست همدگر بگرفته آن هندو و ترک
&&
هر دو در رو می‌فتادند پیش آن مه روی ما
\\
یک قدح پر کرد شاه و داد ظاهر آن به ترک
&&
وز نهان با یک قدح می‌گفت هندو را بیا
\\
ترک را تاجی به سر کایمان لقب دادم تو را
&&
بر رخ هندو نهاده داغ کاین کفرست،ها
\\
آن یکی صوفی مقیم صومعه پاکی شده
&&
وین مقامر در خراباتی نهاده رخت‌ها
\\
چون پدید آمد ز دور آن فتنه جان‌های حور
&&
جام در کف سکر در سر روی چون شمس الضحی
\\
ترس جان در صومعه افتاد زان ترساصنم
&&
می‌کش و زنار بسته صوفیان پارسا
\\
وان مقیمان خراباتی از آن دیوانه تر
&&
می‌شکستند خم‌ها و می‌فکندند چنگ و نا
\\
شور و شر و نفع و ضر و خوف و امن و جان و تن
&&
جمله را سیلاب برده می‌کشاند سوی لا
\\
نیم شب چون صبح شد آواز دادند مؤذنان
&&
ایها العشاق قوموا و استعدوا للصلا
\\
\end{longtable}
\end{center}
