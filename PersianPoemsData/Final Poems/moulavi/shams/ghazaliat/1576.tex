\begin{center}
\section*{غزل شماره ۱۵۷۶: ما زنده به نور کبریاییم}
\label{sec:1576}
\addcontentsline{toc}{section}{\nameref{sec:1576}}
\begin{longtable}{l p{0.5cm} r}
ما زنده به نور کبریاییم
&&
بیگانه و سخت آشناییم
\\
نفس است چو گرگ لیک در سر
&&
بر یوسف مصر برفزاییم
\\
مه توبه کند ز خویش بینی
&&
گر ما رخ خود به مه نماییم
\\
درسوزد پر و بال خورشید
&&
چون ما پر و بال برگشاییم
\\
این هیکل آدم است روپوش
&&
ما قبله جمله سجده‌هاییم
\\
آن دم بنگر مبین تو آدم
&&
تا جانت به لطف دررباییم
\\
ابلیس نظر جدا جدا داشت
&&
پنداشت که ما ز حق جداییم
\\
شمس تبریز خود بهانه‌ست
&&
ماییم به حسن لطف ماییم
\\
با خلق بگو برای روپوش
&&
کو شاه کریم و ما گداییم
\\
ما را چه ز شاهی و گدایی
&&
شادیم که شاه را سزاییم
\\
محویم به حسن شمس تبریز
&&
در محو نه او بود نه ماییم
\\
\end{longtable}
\end{center}
