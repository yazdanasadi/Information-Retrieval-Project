\begin{center}
\section*{غزل شماره ۱۷۶۰: به خدایی که در ازل بوده‌ست}
\label{sec:1760}
\addcontentsline{toc}{section}{\nameref{sec:1760}}
\begin{longtable}{l p{0.5cm} r}
به خدایی که در ازل بوده‌ست
&&
حی و دانا و قادر و قیوم
\\
نور او شمع‌های عشق فروخت
&&
تا بشد صد هزار سر معلوم
\\
از یکی حکم او جهان پر شد
&&
عاشق و عشق و حاکم و محکوم
\\
در طلسمات شمس تبریزی
&&
گشت گنج عجایبش مکتوم
\\
که از آن دم که تو سفر کردی
&&
از حلاوت جدا شدیم چو موم
\\
همه شب همچو شمع می سوزیم
&&
ز آتشش جفت وز انگبین محروم
\\
در فراق جمال او ما را
&&
جسم ویران و جان در او چون بوم
\\
آن عنان را بدین طرف برتاب
&&
زفت کن پیل عیش را خرطوم
\\
بی‌حضورت سماع نیست حلال
&&
همچو شیطان طرب شده مرحوم
\\
یک غزل بی‌تو هیچ گفته نشد
&&
تا رسید آن مشرفه مفهوم
\\
بس به ذوق سماع نامه تو
&&
غزلی پنج شش بشد منظوم
\\
شام ما از تو صبح روشن باد
&&
ای به تو فخر شام و ارمن و روم
\\
\end{longtable}
\end{center}
