\begin{center}
\section*{غزل شماره ۴۴۹: جانا جمال روح بسی خوب و بافرست}
\label{sec:0449}
\addcontentsline{toc}{section}{\nameref{sec:0449}}
\begin{longtable}{l p{0.5cm} r}
جانا جمال روح بسی خوب و بافرست
&&
لیکن جمال و حسن تو خود چیز دیگرست
\\
ای آنک سال‌ها صفت روح می‌کنی
&&
بنمای یک صفت که به ذاتش برابرست
\\
در دیده می‌فزاید نور از خیال او
&&
با این همه به پیش وصالش مکدرست
\\
ماندم دهان باز ز تعظیم آن جمال
&&
هر لحظه بر زبان و دل الله اکبرست
\\
دل یافت دیده‌ای که مقیم هوای توست
&&
آوه که آن هوا چه دل و دیده پرورست
\\
از حور و ماه و روح و پری هیچ دم مزن
&&
کان‌ها به او نماند او چیز دیگرست
\\
چاکرنوازیست که کردست عشق تو
&&
ور نی کجا دلی که بدان عشق درخورست
\\
هر دل که او نخفت شبی در هوای تو
&&
چون روز روشنست و هوا زو منورست
\\
هر کس که بی‌مراد شد او چون مرید توست
&&
بی صورت مراد مرادش میسرست
\\
هر دوزخی که سوخت و در این عشق اوفتاد
&&
در کوثر اوفتاد که عشق تو کوثرست
\\
پایم نمی‌رسد به زمین از امید وصل
&&
هر چند از فراق توم دست بر سرست
\\
غمگین مشو دلا تو از این ظلم دشمنان
&&
اندیشه کن در این که دلارام داورست
\\
از روی زعفران من ار شاد شد عدو
&&
نی روی زعفران من از ورد احمرست
\\
چون برترست خوبی معشوقم از صفت
&&
دردم چه فربه‌ست و مدیحم چه لاغرست
\\
آری چو قاعده‌ست که رنجور زار را
&&
هر چند رنج بیش بود ناله کمترست
\\
همچون قمر بتافت ز تبریز شمس دین
&&
نی خود قمر چه باشد کان روی اقمرست
\\
\end{longtable}
\end{center}
