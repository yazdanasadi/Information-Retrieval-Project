\begin{center}
\section*{غزل شماره ۲۷۱۰: بیا ای غم که تو بس باوفایی}
\label{sec:2710}
\addcontentsline{toc}{section}{\nameref{sec:2710}}
\begin{longtable}{l p{0.5cm} r}
بیا ای غم که تو بس باوفایی
&&
که ابر قطره‌های اشک‌هایی
\\
زنی درویش آمد سوی عباس
&&
که تعلیمم بده نوعی گدایی
\\
در حیلت خدا بر تو گشاده‌ست
&&
تو آموزی گدایان را دغایی
\\
تو نعمانی در این مذهب بگو درس
&&
که خوش تخریج و پاکیزه ادایی
\\
من مسکین دمی دارم فسرده
&&
ندارم روزیی از ژاژخایی
\\
مرا یک کدیه گرمی بیاموز
&&
که تو بس نرگدا و اوستایی
\\
بدانک انبیا عباس دینند
&&
در استرزاق آثار سمایی
\\
ز انواع گدایی‌های طاعات
&&
که برجوشد بدان بحر عطایی
\\
ز صوم و از صلات و از مناسک
&&
ز نهی منکر و شیر غزایی
\\
که بی‌حد است انواع عبادات
&&
و انواع ثقات و ابتلایی
\\
بدو گفتا برو کاین دم ملولم
&&
ببر زحمت مکن طال بقایی
\\
مکرر کرد آن زن لابه کردن
&&
که نومیدم مکن ای لالکایی
\\
مکرر کرد استا دفع راهم
&&
که سودت نیست این زحمت فزایی
\\
ملولم خاطرم کند است این دم
&&
ندارد این نفس مکرم کیایی
\\
سجود آورد و گریان گشت آن زن
&&
که طفلانم مرند از بی‌نوایی
\\
بسی بگریست پس عباس گفتش
&&
همین را باش کاستاتر ز مایی
\\
دو عباسند با تو این دو چشمت
&&
تلین القاسیین بالبکا
\\
به آب دیده چون جنت توان یافت
&&
روان شو چیز دیگر را چه پایی
\\
که آب چشم با خون شهیدان
&&
برابر می‌روند اندر روایی
\\
کسی را که خدا بخشید گریه
&&
بیاموزید راه دلگشایی
\\
بجز این گریه را نفعی دگر هست
&&
ولی سیرم ز شعر و خودنمایی
\\
ولیکن خدمت دل به ز گریه‌ست
&&
که اطلس می‌کند پنجه عبایی
\\
که دل اصل است و اشک تو وسیلت
&&
که خشک و تر نگنجد در خدایی
\\
خمش با دل نشین و رو در او نه
&&
که از سلطان دل صاحب لوایی
\\
\end{longtable}
\end{center}
