\begin{center}
\section*{غزل شماره ۲۲۳۶: جانا تویی کلیم و منم چون عصای تو}
\label{sec:2236}
\addcontentsline{toc}{section}{\nameref{sec:2236}}
\begin{longtable}{l p{0.5cm} r}
جانا تویی کلیم و منم چون عصای تو
&&
گه تکیه گاه خلقم و گه اژدهای تو
\\
در دست فضل و رحمت تو یارم و عصا
&&
ماری شوم چو افکندم اصطفای تو
\\
ای باقی و بقای تو بی‌روز و روزگار
&&
شد روز و روزگار من اندر وفای تو
\\
صد روز و روزگار دگر گر دهی مرا
&&
بادا فدای عشق و فریب و ولای تو
\\
دل چشم گشت جمله چو چشمم به دل بگفت
&&
بی‌کام و بی‌زبان عجب وصف‌های تو
\\
زان دم که از تو چشم خبر برد سوی دل
&&
دل می‌کند دعای دو چشم و دعای تو
\\
می‌گردد آسمان همه شب با دو صد چراغ
&&
در جست و جوی چشم خوش دلربای تو
\\
گر کاسه بی‌نوا شد ور کیسه لاغری
&&
صد جان و دل فزود رخ جان فزای تو
\\
گر خانه و دکان ز هوای تو شد خراب
&&
درتافت لاجرم به خرابم ضیای تو
\\
ای جان اگر رضای تو غم خوردن دل است
&&
صد دل به غم سپارم بهر رضای تو
\\
از زخم هاون غم خود خوش مرا بکوب
&&
زین کوفتن رسد به نظر توتیای تو
\\
جان چیست نیم برگ ز گلزار حسن تو
&&
دل چیست یک شکوفه ز برگ و نوای تو
\\
خامش کنم اگر چه که گوینده من نیم
&&
گفت آن توست و گفتن خلقان صدای تو
\\
\end{longtable}
\end{center}
