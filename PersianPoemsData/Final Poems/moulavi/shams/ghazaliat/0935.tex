\begin{center}
\section*{غزل شماره ۹۳۵: مکن مکن که پشیمان شوی و بد باشد}
\label{sec:0935}
\addcontentsline{toc}{section}{\nameref{sec:0935}}
\begin{longtable}{l p{0.5cm} r}
مکن مکن که پشیمان شوی و بد باشد
&&
که بی‌عنایت جان باغ چون لحد باشد
\\
چه ریشه برکنی از غصه و پشیمانی
&&
چو ریش عقل تو در دست کالبد باشد
\\
بکن مجاهده با نفس و جنگ ریشاریش
&&
که صلح را ز چنین جنگ‌ها مدد باشد
\\
وگر گریز کنی همچو آهو از کف شیر
&&
ز تو گریزد آن ماه بر اسد باشد
\\
نه گوش تو سخن یار مهربان شنود
&&
نه پیش چشم تو دلدار سروقد باشد
\\
نشین به کشتی روح و بگیر دامن نوح
&&
به بحر عشق که هر لحظه جزر و مد باشد
\\
گذر ز ناز و ملولی که ناز آن تو نیست
&&
که آن وظیفه آن یار ماه خد باشد
\\
چه ظلم کردم بر حسن او که مه گفتم
&&
صد آفتاب و فلک را بر او حسد باشد
\\
خموش باش و مگو ریگ را شمار مکن
&&
شمار چون کنی آن را که بی‌عدد باشد
\\
\end{longtable}
\end{center}
