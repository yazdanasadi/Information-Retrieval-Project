\begin{center}
\section*{غزل شماره ۲۶: هر لحظه وحی آسمان آید به سر جان‌ها}
\label{sec:0026}
\addcontentsline{toc}{section}{\nameref{sec:0026}}
\begin{longtable}{l p{0.5cm} r}
هر لحظه وحی آسمان آید به سر جان‌ها
&&
کاخر چو دردی بر زمین تا چند می‌باشی برآ
\\
هر کز گران جانان بود چون درد در پایان بود
&&
آنگه رود بالای خم کان درد او یابد صفا
\\
گل را مجنبان هر دمی تا آب تو صافی شود
&&
تا درد تو روشن شود تا درد تو گردد دوا
\\
جانیست چون شعله ولی دودش ز نورش بیشتر
&&
چون دود از حد بگذرد در خانه ننماید ضیا
\\
گر دود را کمتر کنی از نور شعله برخوری
&&
از نور تو روشن شود هم این سرا هم آن سرا
\\
در آب تیره بنگری نی ماه بینی نی فلک
&&
خورشید و مه پنهان شود چون تیرگی گیرد هوا
\\
باد شمالی می‌وزد کز وی هوا صافی شود
&&
وز بهر این صیقل سحر در می‌دمد باد صبا
\\
باد نفس مر سینه را ز اندوه صیقل می‌زند
&&
گر یک نفس گیرد نفس مر نفس را آید فنا
\\
جان غریب اندر جهان مشتاق شهر لامکان
&&
نفس بهیمی در چرا چندین چرا باشد چرا
\\
ای جان پاک خوش گهر تا چند باشی در سفر
&&
تو باز شاهی بازپر سوی صفیر پادشا
\\
\end{longtable}
\end{center}
