\begin{center}
\section*{غزل شماره ۱۱۶۱: مطربا عیش و نوش از سر گیر}
\label{sec:1161}
\addcontentsline{toc}{section}{\nameref{sec:1161}}
\begin{longtable}{l p{0.5cm} r}
مطربا عیش و نوش از سر گیر
&&
یک دو ابریشمک فروتر گیر
\\
ننگ بگذار و با حریف بساز
&&
جنگ بگذار جام و ساغر گیر
\\
لطف گل بین و جرم خار مبین
&&
جعد بگشا و مشک و عنبر گیر
\\
فربه از توست آسمان و زمین
&&
این یک استاره را تو لاغر گیر
\\
داروی فربهی خلق تویی
&&
فربهش کن چو خواهی و برگیر
\\
خرمش کن به یک شکرخنده
&&
شکری را ز مصر کمتر گیر
\\
بخت و اقبال خاک پای تواند
&&
هر چه می‌بایدت میسر گیر
\\
چونک سعد و ظفر غلام تواند
&&
دشمنت را هزار لشکر گیر
\\
ای دل ار آب کوثرت باید
&&
آتش عشق را تو کوثر گیر
\\
گر غلامی قیصرت باید
&&
بنده‌اش را قباد و قیصر گیر
\\
هر که را نبض عشق می‌نجهد
&&
گر فلاطون بود تواش خر گیر
\\
هر سری کو ز عشق پر نبود
&&
آن سرش را ز دم مأخر گیر
\\
هین مگو راز شمس تبریزی
&&
مکن اسپید و جام احمر گیر
\\
\end{longtable}
\end{center}
