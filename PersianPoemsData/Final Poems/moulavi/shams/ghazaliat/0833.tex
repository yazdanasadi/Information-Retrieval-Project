\begin{center}
\section*{غزل شماره ۸۳۳: مرگ ما هست عروسی ابد}
\label{sec:0833}
\addcontentsline{toc}{section}{\nameref{sec:0833}}
\begin{longtable}{l p{0.5cm} r}
مرگ ما هست عروسی ابد
&&
سر آن چیست هو الله احد
\\
شمس تفریق شد از روزنه‌ها
&&
بسته شد روزنه‌ها رفت عدد
\\
آن عددها که در انگور بود
&&
نیست در شیره کز انگور چکد
\\
هر کی زنده‌ست به نورالله
&&
مرگ این روح مر او راست مدد
\\
بد مگو نیک مگو ایشان را
&&
که گذشتند ز نیکو و ز بد
\\
دیده در حق نه و نادیده مگو
&&
تا که در دیده دگر دیده نهد
\\
دیده دیده بود آن دیده
&&
هیچ غیبی و سری زو نجهد
\\
نظرش چونک به نورالله است
&&
بر چنان نور چه پوشیده شود
\\
نورها گر چه همه نور حقند
&&
تو مخوان آن همه را نور صمد
\\
نور باقیست که آن نور خدا است
&&
نور فانی صفت جسم و جسد
\\
نور ناریست در این دیده خلق
&&
مگر آن را که حقش سرمه کشد
\\
نار او نور شد از بهر خلیل
&&
چشم خر شد به صفت چشم خرد
\\
ای خدایی که عطایت دیدست
&&
مرغ دیده به هوای تو پرد
\\
قطب این که فلک افلاکست
&&
در پی جستن تو بست رصد
\\
یا ز دیدار تو دید آر او را
&&
یا بدین عیب مکن او را رد
\\
دیده تر دار تو جان را هر دم
&&
نگهش دار ز دام قد و خد
\\
دیده در خواب ز تو بیداری
&&
این چنین خواب کمالست و رشد
\\
لیک در خواب نیابد تعبیر
&&
تو ز خوابش به جهان رغم حسد
\\
ور نه می‌کوشد و بر می‌جوشد
&&
ز آتش عشق احد تا به لحد
\\
\end{longtable}
\end{center}
