\begin{center}
\section*{غزل شماره ۱۸۰۲: چندان بگردم گرد دل کز گردش بسیار من}
\label{sec:1802}
\addcontentsline{toc}{section}{\nameref{sec:1802}}
\begin{longtable}{l p{0.5cm} r}
چندان بگردم گرد دل کز گردش بسیار من
&&
نی تن کشاند بار من نی جان کند پیکار من
\\
چندان طواف کان کنم چندان مصاف جان کنم
&&
تا بگسلد یک بارگی هم پود من هم تار من
\\
گر تو لجوجی سخت سر من هم لجوجم ای پسر
&&
سر می نهد هر شیر نر در صبر پاافشار من
\\
تن چون نگردد گرد جان با مشعل چون آسمان
&&
ای نقطه خوبی و کش در جان چون پرگار من
\\
تا آب باشد پیشوا گردان بود این آسیا
&&
تو بی‌خبر گویی که بس که آرد شد خروار من
\\
او فارغ است از کار تو وز گندم و خروار تو
&&
تا آب هست او می تپد چون چرخ در اسرار من
\\
غلبیرم اندر دست او در دست می گرداندم
&&
غلبیر کردن کار او غلبیر بودن کار من
\\
نی صدق ماند و نی ریا نی آب ماند و نی گیا
&&
وانگه بگفتم هین بیا ای یار گل رخسار من
\\
ای جان جان مست من ای جسته دوش از دست من
&&
مشکن ببین اشکست من خیز ای سپه سالار من
\\
ای جان خوش رفتار من می پیچ پیش یار من
&&
تا گویدت دلدار من ای جان و ای جاندار من
\\
مثل کلابه‌ست این تنم حق می تند چون تن زنم
&&
تا چه گولم می کند او زین کلابه و تار من
\\
پنهان بود تار و کشش پیدا کلابه و گردشش
&&
گوید کلابه کی بود بی‌جذبه این پیکار من
\\
تن چون عصابه جان چو سر کان هست پیچان گرد سر
&&
هر پیچ بر پیچ دگر توتوست چون دستار من
\\
ای شمس تبریزی طری گاهی عصابه گه سری
&&
ترسم که تو پیچی کنی در مغلطه دیدار من
\\
\end{longtable}
\end{center}
