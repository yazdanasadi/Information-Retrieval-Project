\begin{center}
\section*{غزل شماره ۶۷۸: به صورت یار من چون خشمگین شد}
\label{sec:0678}
\addcontentsline{toc}{section}{\nameref{sec:0678}}
\begin{longtable}{l p{0.5cm} r}
به صورت یار من چون خشمگین شد
&&
دلم گفت اه مگر با من به کین شد
\\
به صد وادی فرورفتم به سودا
&&
که چه چاره که چاره گر چنین شد
\\
به سوی آسمان رفتم چو دیوان
&&
از این درد آسمان من زمین شد
\\
مرا گفتند راه راست برگیر
&&
چه ره گیرم که یار راستین شد
\\
مرا هم راه و همراهست یارم
&&
که روی او مرا ایمان و دین شد
\\
به زیر گلبنش هر کس که بنشست
&&
سعادت با نشستش همنشین شد
\\
در این گفتارم آن معنی طلب کن
&&
نفس‌های خوشم او را کمین شد
\\
ازیرا اسم‌ها عین مسماست
&&
ز عین اسم آدم عین بین شد
\\
اگر خواهی که عین جمع باشی
&&
همین شد چاره و درمان همین شد
\\
مخوان این گنج نامه دیگر ای جان
&&
که این گنج از پی حکمت دفین شد
\\
به کهگل چون بپوشم آفتابی
&&
جهانی کی درون آستین شد
\\
اگر تو زین ملولی وای بر تو
&&
که تو پیرار مردی این یقین شد
\\
زره بر آب می‌دان این سخن را
&&
همان آبست الا شکل چین شد
\\
ز خود محجوبشان کردم به گفتن
&&
به پیش حاسدان واجب چنین شد
\\
خمش باشم لب از گفتن ببندم
&&
که مشتی بیس با پیری قرین شد
\\
\end{longtable}
\end{center}
