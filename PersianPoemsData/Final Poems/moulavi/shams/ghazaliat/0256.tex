\begin{center}
\section*{غزل شماره ۲۵۶: داد دهی ساغر و پیمانه را}
\label{sec:0256}
\addcontentsline{toc}{section}{\nameref{sec:0256}}
\begin{longtable}{l p{0.5cm} r}
داد دهی ساغر و پیمانه را
&&
مایه دهی مجلس و میخانه را
\\
مست کنی نرگس مخمور را
&&
پیش کشی آن بت دردانه را
\\
جز ز خداوندی تو کی رسد
&&
صبر و قرار این دل دیوانه را
\\
تیغ برآور هله ای آفتاب
&&
نور ده این گوشه ویرانه را
\\
قاف تویی مسکن سیمرغ را
&&
شمع تویی جان چو پروانه را
\\
چشمه حیوان بگشا هر طرف
&&
نقل کن آن قصه و افسانه را
\\
مست کن ای ساقی و در کار کش
&&
این بدن کافر بیگانه را
\\
گر نکند رام چنین دیو را
&&
پس چه شد آن ساغر مردانه را
\\
نیم دلی را به چه آرد که او
&&
پست کند صد دل فرزانه را
\\
از پگه امروز چه خوش مجلسیست
&&
آن صنم و فتنه فتانه را
\\
بشکند آن چشم تو صد عهد را
&&
مست کند زلف تو صد شانه را
\\
یک نفسی بام برآ ای صنم
&&
رقص درآر استن حنانه را
\\
شرح فتحنا و اشارات آن
&&
قفل بگوید سر دندانه را
\\
شاه بگوید شنود پیش من
&&
ترک کنم گفت غلامانه را
\\
\end{longtable}
\end{center}
