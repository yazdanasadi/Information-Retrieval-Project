\begin{center}
\section*{غزل شماره ۱۳۸۰: دی بر سرم تاج زری بنهاده است آن دلبرم}
\label{sec:1380}
\addcontentsline{toc}{section}{\nameref{sec:1380}}
\begin{longtable}{l p{0.5cm} r}
دی بر سرم تاج زری بنهاده است آن دلبرم
&&
چندانک سیلی می زنی آن می نیفتد از سرم
\\
شاه کله دوز ابد بر فرق من از فرق خود
&&
شب پوش عشق خود نهد پاینده باشد لاجرم
\\
ور سر نماند با کله من سر شوم جمله چو مه
&&
زیرا که بی‌حقه و صدف رخشانتر آید گوهرم
\\
اینک سر و گرز گران می زن برای امتحان
&&
ور بشکند این استخوان از عقل و جان مغزینترم
\\
آن جوز بی‌مغزی بود کو پوست بگزیده بود
&&
او ذوق کی دیده بود از لوزی پیغامبرم
\\
لوزینه پرجوز او پرشکر و پرلوز او
&&
شیرین کند حلق و لبم نوری نهد در منظرم
\\
چون مغز یابی ای پسر از پوست برداری نظر
&&
در کوی عیسی آمدی دیگر نگویی کو خرم
\\
ای جان من تا کی گله یک خر تو کم گیر از گله
&&
در زفتی فارس نگر نی بارگیر لاغرم
\\
زفتی عاشق را بدان از زفتی معشوق او
&&
زیرا که کبر عاشقان خیزد ز الله اکبرم
\\
ای دردهای آه گو اه اه مگو الله گو
&&
از چه مگو از جان گو ای یوسف جان پرورم
\\
\end{longtable}
\end{center}
