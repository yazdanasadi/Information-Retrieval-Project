\begin{center}
\section*{غزل شماره ۲۹۱۵: خوش بود گر کاهلی یک سو نهی}
\label{sec:2915}
\addcontentsline{toc}{section}{\nameref{sec:2915}}
\begin{longtable}{l p{0.5cm} r}
خوش بود گر کاهلی یک سو نهی
&&
وز همه یاران تو زوتر برجهی
\\
هست سرتیزی شعار شیر نر
&&
هست دم داری در این ره روبهی
\\
برفروز آتش زنه در دست توست
&&
یوسفت با توست اگر خود در چهی
\\
گر غروب آمد به گور اندرشدی
&&
باز طالع شو ز مشرق چون مهی
\\
گرم شد آن یخ ز جنبش بس گداخت
&&
پس بجنب ای قد تو سرو سهی
\\
برجهان تو اسب را ترکانه زود
&&
که به گوش توست خوب خرگهی
\\
سارعوا فرمود پس مردانه رو
&&
گفت شاهنشاه جان نبود تهی
\\
همچو زهره ناله کن هر صبحگاه
&&
وآنگه از خورشید بین شاهنشهی
\\
بدر هر شب در روش لاغرتر است
&&
بعد کاهش یافت آن مه فربهی
\\
وقت دوری شاه پروردت به لطف
&&
تا چه‌ها بخشد چو باشی درگهی
\\
بس کن آخر توبه کردی از مقال
&&
در خموشی‌هاست دخل آگهی
\\
\end{longtable}
\end{center}
