\begin{center}
\section*{غزل شماره ۲۸۴۹: ز بهار جان خبر ده هله ای دم بهاری}
\label{sec:2849}
\addcontentsline{toc}{section}{\nameref{sec:2849}}
\begin{longtable}{l p{0.5cm} r}
ز بهار جان خبر ده هله ای دم بهاری
&&
ز شکوفه‌هات دانم که تو هم ز وی خماری
\\
بشکف که من شکفتم تو بگو که من بگفتم
&&
صفت صفا و یاری ز جمال شهریاری
\\
اثری که هست باقی ز ورای وهم اکنون
&&
برود به آفتابی که فزود از شراری
\\
چو رسید نوبهاران بدرید زهره دی
&&
چو کسی به نزع افتد بزند دم شماری
\\
همه باغ دام گشته همه سبزفام گشته
&&
گل و لاله جام بر کف که هلا بیا چه داری
\\
گل و لاله‌ها چو دام‌اند و نظاره گر چو صیدی
&&
که شکوفه‌ها چو دام و همه میوه‌ها شکاری
\\
به سمن بگفت سوسن به دو چشم راست روشن
&&
که گذاشت خاک خاکی و گذاشت خار خاری
\\
صنما چه رنگ رنگی ز شراب لطف دنگی
&&
بر شاه عذرت این بس که خوشی و خوش عذاری
\\
رخ لاله برفروزان و رمان ز چشم نرگس
&&
که به چشم شوخ منگر به بتان به طبل خواری
\\
چو نسیم شاخه‌ها را به نشاط اندرآرد
&&
بوزد به دشت و صحرا دم نافه تتاری
\\
چو گذشت رنج و نقصان همه باغ گشت رقصان
&&
که ز بعد عسر یسری بگشاد فضل باری
\\
همه شاخه‌هاش رقصان همه گوشه‌هاش خندان
&&
چو دو دست نوعروسان همه دستشان نگاری
\\
همه مریمند گویی به دم فرشته حامل
&&
همه حوریند زاده ز میان خاک تاری
\\
چو بهشت جمله خوبان شب و روز پای کوبان
&&
سر و آستین فشانان ز نشاط بی‌قراری
\\
به بهار ابر گوید بدی ار نثار کردم
&&
جهت تو کردم آن هم که تو لایق نثاری
\\
به بهار بنگر ای دل که قیامت است مطلق
&&
بد و نیک بردمیده همه ساله هر چه کاری
\\
که بهار گوید ای جان دم خود چو دانه‌ها دان
&&
بنشان تو دانه دم که عوض درخت آری
\\
چو گشاد رازها را به بهار آشکارا
&&
چه کنی بدین نهانی که تو نیک آشکاری
\\
\end{longtable}
\end{center}
