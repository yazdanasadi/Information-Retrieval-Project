\begin{center}
\section*{غزل شماره ۲۸۳۹: بکشید یار گوشم که تو امشب آن مایی}
\label{sec:2839}
\addcontentsline{toc}{section}{\nameref{sec:2839}}
\begin{longtable}{l p{0.5cm} r}
بکشید یار گوشم که تو امشب آن مایی
&&
صنما بلی ولیکن تو نشان بده کجایی
\\
چو رها کنی بهانه بدهی نشان خانه
&&
به سر و دو دیده آیم که تو کان کیمیایی
\\
و اگر به حیله کوشی دغل و دغا فروشی
&&
ز فلک ستاره دزدی ز خرد کله ربایی
\\
شب من نشان مویت سحرم نشان رویت
&&
قمر از فلک درافتد چو نقاب برگشایی
\\
صنما تو همچو شیری من اسیر تو چو آهو
&&
به جهان کی دید صیدی که بترسد از رهایی
\\
صنما هوای ما کن طلب رضای ما کن
&&
که ز بحر و کان شنیدم که تو معدن عطایی
\\
همگی وبالم از تو به خدا بنالم از تو
&&
بنشان تکبرش را تو خدا به کبریایی
\\
ره خواب من چو بستی بمبند راه مستی
&&
ز همه جدام کردی مده از خودم جدایی
\\
مه و مهر یار ما شد به امید تو خدا شد
&&
که زهی امید زفتی که زند در خدایی
\\
همه مال و دل بداده سر کیسه برگشاده
&&
به امید کیسه تو که خلاصه وفایی
\\
همه را دکان شکسته ره خواب و خور ببسته
&&
به امید آن نشسته که ز گوشه‌ای درآیی
\\
به امید کس چه باشی که تویی امید عالم
&&
تو به گوش می چه باشی که تویی می عطایی
\\
به درون توست یوسف چه روی به مصر هرزه
&&
تو درآ درون پرده بنگر چه خوش لقایی
\\
به درون توست مطرب چه دهی کمر به مطرب
&&
نه کم است تن ز نایی نه کم است جان ز نایی
\\
\end{longtable}
\end{center}
