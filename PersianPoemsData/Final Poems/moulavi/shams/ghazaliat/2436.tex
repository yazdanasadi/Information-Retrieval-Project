\begin{center}
\section*{غزل شماره ۲۴۳۶: ای یار اگر نیکو کنی اقبال خود صدتو کنی}
\label{sec:2436}
\addcontentsline{toc}{section}{\nameref{sec:2436}}
\begin{longtable}{l p{0.5cm} r}
ای یار اگر نیکو کنی اقبال خود صدتو کنی
&&
تا بوک رو این سو کنی باشد که با ما خو کنی
\\
من گرد ره را کاستم آفاق را آراستم
&&
وز جرم تو برخاستم باشد که با ما خو کنی
\\
من از عدم زادم تو را بر تخت بنهادم تو را
&&
آیینه‌ای دادم تو را باشد که با ما خو کنی
\\
ای گوهری از کان من وی طالب فرمان من
&&
آخر ببین احسان من باشد که با ما خو کنی
\\
شرب مرا پیمانه شو وز خویشتن بیگانه شو
&&
با درد من همخانه شو باشد که با ما خو کنی
\\
ای شاه زاده داد کن خود را ز خود آزاد کن
&&
روز اجل را یاد کن باشد که با ما خو کنی
\\
مانند تیری از کمان بجهد ز تن سیمرغ جان
&&
آن را بیندیش ای فلان باشد که با ما خو کنی
\\
ای جمع کرده سیم و زر ای عاشق هر لب شکر
&&
باری بیا خوبی نگر باشد که با ما خو کنی
\\
تخم وفاها کاشتم نقشی عجب بنگاشتم
&&
بس پرده‌ها برداشتم باشد که با ما خو کنی
\\
استوثقوا ادیانکم و استغنموا اخوانکم
&&
و استعشقوا ایمانکم باشد که با ما خو کنی
\\
شه شمس تبریزی تو را گوید به پیش ما بیا
&&
بگذر ز زرق و از ریا باشد که با ما خو کنی
\\
\end{longtable}
\end{center}
