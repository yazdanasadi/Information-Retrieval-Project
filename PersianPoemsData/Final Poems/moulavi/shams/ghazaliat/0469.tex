\begin{center}
\section*{غزل شماره ۴۶۹: ای غم اگر مو شوی پیش منت بار نیست}
\label{sec:0469}
\addcontentsline{toc}{section}{\nameref{sec:0469}}
\begin{longtable}{l p{0.5cm} r}
ای غم اگر مو شوی پیش منت بار نیست
&&
پر شکرست این مقام هیچ تو را کار نیست
\\
غصه در آن دل بود کز هوس او تهیست
&&
غم همه آن جا رود کان بت عیار نیست
\\
ای غم اگر زر شوی ور همه شکر شوی
&&
بندم لب گویمت خواجه شکرخوار نیست
\\
در دل اگر تنگیست تنگ شکرهای اوست
&&
ور سفری در دلست جز بر دلدار نیست
\\
ای که تو بی‌غم نه‌ای می‌کن دفع غمش
&&
شاد شو از بوی یار کت نظر یار نیست
\\
ماه ازل روی او بیت و غزل بوی او
&&
بوی بود قسم آنک محرم دیدار نیست
\\
\end{longtable}
\end{center}
