\begin{center}
\section*{غزل شماره ۳۶۸: گویم سخن شکرنباتت}
\label{sec:0368}
\addcontentsline{toc}{section}{\nameref{sec:0368}}
\begin{longtable}{l p{0.5cm} r}
گویم سخن شکرنباتت
&&
یا قصه چشمه حیاتت
\\
رخ بر رخ من نهی بگویم
&&
کز بهر چه شاه کرد ماتت
\\
در خرمنت آتشی درانداخت
&&
کز خرمن خود دهد زکاتت
\\
سرسبز کند چو تره زارت
&&
تا بازخرد ز ترهاتت
\\
در آتش عشق چون خلیلی
&&
خوش باش که می‌دهد نجاتت
\\
عقلت شب قدر دید و صد عید
&&
کز عشق دریده شد براتت
\\
سوگند به سایه لطیفت
&&
سوگند نمی‌خورم به ذاتت
\\
در ذات تو کی رسند جان‌ها
&&
چون غرقه شدند در صفاتت
\\
چون جوی روان و ساجدت کرد
&&
تا پاک کند ز سیئاتت
\\
از هر جهتی تو را بلا داد
&&
تا بازکشد به بی‌جهاتت
\\
گفتی که خمش کنم نکردی
&&
می‌خندد عشق بر ثباتت
\\
\end{longtable}
\end{center}
