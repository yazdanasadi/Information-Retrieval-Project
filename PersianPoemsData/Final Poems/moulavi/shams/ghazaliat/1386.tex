\begin{center}
\section*{غزل شماره ۱۳۸۶: آمد بهار ای دوستان منزل به سروستان کنیم}
\label{sec:1386}
\addcontentsline{toc}{section}{\nameref{sec:1386}}
\begin{longtable}{l p{0.5cm} r}
آمد بهار ای دوستان منزل به سروستان کنیم
&&
تا بخت در رو خفته را چون بخت سرو استان کنیم
\\
همچون غریبان چمن بی‌پا روان گشته به فن
&&
هم بسته پا هم گام زن عزم غریبستان کنیم
\\
جانی که رست از خاکدان نامش روان آمد روان
&&
ما جان زانوبسته را هم منزل ایشان کنیم
\\
ای برگ قوت یافتی تا شاخ را بشکافتی
&&
چون رستی از زندان بگو تا ما در این حبس آن کنیم
\\
ای سرو بر سرور زدی تا از زمین سر ورزدی
&&
سر در چه سیر آموختت تا ما در آن سیران کنیم
\\
ای غنچه گلگون آمدی وز خویش بیرون آمدی
&&
با ما بگو چون آمدی تا ما ز خود خیزان کنیم
\\
آن رنگ عبهر از کجا وان بوی عنبر از کجا
&&
وین خانه را در از کجا تا خدمت دربان کنیم
\\
ای بلبل آمد داد تو من بنده فریاد تو
&&
تو شاد گل ما شاد تو کی شکر این احسان کنیم
\\
ای سبزپوشان چون خضر ای غیب‌ها گویان به سر
&&
تا حلقه گوش از شما پردر و پرمرجان کنیم
\\
بشنو ز گلشن رازها بی‌حرف و بی‌آوازها
&&
برساخت بلبل سازها گر فهم آن دستان کنیم
\\
آواز قمری تا قمر بررفت و طوطی بر شکر
&&
می آورد الحان تر جان مست آن الحان کنیم
\\
\end{longtable}
\end{center}
