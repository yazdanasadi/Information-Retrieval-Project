\begin{center}
\section*{غزل شماره ۱۵۰۹: سفر کردم به هر شهری دویدم}
\label{sec:1509}
\addcontentsline{toc}{section}{\nameref{sec:1509}}
\begin{longtable}{l p{0.5cm} r}
سفر کردم به هر شهری دویدم
&&
چو شهر عشق من شهری ندیدم
\\
ندانستم ز اول قدر آن شهر
&&
ز نادانی بسی غربت کشیدم
\\
رها کردم چنان شکرستانی
&&
چو حیوان هر گیاهی می چریدم
\\
پیاز و گندنا چون قوم موسی
&&
چرا بر من و سلوی برگزیدم
\\
به غیر عشق آواز دهل بود
&&
هر آوازی که در عالم شنیدم
\\
از آن بانگ دهل از عالم کل
&&
بدین دنیای فانی اوفتیدم
\\
میان جان‌ها جان مجرد
&&
چو دل بی‌پر و بی‌پا می پریدم
\\
از آن باده که لطف و خنده بخشد
&&
چو گل بی‌حلق و بی‌لب می چشیدم
\\
ندا آمد ز عشق ای جان سفر کن
&&
که من محنت سرایی آفریدم
\\
بسی گفتم که من آن جا نخواهم
&&
بسی نالیدم و جامه دریدم
\\
چنانک اکنون ز رفتن می گریزم
&&
از آن جا آمدن هم می رمیدم
\\
بگفت ای جان برو هر جا که باشی
&&
که من نزدیک چون حبل الوریدم
\\
فسون کرد و مرا بس عشوه‌ها داد
&&
فسون و عشوه او را خریدم
\\
فسون او جهان را برجهاند
&&
کی باشم من که من خود ناپدیدم
\\
ز راهم برد وان گاهم به ره کرد
&&
گر از ره می نرفتم می رهیدم
\\
بگویم چون رسی آن جا ولیکن
&&
قلم بشکست چون این جا رسیدم
\\
\end{longtable}
\end{center}
