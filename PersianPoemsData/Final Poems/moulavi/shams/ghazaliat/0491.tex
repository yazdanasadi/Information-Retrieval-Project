\begin{center}
\section*{غزل شماره ۴۹۱: جهان و کار جهان سر به سر اگر بادست}
\label{sec:0491}
\addcontentsline{toc}{section}{\nameref{sec:0491}}
\begin{longtable}{l p{0.5cm} r}
جهان و کار جهان سر به سر اگر بادست
&&
چرا ز باد مکافات داد و بیدادست
\\
به باد و بود محمد نگر که چون باقیست
&&
ز بعد ششصد و پنجاه سخت بنیادست
\\
ز باد بولهب و جنس او نمی‌بینی
&&
که از برای فضیحت فسانه شان یادست
\\
چنین ثبات و بقا باد را کجا باشد
&&
در این ثبات که قاف کمتر آحادست
\\
نبود باد دم عیسی و دعای عزیر
&&
عنایت ازلی بد که نورست ادست
\\
اگر چه باد سخن بگذرد سخن باقیست
&&
اگر چه باد صبا بگذرد چمن شادست
\\
ز بیم باد جهان همچو برگ می‌لرزد
&&
درون باد ندانی که تیغ پولادست
\\
کهی بود که به جز باد در جهان نشناخت
&&
کهی کهی نکند ز آنک که نه فرهادست
\\
تو باخبر نشوی گر کنم بسی فریاد
&&
که از درون دلم موج‌های فریادست
\\
اگر تو بحر ببینی و موج بر تو زند
&&
یقین شود که نه بادست ملک آبادست
\\
\end{longtable}
\end{center}
