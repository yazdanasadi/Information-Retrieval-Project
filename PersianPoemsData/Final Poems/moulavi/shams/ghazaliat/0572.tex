\begin{center}
\section*{غزل شماره ۵۷۲: ورای پرده جانت دلا خلقان پنهانند}
\label{sec:0572}
\addcontentsline{toc}{section}{\nameref{sec:0572}}
\begin{longtable}{l p{0.5cm} r}
ورای پرده جانت دلا خلقان پنهانند
&&
ز زخم تیغ فردیت همه جانند و بی‌جانند
\\
تو از نقصان و از بیشی نگویی چند اندیشی
&&
درآ در دین بی‌خویشی که بس بی‌خویش خویشانند
\\
چه دریاها که می‌نوشند چو دریاها همی‌جوشند
&&
اگر چه خود که خاموشند دانااند و می‌دانند
\\
در آن دریای پرمرجان یکی قومند همچون جان
&&
ورای گنبد گردان براق جان همی‌رانند
\\
ایا درویش باتمکین سبک دل گرد زوتر هین
&&
میان بزم مردان شین که ایشان جمله رندانند
\\
ملوکانند درویشان ز مستی جمله بی‌خویشان
&&
اگر چه خاکیند ایشان ولیکن شاه و سلطانند
\\
ز گنج عشق زر ریزند غلام شمس تبریزند
&&
و کان لعل و یاقوتند و در کان جان ارکانند
\\
\end{longtable}
\end{center}
