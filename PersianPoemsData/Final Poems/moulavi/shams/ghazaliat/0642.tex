\begin{center}
\section*{غزل شماره ۶۴۲: در خانه نشسته بت عیار کی دارد}
\label{sec:0642}
\addcontentsline{toc}{section}{\nameref{sec:0642}}
\begin{longtable}{l p{0.5cm} r}
در خانه نشسته بت عیار کی دارد
&&
معشوق قمرروی شکربار کی دارد
\\
بی زحمت دیده رخ خورشید که بیند
&&
بی پرده عیان طاقت دیدار کی دارد
\\
گفتی به خرابات دگر کار ندارم
&&
خود کار تو داری و دگر کار کی دارد
\\
زندان صبوحی همه مخمور خمارند
&&
ای زهره کلید در خمار کی دارد
\\
ما طوطی غیبیم شکرخواره و عاشق
&&
آن کان شکرهای به قنطار کی دارد
\\
یک غمزه دیدار به از دامن دینار
&&
دیدار چو باشد غم دینار کی دارد
\\
جان‌ها چو از آن شیر ره صید بدیدند
&&
اکنون چو سگان میل به مردار کی دارد
\\
چون عین عیانست ز اقرار کی لافد
&&
اقرار چو کاسد شود انکار کی دارد
\\
ای در رخ تو زلزله روز قیامت
&&
در جنت حسن تو غم نار کی دارد
\\
با غمزه غمازه آن یار وفادار
&&
اندیشه این عالم غدار کی دارد
\\
گفتی که ز احوال عزیزان خبری ده
&&
با مخبر خوبت سر اخبار کی دارد
\\
ای مطرب خوش لهجه شیرین دم عارف
&&
یاری ده و برگو که چنین یار کی دارد
\\
بازار بتان از تو خرابست و کسادست
&&
بازار چه باشد دل بازار کی دارد
\\
امروز ز سودای تو کس را سر سر نیست
&&
دستار کی دارد سر دستار کی دارد
\\
شمس الحق تبریز چو نقد آمد و پیدا
&&
از پار کی گوید غم پیرار کی دارد
\\
\end{longtable}
\end{center}
