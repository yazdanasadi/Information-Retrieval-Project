\begin{center}
\section*{غزل شماره ۹۸۳: هر کی در ذوق عشق دنگ آمد}
\label{sec:0983}
\addcontentsline{toc}{section}{\nameref{sec:0983}}
\begin{longtable}{l p{0.5cm} r}
هر کی در ذوق عشق دنگ آمد
&&
نیک فارغ ز نام و ننگ آمد
\\
نشود بند گفت و گوی جهان
&&
شیرگیری که چون پلنگ آمد
\\
شیشه عشق را فراغت‌ها است
&&
گر بر او صد هزار سنگ آمد
\\
نام و ناموس کی شود مانع
&&
چونک آن دلربای شنگ آمد
\\
صد هزاران چو آسمان و زمین
&&
پیش جولان عشق تنگ آمد
\\
قیصر روم عشق غالب باد
&&
گر کسل چون سپاه زنگ آمد
\\
زهره بر چنگ این نوا می‌زد
&&
کان قمر عاقبت به چنگ آمد
\\
شمس تبریز هر کی بی‌تو نشست
&&
عذر او پیش عشق لنگ آمد
\\
\end{longtable}
\end{center}
