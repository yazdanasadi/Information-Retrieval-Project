\begin{center}
\section*{غزل شماره ۲۱۷۶: آن دلبر عیار جگرخواره ما کو}
\label{sec:2176}
\addcontentsline{toc}{section}{\nameref{sec:2176}}
\begin{longtable}{l p{0.5cm} r}
آن دلبر عیار جگرخواره ما کو
&&
آن خسرو شیرین شکرپاره ما کو
\\
بی‌صورت او مجلس ما را نمکی نیست
&&
آن پرنمک و پرفن و عیاره ما کو
\\
باریک شده‌ست از غم او ماه فلک نیز
&&
آن زهره بابهره سیاره ما کو
\\
پربسته چو هاروتم و لب تشنه چو ماروت
&&
آن رشک چه بابل سحاره ما کو
\\
موسی که در این خشک بیابان به عصایی
&&
صد چشمه روان کرد از این خاره ما کو
\\
زین پنج حسن ظاهر و زین پنج حسن سر
&&
ده چشمه گشاینده در این قاره ما کو
\\
از فرقت آن دلبر دردی است در این دل
&&
آن داروی درد دل و آن چاره ما کو
\\
استاره روز او است چو بر می‌ندمد صبح
&&
گویم که بدم گوید کاستاره ما کو
\\
اندر ظلمات است خضر در طلب آب
&&
کان عین حیات خوش فواره ما کو
\\
جان همچو مسیحی است به گهواره قالب
&&
آن مریم بندنده گهواره ما کو
\\
آن عشق پر از صورت بی‌صورت عالم
&&
هم دوز ز ما هم زه قواره ما کو
\\
هر کنج یکی پرغم مخمور نشسته‌ست
&&
کان ساقی دریادل خماره ما کو
\\
آن زنده کن این در و دیوار بدن کو
&&
و آن رونق سقف و در و درساره ما کو
\\
لوامه و اماره بجنگند شب و روز
&&
جنگ افکن لوامه و اماره ما کو
\\
ما مشت گلی در کف قدرت متقلب
&&
از غفلت خود گفته که گل کاره ما کو
\\
شمس الحق تبریز کجا رفت و کجا نیست
&&
و اندر پی او آن دل آواره ما کو
\\
\end{longtable}
\end{center}
