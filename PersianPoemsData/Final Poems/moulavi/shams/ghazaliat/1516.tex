\begin{center}
\section*{غزل شماره ۱۵۱۶: چه نزدیک است جان تو به جانم}
\label{sec:1516}
\addcontentsline{toc}{section}{\nameref{sec:1516}}
\begin{longtable}{l p{0.5cm} r}
چه نزدیک است جان تو به جانم
&&
که هر چیزی که اندیشی بدانم
\\
ضمیر همدگر دانند یاران
&&
نباشم یار صادق گر ندانم
\\
چو آب صاف باشد یار با یار
&&
که بنماید در او عکس بنانم
\\
اگر چه عامه هم آیینه‌هااند
&&
که بنماید در او سود و زیانم
\\
ولیکن آن به هر دم تیره گردد
&&
که او را نیست صیقل‌های جانم
\\
ولی آیینه ای عارف نگردد
&&
اگر خاک جهان بر وی فشانم
\\
از این آیینه روی خود مگردان
&&
که می گوید که جانت را امانم
\\
من و گفت من آیینه‌ست جان را
&&
بیابد حال خویش اندر بیانم
\\
خمش کن تا به ابرو و به غمزه
&&
هزاران ماجرا بر وی بخوانم
\\
\end{longtable}
\end{center}
