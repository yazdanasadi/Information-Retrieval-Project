\begin{center}
\section*{غزل شماره ۲۵۵۰: یکی دودی پدید آمد سحرگاهی به هامونی}
\label{sec:2550}
\addcontentsline{toc}{section}{\nameref{sec:2550}}
\begin{longtable}{l p{0.5cm} r}
یکی دودی پدید آمد سحرگاهی به هامونی
&&
دل عشاق چون آتش تن عشاق کانونی
\\
بیا بخرام و دامن کش در آن دود و در آن آتش
&&
که می‌سوزد در آن جا خوش به هر اطراف ذاالنونی
\\
چو شمعی برفروزی تو ایا اقبال و روزی تو
&&
چو چونی را بسوزی تو درآید جان بی‌چونی
\\
نیاید جز ز مه رویی طواف برج‌ها کردن
&&
که مادون را رها کردن نباشد کار هر دونی
\\
برو تو دست اندازان به سوی شاه چون باران
&&
ببینی بحر را تازان در آن بحر پر از خونی
\\
چه لاله است و گل و ریحان از آن خون رسته در بستان
&&
ببینی و بشوید جان دو دست خود به صابونی
\\
چو دررفتی در آن مخزن منزه از در و روزن
&&
چو عیسی سوزنت گردد حجب چون گنج قارونی
\\
ببینی شاه قدوسی بیابی بی‌دهن بوسی
&&
ز سر خضر چون موسی شوی در فقر هارونی
\\
چو آبی ساکن و خفته و چون موجی برآشفته
&&
به بحر کم زنان رفته شده اندر کم افزونی
\\
چو اندر شه نظر کردی ز مستی آن چنان گردی
&&
که گویی تو مگر خوردی هزاران رطل افیونی
\\
چو دیدی شمس تبریزی ز جان کردی شکرریزی
&&
در آن دم هر دو جا باشی درون مصر و بیرونی
\\
\end{longtable}
\end{center}
