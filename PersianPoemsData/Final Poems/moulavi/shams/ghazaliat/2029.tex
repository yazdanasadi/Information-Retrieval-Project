\begin{center}
\section*{غزل شماره ۲۰۲۹: ای مرغ آسمانی آمد گه پریدن}
\label{sec:2029}
\addcontentsline{toc}{section}{\nameref{sec:2029}}
\begin{longtable}{l p{0.5cm} r}
ای مرغ آسمانی آمد گه پریدن
&&
وی آهوی معانی آمد گه چریدن
\\
ای عاشق جریده بر عاشقان گزیده
&&
بگذر ز آفریده بنگر در آفریدن
\\
آمد تو را فتوحی روحی چگونه روحی
&&
کو چون خیال داند در دیده‌ها دویدن
\\
این دم حکم بیاید تعلیم نو نماید
&&
بی‌گوش سر شنیدن بی‌دیده ماه دیدن
\\
داند سبل ببردن هم مرده زنده کردن
&&
هم تخت و بخت دادن هم بنده پروریدن
\\
آن یوسف معانی و آن گنج رایگانی
&&
خود را اگر فروشد دانی عجب خریدن
\\
کو مشتری واقف در دو دم مخالف
&&
در پرده ساز کردن در پرده‌ها دویدن
\\
ای عاشق موفق وی صادق مصدق
&&
می‌بایدت چو گردون بر قطب خود تنیدن
\\
در بیخودی تو خود را می‌جوی تا بیابی
&&
زیرا فراق صعب است خاصه ز حق بریدن
\\
لب را ز شیر شیطان می‌کوش تا بشویی
&&
چون شسته شد توانی پستان دل مکیدن
\\
ای عشق آن جهانی ما را همی‌کشانی
&&
احسنت ای کشنده شاباش ای کشیدن
\\
هم آفتاب داند از شرق رو نمودن
&&
ار نی به مرکز او نتوان به تک رسیدن
\\
خامش که شرح دل را گر راه گفت بودی
&&
در کوه درفتادی چون بحر برطپیدن
\\
تبریز شمس دین را هم ناگهان ببینی
&&
وآنگه از او بیابی صبح ابد دمیدن
\\
\end{longtable}
\end{center}
