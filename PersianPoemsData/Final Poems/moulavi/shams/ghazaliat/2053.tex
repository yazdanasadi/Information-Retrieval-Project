\begin{center}
\section*{غزل شماره ۲۰۵۳: با عاشقان نشین و همه عاشقی گزین}
\label{sec:2053}
\addcontentsline{toc}{section}{\nameref{sec:2053}}
\begin{longtable}{l p{0.5cm} r}
با عاشقان نشین و همه عاشقی گزین
&&
با آنک نیست عاشق یک دم مشو قرین
\\
ور ز آنک یار پرده عزت فروکشید
&&
آن را که پرده نیست برو روی او ببین
\\
آن روی بین که بر رخش آثار روی او است
&&
آن را نگر که دارد خورشید بر جبین
\\
از بس که آفتاب دو رخ بر رخش نهاد
&&
شهمات می‌شود ز رخش ماه بر زمین
\\
در طره‌هاش نسخه ایاک نعبد است
&&
در چشم‌هاش غمزه ایاک نستعین
\\
بی‌خون و بی‌رگ است تنش چون تن خیال
&&
بیرون و اندرون همه شیر است و انگبین
\\
از بس که در کنار همی‌گیردش نگار
&&
بگرفت بوی یار و رها کرد بوی طین
\\
صبحی است بی‌سپیده و شامی است بی‌خضاب
&&
ذاتی است بی‌جهات و حیاتی است بی‌حنین
\\
کی نور وام خواهد خورشید از سپهر
&&
کی بوی وام خواهد گلبن ز یاسمین
\\
بی‌گفت شو چو ماهی و صافی چو آب بحر
&&
تا زود بر خزینه گوهر شوی امین
\\
در گوش تو بگویم با هیچ کس مگو
&&
این جمله کیست مفتخر تبریز شمس دین
\\
\end{longtable}
\end{center}
