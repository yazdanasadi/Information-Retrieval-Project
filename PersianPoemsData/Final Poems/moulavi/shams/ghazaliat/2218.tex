\begin{center}
\section*{غزل شماره ۲۲۱۸: همه خوردند و برفتند و بماندم من و تو}
\label{sec:2218}
\addcontentsline{toc}{section}{\nameref{sec:2218}}
\begin{longtable}{l p{0.5cm} r}
همه خوردند و برفتند و بماندم من و تو
&&
چو مرا یافته‌ای صحبت هر خام مجو
\\
همه سرسبزی جان تو ز اقبال دل است
&&
هله چون سبزه و چون بید مرو زین لب جو
\\
پر شود خانه دل ماه رخان زیبا
&&
گرهی همچو زلیخا گرهی یوسف رو
\\
حلقه حلقه بر او رقص کنان دست زنان
&&
سوی او خنبد هر یک که منم بنده تو
\\
هر ضمیری که در او آن شه تشریف دهد
&&
هر سوی باغ بود هر طرفی مجلس و طو
\\
چند هنگامه نهی هر طرفی بهر طمع
&&
تو پراکنده شدی جمع نشد نیم تسو
\\
هله ای عشق که من چاکر و شاگرد توام
&&
که بسی خوب و لطیف است تو را صورت و خو
\\
گر می مجلسی و آب حیات همه‌ای
&&
همه دل گشته و فارغ شده از فرج و گلو
\\
هله ای دل که ز من دیده تو تیزتر است
&&
عجب آن کیست چو شمس و چو قمر بر سر کو
\\
آنک در زلزله او است دو صد چون مه و چرخ
&&
و آنک که در سلسله او است دو صد سلسله مو
\\
هفت بحر ار بفزایند و به هفتاد رسند
&&
بود او را به گه عبره به زیر زانو
\\
او مگر صورت عشق است و نماند به بشر
&&
خسروان بر در او گشته ایاز و قتلو
\\
فلک و مهر و ستاره لمع از وی دزدند
&&
یوسف و پیرهنش برده از او صورت و بو
\\
همه شیران بده در حمله او چون سگ لنگ
&&
همه ترکان شده زیبایی او را هندو
\\
لب ببند و صفت لعل لب او کم کن
&&
همه هیچند به پیش لب او هیچ مگو
\\
\end{longtable}
\end{center}
