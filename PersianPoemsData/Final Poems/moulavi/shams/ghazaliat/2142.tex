\begin{center}
\section*{غزل شماره ۲۱۴۲: چون بجهد خنده ز من خنده نهان دارم از او}
\label{sec:2142}
\addcontentsline{toc}{section}{\nameref{sec:2142}}
\begin{longtable}{l p{0.5cm} r}
چون بجهد خنده ز من خنده نهان دارم از او
&&
روی ترش سازم از او بانگ و فغان آرم از او
\\
با ترشان لاغ کنی خنده زنی جنگ شود
&&
خنده نهان کردم من اشک همی‌بارم از او
\\
شهر بزرگ است تنم غم طرفی من طرفی
&&
یک طرفی آبم از او یک طرفی نارم از او
\\
با ترشانش ترشم با شکرانش شکرم
&&
روی من او پشت من او پشت طرب خارم از او
\\
صد چو تو و صد چو منش مست شده در چمنش
&&
رقص کنان دست زنان بر سر هر طارم از او
\\
طوطی قند و شکرم غیر شکر می نخورم
&&
هر چه به عالم ترشی دورم و بیزارم از او
\\
گر ترشی داد تو را شهد و شکر داد مرا
&&
سکسک و لنگی تو از او من خوش و رهوارم از او
\\
هر کی در این ره نرود دره و دوله‌ست رهش
&&
من که در این شاه رهم بر ره هموارم از او
\\
مسجد اقصاست دلم جنت مأواست دلم
&&
حور شده نور شده جمله آثارم از او
\\
هر کی حقش خنده دهد از دهنش خنده جهد
&&
تو اگر انکاری از او من همه اقرارم از او
\\
قسمت گل خنده بود گریه ندارد چه کند
&&
سوسن و گل می‌شکفد در دل هشیارم از او
\\
صبر همی‌گفت که من مژده ده وصلم از او
&&
شکر همی‌گفت که من صاحب انبارم از او
\\
عقل همی‌گفت که من زاهد و بیمارم از او
&&
عشق همی‌گفت که من ساحر و طرارم از او
\\
روح همی‌گفت که من گنج گهر دارم از او
&&
گنج همی‌گفت که من در بن دیوارم از او
\\
جهل همی‌گفت که من بی‌خبرم بیخود از او
&&
علم همی‌گفت که من مهتر بازارم از او
\\
زهد همی‌گفت که من واقف اسرارم از او
&&
فقر همی‌گفت که من بی‌دل و دستارم از او
\\
از سوی تبریز اگر شمس حقم بازرسد
&&
شرح شود کشف شود جمله گفتارم از او
\\
\end{longtable}
\end{center}
