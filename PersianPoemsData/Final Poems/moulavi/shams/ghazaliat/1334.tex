\begin{center}
\section*{غزل شماره ۱۳۳۴: این بوالعجب کاندر خزان شد آفتاب اندر حمل}
\label{sec:1334}
\addcontentsline{toc}{section}{\nameref{sec:1334}}
\begin{longtable}{l p{0.5cm} r}
این بوالعجب کاندر خزان شد آفتاب اندر حمل
&&
خونم به جوش آمد کند در جوی تن رقص الجمل
\\
این رقص موج خون نگر صحرا پر از مجنون نگر
&&
وین عشرت بی‌چون نگر ایمن ز شمشیر اجل
\\
مردار جانی می‌شود پیری جوانی می‌شود
&&
مس زر کانی می‌شود در شهر ما نعم البدل
\\
شهری پر از عشق و فرح بر دست هر مستی قدح
&&
این سوی نوش آن سوی صح این جوی شیر و آن عسل
\\
در شهر یک سلطان بود وین شهر پرسلطان عجب
&&
بر چرخ یک ماهست بس وین چرخ پرماه و زحل
\\
رو رو طبیبان را بگو کان جا شما را کار نیست
&&
کان جا نباشد علتی وان جا نبیند کس خلل
\\
نی قاضیی نی شحنه‌ای نی میر شهر و محتسب
&&
بر آب دریا کی رود دعوی و خصمی و جدل
\\
\end{longtable}
\end{center}
