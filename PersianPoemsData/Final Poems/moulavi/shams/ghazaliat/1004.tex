\begin{center}
\section*{غزل شماره ۱۰۰۴: دوش دل عربده گر با کی بود}
\label{sec:1004}
\addcontentsline{toc}{section}{\nameref{sec:1004}}
\begin{longtable}{l p{0.5cm} r}
دوش دل عربده گر با کی بود
&&
مشت کی کردست دو چشمش کبود
\\
آن دل پرخواره ز عشق شراب
&&
هفت قدح از دگران برفزود
\\
مست شد و بر سر کوی اوفتاد
&&
دست زنان ناگه خوابش ربود
\\
آن عسسی رفت قبایش ببرد
&&
وان دگری شد کمرش را گشود
\\
آمد چنگی بنوازید تار
&&
جست ز خواب آن دل بی‌تار و پود
\\
دید قبا رفته خمارش نماند
&&
دید زیان کم شد سودای سود
\\
دیدش ساقی که در آتش فتاد
&&
جام گرفت و سوی او شد چو دود
\\
بر غم او ریخت می دلگشا
&&
صورت اقبال بدو رو نمود
\\
بخت بقا یافت قبا گو برو
&&
ذوق فنا دید چه جوید وجود
\\
عالم ویرانه به جغدان حلال
&&
باد دو صد شنبه از آن جهود
\\
ما چو خرابیم و خراباتییم
&&
خیز قدح پر کن و پیش آر زود
\\
این قدح از لطف نیاید به چشم
&&
جسم نداند می جان آزمود
\\
زان سوی گوش آمد این طبل عید
&&
در دلش آتش بزن افغان عود
\\
بس کن و اندر تتق عشق رو
&&
دلبر خوبست و هزاران حسود
\\
\end{longtable}
\end{center}
