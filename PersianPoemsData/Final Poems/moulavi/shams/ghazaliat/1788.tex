\begin{center}
\section*{غزل شماره ۱۷۸۸: تا کی گریزی از اجل در ارغوان و ارغنون}
\label{sec:1788}
\addcontentsline{toc}{section}{\nameref{sec:1788}}
\begin{longtable}{l p{0.5cm} r}
تا کی گریزی از اجل در ارغوان و ارغنون
&&
نک کش کشانت می برند انا الیه راجعون
\\
تا کی زنی بر خانه‌ها تو قفل با دندانه‌ها
&&
تا چند چینی دانه‌ها دام اجل کردت زبون
\\
شد اسب و زین نقره گین بر مرکب چوبین نشین
&&
زین بر جنازه نه ببین دستان این دنیای دون
\\
برکن قبا و پیرهن تسلیم شو اندر کفن
&&
بیرون شو از باغ و چمن ساکن شو اندر خاک و خون
\\
دزدیده چشمک می زدی همراز خوبان می شدی
&&
دستک زنان می آمدی کو یک نشان ز آن‌ها کنون
\\
ای کرده بر پاکان زنخ امروز بستندت زنخ
&&
فرزند و اهل و خانه‌ات از خانه کردندت برون
\\
کو عشرت شب‌های تو کو شکرین لب‌های تو
&&
کو آن نفس کز زیرکی بر ماه می خواندی فسون
\\
کو صرفه و استیزه‌ات بر نان و بر نان ریزه‌ات
&&
کو طوق و کو آویزه‌ات ای در شکافی سرنگون
\\
کو آن فضولی‌های تو کو آن ملولی‌های تو
&&
کو آن نغولی‌های تو در فعل و مکر ای ذوفنون
\\
این باغ من آن خان من این آن من آن آن من
&&
ای هر منت هفتاد من اکنون کهی از تو فزون
\\
کو آن دم دولت زدن بر این و آن سبلت زدن
&&
کو حمله‌ها و مشت تو وان سرخ گشتن از جنون
\\
هرگز شبی تا روز تو در توبه و در سوز تو
&&
نابوده مهراندوز تو از خالق ریب المنون
\\
امروز ضربت‌ها خوری وز رفته حسرت‌ها خوری
&&
زان اعتقاد سرسری زان دین سست بی‌سکون
\\
زان سست بودن در وفا بیگانه بودن با خدا
&&
زان ماجرا با انبیا کاین چون بود ای خواجه چون
\\
چون آینه باش ای عمو خوش بی‌زبان افسانه گو
&&
زیرا که مستی کم شود چون ماجرا گردد شجون
\\
\end{longtable}
\end{center}
