\begin{center}
\section*{غزل شماره ۴۵۲: ساقی و سردهی ز لب یارم آرزوست}
\label{sec:0452}
\addcontentsline{toc}{section}{\nameref{sec:0452}}
\begin{longtable}{l p{0.5cm} r}
ساقی و سردهی ز لب یارم آرزوست
&&
بدمستی ز نرگس خمارم آرزوست
\\
هندوی طره‌ات چه رسن باز لولییست
&&
لولی گری طره طرارم آرزوست
\\
اندر دلم ز غمزه غماز فتنه‌هاست
&&
فتنه نشان جادوی بیمارم آرزوست
\\
زان رو که غدرها و دغاهاش بس خوش‌ست
&&
غدرش مرا بسوزد غدارم آرزوست
\\
زان شمع بی‌نظیر که در لامکان بتافت
&&
پروانه وار سوخته هموارم آرزوست
\\
گلزار حسن رو بگشا زانک از رخت
&&
مه شرمسار گشته و گلزارم آرزوست
\\
بعد از چهار سال نشستیم دو به دو
&&
یک ره به کوی وصل تو دوچارم آرزوست
\\
انکار کرد عقل تو وین کار کرده عشق
&&
انکار سود نیست چو این کارم آرزوست
\\
رانیم بالش شه و رانی به زخم مار
&&
با مصطفای حسن در آن غارم آرزوست
\\
تاتار هجر کرد سیاهی و عنبری
&&
زان مشک‌های آهوی تاتارم آرزوست
\\
باریست بر دلم که مرا هیچ بار نیست
&&
ای شاه بار ده که یکی بارم آرزوست
\\
عارست ای خفاش تو را ناز آفتاب
&&
صد سجده من بکرده بر آن عارم آرزوست
\\
با داردار وعده وصلت رسید صبر
&&
هجران دو چشم بسته و بر دارم آرزوست
\\
هست این سپاه عشق تو جان سوز و دلفروز
&&
و اندر سپاه عشق تو سالارم آرزوست
\\
دجال هجر بر سرم از غم قیامتیست
&&
لابد فسون عیسی و تیمارم آرزوست
\\
مکری بکرد بنده و مکری بکرد وصل
&&
از مکر توبه کردم مکارم آرزوست
\\
تا سوی گلشن طرب آیم خراب و مست
&&
از گلشن وصال تو یک خارم آرزوست
\\
زان طره‌های زلف کمرساز بنده را
&&
کز شهر دررمیدم کهسارم آرزوست
\\
موسی جان بدید درختی ز نور نار
&&
آن شعله درخت و از آن نارم آرزوست
\\
تبریز چون بهشت ز دیدار شمس دین
&&
اندر بهشت رفته و دیدارم آرزوست
\\
\end{longtable}
\end{center}
