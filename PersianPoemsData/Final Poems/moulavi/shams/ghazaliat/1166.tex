\begin{center}
\section*{غزل شماره ۱۱۶۶: چند از این راه نو روزگار}
\label{sec:1166}
\addcontentsline{toc}{section}{\nameref{sec:1166}}
\begin{longtable}{l p{0.5cm} r}
چند از این راه نو روزگار
&&
پرده آن یار قدیمی بیار
\\
آتش فرعون بکش ز آب بحر
&&
مفرش نمرود به آتش سپار
\\
چرخ فلک را به خدایی مگیر
&&
انجم و مه را مشناس اختیار
\\
شمس و شموسی که سرآخر شدست
&&
چون خر لنگست در آن مستدار
\\
باد چو راکع شد و خود را شناخت
&&
نیست در آخر چو خسان بی‌مدار
\\
چشم در آن باد نهادست خس
&&
کو کشدش جانب هر دشت و غار
\\
خیره در آن آب بماندست سنگ
&&
کوش بغلطاند در سیل بار
\\
گر بد و نیکیم تو از ما مگیر
&&
ما همه چنگیم و دل ما چو تار
\\
گاه یکی نغمه تر می‌نواز
&&
گاه ز تر بگذر و رو خشک آر
\\
گر ننوازی دل این چنگ را
&&
بس بود اینش که نهی برکنار
\\
نور علی نور چو بنوازیش
&&
باده خوش و خاصه به فصل بهار
\\
در کف عشقست مهار همه
&&
اشتر مستیم در این زیر بار
\\
گاه چو شیری متمثل شود
&&
تا برمد خلق از او چون شکار
\\
گاه چو آبی متشکل شود
&&
خلق رود تشنه بدو جان سپار
\\
\end{longtable}
\end{center}
