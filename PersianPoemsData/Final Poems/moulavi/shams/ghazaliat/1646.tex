\begin{center}
\section*{غزل شماره ۱۶۴۶: روز آن است که ما خویش بر آن یار زنیم}
\label{sec:1646}
\addcontentsline{toc}{section}{\nameref{sec:1646}}
\begin{longtable}{l p{0.5cm} r}
روز آن است که ما خویش بر آن یار زنیم
&&
نظری سیر بر آن روی چو گلنار زنیم
\\
مشتری وار سر زلف مه خود گیریم
&&
فتنه و غلغله اندر همه بازار زنیم
\\
اندرافتیم در آن گلشن چون باد صبا
&&
همه بر جیب گل و جعد سمن زار زنیم
\\
نفسی کوزه زنیم و نفسی کاسه خوریم
&&
تا سبووار همه بر خم خمار زنیم
\\
تا به کی نامه بخوانیم گه جام رسید
&&
نامه را یک نفسی در سر دستار زنیم
\\
چنگ اقبال ز فر رخ تو ساخته شد
&&
واجب آید که دو سه زخمه بر آن تار زنیم
\\
وقت شور آمد و هنگام نگه داشت نماند
&&
ما که مستیم چه دانیم چه مقدار زنیم
\\
خاک زر می شود اندر کف اخوان صفا
&&
خاک در دیده این عالم غدار زنیم
\\
می کشانند سوی میمنه ما را به طناب
&&
خیمه عشرت از این بار در اسرار زنیم
\\
شد جهان روشن و خوش از رخ آتشرویی
&&
خیز تا آتش در مکسبه و کار زنیم
\\
پاره پاره شود و زنده شود چون که طور
&&
گر ز برق دل خود بر که و کهسار زنیم
\\
هله باقیش تو گو که به وجود چو توی
&&
سرد و حیف است که ما حلقه گفتار زنیم
\\
\end{longtable}
\end{center}
