\begin{center}
\section*{غزل شماره ۲۸۸۸: به شکرخنده اگر می‌ببرد جان ز کسی}
\label{sec:2888}
\addcontentsline{toc}{section}{\nameref{sec:2888}}
\begin{longtable}{l p{0.5cm} r}
به شکرخنده اگر می‌ببرد جان ز کسی
&&
می‌دهد جان خوشی پرطربی پرهوسی
\\
گه سحر حمله برد بر همه چون خورشیدی
&&
گه به شب گشت کند بر دل و جان چون عسسی
\\
گه یکی تنگ شکربار کند بهر نثار
&&
گه شود طوطی جان گر بچشد زان مگسی
\\
گه مدرس شود و درس کند بر سر صدر
&&
تا شود کن فیکون صدر جهان مرتبسی
\\
گه دمد یک نفسی عیسی مریم سازد
&&
تا گواه نفسش باشد عیسی نفسی
\\
گه خسی را بکشد سرمه جان در دیده
&&
گه نماید دو جهان در نظرش همچو خسی
\\
متزمن نظری داری و هرچ آید پیش
&&
هم بر آن چفسد و حمله نبرد پیش و پسی
\\
صالح او آمد و این هر دو جهان یک اشتر
&&
ما همه نعره زنان زنگله همچون جرسی
\\
\end{longtable}
\end{center}
