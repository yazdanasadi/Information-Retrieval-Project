\begin{center}
\section*{غزل شماره ۳۰۵۳: ز قیل و قال تو گر خلق بو نبردندی}
\label{sec:3053}
\addcontentsline{toc}{section}{\nameref{sec:3053}}
\begin{longtable}{l p{0.5cm} r}
ز قیل و قال تو گر خلق بو نبردندی
&&
ز حسرت و ز فراقت همه بمردندی
\\
ز جان خویش اگر بوی تو نیابندی
&&
چو استخوان دل و جان را به سگ سپردندی
\\
اگر نه پرتو لطفت بر آب می‌تابید
&&
به جای آب همه زهر ناب خوردندی
\\
اگر نه جرعه آن می بریختی بر خاک
&&
ستارگان ز چه رو گرد خاک گردندی
\\
گر آفتاب ازل گرمیی نبخشیدی
&&
تموز و جمله نباتان او فسردندی
\\
منزهی و درآمیختن عجب صفتی است
&&
دریغ پرده اسرار درنوردندی
\\
اگر نه پرده بدی ره روان پنهانی
&&
ز انبهی همه پاهای ما فشردندی
\\
ز پرده‌ها اگر آن روح قدس بنمودی
&&
عقول و جان بشر را بدن شمردندی
\\
گر آن بدی که تو اندیشه کرده‌ای ز زحیر
&&
بتان و لاله رخان جمله زار و زردندی
\\
چو صورتی نبدی خوب جز تصور تو
&&
شراب‌های مروق ز درد دردندی
\\
اگر خمش کنمی راز عشق فهم شدی
&&
وگر چه خلق همه هند و ترک و کردندی
\\
\end{longtable}
\end{center}
