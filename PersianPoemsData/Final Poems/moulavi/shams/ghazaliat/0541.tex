\begin{center}
\section*{غزل شماره ۵۴۱: صرفه مکن صرفه مکن صرفه گدارویی بود}
\label{sec:0541}
\addcontentsline{toc}{section}{\nameref{sec:0541}}
\begin{longtable}{l p{0.5cm} r}
صرفه مکن صرفه مکن صرفه گدارویی بود
&&
در پاکبازان ای پسر فیض و خداخویی بود
\\
خود عاقبت اندر ولا نی بخل ماند نی سخا
&&
اندر سخا هم بی‌شکی پنهان عوض جویی بود
\\
هست این سخا چون سیر ره وین بخل منزل کردنت
&&
در کشتی نوح آمدی کی وقف و ره‌پویی بود
\\
حاصل عصای موسوی عشقست در کون ای روی
&&
عین و عرض در پیش او اشکال جادویی بود
\\
یک سو رو از گرداب تن پیش از دم غرقه شدن
&&
زیرا بقا و خرمی زان سوی شش سویی بود
\\
خود را بیفشان چون شجر از برگ خشک و برگ تر
&&
بی رنگ نیک و رنگ بد توحید و یک تویی بود
\\
ره رو مگو این چون بود زیرا ز چون بیرون بود
&&
کی شیر را همدم شوی تا در تو آهویی بود
\\
خاموش کاین گفت زبان دارد نشان فرقتی
&&
ور نی چو نان خاید فتی کی وقت نان گویی بود
\\
\end{longtable}
\end{center}
