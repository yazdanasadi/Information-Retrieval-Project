\begin{center}
\section*{غزل شماره ۱۷۱۵: هر کی بمیرد شود دشمن او دوستکام}
\label{sec:1715}
\addcontentsline{toc}{section}{\nameref{sec:1715}}
\begin{longtable}{l p{0.5cm} r}
هر کی بمیرد شود دشمن او دوستکام
&&
دشمنم از مرگ من کور شود والسلام
\\
آن شکرستان مرا می کشد اندر شکر
&&
ای که چنین مرگ را جان و دل من غلام
\\
در غلط افکنده‌ست نام و نشان خلق را
&&
عمر شکربسته را مرگ نهادند نام
\\
از جهت این رسول گفت که الفقر کنز
&&
فقر کند نام گنج تا غلط افتند عام
\\
وحی در ایشان بود گنج به ویران بود
&&
تا که زر پخته را ره نبرد هیچ خام
\\
گفتم ای جان ببین زین دلم سست تنگ
&&
گفت که زین پس ز جهل وامکش از پس لگام
\\
تا که سرانجام تو گردد بر کام تو
&&
توسن خنگ فلک باشد زیر تو رام
\\
گر تو بدانی که مرگ دارد صد باغ و برگ
&&
هست حیات ابد جوییش از جان مدام
\\
خامش کن لب ببند بی‌دهنی خای قند
&&
نیست شو از خود که تا هست شوی زو تمام
\\
\end{longtable}
\end{center}
