\begin{center}
\section*{غزل شماره ۲۵۵۵: چرا چون ای حیات جان در این عالم وطن داری}
\label{sec:2555}
\addcontentsline{toc}{section}{\nameref{sec:2555}}
\begin{longtable}{l p{0.5cm} r}
چرا چون ای حیات جان در این عالم وطن داری
&&
نباشد خاک ره ناطق ندارد سنگ هشیاری
\\
چرا زهری دهد تلخی چرا خاری کند تیزی
&&
چرا خشمی کند تندی چرا باشد شبی تاری
\\
در آن گلزار روی او عجب می‌ماندم روزی
&&
که خاری اندر این عالم کند در عهد او خاری
\\
مگر حضرت نقابی بست از غیرت بر آن چهره
&&
که تا غیری نبیند آن برون ناید ز اغیاری
\\
مگر خود دیده عالم غلیظ و درد و قلب آمد
&&
نمی‌تاند که دریابد ز لطف آن چهره ناری
\\
دو چشم زشت رویان را لباس زشت می‌باید
&&
و کی شاید که درپوشد لباس زشت آن عاری
\\
که از عریانی لطفش لباس لطف شرمنده
&&
که از شرم صفای او عرق‌ها می‌شود جاری
\\
و او با این همه جسمی فروبرید و درپوشید
&&
برون زد لطف از چشمش ز هر سو شد به دیداری
\\
فروپوشید لطف او نهانی کرده چشمش را
&&
که تا شد دیده‌ها محروم و کند از سیر و سیاری
\\
ولیک آن نور ناپیدا همی‌فرمایدت هر دم
&&
شراب می که بفزاید ز بی‌هوشیت هشیاری
\\
که خوبان به غایت را فراغت باشد از شیوه
&&
ولیکن عشقشان دارد هزاران مکر و عیاری
\\
چنانک از شهوتی تو خوش به جسم و جان شهوانی
&&
نباشی زان طرب غافل اگر تو جان جان داری
\\
درون خود طلب آن را نه پیش و پس نه بر گردون
&&
نمی‌بینی که اندر خواب تو در باغ و گلزاری
\\
کدامین سوی می‌دانی کدامین سوی می‌بینی
&&
تو آن باغی که می‌بینی به خواب اندر به بیداری
\\
چو دیده جان گشادی تو بدیدی ملک روحانی
&&
از آن جا طفل ره باشی چو رو زین سو به شه آری
\\
کدامین شه نیارم گفت رمزی از صفات او
&&
ولیکن از مثالی تو بدانی گر خرد داری
\\
خردهایی نمی‌خواهم که از دونی و طماعی
&&
سر و سرور نمی‌جوید همی‌جوید کلهداری
\\
که بگذار و سر می‌جو کز آن سر سر به دست آید
&&
به سر بنشین به بزم سر ببین زان سر تو خماری
\\
ز جامی کز صفای آن نماید غیب‌ها یک یک
&&
چه مه رویان نماید غیب اندر حجب و عماری
\\
به روی هر مهی بینی تو داغی بس ظریف و کش
&&
نشان بندگی شه که فرد است او به دلداری
\\
به نزد حسن انس و جن مخدومی شمس الدین
&&
زهی تبریز دریاوش که بر هر ابر در باری
\\
\end{longtable}
\end{center}
