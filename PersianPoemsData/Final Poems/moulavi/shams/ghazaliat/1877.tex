\begin{center}
\section*{غزل شماره ۱۸۷۷: ای دل چو نمی‌گردد در شرح زبان من}
\label{sec:1877}
\addcontentsline{toc}{section}{\nameref{sec:1877}}
\begin{longtable}{l p{0.5cm} r}
ای دل چو نمی‌گردد در شرح زبان من
&&
وان حرف نمی‌گنجد در صحن بیان من
\\
می گردد تن در کد بر جای زبان خود
&&
در پرده آن مطرب کو زد ضربان من
\\
هم ساغر و هم باده سرمست از آن ساقی
&&
هم جان و جهان حیران در جان و جهان من
\\
از غیب یکی لعلی در غار جهان آمد
&&
وان لعل شده حیران در عزت کان من
\\
ما را تو کجا یابی گر موی به مو جویی
&&
چون در سر زلف او گشته‌ست مکان من
\\
جان دوش مر آن مه را می گفت دلم خستی
&&
پیکان پر از خون بین ای سخته کمان من
\\
گفتا که شکار من جز شیر کجا باشد
&&
جز لعل بدخشانی کی یافت نشان من
\\
جز دلق دو صدپاره من پاره کجا گیرم
&&
باقی قماشت کو ای دلق کشان من
\\
شمس الحق تبریزی از دور زمان برتر
&&
و افزوده ز هر دوری از وی دوران من
\\
\end{longtable}
\end{center}
