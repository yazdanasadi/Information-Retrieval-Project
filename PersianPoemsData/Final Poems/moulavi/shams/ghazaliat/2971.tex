\begin{center}
\section*{غزل شماره ۲۹۷۱: اندر قمارخانه چون آمدی به بازی}
\label{sec:2971}
\addcontentsline{toc}{section}{\nameref{sec:2971}}
\begin{longtable}{l p{0.5cm} r}
اندر قمارخانه چون آمدی به بازی
&&
کارت شود حقیقت هر چند تو مجازی
\\
با جمله سازواری ای جان به نیک خویی
&&
این جا که اصل کار است جانا چرا نسازی
\\
گویی که من شب و روز مرد نمازکارم
&&
چون نیست ای برادر گفتار تو نمازی
\\
با ناکسان تو صحبت زنهار تا نداری
&&
شو همنشین شاهان گر مرد سرفرازی
\\
آخر چرا تو خود را کردی چو پای تابه
&&
چون بر لباس آدم تو بهترین طرازی
\\
بر خر چرا نشینی ای همنشین شاهان
&&
چون هست در رکابت چندین هزار تازی
\\
شیشه دلی که داری بربا ز سنگ جانان
&&
باری به بزم شاه آ بنگر تو دلنوازی
\\
در جانت دردمد شه از شادیی که جانت
&&
هم وارهد ز مطرب وز پرده حجازی
\\
سرمست و پای کوبان با جمع ماه رویان
&&
در نور روی آن شه شاهانه می‌گرازی
\\
شاهت همی‌نوازد کای پیشوای خاصان
&&
پیوسته پیش ما باش چون تو امین رازی
\\
گاه از جمال پستی گاه از شراب مستی
&&
گه با قدم قرینی گه با کرشم و نازی
\\
مقصود شمس دین است هم صدر و هم خداوند
&&
وصلم به خدمت او است چون مرغزی و رازی
\\
هر کس که در دل او باشد هوای تبریز
&&
گردد اگر چه هندو است او گلرخ طرازی
\\
\end{longtable}
\end{center}
