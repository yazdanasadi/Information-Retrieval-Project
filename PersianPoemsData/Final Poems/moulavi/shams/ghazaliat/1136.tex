\begin{center}
\section*{غزل شماره ۱۱۳۶: نبشتست خدا گرد چهره دلدار}
\label{sec:1136}
\addcontentsline{toc}{section}{\nameref{sec:1136}}
\begin{longtable}{l p{0.5cm} r}
نبشتست خدا گرد چهره دلدار
&&
خطی که فاعتبروا منه یا اولی الابصار
\\
چو عشق مردم خوارست مردمی باید
&&
که خویش لقمه کند پیش عشق مردم خوار
\\
تو لقمه ترشی دیر دیر هضم شوی
&&
ولیست لقمه شیرین نوش نوش گوار
\\
تو لقمه‌ای بشکن زانک آن دهان تنگست
&&
سه پیل هم نخورد مر تو را مگر به سه بار
\\
به پیش حرص تو خود پیل لقمه‌ای باشد
&&
تویی چو مرغ ابابیل پیل کرده شکار
\\
تو زاده عدمی آمده ز قحط دراز
&&
تو را چه مرغ مسمن غذا چه کژدم و مار
\\
به دیگ گرم رسیدی گهی دهان سوزی
&&
گهی سیاه کنی جامه و لب و دستار
\\
به هیچ سیر نگردی چو معده دوزخ
&&
مگر که بر تو نهد پای خالق جبار
\\
چنانک بر سر دوزخ قدم نهد خالق
&&
ندا کند که شدم سیر هین قدم بردار
\\
خداست سیرکن چشم اولیا و خواص
&&
که رسته‌اند ز خویش و ز حرص این مردار
\\
نه حرص علم و هنر ماندشان نه حرص بهشت
&&
نجوید او خر و اشتر که هست شیرسوار
\\
خموش اگر شمرم من عطا و بخشش‌هاش
&&
از آن شمار شود گیج و خیره روز شمار
\\
بیا تو مفخر تبریز شمس دین به حق
&&
کمینه چاکر تو شمس گنبد دوار
\\
\end{longtable}
\end{center}
