\begin{center}
\section*{غزل شماره ۲۱۱۵: بازرسید آن بت زیبای من}
\label{sec:2115}
\addcontentsline{toc}{section}{\nameref{sec:2115}}
\begin{longtable}{l p{0.5cm} r}
بازرسید آن بت زیبای من
&&
خرمی این دم و فردای من
\\
در نظرش روشنی چشم من
&&
در رخ او باغ و تماشای من
\\
عاقبت امر به گوشش رسید
&&
بانگ من و نعره و هیهای من
\\
بر در من کیست که در می‌زند
&&
جان و جهان است و تمنای من
\\
گر نزند او در من درد من
&&
ور نکند یاد من او وای من
\\
دور مکن سایه خود از سرم
&&
باز مکن سلسله از پای من
\\
در چه خیالی هله ای روترش
&&
رو بر حلوایی و حلوای من
\\
هم بخور و هم کف حلوا بیار
&&
تا که بیفزاید صفرای من
\\
ریش تو را سخت گرفته‌ست غم
&&
چیست زبونی تو بابای من
\\
در زنخش کوب دو سه مشت سخت
&&
ای نر و نرزاده و مولای من
\\
مشک بدرید و بینداخت دلو
&&
غرقه آب آمد سقای من
\\
بانگ زدم کای کر سقا بیا
&&
رفت و بنشنید علالای من
\\
آن من است او و به هر جا رود
&&
عاقبت آید سوی صحرای من
\\
جوشش دریای معلق مگر
&&
از لمع گوهر گویای من
\\
گوید دریا که ز کشتی بجه
&&
دررو در آب مصفای من
\\
قطره به دریا چو رود در شود
&&
قطره شود بحر به دریای من
\\
ترک غزل گیر و نگر در ازل
&&
کز ازل آمد غم و سودای من
\\
\end{longtable}
\end{center}
