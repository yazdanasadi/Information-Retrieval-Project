\begin{center}
\section*{غزل شماره ۱۲۱۶: گر عاشقی از جان و دل جور و جفای یار کش}
\label{sec:1216}
\addcontentsline{toc}{section}{\nameref{sec:1216}}
\begin{longtable}{l p{0.5cm} r}
گر عاشقی از جان و دل جور و جفای یار کش
&&
ور زانک تو عاشق نه‌ای رو سخره می‌کن خار کش
\\
جانی بباید گوهری تا ره برد در دلبری
&&
این ننگ جان‌ها را ز خود بیرون کن و بر دار کش
\\
گاهی بود در تیرگی گاهی بود در خیرگی
&&
بیزار شو زین جان هله بر وی خط بیزار کش
\\
خود را مبین در من نگر کز جان شدستم بی‌اثر
&&
مانند بلبل مست شو زو رخت بر گلزار کش
\\
این کره تند فلک از روح تو سر می‌کشد
&&
چابک سوار حضرتی این کره را در کار کش
\\
چون شهسوار فارسی خربندگی تا کی کنی
&&
ننگت نمی‌آید که خر گوید تو را خروار کش
\\
همچون جهودان می‌زیی ترسان و خوار و متهم
&&
پس چون جهودان کن نشان عصابه بر دستار کش
\\
یا از جهودی توبه کن از خاک پای مصطفی
&&
بهر گشاد دیده را در دیده افکار کش
\\
\end{longtable}
\end{center}
