\begin{center}
\section*{غزل شماره ۱۸۵۵: چه دانستم که این سودا مرا زین سان کند مجنون}
\label{sec:1855}
\addcontentsline{toc}{section}{\nameref{sec:1855}}
\begin{longtable}{l p{0.5cm} r}
چه دانستم که این سودا مرا زین سان کند مجنون
&&
دلم را دوزخی سازد دو چشمم را کند جیحون
\\
چه دانستم که سیلابی مرا ناگاه برباید
&&
چو کشتی ام دراندازد میان قلزم پرخون
\\
زند موجی بر آن کشتی که تخته تخته بشکافد
&&
که هر تخته فروریزد ز گردش‌های گوناگون
\\
نهنگی هم برآرد سر خورد آن آب دریا را
&&
چنان دریای بی‌پایان شود بی‌آب چون هامون
\\
شکافد نیز آن هامون نهنگ بحرفرسا را
&&
کشد در قعر ناگاهان به دست قهر چون قارون
\\
چو این تبدیل‌ها آمد نه هامون ماند و نه دریا
&&
چه دانم من دگر چون شد که چون غرق است در بی‌چون
\\
چه دانم‌های بسیار است لیکن من نمی‌دانم
&&
که خوردم از دهان بندی در آن دریا کفی افیون
\\
\end{longtable}
\end{center}
