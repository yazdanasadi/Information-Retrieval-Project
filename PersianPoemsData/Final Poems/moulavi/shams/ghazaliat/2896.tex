\begin{center}
\section*{غزل شماره ۲۸۹۶: قره العین منی ای جان بلی}
\label{sec:2896}
\addcontentsline{toc}{section}{\nameref{sec:2896}}
\begin{longtable}{l p{0.5cm} r}
قره العین منی ای جان بلی
&&
ماه بدری گرد ما گردان بلی
\\
صد هزاران آفرین بر روی تو
&&
می‌فرستد حوری و رضوان بلی
\\
ای چراغ و مشعله هفت آسمان
&&
خاکیان را آمدی مهمان بلی
\\
از کمال رحمت و شاهنشهی
&&
گنج آید جانب ویران بلی
\\
سرو رحمت چون خرامان شد به باغ
&&
یابد ابلیس لعین ایمان بلی
\\
چون شکستی شیشه درویش را
&&
واجب آید دادن تاوان بلی
\\
ملک بخشد مالک الملک از کرم
&&
علم بخشد علم القرآن بلی
\\
آفتابی چون ز مشرق سر زند
&&
ذره‌ها آیند در جولان بلی
\\
جاء ربک و الملائک چون رسید
&&
هر محال اکنون شود امکان بلی
\\
در فتوح فتحت ابوابها
&&
گرددت دشوارها آسان بلی
\\
امشب ای دلدار خواب آلود من
&&
خواب را رانی ز نرگسدان بلی
\\
چشم نرگس چون به ترک خواب گفت
&&
بر خورد از فرجه بستان بلی
\\
مغز خود را چون ز غفلت پاک روفت
&&
بو برد از گلبن و ریحان بلی
\\
روز تا شب مست و شب تا روز مست
&&
سخت شیرین باشد این دوران بلی
\\
بلبلا بر منبر گلبن بگو
&&
هست محسن درخور احسان بلی
\\
چون فزون شد اشتهای مستمع
&&
سنگ آرد منطق لقمان بلی
\\
از دیار مصر مر یعقوب را
&&
ریح یوسف شد سوی کنعان بلی
\\
گر خمش باشی و سر پنهان کنی
&&
سر شود پیدا از آن سلطان بلی
\\
خامشی صبر آمد و آثار صبر
&&
هر فرج را می‌کشد از کان بلی
\\
\end{longtable}
\end{center}
