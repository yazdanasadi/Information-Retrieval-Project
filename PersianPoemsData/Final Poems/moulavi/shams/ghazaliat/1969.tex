\begin{center}
\section*{غزل شماره ۱۹۶۹: از بدی‌ها آن چه گویم هست قصدم خویشتن}
\label{sec:1969}
\addcontentsline{toc}{section}{\nameref{sec:1969}}
\begin{longtable}{l p{0.5cm} r}
از بدی‌ها آن چه گویم هست قصدم خویشتن
&&
زانک زهری من ندیدم در جهان چون خویشتن
\\
گر اشارت با کسی دیدی ندارم قصد او
&&
نی به حق ذوالجلال و ذوالکمال و ذوالمنن
\\
تا ز خود فارغ نیایم با دگر کس چون رسم
&&
ور بگویم فارغم از خود بود سودا و ظن
\\
ور بگفتم نکته‌ای هستش بسی تأویل‌ها
&&
گر غرض نقصان کس دارم نه مردم من نه زن
\\
از تو دارم التماسی ای حریف رازدار
&&
حسن ظنی در هوی و مهر من با خویشتن
\\
دشمن جانم منم افغان من هم از خود است
&&
کز خودی خود من بخواهم همچو هیزم سوختن
\\
چونک یاری را هزاران بار با نام و نشان
&&
مدح‌های بی‌نفاقش کرده باشم در علن
\\
فخر کرده من بر او صد بار پیدا و نهان
&&
بوده ما را از عزیزی با دو دیده مقترن
\\
گر یکی عیبی بگویم قصد من عیب من است
&&
زانک ماهم را بپوشد ابر من اندر بدن
\\
رو بدان یک وصف کردم کز ملامت مر ورا
&&
بهر حق دوستی حملش مکن بر مکر و فن
\\
من خودی خویش را گویم که در پنداشتی
&&
رو اگر نور خدایی نیست شو شو ممتحن
\\
ای خود من گر همه سر خدایی محو شو
&&
کان همه خود دیده‌ای پس دیده خودبین بکن
\\
چون خداوند شمس دین را می ستایم تو بدان
&&
کاین همه اوصاف خوبی را ستودم در قرن
\\
\end{longtable}
\end{center}
