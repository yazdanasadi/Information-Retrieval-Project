\begin{center}
\section*{غزل شماره ۱۱۵۴: چو دررسید ز تبریز شمس دین چو قمر}
\label{sec:1154}
\addcontentsline{toc}{section}{\nameref{sec:1154}}
\begin{longtable}{l p{0.5cm} r}
چو دررسید ز تبریز شمس دین چو قمر
&&
ببست شمس و قمر پیش بندگیش کمر
\\
چو روی انور او گشت دیده دیده
&&
مقام دیدن حق یافت دیده‌های بشر
\\
فرشته نعره زنان پیش او چو چاوشان
&&
فلک سجودکنان پیش او به چشم و به سر
\\
به چشم نفس نشد روی ماه او دیدن
&&
که نفس می‌نگشاید به سوی شاه نظر
\\
که لعل آن مه خاصیت زمرد داشت
&&
از آن ببست از او اژدهای نفس به صبر
\\
درخت هر که بدو سر کشید جان نبرد
&&
ز اره‌های فنا و ز زخمه‌های تبر
\\
کنون که ماه نهان شد ز ابر این هجران
&&
ز ابرهای دو دیده فرودوید مطر
\\
ز قطره‌های دو دیده زمین شدی سرسبز
&&
اگر نه قطره برآمیختی به خون جگر
\\
جگر چو آلت رحمست رحم از او خیزد
&&
از این سبب مدد دیده‌ها بکرد مگر
\\
ز عشق جمله اجزای خانه باخبرند
&&
چو کدخدای بود از جمال شه مخبر
\\
تو طالب خبری کم نشین به بی‌خبران
&&
گروه بی‌خبران را به هیچ سگ مشمر
\\
که جفت مرده تو را مرده شوی گرداند
&&
که شوی مرده بود خود ز مرده شوی بتر
\\
به چشم درد به عیسی نگر اگر نگری
&&
سرک مپیچ بدان چشم و در خرش منگر
\\
چو همنشین شود انگور با خم سرکه
&&
شراب او ترشی شد حریف اوست کبر
\\
به حیله حیله تو سوراخ کن خم ترشی
&&
برون گریز و بو سوی بحر شهد و شکر
\\
کدام بحر خداوند شمس دین به حق
&&
به ذات پاک خدا اوست خسرو اکبر
\\
\end{longtable}
\end{center}
