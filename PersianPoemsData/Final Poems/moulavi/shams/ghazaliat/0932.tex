\begin{center}
\section*{غزل شماره ۹۳۲: مها به دل نظری کن که دل تو را دارد}
\label{sec:0932}
\addcontentsline{toc}{section}{\nameref{sec:0932}}
\begin{longtable}{l p{0.5cm} r}
مها به دل نظری کن که دل تو را دارد
&&
که روز و شب به مراعاتت اقتضا دارد
\\
ز شادی و ز فرح در جهان نمی‌گنجد
&&
که چون تو یار دلارام خوش لقا دارد
\\
همی‌رسد به گریبان آسمان دستش
&&
که او چو سایه ز ماه تو مقتدا دارد
\\
به آفتاب تو آن را که پشت گرم شود
&&
چرا دلیر نباشد حذر چرا دارد
\\
چرا به پنجه کمرگاه کوه را نکشد
&&
کسی که ز اطلس عشق خوشت قبا دارد
\\
تو خود جفا نکنی ور کنی جفا بر دل
&&
بکن بکن که به کردار تو رضا دارد
\\
چرا نباشد راضی بدان جفای لطیف
&&
که او طراوت آب و دم صبا دارد
\\
در آتش غم تو همچو عود عطاریست
&&
دل شریف که او داغ انبیا دارد
\\
خمش خمش که سخن آفرین معنی بخش
&&
برون گفت سخن‌های جان فزا دارد
\\
\end{longtable}
\end{center}
