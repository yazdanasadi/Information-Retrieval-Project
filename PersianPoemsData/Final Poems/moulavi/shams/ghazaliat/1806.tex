\begin{center}
\section*{غزل شماره ۱۸۰۶: آن سو مرو این سو بیا ای گلبن خندان من}
\label{sec:1806}
\addcontentsline{toc}{section}{\nameref{sec:1806}}
\begin{longtable}{l p{0.5cm} r}
آن سو مرو این سو بیا ای گلبن خندان من
&&
ای عقل عقل عقل من ای جان جان جان من
\\
زین سو بگردان یک نظر بر کوی ما کن رهگذر
&&
برجوش اندر نیشکر ای چشمه حیوان من
\\
خواهم که شب تاری شود پنهان بیایم پیش تو
&&
از روی تو روشن شود شب پیش رهبانان من
\\
عشق تو را من کیستم از اشک خون ساقیستم
&&
سغراق می چشمان من عصار می مژگان من
\\
ز اشکم شرابت آورم وز دل کبابت آورم
&&
این است تر و خشک من پیدا بود امکان من
\\
دریای چشمم یک نفس خالی مباد از گوهرت
&&
خالی مبادا یک زمان لعل خوشت از کان من
\\
با این همه کو قند تو کو عهد و کو سوگند تو
&&
چون بوریا بر می شکن ای یار خوش پیمان من
\\
نک چشم من تر می زند نک روی من زر می زند
&&
تا بر عقیقت برزند یک زر ز زرافشان من
\\
بنوشته خطی بر رخت حق جددوا ایمانکم
&&
زان چهره و خط خوشت هر دم فزون ایمان من
\\
در سر به چشمم چشم تو گوید به وقت خشم تو
&&
پنهان حدیثی کو شود از آتش پنهان من
\\
گوید قوی کن دل مرم از خشم و ناز آن صنم
&&
اول قدح دردی بخور وانگه ببین پایان من
\\
بر هر گلی خاری بود بر گنج هم ماری بود
&&
شیرین مراد تو بود تلخی و صبرت آن من
\\
گفتم چو خواهی رنج من آن رنج باشد گنج من
&&
من بوهریره آمدم رنج و غمت انبان من
\\
پس دست در انبان کنم خواهنده را سلطان کنم
&&
مر بدر را بدره دهم چون بدر شد مهمان من
\\
هر چه دلم خواهد ز خور ز انبان برآرم بی‌خطر
&&
تا سرخ گردد روی من سرسبز گردد خوان من
\\
گفتا نکو رفت این سخن هشدار و انبان گم مکن
&&
نیکو کلیدی یافتی ای معتمد دربان من
\\
الصبر مفتاح الفرج الصبر معراج الدرج
&&
الصیر تریاق الحرج ای ترک تازی خوان من
\\
بس کن ز لاحول ای پسر چون دیو می غرد بتر
&&
بس کردم از لاحول و شد لاحول گو شیطان من
\\
\end{longtable}
\end{center}
