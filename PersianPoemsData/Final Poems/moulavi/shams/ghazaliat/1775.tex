\begin{center}
\section*{غزل شماره ۱۷۷۵: گر تو کنی روی ترش زحمت از این جا ببرم}
\label{sec:1775}
\addcontentsline{toc}{section}{\nameref{sec:1775}}
\begin{longtable}{l p{0.5cm} r}
گر تو کنی روی ترش زحمت از این جا ببرم
&&
گر تو میی من قدحم ور ترشی من کبرم
\\
عبس وجها سندی کان سناه مددی
&&
کل هوی یهویه ذاک جمیل و کرم
\\
زنده نباشد دل من گر به مهش دل ندهم
&&
عقل ندارد سر من گر ز نباتش نچرم
\\
مبسمه بلبلنی عابسه زلزلنی
&&
ما شطه شیبنی غیبته الف هرم
\\
گر کژی آرم سوی او همچو کمان تیر خورم
&&
ور هنر آرم سوی او عرضه کنم بی‌هنرم
\\
بارحتی فکرته هیجنی قلقلنی
&&
قمت اطوف سکرا مغتنما حول حرم
\\
گر پی رایش نروم باد گسسته رگ من
&&
ور سوی بحرش نروم باد شکسته گهرم
\\
ظلت به مقتنیا مرتزقا مجتنیا
&&
نخله خلد نبتت وسط ریاض و ارم
\\
چونک شکارش نشوم خواجه یقین دان که سگم
&&
چون پی اسپش ندوم خواجه یقین دان که خرم
\\
کنت ثقیلا کسلا خففنی جذبته
&&
نمت علی قارعه عاصفنی سیل عرم
\\
گفتم بسته‌ست دلم گفت منم قفل گشا
&&
گفتم کشتی تو مرا گفت من از تو بترم
\\
رو سخن کار مگو کز همه آزاد شدم
&&
رو سخن خار مگو چون همه گل می سپرم
\\
\end{longtable}
\end{center}
