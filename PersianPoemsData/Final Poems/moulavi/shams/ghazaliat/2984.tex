\begin{center}
\section*{غزل شماره ۲۹۸۴: ای مرغ گیر دام نهانی نهاده‌ای}
\label{sec:2984}
\addcontentsline{toc}{section}{\nameref{sec:2984}}
\begin{longtable}{l p{0.5cm} r}
ای مرغ گیر دام نهانی نهاده‌ای
&&
بر روی دام شعر دخانی نهاده‌ای
\\
چندین هزار مرغ بدین فن بکشته‌ای
&&
پرهای کشته بهر نشانی نهاده‌ای
\\
مرغان پاسبان تو هیهای می‌زنند
&&
درهای هویشان چه معانی نهاده‌ای
\\
مرغان تشنه را به خرابات قرب خویش
&&
خم‌ها و باده‌های معانی نهاده‌ای
\\
آن خنب را که ساقی و مستیش بود نبرد
&&
از بهر شب روی که تو دانی نهاده‌ای
\\
در صبر و توبه عصمت اسپر سرشته‌ای
&&
و اندر جفا و خشم سنانی نهاده‌ای
\\
بی زحمت سنان و سپر بهر مخلصان
&&
ملکی درون سبع مثانی نهاده‌ای
\\
زیر سواد چشم روان کرده موج نور
&&
و اندر جهان پیر جوانی نهاده‌ای
\\
در سینه کز مخیله تصویر می‌رود
&&
بی کلک و بی‌بنان تو بنانی نهاده‌ای
\\
چندین حجاب لحم و عصب بر فراز دل
&&
دل را نفوذ و سیر عیانی نهاده‌ای
\\
غمزه عجبتر است که چون تیر می‌پرد
&&
یا ابروی که بهر کمانی نهاده‌ای
\\
اخلاق مختلف چو شرابات تلخ و نوش
&&
در جسم‌های همچو اوانی نهاده‌ای
\\
وین شربت نهان مترشح شد از زبان
&&
سرجوش نطق را به لسانی نهاده‌ای
\\
هر عین و هر عرض چو دهان بسته غنچه‌ای است
&&
کان را حجاب مهد غوانی نهاده‌ای
\\
روزی که بشکفانی و آن پرده برکشی
&&
ای جان جان جان که تو جانی نهاده‌ای
\\
دل‌های بی‌قرار ببیند که در فراق
&&
از بهر چه نیاز و کشانی نهاده‌ای
\\
خاموش تا بگوید آن جان گفته‌ها
&&
این چه دراز شعبده خوانی نهاده‌ای
\\
\end{longtable}
\end{center}
