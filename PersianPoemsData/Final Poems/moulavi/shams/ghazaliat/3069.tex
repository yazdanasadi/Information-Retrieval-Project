\begin{center}
\section*{غزل شماره ۳۰۶۹: نگاهبان دو دیده‌ست چشم دلداری}
\label{sec:3069}
\addcontentsline{toc}{section}{\nameref{sec:3069}}
\begin{longtable}{l p{0.5cm} r}
نگاهبان دو دیده‌ست چشم دلداری
&&
نگاه دار نظر از رخ دگر یاری
\\
وگر نه به سینه درآید به غیر آن دلبر
&&
بگو برو که همی‌ترسم از جگرخواری
\\
هلا مباد که چشمش به چشم تو نگرد
&&
درون چشم تو بیند خیال اغیاری
\\
به من نگر که مرا یار امتحان‌ها کرد
&&
به حیله برد مرا کشکشان به گلزاری
\\
گلی نمود که گل‌ها ز رشک او می‌ریخت
&&
بتی که جمله بتان پیش او گرفتاری
\\
چنین چنین به تعجب سری بجنبانید
&&
که نادرست و غریبست درنگر باری
\\
چنانک گفت طراریم دزد در پی توست
&&
چو من سپس نگریدم ربود دستاری
\\
ز آب دیده داوود سبزه‌ها بررست
&&
به عذر آنک به نقشی بکرد نظاری
\\
براند مر پدرت را کشان کشان ز بهشت
&&
نظر به سنبله تر یکی ستمکاری
\\
حذر ز سنبل ابرو که چشم شه بر توست
&&
هلا که می‌نگرد سوی تو خریداری
\\
چو مشتری دو چشم تو حی قیومست
&&
به چنگ زاغ مده چشم را چو مرداری
\\
دهی تو کاله فانی بری عوض باقی
&&
لطیف مشتریی سودمند بازاری
\\
خمش خمش که اگر چه تو چشم را بستی
&&
ریای خلق کشیدت به نظم و اشعاری
\\
ولیک مفخر تبریز شمس دین با توست
&&
چه غم خوری ز بد و نیک با چنین یاری
\\
\end{longtable}
\end{center}
