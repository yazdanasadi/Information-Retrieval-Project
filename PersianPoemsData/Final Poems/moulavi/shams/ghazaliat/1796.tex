\begin{center}
\section*{غزل شماره ۱۷۹۶: دلدار من در باغ دی می گشت و می گفت ای چمن}
\label{sec:1796}
\addcontentsline{toc}{section}{\nameref{sec:1796}}
\begin{longtable}{l p{0.5cm} r}
دلدار من در باغ دی می گشت و می گفت ای چمن
&&
صد حور کش داری ولی بنگر یکی داری چو من
\\
قدر لبم نشناختی با من دغاها باختی
&&
اینک چنین بگداختی حیران فی هذا الزمن
\\
ای فتنه‌ها انگیخته بر خلق آتش ریخته
&&
وز آسمان آویخته بر هر دلی پنهان رسن
\\
در بحر صاف پاک تو جمله جهان خاشاک تو
&&
در بحر تو رقصان شده خاشاک نقش مرد و زن
\\
خاشاک اگر گردان بود از موج جان از جا مرو
&&
سرنای خود را گفته تو من دم زنم تو دم مزن
\\
بس شمع‌ها افروختی بیرون ز سقف آسمان
&&
بس نقش‌ها بنگاشتی بیرون ز شهر جان و تن
\\
ای بی‌خیال روی تو جمله حقیقت‌ها خیال
&&
ای بی‌تو جان اندر تنم چون مرده‌ای اندر کفن
\\
بی‌نور نورافروز او ای چشم من چیزی مبین
&&
بی‌جان جان انگیز او ای جان من رو جان مکن
\\
گفتم صلای ماجرا ما را نمی‌پرسی چرا
&&
گفتا که پرسش‌های ما بیرون ز گوش است و دهن
\\
ای سایه معشوق را معشوق خود پنداشته
&&
ای سال‌ها نشناخته تو خویش را از پیرهن
\\
تا جان بااندازه‌ات بر جان بی‌اندازه زد
&&
جانت نگنجد در بدن شمعت نگنجد در لگن
\\
\end{longtable}
\end{center}
