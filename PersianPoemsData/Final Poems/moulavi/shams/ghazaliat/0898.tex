\begin{center}
\section*{غزل شماره ۸۹۸: بانگ زدم من که دل مست کجا می‌رود}
\label{sec:0898}
\addcontentsline{toc}{section}{\nameref{sec:0898}}
\begin{longtable}{l p{0.5cm} r}
بانگ زدم من که دل مست کجا می‌رود
&&
گفت شهنشه خموش جانب ما می‌رود
\\
گفتم تو با منی دم ز درون می‌زنی
&&
پس دل من از برون خیره چرا می‌رود
\\
گفت که دل آن ماست رستم دستان ماست
&&
سوی خیال خطا بهر غزا می‌رود
\\
هر طرفی کو رود بخت از آن سو رود
&&
هیچ مگو هر طرف خواهد تا می‌رود
\\
گه مثل آفتاب گنج زمین می‌شود
&&
گه چو دعا رسول سوی سما می‌رود
\\
گاه ز پستان ابر شیر کرم می‌دهد
&&
گه به گلستان جان همچو صبا می‌رود
\\
بر اثر دل برو تا تو ببینی درون
&&
سبزه و گل می‌دمد جوی وفا می‌رود
\\
صورت بخش جهان ساده و بی‌صورتست
&&
آن سر و پای همه بی‌سر و پا می‌رود
\\
هست صواب صواب گر چه خطایی کند
&&
هست وفای وفا گر به جفا می‌رود
\\
دل مثل روزنست خانه بدو روشنست
&&
تن به فنا می‌رود دل به بقا می‌رود
\\
فتنه برانگیخت دل خون شهان ریخت دل
&&
با همه آمیخت دل گر چه جدا می‌رود
\\
سحر خدا آفرید در دل هر کس پدید
&&
کیسه جوزا برید همچو سها می‌رود
\\
با تو دلا ابلهیست کیسه نگه داشتن
&&
کیسه شد و جان پی کیسه ربا می‌رود
\\
گفتم جادو کسی سست بخندید و گفت
&&
سحر اثر کی کند ذکر خدا می‌رود
\\
گفتم آری ولیک سحر تو سر خداست
&&
سحر خوشت هم تک حکم قضا می‌رود
\\
دایم دلدار را با دل و جان ماجراست
&&
پوست بر او نیست اینک پیش شما می‌رود
\\
اسب سقاست این بانگ دراست این
&&
بانگ کنان کز برون اسب سقا می‌رود
\\
\end{longtable}
\end{center}
