\begin{center}
\section*{غزل شماره ۸۴۰: بعد از سماع گویی کان شورها کجا شد}
\label{sec:0840}
\addcontentsline{toc}{section}{\nameref{sec:0840}}
\begin{longtable}{l p{0.5cm} r}
بعد از سماع گویی کان شورها کجا شد
&&
یا خود نبود چیزی یا بود و آن فنا شد
\\
منکر مباش بنگر اندر عصای موسی
&&
یک لحظه آن عصا بد یک لحظه اژدها شد
\\
چون اژدهاست قالب لب را نهاده بر لب
&&
کو خورد عالمی را وانگه همان عصا شد
\\
یک گوهری چون بیضه جوشید و گشت دریا
&&
کف کرد و کف زمین شد وز دود او سما شد
\\
الحق نهان سپاهی پوشیده پادشاهی
&&
هر لحظه حمله آرد وانگه به اصل واشد
\\
گر چه ز ما نهان شد در عالمی روان شد
&&
تا نیستش نخوانی گر از نظر جدا شد
\\
هر حالتی چو تیرست اندر کمان قالب
&&
رو در نشانه جویش گر از کمان رها شد
\\
گر چه صدف ز ساحل قطره ربود و گم شد
&&
در بحر جوید او را غواص کشنا شد
\\
از میل مرد و زن خون جوشید وان منی شد
&&
وانگه از آن دو قطره یک خیمه در هوا شد
\\
وانگه ز عالم جان آمد سپاه انسان
&&
عقلش وزیر گشت و دل رفت پادشا شد
\\
تا بعد چند گاهی دل یاد شهر جان کرد
&&
واگشت جمله لشکر در عالم بقا شد
\\
گویی چگونه باشد آمدشد معانی
&&
اینک به وقت خفتن بنگر گره گشا شد
\\
\end{longtable}
\end{center}
