\begin{center}
\section*{غزل شماره ۱۴۲۴: بگفتم عذر با دلبر که بی‌گه بود و ترسیدم}
\label{sec:1424}
\addcontentsline{toc}{section}{\nameref{sec:1424}}
\begin{longtable}{l p{0.5cm} r}
بگفتم عذر با دلبر که بی‌گه بود و ترسیدم
&&
جوابم داد کای زیرک بگاهت نیز هم دیدم
\\
بگفتم ای پسندیده چو دیدی گیر نادیده
&&
بگفت او ناپسندت را به لطف خود پسندیدم
\\
بگفتم گر چه شد تقصیر دل هرگز نگردیده‌ست
&&
بگفت آن را هم از من دان که من از دل نگردیدم
\\
بگفتم هجر خونم خورد بشنو آه مهجوران
&&
بگفت آن دام لطف ماست کاندر پات پیچیدم
\\
چو یوسف کابن یامین را به مکر از دشمنان بستد
&&
تو را هم متهم کردند و من پیمانه دزدیدم
\\
بگفتم روز بی‌گاه است و بس ره دور گفتا رو
&&
به من بنگر به ره منگر که من ره را نوردیدم
\\
به گاه و بی‌گه عالم چه باشد پیش این قدرت
&&
که من اسرار پنهان را بر این اسباب نبریدم
\\
اگر عقل خلایق را همه بر همدگر بندی
&&
نیابد سر لطف ما مگر آن جان که بگزیدم
\\
\end{longtable}
\end{center}
