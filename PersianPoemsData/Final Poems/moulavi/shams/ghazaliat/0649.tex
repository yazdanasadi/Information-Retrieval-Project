\begin{center}
\section*{غزل شماره ۶۴۹: بر چرخ سحرگاه یکی ماه عیان شد}
\label{sec:0649}
\addcontentsline{toc}{section}{\nameref{sec:0649}}
\begin{longtable}{l p{0.5cm} r}
بر چرخ سحرگاه یکی ماه عیان شد
&&
از چرخ فرود آمد و در ما نگران شد
\\
چون باز که برباید مرغی به گه صید
&&
بربود مرا آن مه و بر چرخ دوان شد
\\
در خود چو نظر کردم خود را بندیدم
&&
زیرا که در آن مه تنم از لطف چو جان شد
\\
در جان چو سفر کردم جز ماه ندیدم
&&
تا سر تجلی ازل جمله بیان شد
\\
نه چرخ فلک جمله در آن ماه فروشد
&&
کشتی وجودم همه در بحر نهان شد
\\
آن بحر بزد موج و خرد باز برآمد
&&
و آوازه درافکند چنین گشت و چنان شد
\\
آن بحر کفی کرد و به هر پاره از آن کف
&&
نقشی ز فلان آمد و جسمی ز فلان شد
\\
هر پاره کف جسم کز آن بحر نشان یافت
&&
در حال گذارید و در آن بحر روان شد
\\
بی دولت مخدومی شمس الحق تبریز
&&
نی ماه توان دیدن و نی بحر توان شد
\\
\end{longtable}
\end{center}
