\begin{center}
\section*{غزل شماره ۲۱۲۸: ابشر ثم ابشر یا مؤتمن}
\label{sec:2128}
\addcontentsline{toc}{section}{\nameref{sec:2128}}
\begin{longtable}{l p{0.5cm} r}
ابشر ثم ابشر یا مؤتمن
&&
اقترب الوصل و افنی المحن
\\
فاجتمعوا نقضی ما فاتنا
&&
من سکر یلقب‌ام الفتن
\\
قد قدم الساقی نعم السقا
&&
قد قرب المنزل نعم الوطن
\\
کار تو این است که دل پروری
&&
پرورش آمد همه کار چمن
\\
خلدک الله لنا ساقیا
&&
انت لنا البر ولی المنن
\\
نحن عطاش سندی فاسقنا
&&
من سکر یقطع راس الحزن
\\
ینشئنا صفوته نشأه
&&
طیبه السر ملیح العلن
\\
ترک کن این گفت و همی‌باش جفت
&&
و اغتنم الفرض و خل السنن
\\
فاغتنم السکر و زمزم لنا
&&
تن تنتن تن تنتن تن تنن
\\
قد ظهر الصبح و خل الحرس
&&
قد وضع الحرب فخل المحن
\\
طیبنا الراح و نعم المطیب
&&
و اختلط الشهد لنا باللبن
\\
نطمع فی الزاید فازدد لنا
&&
فاسق و اسرف سرفا مشبعا
\\
سن لنا سنتک المرتضی
&&
رن لنا رنه ظبی الاغن
\\
نخ هنا جمله بعراننا
&&
لیس علی الارض کهذا العطن
\\
من هو لا یغبط هذ السقا
&&
من هو لا یعبد هذ الوثن
\\
ما لرسالات هوی منتهی
&&
فاقنع بالاوجز یا ممتحن
\\
قد سکر القوم و نام الندیم
&&
نشرب بالوحده نحن اذن
\\
مفتعلن مفتعلن مفتعل
&&
فعلللن فعلللن فعللن
\\
\end{longtable}
\end{center}
