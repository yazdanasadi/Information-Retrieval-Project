\begin{center}
\section*{غزل شماره ۲۰۵: چند گریزی ز ما چند روی جا به جا}
\label{sec:0205}
\addcontentsline{toc}{section}{\nameref{sec:0205}}
\begin{longtable}{l p{0.5cm} r}
چند گریزی ز ما چند روی جا به جا
&&
جان تو در دست ماست همچو گلوی عصا
\\
چند بکردی طواف گرد جهان از گزاف
&&
زین رمه پر ز لاف هیچ تو دیدی وفا
\\
روز دو سه‌ای زحیر گرد جهان گشته گیر
&&
همچو سگان مرده گیر گرسنه و بی‌نوا
\\
مرده دل و مرده جو چون پسر مرده شو
&&
از کفن مرده ایست در تن تو آن قبا
\\
زنده ندیدی که تا مرده نماید تو را
&&
چند کشی در کنار صورت گرمابه را
\\
دامن تو پرسفال پیش تو آن زر و مال
&&
باورم آنگه کنی که اجل آرد فنا
\\
گویی که زر کهن من چه کنم بخش کن
&&
من به سما می‌روم نیست زر آن جا روا
\\
جغد نه‌ای بلبلی از چه در این منزلی
&&
باغ و چمن را چه شد سبزه و سرو و صبا
\\
\end{longtable}
\end{center}
