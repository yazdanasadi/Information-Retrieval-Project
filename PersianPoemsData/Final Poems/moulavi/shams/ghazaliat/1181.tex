\begin{center}
\section*{غزل شماره ۱۱۸۱: به سوی ما نگر چشمی برانداز}
\label{sec:1181}
\addcontentsline{toc}{section}{\nameref{sec:1181}}
\begin{longtable}{l p{0.5cm} r}
به سوی ما نگر چشمی برانداز
&&
وگر فرصت بود بوسی درانداز
\\
چو کردی نیت نیکو مگردان
&&
از آن گلشن گلی بر چاکر انداز
\\
اگر خواهی که روزافزون بود کار
&&
نظر بر کار ما افزونتر انداز
\\
وگر تو فتنه انگیزی و خودکام
&&
رها کن داد و رسمی دیگر انداز
\\
نگون کن سرو را همچون بنفشه
&&
گناه غنچه بر نیلوفر انداز
\\
ز باد و بوی توست امروز در باغ
&&
درختان جمله رقاص و سرانداز
\\
چو شاخ لاغری افزون کند رقص
&&
تو میوه سوی شاخ لاغر انداز
\\
چو آمد خار گل را اسپری بخش
&&
چو خصم آمد به سوسن خنجر انداز
\\
بر عاشق بری چون سیم بگشا
&&
سوی مفلس یکی مشتی زر انداز
\\
برآ ای شاه شمس الدین تبریز
&&
یکی نوری عجب بر اختر انداز
\\
\end{longtable}
\end{center}
