\begin{center}
\section*{غزل شماره ۴۳۶: گفتا که کیست بر در گفتم کمین غلامت}
\label{sec:0436}
\addcontentsline{toc}{section}{\nameref{sec:0436}}
\begin{longtable}{l p{0.5cm} r}
گفتا که کیست بر در گفتم کمین غلامت
&&
گفتا چه کار داری گفتم مها سلامت
\\
گفتا که چند رانی گفتم که تا بخوانی
&&
گفتا که چند جوشی گفتم که تا قیامت
\\
دعوی عشق کردم سوگندها بخوردم
&&
کز عشق یاوه کردم من ملکت و شهامت
\\
گفتا برای دعوی قاضی گواه خواهد
&&
گفتم گواه اشکم زردی رخ علامت
\\
گفتا گواه جرحست تردامنست چشمت
&&
گفتم به فر عدلت عدلند و بی‌غرامت
\\
گفتا که بود همره گفتم خیالت ای شه
&&
گفتا که خواندت این جا گفتم که بوی جامت
\\
گفتا چه عزم داری گفتم وفا و یاری
&&
گفتا ز من چه خواهی گفتم که لطف عامت
\\
گفتا کجاست خوشتر گفتم که قصر قیصر
&&
گفتا چه دیدی آن جا گفتم که صد کرامت
\\
گفتا چراست خالی گفتم ز بیم رهزن
&&
گفتا که کیست رهزن گفتم که این ملامت
\\
گفتا کجاست ایمن گفتم که زهد و تقوا
&&
گفتا که زهد چه بود گفتم ره سلامت
\\
گفتا کجاست آفت گفتم به کوی عشقت
&&
گفتا که چونی آن جا گفتم در استقامت
\\
خامش که گر بگویم من نکته‌های او را
&&
از خویشتن برآیی نی در بود نه بامت
\\
\end{longtable}
\end{center}
