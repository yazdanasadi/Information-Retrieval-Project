\begin{center}
\section*{غزل شماره ۳۱۲۷: عجب‌العجایب توی در کیایی}
\label{sec:3127}
\addcontentsline{toc}{section}{\nameref{sec:3127}}
\begin{longtable}{l p{0.5cm} r}
عجب‌العجایب توی در کیایی
&&
نما روی خود، گر عجب می‌نمایی
\\
توی محرم دل توی همدم دل
&&
بجز تو که داند ره دلگشایی
\\
تو دانی که دل در کجاها فتادست
&&
اگر دل نداند ترا که کجایی
\\
برافکن برو سایهٔ از سعادت
&&
که مسجود قانی و جان همایی
\\
جهان را بیارا به نور نبوت
&&
که استاد جان همه انبیایی
\\
گهر سنگ بود وز تو گشت گوهر
&&
عطا کن، عطا کن، که بحر عطایی
\\
نه آب منی بد، که شخص سنی شد؟!
&&
چو رست از منی، وارهانش ز مایی
\\
کف آب را تو بدادی زمینی
&&
سیه دود را تو بدادی سمایی
\\
چو تبدیل اشیا ترا بد میسر
&&
همه حلم و علمی همه کیمیایی
\\
حرامست خواب شب، ایرا تو ماهی
&&
که در شب چو بدری ز جانها برآیی
\\
میا خواب! اینجا، برو جای دیگر
&&
که بحرست چشمم، در او غرقه آبی
\\
شبا، در تهیج چو مار سیاهی
&&
جهان را بخوردی، مگر اژدهایی
\\
چو خلاق بیچون فسون بر تو خواند
&&
هرانچ بخوردی سحرگه بزایی
\\
الا ماه گردون! که سیاح چرخی
&&
پی من باشد دمی گر بپایی؟!
\\
تو در چشم بعضی مقیمی و ساکن
&&
تو هر دیده را شیوهٔ می‌نمایی
\\
اسکان قلبی! علیکم ثنایی
&&
افیضوا علینا، کووس البقء
\\
گر آن جان جان را ندیدی دلا تو
&&
اگر جمله چشمی، اسیر عمایی
\\
چو هفتاد و دو ملتی عقل دارد
&&
بجو در جنونش دلا اصطفایی
\\
اجیبوا، اجیبوا هواکم عجیب
&&
صفا من هواکم نسیم الهوایی
\\
تن اندر جنونش، دلم ارغنونش
&&
روانم زبونش، ز بی‌دست و پایی
\\
مگر اختران دیده‌اندت ز بالا
&&
فرو کرده سرها برای گوایی
\\
غلط، کیست اختر؟! که بویی نبردست
&&
دل عقل کل با همه ارتقایی
\\
فلا عیش یا سادتی ما عداکم
&&
بظعن و سیر ولا فی ثواء
\\
\end{longtable}
\end{center}
