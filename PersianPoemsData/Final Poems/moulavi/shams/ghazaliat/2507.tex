\begin{center}
\section*{غزل شماره ۲۵۰۷: بیامد عید ای ساقی عنایت را نمی‌دانی}
\label{sec:2507}
\addcontentsline{toc}{section}{\nameref{sec:2507}}
\begin{longtable}{l p{0.5cm} r}
بیامد عید ای ساقی عنایت را نمی‌دانی
&&
غلامانند سلطان را بیارا بزم سلطانی
\\
منم مخمور و مست تو قدح خواهم ز دست تو
&&
قدح از دست تو خوشتر که می جان است و تو جانی
\\
بیا ساقی کم آزارم که من از خویش بیزارم
&&
بنه بر دست آن شیشه به قانون پری خوانی
\\
چنان کن شیشه را ساده که گوید خود منم باده
&&
به حق خویشی ای ساقی که بی‌خویشم تو بنشانی
\\
به عشق و جست و جوی تو سبو بردم به جوی تو
&&
بحمدالله که دانستم که ما را خود تو جویانی
\\
تو خواهم کز نکوکاری سبو را نیک پر داری
&&
از آن می‌های روحانی وزان خم‌های پنهانی
\\
میی اندر سرم کردی و دیگر وعده‌ام کردی
&&
به جان پاکت ای ساقی که پیمان را نگردانی
\\
که ساقی الستی تو قرار جان مستی تو
&&
در خیبر شکستی تو به بازوی مسلمانی
\\
\end{longtable}
\end{center}
