\begin{center}
\section*{غزل شماره ۴۰۲: نقش بند جان که جان‌ها جانب او مایلست}
\label{sec:0402}
\addcontentsline{toc}{section}{\nameref{sec:0402}}
\begin{longtable}{l p{0.5cm} r}
نقش بند جان که جان‌ها جانب او مایلست
&&
عاقلان را بر زبان و عاشقان را در دلست
\\
آنک باشد بر زبان‌ها لا احب الافلین
&&
باقیات الصالحات است آنک در دل حاصلست
\\
دل مثال آسمان آمد زبان همچون زمین
&&
از زمین تا آسمان‌ها منزل بس مشکلست
\\
دل مثال ابر آمد سینه‌ها چون بام‌ها
&&
وین زبان چون ناودان باران از این جا نازلست
\\
آب از دل پاک آمد تا به بام سینه‌ها
&&
سینه چون آلوده باشد این سخن‌ها باطلست
\\
این خود آن کس را بود کز ابر او باران چکد
&&
بام کو از ابر گیرد ناودانش قایلست
\\
آنک برد از ناودان دیگران او سارقست
&&
آنک دزدد آب بام دیگران او ناقلست
\\
هر که روید نرگس گل ز آب چشمش عاشقست
&&
هر که نرگس‌ها بچیند دسته بند عاملست
\\
گر چه کف‌های ترازو شد برابر وقت وزن
&&
چون زبانه ش راست نبود آن ترازو مایلست
\\
هر کی پوشیده‌ست بر وی حال و رنگ جان او
&&
هر جوابی که بگوید او به معنی سائلست
\\
گر طبیبی حاذقی رنجور را تلخی دهد
&&
گر چه ظالم می‌نماید نیست ظالم عادلست
\\
پا شناسد کفش خویش ار چه که تاریکی بود
&&
دل ز راه ذوق داند کاین کدامین منزلست
\\
در دل و کشتی نوح افکن در این طوفان تو خویش
&&
دل مترسان ای برادر گر چه منزل‌هایلست
\\
هر که را خواهی شناسی همنشینش را نگر
&&
زانک مقبل در دو عالم همنشین مقبل‌ست
\\
هر چه بر تو ناخوش آید آن منه بر دیگران
&&
زانک این خو و طبیعت جملگان را شاملست
\\
پنبه‌ها در گوش کن تا نشنوی هر نکته‌ای
&&
زانک روح ساده تو زنگ‌ها را قابلست
\\
هر که روحش از هوای هفتمین بگذشت رست
&&
می خور از انفاس روح او که روحش بسملست
\\
این هوا اندر کمین باشد چو بیند بی‌رفیق
&&
مرد را تنها بگوید هین که مردک غافل‌ست
\\
وصل خواهی با کسان بنشین که ایشان واصلند
&&
وصل از آن کس خواه باری کو به معنی واصل‌ست
\\
گرد مستان گرد اگر می کم رسد بویی رسد
&&
خود مذاق می چه داند آنک مرد عاقلست
\\
نکته‌ها را یاد می‌گیری جواب هر سال
&&
تا به وقت امتحان گویند مرد فاضلست
\\
گر بنتوانی ز نقص خود شدن سوی کمال
&&
شمس تبریزی کنون اندر کمالت کاملست
\\
\end{longtable}
\end{center}
