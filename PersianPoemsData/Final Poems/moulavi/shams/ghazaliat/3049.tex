\begin{center}
\section*{غزل شماره ۳۰۴۹: ربود عقل و دلم را جمال آن عربی}
\label{sec:3049}
\addcontentsline{toc}{section}{\nameref{sec:3049}}
\begin{longtable}{l p{0.5cm} r}
ربود عقل و دلم را جمال آن عربی
&&
درون غمزه مستش هزار بوالعجبی
\\
هزار عقل و ادب داشتم من ای خواجه
&&
کنون چو مست و خرابم صلای بی‌ادبی
\\
مسبب سبب این جا در سبب بربست
&&
تو آن ببین که سبب می‌کشد ز بی‌سببی
\\
پریر رفتم سرمست بر سر کویش
&&
به خشم گفت چه گم کرده‌ای چه می‌طلبی
\\
شکسته بسته بگفتم یکی دو لفظ عرب
&&
اتیت اطلب فی حیکم مقام ابی
\\
جواب داد کجا خفته‌ای چه می‌جویی
&&
به پیش عقل محمد پلاس بولهبی
\\
ز عجز خوردم سوگندها و گرم شدم
&&
به ذات پاک خدا و به جان پاک نبی
\\
چه جای گرمی و سوگند پیش آن بینا
&&
و کیف یصرع صقر بصوله الخرب
\\
روان شد اشک ز چشم من و گواهی داد
&&
کما یسیل میاه السقا من القرب
\\
چه چاره دارم غماز من هم از خانه‌ست
&&
رخم چو سکه زر آب دیده‌ام سحبی
\\
دریغ دلبر جان را به مال میل بدی
&&
و یا فریفته گشتی به سیدی چلبی
\\
و یا به حیله و مکری ز ره درافتادی
&&
و یا که مست شدی او ز باده عنبی
\\
دهان به گوش من آرد به گاه نومیدی
&&
چه می‌کند سر و گوش مرا به شهد لبی
\\
غلام ساعت نومیدیم که آن ساعت
&&
شراب وصل بتابد ز شیشه‌ای حلبی
\\
از آن شراب پرستم که یار می بخشست
&&
رخم چو شیشه می کرد و بود رخ ذهبی
\\
برادرم پدرم اصل و فصل من عشقست
&&
که خویش عشق بماند نه خویشی نسبی
\\
خمش که مفخر آفاق شمس تبریزی
&&
بشست نام و نشان مرا به خوش لقبی
\\
\end{longtable}
\end{center}
