\begin{center}
\section*{غزل شماره ۳۰۸۹: به اهل پرده اسرارها ببر خبری}
\label{sec:3089}
\addcontentsline{toc}{section}{\nameref{sec:3089}}
\begin{longtable}{l p{0.5cm} r}
به اهل پرده اسرارها ببر خبری
&&
که پرده‌های شما بردرید از قمری
\\
نشسته بودند یک شب نجوم و سیارات
&&
برای طلعت آن آفتاب در سمری
\\
برید غیرت شمشیر برکشید و برفت
&&
که در چه‌اید بگفتند نیستمان خبری
\\
برید غیرت واگشت و هر یکی می‌گفت
&&
به ناله‌های پرآتش که آه واحذری
\\
شبانگهانی عقرب چو کزدمک می‌رفت
&&
به گوش‌های سراپرده‌هاش بر خطری
\\
که پاسبان سراپرده جلالت او
&&
به نفط قهر بزد تا بسوخت از شرری
\\
دریغ دیده بختم به کحل خاک درش
&&
ز بهر روشنی چشم یافتی نظری
\\
که تا به قوت آن یک نظر بدو کردی
&&
که مهر و ماه نیابند اندر او اثری
\\
که نسر طایر بگذشت از هوس آن سو
&&
به اعتماد که او راست بسته بال و پری
\\
یکی مگس ز شکرهای بی‌کرانه او
&&
پرید در پی آن نسر و برسکست سری
\\
چو بوی خمر رحیقش برون زند ز جهان
&&
خراب و مست ببینی به هر طرف عمری
\\
به بر و بحر فتادست ولوله شادی
&&
که بحر رحمت پوشید قالب بشری
\\
فکند ایمن و ساکن حذرکنان بلا
&&
سلاح‌ها بفراغت ز تیغ یا سپری
\\
که ذره‌های هواها و قطره‌های بحار
&&
به گوش حلقه او کرد و بر میان کمری
\\
چو حق خدمت او ماجرا کند آغاز
&&
یقین شود همه را زانک نیستشان هنری
\\
نگارگر بگه نقش شهرها می‌کرد
&&
گشاد هندسه را پس مهندسانه دری
\\
چو دررسید به تبریز و نقش او ناگاه
&&
برو فتاد شعاعات روح سیمبری
\\
قلم شکست و بیفتاد بی‌خبر بر جای
&&
چو مستیان شبانه ز خوردن سکری
\\
تمام چون کنم این را که خاطر از آتش
&&
همی‌گدازد در آب شکر چون شکری
\\
\end{longtable}
\end{center}
