\begin{center}
\section*{غزل شماره ۸۳: ای یار قمرسیما ای مطرب شکرخا}
\label{sec:0083}
\addcontentsline{toc}{section}{\nameref{sec:0083}}
\begin{longtable}{l p{0.5cm} r}
ای یار قمرسیما ای مطرب شکرخا
&&
آواز تو جان افزا تا روز مشین از پا
\\
سودی همگی سودی بر جمله برافزودی
&&
تا بود چنین بودی تا روز مشین از پا
\\
صد شهر خبر رفته کای مردم آشفته
&&
بیدار شد آن خفته تا روز مشین از پا
\\
بیدار شد آن فتنه کو چون بزند طعنه
&&
در کوه کند رخنه تا روز مشین از پا
\\
در خانه چنین جمعی در جمع چنین شمعی
&&
دارم ز تو من طمعی تا روز مشین از پا
\\
میر آمد میر آمد وان بدر منیر آمد
&&
وان شکر و شیر آمد تا روز مشین از پا
\\
ای بانگ و نوایت تر وز باد صبا خوشتر
&&
ما را تو بری از سر تا روز مشین از پا
\\
مجلس به تو فرخنده عشرت ز دمت زنده
&&
چون شمع فروزنده تا روز مشین از پا
\\
این چرخ و زمین خیمه کس دید چنین خیمه
&&
ای استن این خیمه تا روز مشین از پا
\\
این قوم پرند از تو باکر و فرند از تو
&&
زیر و زبرند از تو تا روز مشین از پا
\\
در بحر چو کشتیبان آن پیل همی‌جنبان
&&
تا منزل آباقان تا روز مشین از پا
\\
ای خوش نفس نایی بس نادره برنایی
&&
چون با همه برنایی تا روز مشین از پا
\\
دف از کف دست آید نی از دم مست آید
&&
با نی همه پست آید تا روز مشین از پا
\\
چون جان خمشیم اما کی خسبد جان جانا
&&
تو باش زبان ما تا روز مشین از پا
\\
\end{longtable}
\end{center}
