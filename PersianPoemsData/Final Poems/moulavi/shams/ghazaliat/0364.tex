\begin{center}
\section*{غزل شماره ۳۶۴: تا نقش خیال دوست با ماست}
\label{sec:0364}
\addcontentsline{toc}{section}{\nameref{sec:0364}}
\begin{longtable}{l p{0.5cm} r}
تا نقش خیال دوست با ماست
&&
ما را همه عمر خود تماشاست
\\
آن جا که وصال دوستانست
&&
والله که میان خانه صحراست
\\
وان جا که مراد دل برآید
&&
یک خار به از هزار خرماست
\\
چون بر سر کوی یار خسبیم
&&
بالین و لحاف ما ثریاست
\\
چون در سر زلف یار پیچیم
&&
اندر شب قدر قدر ما راست
\\
چون عکس جمال او بتابد
&&
کهسار و زمین حریر و دیباست
\\
از باد چو بوی او بپرسیم
&&
در باد صدای چنگ و سرناست
\\
بر خاک چو نام او نویسیم
&&
هر پاره خاک حور و حوراست
\\
بر آتش از او فسون بخوانیم
&&
زو آتش تیزاب سیماست
\\
قصه چه کنم که بر عدم نیز
&&
نامش چو بریم هستی افزاست
\\
آن نکته که عشق او در آن جاست
&&
پرمغزتر از هزار جوزاست
\\
وان لحظه که عشق روی بنمود
&&
این‌ها همه از میانه برخاست
\\
خامش که تمام ختم گشته‌ست
&&
کلی مراد حق تعالاست
\\
\end{longtable}
\end{center}
