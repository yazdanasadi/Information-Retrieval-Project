\begin{center}
\section*{غزل شماره ۷۳۷: نام آن کس بر که مرده از جمالش زنده شد}
\label{sec:0737}
\addcontentsline{toc}{section}{\nameref{sec:0737}}
\begin{longtable}{l p{0.5cm} r}
نام آن کس بر که مرده از جمالش زنده شد
&&
گریه‌های جمله عالم در وصالش خنده شد
\\
یاد آن کس کن که چون خوبی او رویی نمود
&&
حسن‌های جمله عالم حسن او را بنده شد
\\
جمله آب زندگانی زیر تختش می‌رود
&&
هر کی خورد از آب جویش تا ابد پاینده شد
\\
یک شبی خورشید پایه تخت او را بوسه داد
&&
لاجرم بر چرخ گردون تا ابد تابنده شد
\\
زندگی عاشقانش جمله در افکندگیست
&&
خاک طامع بهر این در زیر پا افکنده شد
\\
آهوان را بوی مشک از طره‌اش بر ناف زد
&&
تا مشام شیر صید مرج‌ها غرنده شد
\\
بال و پر وهم عاشق ز آتش دل چون بسوخت
&&
همچو خورشید و قمر بی‌بال و پر پرنده شد
\\
ای خنک جانی که لطف شمس تبریزی بیافت
&&
برگذشت از نه فلک بر لامکان باشنده شد
\\
\end{longtable}
\end{center}
