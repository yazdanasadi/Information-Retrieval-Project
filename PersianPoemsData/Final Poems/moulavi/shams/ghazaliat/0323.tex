\begin{center}
\section*{غزل شماره ۳۲۳: آن نفسی که باخودی یار چو خار آیدت}
\label{sec:0323}
\addcontentsline{toc}{section}{\nameref{sec:0323}}
\begin{longtable}{l p{0.5cm} r}
آن نفسی که باخودی یار چو خار آیدت
&&
وان نفسی که بیخودی یار چه کار آیدت
\\
آن نفسی که باخودی خود تو شکار پشه‌ای
&&
وان نفسی که بیخودی پیل شکار آیدت
\\
آن نفسی که باخودی بسته ابر غصه‌ای
&&
وان نفسی که بیخودی مه به کنار آیدت
\\
آن نفسی که باخودی یار کناره می‌کند
&&
وان نفسی که بیخودی باده یار آیدت
\\
آن نفسی که باخودی همچو خزان فسرده‌ای
&&
وان نفسی که بیخودی دی چو بهار آیدت
\\
جمله بی‌قراریت از طلب قرار تست
&&
طالب بی‌قرار شو تا که قرار آیدت
\\
جمله ناگوارشت از طلب گوارش است
&&
ترک گوارش ار کنی زهر گوار آیدت
\\
جمله بی‌مرادیت از طلب مراد تست
&&
ور نه همه مرادها همچو نثار آیدت
\\
عاشق جور یار شو عاشق مهر یار نی
&&
تا که نگار نازگر عاشق زار آیدت
\\
خسرو شرق شمس دین از تبریز چون رسد
&&
از مه و از ستاره‌ها والله عار آیدت
\\
\end{longtable}
\end{center}
