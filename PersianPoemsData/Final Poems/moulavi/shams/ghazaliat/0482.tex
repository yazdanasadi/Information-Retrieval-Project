\begin{center}
\section*{غزل شماره ۴۸۲: برات عاشق نو کن رسید روز برات}
\label{sec:0482}
\addcontentsline{toc}{section}{\nameref{sec:0482}}
\begin{longtable}{l p{0.5cm} r}
برات عاشق نو کن رسید روز برات
&&
زکات لعل ادا کن رسید وقت زکات
\\
برات و قدر خیالت دو عید چیست وصال
&&
چو این و آن نبود هست نوبت حسرات
\\
به باغ‌های حقایق برات دوست رسید
&&
ز تخته بند زمستان شکوفه یافت نجات
\\
چو طوطیان خبر قند دوست آوردند
&&
ز دشت و کوه برویید صد هزار نبات
\\
دو شادیست عروسان باغ را امروز
&&
وفات در بگشاد و خریف یافت وفات
\\
بیا که نور سماوات خاک را آراست
&&
شکوفه نور حقست و درخت چون مشکات
\\
جهان پر از خضر سبزپوش دانی چیست
&&
که جوش کرد ز خاک و درخت آب حیات
\\
ز لامکان برسیدست حور سوی ملک
&&
ز بی‌جهت برسیدست خلد سوی جهات
\\
طیور نعره ارنی همی‌زنند چرا
&&
که طور یافت ربیع و کلیم جان میقات
\\
به باغ آی و قیامت ببین و حشر عیان
&&
که رعد نفخه صور آمد و نشور موات
\\
اذان فاخته دیدیم و قامت اشجار
&&
خموش کن که سخن شرط نیست وقت صلات
\\
\end{longtable}
\end{center}
