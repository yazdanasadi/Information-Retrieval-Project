\begin{center}
\section*{غزل شماره ۲۵۵۹: الا ای یوسف مصری از این دریای ظلمانی}
\label{sec:2559}
\addcontentsline{toc}{section}{\nameref{sec:2559}}
\begin{longtable}{l p{0.5cm} r}
الا ای یوسف مصری از این دریای ظلمانی
&&
روان کن کشتی وصلت برای پیر کنعانی
\\
یکی کشتی که این دریا ز نور او بگیرد چشم
&&
که از شعشاع آن کشتی بگردد بحر نورانی
\\
نه زان نوری که آن باشد به جان چاکران لایق
&&
از آن نوری که آن باشد جمال و فر سلطانی
\\
در آن بحر جلالت‌ها که آن کشتی همی‌گردد
&&
چو باشد عاشق او حق که باشد روح روحانی
\\
چو آن کشتی نماید رخ برآید گرد آن دریا
&&
نماند صعبیی دیگر بگردد جمله آسانی
\\
چه آسانی که از شادی ز عاشق هر سر مویی
&&
در آن دریا به رقص اندرشده غلطان و خندانی
\\
نبیند خنده جان را مگر که دیده جان‌ها
&&
نماید خدها در جسم آب و خاک ارکانی
\\
ز عریانی نشانی‌هاست بر درز لباس او
&&
ز چشم و گوش و فهم و وهم اگر خواهی تو برهانی
\\
تو برهان را چه خواهی کرد که غرق عالم حسی
&&
برو می‌چر چو استوران در این مرعای شهوانی
\\
مگر الطاف مخدومی خداوندی شمس دین
&&
رباید مر تو را چون باد از وسواس شیطانی
\\
کز این جمله اشارت‌ها هم از کشتی هم از دریا
&&
مکن فهمی مگر در حق آن دریای ربانی
\\
چو این را فهم کردی تو سجودی بر سوی تبریز
&&
که تا او را بیابد جان ز رحمت‌های یزدانی
\\
\end{longtable}
\end{center}
