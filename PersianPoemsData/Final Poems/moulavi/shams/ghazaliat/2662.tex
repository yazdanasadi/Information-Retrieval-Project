\begin{center}
\section*{غزل شماره ۲۶۶۲: دلا رو رو همان خون شو که بودی}
\label{sec:2662}
\addcontentsline{toc}{section}{\nameref{sec:2662}}
\begin{longtable}{l p{0.5cm} r}
دلا رو رو همان خون شو که بودی
&&
بدان صحرا و هامون شو که بودی
\\
در این خاکستر هستی چو غلطی
&&
در آتشدان و کانون شو که بودی
\\
در این چون شد چگونه چند مانی
&&
بدان تصریف بی‌چون شو که بودی
\\
نه گاوی که کشی بیگار گردون
&&
بر آن بالای گردون شو که بودی
\\
در این کاهش چو بیماران دقی
&&
به عمر روزافزون شو که بودی
\\
زبون طب افلاطون چه باشی
&&
فلاطون فلاطون شو که بودی
\\
ایم هو کی اسیرانه چه باشی
&&
همان سلطان و بارون شو که بودی
\\
اگر رویین تنی جسم آفت توست
&&
همان جان فریدون شو که بودی
\\
همان اقبال و دولت بین که دیدی
&&
همان بخت همایون شو که بودی
\\
رها کن نظم کردن درها را
&&
به دریا در مکنون شو که بودی
\\
\end{longtable}
\end{center}
