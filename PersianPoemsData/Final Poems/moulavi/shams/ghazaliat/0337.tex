\begin{center}
\section*{غزل شماره ۳۳۷: ببستی چشم یعنی وقت خوابست}
\label{sec:0337}
\addcontentsline{toc}{section}{\nameref{sec:0337}}
\begin{longtable}{l p{0.5cm} r}
ببستی چشم یعنی وقت خواب است
&&
نه خوابت آن حریفان را جواب است
\\
تو می‌دانی که ما چندان نپاییم
&&
ولیکن چشم مستت را شتاب است
\\
جفا می‌کن جفاات جمله لطف است
&&
خطا می‌کن خطای تو صواب است
\\
تو چشم آتشین در خواب می‌کن
&&
که ما را چشم و دل باری کباب است
\\
بسی سرها ربوده چشم ساقی
&&
به شمشیری که آن یک قطره آب است
\\
یکی گوید که این از عشق ساقیست
&&
یکی گوید که این فعل شراب است
\\
می و ساقی چه باشد نیست جز حق
&&
خدا داند که این عشق از چه باب است
\\
\end{longtable}
\end{center}
