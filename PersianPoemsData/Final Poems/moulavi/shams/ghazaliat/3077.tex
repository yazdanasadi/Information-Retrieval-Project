\begin{center}
\section*{غزل شماره ۳۰۷۷: ایا مربی جان از صداع جان چونی}
\label{sec:3077}
\addcontentsline{toc}{section}{\nameref{sec:3077}}
\begin{longtable}{l p{0.5cm} r}
ایا مربی جان از صداع جان چونی
&&
ایا ببرده دل از جمله دلبران چونی
\\
ز زحمت شب ما و ز ناله‌های صبوح
&&
که می‌رسد به تو ای ماه مهربان چونی
\\
ایا کسی که نخفت و نخفت چشم خوشت
&&
ز لکلک جرس و بانگ پاسبان چونی
\\
ایا غریب فلک تو بر این زمین حیفی
&&
ایا جهان ملاحت در این جهان چونی
\\
ز آفتاب کی پرسد که چون همی‌گردی
&&
به گلستان که بگوید که گلستان چونی
\\
ز روی زرد بپرسند درد دل چونست
&&
ولی کسی بنپرسد که ارغوان چونی
\\
چو روی زشت به آیینه گفت چونی تو
&&
بگفت من چو چراغم تو قلتبان چونی
\\
جواب گفت که من بازگونه می‌پرسم
&&
مثال کشت که گوید به آسمان چونی
\\
دهان گشادم یعنی ببین که لب خشکم
&&
که تا شراب تو گوید که ای دهان چونی
\\
ز گفت چون تو جویی روان شود در حال
&&
میان جان و روانم که ای روان چونی
\\
بگو تو باقی این را که از خمار لبت
&&
سرم گران شد پرسش که سرگران چونی
\\
\end{longtable}
\end{center}
