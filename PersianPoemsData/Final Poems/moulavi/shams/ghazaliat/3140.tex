\begin{center}
\section*{غزل شماره ۳۱۴۰: صنما بر همه جهان تو چو خورشید سروری}
\label{sec:3140}
\addcontentsline{toc}{section}{\nameref{sec:3140}}
\begin{longtable}{l p{0.5cm} r}
صنما بر همه جهان تو چو خورشید سروری
&&
قمرا می‌رسد تو را که به خورشید بنگری
\\
همه عالم چو جان شود همگی گلستان شود
&&
شکم خاک کان شود چو تو بر خاک بگذری
\\
تن من همچو رشته شد به دلم مهر کشته شد
&&
چو به سر این نوشته شد نبود کار سرسری
\\
چو سحر پرده می‌درد تو پس پرده می‌روی
&&
چو به شب پرده می‌کشد تو به شب پرده می‌دری
\\
صنما خاک پای خود تو مرا سرمه وام ده
&&
که نظر در تو خیره شد که تو خورشیدمنظری
\\
رخ خوبان این جهان همه ابرست و تو مهی
&&
سر شاهان این جهان همه پایست و تو سری
\\
چو درآمد خیال تو مه نو تیره شد بگفت
&&
چه عجب گر تو روشنی که از او آب می‌خوری
\\
\end{longtable}
\end{center}
