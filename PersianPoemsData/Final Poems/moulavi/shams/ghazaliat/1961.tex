\begin{center}
\section*{غزل شماره ۱۹۶۱: نوبهارا جان مایی جان‌ها را تازه کن}
\label{sec:1961}
\addcontentsline{toc}{section}{\nameref{sec:1961}}
\begin{longtable}{l p{0.5cm} r}
نوبهارا جان مایی جان‌ها را تازه کن
&&
باغ‌ها را بشکفان و کشت‌ها را تازه کن
\\
گل جمال افروخته‌ست و مرغ قول آموخته‌ست
&&
بی صبا جنبش ندارند هین صبا را تازه کن
\\
سرو سوسن را همی‌گوید زبان را برگشا
&&
سنبله با لاله می گوید وفا را تازه کن
\\
شد چناران دف زنان و شد صنوبر کف زنان
&&
فاخته نعره زنان کوکو عطا را تازه کن
\\
از گل سوری قیام و از بنفشه بین رکوع
&&
برگ رز اندر سجود آمد صلا را تازه کن
\\
جمله گل‌ها صلح جو و خار بدخو جنگ جو
&&
خیز ای وامق تو باری عهد عذرا تازه کن
\\
رعد گوید ابر آمد مشک‌ها بر خاک ریخت
&&
ای گلستان رو بشو و دست و پا را تازه کن
\\
نرگس آمد سوی بلبل خفته چشمک می زند
&&
کاندرآ اندر نوا عشق و هوا را تازه کن
\\
بلبل این بشنید از او و با گل صدبرگ گفت
&&
گر سماعت میل شد این بی‌نوا را تازه کن
\\
سبزپوشان خضرکسوه همی‌گویند رو
&&
چون شکوفه سر سر اولیا را تازه کن
\\
وان سه برگ و آن سمن وان یاسمین گویند نی
&&
در خموشی کیمیا بین کیمیا را تازه کن
\\
\end{longtable}
\end{center}
