\begin{center}
\section*{غزل شماره ۸۵۹: نی دیده هر دلی را دیدار می‌نماید}
\label{sec:0859}
\addcontentsline{toc}{section}{\nameref{sec:0859}}
\begin{longtable}{l p{0.5cm} r}
نی دیده هر دلی را دیدار می‌نماید
&&
نی هر خسیس را شه رخسار می‌نماید
\\
الا حقیر ما را الا خسیس ما را
&&
کز خار می‌رهاند گلزار می‌نماید
\\
دود سیاه ما را در نور می‌کشاند
&&
زهد قدیم ما را خمار می‌نماید
\\
هرگز غلام خود را نفروشد و نبخشد
&&
تا چیست اینک او را بازار می‌نماید
\\
شیریست پور آدم صندوق عالم اندر
&&
صندوق درشدست او بیمار می‌نماید
\\
روزی که او بغرد صندوق را بدرد
&&
کاری نماید اکنون بی‌کار می‌نماید
\\
صدیق با محمد بر هفت آسمانست
&&
هر چند کو به ظاهر در غار می‌نماید
\\
یکیست عشق لیکن هر صورتی نماید
&&
وین احولان خس را دوچار می‌نماید
\\
جمله گلست این ره گر ظاهرش چو خارست
&&
نور از درخت موسی چون نار می‌نماید
\\
آب حیات آمد وین بانگ سیلابست
&&
گفتار نیست لیکن گفتار می‌نماید
\\
سوگند خورده بودم کز دل سخن نگویم
&&
دل آینه‌ست و رو را ناچار می‌نماید
\\
شمس الحقی که نورش بر آینه‌ست تابان
&&
در جنبش این و آن را دیوار می‌نماید
\\
هر طبله که گشایم زان قند بی‌کرانست
&&
کان را به نوع دیگر عطار می‌نماید
\\
\end{longtable}
\end{center}
