\begin{center}
\section*{غزل شماره ۲۰۷: ای که به هنگام درد راحت جانی مرا}
\label{sec:0207}
\addcontentsline{toc}{section}{\nameref{sec:0207}}
\begin{longtable}{l p{0.5cm} r}
ای که به هنگام درد راحت جانی مرا
&&
وی که به تلخی فقر گنج روانی مرا
\\
آن چه نبردست وهم عقل ندیدست و فهم
&&
از تو به جانم رسید قبله ازانی مرا
\\
از کرمت من به ناز می‌نگرم در بقا
&&
کی بفریبد شها دولت فانی مرا
\\
نغمت آن کس که او مژده تو آورد
&&
گر چه به خوابی بود به ز اغانی مرا
\\
در رکعات نماز هست خیال تو شه
&&
واجب و لازم چنانک سبع مثانی مرا
\\
در گنه کافران رحم و شفاعت تو راست
&&
مهتری و سروری سنگ دلانی مرا
\\
گر کرم لایزال عرضه کند ملک‌ها
&&
پیش نهد جمله‌ای کنز نهانی مرا
\\
سجده کنم من ز جان روی نهم من به خاک
&&
گویم از این‌ها همه عشق فلانی مرا
\\
عمر ابد پیش من هست زمان وصال
&&
زانک نگنجد در او هیچ زمانی مرا
\\
عمر اوانی‌ست و وصل شربت صافی در آن
&&
بی تو چه کار آیدم رنج اوانی مرا
\\
بیست هزار آرزو بود مرا پیش از این
&&
در هوسش خود نماند هیچ امانی مرا
\\
از مدد لطف او ایمن گشتم از آنک
&&
گوید سلطان غیب لست ترانی مرا
\\
گوهر معنی اوست پر شده جان و دلم
&&
اوست اگر گفت نیست ثالث و ثانی مرا
\\
رفت وصالش به روح جسم نکرد التفات
&&
گر چه مجرد ز تن گشت عیانی مرا
\\
پیر شدم از غمش لیک چو تبریز را
&&
نام بری بازگشت جمله جوانی مرا
\\
\end{longtable}
\end{center}
