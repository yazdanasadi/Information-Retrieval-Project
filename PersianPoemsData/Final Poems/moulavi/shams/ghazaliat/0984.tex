\begin{center}
\section*{غزل شماره ۹۸۴: هین که هنگام صابران آمد}
\label{sec:0984}
\addcontentsline{toc}{section}{\nameref{sec:0984}}
\begin{longtable}{l p{0.5cm} r}
هین که هنگام صابران آمد
&&
وقت سختی و امتحان آمد
\\
این چنین وقت عهدها شکنند
&&
کارد چون سوی استخوان آمد
\\
عهد و سوگند سخت سست شود
&&
مرد را کار چون به جان آمد
\\
هله ای دل تو خویش سست مکن
&&
دل قوی کن که وقت آن آمد
\\
چون زر سرخ اندر آتش خند
&&
تا بگویند زر کان آمد
\\
گرم خوش رو به پیش تیغ اجل
&&
بانگ برزن که پهلوان آمد
\\
با خدا باش و نصرت از وی خواه
&&
که مددها ز آسمان آمد
\\
ای خدا آستین فضل فشان
&&
چونک بنده بر آستان آمد
\\
چون صدف ما دهان گشادستیم
&&
کابر فضل تو درفشان آمد
\\
ای بسا خار خشک کز دل او
&&
در پناه تو گلستان آمد
\\
من نشان کرده‌ام تو را که ز تو
&&
دلخوشی‌های بی‌نشان آمد
\\
وقت رحمست و وقت عاطفت است
&&
که مرا زخم بس گران آمد
\\
ای ابابیل هین که بر کعبه
&&
لشکر و پیل بی‌کران آمد
\\
عقل گوید مرا خمش کن بس
&&
که خداوند غیب دان آمد
\\
من خمش کردم ای خدا لیکن
&&
بی من از خان من فغان آمد
\\
ما رمیت اذ رمیت هم ز خداست
&&
تیر ناگه کز این کمان آمد
\\
\end{longtable}
\end{center}
