\begin{center}
\section*{غزل شماره ۵۳۰: امروز خندانیم و خوش کان بخت خندان می‌رسد}
\label{sec:0530}
\addcontentsline{toc}{section}{\nameref{sec:0530}}
\begin{longtable}{l p{0.5cm} r}
امروز خندانیم و خوش کان بخت خندان می‌رسد
&&
سلطان سلطانان ما از سوی میدان می‌رسد
\\
امروز توبه بشکنم پرهیز را برهم زنم
&&
کان یوسف خوبان من از شهر کنعان می‌رسد
\\
مست و خرامان می‌روم پوشیده چون جان می‌روم
&&
پرسان و جویان می‌روم آن سو که سلطان می‌رسد
\\
اقبال آبادان شده دستار دل ویران شده
&&
افتان شده خیزان شده کز بزم مستان می‌رسد
\\
فرمان ما کن ای پسر با ما وفا کن ای پسر
&&
نسیه رها کن ای پسر کامروز فرمان می‌رسد
\\
پرنور شو چون آسمان سرسبزه شو چون بوستان
&&
شو آشنا چون ماهیان کان بحر عمان می‌رسد
\\
هان ای پسر هان ای پسر خود را ببین در من نگر
&&
زیرا ز بوی زعفران گویند خندان می‌رسد
\\
بازآمدی کف می‌زنی تا خانه‌ها ویران کنی
&&
زیرا که در ویرانه‌ها خورشید رخشان می‌رسد
\\
ای خانه را گشته گرو تو سایه پروردی برو
&&
کز آفتاب آن سنگ را لعل بدخشان می‌رسد
\\
گه خونی و خون خواره‌ای گه خستگان را چاره‌ای
&&
خاصه که این بیچاره را کز سوی ایشان می‌رسد
\\
امروز مستان را بجو غیبم ببین عیبم مگو
&&
زیرا ز مستی‌های او حرفم پریشان می‌رسد
\\
\end{longtable}
\end{center}
