\begin{center}
\section*{غزل شماره ۱۹۴۵: هر چه آن سرخوش کند بویی بود از یار من}
\label{sec:1945}
\addcontentsline{toc}{section}{\nameref{sec:1945}}
\begin{longtable}{l p{0.5cm} r}
هر چه آن سرخوش کند بویی بود از یار من
&&
هر چه دل واله کند آن پرتو دلدار من
\\
خاک را و خاکیان را این همه جوشش ز چیست
&&
ریخت بر روی زمین یک جرعه از خمار من
\\
هر که را افسرده دیدی عاشق کار خود است
&&
منگر اندر کار خویش و بنگر اندر کار من
\\
در بهاران گشت ظاهر جمله اسرار زمین
&&
چون بهار من بیاید بردمد اسرار من
\\
چون به گلزار زمین خار زمین پوشیده شد
&&
خارخار من نماند چون دمد گلزار من
\\
هر کی بیمار خزان شد شربتی خورد از بهار
&&
چون بهار من بخندد برجهد بیمار من
\\
چیست این باد خزانی آن دم انکار تو
&&
چیست آن باد بهاری آن دم اقرار من
\\
\end{longtable}
\end{center}
