\begin{center}
\section*{غزل شماره ۲۸۰۶: آه کان سایه خدا گوهردلی پرمایه‌ای}
\label{sec:2806}
\addcontentsline{toc}{section}{\nameref{sec:2806}}
\begin{longtable}{l p{0.5cm} r}
آه کان سایه خدا گوهردلی پرمایه‌ای
&&
آفتاب او نهشت اندر دو عالم سایه‌ای
\\
آفتاب و چرخ را چون ذره‌ها برهم زند
&&
وز جمال خود دهدشان نو به نو سرمایه‌ای
\\
عشق و عاشق را چه خوش خندان کنی رقصان کنی
&&
عشق سازی عقل سوزی طرفه‌ای خودرایه‌ای
\\
چشم مرده وام کرده جان ز بهر عشق او
&&
ز آنک در دیده بدیده جان از آن سر پایه‌ای
\\
قهر صد دندان ز لطفش پیر بی‌دندان شده
&&
عقل پابرجا ز عشقش یاوه و هرجایه‌ای
\\
صد هزاران ساله از هست و عدم زان سوتری
&&
وز تواضع مر عدم را هست خوش همسایه‌ای
\\
کوه حلمی شمس تبریزی دو عالم تخت تو
&&
بر نهان و آشکارش می‌نگر از قایه‌ای
\\
\end{longtable}
\end{center}
