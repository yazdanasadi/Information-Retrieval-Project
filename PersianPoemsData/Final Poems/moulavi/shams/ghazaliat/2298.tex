\begin{center}
\section*{غزل شماره ۲۲۹۸: زهی بزم خداوندی زهی می‌های شاهانه}
\label{sec:2298}
\addcontentsline{toc}{section}{\nameref{sec:2298}}
\begin{longtable}{l p{0.5cm} r}
زهی بزم خداوندی زهی می‌های شاهانه
&&
زهی یغما که می‌آرد شه قفجاق ترکانه
\\
دلم آهن همی‌خاید از آن لعلین لبی که او
&&
کنار لطف بگشاید میان حلقه مستانه
\\
هر آن جانی که شد مجنون به عشق حالت بی‌چون
&&
کجا گیرد قرار اکنون بدین افسون و افسانه
\\
چو او طره برافشاند سوی عاشق همی‌داند
&&
که از زنجیر جنبیدن بجنبد شور دیوانه
\\
به عشق طره‌های او که جعد و شاخ شاخ آمد
&&
دل من شاخ شاخ آید چو دندان در سر شانه
\\
چه برهم گشته‌اند این دم حریفان دل از مستی
&&
برای جانت ای مه رو سری درکن در این خانه
\\
اگر ساقی ندادت می دلا در گل چه افتادی
&&
وگر آن مشک نگشاد او چرا پر گشت پیمانه
\\
خداوندا در این بیشه چه گم گشته‌ست اندیشه
&&
تنی تن کجا ماند میان جان و جانانه
\\
بیا ای شمس تبریزی که در رفعت سلیمانی
&&
که از عشقت همه مرغان شدند از دام و از دانه
\\
\end{longtable}
\end{center}
