\begin{center}
\section*{غزل شماره ۱۹۴۱: مهره‌ای از جان ربودم بی‌دهان و بی‌دهان}
\label{sec:1941}
\addcontentsline{toc}{section}{\nameref{sec:1941}}
\begin{longtable}{l p{0.5cm} r}
مهره‌ای از جان ربودم بی‌دهان و بی‌دهان
&&
گر رقیب او بداند گو بدان و گو بدان
\\
سر او را نقش کردم نقش کردم نقش کرد
&&
هر که خواهد گو بخوان و گو بخوان و گو بخوان
\\
پیش منکر می شدم من نیستم من نیستم
&&
هستم اکنون در میان و در میان و در میان
\\
گر تو گویی کو درستی کو درستی کو گواه
&&
در شکست من بیان و صد بیان و صد بیان
\\
اشک چشمم بس گواه و بس گواه و بس گواه
&&
رنگ رویم بس نشان و بس نشان و بس نشان
\\
نک نشان لاله رویی لاله رویی لاله‌ای
&&
بر رخ من زعفران و زعفران و زعفران
\\
جز صلاح الدین نداند این سخن را این سخن
&&
من غلام زیرکان و زیرکان و زیرکان
\\
\end{longtable}
\end{center}
