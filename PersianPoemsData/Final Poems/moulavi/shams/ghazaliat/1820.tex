\begin{center}
\section*{غزل شماره ۱۸۲۰: هی چه گریزی چندین یک نفس این جا بنشین}
\label{sec:1820}
\addcontentsline{toc}{section}{\nameref{sec:1820}}
\begin{longtable}{l p{0.5cm} r}
هی چه گریزی چندین یک نفس این جا بنشین
&&
صبر تو کو ای صابر ای همه صبر و تمکین
\\
ما دو سه کس نو مرده منتظر آن پرده
&&
زنده شویم از تلقین بازرهیم از تکفین
\\
هی به سلف نفخی کن پیشتر از یوم الدین
&&
تا شنود چرخ فلک از حشر تو تحسین
\\
هی به زبان ما گو رمز مگو پیدا گو
&&
چند خوری خون به ستم ای همه خویت خونین
\\
چند گزی بر جگرش چند کنی قصد سرش
&&
چند دهی بد خبرش کار چنین است و چنین
\\
چند کنی تلخ لبش چند کنی تیره شبش
&&
ای لب تو همچو شکر ای شب تو خلد برین
\\
هیچ عسل زهر دهد یا ز شکر سرکه جهد
&&
مغلطه تا چند دهی ای غلط انداز مهین
\\
هر چه کنی آن لب تو باشد غماز شکر
&&
هر حرکت که تو کنی هست در آن لطف دفین
\\
سرو چه ماند به خسی زر به چه ماند به مسی
&&
تو به چه مانی به کسی ای ملک یوم الدین
\\
\end{longtable}
\end{center}
