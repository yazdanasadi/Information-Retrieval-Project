\begin{center}
\section*{غزل شماره ۷۲۴: گرمابه دهر جان فزا بود}
\label{sec:0724}
\addcontentsline{toc}{section}{\nameref{sec:0724}}
\begin{longtable}{l p{0.5cm} r}
گرمابه دهر جان فزا بود
&&
زیرا که در او پری ما بود
\\
مر پریان را ز حیرت او
&&
هر گوشه مقال و ماجرا بود
\\
عقلست چراغ ماجراها
&&
آن جا هش و عقل از کجا بود
\\
در صرصر عشق عقل پشه‌ست
&&
آن جا چه مجال عقل‌ها بود
\\
از احمد پا کشید جبریل
&&
از سدره سفر چو ماورا بود
\\
گفتا که بسوزم ار بیایم
&&
کان سو همه عشق بد ولا بود
\\
تعظیم و مواصلت دو ضدند
&&
در فسحت وصل آن هبا بود
\\
آن جا لیلی شدست مجنون
&&
زیرا که جنون هزار تا بود
\\
آن جا حسنی نقاب بگشود
&&
پیراهن حسن‌ها قبا بود
\\
یوسف در عشق بد زلیخا
&&
نی زهره و چنگ و نی نوا بود
\\
وان نافخ صور مانده بی‌روح
&&
کان جا جز روح دوست لا بود
\\
در بحر گریخت این مقالات
&&
زیرا هنگام آشنا بود
\\
\end{longtable}
\end{center}
