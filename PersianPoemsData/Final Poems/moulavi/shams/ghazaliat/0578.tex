\begin{center}
\section*{غزل شماره ۵۷۸: مرا عهدیست با شادی که شادی آن من باشد}
\label{sec:0578}
\addcontentsline{toc}{section}{\nameref{sec:0578}}
\begin{longtable}{l p{0.5cm} r}
مرا عهدیست با شادی که شادی آن من باشد
&&
مرا قولیست با جانان که جانان جان من باشد
\\
به خط خویشتن فرمان به دستم داد آن سلطان
&&
که تا تختست و تا بختست او سلطان من باشد
\\
اگر هشیار اگر مستم نگیرد غیر او دستم
&&
وگر من دست خود خستم همو درمان من باشد
\\
چه زهره دارد اندیشه که گرد شهر من گردد
&&
کی قصد ملک من دارد چو او خاقان من باشد
\\
نبیند روی من زردی به اقبال لب لعلش
&&
بمیرد پیش من رستم چو از دستان من باشد
\\
بدرم زهره زهره خراشم ماه را چهره
&&
برم از آسمان مهره چو او کیوان من باشد
\\
بدرم جبه مه را بریزم ساغر شه را
&&
وگر خواهند تاوانم همو تاوان من باشد
\\
چراغ چرخ گردونم چو اجری خوار خورشیدم
&&
امیر گوی و چوگانم چو دل میدان من باشد
\\
منم مصر و شکرخانه چو یوسف در برم گیرم
&&
چه جویم ملک کنعان را چو او کنعان من باشد
\\
زهی حاضر زهی ناظر زهی حافظ زهی ناصر
&&
زهی الزام هر منکر چو او برهان من باشد
\\
یکی جانیست در عالم که ننگش آید از صورت
&&
بپوشد صورت انسان ولی انسان من باشد
\\
سر ما هست و من مجنون مجنبانید زنجیرم
&&
مرا هر دم سر مه شد چو مه بر خوان من باشد
\\
سخن بخش زبان من چو باشد شمس تبریزی
&&
تو خامش تا زبان‌ها خود چو دل جنبان من باشد
\\
\end{longtable}
\end{center}
