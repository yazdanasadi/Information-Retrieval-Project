\begin{center}
\section*{غزل شماره ۱۸۴۷: خرامان می روی در دل چراغ افروز جان و تن}
\label{sec:1847}
\addcontentsline{toc}{section}{\nameref{sec:1847}}
\begin{longtable}{l p{0.5cm} r}
خرامان می روی در دل چراغ افروز جان و تن
&&
زهی چشم و چراغ دل زهی چشمم به تو روشن
\\
زهی دریای پرگوهر زهی افلاک پراختر
&&
زهی صحرای پرعبهر زهی بستان پرسوسن
\\
ز تو اجسام را چستی ز تو ارواح را مستی
&&
ایا پر کرده گوهرها جهان خاک را دامن
\\
چه می گویم من ای دلبر نظیر تو دو سه ابتر
&&
چه تشبیهت کنم دیگر چه دارم من چه دانم من
\\
بگو ای چشم حیران را چو دیدی لطف جانان را
&&
چه خواهی دید خلقان را چه گردی گرد آهرمن
\\
شکار شیر بگذاری شکار خوک برداری
&&
زهی تدبیر و هشیاری زهی بیگار و جان کندن
\\
مرا باری عنایاتش خطابات و مراعاتش
&&
شعاعات و ملاقاتش یکی طوقی است در گردن
\\
حلاوت‌های آن مفضل قرار و صبر برد از دل
&&
که دیدم غیر او تا من سکون یابم در این مسکن
\\
به غیر آن جلال و عز که او دیگر نشد هرگز
&&
همه درمانده و عاجز ز خاص و عام و مرد و زن
\\
منم از عشق افروزان مثال آتش از هیزم
&&
ز غیر عشق بیگانه مثال آب با روغن
\\
بسوزان هر چه من دارم به غیر دل که اندر دل
&&
به هر ساعت همی‌سازی ز کر و فر خود گلشن
\\
غلام زنگی شب را تو کردی ساقی خلقان
&&
غلام روز رومی را بدادی دار و گیر و فن
\\
وانگه این دو لالا را رقیب مرد و زن کردی
&&
که تا چون دانه شان از که گزینی اندر این خرمن
\\
همه صاحب دلان گندم که بامغزند و بالذت
&&
همه جسمانیان چون که که بی‌مغزند در مطحن
\\
درخت سبز صاحب دل میان باغ دین خندان
&&
درخت خشک بی‌معنی چه باشد هیزم گلخن
\\
خیالت می رود در دل چو عیسی بهر جان بخشی
&&
چنانک وحی ربانی به موسی جانب ایمن
\\
خیالت را نشانی‌ها زر و گوهرفشانی‌ها
&&
کز او خندان شود دندان کز او گویا شود الکن
\\
دو غماز دگر دارم یکی عشق و دگر مستی
&&
حریفان را نمی‌گویم یکی از دیگری احسن
\\
ز تو ای دیده و دینم هزاران لطف می بینم
&&
ولیکن خاطر عاشق بداندیش آمد و بدظن
\\
ز چشم روز می ترسم که چشمش سحرها دارد
&&
ز زلف شام می ترسم که شب فتنه است و آبستن
\\
مرا گوید چه می ترسی که کوبد مر تو را محنت
&&
که سرمه نور دیده شد چو شد ساییده در هاون
\\
همه خوف از وجود آید بر او کم لرز و کم می زن
&&
همه ترس از شکست آید شکسته شو ببین مؤمن
\\
ز ارکان من بدزدیدم زر و در کیسه پیچیدم
&&
ز ترس بازدادن من چو دزدانم در این مکمن
\\
سبوس ار چه که پنهان شد میان آرد چون دزدان
&&
کشاند شحنه دادش ز هر گوشه به پرویزن
\\
چو هیزم بی‌خبر بودی ز عشق آتش به تو درزد
&&
بجه چون برق از این آتش برآ چون دود از این روزن
\\
چه خنجر می کشی این جا تو گردن پیش خنجر نه
&&
که تا زفتی نگنجی تو درون چشمه سوزن
\\
در جنت چو تنگ آمد مثال چشمه سوزن
&&
اگر خواهی چو پشمی شو لتغزل ذاک تغزیلا
\\
بود کان غزل در سوزن نگنجد کاین دمت غزل است
&&
که می ریسی ز پنبه تن که بافی حله ادکن
\\
لباس حله ادکن ز غزل پنبگی ناید
&&
مگر این پنبه ابریشم شود ز اکسیر آن مخزن
\\
چو ابریشم شوی آید و ریشم تاب وحی او
&&
تو را گوید بریس اکنون بدم پیغام مستحسن
\\
چه باشد وحی در تازی به گوش اندر سخن گفتن
&&
دهل می نشنود گوشت به جهد و جد نوبت زن
\\
گران گوشی وانگه تو به گوش اندرکنی پنبه
&&
چنانک گفت واستغشوا بپیچی سر به پیراهن
\\
گران گوشی گران جسمی گران جانی نذیر آمد
&&
که می گوید تو را هر یک الا یا علج لا تؤمن
\\
سبک گوشی سبک جسمی سبک جانی بشیر آمد
&&
که می گوید تو را هر یک الا یا لیث لا تحزن
\\
بهاری باش تا خوبان به بستان در تو آویزند
&&
که بگریزند این خوبان ز شکل بارد بهمن
\\
بهار ار نیستی اکنون چو تابستان در آتش رو
&&
که بی‌آن حسن و بی‌آن عشق باشد مرد مستهجن
\\
اگر خواهی که هر جزوت شود گویا و شاعر رو
&&
خمش کن سوی این منطق به نظم و نثر لاترکن
\\
که برکنده شوی از فکر چون در گفت می آیی
&&
مکن از فکر دل خود را از این گفت زبان برکن
\\
قضا خنبک زند گوید که مردان عهدها کردند
&&
شکستم عهدهاشان را هلا می کوش ما امکن
\\
ستیزه می کنی با خود کز این پس من چنین باشم
&&
ز استیزه چه بربندی قضا را بنگر ای کودن
\\
نکاحی می کند با دل به هر دم صورت غیبی
&&
نزاید گر چه جمع آیند صد عنین و استرون
\\
صور را دل شده جاذب چو عنین شهوت کاذب
&&
ز خوبان نیست عنین را به جز بخشیدن وجکن
\\
بیا ای شمس تبریزی که سلطانی و خون ریزی
&&
قضا را گو که از بالا جهان را در بلا مفکن
\\
\end{longtable}
\end{center}
