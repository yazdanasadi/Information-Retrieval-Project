\begin{center}
\section*{غزل شماره ۶۹: چه باشد گر نگارینم بگیرد دست من فردا}
\label{sec:0069}
\addcontentsline{toc}{section}{\nameref{sec:0069}}
\begin{longtable}{l p{0.5cm} r}
چه باشد گر نگارینم بگیرد دست من فردا
&&
ز روزن سر درآویزد چو قرص ماه خوش سیما
\\
درآید جان فزای من گشاید دست و پای من
&&
که دستم بست و پایم هم کف هجران پابرجا
\\
بدو گویم به جان تو که بی‌تو ای حیات جان
&&
نه شادم می‌کند عشرت نه مستم می‌کند صهبا
\\
وگر از ناز او گوید برو از من چه می‌خواهی
&&
ز سودای تو می‌ترسم که پیوندد به من سودا
\\
برم تیغ و کفن پیشش چو قربانی نهم گردن
&&
که از من دردسر داری مرا گردن بزن عمدا
\\
تو می‌دانی که من بی‌تو نخواهم زندگانی را
&&
مرا مردن به از هجران به یزدان کاخرج الموتی
\\
مرا باور نمی‌آمد که از بنده تو برگردی
&&
همی‌گفتم اراجیفست و بهتان گفته اعدا
\\
تویی جان من و بی‌جان ندانم زیست من باری
&&
تویی چشم من و بی‌تو ندارم دیده بینا
\\
رها کن این سخن‌ها را بزن مطرب یکی پرده
&&
رباب و دف به پیش آور اگر نبود تو را سرنا
\\
\end{longtable}
\end{center}
