\begin{center}
\section*{غزل شماره ۱۷۲۷: به غم فرونروم باز سوی یار روم}
\label{sec:1727}
\addcontentsline{toc}{section}{\nameref{sec:1727}}
\begin{longtable}{l p{0.5cm} r}
به غم فرونروم باز سوی یار روم
&&
در آن بهشت و گلستان و سبزه زار روم
\\
ز برگ ریز خزان فراق سیر شدم
&&
به گلشن ابد و سرو پایدار روم
\\
من از شمار بشر نیستم وداع وداع
&&
به نقل و مجلس و سغراق بی‌شمار روم
\\
نمی‌شکیبد ماهی ز آب من چه کنم
&&
چو آب سجده کنان سوی جویبار روم
\\
به عاقبت غم عشقم کشان کشان ببرد
&&
همان به‌ست که اکنون به اختیار روم
\\
ز داد عشق بود کار و بار سلطانان
&&
به عشق درنروم در کدام کار روم
\\
شنیده‌ام که امیر بتان به صید شده‌ست
&&
اگر چه لاغرم سوی مرغزار روم
\\
چو شیر عشق فرستد سگان خود به شکار
&&
به عشق دل به دهان سگ شکار روم
\\
چو بر براق سعادت کنون سوار شدم
&&
به سوی سنجق سلطان کامیار روم
\\
جهان عشق به زیر لوای سلطانی است
&&
چو از رعیت عشقم بدان دیار روم
\\
منم که در نظرم خوار گشت جان و جهان
&&
بدان جهان و بدان جان بی‌غبار روم
\\
غبار تن نبود ماه جان بود آن جا
&&
سزد سزد که بر آن چرخ برق وار روم
\\
اگر کلیم حلیمم بدان درخت شوم
&&
وگر خلیل جلیلم در آن شرار روم
\\
خموش کی هلدم تشنگی این یاران
&&
مگر که از بر یاران به یار غار روم
\\
جوار مفخر آفاق شمس تبریزی
&&
بهشت عدن بود هم در آن جوار روم
\\
\end{longtable}
\end{center}
