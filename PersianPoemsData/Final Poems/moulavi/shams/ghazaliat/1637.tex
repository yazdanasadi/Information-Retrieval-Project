\begin{center}
\section*{غزل شماره ۱۶۳۷: منم آن دزد که شب نقب زدم ببریدم}
\label{sec:1637}
\addcontentsline{toc}{section}{\nameref{sec:1637}}
\begin{longtable}{l p{0.5cm} r}
منم آن دزد که شب نقب زدم ببریدم
&&
سر صندوق گشادم گهری دزدیدم
\\
ز زلیخای حرم چادر سر بربودم
&&
چو بدیدم رخ یوسف کف خود ببریدم
\\
سر سودای کسی قصد سر من دارد
&&
کی برد سر ز کف آنک از آن سر دیدم
\\
چو بگفتم نبرم سر سر من گفت آمین
&&
چون غمش کند ز بیخم پس از آن روییدم
\\
این چه ماه است که اندر دل و جان‌ها گردد
&&
که من از گردش او بس چو فلک گردیدم
\\
جان اخوان صفا اوست که اندر هوسش
&&
همه دردی جهان در سر خود مالیدم
\\
اندر این چاه جهان یوسف حسنی است نهان
&&
من بر این چرخ از او همچو رسن پیچیدم
\\
هله ای عشق بیا یار منی در دو جهان
&&
از همه خلق بریدم به تو برچفسیدم
\\
زان چنین در فرحم کز قدحت سرمستم
&&
زان گزیده‌ست مرا حق که تو را بگزیدم
\\
بنهان از همه خلقان چه خوش آیین باغی است
&&
که چو گل در چمنش جامه جان بدریدم
\\
اندر آن باغ یکی دلبر بالاشجری است
&&
که چو برگ از شجر اندر قدمش ریزیدم
\\
بس کنم آنچ بگفت او که بگو من گفتم
&&
و آنچ فرمود بپوشان و مگو پوشیدم
\\
شمس تبریز که آفاق از او شد پرنور
&&
من به هر سوی چو سایه ز پیش گردیدم
\\
\end{longtable}
\end{center}
