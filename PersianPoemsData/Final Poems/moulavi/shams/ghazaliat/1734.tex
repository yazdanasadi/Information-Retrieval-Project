\begin{center}
\section*{غزل شماره ۱۷۳۴: سماع چیست ز پنهانیان دل پیغام}
\label{sec:1734}
\addcontentsline{toc}{section}{\nameref{sec:1734}}
\begin{longtable}{l p{0.5cm} r}
سماع چیست ز پنهانیان دل پیغام
&&
دل غریب بیابد ز نامه شان آرام
\\
شکفته گردد از این باد شاخه‌های خرد
&&
گشاده گردد از این زخمه در وجود مسام
\\
سحر رسد ز ندای خروس روحانی
&&
ظفر رسد ز صدای نقاره بهرام
\\
عصیر جان به خم جسم تیر می انداخت
&&
چو دف شنید برآرد کفی نشان قوام
\\
حلاوتی عجبی در بدن پدید آید
&&
که از نی و لب مطرب شکر رسید به کام
\\
هزار کزدم غم را کنون ببین کشته
&&
هزار دور فرح بین میان ما بی‌جام
\\
فسون رقیه کزدم نویس عید رسید
&&
که هست رقیه کزدم به کوی عشق مدام
\\
ز هر طرف بجهد بی‌قرار یعقوبی
&&
که بوی پیرهن یوسفی بیافت مشام
\\
چو جان ما ز نفخت است فیه من روحی
&&
روا بود که نفختش بود شراب و طعام
\\
چو حشر جمله خلایق به نفخ خواهد بود
&&
ز ذوق زمزمه بجهند مردگان ز منام
\\
که خاک بر سر جان کسی که افسرده‌ست
&&
اثر نگیرد از آن نفخ و کم بود ز اعدام
\\
تن و دلی که بنوشید از این رحیق حلال
&&
بر آتش غم هجران حرام گشت حرام
\\
جمال صورت غیبی ز وصف بیرون است
&&
هزار دیده روشن به وام خواه به وام
\\
درون توست یکی مه کز آسمان خورشید
&&
ندا همی‌کندش کای منت غلام غلام
\\
ز جیب خویش بجو مه چو موسی عمران
&&
نگر به روزن خویش و بگو سلام سلام
\\
سماع گرم کن و خاطر خران کم جو
&&
که جان جان سماعی و رونق ایام
\\
زبان خود بفروشم هزار گوش خرم
&&
که رفت بر سر منبر خطیب شهدکلام
\\
\end{longtable}
\end{center}
