\begin{center}
\section*{غزل شماره ۳۹۵: عشق اندر فضل و علم و دفتر و اوراق نیست}
\label{sec:0395}
\addcontentsline{toc}{section}{\nameref{sec:0395}}
\begin{longtable}{l p{0.5cm} r}
عشق اندر فضل و علم و دفتر و اوراق نیست
&&
هر چه گفت و گوی خلق آن ره ره عشاق نیست
\\
شاخ عشق اندر ازل دان بیخ عشق اندر ابد
&&
این شجر را تکیه بر عرش و ثری و ساق نیست
\\
عقل را معزول کردیم و هوا را حد زدیم
&&
کاین جلالت لایق این عقل و این اخلاق نیست
\\
تا تو مشتاقی بدان کاین اشتیاق تو بتی است
&&
چون شدی معشوق از آن پس هستیی مشتاق نیست
\\
مرد بحری دایما بر تخته خوف و رجا است
&&
چونک تخته و مرد فانی شد جز استغراق نیست
\\
شمس تبریزی تویی دریا و هم گوهر تویی
&&
زانک بود تو سراسر جز سر خلاق نیست
\\
\end{longtable}
\end{center}
