\begin{center}
\section*{غزل شماره ۲۴۱۱: دلم چو دیده و تو چون خیال در دیده}
\label{sec:2411}
\addcontentsline{toc}{section}{\nameref{sec:2411}}
\begin{longtable}{l p{0.5cm} r}
دلم چو دیده و تو چون خیال در دیده
&&
زهی مبارک و زیبا به فال در دیده
\\
به بوی وصل دو دیده خراب و مست شده‌ست
&&
چگونه باشد یا رب وصال در دیده
\\
چو دیده بیشه آن شیرمست من باشد
&&
چه زهره دارد گرگ و شکال در دیده
\\
دو دیده را بگشا نور ذوالجلال ببین
&&
ز فر دولت آن خوش خصال در دیده
\\
چو چتر و سنجق آن رشک صد سلیمان دید
&&
گشاد هدهد جان پر و بال در دیده
\\
چو آفتاب جمالش بدیده‌ها درتافت
&&
چه شعله‌هاست ز نور جلال در دیده
\\
چو عقل عقل قنق شد درون خرگه جسم
&&
عقول هیچ ندارد مجال در دیده
\\
دو دیده مست شد از جان صدر شمس الدین
&&
چه باده‌هاست از او مال مال در دیده
\\
\end{longtable}
\end{center}
