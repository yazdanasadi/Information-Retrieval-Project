\begin{center}
\section*{غزل شماره ۲۵۹: پیش کش آن شاه شکرخانه را}
\label{sec:0259}
\addcontentsline{toc}{section}{\nameref{sec:0259}}
\begin{longtable}{l p{0.5cm} r}
پیش کش آن شاه شکرخانه را
&&
آن گهر روشن دردانه را
\\
آن شه فرخ رخ بی‌مثل را
&&
آن مه دریادل جانانه را
\\
روح دهد مرده پوسیده را
&&
مهر دهد سینه بیگانه را
\\
دامن هر خار پر از گل کند
&&
عقل دهد کله دیوانه را
\\
در خرد طفل دوروزه نهد
&&
آنچ نباشد دل فرزانه را
\\
طفل کی باشد تو مگر منکری
&&
عربده استن حنانه را
\\
مست شوی و شه مستان شوی
&&
چونک بگرداند پیمانه را
\\
بیخودم و مست و پراکنده مغز
&&
ور نه نکو گویم افسانه را
\\
با همه بشنو که بباید شنود
&&
قصه شیرین غریبانه را
\\
بشکند آن روی دل ماه را
&&
بشکند آن زلف دو صد شانه را
\\
قصه آن چشم کی یارد گزارد
&&
ساحر ساحرکش فتانه را
\\
بیند چشمش که چه خواهد شدن
&&
تا ابد او بیند پیشانه را
\\
راز مگو رو عجمی ساز خویش
&&
یاد کن آن خواجه علیانه را
\\
\end{longtable}
\end{center}
