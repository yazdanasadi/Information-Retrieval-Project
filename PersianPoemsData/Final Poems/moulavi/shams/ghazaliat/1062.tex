\begin{center}
\section*{غزل شماره ۱۰۶۲: چون نبینم من جمالت صد جهان خود دیده گیر}
\label{sec:1062}
\addcontentsline{toc}{section}{\nameref{sec:1062}}
\begin{longtable}{l p{0.5cm} r}
چون نبینم من جمالت صد جهان خود دیده گیر
&&
چون حدیث تو نباشد سر سر بشنیده گیر
\\
ای که در خوابت ندیده آدم و ذریتش
&&
از کی پرسم وصف حسنت از همه پرسیده گیر
\\
چون نباشم در وصالت ای ز بینایان نهان
&&
در بهشت و حور و دولت تا ابد باشیده گیر
\\
چون نبینم خشم و ناز شکرینت هر دمی
&&
بر سر شاهان معنی مر مرا نازیده گیر
\\
چونک ابر هجر تو ماه تو را پوشیده کرد
&&
صد هزاران در و گوهر بر سرم باریده گیر
\\
چونک مستان را نباشد شمع و شاهد روی تو
&&
صد هزاران خم باده هر طرف جوشیده گیر
\\
خضر بی‌من گر ببیند روی تو ای وای من
&&
ور نبیند آب حیوان هر دمش نوشیده گیر
\\
چون فنا خواهد شدن این ساحره دنیای دون
&&
تخت و بخت و گنج و عالم را به من بخشیده گیر
\\
در ازل جان‌های صدیقان نثار روی تو
&&
چونک رویت را نبینم خود نثاری چیده گیر
\\
این عزیز مصر جانم تا نبیند روی تو
&&
هر دو روزی یوسفی شکرلبی بخریده گیر
\\
ای خروشیده ز دردم سنگ و آهن دم به دم
&&
چون نجست از سنگ و آهن برق بخروشیده گیر
\\
یک شب این دیوانه را مهمان آن زنجیر کن
&&
ور بژولاند سر زلف تو را ژولیده گیر
\\
ور جهان در عشق تو بدگوی من شد باک نیست
&&
صد دروغ و افترا بر صادقی بافیده گیر
\\
با فراقت از دو عالم چون منم مظلومتر
&&
گر بنالد ظالم از مظلوم تو نالیده گیر
\\
چون نلافم شمس تبریز از سگان کوی تو
&&
بر سر شیران عالم مر مرا لافیده گیر
\\
\end{longtable}
\end{center}
