\begin{center}
\section*{غزل شماره ۱۰۷۳: خوی بد دارم ملولم تو مرا معذور دار}
\label{sec:1073}
\addcontentsline{toc}{section}{\nameref{sec:1073}}
\begin{longtable}{l p{0.5cm} r}
خوی بد دارم ملولم تو مرا معذور دار
&&
خوی من کی خوش شود بی‌روی خوبت ای نگار
\\
بی‌تو هستم چون زمستان خلق از من در عذاب
&&
با تو هستم چون گلستان خوی من خوی بهار
\\
بی‌تو بی‌عقلم ملولم هر چه گویم کژ بود
&&
من خجل از عقل و عقل از نور رویت شرمسار
\\
آب بد را چیست درمان باز در جیحون شدن
&&
خوی بد را چیست درمان بازدیدن روی یار
\\
آب جان محبوس می‌بینم در این گرداب تن
&&
خاک را بر می‌کنم تا ره کنم سوی بحار
\\
شربتی داری که پنهانی به نومیدان دهی
&&
تا فغان در ناورد از حسرتش اومیدوار
\\
چشم خود ای دل ز دلبر تا توانی برمگیر
&&
گر ز تو گیرد کناره ور تو را گیرد کنار
\\
\end{longtable}
\end{center}
