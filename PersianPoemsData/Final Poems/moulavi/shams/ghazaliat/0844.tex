\begin{center}
\section*{غزل شماره ۸۴۴: گر ساعتی ببری ز اندیشه‌ها چه باشد}
\label{sec:0844}
\addcontentsline{toc}{section}{\nameref{sec:0844}}
\begin{longtable}{l p{0.5cm} r}
گر ساعتی ببری ز اندیشه‌ها چه باشد
&&
غوطی خوری چو ماهی در بحر ما چه باشد
\\
ز اندیشه‌ها نخسپی ز اصحاب کهف باشی
&&
نوری شوی مقدس از جان و جا چه باشد
\\
آخر تو برگ کاهی ما کهربای دولت
&&
زین کاهدان بپری تا کهربا چه باشد
\\
صد بار عهد کردی کاین بار خاک باشم
&&
یک بار پاس داری آن عهد را چه باشد
\\
تو گوهری نهفته در کاه گل گرفته
&&
گر رخ ز گل بشویی ای خوش لقا چه باشد
\\
از پشت پادشاهی مسجود جبرئیلی
&&
ملک پدر بجویی ای بی‌نوا چه باشد
\\
ای اولیای حق را از حق جدا شمرده
&&
گر ظن نیک داری بر اولیا چه باشد
\\
جزوی ز کل بمانده دستی ز تن بریده
&&
گر زین سپس نباشی از ما جدا چه باشد
\\
بی سر شوی و سامان از کبر و حرص خالی
&&
آنگه سری برآری از کبریا چه باشد
\\
از ذکر نوش شربت تا وارهی ز فکرت
&&
در جنگ اگر نپیچی ای مرتضا چه باشد
\\
بس کن که تو چو کوهی در کوه کان زر جو
&&
که را اگر نیاری اندر صدا چه باشد
\\
\end{longtable}
\end{center}
