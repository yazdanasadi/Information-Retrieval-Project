\begin{center}
\section*{غزل شماره ۲۲۹۴: ز بردابرد عشق او چو بشنید این دل پاره}
\label{sec:2294}
\addcontentsline{toc}{section}{\nameref{sec:2294}}
\begin{longtable}{l p{0.5cm} r}
ز بردابرد عشق او چو بشنید این دل پاره
&&
برآمد از وجود خویش و هر دو کون یک باره
\\
به بحر نیستی درشد همه هستی محقر شد
&&
به ناگه شعله‌ای برشد شگرف از جان خون خواره
\\
کجا اسراربین آمد دمی کز کبر و کین آمد
&&
حیاتی کز زمین آمد بود در بحر بیچاره
\\
الا ای جان انسانی چو از اقلیم نقصانی
&&
به شب هنگام ظلمانی چو اختر باش سیاره
\\
چو از مردان مدد یابی یکی عیش ابد یابی
&&
سپاه بی‌عدد یابی به قهر نفس اماره
\\
چو هستی را همی‌روبی سر هر نفس می‌کوبی
&&
بدید آید یکی خوبی نه رو باشد نه رخساره
\\
چه باشد صد قمر آن جا شود هر خاک زر آن جا
&&
به غیر دل مبر آن جا که آن جا هست دل پاره
\\
زهی دربخش دریایی برای جان بینایی
&&
شمار ریگ هر جایی ز عشقش هست آواره
\\
خوشا مشکا که می‌بیزی به راه شمس تبریزی
&&
زهی باده که می‌ریزی برای جان میخواره
\\
\end{longtable}
\end{center}
