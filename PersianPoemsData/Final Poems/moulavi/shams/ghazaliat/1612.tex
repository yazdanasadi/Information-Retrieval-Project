\begin{center}
\section*{غزل شماره ۱۶۱۲: منم آن کس که نبینم بزنم فاخته گیرم}
\label{sec:1612}
\addcontentsline{toc}{section}{\nameref{sec:1612}}
\begin{longtable}{l p{0.5cm} r}
منم آن کس که نبینم بزنم فاخته گیرم
&&
من از آن خارکشانم که شود خار حریرم
\\
به کی مانم به کی مانم که سطرلاب جهانم
&&
همه اشکال فلک را به یکایک بپذیرم
\\
ز پس کوه معانی علم عشق برآمد
&&
چو علمدار برآمد برهاند ز زحیرم
\\
ز سحر گر بگریزم تو یقین دان که خفاشم
&&
ز ضرر گر بگریزم تو یقین دان که ضریرم
\\
چو ز بادی بگریزم چو خسم سخره بادم
&&
چو دهانم نپذیرد به خدا خام و خمیرم
\\
نه چو خورشید جهانم شه یک روزه فانی
&&
که نیندیشد و گوید که چه میرم که بمیرم
\\
نه چو گردون نه چو چرخم نه چو مرغم نه چو فرخم
&&
نه چو مریخ سلح کش نه چو مه نیمه و زیرم
\\
چو منی خوار نباشد که تویی حافظ و یارم
&&
بر خلق ابن قلیلم بر تو ابن کثیرم
\\
هنر خویش بپوشم ز همه تا نخرندم
&&
بدو صد عیب بلنگم که خرد جز تو امیرم
\\
نخورم جز جگر و دل که جگرگوشه شیرم
&&
نه چون یوزان خسیسم که بود طعمه پنیرم
\\
ز شرر زان نگریزم که زرم نی زر قلبم
&&
ز خطر زان نگریزم که در این ملک خطیرم
\\
همگان مردنیانند نمایند و نپایند
&&
تو بیا کآب حیاتی که ز تو نیست گزیرم
\\
تو مرا جان بقایی که دهی جام حیاتم
&&
تو مرا گنج عطایی که نهی نام فقیرم
\\
هله بس کن هله بس کن کم آواز جرس کن
&&
که کهم من نه صدایم قلمم من نه صریرم
\\
فعلاتن فعلاتن فعلاتن فعلاتن
&&
همه می گوی و مزن دم ز شهنشاه شهیرم
\\
\end{longtable}
\end{center}
