\begin{center}
\section*{غزل شماره ۲۷۲۰: تو تا بنشسته‌ای بر دار فانی}
\label{sec:2720}
\addcontentsline{toc}{section}{\nameref{sec:2720}}
\begin{longtable}{l p{0.5cm} r}
تو تا بنشسته‌ای بر دار فانی
&&
نشسته می‌روی و می نبینی
\\
نشسته می‌روی این نیز نیکو است
&&
اگر رویت در این گفتن سوی او است
\\
بسی گشتی در این گرداب گردان
&&
به سوی جوی رحمت رو بگردان
\\
بزن پایی بر این پابند عالم
&&
که تا دست از تبرک بر تو مالم
\\
تو را زلفی است به از مشک و عنبر
&&
تو ده کل را کلاهی ای برادر
\\
کله کم جو چو داری جعد فاخر
&&
کله بر آسمان انداز آخر
\\
چرا دنیا به نکته مستحیله
&&
فریبد چون تو زیرک را به حیله
\\
به سردی نکته گوید سرد سیلی
&&
نداری پای آن خر را شکالی
\\
اگر دوران دلیل آرد در آن قال
&&
تخلف دیده‌ای در روی او مال
\\
تو را عمری کشید این غول در تیه
&&
بکن با غول خود بحثی به توجیه
\\
چرا الزام اویی چیست سکته
&&
جوابش گو که مقلوب است نکته
\\
\end{longtable}
\end{center}
