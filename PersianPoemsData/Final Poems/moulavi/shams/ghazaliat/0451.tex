\begin{center}
\section*{غزل شماره ۴۵۱: پنهان مشو که روی تو بر ما مبارکست}
\label{sec:0451}
\addcontentsline{toc}{section}{\nameref{sec:0451}}
\begin{longtable}{l p{0.5cm} r}
پنهان مشو که روی تو بر ما مبارکست
&&
نظاره تو بر همه جان‌ها مبارکست
\\
یک لحظه سایه از سر ما دورتر مکن
&&
دانسته‌ای که سایه عنقا مبارکست
\\
ای نوبهار حسن بیا کان هوای خوش
&&
بر باغ و راغ و گلشن و صحرا مبارکست
\\
ای صد هزار جان مقدس فدای او
&&
کآید به کوی عشق که آن جا مبارکست
\\
سودایییم از تو و بطال و کو به کو
&&
ما را چنین بطالت و سودا مبارکست
\\
ای بستگان تن به تماشای جان روید
&&
کآخر رسول گفت تماشا مبارکست
\\
هر برگ و هر درخت رسولیست از عدم
&&
یعنی که کشت‌های مصفا مبارکست
\\
چون برگ و چون درخت بگفتند بی‌زبان
&&
بی گوش بشنوید که این‌ها مبارکست
\\
ای جان چار عنصر عالم جمال تو
&&
بر آب و باد و آتش و غبرا مبارکست
\\
یعنی که هر چه کاری آن گم نمی‌شود
&&
کس تخم دین نکارد الا مبارکست
\\
سجده برم که خاک تو بر سر چو افسرست
&&
پا درنهم که راه تو بر پا مبارکست
\\
می‌آیدم به چشم همین لحظه نقش تو
&&
والله خجسته آمد و حقا مبارکست
\\
نقشی که رنگ بست از این خاک بی‌وفاست
&&
نقشی که رنگ بست ز بالا مبارکست
\\
بر خاکیان جمال بهاران خجسته‌ست
&&
بر ماهیان طپیدن دریا مبارکست
\\
آن آفتاب کز دل در سینه‌ها بتافت
&&
بر عرش و فرش و گنبد خضرا مبارکست
\\
دل را مجال نیست که از ذوق دم زند
&&
جان سجده می‌کند که خدایا مبارکست
\\
هر دل که با هوای تو امشب شود حریف
&&
او را یقین بدان تو که فردا مبارکست
\\
بفزا شراب خامش و ما را خموش کن
&&
کاندر درون نهفتن اشیاء مبارکست
\\
\end{longtable}
\end{center}
