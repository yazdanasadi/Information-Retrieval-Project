\begin{center}
\section*{غزل شماره ۵۸۲: اگر خواب آیدم امشب سزای ریش خود بیند}
\label{sec:0582}
\addcontentsline{toc}{section}{\nameref{sec:0582}}
\begin{longtable}{l p{0.5cm} r}
اگر خواب آیدم امشب سزای ریش خود بیند
&&
به جای مفرش و بالی همه مشت و لگد بیند
\\
ازیرا خواب کژ بیند که آیینه خیالست او
&&
که معلوم‌ست تعبیرش اگر او نیک و بد بیند
\\
خصوصا اندر این مجلس که امشب در نمی‌گنجد
&&
دو چشم عقل پایان بین که صدساله رصد بیند
\\
شب قدرست وصل او شب قبرست هجر او
&&
شب قبر از شب قدرش کرامات و مدد بیند
\\
خنک جانی که بر بامش همی چوبک زند امشب
&&
شود همچون سحر خندان عطای بی‌عدد بیند
\\
برو ای خواب خاری زن تو اندر چشم نامحرم
&&
که حیفست آن که بیگانه در این شب قد و خد بیند
\\
شرابش ده بخوابانش برون بر از گلستانش
&&
که تا در گردن او فردا ز غم حبل مسد بیند
\\
ببردی روز در گفتن چو آمد شب خمش باری
&&
که هرک از گفت خامش شد عوض گفت ابد بیند
\\
\end{longtable}
\end{center}
