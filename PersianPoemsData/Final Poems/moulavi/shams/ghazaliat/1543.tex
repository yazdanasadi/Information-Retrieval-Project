\begin{center}
\section*{غزل شماره ۱۵۴۳: کجایی ساقیا درده مدامم}
\label{sec:1543}
\addcontentsline{toc}{section}{\nameref{sec:1543}}
\begin{longtable}{l p{0.5cm} r}
کجایی ساقیا درده مدامم
&&
که من از جان غلامت را غلامم
\\
می اندرده تهی دستم چه داری
&&
که از خون جگر پر گشت جامم
\\
ز ننگ من نگوید نام من کس
&&
چو من مردی چه جای ننگ و نامم
\\
چو بر جانم زدی شمشیر عشقت
&&
تمامم کن که زنده ناتمامم
\\
گهم زاهد همی‌خوانند و گه رند
&&
من مسکین ندانم تا کدامم
\\
ز من چون شمع تا یک ذره باقی است
&&
نخواهد بود جز آتش مقامم
\\
مرا جز سوختن راه دگر نیست
&&
بیا تا خوش بسوزم زانک خامم
\\
\end{longtable}
\end{center}
