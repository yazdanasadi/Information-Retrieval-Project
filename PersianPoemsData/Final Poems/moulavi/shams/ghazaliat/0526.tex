\begin{center}
\section*{غزل شماره ۵۲۶: ای لولیان ای لولیان یک لولیی دیوانه شد}
\label{sec:0526}
\addcontentsline{toc}{section}{\nameref{sec:0526}}
\begin{longtable}{l p{0.5cm} r}
ای لولیان ای لولیان یک لولیی دیوانه شد
&&
طشتش فتاد از بام ما نک سوی مجنون خانه شد
\\
می‌گشت گرد حوض او چون تشنگان در جست و جو
&&
چون خشک نانه ناگهان در حوض ما ترنانه شد
\\
ای مرد دانشمند تو دو گوش از این بربند تو
&&
مشنو تو این افسون که او ز افسون ما افسانه شد
\\
زین حلقه نجهد گوش‌ها کو عقل برد از هوش‌ها
&&
تا سر نهد بر آسیا چون دانه در پیمانه شد
\\
بازی مبین بازی مبین این جا تو جانبازی گزین
&&
سرها ز عشق جعد او بس سرنگون چون شانه شد
\\
غره مشو با عقل خود بس اوستاد معتمد
&&
کاستون عالم بود او نالانتر از حنانه شد
\\
من که ز جان ببریده‌ام چون گل قبا بدریده‌ام
&&
زان رو شدم که عقل من با جان من بیگانه شد
\\
این قطره‌های هوش‌ها مغلوب بحر هوش شد
&&
ذرات این جان ریزه‌ها مستهلک جانانه شد
\\
خامش کنم فرمان کنم وین شمع را پنهان کنم
&&
شمعی که اندر نور او خورشید و مه پروانه شد
\\
\end{longtable}
\end{center}
