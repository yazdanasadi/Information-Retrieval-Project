\begin{center}
\section*{غزل شماره ۲۴۴۸: ای داده جان را لطف تو خوشتر ز مستی حالتی}
\label{sec:2448}
\addcontentsline{toc}{section}{\nameref{sec:2448}}
\begin{longtable}{l p{0.5cm} r}
ای داده جان را لطف تو خوشتر ز مستی حالتی
&&
خوشتر ز مستی ابد بی‌باده و بی‌آلتی
\\
یک ساعتی تشریف ده جان را چنان تلطیف ده
&&
آن ساعتی پاک از کی و تا کی عجایب ساعتی
\\
شاهنشه یغماییی کز دولت یغمای تو
&&
یاغی به شادی منتظر تا کی کنی تو غارتی
\\
جان چون نداند نقش خود یا عالم جان بخش خود
&&
پا می نداند کفش خود کان لایق است و بابتی
\\
پا را ز کفش دیگری هر لحظه تنگی و شری
&&
وز کفش خود شد خوشتری پا را در آن جا راحتی
\\
جان نیز داند جفت خود وز غیب داند نیک و بد
&&
کز غیب هر جان را بود درخورد هر جان ساحتی
\\
جانی که او را هست آن محبوس از آن شد در جهان
&&
چون نیست او را این زمان از بهر آن دم طاقتی
\\
چون شاه زاده طفل بد پس مخزنش بر قفل بد
&&
خلعت نهاده بهر او تا برکشد او قامتی
\\
تو قفل دل را باز کن قصد خزینه راز کن
&&
در مشکلات دو جهان نبود سؤالت حاجتی
\\
خمخانه مردان دل است وز وی چه مستی حاصل است
&&
طفلی و پایت در گل است پس صبر کن تا غایتی
\\
تا غایتی کز گوشه‌ای دولت برآرد جوشه‌ای
&&
از دور گردی خاسته تابان شده یک رایتی
\\
بنوشته بر رایت که این نقش خداوند شمس دین
&&
از مفخر تبریز و چین اندر بصیرت آیتی
\\
\end{longtable}
\end{center}
