\begin{center}
\section*{غزل شماره ۲۰۴۴: جانا بیار باده و بختم بلند کن}
\label{sec:2044}
\addcontentsline{toc}{section}{\nameref{sec:2044}}
\begin{longtable}{l p{0.5cm} r}
جانا بیار باده و بختم بلند کن
&&
زان حلقه‌های زلف دلم را کمند کن
\\
مجلس خوش است و ما و حریفان همه خوشیم
&&
آتش بیار و چاره مشتی سپند کن
\\
زان جام بی‌دریغ در اندیشه‌ها بریز
&&
در بیخودی سزای دل خودپسند کن
\\
ای غم برو برو بر مستانت کار نیست
&&
آن را که هوشیار بیابی گزند کن
\\
مستان مسلمند ز اندیشه‌ها و غم
&&
آن کو نشد مسلم او را نژند کن
\\
ای جان مست مجلس ابرار یشربون
&&
بر گربه اسیر هوا ریش خند کن
\\
ریش همه به دست اجل بین و رحم کن
&&
از مرگ وارهان همه را سودمند کن
\\
عزم سفر کن ای مه و بر گاو نه تو رخت
&&
با شیرگیر مست مگو ترک پند کن
\\
در چشم ما نگر اثر بیخودی ببین
&&
ما را سوار اشقر و پشت سمند کن
\\
یک رگ اگر در این تن ما هوشیار هست
&&
با او حساب دفتر هفتاد و اند کن
\\
ای طبع روسیاه سوی هند بازرو
&&
وی عشق ترک تاز سفر سوی جند کن
\\
آن جا که مست گشتی بنشین مقیم شو
&&
و آن جا که باده خوردی آن جا فکند کن
\\
در مطبخ خدا اگرت قوت روح نیست
&&
آن گاه سر در آخر این گوسفند کن
\\
خواهی که شاهدان فلک جلوه گر شوند
&&
دل را حریف صیقل آیینه رند کن
\\
ای دل خموش کن همه بی‌حرف گو سخن
&&
بی‌لب حدیث عالم بی‌چون و چند کن
\\
\end{longtable}
\end{center}
