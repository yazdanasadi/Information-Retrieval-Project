\begin{center}
\section*{غزل شماره ۱۱۴۲: درخت اگر متحرک بدی به پا و به پر}
\label{sec:1142}
\addcontentsline{toc}{section}{\nameref{sec:1142}}
\begin{longtable}{l p{0.5cm} r}
درخت اگر متحرک بدی به پا و به پر
&&
نه رنج اره کشیدی نه زخمه‌های تبر
\\
ور آفتاب نرفتی به پر و پا همه شب
&&
جهان چگونه منور شدی بگاه سحر
\\
ور آب تلخ نرفتی ز بحر سوی افق
&&
کجا حیات گلستان شدی به سیل و مطر
\\
چو قطره از وطن خویش رفت و بازآمد
&&
مصادف صدف او گشت و شد یکی گوهر
\\
نه یوسفی به سفر رفت از پدر گریان
&&
نه در سفر به سعادت رسید و ملک و ظفر
\\
نه مصطفی به سفر رفت جانب یثرب
&&
بیافت سلطنت و گشت شاه صد کشور
\\
وگر تو پای نداری سفر گزین در خویش
&&
چو کان لعل پذیرا شو از شعاع اثر
\\
ز خویشتن سفری کن به خویش ای خواجه
&&
که از چنین سفری گشت خاک معدن زر
\\
ز تلخی و ترشی رو به سوی شیرینی
&&
چنانک رست ز تلخی هزار گونه ثمر
\\
ز شمس مفخر تبریز جوی شیرینی
&&
از آنک هر ثمر از نور شمس یابد فر
\\
\end{longtable}
\end{center}
