\begin{center}
\section*{غزل شماره ۱۲۲۴: پریشان باد پیوسته دل از زلف پریشانش}
\label{sec:1224}
\addcontentsline{toc}{section}{\nameref{sec:1224}}
\begin{longtable}{l p{0.5cm} r}
پریشان باد پیوسته دل از زلف پریشانش
&&
وگر برناورم فردا سر خویش از گریبانش
\\
الا ای شحنه خوبی ز لعل تو بسی گوهر
&&
بدزدیدست جان من برنجانش برنجانش
\\
گر ایمان آورد جانی به غیر کافر زلفت
&&
بزن از آتش شوقت تو اندر کفر و ایمانش
\\
پریشان باد زلف او که تا پنهان شود رویش
&&
که تا تنها مرا باشد پریشانی ز پنهانش
\\
منم در عشق بی‌برگی که اندر باغ عشق او
&&
چو گل پاره کنم جامه ز سودای گلستانش
\\
در آن گل‌های رخسارش همی‌غلطید روزی دل
&&
بگفتم چیست این گفتا همی‌غلطم در احسانش
\\
یکی خطی نویسم من ز حال خود بر آن عارض
&&
که تا برخواند آن عارض که استادست خط خوانش
\\
ولیکن سخت می‌ترسم از آن زلف سیه کاوش
&&
که بس دل در رسن بستست آن هندو ز بهتانش
\\
به چاه آن ذقن بنگر مترس ای دل ز افتادن
&&
که هر دل کان رسن بیند چنان چاهست زندانش
\\
\end{longtable}
\end{center}
