\begin{center}
\section*{غزل شماره ۱۶۲۴: خبری اگر شنیدی ز جمال و حسن یارم}
\label{sec:1624}
\addcontentsline{toc}{section}{\nameref{sec:1624}}
\begin{longtable}{l p{0.5cm} r}
خبری اگر شنیدی ز جمال و حسن یارم
&&
سر مست گفته باشد من از این خبر ندارم
\\
شب و روز می بکوشم که برهنه را بپوشم
&&
نه چنان دکان فروشم که دکان نو برآرم
\\
علمی به دست مستی دو هزار مست با وی
&&
به میان شهر گردان که خمار شهریارم
\\
به چه میخ بندم آن را که فقاع از او گشاید
&&
چه شکار گیرم آن جا که شکار آن شکارم
\\
دهلی بدین عظیمی به گلیم درنگنجد
&&
فر و نور مه بگوید که من اندر این غبارم
\\
به سر مناره اشتر رود و فغان برآرد
&&
که نهان شدم من این جا مکنید آشکارم
\\
شتر است مرد عاشق سر آن مناره عشق است
&&
که مناره‌هاست فانی و ابدی است این منارم
\\
تو پیازهای گل را به تک زمین نهان کن
&&
به بهار سر برآرد که من آن قمرعذارم
\\
سر خنب چون گشادی برسان وظیفه‌ها را
&&
به میان دور ما آ که غلام این دوارم
\\
پی جیب توست این جا همه جیب‌ها دریده
&&
پی سیب توست ای جان که چو برگ بی‌قرارم
\\
همه را به لطف جان کن همه را ز سر جوان کن
&&
به شراب اختیاری که رباید اختیارم
\\
همه پرده‌ها بدران دل بسته را بپران
&&
هله ای تو اصل اصلم به تو است هم مطارم
\\
به خدا که روز نیکو ز بگه بدید باشد
&&
که درآید آفتابش به وصال در کنارم
\\
تو خموش تا قرنفل بکند حکایت گل
&&
بر شاهدان گلشن چو رسید نوبهارم
\\
\end{longtable}
\end{center}
