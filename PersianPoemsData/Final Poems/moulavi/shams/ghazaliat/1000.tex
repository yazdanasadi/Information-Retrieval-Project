\begin{center}
\section*{غزل شماره ۱۰۰۰: آنچ گل سرخ قبا می‌کند}
\label{sec:1000}
\addcontentsline{toc}{section}{\nameref{sec:1000}}
\begin{longtable}{l p{0.5cm} r}
آنچ گل سرخ قبا می‌کند
&&
دانم من کان ز کجا می‌کند
\\
بید پیاده که کشیدست صف
&&
آنچ گذشتست قضا می‌کند
\\
سوسن با تیغ و سمن با سپر
&&
هر یک تکبیر غزا می‌کند
\\
بلبل مسکین که چه‌ها می‌کشد
&&
آه از آن گل که چه‌ها می‌کند
\\
گوید هر یک ز عروسان باغ
&&
کان گل اشارت سوی ما می‌کند
\\
گوید بلبل که گل آن شیوه‌ها
&&
بهر من بی‌سر و پا می‌کند
\\
دست برآورده به زاری چنار
&&
با تو بگویم چه دعا می‌کند
\\
بر سر غنچه کی کله می‌نهد
&&
پشت بنفشه کی دوتا می‌کند
\\
گر چه خزان کرد جفاها بسی
&&
بین که بهاران چه وفا می‌کند
\\
فصل خزان آنچ به تاراج برد
&&
فصل بهار آمد ادا می‌کند
\\
ذکر گل و بلبل و خوبان باغ
&&
جمله بهانه‌ست چرا می‌کند
\\
غیرت عشق است وگر نه زبان
&&
شرح عنایات خدا می‌کند
\\
مفخر تبریز و جهان شمس دین
&&
باز مراعات شما می‌کند
\\
\end{longtable}
\end{center}
