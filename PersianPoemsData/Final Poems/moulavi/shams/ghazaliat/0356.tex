\begin{center}
\section*{غزل شماره ۳۵۶: تو را در دلبری دستی تمامست}
\label{sec:0356}
\addcontentsline{toc}{section}{\nameref{sec:0356}}
\begin{longtable}{l p{0.5cm} r}
تو را در دلبری دستی تمامست
&&
مرا در بی‌دلی درد و سقامست
\\
بجز با روی خوبت عشقبازی
&&
حرامست و حرامست و حرامست
\\
همه فانی و خوان وحدت تو
&&
مدامست و مدامست و مدامست
\\
چو چشم خود بمالم خود جز تو
&&
کدامست و کدامست و کدامست
\\
جهان بر روی تو از بهر روپوش
&&
لثامست و لثامست و لثامست
\\
به هر دم از زبان عشق بر ما
&&
سلامست و سلامست و سلامست
\\
ز هر ذره به گفت بی‌زبانی
&&
پیامست و پیامست و پیامست
\\
غم و شادی ما در پیش تختت
&&
غلامست و غلامست و غلامست
\\
اگر چه اشتر غم هست گرگین
&&
امامست و امامست و امامست
\\
پس آن اشتر شادی پرشیر
&&
ختامست و ختامست و ختامست
\\
تو را در بینی این هر دو اشتر
&&
زمامست و زمامست و زمامست
\\
نه آن شیری که آخر طفل جان را
&&
فطامست و فطامست و فطامست
\\
از آن شیری که جوی خلد از وی
&&
نظامست و نظامست و نظامست
\\
خمش کردم که غیرت بر دهانم
&&
لگامست و لگامست و لگامست
\\
\end{longtable}
\end{center}
