\begin{center}
\section*{غزل شماره ۳۷۱: گر جام سپهر زهرپیماست}
\label{sec:0371}
\addcontentsline{toc}{section}{\nameref{sec:0371}}
\begin{longtable}{l p{0.5cm} r}
گر جام سپهر زهرپیماست
&&
آن در لب عاشقان چو حلواست
\\
زین واقعه گر ز جای رفتی
&&
از جای برو که جای این جاست
\\
مگریز ز سوز عشق زیرا
&&
جز آتش عشق دود و سوداست
\\
دودت نپزد کند سیاهت
&&
در پختنت آتشست کاستاست
\\
پروانه که گرد دود گردد
&&
دودآلودست و خام و رسواست
\\
از خانه و مان به یاد ناید
&&
آن را که چنین سفر مهیاست
\\
از شهر مگو که در بیابان
&&
موسیست رفیق من و سلواست
\\
صحبت چه کنی که در سقیمی
&&
هر لحظه طبیب تو مسیحاست
\\
دلتنگ خوشم که در فراخی
&&
هر مسخره را رهست و گنجاست
\\
چون خانه دل ز غم شود تنگ
&&
در وی شه دلنواز تنهاست
\\
دل تنگ بود جز او نگنجد
&&
تنگی دلم امان و غوغاست
\\
دندان عدو ز ترس کندست
&&
پس روترشی رهایی ماست
\\
خاموش که بحر اگر ترش روست
&&
هم معدن گوهرست و دریاست
\\
\end{longtable}
\end{center}
