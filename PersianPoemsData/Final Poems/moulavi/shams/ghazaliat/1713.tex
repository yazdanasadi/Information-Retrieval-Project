\begin{center}
\section*{غزل شماره ۱۷۱۳: خیزید عاشقان که سوی آسمان رویم}
\label{sec:1713}
\addcontentsline{toc}{section}{\nameref{sec:1713}}
\begin{longtable}{l p{0.5cm} r}
خیزید عاشقان که سوی آسمان رویم
&&
دیدیم این جهان را تا آن جهان رویم
\\
نی نی که این دو باغ اگر چه خوش است و خوب
&&
زین هر دو بگذریم و بدان باغبان رویم
\\
سجده کنان رویم سوی بحر همچو سیل
&&
بر روی بحر زان پس ما کف زنان رویم
\\
زین کوی تعزیت به عروسی سفر کنیم
&&
زین روی زعفران به رخ ارغوان رویم
\\
از بیم اوفتادن لرزان چو برگ و شاخ
&&
دل‌ها همی‌طپند به دارالامان رویم
\\
از درد چاره نیست چو اندر غریبییم
&&
وز گرد چاره نیست چو در خاکدان رویم
\\
چون طوطیان سبز به پر و به بال نغز
&&
شکرستان شویم و به شکرستان رویم
\\
این نقش‌ها نشانه نقاش بی‌نشان
&&
پنهان ز چشم بد هله تا بی‌نشان رویم
\\
راهی پر از بلاست ولی عشق پیشواست
&&
تعلیممان دهد که در او بر چه سان رویم
\\
هر چند سایه کرم شاه حافظ است
&&
در ره همان به‌ست که با کاروان رویم
\\
ماییم همچو باران بر بام پرشکاف
&&
بجهیم از شکاف و بدان ناودان رویم
\\
همچون کمان کژیم که زه در گلوی ماست
&&
چون راست آمدیم چو تیر از کمان رویم
\\
در خانه مانده‌ایم چو موشان ز گربگان
&&
گر شیرزاده‌ایم بدان ارسلان رویم
\\
جان آینه کنیم به سودای یوسفی
&&
پیش جمال یوسف با ارمغان رویم
\\
خامش کنیم تا که سخن بخش گوید این
&&
او آن چنانک گوید ما آن چنان رویم
\\
\end{longtable}
\end{center}
