\begin{center}
\section*{غزل شماره ۹۱۲: نگفتمت مرو آن جا که مبتلات کنند}
\label{sec:0912}
\addcontentsline{toc}{section}{\nameref{sec:0912}}
\begin{longtable}{l p{0.5cm} r}
نگفتمت مرو آن جا که مبتلات کنند
&&
که سخت دست درازند بسته پات کنند
\\
نگفتمت که بدان سوی دام در دامست
&&
چو درفتادی در دام کی رهات کنند
\\
نگفتمت به خرابات طرفه مستانند
&&
که عقل را هدف تیر ترهات کنند
\\
چو تو سلیم دلی را چو لقمه بربایند
&&
به هر پیاده شهی را به طرح مات کنند
\\
بسی مثال خمیرت دراز و گرد کنند
&&
کهت کنند و دو صد بار کهربات کنند
\\
تو مرد دل تنکی پیش آن جگرخواران
&&
اگر روی چو جگربند شوربات کنند
\\
تو اعتماد مکن بر کمال و دانش خویش
&&
که کوه قاف شوی زود در هوات کنند
\\
هزار مرغ عجب از گل تو برسازند
&&
چو ز آب و گل گذری تا دگر چه‌هات کنند
\\
برون کشندت از این تن چنان که پنبه ز پوست
&&
مثال شخص خیالیت بی‌جهات کنند
\\
چو در کشاکش احکام راضیت یابند
&&
ز رنج‌ها برهانند و مرتضات کنند
\\
خموش باش که این کودنان پست سخن
&&
حشیشی‌اند و همین لحظه ژاژخات کنند
\\
\end{longtable}
\end{center}
