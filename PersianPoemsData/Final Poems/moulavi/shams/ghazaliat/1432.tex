\begin{center}
\section*{غزل شماره ۱۴۳۲: تو خود دانی که من بی‌تو عدم باشم عدم باشم}
\label{sec:1432}
\addcontentsline{toc}{section}{\nameref{sec:1432}}
\begin{longtable}{l p{0.5cm} r}
تو خود دانی که من بی‌تو عدم باشم عدم باشم
&&
عدم خود قابل هست است از آن هم نیز کم باشم
\\
چو زان یوسف جدا مانم یقین در بیت احزانم
&&
حریف ظن بد باشم ندیم هر ندم باشم
\\
چو شحنه شهر شه باشم عسس گردم چو مه باشم
&&
شکنجه دزد غم باشم سقام هر سقم باشم
\\
ببندم گردن غم را چو اشتر می کشم هر جا
&&
بجز خارش ننوشانم چو در باغ ارم باشم
\\
قضایش گر قصاص آرد مرا اشتر کند روزی
&&
جمازه حج او گردم حمول آن حرم باشم
\\
منم محکوم امر مر گه اشتربان و گه اشتر
&&
گهی لت خواره چون طبلم گهی شقه علم باشم
\\
اگر طبال اگر طبلم به لشکرگاه آن فضلم
&&
از این تلوین چه غم دارم چو سلطان را حشم باشم
\\
بگیرم خرس فکرت را ره رقصش بیاموزم
&&
به هنگامه بتان آرم ز رقصش مغتنم باشم
\\
چو شمعی ام که بی‌گفتن نمایم نقش هر چیزی
&&
مکن اندیشه کژمژ که غماز رقم باشم
\\
یقول العشق یا صاحی تساکر و اغتنم راحی
&&
فاشبعناک یا طاوی و داویناک یا اخشم
\\
شکرنا نعمه المولی و مولانا به اولی
&&
فهذا العیش لا یفنی و هذا الکاس لا یهشم
\\
افندی کالی میراسوذ لزمونو تا کالاسو
&&
اذی نازس کنا خارس که تا من محتشم باشم
\\
یزک ای یار روحانی ورر عیسی بکی جانی
&&
سنک اول ایلکل قانی اگر من متهم باشم
\\
خمش باشم ترش باشم به قاصد تا بگوید او
&&
خمش چونی ترش چونی تو را چون من صنم باشم
\\
\end{longtable}
\end{center}
