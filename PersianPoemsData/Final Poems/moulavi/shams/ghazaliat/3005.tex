\begin{center}
\section*{غزل شماره ۳۰۰۵: آن دل که گم شده‌ست هم از جان خویش جوی}
\label{sec:3005}
\addcontentsline{toc}{section}{\nameref{sec:3005}}
\begin{longtable}{l p{0.5cm} r}
آن دل که گم شده‌ست هم از جان خویش جوی
&&
آرام جان خویش ز جانان خویش جوی
\\
اندر شکر نیابی ذوق نبات غیب
&&
آن ذوق را هم از لب و دندان خویش جوی
\\
دو چشم را تو ناظر هر بی‌نظر مکن
&&
در ناظری گریز و ازو آن خویش جوی
\\
نقلست از رسول که مردم معادنند
&&
پس نقد خویش را برو از کان خویش جوی
\\
از تخت تن برون رو و بر تخت جان نشین
&&
از آسمان گذر کن و کیوان خویش جوی
\\
برقی که بر دلت زد و دل بی‌قرار شد
&&
آن برق را در اشک چو باران خویش جوی
\\
انبان بوهریره وجود توست و بس
&&
هر چه مراد توست در انبان خویش جوی
\\
ای بی‌نشان محض نشان از کی جویمت
&&
هم تو بجو مرا و به احسان خویش جوی
\\
\end{longtable}
\end{center}
