\begin{center}
\section*{غزل شماره ۸۹۰: صبحدمی همچو صبح پرده ظلمت درید}
\label{sec:0890}
\addcontentsline{toc}{section}{\nameref{sec:0890}}
\begin{longtable}{l p{0.5cm} r}
صبحدمی همچو صبح پرده ظلمت درید
&&
نیم شبی ناگهان صبح قیامت دمید
\\
واسطه‌ها را برید دید به خود خویش را
&&
آنچ زبانی نگفت بی‌سر و گوشی شنید
\\
پوست بدرد ز ذوق عشق چو پیدا شود
&&
لیک کجا ذوق آن کو کندت ناپدید
\\
فقر ببرده سبق رفته طبق بر طبق
&&
باز کند قفل را فقر مبارک کلید
\\
کشته شهوت پلید کشته عقلست پاک
&&
فقر زده خیمه‌ای زان سوی پاک و پلید
\\
جمله دل عاشقان حلقه زده گرد فقر
&&
فقر چو شیخ الشیوخ جمله دل‌ها مرید
\\
چونک به تبریز چشم شمس حقم را بدید
&&
گفت حقش پر شدی گفت که هل من مزید
\\
\end{longtable}
\end{center}
