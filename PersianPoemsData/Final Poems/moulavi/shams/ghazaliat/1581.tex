\begin{center}
\section*{غزل شماره ۱۵۸۱: گر به خوبی می بلافد لا نسلم لا نسلم}
\label{sec:1581}
\addcontentsline{toc}{section}{\nameref{sec:1581}}
\begin{longtable}{l p{0.5cm} r}
گر به خوبی می بلافد لا نسلم لا نسلم
&&
کاندر این مکتب ندارد کر و فری هر معلم
\\
متهم شو همچو یوسف تا در آن زندان درآیی
&&
زانک در زندان نیاید جز مگر بدنام و ظالم
\\
جای عاقل صدر دیوان جای مجنون قعر زندان
&&
حبس و تهمت قسم عاشق تخت و منبر جای عالم
\\
کم طمع شد آن کسی کو طمع در عشق تو بندد
&&
کم سخن شد آن کسی که عشق با او شد مکالم
\\
پنجه اندر خون شیران دارد آن شیر سمایی
&&
غمزه خون خوار دارد غم ندارد از مظالم
\\
گر بگویم ور خموشم ور بجوشم ور نجوشم
&&
اندر این فتنه خوشم من تو برو می باش سالم
\\
مشک بربند ای سقا تو گر چه اندر وقت خوردن
&&
مستی آرد این معانی حیرت آرد این معالم
\\
\end{longtable}
\end{center}
