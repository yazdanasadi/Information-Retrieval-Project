\begin{center}
\section*{غزل شماره ۱۰۶۸: نیشکر باید که بندد پیش آن لب‌ها کمر}
\label{sec:1068}
\addcontentsline{toc}{section}{\nameref{sec:1068}}
\begin{longtable}{l p{0.5cm} r}
نیشکر باید که بندد پیش آن لب‌ها کمر
&&
خسروی باید که نوشم زان لب شیرین شکر
\\
بلک دریاییست عشق و موج رحمت می‌زند
&&
ابر بفرستد به دوران و به نزدیکان گهر
\\
صد سلام و بندگی ای جان از این مستان بخوان
&&
جام زرین پیش آر و سیم بر ای سیمبر
\\
پشت آنی تو که پشتش از غم و محنت شکست
&&
آب آنی که ندارد هیچ آبی بر جگر
\\
پخته شد نان دلی کز تف عشق تو بسوخت
&&
شد زبردست ابد آن کز تو شد زیر و زبر
\\
زان سر مستانش رست از خنجر قصاب مرگ
&&
که نبودند اندر این سودا چو ساطوری دوسر
\\
می بیار ای عشق بهر جان فرزندان خویش
&&
محو کن اندیشه‌ها را زان شراب چون شرر
\\
دی بدادی آنچ دادی جمع را ای میرداد
&&
بخش امروزینه کو ای هر دمی بخشنده تر
\\
بس کن و پرده دگر زن تا نگردد کس ملول
&&
می پر از باغی به باغی این چنین کن پرشکر
\\
\end{longtable}
\end{center}
