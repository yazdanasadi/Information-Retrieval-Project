\begin{center}
\section*{غزل شماره ۷۲۸: دشمن خویشیم و یار آنک ما را می‌کشد}
\label{sec:0728}
\addcontentsline{toc}{section}{\nameref{sec:0728}}
\begin{longtable}{l p{0.5cm} r}
دشمن خویشیم و یار آنک ما را می‌کشد
&&
غرق دریاییم و ما را موج دریا می‌کشد
\\
زان چنین خندان و خوش ما جان شیرین می‌دهیم
&&
کان ملک ما را به شهد و قند و حلوا می‌کشد
\\
خویش فربه می‌نماییم از پی قربان عید
&&
کان قصاب عاشقان بس خوب و زیبا می‌کشد
\\
آن بلیس بی‌تبش مهلت همی‌خواهد از او
&&
مهلتی دادش که او را بعد فردا می‌کشد
\\
همچو اسماعیل گردن پیش خنجر خوش بنه
&&
درمدزد از وی گلو گر می‌کشد تا می‌کشد
\\
نیست عزرائیل را دست و رهی بر عاشقان
&&
عاشقان عشق را هم عشق و سودا می‌کشد
\\
کشتگان نعره زنان یا لیت قومی یعلمون
&&
خفیه صد جان می‌دهد دلدار و پیدا می‌کشد
\\
از زمین کالبد برزن سری وانگه ببین
&&
کو تو را بر آسمان بر می‌کشد یا می‌کشد
\\
روح ریحی می‌ستاند راح روحی می‌دهد
&&
باز جان را می‌رهاند جغد غم را می‌کشد
\\
آن گمان ترسا برد مؤمن ندارد آن گمان
&&
کو مسیح خویشتن را بر چلیپا می‌کشد
\\
هر یکی عاشق چو منصورند خود را می‌کشند
&&
غیر عاشق وانما که خویش عمدا می‌کشد
\\
صد تقاضا می‌کند هر روز مردم را اجل
&&
عاشق حق خویشتن را بی‌تقاضا می‌کشد
\\
بس کنم یا خود بگویم سر مرگ عاشقان
&&
گر چه منکر خویش را از خشم و صفرا می‌کشد
\\
شمس تبریزی برآمد بر افق چون آفتاب
&&
شمع‌های اختران را بی‌محابا می‌کشد
\\
\end{longtable}
\end{center}
