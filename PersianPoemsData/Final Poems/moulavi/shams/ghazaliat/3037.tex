\begin{center}
\section*{غزل شماره ۳۰۳۷: به خاک پای تو ای مه هر آن شبی که بتابی}
\label{sec:3037}
\addcontentsline{toc}{section}{\nameref{sec:3037}}
\begin{longtable}{l p{0.5cm} r}
به خاک پای تو ای مه هر آن شبی که بتابی
&&
به جای عمر عزیزی چو عمر ما نشتابی
\\
چو شب روان هوس را تو چشمی و تو چراغی
&&
مسافران فلک را تو آتشی و تو آبی
\\
در این منازل گردون در این طواف همایون
&&
گر از قضا مه ما را به اتفاق بیابی
\\
اگر چه روح جهانست و روح سوی ندارد
&&
ثواب کن سوی او رو اگر چه غرق ثوابی
\\
بگو به تست پیامی اگر چه حاضر جانی
&&
جواب ده به حق آنک بس لطیف جوابی
\\
هزار مهره ربودی هنوز اول بازیست
&&
هزار پرده دریدی هنوز زیر نقابی
\\
چه ناله‌هاست نهان و چه زخم‌هاست دلم را
&&
زهی رباب دل من به دست چون تو ربابی
\\
دلم تو را چو ربابی تنم تو را چو خرابی
&&
رباب می‌زن و می‌گرد مست گرد خرابی
\\
همه ز جام تو مستند هر یکی ز شرابی
&&
ز جام خویش نپرسی که مست از چه شرابی
\\
کجاست ساحل دریا دلا که هر دم غرقی
&&
کجاست آتش غیبی که لحظه لحظه کبابی
\\
\end{longtable}
\end{center}
