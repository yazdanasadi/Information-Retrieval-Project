\begin{center}
\section*{غزل شماره ۱۱۴۵: به من نگر که منم مونس تو اندر گور}
\label{sec:1145}
\addcontentsline{toc}{section}{\nameref{sec:1145}}
\begin{longtable}{l p{0.5cm} r}
به من نگر که منم مونس تو اندر گور
&&
در آن شبی که کنی از دکان و خانه عبور
\\
سلام من شنوی در لحد خبر شودت
&&
که هیچ وقت نبودی ز چشم من مستور
\\
منم چو عقل و خرد در درون پرده تو
&&
به وقت لذت و شادی به گاه رنج و فتور
\\
شب غریب چو آواز آشنا شنوی
&&
رهی ز ضربت مار و جهی ز وحشت مور
\\
خمار عشق درآرد به گور تو تحفه
&&
شراب و شاهد و شمع و کباب و نقل و بخور
\\
در آن زمان که چراغ خرد بگیرانیم
&&
چه‌های و هوی برآید ز مردگان قبور
\\
ز های و هوی شود خیره خاک گورستان
&&
ز بانگ طبل قیامت ز طمطراق نشور
\\
کفن دریده گرفته دو گوش خود از بیم
&&
دماغ و گوش چه باشد به پیش نفخه صور
\\
به هر طرف نگری صورت مرا بینی
&&
اگر به خود نگری یا به سوی آن شر و شور
\\
ز احولی بگریز و دو چشم نیکو کن
&&
که چشم بد بود آن روز از جمالم دور
\\
به صورت بشرم‌هان و هان غلط نکنی
&&
که روح سخت لطیفست عشق سخت غیور
\\
چه جای صورت اگر خود نمد شود صدتو
&&
شعاع آینه جان علم زند به ظهور
\\
دهل زنید و سوی مطربان شهر تنید
&&
مراهقان ره عشق راست روز ظهور
\\
به جای لقمه و پول ار خدای را جستی
&&
نشسته بر لب خندق ندیدیی یک کور
\\
به شهر ما تو چه غمازخانه بگشادی
&&
دهان بسته تو غماز باش همچون نور
\\
\end{longtable}
\end{center}
