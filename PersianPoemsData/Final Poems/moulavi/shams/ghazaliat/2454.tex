\begin{center}
\section*{غزل شماره ۲۴۵۴: عیش جهان پیسه بود گاه خوشی گاه بدی}
\label{sec:2454}
\addcontentsline{toc}{section}{\nameref{sec:2454}}
\begin{longtable}{l p{0.5cm} r}
عیش جهان پیسه بود گاه خوشی گاه بدی
&&
عاشق او شو که دهد ملکت عیش ابدی
\\
چونک سپید است و سیه روز و شب عمر همه
&&
عمر دگر جو که بود ساده چو نور صمدی
\\
ای تو فرورفته به خود گاه از آن گور و لحد
&&
غافل از این لحظه که تو در لحد بود خودی
\\
دیدن روزی ده تو رزق حلال است تو را
&&
گرم به دکان چه روی در پی رزق عددی
\\
نادره طوطی که تویی کان شکر باطن تو
&&
نادره بلبل که تویی گلشنی و لعل خدی
\\
لیلی و مجنون عجب هر دو به یک پوست درون
&&
آینه هر دو تویی لیک درون نمدی
\\
عالم جان بحر صفا صورت و قالب کف او
&&
بحر صفا را بنگر چنگ در این کف چه زدی
\\
هیچ قراری نبود بر سر دریا کف را
&&
ز آنک قرارش ندهد جنبش موج مددی
\\
ز آنک کف از خشک بود لایق دریا نبود
&&
نیک به نیکی رود و بد برود سوی بدی
\\
کف همگی آب شود یا به کناری برود
&&
ز آنک دورنگی نبود در دل بحر احدی
\\
موج برآید ز خود و در خود نظاره کند
&&
سجده کنان کای خود من آه چه بیرون ز حدی
\\
جمله جان‌هاست یکی وین همه عکس ملکی
&&
دیده احول بگشا خوش نگر ار باخردی
\\
\end{longtable}
\end{center}
