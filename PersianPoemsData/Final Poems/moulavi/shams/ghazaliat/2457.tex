\begin{center}
\section*{غزل شماره ۲۴۵۷: ای دل سرگشته شده در طلب یاوه روی}
\label{sec:2457}
\addcontentsline{toc}{section}{\nameref{sec:2457}}
\begin{longtable}{l p{0.5cm} r}
ای دل سرگشته شده در طلب یاوه روی
&&
چند بگفتم که مده دل به کسی بی‌گروی
\\
بر سر شطرنج بتی جامه کنی کیسه بری
&&
با چو منی ساده دلی خیره سری خیره شوی
\\
برد همه رخت مرا نیست مرا برگ کهی
&&
آنک ز گنج زر او من نرسیدم به جوی
\\
تا بخورد تا ببرد جان مرا عشق کهن
&&
آن کهنی کو دهدم هر نفسی جان نوی
\\
آن کهنی نوصفتی همچو خدا بی‌جهتی
&&
خوش گهری خوش نظری خوش خبری خوش شنوی
\\
خرمن گل گشت جهان از رخت ای سرو روان
&&
دشمن تو جو دروی یار تو گندم دروی
\\
جذب کن ای بادصفت آب وجود همه را
&&
برکش خورشیدصفت شبنمه‌ای رازگوی
\\
ای تو چو خورشید ولی نی چو تفش داغ کنی
&&
ای چو صبا بالطفی نی چو صبا خیره دوی
\\
گر صفتی در دل من کژ شود آن را تو بکن
&&
شاخ کژی را بکند صاحب بستان به خوی
\\
گر چه شود خانه دین رخنه ز موش حسدی
&&
موش کی باشد برمد از دم گربه به موی
\\
سبز شود آب و گلی چون دهدش وصل دلی
&&
دلبر و دل جمع شدند لیک نباشند دوی
\\
پیشتر آ تا که نه من مانم این جا نه سخن
&&
ظلمت هستی چه زند پیش صبوح چو تویی
\\
\end{longtable}
\end{center}
