\begin{center}
\section*{غزل شماره ۳۱۹۴: وقتت خوش ای حبیبی، بشنو بحق یاری}
\label{sec:3194}
\addcontentsline{toc}{section}{\nameref{sec:3194}}
\begin{longtable}{l p{0.5cm} r}
وقتت خوش ای حبیبی، بشنو بحق یاری
&&
ارحم حنین قلبی لا تسع فی ضراری
\\
دل را مکن چو خاره، مگزین ز ما کناره
&&
یا منیة الفاد، دار ولا تمار
\\
ساقی خاص روحی، در ده می صبوحی
&&
اللیل قد تولی و البدر فی‌التواری
\\
ای برده هوش ما را، یار آر دوش ما را
&&
اسقیتنا کسا صرفا علی‌الخمار
\\
مار را خراب کردی، غرق شراب کردی
&&
حتی بدا و افشا، ما کان فی سراری
\\
سلطان خیل مایی، لیلی لیل مایی
&&
یا لدةاللیالی، یا بهجةالنهار
\\
ای سر طور سینا وی نور چشم بینا
&&
انت‌الکبیر فینا، فارحم علی‌اصغار
\\
هین نوبت جنون شد، مستی ما فزون شد
&&
یا مسکرالعقول، یا هادم‌الوقار
\\
شاه سخن‌ور آمد، موج سخن درآمد
&&
نحن‌الصدا نصدی، والله خیر قاری
\\
\end{longtable}
\end{center}
