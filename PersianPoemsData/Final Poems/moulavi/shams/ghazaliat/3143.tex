\begin{center}
\section*{غزل شماره ۳۱۴۳: مستی و عاشقانه می‌گویی}
\label{sec:3143}
\addcontentsline{toc}{section}{\nameref{sec:3143}}
\begin{longtable}{l p{0.5cm} r}
مستی و عاشقانه می‌گویی
&&
تو غریبی و یا از این کویی
\\
پیش آن چشم‌های جادوی تو
&&
چون نباشد حرام جادویی
\\
پیش رویت چو قرص مه خجلست
&&
به چه رو کرد زهره بی‌رویی
\\
عاشقان را چه سود دارد پند
&&
سیل شان برد رو چه می‌جویی
\\
تو چه دانی ز خوبی بت ما
&&
ما از آن سو و تو از این سویی
\\
ما ز دستان او ز دست شدیم
&&
دست از ما چرا نمی‌شویی
\\
رو به میدان عشق سجده کنان
&&
پیش چوگان عشق چون گویی
\\
پیش آن چشم‌های ترکانه
&&
بنده‌ای و کمینه هندویی
\\
به ستیزه در این حرم ای صبر
&&
گاه لاله و گاه لولویی
\\
آفتابا نه حد تو پیداست
&&
که نه در خانه ترازویی
\\
هله ای ماه خویش را بشناس
&&
نی به وقت محاق چون مویی
\\
هله ای زهره زیر چادر رو
&&
رو نداری وقیحه بانویی
\\
تو بیا ای کمال صورت عشق
&&
نور ذات حقی و یا اویی
\\
اندر این ره نماند پای مرا
&&
زانوم را نماند زانویی
\\
همچو کشتی روم به پهلو من
&&
ای دل من هزارپهلویی
\\
مست و بی‌خویش می‌روی چپ و راست
&&
سوی بی‌چپ و راست می‌پویی
\\
نی چپست و نه راست در جانست
&&
بو ز جان یابی ار بینبویی
\\
ز آن شکر روی اگر بگردانی
&&
گر نباتی بدان که بدخویی
\\
ور تو دیوی و رو بدو آری
&&
الله الله چه ماه ده تویی
\\
دلم از جا رود چو گویم او
&&
همه اوها غلام این اویی
\\
هین ز خوهای او یکی بشنو
&&
گاه شیری کند گه آهویی
\\
هین خمش که ار دیده کف نکند
&&
نکند سیب و نار آلویی
\\
\end{longtable}
\end{center}
