\begin{center}
\section*{غزل شماره ۷۴۵: آمدم تا رو نهم بر خاک پای یار خود}
\label{sec:0745}
\addcontentsline{toc}{section}{\nameref{sec:0745}}
\begin{longtable}{l p{0.5cm} r}
آمدم تا رو نهم بر خاک پای یار خود
&&
آمدم تا عذر خواهم ساعتی از کار خود
\\
آمدم کز سر بگیرم خدمت گلزار او
&&
آمدم کآتش بیارم درزنم در خار خود
\\
آمدم تا صاف گردم از غبار هر چه رفت
&&
نیک خود را بد شمارم از پی دلدار خود
\\
آمدم با چشم گریان تا ببیند چشم من
&&
چشمه‌های سلسبیل از مهر آن عیار خود
\\
خیز ای عشق مجرد مهر را از سر بگیر
&&
مردم و خالی شدم ز اقرار و از انکار خود
\\
زانک بی‌صاف تو نتوان صاف گشتن در وجود
&&
بی تو نتوان رست هرگز از غم و تیمار خود
\\
من خمش کردم به ظاهر لیک دانی کز درون
&&
گفت خون آلود دارم در دل خون خوار خود
\\
درنگر در حال خاموشی به رویم نیک نیک
&&
تا ببینی بر رخ من صد هزار آثار خود
\\
این غزل کوتاه کردم باقی این در دل است
&&
گویم ار مستم کنی از نرگس خمار خود
\\
ای خموش از گفت خویش و ای جدا از جفت خویش
&&
چون چنین حیران شدی از عقل زیرکسار خود
\\
ای خمش چونی از این اندیشه‌های آتشین
&&
می‌رسد اندیشه‌ها با لشکر جرار خود
\\
وقت تنهایی خمش باشند و با مردم بگفت
&&
کس نگوید راز دل را با در و دیوار خود
\\
تو مگر مردم نمی‌یابی که خامش کرده‌ای
&&
هیچ کس را می‌نبینی محرم گفتار خود
\\
تو مگر از عالم پاکی نیامیزی به طبع
&&
با سگان طبع کآلودند از مردار خود
\\
\end{longtable}
\end{center}
