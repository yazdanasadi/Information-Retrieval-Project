\begin{center}
\section*{غزل شماره ۲۹۳۰: تو چرا جمله نبات و شکری}
\label{sec:2930}
\addcontentsline{toc}{section}{\nameref{sec:2930}}
\begin{longtable}{l p{0.5cm} r}
تو چرا جمله نبات و شکری
&&
تو چرا دلبر و شیرین نظری
\\
تو چرا همچو گل خندانی
&&
تو چرا تازه چو شاخ شجری
\\
تو به یک خنده چرا راه زنی
&&
تو به یک غمزه چرا عقل بری
\\
تو چرا صاف چو صحن فلکی
&&
تو چرا چست چو قرص قمری
\\
تو چرا بی‌بنه چون دریایی
&&
تو چرا روشن و خوش چون گهری
\\
عاقلان را ز چه دیوانه کنی
&&
ای همه پیشه تو فتنه گری
\\
ساکنان را ز چه در رقص آری
&&
ز آدمی و ملک و دیو و پری
\\
تو چرا توبه مردم شکنی
&&
تو چرا پرده مردم بدری
\\
همه دل‌ها چو در اندیشه توست
&&
تو کجایی به چه اندیشه دری
\\
\end{longtable}
\end{center}
