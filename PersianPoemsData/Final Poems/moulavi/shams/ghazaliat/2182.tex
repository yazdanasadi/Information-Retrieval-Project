\begin{center}
\section*{غزل شماره ۲۱۸۲: به پیشت نام جان گویم زهی رو}
\label{sec:2182}
\addcontentsline{toc}{section}{\nameref{sec:2182}}
\begin{longtable}{l p{0.5cm} r}
به پیشت نام جان گویم زهی رو
&&
حدیث گلستان گویم زهی رو
\\
تو این جا حاضر و شرمم نباشد
&&
که از حسن بتان گویم زهی رو
\\
بهار و صد بهار از تو خجل شد
&&
من افسانه خزان گویم زهی رو
\\
تو شاهنشاه صد جان و جهانی
&&
من از جان و جهان گویم زهی رو
\\
حدیثت در دهان جان نگنجد
&&
حدیثت از زبان گویم زهی رو
\\
جهان گم گشت و ماهت آشکارا
&&
چنین مه را نهان گویم زهی رو
\\
همه عالم ز نورت لعل در لعل
&&
به پیش تو ز کان گویم زهی رو
\\
ز تو دل‌ها پر از نور یقین است
&&
یقین را از گمان گویم زهی رو
\\
چو خورشید جمالت بر زمین تافت
&&
ز ماه و اختران گویم زهی رو
\\
چو لطف شمس تبریزی ز حد رفت
&&
من از وی گر فغان گویم زهی رو
\\
\end{longtable}
\end{center}
