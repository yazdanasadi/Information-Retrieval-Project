\begin{center}
\section*{غزل شماره ۲۵۱۹: غلام پاسبانانم که یارم پاسبانستی}
\label{sec:2519}
\addcontentsline{toc}{section}{\nameref{sec:2519}}
\begin{longtable}{l p{0.5cm} r}
غلام پاسبانانم که یارم پاسبانستی
&&
به چستی و به شبخیزی چو ماه و اخترانستی
\\
غلام باغبانانم که یارم باغبانستی
&&
به تری و به رعنایی چو شاخ ارغوانستی
\\
نباشد عاشقی عیبی وگر عیب است تا باشد
&&
که نفسم عیب دان آمد و یارم غیب دانستی
\\
اگر عیب همه عالم تو را باشد چو عشق آمد
&&
بسوزد جمله عیبت را که او بس قهرمانستی
\\
گذشتم بر گذرگاهی بدیدم پاسبانی را
&&
نشسته بر سر بامی که برتر ز آسمانستی
\\
کلاه پاسبانانه قبای پاسبانانه
&&
ولیک از های های او در عالم در امانستی
\\
به دست دیدبان او یکی آیینه‌ای شش سو
&&
که حال شش جهت یک یک در آیینه بیانستی
\\
چو من دزدی بدم رهبر طمع کردم بدان گوهر
&&
برآوردم یکی شکلی که بیرون از گمانستی
\\
ز هر سویی که گردیدم نشانه تیر او دیدم
&&
ز هر شش سو برون رفتم که آن ره بی‌نشانستی
\\
همه سوها ز بی‌سو شد نشان از بی‌نشان آمد
&&
چو آمد راه واگشتن ز آینده نهانستی
\\
چو زان شش پرده تاری برون رفتم به عیاری
&&
ز نور پاسبان دیدم که او شاه جهانستی
\\
چو باغ حسن شه دیدم حقیقت شد بدانستم
&&
که هم شه باغبانستی و هم شه باغ جانستی
\\
از او گر سنگسار آیی تو شیشه عشق را مشکن
&&
ازیرا رونق نقدت ز سنگ امتحانستی
\\
ز شاهان پاسبانی خود ظریف و طرفه می‌آید
&&
چنان خود را خلق کرده که نشناسی که آنستی
\\
لباس جسم پوشیده که کمتر کسوه آن است
&&
سخن در حرف آورده که آن دونتر زبانستی
\\
به گل اندوده خورشیدی میان خاک ناهیدی
&&
درون دلق جمشیدی که گنج خاکدانستی
\\
زبان وحییان را او ز ازل وجه العرب بوده
&&
زبان هندوی گوید که خود از هندوانستی
\\
زمین و آسمان پیشش دو که برگ است پنداری
&&
که در جسم از زمینستی و در عمر از زمانستی
\\
ز یک خندش مصور شد بهشت ار هشت ور بیش است
&&
به چشم ابلهان گویی ز جنت ارمغانستی
\\
بر او صفرا کنند آنگه ز نخوت اصل سیم و زر
&&
که ما زر و هنر داریم و غافل زو که کانستی
\\
چه عذر آرند آن روزی که عذرا گردد از پرده
&&
چه خون گریند آن صبحی که خورشیدش عیانستی
\\
میان بلغم و صفرا و خون و مره و سودا
&&
نماید روح از تأثیر گویی در میانستی
\\
ز تن تا جان بسی راه است و در تن می‌نماند جان
&&
چنین دان جان عالم را کز او عالم جوانستی
\\
نه شخص عالم کبری چنین بر کار بی‌جان است
&&
که چرخ ار بی‌روانستی بدین سان کی روانستی
\\
زمین و آسمان‌ها را مدد از عالم عقل است
&&
که عقل اقلیم نورانی و پاک درفشانستی
\\
جهان عقل روشن را مددها از صفات آید
&&
صفات ذات خلاقی که شاه کن فکانستی
\\
که این تیر عوارض را که می‌پرد به هر سویی
&&
کمان پنهان کند صانع ولی تیر از کمانستی
\\
اگر چه عقل بیدار است آن از حی قیوم است
&&
اگر چه سگ نگهبان است تأثیر شبانستی
\\
چو سگ آن از شبان بیند زیانش جمله سودستی
&&
چو سگ خود را شبان بیند همه سودش زیانستی
\\
چو خود را ملک او بینی جهان اندر جهان باشی
&&
وگر خود را ملک دانی جهان از تو جهانستی
\\
تو عقل کل چو شهری دان سواد شهر نفس کل
&&
و این اجزا در آمدشد مثال کاروانستی
\\
خنک آن کاروانی کان سلامت با وطن آید
&&
غنیمت برده و صحت و بختش همعنانستی
\\
خفیر ارجعی با او بشیر ابشروا بر ره
&&
سلام شاه می‌آرند و جان دامن کشانستی
\\
خواطر چون سوارانند و زوتر زی وطن آیند
&&
و یا بازان و زاغانند پس در آشیانستی
\\
خواطر رهبرانند و چو رهبر مر تو را بار است
&&
مقامت ساعد شه دان که شاه شه نشانستی
\\
وگر زاغ است آن خاطر که چشمش سوی مردار است
&&
کسی کش زاغ رهبر شد به گورستان روانستی
\\
چو در مازاغ بگریزی شود زاغ تو شهبازی
&&
که اکسیر است شادی ساز او را کاندهانستی
\\
گر آن اصلی که زاغ و باز از او تصویر می‌یابد
&&
تجلی سازدی مطلق اصالت را یگانستی
\\
ور آن نوری کز او زاید غم و شادی به یک اشکم
&&
دمی پهلو تهی کردی همه کس شادمانستی
\\
همه اجزا همی‌گویند هر یک ای همه تو تو
&&
همین گفت ار نه پرده ستی همه با همگنانستی
\\
درخت جان‌ها رقصان ز باد این چنین باده
&&
گران باد آشکارستی نه لنگر بادبانستی
\\
درای کاروان دل به گوشم بانگ می‌آرد
&&
گر آن بانگش به حس آید هر اشتر ساربانستی
\\
درافتد از صدف هر دم صدف بازش خورد در دم
&&
وگر نه عین کری هم کران را ترجمانستی
\\
سهیل شمس تبریزی نتابد در یمن ور نی
&&
ادیم طایفی گشتی به هر جا سختیانستی
\\
ضیاوار ای حسام الدین ضیاء الحق گواهی ده
&&
ندیدی هیچ دیده گر ضیا نه دیدبانستی
\\
گواهی ضیا هم او گواهی قمر هم رو
&&
گواهی مشک اذفربو که بر عالم وزانستی
\\
اگر گوشت شود دیده گواهی ضیا بشنو
&&
ولی چشم تو گوش آمد که حرفش گلستانستی
\\
چو از حرفی گلستانی ز معنی کی گل استانی
&&
چو پا در قیر جزوستت حجابت قیروانستی
\\
کتاب حس به دست چپ کتاب عقل دست راست
&&
تو را نامه به چپ دادند که بیرون ز آستانستی
\\
چو عقلت طبع حس دارد و دست راست خوی چپ
&&
و تبدیل طبیعت هم نه کار داستانستی
\\
خداوندا تو کن تبدیل که خود کار تو تبدیل است
&&
که اندر شهر تبدیلت زبان‌ها چون سنانستی
\\
عدم را در وجود آری از این تبدیل افزونتر
&&
تو نور شمع می‌سازی که اندر شمعدانستی
\\
تو بستان نامه از چپم به دست راستم درنه
&&
تو تانی کرد چپ را راست بنده ناتوانستی
\\
ترازوی سبک دارم گرانش کن به فضل خود
&&
تو که را که کنی زیرا نه کوه از خود گرانستی
\\
کمال لطف داند شد کمال نقص را چاره
&&
که قعر دوزخ ار خواهی به از صدر جنانستی
\\
\end{longtable}
\end{center}
