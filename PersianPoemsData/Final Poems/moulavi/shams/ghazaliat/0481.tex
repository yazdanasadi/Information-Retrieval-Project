\begin{center}
\section*{غزل شماره ۴۸۱: چه گوهری تو که کس را به کف بهای تو نیست}
\label{sec:0481}
\addcontentsline{toc}{section}{\nameref{sec:0481}}
\begin{longtable}{l p{0.5cm} r}
چه گوهری تو که کس را به کف بهای تو نیست
&&
جهان چه دارد در کف که آن عطای تو نیست
\\
سزای آنک زید بی‌رخ تو زین بترست
&&
سزای بنده مده گر چه او سزای تو نیست
\\
نثار خاک تو خواهم به هر دمی دل و جان
&&
که خاک بر سر جانی که خاک پای تو نیست
\\
مبارکست هوای تو بر همه مرغان
&&
چه نامبارک مرغی که در هوای تو نیست
\\
میان موج حوادث هر آنک استادست
&&
به آشنا نرهد چونک آشنای تو نیست
\\
بقا ندارد عالم وگر بقا دارد
&&
فناش گیر چو او محرم بقای تو نیست
\\
چه فرخست رخی کو شهیت را ماتست
&&
چه خوش لقا بود آن کس که بی‌لقای تو نیست
\\
ز زخم تو نگریزم که سخت خام بود
&&
دلی که سوخته آتش بلای تو نیست
\\
دلی که نیست نشد روی در مکان دارد
&&
ز لامکانش برانی که رو که جای تو نیست
\\
کرانه نیست ثنا و ثناگران تو را
&&
کدام ذره که سرگشته ثنای تو نیست
\\
نظیر آنک نظامی به نظم می‌گوید
&&
جفا مکن که مرا طاقت جفای تو نیست
\\
\end{longtable}
\end{center}
