\begin{center}
\section*{غزل شماره ۱۰۳۷: گیرم که بود میر تو را زر به خروار}
\label{sec:1037}
\addcontentsline{toc}{section}{\nameref{sec:1037}}
\begin{longtable}{l p{0.5cm} r}
گیرم که بود میر تو را زر به خروار
&&
رخساره چون زر ز کجا یابد زردار
\\
از دلشده زار چو زاری بشنیدند
&&
از خاک برآمد به تماشا گل و گلزار
\\
هین جامه بکن زود در این حوض فرورو
&&
تا بازرهی از سر و از غصه دستار
\\
ما نیز چو تو منکر این غلغله بودیم
&&
گشتیم به یک غمزه چنین سغبه دلدار
\\
تا کی شکنی عاشق خود را تو ز غیرت
&&
هل تا دو سه ناله بکند این دل بیمار
\\
نی نی مهلش زانک از آن ناله زارش
&&
نی خلق زمین ماند و نی چرخه دوار
\\
امروز عجب نیست اگر فاش نگردد
&&
آن عالم مستور به دستوری ستار
\\
باز این دل دیوانه ز زنجیر برون جست
&&
بدرید گریبان خود از عشق دگربار
\\
خامش که اشارت ز شه عشق چنین است
&&
کز صبر گلوی دل و جان گیر و بیفشار
\\
\end{longtable}
\end{center}
