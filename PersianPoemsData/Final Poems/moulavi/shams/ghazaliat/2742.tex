\begin{center}
\section*{غزل شماره ۲۷۴۲: ای آنک تو خواب ما ببستی}
\label{sec:2742}
\addcontentsline{toc}{section}{\nameref{sec:2742}}
\begin{longtable}{l p{0.5cm} r}
ای آنک تو خواب ما ببستی
&&
رفتی و به گوشه‌ای نشستی
\\
اندر دلم آمدی چو ماهی
&&
چون دل به تو بنگرید جستی
\\
چون گلشن نیستی نمودی
&&
چون صبر کنیم ما به هستی
\\
چون باشد در خمار هجران
&&
آن روح که یافت وصل و مستی
\\
آن خانه چگونه خانه ماند
&&
کز هجر ستون او شکستی
\\
پنداشتی ای دماغ سرمست
&&
کز رنج خمار بازرستی
\\
در عشق وصال هست و هجران
&&
در راه بلندی است و پستی
\\
از یک جهت ار چه حق شناسی
&&
از ده جهت آب و گل پرستی
\\
بسیار ره است تا به جایی
&&
کاندر سوداش طمع بستی
\\
\end{longtable}
\end{center}
