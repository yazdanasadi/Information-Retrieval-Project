\begin{center}
\section*{غزل شماره ۱۸۶: ای میرآب بگشا آن چشمه روان را}
\label{sec:0186}
\addcontentsline{toc}{section}{\nameref{sec:0186}}
\begin{longtable}{l p{0.5cm} r}
ای میرآب بگشا آن چشمه روان را
&&
تا چشم‌ها گشاید ز اشکوفه بوستان را
\\
آب حیات لطفت در ظلمت دو چشم است
&&
زان مردمک چو دریا کردست دیدگان را
\\
هرگز کسی نرقصد تا لطف تو نبیند
&&
کاندر شکم ز لطفت رقص است کودکان را
\\
اندر شکم چه باشد و اندر عدم چه باشد
&&
کاندر لحد ز نورت رقص است استخوان را
\\
بر پرده‌های دنیا بسیار رقص کردیم
&&
چابک شوید یاران مر رقص آن جهان را
\\
جان‌ها چو می‌برقصد با کندهای قالب
&&
خاصه چو بسکلاند این کنده گران را
\\
پس ز اول ولادت بودیم پای کوبان
&&
در ظلمت رحم‌ها از بهر شکر جان را
\\
پس جمله صوفیانیم از خانقه رسیده
&&
رقصان و شکرگویان این لوت رایگان را
\\
این لوت را اگر جان بدهیم رایگانست
&&
خود چیست جان صوفی این گنج شایگان را
\\
چون خوان این جهان را سرپوش آسمانست
&&
از خوان حق چه گویم زهره بود زبان را
\\
ما صوفیان راهیم ما طبل خوار شاهیم
&&
پاینده دار یا رب این کاسه را و خوان را
\\
در کاسه‌های شاهان جز کاسه شست ما نی
&&
هر خام درنیابد این کاسه را و نان را
\\
از کاسه‌های نعمت تا کاسه ملوث
&&
پیش مگس چه فرق است آن ننگ میزبان را
\\
وان کس که کس بود او ناخورده و چشیده
&&
گه می‌گزد زبان را گه می‌زند دهان را
\\
\end{longtable}
\end{center}
