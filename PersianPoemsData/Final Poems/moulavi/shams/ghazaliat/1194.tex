\begin{center}
\section*{غزل شماره ۱۱۹۴: گر نه‌ای دیوانه رو مر خویش را دیوانه ساز}
\label{sec:1194}
\addcontentsline{toc}{section}{\nameref{sec:1194}}
\begin{longtable}{l p{0.5cm} r}
گر نه‌ای دیوانه رو مر خویش را دیوانه ساز
&&
گر چه صد ره مات گشتی مهره دیگر بباز
\\
گر چه چون تاری ز زخمش زخمه دیگر بزن
&&
بازگرد ای مرغ گر چه خسته‌ای از چنگ باز
\\
چند خانه گم کنی و یاوه گردی گرد شهر
&&
ور ز شهری نیز یاوه با قلاوزی بساز
\\
اسب چوبین برتراشیدی که این اسب منست
&&
گر نه چوبینست اسبت خواجه یک منزل بتاز
\\
دعوت حق نشنوی آنگه دعاها می‌کنی
&&
شرم بادت ای برادر زین دعای بی‌نماز
\\
سر به سر راضی نه‌ای که سر بری از تیغ حق
&&
کی دهد بو همچو عنبر چونک سیری و پیاز
\\
گر نیازت را پذیرد شمس تبریزی ز لطف
&&
بعد از آن بر عرش نه تو چاربالش بهر ناز
\\
\end{longtable}
\end{center}
