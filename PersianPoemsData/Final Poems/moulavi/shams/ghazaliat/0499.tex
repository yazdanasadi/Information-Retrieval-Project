\begin{center}
\section*{غزل شماره ۴۹۹: عشق جز دولت و عنایت نیست}
\label{sec:0499}
\addcontentsline{toc}{section}{\nameref{sec:0499}}
\begin{longtable}{l p{0.5cm} r}
عشق جز دولت و عنایت نیست
&&
جز گشاد دل و هدایت نیست
\\
عشق را بوحنیفه درس نکرد
&&
شافعی را در او روایت نیست
\\
لایجوز و یجوز تا اجل‌ست
&&
علم عشاق را نهایت نیست
\\
عاشقان غرقه‌اند در شکراب
&&
از شکر مصر را شکایت نیست
\\
جان مخمور چون نگوید شکر
&&
باده‌ای را که حد و غایت نیست
\\
هر که را پرغم و ترش دیدی
&&
نیست عاشق و زان ولایت نیست
\\
گر نه هر غنچه پرده باغی‌ست
&&
غیرت و رشک را سرایت نیست
\\
مبتدی باشد اندر این ره عشق
&&
آنک او واقف از بدایت نیست
\\
نیست شو نیست از خودی زیرا
&&
بتر از هستیت جنایت نیست
\\
هیچ راعی مشو رعیت شو
&&
راعیی جز سد رعایت نیست
\\
بس بدی بنده را کفی بالله
&&
لیکش این دانش و کفایت نیست
\\
گوید این مشکل و کنایاتست
&&
این صریح است این کنایت نیست
\\
پای کوری به کوزه‌ای برزد
&&
گفت فراش را وقایت نیست
\\
کوزه و کاسه چیست بر سر ره
&&
راه را زین خزف نقایت نیست
\\
کوزه‌ها را ز راه برگیرید
&&
یا که فراش در سعایت نیست
\\
گفت ای کور کوزه بر ره نیست
&&
لیک بر ره تو را درایت نیست
\\
ره رها کرده‌ای سوی کوزه
&&
می‌روی آن به جز غوایت نیست
\\
خواجه جز مستی تو در ره دین
&&
آیتی ز ابتدا و غایت نیست
\\
آیتی تو و طالب آیت
&&
به ز آیت طلب خود آیت نیست
\\
بی رهی ور نه در ره کوشش
&&
هیچ کوشنده بی‌جرایت نیست
\\
چونک مثقال ذره یره است
&&
ذره زله بی‌نکایت نیست
\\
ذره خیر بی‌گشادی نیست
&&
چشم بگشا اگر عمایت نیست
\\
هر نباتی نشانی آب است
&&
چیست کان را از او جبایت نیست
\\
بس کن این آب را نشانی‌هاست
&&
تشنه را حاجت وصایت نیست
\\
\end{longtable}
\end{center}
