\begin{center}
\section*{غزل شماره ۳۱۰۹: کالی تیشبی آپانسو، ای افندی چلبی}
\label{sec:3109}
\addcontentsline{toc}{section}{\nameref{sec:3109}}
\begin{longtable}{l p{0.5cm} r}
کالی تیشبی آپانسو، ای افندی چلبی
&&
نیمشب بر بام مایی، تا کرمی طلبی
\\
گه سیه‌پوش و عصا، که منم کالویروس
&&
گه عمامه و نیزهٔ که غریبم عربی
\\
هرچه هستی ای امیر، سخت مستی شیرگیر
&&
هر زبان خواهی بگو، خسروا شیرین لبی
\\
ارتمی آغاپسو، کایکاپر ترا
&&
نور حقی یا حقی، یا فرشته یا نبی
\\
چون غم دل می‌خورم، رحم بر دل می‌برم
&&
کای دل مسکین چرا در چنین تاب و تبی
\\
دل همی‌گوید که:« تو از کجا من از کجا
&&
من دلم تو قالبی، رو همی‌کن قالبی
\\
پوستها را رنگها، مغزها را ذوقها
&&
پوستها با مغزها کی کند هم مذهبی؟»
\\
کالی میرا لییری، پوستن کالاستن
&&
شب شما را روز شد، نیست شبها را شبی
\\
اشکلفیس چلپی، انپا پیسوایلادو
&&
سردهی کن لحظهٔ، زانک شیرین مشربی
\\
من خمش کردم، مرا بی‌زبان تعلیم ده
&&
آنچ ازو لرزد دل مشرقی و مغربی
\\
\end{longtable}
\end{center}
