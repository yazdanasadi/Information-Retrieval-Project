\begin{center}
\section*{غزل شماره ۲۴۱۳: عجب دلی که به عشق بت است پیوسته}
\label{sec:2413}
\addcontentsline{toc}{section}{\nameref{sec:2413}}
\begin{longtable}{l p{0.5cm} r}
عجب دلی که به عشق بت است پیوسته
&&
عجبتر این که بتش پیش او است بنشسته
\\
بمال چشم دلا بهترک از این بنگر
&&
مدو به هر طرف ای دل تو نیز آهسته
\\
دو کف به سوی دعا سوی بحر می‌رانی
&&
نه گوهر تو به جیب تو است پیوسته
\\
خنک کسی که ورا دست گرد جیب بود
&&
که او لطیف و سبک روح گشت و برجسته
\\
اگر چه هر طرفی بازگشت در طلبش
&&
از آن طلب چو به خود وانگشت شد خسته
\\
میان گلبن دل جان بخسته از خاری
&&
ببین دلا تو ز خاری هزار گلدسته
\\
میان دل چو برآید غبار و طبل و علم
&&
هزار سنجق هستی ببین تو بشکسته
\\
بیا به شهر عدم درنگر در آن مستان
&&
ببین ز خویش و هزاران چو خویش وارسته
\\
نهاده هر دو قدم شاد در سرای بقا
&&
و زین بساط فنا هر دو دست خود شسته
\\
\end{longtable}
\end{center}
