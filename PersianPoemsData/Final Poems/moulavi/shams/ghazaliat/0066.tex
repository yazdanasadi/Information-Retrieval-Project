\begin{center}
\section*{غزل شماره ۶۶: تو را ساقی جان گوید برای ننگ و نامی را}
\label{sec:0066}
\addcontentsline{toc}{section}{\nameref{sec:0066}}
\begin{longtable}{l p{0.5cm} r}
تو را ساقی جان گوید برای ننگ و نامی را
&&
فرومگذار در مجلس چنین اشگرف جامی را
\\
ز خون ما قصاصت را بجو این دم خلاصت را
&&
مهل ساقی خاصت را برای خاص و عامی را
\\
بکش جام جلالی را فدا کن نفس و مالی را
&&
مشو سخره حلالی را مخوان باده حرامی را
\\
غلط کردار نادانی همه نامیست یا نانی
&&
تو را چون پخته شد جانی مگیر ای پخته خامی را
\\
کسی کز نام می‌لافد بهل کز غصه بشکافد
&&
چو آن مرغی که می‌بافد به گرد خویش دامی را
\\
در این دام و در این دانه مجو جز عشق جانانه
&&
مگو از چرخ وز خانه تو دیده گیر بامی را
\\
تو شین و کاف و ری را خود مگو شکر که هست از نی
&&
مگو القاب جان حی یکی نقش و کلامی را
\\
چو بی‌صورت تو جان باشی چه نقصان گر نهان باشی
&&
چرا دربند آن باشی که واگویی پیامی را
\\
بیا ای هم دل محرم بگیر این باده خرم
&&
چنان سرمست شو این دم که نشناسی مقامی را
\\
برو ای راه ره پیما بدان خورشید جان افزا
&&
از این مجنون پرسودا ببر آن جا سلامی را
\\
بگو ای شمس تبریزی از آن می‌های پاییزی
&&
به خود در ساغرم ریزی نفرمایی غلامی را
\\
\end{longtable}
\end{center}
