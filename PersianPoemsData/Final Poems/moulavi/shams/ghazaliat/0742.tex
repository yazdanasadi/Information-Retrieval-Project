\begin{center}
\section*{غزل شماره ۷۴۲: عشق عاشق را ز غیرت نیک دشمن رو کند}
\label{sec:0742}
\addcontentsline{toc}{section}{\nameref{sec:0742}}
\begin{longtable}{l p{0.5cm} r}
عشق عاشق را ز غیرت نیک دشمن رو کند
&&
چونک رد خلق کردش عشق رو با او کند
\\
کنک شاید خلق را آن کس نشاید عشق را
&&
زانک جان روسپی باشد که او صد شو کند
\\
چون نشاید دیگران را تا همه ردش کنند
&&
شاه عشقش بعد از آن با خویش همزانو کند
\\
زانک خلقش چون براند خو ز خلقان واکند
&&
باطن و ظاهر همه با عشق خوش خو خو کند
\\
جان قبول خلق یابد خاطرش آن جا کشد
&&
دل به مهر هر کسی دزدیده رو هر سو کند
\\
چون ببیند عشق گوید زلف من سایه فکند
&&
وانگهی عاشق در این دم مشک و عنبر بو کند
\\
مشک و عنبر را کنم من خصم آن مغز و دماغ
&&
تا که عاشق از ضرورت ترک این هر دو کند
\\
گر چه هم بر یاد ما بو کرد عاشق مشک را
&&
نوطلب باشد که همچون طفلکان کوکو کند
\\
چونک از طفلی برون شد چشم دانش برگشاد
&&
بر لب جو کی دوادو بر نشان جو کند
\\
عاشق نوکار باشی تلخ گیر و تلخ نوش
&&
تا تو را شیرین ز شهد خسروی دارو کند
\\
تا بود کز شمس تبریزی بیابی مستیی
&&
از ورای هر دو عالم کان تو را بی‌تو کند
\\
\end{longtable}
\end{center}
