\begin{center}
\section*{غزل شماره ۲۷۱۵: مرا اندر جگر بنشست خاری}
\label{sec:2715}
\addcontentsline{toc}{section}{\nameref{sec:2715}}
\begin{longtable}{l p{0.5cm} r}
مرا اندر جگر بنشست خاری
&&
بحمدالله ز باغ او است باری
\\
یکی اقبال زفتی یافت جانم
&&
وگر چه شد تنم در عشق زاری
\\
کناری نیست این اقبال ما را
&&
چو بگرفتم چنین مه در کناری
\\
بگیر این عقل را بر دار او کش
&&
تماشا کن از این پس گیر و داری
\\
چو اندربافت این جانم به عشقش
&&
ز هستم تا نماند پود و تاری
\\
رخ گلنار گر در ره حجاب است
&&
چو گل در جان زنیمش زود ناری
\\
مشو غره به گلزار فنا تو
&&
که او گنده شود روزی سه چاری
\\
جمالی بین که حضرت عاشقستش
&&
بشو بهر چنین جان جان سپاری
\\
خداوندی شمس الدین تبریز
&&
کز او دارد خداوند افتخاری
\\
\end{longtable}
\end{center}
