\begin{center}
\section*{غزل شماره ۱۲۲: دیدم رخ خوب گلشنی را}
\label{sec:0122}
\addcontentsline{toc}{section}{\nameref{sec:0122}}
\begin{longtable}{l p{0.5cm} r}
دیدم رخ خوب گلشنی را
&&
آن چشم و چراغ روشنی را
\\
آن قبله و سجده گاه جان را
&&
آن عشرت و جای ایمنی را
\\
دل گفت که جان سپارم آن جا
&&
بگذارم هستی و منی را
\\
جان هم به سماع اندرآمد
&&
آغاز نهاد کف زنی را
\\
عقل آمد و گفت من چه گویم
&&
این بخت و سعادت سنی را
\\
این بوی گلی که کرد چون سرو
&&
هر پشت دوتای منحنی را
\\
در عشق بدل شود همه چیز
&&
ترکی سازند ارمنی را
\\
ای جان تو به جان جان رسیدی
&&
وی تن بگذاشتی تنی را
\\
یاقوت زکات دوست ما راست
&&
درویش خورد زر غنی را
\\
آن مریم دردمند یابد
&&
تازه رطب تر جنی را
\\
تا دیده غیر برنیفتد
&&
منمای به خلق محسنی را
\\
ز ایمان اگرت مراد امنست
&&
در عزلت جوی ایمنی را
\\
عزلت گه چیست خانه دل
&&
در دل خو گیر ساکنی را
\\
در خانه دل همی‌رسانند
&&
آن ساغر باقی هنی را
\\
خامش کن و فن خامشی گیر
&&
بگذار تو لاف پرفنی را
\\
زیرا که دلست جای ایمان
&&
در دل می‌دارمؤمنی را
\\
\end{longtable}
\end{center}
