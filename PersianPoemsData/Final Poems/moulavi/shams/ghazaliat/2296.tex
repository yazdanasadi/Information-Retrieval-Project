\begin{center}
\section*{غزل شماره ۲۲۹۶: مرا گویی که چونی تو لطیف و لمتر و تازه}
\label{sec:2296}
\addcontentsline{toc}{section}{\nameref{sec:2296}}
\begin{longtable}{l p{0.5cm} r}
مرا گویی که چونی تو لطیف و لمتر و تازه
&&
مثال حسن و احسانت برون از حد و اندازه
\\
خوش آن باشد که می‌راند به سوی اصل شیرینی
&&
در آن سیران سقط کرده هزاران اسب و جمازه
\\
همی‌کوشم به خاموشی ولیکن از شکرنوشی
&&
شدم همخوی آن غمزه که آن غمزه‌ست غمازه
\\
دلا سرسخت و پاسستی چنین باشند در مستی
&&
ولی بشتاب لنگانه که می‌بندند دروازه
\\
بدان صبح نجاتی رو بدان بحر حیاتی رو
&&
بزن سنگی بر این کوزه بزن نفطی در آن کازه
\\
بهل می را به میخواران بهل تب را به غمخواران
&&
که این را جملگی نقش است و آن را جمله آوازه
\\
که کنزا کنت مخفیا فاحببت بان اعرف
&&
برای جان مشتاقان به رغم نفس طنازه
\\
تعالوا یا موالینا الی اعلی معالینا
&&
فان الجسم کالاعمی و ان الحس عکازه
\\
الی نور هو الله تری فی ضؤ لقیاه
&&
کمال البدر نقصانا و عین الشمس خبازه
\\
\end{longtable}
\end{center}
