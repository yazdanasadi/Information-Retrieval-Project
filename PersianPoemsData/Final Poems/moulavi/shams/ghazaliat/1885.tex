\begin{center}
\section*{غزل شماره ۱۸۸۵: ای سرده صد سودا دستار چنین می کن}
\label{sec:1885}
\addcontentsline{toc}{section}{\nameref{sec:1885}}
\begin{longtable}{l p{0.5cm} r}
ای سرده صد سودا دستار چنین می کن
&&
خوب است همین شیوه ای دوست همین می کن
\\
فرمانده خوبانی ابرو چو بجنبانی
&&
این بنده تو را گوید آن می کن و این می کن
\\
از خون مسلمانان در ساغر رهبان کن
&&
وز کافر زلفینت ویرانی دین می کن
\\
مأمون امین را تو می ران که رو ای خاین
&&
وان غیرت رهزن را بر روح امین می کن
\\
آن حکم که از هیبت در عرش نمی‌گنجد
&&
بر پشت زمان می نه بر روی زمین می کن
\\
آن را که ندارد جان جان ده به دم عیسی
&&
وان را که ندارد زر ز اکسیر زرین می کن
\\
تا دور ابد شاها شمس الحق تبریزی
&&
حکمی است به دور تو آری هله هین می کن
\\
\end{longtable}
\end{center}
