\begin{center}
\section*{غزل شماره ۱۸۰۳: بخت نگار و چشم من هر دو نخسبد در زمن}
\label{sec:1803}
\addcontentsline{toc}{section}{\nameref{sec:1803}}
\begin{longtable}{l p{0.5cm} r}
بخت نگار و چشم من هر دو نخسبد در زمن
&&
ای نقش او شمع جهان ای چشم من او را لگن
\\
چشم و دماغ از عشق تو بی‌خواب و خور پرورده شد
&&
چون سرو و گل هر دو خورند از آب لطفت بی‌دهن
\\
ای کار جان پاک از عبث روزی جان پاک از حدث
&&
هر لحظه زاید صورتی در شهر جان بی‌مرد و زن
\\
هر صورتی به از قمر شیرینتر از شهد و شکر
&&
با صد هزاران کر و فر در خدمت معشوق من
\\
حیران ملک در رویشان آب فلک در جویشان
&&
ای دل چو اندر کویشان مست آمدی دستی بزن
\\
زان ماه روی مه جبین شد چون فلک روی زمین
&&
المستغاث ای مسلمین زین نقش‌های پرفتن
\\
\end{longtable}
\end{center}
