\begin{center}
\section*{غزل شماره ۲۵۹۴: از مرگ چه اندیشی چون جان بقا داری}
\label{sec:2594}
\addcontentsline{toc}{section}{\nameref{sec:2594}}
\begin{longtable}{l p{0.5cm} r}
از مرگ چه اندیشی چون جان بقا داری
&&
در گور کجا گنجی چون نور خدا داری
\\
خوش باش کز آن گوهر عالم همه شد چون زر
&&
ماننده آن دلبر بنما که کجا داری
\\
در عشق نشسته تن در عشرت تا گردن
&&
تو روی ترش با من ای خواجه چرا داری
\\
در عالم بی‌رنگی مستی بود و شنگی
&&
شیخا تو چو دلتنگی با غم چه هواداری
\\
چندین بمخور این غم تا چند نهی ماتم
&&
همرنگ شو آخر هم گر بخشش ما داری
\\
از تابش تو جانا جان گشت چنین دانا
&&
بسم الله مولانا چون ساغرها داری
\\
شمس الحق تبریزی چون صاف شکرریزی
&&
با تیره نیامیزی چون بحر صفا داری
\\
\end{longtable}
\end{center}
