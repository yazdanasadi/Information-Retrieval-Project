\begin{center}
\section*{غزل شماره ۳۰۸۷: اگر تو همره بلبل ز بهر گلزاری}
\label{sec:3087}
\addcontentsline{toc}{section}{\nameref{sec:3087}}
\begin{longtable}{l p{0.5cm} r}
اگر تو همره بلبل ز بهر گلزاری
&&
تو خار را همه گل بین چو بهر گل زاری
\\
نمی‌شناسی باشد که خار گل باشد
&&
اگر چه می خلدت عاقبت کند یاری
\\
درون خار گلست و برون خار گلست
&&
به احتیاط نگر تا سر کی می‌خاری
\\
چه احتیاط مرا عقل و احتیاط نماند
&&
تو احتیاط کن آخر که مرد هشیاری
\\
غلط تو هم نتوانی نگاه داشت مرا
&&
عجب ز شمع تو پروانه را نگه داری
\\
خوشست تلخی دارو و سیلی استاد
&&
غنیمتست ز یار وفا جفاکاری
\\
به دست دلبر اگر عاشقی زبون باشد
&&
ز عشق و عقل ویست آن نه از سبکساری
\\
به غیر ناز و جفا هر چه می‌کند معشوق
&&
مباش ایمن کان فتنه است و طراری
\\
زبون و دستخوش و عشوه می خوریم ای عشق
&&
اگر دروغ فروشی و گر محال آری
\\
دروغ و عشوه و صدق و محال او حالست
&&
ولیک غیر نبیند به چشم اغیاری
\\
\end{longtable}
\end{center}
