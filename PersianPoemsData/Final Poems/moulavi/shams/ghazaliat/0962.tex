\begin{center}
\section*{غزل شماره ۹۶۲: سحر این دل من ز سودا چه می‌شد}
\label{sec:0962}
\addcontentsline{toc}{section}{\nameref{sec:0962}}
\begin{longtable}{l p{0.5cm} r}
سحر این دل من ز سودا چه می‌شد
&&
از آن برق رخسار و سیما چه می‌شد
\\
از آن طلعت خوش و زان آب و آتش
&&
ز فرق سر بنده تا پا چه می‌شد
\\
خدایا تو دانی که بر ما چه آمد
&&
خدایا تو دانی که ما را چه می‌شد
\\
ز ریحان و گل‌ها که روید ز دل‌ها
&&
سراسر همه دشت و صحرا چه می‌شد
\\
ز خورشید پرسی که گردون چه سان بد
&&
ز مه پرس باری که جوزا چه می‌شد
\\
ز معشوق اعظم به هر جان خرم
&&
به پستی چه آمد به بالا چه می‌شد
\\
تعالی تقدس چو بنمود خود را
&&
مقدس دلی از تعالی چه می‌شد
\\
چو می‌کرد بخشش نظر شمس تبریز
&&
به بینا چه بخشید و بینا چه می‌شد
\\
\end{longtable}
\end{center}
