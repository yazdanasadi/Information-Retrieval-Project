\begin{center}
\section*{غزل شماره ۲۸۵۲: چو یقین شده‌ست دل را که تو جان جان جانی}
\label{sec:2852}
\addcontentsline{toc}{section}{\nameref{sec:2852}}
\begin{longtable}{l p{0.5cm} r}
چو یقین شده‌ست دل را که تو جان جان جانی
&&
بگشا در عنایت که ستون صد جهانی
\\
چو فراق گشت سرکش بزنی تو گردنش خوش
&&
به قصاص عاشقانت که تو صارم زمانی
\\
چو وصال گشت لاغر تو بپرورش به ساغر
&&
همه چیز را به پیشت خورشی است رایگانی
\\
به حمل رسید آخر به سعادت آفتابت
&&
که جهان پیر یابد ز تو تابش جوانی
\\
چه سماع‌هاست در جان چه قرابه‌های ریزان
&&
که به گوش می‌رسد زان دف و بربط و اغانی
\\
چه پر است این گلستان ز دم هزاردستان
&&
که ز های و هوی مستان تو می از قدح ندانی
\\
همه شاخه‌ها شکفته ملکان قدح گرفته
&&
همگان ز خویش رفته به شراب آسمانی
\\
برسان سلام جانم تو بدان شهان ولیکن
&&
تو کسی به هش نیابی که سلامشان رسانی
\\
پشه نیز باده خورده سر و ریش یاوه کرده
&&
نمرود را به دشنه ز وجود کرده فانی
\\
چو به پشه این رساند تو بگو به پیل چه دهد
&&
چه کنم به شرح ناید می جام لامکانی
\\
ز شراب جان پذیرش سگ کهف شیرگیرش
&&
که به گرد غار مستان نکند به جز شبانی
\\
چو سگی چنین ز خود شد تو ببین که شیر شرزه
&&
چو وفا کند چه یابد ز رحیق آن اوانی
\\
تبریز مشرقی شد به طلوع شمس دینی
&&
که از او رسد شرارت به کواکب معانی
\\
\end{longtable}
\end{center}
