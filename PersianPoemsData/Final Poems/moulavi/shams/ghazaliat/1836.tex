\begin{center}
\section*{غزل شماره ۱۸۳۶: باز بهار می کشد زندگی از بهار من}
\label{sec:1836}
\addcontentsline{toc}{section}{\nameref{sec:1836}}
\begin{longtable}{l p{0.5cm} r}
باز بهار می کشد زندگی از بهار من
&&
مجلس و بزم می نهد تا شکند خمار من
\\
من دل پردلان بدم قوت صابران بدم
&&
برد هوای دلبری هم دل و هم قرار من
\\
تند نمود عشق او تیز شدم ز تندیش
&&
گفت برو ندیده‌ای تیزی ذوالفقار من
\\
از قدم درشت او نرم شده‌ست گردنم
&&
تا چه کشد دگر از او گردن نرمسار من
\\
پخته نجوشد ای صنم جوش مده که پخته‌ام
&&
کز سر دیگ می رود تا به فلک بخار من
\\
هین که بخار خون من باخبر است از غمت
&&
تا نبرد به آسمان راز دل نزار من
\\
روح گریخت پیش تو از تن همچو دوزخم
&&
شرم بریخت پیش تو دیده شرمسار من
\\
\end{longtable}
\end{center}
