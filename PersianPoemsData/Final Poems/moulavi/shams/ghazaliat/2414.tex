\begin{center}
\section*{غزل شماره ۲۴۱۴: ز لقمه‌ای که بشد دیده تو را پرده}
\label{sec:2414}
\addcontentsline{toc}{section}{\nameref{sec:2414}}
\begin{longtable}{l p{0.5cm} r}
ز لقمه‌ای که بشد دیده تو را پرده
&&
مخور تو بیش که ضایع کنی سراپرده
\\
حیات خویش در آن لقمه گر چه پنداری
&&
ضمیر را سبل است آن و دیده را پرده
\\
چرا مکن تو در این جا مگو چرا نکنم
&&
که چشم جان را گشته است این چرا پرده
\\
طلسم تن که ز هر زهر شهد بنموده‌ست
&&
عروس پرده نموده‌ست مر تو را پرده
\\
چو لقمه را ببریدی خیال پیش آید
&&
خیال‌هاست شده بر در صفا پرده
\\
خیال طبع به روی خیال روح آید
&&
ز عقل نعره برآید که جان فزا پرده
\\
دلا جدا شو از این پرده‌های گوناگون
&&
هلا که تا نکند مر تو را جدا پرده
\\
\end{longtable}
\end{center}
