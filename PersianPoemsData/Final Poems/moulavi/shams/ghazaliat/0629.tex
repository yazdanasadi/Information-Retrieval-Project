\begin{center}
\section*{غزل شماره ۶۲۹: عاشق چو منی باید می‌سوزد و می‌سازد}
\label{sec:0629}
\addcontentsline{toc}{section}{\nameref{sec:0629}}
\begin{longtable}{l p{0.5cm} r}
عاشق چو منی باید می‌سوزد و می‌سازد
&&
ور نی مثل کودک تا کعب همی‌بازد
\\
مه رو چو تویی باید ای ماه غلام تو
&&
تا بر همه مه رویان می‌چربد و می‌نازد
\\
عاشق چو منی باید کز مستی و بی‌خویشی
&&
با خلق نپیوندد با خویش نپردازد
\\
فارس چو تویی باید ای شاه سوار من
&&
کز وهم و گمان زان سو می‌راند و می‌تازد
\\
عشق آب حیات آمد برهاندت از مردن
&&
ای شاه که او خود را در عشق دراندازد
\\
چون شاخ زرست این جان می‌کش به خودش می‌دان
&&
چندان که کشش بیند سوی تو همی‌یازد
\\
باری دل و جان من مستست در آن معدن
&&
هر روز چو نوعشقان فرهنگ نو آغازد
\\
چون چنگ شوی از غم خم داده وانگه او
&&
در بر کشدت شیرین بی‌واسطه بنوازد
\\
آن آهوی مفتونش چون تازه شود خونش
&&
آن شیر بدان آهو در میمنه بگرازد
\\
شمس الحق تبریزی بر شمس فلک روزی
&&
باشد که طراز نو شعشاع تو بطرازد
\\
\end{longtable}
\end{center}
