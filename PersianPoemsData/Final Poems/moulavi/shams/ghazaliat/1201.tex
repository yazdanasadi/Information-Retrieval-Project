\begin{center}
\section*{غزل شماره ۱۲۰۱: برای عاشق و دزدست شب فراخ و دراز}
\label{sec:1201}
\addcontentsline{toc}{section}{\nameref{sec:1201}}
\begin{longtable}{l p{0.5cm} r}
برای عاشق و دزدست شب فراخ و دراز
&&
هلا بیا شب لولی و کار هر دو بساز
\\
من از خزینه سلطان عقیق و در دزدم
&&
نیم خسیس که دزدم قماشه بزاز
\\
درون پرده شب‌ها لطیف دزدانند
&&
که ره برند به حیلت به بام خانه راز
\\
طمع ندارم از شب روی و عیاری
&&
بجز خزینه شاه و عقیق آن شه ناز
\\
رخی که از کر و فرش نماند شب به جهان
&&
زهی چراغ که خورشید سوزی و مه ساز
\\
روا شود همه حاجات خلق در شب قدر
&&
که قدر از چو تو بدری بیافت آن اعزاز
\\
همه تویی و ورای همه دگر چه بود
&&
که تا خیال درآید کسی تو را انباز
\\
هلا گذر کن از این پهن گوش‌ها بگشا
&&
که من حکایت نادر همه کنم آغاز
\\
مسیح را چو ندیدی فسون او بشنو
&&
بپر چو باز سفیدی به سوی طبلک باز
\\
چو نقده زر سرخی تو مهر شه بپذیر
&&
اگر نه تو زر سرخی چراست چندین گاز
\\
تو آن زمان که شدی گنج این ندانستی
&&
که هر کجا که بود گنج سر کند غماز
\\
بیار گنج و مکن حیله که نخواهی رست
&&
به تف تف و به مصلا و ذکر و زهد و نماز
\\
بدزدی و بنشینی به گوشه مسجد
&&
که من جنید زمانم ابایزید نیاز
\\
قماش بازده آن گاه زهد خود می‌کن
&&
مکن بهانه ضعف و فرومکش آواز
\\
خموش کن ز بهانه که حبه‌ای نخرند
&&
در این مقام ز تزویر و حیله طناز
\\
بگیر دامن اقبال شمس تبریزی
&&
که تا کمال تو یابد ز آستینش طراز
\\
\end{longtable}
\end{center}
