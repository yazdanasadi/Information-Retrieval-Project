\begin{center}
\section*{غزل شماره ۵۹: تو از خواری همی‌نالی نمی‌بینی عنایت‌ها}
\label{sec:0059}
\addcontentsline{toc}{section}{\nameref{sec:0059}}
\begin{longtable}{l p{0.5cm} r}
تو از خواری همی‌نالی نمی‌بینی عنایت‌ها
&&
مخواه از حق عنایت‌ها و یا کم کن شکایت‌ها
\\
تو را عزت همی‌باید که آن فرعون را شاید
&&
بده آن عشق و بستان تو چو فرعون این ولایت‌ها
\\
خنک جانی که خواری را به جان ز اول نهد بر سر
&&
پی اومید آن بختی که هست اندر نهایت‌ها
\\
دهان پرپست می‌خواهی مزن سرنای دولت را
&&
نتاند خواندن مقری دهان پرپست آیت‌ها
\\
ازان دریا هزاران شاخ شد هر سوی و جویی شد
&&
به باغ جان هر خلقی کند آن جو کفایت‌ها
\\
دلا منگر به هر شاخی که در تنگی فرومانی
&&
به اول بنگر و آخر که جمع آیند غایت‌ها
\\
اگر خوکی فتد در مشک و آدم زاد در سرگین
&&
رود هر یک به اصل خود ز ارزاق و کفایت‌ها
\\
سگ گرگین این در به ز شیران همه عالم
&&
که لاف عشق حق دارد و او داند وقایت‌ها
\\
تو بدنامی عاشق را منه با خواری دونان
&&
که هست اندر قفای او ز شاه عشق رایت‌ها
\\
چو دیگ از زر بود او را سیه رویی چه غم آرد
&&
که از جانش همی‌تابد به هر زخمی حکایت‌ها
\\
تو شادی کن ز شمس الدین تبریزی و از عشقش
&&
که از عشقش صفا یابی و از لطفش حمایت‌ها
\\
\end{longtable}
\end{center}
