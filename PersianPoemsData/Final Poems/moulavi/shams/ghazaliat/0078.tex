\begin{center}
\section*{غزل شماره ۷۸: ساقی ز شراب حق پر دار شرابی را}
\label{sec:0078}
\addcontentsline{toc}{section}{\nameref{sec:0078}}
\begin{longtable}{l p{0.5cm} r}
ساقی ز شراب حق پر دار شرابی را
&&
درده می ربانی دل‌های کبابی را
\\
کم گوی حدیث نان در مجلس مخموران
&&
جز آب نمی‌سازد مر مردم آبی را
\\
از آب و خطاب تو تن گشت خراب تو
&&
آراسته دار ای جان زین گنج خرابی را
\\
گلزار کند عشقت آن شوره خاکی را
&&
دربار کند موجت این جسم سحابی را
\\
بفزای شراب ما بربند تو خواب ما
&&
از شب چه خبر باشد مر مردم خوابی را
\\
همکاسه ملک باشد مهمان خدایی را
&&
باده ز فلک آید مردان ثوابی را
\\
نوشد لب صدیقش ز اکواب و اباریقش
&&
در خم تقی یابی آن باده نابی را
\\
هشیار کجا داند بی‌هوشی مستان را
&&
بوجهل کجا داند احوال صحابی را
\\
استاد خدا آمد بی‌واسطه صوفی را
&&
استاد کتاب آمد صابی و کتابی را
\\
چون محرم حق گشتی وز واسطه بگذشتی
&&
بربای نقاب از رخ خوبان نقابی را
\\
منکر که ز نومیدی گوید که نیابی این
&&
بنده ره او سازد آن گفت نیابی را
\\
نی باز سپیدست او نی بلبل خوش نغمه
&&
ویرانه دنیا به آن جغد غرابی را
\\
خاموش و مگو دیگر مفزای تو شور و شر
&&
کز غیب خطاب آید جان‌های خطابی را
\\
\end{longtable}
\end{center}
