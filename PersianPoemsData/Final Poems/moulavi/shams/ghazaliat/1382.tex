\begin{center}
\section*{غزل شماره ۱۳۸۲: ای ساقی روشن دلان بردار سغراق کرم}
\label{sec:1382}
\addcontentsline{toc}{section}{\nameref{sec:1382}}
\begin{longtable}{l p{0.5cm} r}
ای ساقی روشن دلان بردار سغراق کرم
&&
کز بهر این آورده‌ای ما را ز صحرای عدم
\\
تا جان ز فکرت بگذرد وین پرده‌ها را بردرد
&&
زیرا که فکرت جان خورد جان را کند هر لحظه کم
\\
ای دل خموش از قال او واقف نه‌ای ز احوال او
&&
بر رخ نداری خال او گر چون مهی ای جان عم
\\
خوبی جمال عالمان وان حال حال عارفان
&&
کو دیده کو دانش بگو کو گلستان کو بوی و شم
\\
زان می که او سرکه شود زو ترش رویی کی رود
&&
این می مجو آن می بجو کو جام غم کو جام جم
\\
آن می بیار ای خوبرو کاشکوفه‌اش حکمت بود
&&
کز بحر جان دارد مدد تا درج در شد زو شکم
\\
بر ریز آن رطل گران بر آه سرد منکران
&&
تا سردشان سوزان شود گردد همه لاشان نعم
\\
گر مجسم خالی بدی گفتار من عالی بدی
&&
یا نور شو یا دور شو بر ما مکن چندین ستم
\\
مانند درد دیده‌ای بر دیده برچفسیده‌ای
&&
ای خواجه برگردان ورق ور نه شکستم من قلم
\\
هر کس که هایی می کند آخر ز جایی می کند
&&
شاهی بود یا لشکری تنها نباشد آن علم
\\
خالی نمی‌گردد وطن خالی کن این تن را ز من
&&
مستست جان در آب و گل ترسم که درلغزد قدم
\\
ای شمس تبریزی ببین ما را تو این نعم المعین
&&
ای قوت پا در روش وی صحت جان در سقم
\\
\end{longtable}
\end{center}
