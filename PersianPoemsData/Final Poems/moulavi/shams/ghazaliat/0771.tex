\begin{center}
\section*{غزل شماره ۷۷۱: هله عاشقان بکوشید که چو جسم و جان نماند}
\label{sec:0771}
\addcontentsline{toc}{section}{\nameref{sec:0771}}
\begin{longtable}{l p{0.5cm} r}
هله عاشقان بکوشید که چو جسم و جان نماند
&&
دلتان به چرخ پرد چو بدن گران نماند
\\
دل و جان به آب حکمت ز غبارها بشویید
&&
هله تا دو چشم حسرت سوی خاکدان نماند
\\
نه که هر چه در جهانست نه که عشق جان آنست
&&
جز عشق هر چه بینی همه جاودان نماند
\\
عدم تو همچو مشرق اجل تو همچو مغرب
&&
سوی آسمان دیگر که به آسمان نماند
\\
ره آسمان درونست پر عشق را بجنبان
&&
پر عشق چون قوی شد غم نردبان نماند
\\
تو مبین جهان ز بیرون که جهان درون دیده‌ست
&&
چو دو دیده را ببستی ز جهان جهان نماند
\\
دل تو مثال بامست و حواس ناودان‌ها
&&
تو ز بام آب می‌خور که چو ناودان نماند
\\
تو ز لوح دل فروخوان به تمامی این غزل را
&&
منگر تو در زبانم که لب و زبان نماند
\\
تن آدمی کمان و نفس و سخن چو تیرش
&&
چو برفت تیر و ترکش عمل کمان نماند
\\
\end{longtable}
\end{center}
