\begin{center}
\section*{غزل شماره ۲۴۶۸: نیست بجز دوام جان ز اهل دلان روایتی}
\label{sec:2468}
\addcontentsline{toc}{section}{\nameref{sec:2468}}
\begin{longtable}{l p{0.5cm} r}
نیست به جز دوام جان ز اهل دلان روایتی
&&
راحت‌های عشق را نیست چو عشق غایتی
\\
شکر شنیدم از همه تا چه خوشند این رمه
&&
هان مپذیر دمدمه ز آنک کند شکایتی
\\
عشق مه است جمله رو ماه حسد برد بدو
&&
جز که ندای ابشروا این است ورا قرائتی
\\
هر سحری حلاوتی هر طرفی طراوتی
&&
هر قدمی عجایبی هر نفسی عنایتی
\\
خوبی جان چو شد ز حد و آن مدد است بر مدد
&&
هست برای چشم بد نیک بلا حمایتی
\\
پشت فلک ز جست و جو گشته چو عاشقان دوتو
&&
ز آنک جمال حسن هو نادره است و آیتی
\\
پرتو روی عشق دان آنک به هر سحرگهان
&&
شمس کشید نیزه‌ای صبح فراشت رایتی
\\
عشق چو رهنمون کند روح در او سکون کند
&&
سر ز فلک برون کند گوید خوش ولایتی
\\
ایزد گفت عشق را گر نبدی جمال تو
&&
آینه وجود را کی کنمی رعایتی
\\
گر چه که میوه آخر است ور چه درخت اول است
&&
میوه ز روی مرتبت داشت بر او بدایتی
\\
چند بود بیان تو بیش مگو به جان تو
&&
هست دل از زبان تو در غم و در نکایتی
\\
خلوتیان گریخته نقل سکوت ریخته
&&
ز آنک سکوت مست را هست قوی وقایتی
\\
گر چه نوای بلبلان هست دوای بی‌دلان
&&
خامش تا دهد تو را عشق جز این جرایتی
\\
\end{longtable}
\end{center}
