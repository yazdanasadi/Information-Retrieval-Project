\begin{center}
\section*{غزل شماره ۱۳۴: ساقیا در نوش آور شیره عنقود را}
\label{sec:0134}
\addcontentsline{toc}{section}{\nameref{sec:0134}}
\begin{longtable}{l p{0.5cm} r}
ساقیا در نوش آور شیره عنقود را
&&
در صبوح آور سبک مستان خواب آلود را
\\
یک به یک در آب افکن جمله تر و خشک را
&&
اندر آتش امتحان کن چوب را و عود را
\\
سوی شورستان روان کن شاخی از آب حیات
&&
چون گل نسرین بخندان خار غم فرسود را
\\
بلبلان را مست گردان مطربان را شیرگیر
&&
تا که درسازند با هم نغمه داوود را
\\
بادپیما بادپیمایان خود را آب ده
&&
کوری آن حرص افزون جوی کم پیمود را
\\
هم بزن بر صافیان آن درد دردانگیز را
&&
هم بخور با صوفیان پالوده بی‌دود را
\\
می میاور زان بیاور که می از وی جوش کرد
&&
آنک جوشش در وجود آورد هر موجود را
\\
زان میی کاندر جبل انداخت صد رقص الجمل
&&
زان میی کو روشنی بخشد دل مردود را
\\
هر صباحی عید داریم از تو خاصه این صبوح
&&
کز کرم بر می فشانی باده موعود را
\\
برفشان چندانک ما افشانده گردیم از وجود
&&
تا که هر قاصد بیابد در فنا مقصود را
\\
همچو آبی دیده در خود آفتاب و ماه را
&&
چون ایازی دیده در خود هستی محمود را
\\
شمس تبریزی برآر از چاه مغرب مشرقی
&&
همچو صبحی کو برآرد خنجر مغمود را
\\
\end{longtable}
\end{center}
