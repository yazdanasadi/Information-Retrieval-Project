\begin{center}
\section*{غزل شماره ۲۱۳۴: نبود چنین مه در جهان ای دل همین جا لنگ شو}
\label{sec:2134}
\addcontentsline{toc}{section}{\nameref{sec:2134}}
\begin{longtable}{l p{0.5cm} r}
نبود چنین مه در جهان ای دل همین جا لنگ شو
&&
از جنگ می‌ترسانیم گر جنگ شد گو جنگ شو
\\
ماییم مست ایزدی زان باده‌های سرمدی
&&
تو عاقلی و فاضلی دربند نام و ننگ شو
\\
رفتیم سوی شاه دین با جامه‌های کاغذین
&&
تو عاشق نقش آمدی همچون قلم در رنگ شو
\\
در عشق جانان جان بده بی‌عشق نگشاید گره
&&
ای روح این جا مست شو وی عقل این جا دنگ شو
\\
شد روم مست روی او شد زنگ مست موی او
&&
خواهی به سوی روم رو خواهی به سوی زنگ شو
\\
در دوغ او افتاده‌ای خود تو ز عشقش زاده‌ای
&&
زین بت خلاصی نیستت خواهی به صد فرسنگ شو
\\
گر کافری می‌جویدت ورمؤمنی می‌شویدت
&&
این گو برو صدیق شو و آن گو برو افرنگ شو
\\
چشم تو وقف باغ او گوش تو وقف لاغ او
&&
از دخل او چون نخل شو وز نخل او آونگ شو
\\
هم چرخ قوس تیر او هم آب در تدبیر او
&&
گر راستی رو تیر شو ور کژروی خرچنگ شو
\\
ملکی است او را زفت و خوش هر گونه ای می‌بایدش
&&
خواهی عقیق و لعل شو خواهی کلوخ و سنگ شو
\\
گر لعل و گر سنگی هلا می غلط در سیل بلا
&&
با سیل سوی بحر رو مهمان عشق شنگ شو
\\
بحری است چون آب خضر گر پر خوری نبود مضر
&&
گر آب دریا کم شود آنگه برو دلتنگ شو
\\
می‌باش همچون ماهیان در بحر آیان و روان
&&
گر یاد خشکی آیدت از بحر سوی گنگ شو
\\
گه بر لبت لب می‌نهد گه بر کنارت می‌نهد
&&
چون آن کند رو نای شو چون این کند رو چنگ شو
\\
هر چند دشمن نیستش هر سو یکی مستیستش
&&
مستان او را جام شو بر دشمنان سرهنگ شو
\\
سودای تنهایی مپز در خانه خلوت مخز
&&
شد روز عرض عاشقان پیش آ و پیش آهنگ شو
\\
آن کس بود محتاج می کو غافل است از باغ وی
&&
باغ پرانگور ویی گه باده شو گه بنگ شو
\\
خاموش همچون مریمی تا دم زند عیسی دمی
&&
کت گفت کاندر مشغله یار خران عنگ شو
\\
\end{longtable}
\end{center}
