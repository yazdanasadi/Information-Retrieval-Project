\begin{center}
\section*{غزل شماره ۲۵۴۸: تو استظهار آن داری که رو از ما بگردانی}
\label{sec:2548}
\addcontentsline{toc}{section}{\nameref{sec:2548}}
\begin{longtable}{l p{0.5cm} r}
تو استظهار آن داری که رو از ما بگردانی
&&
ولی چون کعبه برپرد کجا ماند مسلمانی
\\
تو سلطانی و جانداری تو هم آنی و آن داری
&&
مشوران مرغ جان‌ها را که ایشان را سلیمانی
\\
فلک ایمن ز هر غوغا زمین پرغارت و یغما
&&
ولیکن از فلک دارد زمین جمع و پریشانی
\\
زمین مانند تن آمد فلک چون عقل و جان آمد
&&
تن ار فربه وگر لاغر ز جان باشد همی‌دانی
\\
چو تن را عقل بگذارد پریشانی کند این تن
&&
بگوید تن که معذورم تو رفتی که نگهبانی
\\
عنایت‌های تو جان را چو عقل عقل ما آمد
&&
چو تو از عقل برگردی چه دارد عقل عقلانی
\\
شود یوسف یکی گرگی شود موسی چو فرعونی
&&
چو بیرون شد رکاب تو سرآخر گشت پالانی
\\
چو ما دستیم و تو کانی بیاور هر چه می‌آری
&&
چو ما خاکیم و تو آبی برویان هر چه رویانی
\\
تو جویایی و ناجویا چو مغناطیس ای مولا
&&
تو گویایی و ناگویا چو اسطرلاب و میزانی
\\
\end{longtable}
\end{center}
