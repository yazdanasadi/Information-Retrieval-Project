\begin{center}
\section*{غزل شماره ۱۷۹۴: ای باغبان ای باغبان آمد خزان آمد خزان}
\label{sec:1794}
\addcontentsline{toc}{section}{\nameref{sec:1794}}
\begin{longtable}{l p{0.5cm} r}
ای باغبان ای باغبان آمد خزان آمد خزان
&&
بر شاخ و برگ از درد دل بنگر نشان بنگر نشان
\\
ای باغبان هین گوش کن ناله درختان نوش کن
&&
نوحه کنان از هر طرف صد بی‌زبان صد بی‌زبان
\\
هرگز نباشد بی‌سبب گریان دو چشم و خشک لب
&&
نبود کسی بی‌درد دل رخ زعفران رخ زعفران
\\
حاصل درآمد زاغ غم در باغ و می کوبد قدم
&&
پرسان به افسوس و ستم کو گلستان کو گلستان
\\
کو سوسن و کو نسترن کو سرو و لاله و یاسمن
&&
کو سبزپوشان چمن کو ارغوان کو ارغوان
\\
کو میوه‌ها را دایگان کو شهد و شکر رایگان
&&
خشک است از شیر روان هر شیردان هر شیردان
\\
کو بلبل شیرین فنم کو فاخته کوکوزنم
&&
طاووس خوب چون صنم کو طوطیان کو طوطیان
\\
خورده چو آدم دانه‌ای افتاده از کاشانه‌ای
&&
پریده تاج و حله شان زین افتنان زین افتنان
\\
گلشن چو آدم مستضر هم نوحه گر هم منتظر
&&
چون گفتشان لا تقنطوا ذو الامتنان ذو الامتنان
\\
جمله درختان صف زده جامه سیه ماتم زده
&&
بی‌برگ و زار و نوحه گر زان امتحان زان امتحان
\\
ای لک لک و سالار ده آخر جوابی بازده
&&
در قعر رفتی یا شدی بر آسمان بر آسمان
\\
گفتند ای زاغ عدو آن آب بازآید به جو
&&
عالم شود پررنگ و بو همچون جنان همچون جنان
\\
ای زاغ بیهوده سخن سه ماه دیگر صبر کن
&&
تا دررسد کوری تو عید جهان عید جهان
\\
ز آواز اسرافیل ما روشن شود قندیل ما
&&
زنده شویم از مردن آن مهر جان آن مهر جان
\\
تا کی از این انکار و شک کان خوشی بین و نمک
&&
بر چرخ پرخون مردمک بی نردبان بی نردبان
\\
میرد خزان همچو دد بر گور او کوبی لگد
&&
نک صبح دولت می دمد ای پاسبان ای پاسبان
\\
صبحا جهان پرنور کن این هندوان را دور کن
&&
مر دهر را محرور کن افسون بخوان افسون بخوان
\\
ای آفتاب خوش عمل بازآ سوی برج حمل
&&
نی یخ گذار و نی وحل عنبرفشان عنبرفشان
\\
گلزار را پرخنده کن وان مردگان را زنده کن
&&
مر حشر را تابنده کن هین العیان هین العیان
\\
از حبس رسته دانه‌ها ما هم ز کنج خانه‌ها
&&
آورده باغ از غیب‌ها صد ارمغان صد ارمغان
\\
گلشن پر از شاهد شود هم پوستین کاسد شود
&&
زاینده و والد شود دور زمان دور زمان
\\
لک لک بیاید با یدک بر قصر عالی چون فلک
&&
لک لک کنان کالملک لک یا مستعان یا مستعان
\\
بلبل رسد بربط زنان وان فاخته کوکوکنان
&&
مرغان دیگر مطرب بخت جوان بخت جوان
\\
من زین قیامت حاملم گفت زبان را می هلم
&&
می ناید اندیشه دلم اندر زبان اندر زبان
\\
خاموش و بشنو ای پدر از باغ و مرغان نو خبر
&&
پیکان پران آمده از لامکان از لامکان
\\
\end{longtable}
\end{center}
