\begin{center}
\section*{غزل شماره ۲۴۷۲: چشم تو خواب می‌رود یا که تو ناز می‌کنی}
\label{sec:2472}
\addcontentsline{toc}{section}{\nameref{sec:2472}}
\begin{longtable}{l p{0.5cm} r}
چشم تو خواب می‌رود یا که تو ناز می‌کنی
&&
نی به خدا که از دغل چشم فراز می‌کنی
\\
چشم ببسته‌ای که تا خواب کنی حریف را
&&
چونک بخفت بر زرش دست دراز می‌کنی
\\
سلسله‌ای گشاده‌ای دام ابد نهاده‌ای
&&
بند کی سخت می‌کنی بند کی باز می‌کنی
\\
عاشق بی‌گناه را بهر ثواب می‌کشی
&&
بر سر گور کشتگان بانگ نماز می‌کنی
\\
گه به مثال ساقیان عقل ز مغز می‌بری
&&
گه به مثال مطربان نغنغه ساز می‌کنی
\\
طبل فراق می‌زنی نای عراق می‌زنی
&&
پرده بوسلیک را جفت حجاز می‌کنی
\\
جان و دل فقیر را خسته دل اسیر را
&&
از صدقات حسن خود گنج نیاز می‌کنی
\\
پرده چرخ می‌دری جلوه ملک می‌کنی
&&
تاج شهان همی‌بری ملک ایاز می‌کنی
\\
عشق منی و عشق را صورت شکل کی بود
&&
اینک به صورتی شدی این به مجاز می‌کنی
\\
گنج بلا نهایتی سکه کجاست گنج را
&&
صورت سکه گر کنی آن پی گاز می‌کنی
\\
غرق غنا شو و خمش شرم بدار چند چند
&&
در کنف غنای او ناله آز می‌کنی
\\
\end{longtable}
\end{center}
