\begin{center}
\section*{غزل شماره ۸۶۰: ای دل اگر کم آیی کارت کمال گیرد}
\label{sec:0860}
\addcontentsline{toc}{section}{\nameref{sec:0860}}
\begin{longtable}{l p{0.5cm} r}
ای دل اگر کم آیی کارت کمال گیرد
&&
مرغت شکار گردد صید حلال گیرد
\\
مه می‌دود چو آیی در ظل آفتابی
&&
بدری شود اگر چه شکل هلال گیرد
\\
در دل مقام سازد همچون خیال آن کس
&&
کاندر ره حقیقت ترک خیال گیرد
\\
کو آن خلیل گویا وجهت وجه حقا
&&
وان جان گوشمالی کو پای مال گیرد
\\
این گنده پیر دنیا چشمک زند ولیکن
&&
مر چشم روشنان را از وی ملال گیرد
\\
گر در برم کشد او از ساحری و شیوه
&&
اندر برش دل من کی پر و بال گیرد
\\
گلگونه کرده است او تا روی چون گلم را
&&
بویش تباه گردد رنگش زوال گیرد
\\
رخ بر رخش منه تو تا رویت از شهنشه
&&
مانند آفتابی نور جلال گیرد
\\
چه جای آفتابی کز پرتو جمالش
&&
صد آفتاب و مه را بر چرخ حال گیرد
\\
شویان اولینش بنگر که در چه حالند
&&
آن کاین دلیل داند نی آن دلال گیرد
\\
ای صد هزار عاقل او در جوال کرده
&&
کو عقل کاملی تا ترک جوال گیرد
\\
خطی نوشت یزدان بر خد خوش عذاران
&&
کز خط سیه‌تر است او کاین خط و خال گیرد
\\
از ابر خط برون آ وز خال و عم جدا شو
&&
تا مه ز طلعت تو هر شام فال گیرد
\\
\end{longtable}
\end{center}
