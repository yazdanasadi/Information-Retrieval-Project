\begin{center}
\section*{غزل شماره ۲۲۰۰: ای صبا بادی که داری در سر از یاری بگو}
\label{sec:2200}
\addcontentsline{toc}{section}{\nameref{sec:2200}}
\begin{longtable}{l p{0.5cm} r}
ای صبا بادی که داری در سر از یاری بگو
&&
گر نگویی با کسی با عاشقان باری بگو
\\
قصه کن در گوش ما گر دیگران محرم نیند
&&
با دل پرخون ما پیغام دلداری بگو
\\
آن مسیح حسن را دانم که می‌دانی کجاست
&&
با کسی کز عشق دارد بسته زناری بگو
\\
بانگ برزن عاشقی را کو به گل مشغول شد
&&
گو که شرمت باد از آن رخ ترک گلزاری بگو
\\
ای صبا خوش آمدی چون بازگردی سوی دوست
&&
حال من دزدیده اندر گوش عیاری بگو
\\
سوسنی با صد زبان گر حال من با او بگفت
&&
تو چو نرگس بی‌زبان از چشم اسراری بگو
\\
با چنان غیرت که جان دارد بگفتم پیش خلق
&&
شمس تبریزی بگویم گفت جان آری بگو
\\
\end{longtable}
\end{center}
