\begin{center}
\section*{غزل شماره ۲۹۹۵: اندر میان جمع چه جان است آن یکی}
\label{sec:2995}
\addcontentsline{toc}{section}{\nameref{sec:2995}}
\begin{longtable}{l p{0.5cm} r}
اندر میان جمع چه جان است آن یکی
&&
یک جان نخوانمش که جهان است آن یکی
\\
سوگند می‌خورم به جمال و کمال او
&&
کز چشم خویش هم پنهان است آن یکی
\\
بر فرق خاک آب روان کرد عشق او
&&
در باغ عشق سرو روان است آن یکی
\\
جمله شکوفه‌اند اگر میوه است او
&&
جمله قراضه‌اند چو کان است آن یکی
\\
دل موج می‌زند ز صفاتش ولی خموش
&&
زیرا فزون ز شرح و بیان است آن یکی
\\
روزی که او بزاد زمین و زمان نبود
&&
بالاتر از زمین و زمان است آن یکی
\\
قفلی است بر دهان من از رشک عاشقان
&&
تا من نگویم این که فلان است آن یکی
\\
هر دم که کنج چشمم بر روی او فتد
&&
گویم که ای خدای چه سان است آن یکی
\\
گر چشم درد نیست تو را چشم باز کن
&&
زیرا چو آفتاب عیان است آن یکی
\\
پیشش تو سجده می‌کن تا پادشا شوی
&&
زیرا که پادشاه نشان است آن یکی
\\
گر صد هزار خلق تو را رهزند که نیست
&&
اندر گمان مباش که آن است آن یکی
\\
گفتم به شمس مفخر تبریز بنگرش
&&
گفتا عجب مدار چنان است آن یکی
\\
\end{longtable}
\end{center}
