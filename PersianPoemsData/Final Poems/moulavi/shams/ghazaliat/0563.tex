\begin{center}
\section*{غزل شماره ۵۶۳: دلا نزد کسی بنشین که او از دل خبر دارد}
\label{sec:0563}
\addcontentsline{toc}{section}{\nameref{sec:0563}}
\begin{longtable}{l p{0.5cm} r}
دلا نزد کسی بنشین که او از دل خبر دارد
&&
به زیر آن درختی رو که او گل‌های تر دارد
\\
در این بازار عطاران مرو هر سو چو بی‌کاران
&&
به دکان کسی بنشین که در دکان شکر دارد
\\
ترازو گر نداری پس تو را زو رهزند هر کس
&&
یکی قلبی بیاراید تو پنداری که زر دارد
\\
تو را بر در نشاند او به طراری که می‌آید
&&
تو منشین منتظر بر در که آن خانه دو در دارد
\\
به هر دیگی که می‌جوشد میاور کاسه و منشین
&&
که هر دیگی که می‌جوشد درون چیزی دگر دارد
\\
نه هر کلکی شکر دارد نه هر زیری زبر دارد
&&
نه هر چشمی نظر دارد نه هر بحری گهر دارد
\\
بنال ای بلبل دستان ازیرا ناله مستان
&&
میان صخره و خارا اثر دارد اثر دارد
\\
بنه سر گر نمی‌گنجی که اندر چشمه سوزن
&&
اگر رشته نمی‌گنجد از آن باشد که سر دارد
\\
چراغست این دل بیدار به زیر دامنش می‌دار
&&
از این باد و هوا بگذر هوایش شور و شر دارد
\\
چو تو از باد بگذشتی مقیم چشمه‌ای گشتی
&&
حریف همدمی گشتی که آبی بر جگر دارد
\\
چو آبت بر جگر باشد درخت سبز را مانی
&&
که میوه نو دهد دایم درون دل سفر دارد
\\
\end{longtable}
\end{center}
