\begin{center}
\section*{غزل شماره ۱۴۳۹: من این ایوان نه تو را نمی‌دانم نمی‌دانم}
\label{sec:1439}
\addcontentsline{toc}{section}{\nameref{sec:1439}}
\begin{longtable}{l p{0.5cm} r}
من این ایوان نه تو را نمی‌دانم نمی‌دانم
&&
من این نقاش جادو را نمی‌دانم نمی‌دانم
\\
مرا گوید مرو هر سو تو استادی بیا این سو
&&
که من آن سوی بی‌سو را نمی‌دانم نمی‌دانم
\\
همی‌گیرد گریبانم همی‌دارد پریشانم
&&
من این خوش خوی بدخو را نمی‌دانم نمی‌دانم
\\
مرا جان طرب پیشه‌ست که بی‌مطرب نیارامد
&&
من این جان طرب جو را نمی‌دانم نمی‌دانم
\\
یکی شیری همی‌بینم جهان پیشش گله آهو
&&
که من این شیر و آهو را نمی‌دانم نمی‌دانم
\\
مرا سیلاب بربوده مرا جویای جو کرده
&&
که این سیلاب و این جو را نمی‌دانم نمی‌دانم
\\
چو طفلی گم شدستم من میان کوی و بازاری
&&
که این بازار و این کو را نمی‌دانم نمی‌دانم
\\
مرا گوید یکی مشفق بدت گویند بدگویان
&&
نکوگو را و بدگو را نمی‌دانم نمی‌دانم
\\
زمین چون زن فلک چو شو خورد فرزند چون گربه
&&
من این زن را و این شو را نمی‌دانم نمی‌دانم
\\
مرا آن صورت غیبی به ابرو نکته می گوید
&&
که غمزه چشم و ابرو را نمی‌دانم نمی‌دانم
\\
منم یعقوب و او یوسف که چشمم روشن از بویش
&&
اگر چه اصل این بو را نمی‌دانم نمی‌دانم
\\
جهان گر رو ترش دارد چو مه در روی من خندد
&&
که من جز میر مه رو را نمی‌دانم نمی‌دانم
\\
ز دست و بازوی قدرت به هر دم تیر می پرد
&&
که من آن دست و بازو را نمی‌دانم نمی‌دانم
\\
در آن مطبخ درافتادم که جان و دل کباب آمد
&&
من این گندیده تزغو را نمی‌دانم نمی‌دانم
\\
دکان نانبا دیدم که قرصش قرص ماه آمد
&&
من این نان و ترازو را نمی‌دانم نمی‌دانم
\\
چو مردان صف شکستم من به طفلی بازرستم من
&&
که این لالای لولو را نمی‌دانم نمی‌دانم
\\
تو گویی شش جهت منگر به سوی بی‌سوی برپر
&&
بیا این سو من آن سو را نمی‌دانم نمی‌دانم
\\
خمش کن چند می گویی چه قیل و قال می جویی
&&
که قیل و قال و قالو را نمی‌دانم نمی‌دانم
\\
به دستم یرلغی آمد از آن قان همه قانان
&&
که من با چو و با تو را نمی‌دانم نمی‌دانم
\\
دوایی دارم آخر من ز جالینوس پنهانی
&&
که من این درد پهلو را نمی‌دانم نمی‌دانم
\\
مرا دردی است و دارویی که جالینوس می گوید
&&
که من این درد و دارو را نمی‌دانم نمی‌دانم
\\
برو ای شب ز پیش من مپیچان زلف و گیسو را
&&
که جز آن جعد و گیسو را نمی‌دانم نمی‌دانم
\\
برو ای روز گلچهره که خورشیدت چه گلگون است
&&
که من جز نور یاهو را نمی‌دانم نمی‌دانم
\\
برو ای باغ با نقلت برو ای شیره با شیرت
&&
که جز آن نقل و طزغو را نمی‌دانم نمی‌دانم
\\
اگر صد منجنیق آید ز برج آسمان بر من
&&
بجز آن برج و بارو را نمی‌دانم نمی‌دانم
\\
چه رومی چهرگان دارم چه ترکان نهان دارم
&&
چه عیب است ار هلاوو را نمی‌دانم نمی‌دانم
\\
هلاوو را بپرس آخر از آن ترکان حیران کن
&&
کز آن حیرت هلا او را نمی‌دانم نمی‌دانم
\\
دلم چون تیر می پرد کمان تن همی‌غرد
&&
اگر آن دست و بازو را نمی‌دانم نمی‌دانم
\\
رها کن حرف هندو را ببین ترکان معنی را
&&
من آن ترکم که هندو را نمی‌دانم نمی‌دانم
\\
بیا ای شمس تبریزی مکن سنگین دلی با من
&&
که با تو سنگ و لولو را نمی‌دانم نمی‌دانم
\\
\end{longtable}
\end{center}
