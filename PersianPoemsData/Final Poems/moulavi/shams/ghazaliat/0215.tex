\begin{center}
\section*{غزل شماره ۲۱۵: من از کجا غم و شادی این جهان ز کجا}
\label{sec:0215}
\addcontentsline{toc}{section}{\nameref{sec:0215}}
\begin{longtable}{l p{0.5cm} r}
من از کجا غم و شادی این جهان ز کجا
&&
من از کجا غم باران و ناودان ز کجا
\\
چرا به عالم اصلی خویش وانروم
&&
دل از کجا و تماشای خاکدان ز کجا
\\
چو خر ندارم و خربنده نیستم ای جان
&&
من از کجا غم پالان و کودبان ز کجا
\\
هزارساله گذشتی ز عقل و وهم و گمان
&&
تو از کجا و فشارات بدگمان ز کجا
\\
تو مرغ چارپری تا بر آسمان پری
&&
تو از کجا و ره بام و نردبان ز کجا
\\
کسی تو را و تو کس را به بز نمی‌گیری
&&
تو از کجا و هیاهای هر شبان ز کجا
\\
هزار نعره ز بالای آسمان آمد
&&
تو تن زنی و نجویی که این فغان ز کجا
\\
چو آدمی به یکی مار شد برون ز بهشت
&&
میان کژدم و ماران تو را امان ز کجا
\\
دلا دلا به سررشته شو مثل بشنو
&&
که آسمان ز کجایست و ریسمان ز کجا
\\
شراب خام بیار و به پختگان درده
&&
من از کجا غم هر خام قلتبان ز کجا
\\
شرابخانه درآ و در از درون دربند
&&
تو از کجا و بد و نیک مردمان ز کجا
\\
طمع مدار که عمر تو را کران باشد
&&
صفات حقی و حق را حد و کران ز کجا
\\
اجل قفس شکند مرغ را نیازارد
&&
اجل کجا و پر مرغ جاودان ز کجا
\\
خموش باش که گفتی بسی و کس نشنید
&&
که این دهل ز چه بام‌ست و این بیان ز کجا
\\
\end{longtable}
\end{center}
