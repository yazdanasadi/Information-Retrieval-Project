\begin{center}
\section*{غزل شماره ۷۵۷: وصف آن مخدوم می‌کن گر چه می‌رنجد حسود}
\label{sec:0757}
\addcontentsline{toc}{section}{\nameref{sec:0757}}
\begin{longtable}{l p{0.5cm} r}
وصف آن مخدوم می‌کن گر چه می‌رنجد حسود
&&
کاین حسودی کم نخواهد گشت از چرخ کبود
\\
گر چه خود نیکو نیاید وصف می از هوشیار
&&
چون پی مست از خمار غمزه مستش چه سود
\\
مست آن می گر نه‌ای می دو پی دستار و دل
&&
چونک دستار و دلت را غمزه‌های او ربود
\\
گر دو صد هستیت باشد در وجودش نیست شو
&&
زانک شاید نیست گشتن از برای آن وجود
\\
نیم شب برخاستم دل را ندیدم پیش او
&&
گرد خانه جستم این دل را که او را خود چه بود
\\
چون بجستم خانه خانه یافتم بیچاره را
&&
در یکی کنجی به ناله کی خدا اندر سجود
\\
گوش بنهادم که تا خود التماس وصل کیست
&&
دیدمش کاندر پی زاری زبان را برگشود
\\
کای نهان و آشکارا آشکارا پیش تو
&&
این نهانم آتش است و آشکارم آه و دود
\\
از برای آنک خوبان را نجویی در شکست
&&
صد هزاران جوی‌ها در جوی خوبی درفزود
\\
می‌شمرد از شه نشان‌ها لیک نامش می‌نگفت
&&
در درون ظلمت شب اندر آن گفت و شنود
\\
آنگهان زیر زبان می‌گفت یارم نام او
&&
می‌نگویم گر چه نامش هست خوش بوتر ز عود
\\
زانک در وهم من آید دزدگوشی از بشر
&&
کو در این شب گوش می‌دارد حدیثم ای ودود
\\
سخت می‌آید مرا نام خوشش پیش کسی
&&
کو به عزت نشنود آن نام او را از جحود
\\
ور به عزت بشنود غیرت بسوزد مر مرا
&&
اندر این عاجز شدست او بی‌طریق و بی‌ورود
\\
بانگ کردش هاتفی تو نام آن کس یاد کن
&&
غم مخور از هیچ کس در ذکر نامش ای عنود
\\
زانک نامش هست مفتاح مراد جان تو
&&
زود نام او بگو تا در گشاید زود زود
\\
دل نمی‌یارست نامش گفتن و در بسته ماند
&&
تا سحرگه روز شد خورشید ناگه رو نمود
\\
با هزاران لابه‌هاتف همین تبریز گفت
&&
گشت بی‌هوش و فتاد این دل شکستن تار و پود
\\
چون شدم بی‌هوش آنگه نقش شد بر روی او
&&
نام آن مخدوم شمس الدین در آن دریای جود
\\
\end{longtable}
\end{center}
