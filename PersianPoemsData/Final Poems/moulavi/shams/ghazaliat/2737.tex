\begin{center}
\section*{غزل شماره ۲۷۳۷: مجلس چو چراغ و تو چو آبی}
\label{sec:2737}
\addcontentsline{toc}{section}{\nameref{sec:2737}}
\begin{longtable}{l p{0.5cm} r}
مجلس چو چراغ و تو چو آبی
&&
وز آب چراغ را خرابی
\\
خورشید بتافته‌ست بر جمع
&&
رو تو ز میان که چون سحابی
\\
بر خوان منشین که نیک خامی
&&
کو بوی کباب اگر کبابی
\\
در پیش شدی که حاجبم من
&&
والله که نه حاجبی حجابی
\\
چون حاجب باب را نشان‌هاست
&&
دانند تو را که از چه بابی
\\
گشتی تو سوار اسب چوبین
&&
از جهل به حمله می‌شتابی
\\
یا عشق گزین که هر سه نقد است
&&
یا زهد چو طالب ثوابی
\\
با بیداران نشین و برخیز
&&
کاین قافله رفت تو به خوابی
\\
از شمس الدین رسی به منزل
&&
و اندر تبریز راه یابی
\\
\end{longtable}
\end{center}
