\begin{center}
\section*{غزل شماره ۱۰۴۲: منم از جان خود بیزار بیزار}
\label{sec:1042}
\addcontentsline{toc}{section}{\nameref{sec:1042}}
\begin{longtable}{l p{0.5cm} r}
منم از جان خود بیزار بیزار
&&
اگر باشد تو را از بنده آزار
\\
مرا خود جان و دل بهر تو باید
&&
که قربان تو باشد ای نکوکار
\\
ز آزار دلت گر چه نگویی
&&
درون جان من پیداست آثار
\\
بهار از من بگردد چون ندانم
&&
چو در دل جای گلشن پر شود خار
\\
گناهم پیش لطفت سجده آرد
&&
که ای مسجود جان زنهار زنهار
\\
گنه را لطف تو گوید که تا کی
&&
گنه گوید بدو کاین بار این بار
\\
تن و جانی که خاک تو نباشد
&&
تن او سله باشد جان او مار
\\
تو خورشیدی و مرغ روز خواهی
&&
چو مرغ شب بیاید نبودش بار
\\
چو برگیری تو رسم شب ز عالم
&&
چه پرها برکند مرغ شب ای یار
\\
به حق آن که لطف تو جهانست
&&
که آن جا گم شود این چرخ دوار
\\
به چشم جان چه دریا و چه صحرا
&&
در آن عالم چه اقرار و چه انکار
\\
به تنگی درفتد هرک از تو ماند
&&
فروکن دست و او را زود بردار
\\
به قصد از شمس تبریزی نگردم
&&
چگونه زهر نوشد مرد هشیار
\\
\end{longtable}
\end{center}
