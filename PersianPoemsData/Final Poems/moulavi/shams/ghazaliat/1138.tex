\begin{center}
\section*{غزل شماره ۱۱۳۸: چه مایه رنج کشیدم ز یار تا این کار}
\label{sec:1138}
\addcontentsline{toc}{section}{\nameref{sec:1138}}
\begin{longtable}{l p{0.5cm} r}
چه مایه رنج کشیدم ز یار تا این کار
&&
بر آب دیده و خون جگر گرفت قرار
\\
هزار آتش و دود و غمست و نامش عشق
&&
هزار درد و دریغ و بلا و نامش یار
\\
هر آنک دشمن جان خودست بسم الله
&&
صلای دادن جان و صلای کشتن زار
\\
به من نگر که مرا او به صد چنین ارزد
&&
نترسم و نگریزم ز کشتن دلدار
\\
چو آب نیل دو رو دارد این شکنجه عشق
&&
به اهل خویش چو آب و به غیر او خون خوار
\\
چو عود و شمع نسوزد چه قیمتش باشد
&&
که هیچ فرق نماند ز عود و کنده خار
\\
چو زخم تیغ نباشد به جنگ و نیزه و تیر
&&
چه فرق حیز و مخنث ز رستم و جاندار
\\
به پیش رستم آن تیغ خوشتر از شکرست
&&
نثار تیر بر او لذیذتر ز نثار
\\
شکار را به دو صد ناز می‌برد این شیر
&&
شکار در هوس او دوان قطار قطار
\\
شکار کشته به خون اندرون همی‌زارد
&&
که از برای خدایم بکش تو دیگربار
\\
دو چشم کشته به زنده بدان همی‌نگرد
&&
که ای فسرده غافل بیا و گوش مخار
\\
خمش خمش که اشارات عشق معکوسست
&&
نهان شوند معانی ز گفتن بسیار
\\
\end{longtable}
\end{center}
