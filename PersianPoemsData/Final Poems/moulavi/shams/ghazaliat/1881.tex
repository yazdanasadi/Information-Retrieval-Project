\begin{center}
\section*{غزل شماره ۱۸۸۱: از چشمه جان ره شد در خانه هر مسکین}
\label{sec:1881}
\addcontentsline{toc}{section}{\nameref{sec:1881}}
\begin{longtable}{l p{0.5cm} r}
از چشمه جان ره شد در خانه هر مسکین
&&
ماننده کاریزی بی‌تیشه و بی‌میتین
\\
دل روی سوی جان کرد کای عاشق و ای پردرد
&&
بر روزن دلبر رو در خانه خود منشین
\\
ای خواجه سودایی می باش تو صحرایی
&&
در گلشن شادی رو منگر به غم غمگین
\\
چون پوست بود این دل چون آتش باشد غم
&&
وین پوست از آن آتش چون سفره بود پرچین
\\
چون دیده دل از غم پرخاک شود ای غم
&&
تبریز کجا یابی با حضرت شمس الدین
\\
\end{longtable}
\end{center}
