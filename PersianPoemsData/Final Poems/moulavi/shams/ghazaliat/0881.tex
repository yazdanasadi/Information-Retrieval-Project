\begin{center}
\section*{غزل شماره ۸۸۱: آه که بار دگر آتش در من فتاد}
\label{sec:0881}
\addcontentsline{toc}{section}{\nameref{sec:0881}}
\begin{longtable}{l p{0.5cm} r}
آه که بار دگر آتش در من فتاد
&&
وین دل دیوانه باز روی به صحرا نهاد
\\
آه که دریای عشق بار دگر موج زد
&&
وز دل من هر طرف چشمه خون برگشاد
\\
آه که جست آتشی خانه دل درگرفت
&&
دود گرفت آسمان آتش من یافت باد
\\
آتش دل سهل نیست هیچ ملامت مکن
&&
یا رب فریاد رس ز آتش دل داد داد
\\
لشکر اندیشه‌ها می‌رسد از بیشه‌ها
&&
سوی دلم طلب طلب وز غم من شاد شاد
\\
ای دل روشن ضمیر بر همه دل‌ها امیر
&&
صبر گزیدی و یافت جان تو جمله مراد
\\
چشم همه خشک و تر مانده در همدگر
&&
چشم تو سوی خداست چشم همه بر تو باد
\\
دست تو دست خدا چشم تو مست خدا
&&
بر همه پاینده باد سایه رب العباد
\\
ناله خلق از شماست آن شما از کجاست
&&
این همه از عشق زاد عشق عجب از چه زاد
\\
شمس حق دین تویی مالک ملک وجود
&&
ای که ندیده چو تو عشق دگر کیقباد
\\
\end{longtable}
\end{center}
