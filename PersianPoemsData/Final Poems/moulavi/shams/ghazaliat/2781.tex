\begin{center}
\section*{غزل شماره ۲۷۸۱: ساقیا بر خاک ما چون جرعه‌ها می‌ریختی}
\label{sec:2781}
\addcontentsline{toc}{section}{\nameref{sec:2781}}
\begin{longtable}{l p{0.5cm} r}
ساقیا بر خاک ما چون جرعه‌ها می‌ریختی
&&
گر نمی‌جستی جنون ما چرا می‌ریختی
\\
ساقیا آن لطف کو کان روز همچون آفتاب
&&
نور رقص انگیز را بر ذره‌ها می‌ریختی
\\
دست بر لب می‌نهی یعنی خمش من تن زدم
&&
خود بگوید جرعه‌ها کان بهر ما می‌ریختی
\\
ریختی خون جنید و گفت اخ هل من مزید
&&
بایزیدی بردمید از هر کجا می‌ریختی
\\
ز اولین جرعه که بر خاک آمد آدم روح یافت
&&
جبرئیلی هست شد چون بر سما می‌ریختی
\\
می‌گزیدی صادقان را تا چو رحمت مست شد
&&
از گزافه بر سزا و ناسزا می‌ریختی
\\
می‌بدادی جان به نان و نان تو را درخورد نی
&&
آب سقا می‌خریدی بر سقا می‌ریختی
\\
همچو موسی کآتشی بنمودیش وآن نور بود
&&
در لباس آتشی نور و ضیا می‌ریختی
\\
روز جمعه کی بود روزی که در جمع توییم
&&
جمع کردی آخر آن را که جدا می‌ریختی
\\
درج بد بیگانه‌ای با آشنا در هر دمم
&&
خون آن بیگانه را بر آشنا می‌ریختی
\\
ای دل آمد دلبری کاندر ملاقات خوشش
&&
همچو گل در برگ ریزان از حیا می‌ریختی
\\
آمد آن ماهی که چون ابر گران در فرقتش
&&
اشک‌ها چون مشک‌ها بهر لقا می‌ریختی
\\
دلبرا دل را ببر در آب حیوان غوطه ده
&&
آب حیوانی کز آن بر انبیا می‌ریختی
\\
انبیا عامی بدندی گر نه از انعام خاص
&&
بر مس هستی ایشان کیمیا می‌ریختی
\\
این دعا را با دعای ناکسان مقرون مکن
&&
کز برای ردشان آب دعا می‌ریختی
\\
کوشش ما را منه پهلوی کوشش‌های عام
&&
کز بقاشان می‌کشیدی در فنا می‌ریختی
\\
\end{longtable}
\end{center}
