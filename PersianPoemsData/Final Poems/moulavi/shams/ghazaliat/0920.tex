\begin{center}
\section*{غزل شماره ۹۲۰: کسی که عاشق آن رونق چمن باشد}
\label{sec:0920}
\addcontentsline{toc}{section}{\nameref{sec:0920}}
\begin{longtable}{l p{0.5cm} r}
کسی که عاشق آن رونق چمن باشد
&&
عجب مدار که در بی‌دلی چو من باشد
\\
حدیث صبر مگویید صبر را ره نیست
&&
در آن دلی که بدان یار ممتحن باشد
\\
چو عشق سلسله خویش را بجنباند
&&
جنون عقل فلاطون و بوالحسن باشد
\\
به جان عشق که جانی ز عشق جان نبرد
&&
وگر درونه صد برج و صد بدن باشد
\\
اگر چو شیر شوی عشق شیرگیر قویست
&&
وگر چه پیل شوی عشق کرکدن باشد
\\
وگر به قعر چهی درروی برای گریز
&&
چو دلو گردن از او بسته رسن باشد
\\
وگر چو موی شوی موی می‌شکافد عشق
&&
وگر کباب شوی عشق باب زن باشد
\\
امان عالم عشقست و معدلت هم از اوست
&&
وگر چه راه زن عقل مرد و زن باشد
\\
خموش کن که سخن را وطن دمشق دلست
&&
مگو غریب ورا کش چنین وطن باشد
\\
\end{longtable}
\end{center}
