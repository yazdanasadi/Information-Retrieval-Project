\begin{center}
\section*{غزل شماره ۱۰۷۶: لحظه لحظه می برون آمد ز پرده شهریار}
\label{sec:1076}
\addcontentsline{toc}{section}{\nameref{sec:1076}}
\begin{longtable}{l p{0.5cm} r}
لحظه لحظه می برون آمد ز پرده شهریار
&&
باز اندر پرده می‌شد همچنین تا هشت بار
\\
ساعتی بیرونیان را می‌ربود از عقل و دل
&&
ساعتی اهل حرم را می‌ببرد از هوش و کار
\\
دفتری از سحر مطلق پیش چشمش باز بود
&&
گردشی از گردش او در دل هر بی‌قرار
\\
گاه از نوک قلم سوداش نقشی می‌کشید
&&
گاه از سرنای عشقش عقل مسکین سنگسار
\\
چونک شب شد ز آتش رخسار شمعی برفروخت
&&
تا دو صد پروانه جان را پدید آمد مدار
\\
چون ز شب نیمی بشد مستان همه بیخود شدند
&&
ما بماندیم و شب و شمع و شراب و آن نگار
\\
مای ما هم خفته بود و برده زحمت از میان
&&
مای ما با مای او گشته کنار اندر کنار
\\
چون سحر این مای ما مشتاق آن ما گشته بود
&&
ما درآمد سایه وار و شد برون آن مای یار
\\
شمس تبریزی برفت اما شعاع روی او
&&
هر طرف نوری دهد آن را که هستش اختیار
\\
\end{longtable}
\end{center}
