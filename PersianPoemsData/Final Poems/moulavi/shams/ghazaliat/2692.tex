\begin{center}
\section*{غزل شماره ۲۶۹۲: به جان تو پس گردن نخاری}
\label{sec:2692}
\addcontentsline{toc}{section}{\nameref{sec:2692}}
\begin{longtable}{l p{0.5cm} r}
به جان تو پس گردن نخاری
&&
نگویی می‌روم عذری نیاری
\\
بسازی با دو سه مسکین بی‌دل
&&
اگر چه بی‌دلان بسیار داری
\\
نگویی کار دارم در پی کار
&&
چه باشی بسته تو خاوندگاری
\\
تو گویی می‌روم رنجور دارم
&&
نه رنجوران ما را می‌گذاری
\\
ز ما رنجورتر آخر کی باشد
&&
که در چشمت نیاییم از نزاری
\\
خوری سوگند که فردا بیایم
&&
چه دامن گیردت سوگند خواری
\\
تو با سوگند کاری پخته‌ای سر
&&
که بر اسرار پنهانی سواری
\\
تو ماهی ما شبیم از ما بمگریز
&&
که بی‌مه شب بود دلگیر و تاری
\\
تو آبی ما مثال کشت تشنه
&&
مگرد از ما که آب خوشگواری
\\
بپاش ای جان درویشان صادق
&&
چه باشد گر چنین تخمی بکاری
\\
چه درویشان که هر یک گنج ملکند
&&
که شاهان راست ز ایشان شرمساری
\\
به تو درویش و با غیر تو سلطان
&&
ز تو دارند تاج شهریاری
\\
که مه درویش باشد پیش خورشید
&&
کند بر اختران مه شهسواری
\\
منم نای تو معذورم در این بانگ
&&
که بر من هر دمی دم می‌گماری
\\
همه دم‌های این عالم شمرده‌ست
&&
تو ای دم چه دمی که بی‌شماری
\\
\end{longtable}
\end{center}
