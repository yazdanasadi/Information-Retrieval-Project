\begin{center}
\section*{غزل شماره ۱۹۷۵: عشق شمس الدین است یا نور کف موسی است آن}
\label{sec:1975}
\addcontentsline{toc}{section}{\nameref{sec:1975}}
\begin{longtable}{l p{0.5cm} r}
عشق شمس الدین است یا نور کف موسی است آن
&&
این خیال شمس دین یا خود دو صد عیسی است آن
\\
گر همه معنی است پس این چهره چون ماه چیست
&&
صورتش چون گویم آخر چون همه معنی است آن
\\
خواه این و خواه آن باری از آن فتنه لبش
&&
جان ما رقصان و خوش سرمست و سودایی است آن
\\
نیک بنگر در رخ من در فراق جان جان
&&
بی دل و جان می نویسد گر چه در انشی است آن
\\
من چه گویم خود عطارد با همه جان‌های پاک
&&
از برای پاکی او عاشق املی است آن
\\
جان من همچون عصا چون دستبوس او بیافت
&&
پس چو موسی درفکندش جان کنون افعی است آن
\\
دیده من در فراق دولت احیای او
&&
در میان خندان شده در قدرت مولی است آن
\\
هرک او اندر رکاب شاه شمس الدین دوید
&&
فارغ از دنیا و عقبی آخر و اولی است آن
\\
و آنک او بوسید دستش خود چه گویم بهر او
&&
عاقلان دانند کان خود در شرف اولی است آن
\\
جسم او چون دید جانم زود ایمان تازه کرد
&&
گفتمش چه گفت بنگر معجزه کبری است آن
\\
فر تبریز است از فر و جمال آن رخی
&&
کان غبین و حسرت صد آزر و مانی است آن
\\
\end{longtable}
\end{center}
