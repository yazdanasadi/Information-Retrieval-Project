\begin{center}
\section*{غزل شماره ۵۰۳: بر شکرت جمع مگس‌ها چراست}
\label{sec:0503}
\addcontentsline{toc}{section}{\nameref{sec:0503}}
\begin{longtable}{l p{0.5cm} r}
بر شکرت جمع مگس‌ها چراست
&&
نکته لاحول مگسران کجاست
\\
هر نظری بر رخ او راست نیست
&&
جز نظری کو ز ازل بود راست
\\
اسب خسان را به رخی پی بزن
&&
عشوه ده ای شاه که این روی ماست
\\
عشوه و عیاری و جور و دغل
&&
تو نکنی ور کنی از تو رواست
\\
از تو اگر سنگ رسد گوهرست
&&
گر تو کنی جور به از صد وفاست
\\
تیره نظر چونک ببیند دو نقش
&&
جامه درد نعره زند کاین صفاست
\\
چونک هر اندیشه خیالی گزید
&&
مجلس عشاق خیالش جداست
\\
کعبه چو از سنگ پرستان پرست
&&
روی به ما آر که قبله خداست
\\
آنک از این قبله گدایی کند
&&
در نظرش سنجر و سلطان گداست
\\
جز که به تبریز بر شمس دین
&&
روح نیاسود و نخفت و نخاست
\\
\end{longtable}
\end{center}
