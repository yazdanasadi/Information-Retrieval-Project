\begin{center}
\section*{غزل شماره ۲۴۸۶: ای زده مطرب غمت در دل ما ترانه‌ای}
\label{sec:2486}
\addcontentsline{toc}{section}{\nameref{sec:2486}}
\begin{longtable}{l p{0.5cm} r}
ای زده مطرب غمت در دل ما ترانه‌ای
&&
در سر و در دماغ جان جسته ز تو فسانه‌ای
\\
چونک خیال خوش دمت از سوی غیب دردمد
&&
ز آتش عشق برجهد تا به فلک زبانه‌ای
\\
زهره عشق چون بزد پنجه خود در آب و گل
&&
قامت ما چو چنگ شد سینه ما چغانه‌ای
\\
آهوی لنگ چون جهد از کف شیر شرزه‌ای
&&
چون برهد ز باز جان قالب چون سمانه‌ای
\\
ای گل و ای بهار جان وی می و ای خمار جان
&&
شاه و یگانه او بود کز تو خورد یگانه‌ای
\\
باغ و بهار و بخت بین عالم پردرخت بین
&&
وین همگی درخت‌ها رسته شده ز دانه‌ای
\\
از دهش و عطای تو فقر فقیر فخر شد
&&
تا که نماند مرگ را بر فقرا دهانه‌ای
\\
لطف و عطا و رحمتت طبل وصال می‌زند
&&
گر نکند وصال تو بار دگر بهانه‌ای
\\
روزه مریم مرا خوان مسیحیت نوا
&&
تر کنم از فرات تو امشب خشک نانه‌ای
\\
گشته کمان سرمدی سرده تیرهای ما
&&
گشته خدنگ احمدی فخر بنی کنانه‌ای
\\
پیش کشیی آن کمان هر کس می‌کند زهی
&&
بهر قدوم تیر تو رقعه دل نشانه‌ای
\\
جذبه حق یک رسن تافت ز آه تو و من
&&
یوسف جان ز چاه تن رفت به آشیانه‌ای
\\
خامش کن اگر سرت خارش نطق می‌دهد
&&
هست برای جعد تو صبر گزیده شانه‌ای
\\
\end{longtable}
\end{center}
