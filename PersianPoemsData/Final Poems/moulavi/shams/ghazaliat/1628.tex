\begin{center}
\section*{غزل شماره ۱۶۲۸: دیده از خلق ببستم چو جمالش دیدم}
\label{sec:1628}
\addcontentsline{toc}{section}{\nameref{sec:1628}}
\begin{longtable}{l p{0.5cm} r}
دیده از خلق ببستم چو جمالش دیدم
&&
مست بخشایش او گشتم و جان بخشیدم
\\
جهت مهر سلیمان همه تن موم شدم
&&
وز پی نور شدن موم مرا مالیدم
\\
رای او دیدم و رای کژ خود افکندم
&&
نای او گشتم و هم بر لب او نالیدم
\\
او به دست من و کورانه به دستش جستم
&&
من به دست وی و از بی‌خبران پرسیدم
\\
ساده دل بودم و یا مست و یا دیوانه
&&
ترس ترسان ز زر خویش همی‌دزدیدم
\\
از ره رخنه چو دزدان به رز خود رفتم
&&
همچو دزدان سمن از گلشن خود می چیدم
\\
بس کن و راز مرا بر سر انگشت مپیچ
&&
که من از پنجه پیچ تو بسی پیچیدم
\\
شمس تبریز که نور مه و اختر هم از اوست
&&
گر چه زارم ز غمش همچو هلال عیدم
\\
\end{longtable}
\end{center}
