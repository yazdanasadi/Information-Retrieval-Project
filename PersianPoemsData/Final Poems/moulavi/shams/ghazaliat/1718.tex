\begin{center}
\section*{غزل شماره ۱۷۱۸: ای تو ترش کرده رو تا که بترسانیم}
\label{sec:1718}
\addcontentsline{toc}{section}{\nameref{sec:1718}}
\begin{longtable}{l p{0.5cm} r}
ای تو ترش کرده رو تا که بترسانیم
&&
بسته شکرخنده را تا که بگریانیم
\\
ترش نگردم از آنک از تو همه شکرم
&&
گریه نصیب تن است من گهر جانیم
\\
در دل آتش روم تازه و خندان شوم
&&
همچو زر سرخ از آنک جمله زر کانیم
\\
در دل آتش اگر غیر تو را بنگرم
&&
دار مرا سنگسار ز آنچ من ارزانیم
\\
هیچ نشینم به عیش هیچ نخیزم به پا
&&
جز تو که برداریم جز تو که بنشانیم
\\
این دل من صورتی گشت و به من بنگرید
&&
بوسه همی‌داد دل بر سر و پیشانیم
\\
گفتم ای دل بگو خیر بود حال چیست
&&
تو نه که نوری همه من نه که ظلمانیم
\\
ور تو منی من توام خیرگی از خود ز چیست
&&
مست بخندید و گفت دل که نمی‌دانیم
\\
رو مطلب تو محال نیست زبان را مجال
&&
سوره کهفم که تو خفته فروخوانیم
\\
زود بر او درفتاد صورت من پیش دل
&&
گفت بگو راست ای صادق ربانیم
\\
گفت که این حیرت از منظر شمس حق است
&&
مفخر تبریزیان آنک در او فانیم
\\
\end{longtable}
\end{center}
