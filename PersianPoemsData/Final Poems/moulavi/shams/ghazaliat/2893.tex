\begin{center}
\section*{غزل شماره ۲۸۹۳: قدر غم گر چشم سر بگریستی}
\label{sec:2893}
\addcontentsline{toc}{section}{\nameref{sec:2893}}
\begin{longtable}{l p{0.5cm} r}
قدر غم گر چشم سر بگریستی
&&
روز و شب‌ها تا سحر بگریستی
\\
آسمان گر واقفستی زین فراق
&&
انجم و شمس و قمر بگریستی
\\
زین چنین عزلی شه ار واقف شدی
&&
بر خود و تاج و کمر بگریستی
\\
گر شب گردک بدیدی این طلاق
&&
بر کنار و بوسه بربگریستی
\\
گر شراب لعل دیدی این خمار
&&
بر قنینه و شیشه گر بگریستی
\\
گر گلستان واقفستی زین خزان
&&
برگ گل بر شاخ تر بگریستی
\\
مرغ پران واقفستی زین شکار
&&
سست کردی بال و پر بگریستی
\\
گر فلاطون را هنر نفریفتی
&&
نوحه کردی بر هنر بگریستی
\\
روزن ار واقف شدی از دود مرگ
&&
روزن و دیوار و در بگریستی
\\
کشتی اندر بحر رقصان می‌رود
&&
گر بدیدی این خطر بگریستی
\\
آتش این بوته گر ظاهر شدی
&&
محتشم بر سیم و زر بگریستی
\\
رستم ار هم واقفستی زین ستم
&&
بر مصاف و کر و فر بگریستی
\\
این اجل کر است و ناله نشنود
&&
ور نه با خون جگر بگریستی
\\
دل ندارد هیچ این جلاد مرگ
&&
ور دلش بودی حجر بگریستی
\\
گر نمودی ناخنان خویش مرگ
&&
دست و پا بر همدگر بگریستی
\\
وقت پیچاپیچ اگر حاضر شدی
&&
ماده بز بر شیر نر بگریستی
\\
مادر فرزندخوار آمد زمین
&&
ور نه بر مرگ پسر بگریستی
\\
جان شیرین دادن از تلخی مرگ
&&
گر شدی پیدا شکر بگریستی
\\
داندی مقری که عرعر می‌کند
&&
ترک کردی عر و عر بگریستی
\\
گر جنازه واقفستی زین کفن
&&
این جنازه بر گذر بگریستی
\\
کودک نوزاد می‌گرید ز نقل
&&
عاقلستی بیشتر بگریستی
\\
لیک بی‌عقلی نگرید طفل نیز
&&
ور نه چشم گاو و خر بگریستی
\\
با همه تلخی همین شیرین ما
&&
چاره دیدی چون مطر بگریستی
\\
زان که شیرین دید تلخی‌های مرگ
&&
زان چه دید آن دیده ور بگریستی
\\
که گذشت آن من و رفت آنچ رفت
&&
کو خبر تا زین خبر بگریستی
\\
تیر زهرآلود کآمد بر جگر
&&
بر سپر جستی سپر بگریستی
\\
زیر خاکم آن چنانک این جهان
&&
شاید ار زیر و زبر بگریستی
\\
هین خمش کن نیست یک صاحب نظر
&&
ور بدی صاحب نظر بگریستی
\\
شمس تبریزی برفت و کو کسی
&&
تا بر آن فخرالبشر بگریستی
\\
عالم معنی عروسی یافت زو
&&
لیک بی‌او این صور بگریستی
\\
این جهان را غیر آن سمع و بصر
&&
گر بدی سمع و بصر بگریستی
\\
\end{longtable}
\end{center}
