\begin{center}
\section*{غزل شماره ۱۱۷۰: رحم کن ار زخم شوم سر به سر}
\label{sec:1170}
\addcontentsline{toc}{section}{\nameref{sec:1170}}
\begin{longtable}{l p{0.5cm} r}
رحم کن ار زخم شوم سر به سر
&&
مرهم صبرم ده و رنجم ببر
\\
ور همه در زهر دهی غوطه‌ام
&&
زهر مرا غوطه ده اندر شکر
\\
بحر اگر تلخ بود همچو زهر
&&
هست صدف عصمت جان گهر
\\
ابر ترش رو که غم انگیز شد
&&
مژده تو دادیش ز رزق و مطر
\\
مادر اگر چه که همه رحمتست
&&
رحمت حق بین تو ز قهر پدر
\\
سرمه نو باید در چشم دل
&&
ور نه چه داند ره سرمه بصر
\\
بود به بصره به یکی کو خراب
&&
خانه درویش به عهد عمر
\\
مفلس و مسکین بد و صاحب عیال
&&
جمله آن خانه یک از یک بتر
\\
هر یک مشهور بخواهندگی
&&
خلق ز بس کدیه شان بر حذر
\\
بود لحاف شبشان ماهتاب
&&
روز طواف همشان در به در
\\
گر بکنم قصه ز ادبیرشان
&&
درد دل افزاید با درد سر
\\
شاه کریمی برسید از شکار
&&
شد سوی آن خانه ز گرد سفر
\\
در بزد از تشنگی و آب خواست
&&
آمد از آن خانه یتیمی به در
\\
گفت که هست آب ولی کوزه نیست
&&
آب یتیمان بود از چشم تر
\\
شاه در این بود که لشکر رسید
&&
همچو ستاره همه گرد قمر
\\
گفت برای دل من هر یکی
&&
در حق این قوم ببخشید زر
\\
گنج شد آن خانه ز اقبال شاه
&&
روشن و آراسته زیر و زبر
\\
ولوله و آوازه به شهر اوفتاد
&&
شهر به نظاره پی یک دگر
\\
گفت یکی کأخر ای مفلسان
&&
کشت به یک روز نیاید به بر
\\
حال شما دی همگان دیده‌اند
&&
کن فیکون کس نشود بخت ور
\\
ور بشود بخت ور آخر چنین
&&
کی شود او همچو فلک مشتهر
\\
گفت کریمی سوی بر ما گذشت
&&
کرد در این خانه به رحمت نظر
\\
قصه درازست و اشارت بس است
&&
دیده فزون دار و سخن مختصر
\\
\end{longtable}
\end{center}
