\begin{center}
\section*{غزل شماره ۲۲۵۹: هله طبل وفا بزن که بیامد اوان تو}
\label{sec:2259}
\addcontentsline{toc}{section}{\nameref{sec:2259}}
\begin{longtable}{l p{0.5cm} r}
هله طبل وفا بزن که بیامد اوان تو
&&
می چون ارغوان بده که شکفت ارغوان تو
\\
بفشاریم شیره از شکرانگور باغ تو
&&
بفشانیم میوه‌ها ز درخت جوان تو
\\
بمران جان و عقل را ز سر خوان فضل خود
&&
چه خورد یا چه کم کند مگسی دو ز خوان تو
\\
طمع جمله طامعان بود از خرمنت جوی
&&
دو ده مختصر بود دو جهان در جهان تو
\\
همه روز آفتاب اگر ز ضیا تیغ می‌زند
&&
به کم از ذره می‌شود ز نهیب سنان تو
\\
چو زمین بوس می‌کند پی تو جان آسمان
&&
به چه پر برپرد زمین به سوی آسمان تو
\\
بنشیند شکسته پر سوی تو می‌کند نظر
&&
که همین جاش می‌رسد مدد ارمغان تو
\\
نه گذشته‌ست در جهان نه شب و نی سحرگهان
&&
که دمم آتشین نشد ز دم پاسبان تو
\\
نه مرا وعده کرده‌ای نه که سوگند خورده‌ای
&&
که به هنگام برشدن برسد نردبان تو
\\
چو بدان چشم عبهری به سوی بنده بنگری
&&
بپرد جانش از مکان به سوی لامکان تو
\\
بنوازیش کای حزین مخور اندوه بعد از این
&&
که خروشید آسمان ز خروش و فغان تو
\\
منم از مادر و پدر به نوازش رحیمتر
&&
جهت پختگی تو برسید امتحان تو
\\
بکنم باغ و جنتی و دوایی ز درد تو
&&
بکنم آسمان تو به از این از دخان تو
\\
همه گفتیم و اصل را بنگفتیم دلبرا
&&
که همان به که راز تو شنوند از دهان تو
\\
\end{longtable}
\end{center}
