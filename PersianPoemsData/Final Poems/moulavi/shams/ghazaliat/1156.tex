\begin{center}
\section*{غزل شماره ۱۱۵۶: مطرب عاشقان بجنبان تار}
\label{sec:1156}
\addcontentsline{toc}{section}{\nameref{sec:1156}}
\begin{longtable}{l p{0.5cm} r}
مطرب عاشقان بجنبان تار
&&
بزن آتش به مؤمن و کفار
\\
مصلحت نیست عشق را خمشی
&&
پرده از روی مصلحت بردار
\\
تا بنگریست طفل گهواره
&&
کی دهد شیر مادر غمخوار
\\
هر چه غیر خیال معشوقست
&&
خار عشقست اگر بود گلزار
\\
مطربا چون رسی به شرح دلم
&&
پای در خون نهاده‌ای هش دار
\\
پای آهسته نه که تا نجهد
&&
چکره‌ای خون دل به هر دیوار
\\
مطربا زخم‌های دل می‌بین
&&
تا ندانند خویشتن خوش دار
\\
مطربا نام بر ز معشوقی
&&
کز دل ما ببرد صبر و قرار
\\
من چه گفتم کجا بماند دلی
&&
گر دلم کوه بود رفت از کار
\\
نام او گوی و نام من کم کن
&&
تا لقب گویمت نکوگفتار
\\
چون ز رفتار او سخن گویم
&&
دل کجا می‌رود زهی رفتار
\\
شمس تبریز عیسی عهدی
&&
هست در عهد تو چنین بیمار
\\
\end{longtable}
\end{center}
