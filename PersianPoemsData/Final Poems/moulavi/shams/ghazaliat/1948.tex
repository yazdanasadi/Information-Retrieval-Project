\begin{center}
\section*{غزل شماره ۱۹۴۸: بانگ آید هر زمانی زین رواق آبگون}
\label{sec:1948}
\addcontentsline{toc}{section}{\nameref{sec:1948}}
\begin{longtable}{l p{0.5cm} r}
بانگ آید هر زمانی زین رواق آبگون
&&
آیت انا بنیناها و انا موسعون
\\
کی شنود این بانگ را بی‌گوش ظاهر دم به دم
&&
تایبون العابدون الحامدون السایحون
\\
نردبان حاصل کنید از ذی المعارج برروید
&&
تعرج الروح الیه و الملایک اجمعون
\\
کی تراشد نردبان چرخ نجار خیال
&&
ساخت معراجش ید کل الینا راجعون
\\
تا تراشیده نگردی تو به تیشه صبر و شکر
&&
لایلقیها فرو می خوان و الاالصابرون
\\
بنگر این تیشه به دست کیست خوش تسلیم شو
&&
چون گره مستیز با تیشه که نحن الغالبون
\\
پایه‌ای چند ار برآیی باشی اصحاب الیمین
&&
ور رسی بر بام خود السابقون السابقون
\\
گر ز صوفی خانه گردونی ای صوفی برآ
&&
و اندرآ اندر صف انا لنحن الصافون
\\
ور فقیری کوس تم الفقر فهو الله بزن
&&
ور فقیهی پاک باش از انهم لا یفقهون
\\
گر چو نونی در رکوع و چون قلم اندر سجود
&&
پس تو چون نون و قلم پیوند با مایسطرون
\\
چشم شوخ سوف یبصر باش پیش از یبصرون
&&
چو مداهن نرم سازی چیست پیش یدهنون
\\
چون درخت سدره بیخ آور شو از لا ریب فیه
&&
تا نلرزد شاخ و برگت از دم ریب المنون
\\
بنگر آن باغ سیه گشته ز طاف طایف
&&
مکر ایشان باغ ایشان سوخته هم نایمون
\\
\end{longtable}
\end{center}
