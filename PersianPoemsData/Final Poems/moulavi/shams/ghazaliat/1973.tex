\begin{center}
\section*{غزل شماره ۱۹۷۳: موی بر سر شد سپید و روی من بگرفت چین}
\label{sec:1973}
\addcontentsline{toc}{section}{\nameref{sec:1973}}
\begin{longtable}{l p{0.5cm} r}
موی بر سر شد سپید و روی من بگرفت چین
&&
از فراق دلبری کاسدکن خوبان چین
\\
جان ز غیرت گوش را گوید حدیثش کم شنو
&&
دل ز غیرت چشم را گوید که رویش را مبین
\\
دست عشرت برگشادم تا ببندم پای غم
&&
عشرتم همرنگ غم شد ای مسلمانان چنین
\\
دست در سنگی زدم دانم که نرهاند مرا
&&
لیک غرقه گشته هم چنگی زند در آن و این
\\
از در دل درشدم امروز دیدم حال او
&&
زردروی و جامه چاک و بی‌یسار و بی‌یمین
\\
گفتمش چونی دلا او گریه درشدهای های
&&
از فراق ماه روی همنشان همنشین
\\
\end{longtable}
\end{center}
