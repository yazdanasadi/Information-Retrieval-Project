\begin{center}
\section*{غزل شماره ۲۱۱۸: می تلخی که تلخی‌ها بدو گردد همه شیرین}
\label{sec:2118}
\addcontentsline{toc}{section}{\nameref{sec:2118}}
\begin{longtable}{l p{0.5cm} r}
می تلخی که تلخی‌ها بدو گردد همه شیرین
&&
بت چینی که نگذارد که افتد بر رخ ما چین
\\
میش هر دم همی‌گوید که آب خضر را درکش
&&
رخش هر لحظه می‌گوید که گلزار مخلد بین
\\
زبان چرب او کرد درختانی پر از زیتون
&&
لب شیرین او خواند به افسون سوره والتین
\\
ایا من عشق خدیه یذیب الف حور العین
&&
هواه کاشف البلوی کعسق او یاسین
\\
شعاع وجهه یعلو علی شمس الضحی نورا
&&
کمال ساده الوافی یفوق الطور فی المتکین
\\
فکم من عاشق اردی مقال الحب زر غبا
&&
و کم من میت احیا محیاه کیوم الدین
\\
همی‌گوید مگو چیزی وگر نی هست تمییزی
&&
که زنده کردمی هر دم هزاران مرده زین تلقین
\\
سکوتی عند احرار غدا کشاف اسرار
&&
وراء الحرف معلوم بیان النور فی التعیین
\\
چو می‌گوید بگو حاجت دهد گوشی بدین امت
&&
که او ناگفته دریابد چو گوش غیب گو آمین
\\
سکتنا یا صبا نجد فبلغ انت ما تدری
&&
و ترجم ما کتمناه لاهل الحی حتی حین
\\
\end{longtable}
\end{center}
