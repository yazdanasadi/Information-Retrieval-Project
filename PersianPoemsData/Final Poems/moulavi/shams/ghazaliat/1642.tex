\begin{center}
\section*{غزل شماره ۱۶۴۲: ساقیا ما ز ثریا به زمین افتادیم}
\label{sec:1642}
\addcontentsline{toc}{section}{\nameref{sec:1642}}
\begin{longtable}{l p{0.5cm} r}
ساقیا ما ز ثریا به زمین افتادیم
&&
گوش خود بر دم شش تای طرب بنهادیم
\\
دل رنجور به طنبور نوایی دارد
&&
دل صدپاره خود را به نوایش دادیم
\\
به خرابات بدستیم از آن رو مستیم
&&
کوی دیگر نشناسیم در این کو زادیم
\\
ساقیا زین همه بگذر بده آن جام شراب
&&
همه را جمله یکی کن که در این افرادیم
\\
همه را غرق کن و بازرهان زین اعداد
&&
مزه‌ای بخش که ما بی‌مزه اعدادیم
\\
دل ما یافت از این باده عجایب بویی
&&
لاجرم از دم این باده لطیف اورادیم
\\
از برون خسته یاریم و درون رسته یار
&&
لاجرم مست و طربناک و قوی بنیادیم
\\
همه مستیم و خرابیم و فنای ره دوست
&&
در خرابات فنا عاقله ایجادیم
\\
هله خاموش بیارام عروسی داریم
&&
هله گردک بنشینیم که ما دامادیم
\\
\end{longtable}
\end{center}
