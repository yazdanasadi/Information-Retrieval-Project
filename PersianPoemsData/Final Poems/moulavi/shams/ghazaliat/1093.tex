\begin{center}
\section*{غزل شماره ۱۰۹۳: روستایی بچه‌ای هست درون بازار}
\label{sec:1093}
\addcontentsline{toc}{section}{\nameref{sec:1093}}
\begin{longtable}{l p{0.5cm} r}
روستایی بچه‌ای هست درون بازار
&&
دغلی لاف زنی سخره کنی بس عیار
\\
که از او محتسب و مهتر بازار بدرد
&&
در فغانند از او از فقعی تا عطار
\\
چون بگویند چرا می‌کنی این ویرانی
&&
دست کوته کن و دم درکش و شرمی می‌دار
\\
او دو صد عهد کند گوید من بس کردم
&&
توبه کردم نتراشم ز شما چون نجار
\\
بعد از این بد نکنم عاقل و هوشیار شدم
&&
که مرا زخم رسید از بد و گشتم بیدار
\\
باز در حین ببرد از بر همسایه گرو
&&
بخورد بامی و چنگی همه با خمر و خمار
\\
خویشتن را به کناری فکند رنجوری
&&
که به یک ساله تب تیز بود گشته نزار
\\
این هم از مکر که تا درفکند مسکینی
&&
که بر او رحم کند او به گمان و پندار
\\
پس بگوید که مرا مکنت چندین سیم است
&&
پیش هر کس به فلان جای و نقدی بسیار
\\
هر که زین رنج مرا باز یکی یارانه
&&
بکند در عوض آن بکنم من صد بار
\\
تا از این شیفته سر نیز تراشی بکند
&&
به طریق گرو و وام به چار و ناچار
\\
چون بداند برود خاک کند بر سر او
&&
جامه زد چاک به زنهار از این بی‌زنهار
\\
چون شود قصد که گیرند بپوشد ازرق
&&
صوفیی گردد صافی صفت بی‌آزار
\\
یک زبان دارد صد گز که به ظاهر سگزست
&&
چون به زخمش نگری باشد چاهی پرمار
\\
به گهی کز سر عشرت لطف آغاز کند
&&
شکرابت دهد او از شکر آن گفتار
\\
همه مهر و کرم و خاکی و عشق انگیزی
&&
که بجوشد دل تو وز تو رود جمله قرار
\\
و گهی از سر فضل و هنر آغاز کند
&&
که بگویی تو که لقمان زمانست به کار
\\
تا که از زهد و تقزز سخن آغاز کند
&&
سر و گردن بتراشد چو کدو یا چو خیار
\\
روزی از معرفت و فقه بسوزد ما را
&&
که بگویم که جنیدست و ز شیخان کبار
\\
چون بکاوی دغلی گنده بغل مکاری
&&
آفتی مزبله‌ای جمله شکم طبلی خوار
\\
هیچ کاری نه از او جمله شکم خواری و بس
&&
پس از آن گشت به هر مصطبه او اشکم خوار
\\
محتسب کو ز کفایت چو نظام الملکست
&&
کرد از مکر چنین کس رخ خود در دیوار
\\
زاری آغاز کند او که همه خرد و بزرگ
&&
همه یاریش کنند ار چه بدیدند یسار
\\
محتسب عقل تو است دان که صفاتت بازار
&&
وان دغل هست در او نفس پلید مکار
\\
چون همه از کف او عاجز و مسکین گشتند
&&
جمله گفتند که سحرست فن این طرار
\\
چونک سحرست نتانیم مگر یک حیله
&&
برویم از کف او نزد خداوند کبار
\\
صاحب دید و بصیرت شه ما شمس الدین
&&
که از او گشت رخ روح چو صد روی نگار
\\
چو از او داد بخواهیم از این بیدادی
&&
او به یک لحظه رهاند همه را از آزار
\\
که اگر هیبت او دیو پری نشناسد
&&
هر یکی زاهد عصری شود و اهل وقار
\\
برهندی همه از ظلمت این نفس لئیم
&&
گر از او یک نظری فضل بتابند بهار
\\
خاک تبریز که از وی چو حریم حرم است
&&
بس از او برخورد آن جان و روان زوار
\\
\end{longtable}
\end{center}
