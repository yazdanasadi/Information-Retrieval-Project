\begin{center}
\section*{غزل شماره ۲۶۳۹: برخیز که صبح است و صبوح است و سکاری}
\label{sec:2639}
\addcontentsline{toc}{section}{\nameref{sec:2639}}
\begin{longtable}{l p{0.5cm} r}
برخیز که صبح است و صبوح است و سکاری
&&
بگشای کنار آمد آن یار کناری
\\
برخیز بیا دبدبه عمر ابد بین
&&
رستند و گذشتند ز دم‌های شماری
\\
آن رفت که اقبال بخارید سر ما
&&
ای دل سر اقبال از این بار تو خاری
\\
گنجی تو عجب نیست که در توده خاکی
&&
ماهی تو عجب نیست که در گرد و غباری
\\
اندر حرم کعبه اقبال خرامید
&&
از بادیه ایمن شده وز ناز مکاری
\\
گردان شده بین چرخ که صد ماه در او هست
&&
جز تابش یک روزه تو ای چرخ چه داری
\\
آن ساغر جان که ملک الموت اجل شد
&&
نی شورش دل آرد و نی رنج خماری
\\
بس کن که اگر جان بخورد صورت ما را
&&
صد عذر بخواهد لبش از خوب عذاری
\\
\end{longtable}
\end{center}
