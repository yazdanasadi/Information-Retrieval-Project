\begin{center}
\section*{غزل شماره ۲۷۸۸: آه از آن رخسار برق انداز خوش عیاره‌ای}
\label{sec:2788}
\addcontentsline{toc}{section}{\nameref{sec:2788}}
\begin{longtable}{l p{0.5cm} r}
آه از آن رخسار برق انداز خوش عیاره‌ای
&&
صاعقه است از برق او بر جان هر بیچاره‌ای
\\
چون ز پیش رشته‌ای در لعل چون آتش بتافت
&&
موج زد دریای گوهر از میان خاره‌ای
\\
این دل صدپاره مر دربان جان را پاره داد
&&
چون به پیش پرده آمد بهترک شد پاره‌ای
\\
هشت منظر شد بهشت و هر یکی چون دفتری
&&
هشت دفتر درج بین در رقعه‌ای رخساره‌ای
\\
تا چه مرغ است این دلم چون اشتران زانو زده
&&
یا چو اشترمرغ گرد شعله آتشخواره‌ای
\\
هم دکان شد این دلم با عشقت ای کان طرب
&&
خوش حریفی یافت او هم در دکان هم کاره‌ای
\\
ز آفتاب عشق تو ذرات جان‌ها شد چو ماه
&&
وز سعادت در فلک هر ساعتی استاره‌ای
\\
نقش تو نادیده و یک یک حکایت می‌کند
&&
چون مسیح از نور مریم روح در گهواره‌ای
\\
شمس تبریزی تناقض چیست در احوال دل
&&
هم مقیم عشق باشد هم ز عشق آواره‌ای
\\
\end{longtable}
\end{center}
