\begin{center}
\section*{غزل شماره ۹۴۶: میان باغ گل سرخ‌های و هو دارد}
\label{sec:0946}
\addcontentsline{toc}{section}{\nameref{sec:0946}}
\begin{longtable}{l p{0.5cm} r}
میان باغ گل سرخ‌های و هو دارد
&&
که بو کنید دهان مرا چه بو دارد
\\
پیاله‌ای به من آورد لاله که بخوری
&&
خورم چرا نخورم بنده هم گلو دارد
\\
گلو چه حاجت می‌نوش بی‌گلو و دهان
&&
رحیق غیب که طعم سقا همو دارد
\\
چو سال سال نشاطست و روز روز طرب
&&
خنک مرا و کسی را که عیش خو دارد
\\
چرا مقیم نباشد چو ما به مجلس گل
&&
کسی که ساقی باقی ماه رو دارد
\\
به آفتاب جلالت که ذره ذره عشق
&&
نهان به زیر قبا ساغر و کدو دارد
\\
سؤال کردم از گل که بر که می‌خندی
&&
جواب داد بدان زشت کو دو شو دارد
\\
غلام کور که او را دو خواجه می‌باید
&&
چو سگ همیشه مقام او میان کو دارد
\\
سؤال کردم از خار کاین سلاح تو چیست
&&
جواب داد که گلزار صد عدو دارد
\\
هزار بار چمن را بسوخت و بازآراست
&&
چه عشق دارد با ما چه جست و جو دارد
\\
ز شمس مفخر تبریز پرس کاین از چیست
&&
وگر چه دفع دهد دم مخور که او دارد
\\
\end{longtable}
\end{center}
