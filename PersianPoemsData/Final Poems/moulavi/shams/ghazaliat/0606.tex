\begin{center}
\section*{غزل شماره ۶۰۶: با تلخی معزولی میری بنمی ارزد}
\label{sec:0606}
\addcontentsline{toc}{section}{\nameref{sec:0606}}
\begin{longtable}{l p{0.5cm} r}
با تلخی معزولی میری بنمی ارزد
&&
یک روز همی‌خندد صد سال همی‌لرزد
\\
خربندگی و آنگه از بهر خر مرده
&&
بهر گل پژمرده با خار همی‌سازد
\\
زنهار نخندی تو تا اوت نخنداند
&&
زیرا که همه خنده زین خنده همی‌خیزد
\\
ای روی ترش بنگر آن را که ترش کردت
&&
تا او شکری شیرین در سرکه درآمیزد
\\
ای خسته افتاده بنگر که که افکندت
&&
چون درنگری او را هم اوت برانگیزد
\\
گر زانک سگی خسبد بر خاک سر کویش
&&
شیر از حذر آن سگ بگدازد و بگریزد
\\
\end{longtable}
\end{center}
