\begin{center}
\section*{غزل شماره ۱۳۷۷: ای با من و پنهان چو دل از دل سلامت می کنم}
\label{sec:1377}
\addcontentsline{toc}{section}{\nameref{sec:1377}}
\begin{longtable}{l p{0.5cm} r}
ای با من و پنهان چو دل از دل سلامت می کنم
&&
تو کعبه‌ای هر جا روم قصد مقامت می کنم
\\
هر جا که هستی حاضری از دور در ما ناظری
&&
شب خانه روشن می شود چون یاد نامت می کنم
\\
گه همچو باز آشنا بر دست تو پر می زنم
&&
گه چون کبوتر پرزنان آهنگ بامت می کنم
\\
گر غایبی هر دم چرا آسیب بر دل می زنم
&&
ور حاضری پس من چرا در سینه دامت می کنم
\\
دوری به تن لیک از دلم اندر دل تو روزنیست
&&
زان روزن دزدیده من چون مه پیامت می کنم
\\
ای آفتاب از دور تو بر ما فرستی نور تو
&&
ای جان هر مهجور تو جان را غلامت می کنم
\\
من آینه دل را ز تو این جا صقالی می دهم
&&
من گوش خود را دفتر لطف کلامت می کنم
\\
در گوش تو در هوش تو و اندر دل پرجوش تو
&&
این‌ها چه باشد تو منی وین وصف عامت می کنم
\\
ای دل نه اندر ماجرا می گفت آن دلبر تو را
&&
هر چند از تو کم شود از خود تمامت می کنم
\\
ای چاره در من چاره گر حیران شو و نظاره گر
&&
بنگر کز این جمله صور این دم کدامت می کنم
\\
گه راست مانند الف گه کژ چو حرف مختلف
&&
یک لحظه پخته می شوی یک لحظه خامت می کنم
\\
گر سال‌ها ره می روی چون مهره‌ای در دست من
&&
چیزی که رامش می کنی زان چیز رامت می کنم
\\
ای شه حسام الدین حسن می گوی با جانان که من
&&
جان را غلاف معرفت بهر حسامت می کنم
\\
\end{longtable}
\end{center}
