\begin{center}
\section*{غزل شماره ۲۵۲۱: اگر یار مرا از من غم و سودا نبایستی}
\label{sec:2521}
\addcontentsline{toc}{section}{\nameref{sec:2521}}
\begin{longtable}{l p{0.5cm} r}
اگر یار مرا از من غم و سودا نبایستی
&&
مرا صد درد کان بودی مرا صد عقل و رایستی
\\
وگر کشتی رخت من نگشتی غرقه دریا
&&
فلک با جمله گوهرهاش پیش من گدایستی
\\
وگر از راه اندیشه بدین مستان رهی بودی
&&
خرد در کار عشق ما چرا بی‌دست و پایستی
\\
وگر خسرو از این شیرین یکی انگشت لیسیدی
&&
چرا قید کله بودی چرا قید قبایستی
\\
طبیب عشق اگر دادی به جالینوس یک معجون
&&
چرا بهر حشایش او بدین حد ژاژخایستی
\\
ز مستی تجلی گر سر هر کوه را بودی
&&
مثال ابر هر کوهی معلق بر هوایستی
\\
وگر غولان اندیشه همه یک گوشه رفتندی
&&
بیابان‌های بی‌مایه پر از نوش و نوایستی
\\
وگر در عهده عهدی وفایی آمدی از ما
&&
دلارام جهان پرور بر آن عهد و وفایستی
\\
وگر این گندم هستی سبکتر آرد می‌گشتی
&&
متاع هستی خلقان برون زین آسیایستی
\\
وگر خضری دراشکستی به ناگه کشتی تن را
&&
در این دریا همه جان‌ها چو ماهی آشنایستی
\\
ستایش می‌کند شاعر ملک را و اگر او را
&&
ز خویش خود خبر بودی ملک شاعر ستایستی
\\
وگر جبار بربستی شکسته ساق و دستش را
&&
نه در جبر و قدر بودی نه در خوف و رجایستی
\\
در آن اشکستگی او گر بدیدی ذوق اشکستن
&&
نه از مرهم بپرسیدی نه جویای دوایستی
\\
نشان از جان تو این داری که می‌باید نمی‌باید
&&
نمی‌باید شدی باید اگر او را ببایستی
\\
وگر از خرمن خدمت تو ده سالار منبل را
&&
یکی برگ کهی بودی گنه بر کهربایستی
\\
فراز آسمان صوفی همی‌رقصید و می‌گفت این
&&
زمین کل آسمان گشتی گرش چون من صفایستی
\\
خمش کن شعر می‌ماند و می‌پرند معنی‌ها
&&
پر از معنی بدی عالم اگر معنی بپایستی
\\
\end{longtable}
\end{center}
