\begin{center}
\section*{غزل شماره ۲۶۸۴: شنودم من که چاکر را ستودی}
\label{sec:2684}
\addcontentsline{toc}{section}{\nameref{sec:2684}}
\begin{longtable}{l p{0.5cm} r}
شنودم من که چاکر را ستودی
&&
کی باشم من تو لطف خود نمودی
\\
تو کان لعل و جان کهربایی
&&
به رحمت برگ کاهی را ربودی
\\
یکی آهن بدم بی‌قدر و قیمت
&&
توام آیینه ای کردی زدودی
\\
ز طوفان فناام واخریدی
&&
که هم نوحی و هم کشتی جودی
\\
دلا گر سوختی چون عود بوده
&&
وگر خامی بسوز اکنون که عودی
\\
به زیر سایه اقبال خفتم
&&
برون پنج حس راهم گشودی
\\
بدان ره بی‌پر و بی‌پا و بی‌سر
&&
به شرق و غرب شاید شد به زودی
\\
در آن ره نیست خار اختیاری
&&
نه ترسایی است آن جا نه جهودی
\\
برون از خطه چرخ کبودش
&&
رهیده جان ز کوری و کبودی
\\
چه می‌گریی بر خندندگان رو
&&
چه می‌پایی همان جا رو که بودی
\\
از این شهدی که صد گون نیش دارد
&&
بجز دنبل ببین چیزی فزودی
\\
\end{longtable}
\end{center}
