\begin{center}
\section*{غزل شماره ۶۳۸: ملولان همه رفتند در خانه ببندید}
\label{sec:0638}
\addcontentsline{toc}{section}{\nameref{sec:0638}}
\begin{longtable}{l p{0.5cm} r}
ملولان همه رفتند در خانه ببندید
&&
بر آن عقل ملولانه همه جمع بخندید
\\
به معراج برآیید چو از آل رسولید
&&
رخ ماه ببوسید چو بر بام بلندید
\\
چو او ماه شکافید شما ابر چرایید
&&
چو او چست و ظریفست شما چون هلپندید
\\
ملولان به چه رفتید که مردانه در این راه
&&
چو فرهاد و چو شداد دمی کوه نکندید
\\
چو مه روی نباشید ز مه روی متابید
&&
چو رنجور نباشید سر خویش مبندید
\\
چنان گشت و چنین گشت چنان راست نیاید
&&
مدانید که چونید مدانید که چندید
\\
چو آن چشمه بدیدیت چرا آب نگشتید
&&
چو آن خویش بدیدیت چرا خویش پسندید
\\
چو در کان نباتید ترش روی چرایید
&&
چو در آب حیاتید چرا خشک و نژندید
\\
چنین برمستیزید ز دولت مگریزید
&&
چه امکان گریزست که در دام کمندید
\\
گرفتار کمندید کز او هیچ امان نیست
&&
مپیچید مپیچید بر استیزه مرندید
\\
چو پروانه جانباز بسایید بر این شمع
&&
چه موقوف رفیقید چه وابسته بندید
\\
از این شمع بسوزید دل و جان بفروزید
&&
تن تازه بپوشید چو این کهنه فکندید
\\
ز روباه چه ترسید شما شیرنژادید
&&
خر لنگ چرایید چو از پشت سمندید
\\
همان یار بیاید در دولت بگشاید
&&
که آن یار کلیدست شما جمله کلندید
\\
خموشید که گفتار فروخورد شما را
&&
خریدار چو طوطیست شما شکر و قندید
\\
\end{longtable}
\end{center}
