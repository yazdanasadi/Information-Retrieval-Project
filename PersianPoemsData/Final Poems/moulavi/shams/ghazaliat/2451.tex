\begin{center}
\section*{غزل شماره ۲۴۵۱: دریوزه‌ای دارم ز تو در اقتضای آشتی}
\label{sec:2451}
\addcontentsline{toc}{section}{\nameref{sec:2451}}
\begin{longtable}{l p{0.5cm} r}
دریوزه‌ای دارم ز تو در اقتضای آشتی
&&
دی نکته‌ای فرموده‌ای جان را برای آشتی
\\
جان را نشاط و دمدمه جمله مهماتش همه
&&
کاری نمی‌بینم دگر الا نوای آشتی
\\
جان خشم گیرد با کسی گردد جهانش محبسی
&&
جان را فتد یا رب عجب با جسم رای آشتی
\\
با غیر اگر خشمین شوی گیری سر خویش و روی
&&
سر با تو چون خشمین شود آن گاه وای آشتی
\\
گر دستبوس وصل تو یابد دلم در جست و جو
&&
بس بوسه‌ها که دل دهد بر خاک پای آشتی
\\
هر نیکوی که تن کند از لطف داد جان بود
&&
من هر سخا که کرده‌ام بود آن سخای آشتی
\\
چون ابر دی گریان شدم وز برگ و بر عریان شدم
&&
خواهم که ناگه درغژم خوش در قبای آشتی
\\
سلطان و شاهنشه شوم اجری فرست مه شوم
&&
نیکولقا آنگه شود کید لقای آشتی
\\
ای جان صد باغ و چمن تشریف ده سوی وطن
&&
هر چند بدرایی من نگذاشت جای آشتی
\\
از نوبهار لم یکن این باد را تلطیف کن
&&
تا بی‌بخار غم شود از تو فضای آشتی
\\
آلایش ما چیست خود با بحر جان و جر و مد
&&
یا کبر و شیطانی ما با کبریای آشتی
\\
خاموش کن ای بی‌ادب چیزی مگو در زیر لب
&&
تا بی‌ریا باشد طلب اندر دعای آشتی
\\
\end{longtable}
\end{center}
