\begin{center}
\section*{غزل شماره ۸۶۸: به حرم به خود کشید و مرا آشنا ببرد}
\label{sec:0868}
\addcontentsline{toc}{section}{\nameref{sec:0868}}
\begin{longtable}{l p{0.5cm} r}
به حرم به خود کشید و مرا آشنا ببرد
&&
یک یک برد شما را آنک مرا ببرد
\\
آن را که بود آهن آهن ربا کشید
&&
وان را که بود برگ کهی کهربا ببرد
\\
قانون لنگری به ثری گشت منجذب
&&
عیسی مهتری را جذب سما ببرد
\\
هر حس معنوی را در غیب درکشید
&&
هر مس اسعدی را هم کیمیا ببرد
\\
از غارت فنا و اجل ایمنست و دور
&&
آن کس که رخت خویش سوی انبیا ببرد
\\
آن چشم نیک را نرسد هیچ چشم بد
&&
کو شمع حسن را ز ملاء در خلاء ببرد
\\
ما از قضا به قاضی حاجت گریختیم
&&
کنچ از قضا رسید به طالب قضا ببرد
\\
این‌ها گذشت ای خنک آن دل که ناگهش
&&
حسن و جمال آن مه نیکولقا ببرد
\\
\end{longtable}
\end{center}
