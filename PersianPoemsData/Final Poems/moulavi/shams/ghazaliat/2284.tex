\begin{center}
\section*{غزل شماره ۲۲۸۴: باده بده باد مده وز خودمان یاد مده}
\label{sec:2284}
\addcontentsline{toc}{section}{\nameref{sec:2284}}
\begin{longtable}{l p{0.5cm} r}
باده بده باد مده وز خودمان یاد مده
&&
روز نشاط است و طرب برمنشین داد مده
\\
آمده‌ام مست لقا کشته شمشیر فنا
&&
گر نه چنینم تو مرا هیچ دل شاد مده
\\
خواجه تو عارف بده‌ای نوبت دولت زده‌ای
&&
کامل جان آمده‌ای دست به استاد مده
\\
در ده ویرانه تو گنج نهان است ز هو
&&
هین ده ویران تو را نیز به بغداد مده
\\
والله تیره شب تو به ز دو صد روز نکو
&&
شب مده و روز مجو عاج به شمشاد مده
\\
غیر خدا نیست کسی در دو جهان همنفسی
&&
هر چه وجود است تو را جز که به ایجاد مده
\\
گر چه در این خیمه دری دانک تو با خیمه گری
&&
لیک طناب دل خود جز که به اوتاد مده
\\
ساقی جان صرفه مکن روز ببردی به سخن
&&
مال یتیمان بمخور دست به فریاد مده
\\
ای صنم خفته ستان در چمن و لاله ستان
&&
باده ز مستان مستان در کف آحاد مده
\\
دانه به صحرا مکشان بر سر زاغان مفشان
&&
جوهر فردیت خود هرزه به افراد مده
\\
چون بود ای دلشده چون نقد بر از کن فیکون
&&
نقد تو نقد است کنون گوش به میعاد مده
\\
هم تو تویی هم تو منم هیچ مرو از وطنم
&&
مرغ تویی چوژه منم چوزه به هر خاد مده
\\
آنک به خویش است گرو علم و فریبش مشنو
&&
هست تو را دانش نو هوش به اسناد مده
\\
خسرو جانی و جهان وز جهت کوهکنان
&&
با تو کلندی است گران جز که به فرهاد مده
\\
بس کن کاین نطق خرد جنبش طفلانه بود
&&
عارف کامل شده را سبحه عباد مده
\\
\end{longtable}
\end{center}
