\begin{center}
\section*{غزل شماره ۲۴۶۲: طوطی و طوطی بچه‌ای قند به صد ناز خوری}
\label{sec:2462}
\addcontentsline{toc}{section}{\nameref{sec:2462}}
\begin{longtable}{l p{0.5cm} r}
طوطی و طوطی بچه‌ای قند به صد ناز خوری
&&
از شکرستان ازل آمده‌ای بازپری
\\
قند تو فرخنده بود خاصه که در خنده بود
&&
بزم ز آغاز نهم چون تو به آغاز دری
\\
ای طربستان ابد ای شکرستان احد
&&
هم طرب اندر طربی هم شکر اندر شکری
\\
یوسف اندر تتقی یا اسدی بر افقی
&&
یا قمر اندر قمر اندر قمر اندر قمری
\\
ساقی این میکده‌ای نوبت عشرت زده‌ای
&&
تا همه را مست کنی خرقه مستان ببری
\\
مست شدم مست ولی اندککی باخبرم
&&
زین خبرم بازرهان ای که ز من باخبری
\\
پیشتر آ پیش که آن شعشعه چهره تو
&&
می‌نهلد تا نگرم که ملکی یا بشری
\\
رقص کنان هر قدحی نعره زنان وافرحی
&&
شیشه گران شیشه شکن مانده از شیشه گری
\\
جام طرب عام شده عقل و سرانجام شده
&&
از کف حق جام بری به که سرانجام بری
\\
سر ز خرد تافته‌ام عقل دگر یافته‌ام
&&
عقل جهان یک سری و عقل نهانی دوسری
\\
راهب آفاق شدم با همگان عاق شدم
&&
از همگان می‌ببرم تا که تو از من نبری
\\
با غمت آموخته‌ام چشم ز خود دوخته‌ام
&&
در جز تو چون نگرد آنک تو در وی نگری
\\
داد ده ای عشق مرا وز در انصاف درآ
&&
چون ابدا آن توام نی قنقم رهگذری
\\
من به تو مانم فلکا ساکنم و زیر و زبر
&&
ز آنک مقیمی به نظر روز و شب اندر سفری
\\
ناظر آنی که تو را دارد منظور جهان
&&
حاضر آنی که از او در سفر و در حضری
\\
\end{longtable}
\end{center}
