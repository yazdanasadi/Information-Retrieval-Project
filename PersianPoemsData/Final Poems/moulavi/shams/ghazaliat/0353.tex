\begin{center}
\section*{غزل شماره ۳۵۳: زهی می کاندر آن دستست هیهات}
\label{sec:0353}
\addcontentsline{toc}{section}{\nameref{sec:0353}}
\begin{longtable}{l p{0.5cm} r}
زهی می کاندر آن دستست هیهات
&&
که عقل کل بدو مستست هیهات
\\
بر آن بالا برد دل را که آن جا
&&
سر نیزه زحل پستست هیهات
\\
هر آن کو گشت بی‌خویش اندر این بزم
&&
ز خویش و اقربا رسته‌ست هیهات
\\
چو عنقا برپرد بر ذروه قاف
&&
که پیشش که کمربسته‌ست هیهات
\\
عجایب بین که شیشه ناشکسته
&&
هزاران دست و پا خسته‌ست هیهات
\\
مرا گویی که صبر آهسته‌تر ران
&&
چه جای صبر و آهسته‌ست هیهات
\\
بده آن پیر را جامی و بنشان
&&
که این جا پیر بایسته‌ست هیهات
\\
خصوصا جان پیری‌ها که عقل‌ست
&&
که خوش مغزست و شایسته‌ست هیهات
\\
از آن باغ و ریاض بی‌نهایت
&&
همه عالم چو گلدسته‌ست هیهات
\\
چو گلدسته‌ست پوسیده شود زود
&&
به دشتی رو کز او رسته‌ست هیهات
\\
میی درکش به نام دلربایی
&&
که بس زیبا و برجسته‌ست هیهات
\\
ز بس خون‌ها که او دارد به گردن
&&
خرد را طوق بسکسته‌ست هیهات
\\
شکن‌هایی که دارد طره او
&&
بهای مشک بشکسته‌ست هیهات
\\
خمش کردم خموشانه به من ده
&&
که دل را گفت پیوسته‌ست هیهات
\\
\end{longtable}
\end{center}
