\begin{center}
\section*{غزل شماره ۶۰۰: جامم بشکست ای جان پهلوش خلل دارد}
\label{sec:0600}
\addcontentsline{toc}{section}{\nameref{sec:0600}}
\begin{longtable}{l p{0.5cm} r}
جامم بشکست ای جان پهلوش خلل دارد
&&
در جمع چنین مستان جامی چه محل دارد
\\
گر بشکند این جامم من غصه نیاشامم
&&
جامی دگر آن ساقی در زیر بغل دارد
\\
جامست تن خاکی جانست می پاکی
&&
جامی دگرم بخشد کاین جام علل دارد
\\
ساقی وفاداری کز مهر کله دارد
&&
ساقی که قبای او از حلم تگل دارد
\\
شادی و فرح بخشد دل را که دژم باشد
&&
تیزی نظر بخشد گر چشم سبل دارد
\\
عقلی که بر این روزن شد حارس این خانه
&&
خاک در او گردد گر علم و عمل دارد
\\
شهمات کجا گردد آن کو رخ شه بیند
&&
کی تلخ شود آن کو دریای عسل دارد
\\
از آب حیات او آن کس که کشد گردن
&&
در عین حیات خود صد مرگ و اجل دارد
\\
خورشید به هر برجی مسعود و بهی باشد
&&
اما کر و فر خود در برج حمل دارد
\\
جز صورت عشق حق هر چیز که من دیدم
&&
نیمیش دروغ آمد نیمیش دغل دارد
\\
چندان لقبش گفتم از کامل و از ناقص
&&
از غایت بی‌مثلی صد گونه مثل دارد
\\
\end{longtable}
\end{center}
