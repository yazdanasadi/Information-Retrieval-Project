\begin{center}
\section*{غزل شماره ۲۸۵۸: صنما تو همچو آتش قدح مدام داری}
\label{sec:2858}
\addcontentsline{toc}{section}{\nameref{sec:2858}}
\begin{longtable}{l p{0.5cm} r}
صنما تو همچو آتش قدح مدام داری
&&
به جواب هر سلامی که کنند جام داری
\\
ز برای تو اگر تن دو هزار جان سپارد
&&
ز خداش وحی آید که هنوز وام داری
\\
چو حقت ز غیرت خود ز تو نیز کرد پنهان
&&
به درون جان چاکر چه پدید نام داری
\\
چو سلام تو شنیدم ز سلامتی بریدم
&&
صنما هزار آتش تو در آن سلام داری
\\
ز پی غلامی تو چو بسوخت جان شاهان
&&
به کدام روی گویم که چو من غلام داری
\\
تو هنوز روح بودی که تمام شد مرادت
&&
بجز از برای فتنه به جهان چه کام داری
\\
توریز بخت یارت به خدا که راست گویی
&&
که میان شیرمردان چو ویی کدام داری
\\
تبریز شاد بادا که ز نور و فر آن شه
&&
دو هزار بیش چاکر چو یمن چو شام داری
\\
نظر خدای خواهم که تو را به من رساند
&&
به دعا چه خواهمت من که همه تو رام داری
\\
نظر حسود مسکین طرقید از تفکر
&&
نرسید در تو هر چند که تو لطف عام داری
\\
چه حسود بلک عاشق دو هزار هر نواحی
&&
نه خیالشان نمایی نه به کس پیام داری
\\
تو خدای شمس دین را به من غلام بخشی
&&
چو غلامیی ورا تو به شهان حرام داری
\\
لقبت چو می‌بگویم دل من همی‌بلرزد
&&
تو دلا مترس زیرا که شه کرام داری
\\
\end{longtable}
\end{center}
