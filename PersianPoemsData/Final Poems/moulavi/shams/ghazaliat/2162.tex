\begin{center}
\section*{غزل شماره ۲۱۶۲: دگرباره بشوریدم بدان سانم به جان تو}
\label{sec:2162}
\addcontentsline{toc}{section}{\nameref{sec:2162}}
\begin{longtable}{l p{0.5cm} r}
دگرباره بشوریدم بدان سانم به جان تو
&&
که هر بندی که بربندی بدرانم به جان تو
\\
من آن دیوانه بندم که دیوان را همی‌بندم
&&
زبان مرغ می‌دانم سلیمانم به جان تو
\\
نخواهم عمر فانی را تویی عمر عزیز من
&&
نخواهم جان پرغم را تویی جانم به جان تو
\\
چو تو پنهان شوی از من همه تاریکی و کفرم
&&
چو تو پیدا شوی بر من مسلمانم به جان تو
\\
گر آبی خوردم از کوزه خیال تو در او دیدم
&&
وگر یک دم زدم بی‌تو پشیمانم به جان تو
\\
اگر بی‌تو بر افلاکم چو ابر تیره غمناکم
&&
وگر بی‌تو به گلزارم به زندانم به جان تو
\\
سماع گوش من نامت سماع هوش من جامت
&&
عمارت کن مرا آخر که ویرانم به جان تو
\\
درون صومعه و مسجد تویی مقصودم ای مرشد
&&
به هر سو رو بگردانی بگردانم به جان تو
\\
سخن با عشق می‌گویم که او شیر و من آهویم
&&
چه آهویم که شیران را نگهبانم به جان تو
\\
ایا منکر درون جان مکن انکارها پنهان
&&
که سر سرنبشتت را فروخوانم به جان تو
\\
چه خویشی کرد آن بی‌چون عجب با این دل پرخون
&&
که ببریده‌ست آن خویشی ز خویشانم به جان تو
\\
تو عید جان قربانی و پیشت عاشقان قربان
&&
بکش در مطبخ خویشم که قربانم به جان تو
\\
ز عشق شمس تبریزی ز بیداری و شبخیزی
&&
مثال ذره گردان پریشانم به جان تو
\\
\end{longtable}
\end{center}
