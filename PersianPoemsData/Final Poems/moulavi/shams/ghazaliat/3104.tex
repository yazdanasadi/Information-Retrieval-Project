\begin{center}
\section*{غزل شماره ۳۱۰۴: طواف کعبه دل کن اگر دلی داری}
\label{sec:3104}
\addcontentsline{toc}{section}{\nameref{sec:3104}}
\begin{longtable}{l p{0.5cm} r}
طواف کعبه دل کن اگر دلی داری
&&
دلست کعبه معنی تو گل چه پنداری
\\
طواف کعبه صورت حقت بدان فرمود
&&
که تا به واسطه آن دلی به دست آری
\\
هزار بار پیاده طواف کعبه کنی
&&
قبول حق نشود گر دلی بیازاری
\\
بده تو ملکت و مال و دلی به دست آور
&&
که دل ضیا دهدت در لحد شب تاری
\\
هزار بدره زرگر بری به حضرت حق
&&
حقت بگوید دل آر اگر به ما آری
\\
که سیم و زر بر ما لاشیست بی‌مقدار
&&
دلست مطلب ما گر مرا طلبکاری
\\
ز عرش و کرسی و لوح قلم فزون باشد
&&
دل خراب که آن را کهی بنشماری
\\
مدار خوار دلی را اگر چه خوار بود
&&
که بس عزیر عزیزست دل در آن خواری
\\
دل خراب چو منظرگه اله بود
&&
زهی سعادت جانی که کرد معماری
\\
عمارت دل بیچاره دو صدپاره
&&
ز حج و عمره به آید به حضرت باری
\\
کنوز گنج الهی دل خراب بود
&&
که در خرابه بود دفن گنج بسیاری
\\
کمر به خدمت دل‌ها ببند چاکروار
&&
که برگشاید در تو طریق اسراری
\\
گرت سعادت و اقبال گشت مطلوبت
&&
شوی تو طالب دل‌ها و کبر بگذاری
\\
چو همعنان تو گردد عنایت دل‌ها
&&
شود ینابع حکمت ز قلب تو جاری
\\
روان شود ز لسانت چو سیل آب حیات
&&
دمت بود چو مسیحا دوای بیماری
\\
برای یک دل موجود گشت هر دو جهان
&&
شنو تو نکته لولاک از لب قاری
\\
وگر نه کون و مکان را وجود کی بودی
&&
ز مهر و ماه و ز ارض و سمای زنگاری
\\
خموش وصف دل اندر بیان نمی‌گنجد
&&
اگر به هر سر مویی دو صد زبان داری
\\
\end{longtable}
\end{center}
