\begin{center}
\section*{غزل شماره ۱۵۹۷: این چه کژطبعی بود که صد هزاران غم خوریم}
\label{sec:1597}
\addcontentsline{toc}{section}{\nameref{sec:1597}}
\begin{longtable}{l p{0.5cm} r}
این چه کژطبعی بود که صد هزاران غم خوریم
&&
جمع مستان را بخوان تا باده‌ها با هم خوریم
\\
باده‌ای کابرار را دادند اندر یشربون
&&
با جنید و بایزید و شبلی و ادهم خوریم
\\
ابر نبود ماه ما را تا جفای شب کشیم
&&
مرگ نبود عاشقان را تا غم ماتم خوریم
\\
نفس ماده کیست تا ما تیغ خود بر وی زنیم
&&
زخم بر رستم زنیم و زخم از رستم خوریم
\\
بود مردم خوار عالم خلق عالم را بخورد
&&
خالق آورده‌ست ما را تا که ما عالم خوریم
\\
این جهان افسونگرست و وعده فردا دهد
&&
ما از آن زیرکتریم ای خوش پسر که دم خوریم
\\
گر پری زادیم شب جمعیت پریان بود
&&
ور ز آدم زاده‌ایم آن باده با آدم خوریم
\\
گه از آن کف گوهر هستی و سرمستی بریم
&&
گه از آن دف نعره و فریاد زیر و بم خوریم
\\
ماهییم و ساقی ما نیست جز دریای عشق
&&
هیچ دریا کم شود زان رو که بیش و کم خوریم
\\
گه چو گردون از مه و خورشید اشکم پر کنیم
&&
گر چو خورشید آب‌ها را جمله بی‌اشکم خوریم
\\
شمس تبریزی تو سلطانی و ما بنده توییم
&&
لاجرم در دور تو باده به جام جم خوریم
\\
\end{longtable}
\end{center}
