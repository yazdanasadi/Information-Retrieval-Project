\begin{center}
\section*{غزل شماره ۸۳۸: گر نخسپی شبکی جان چه شود}
\label{sec:0838}
\addcontentsline{toc}{section}{\nameref{sec:0838}}
\begin{longtable}{l p{0.5cm} r}
گر نخسپی شبکی جان چه شود
&&
ور نکوبی در هجران چه شود
\\
ور بیاری شبکی روز آری
&&
از برای دل یاران چه شود
\\
ور دو دیده به تو روشن گردد
&&
کوری دیده شیطان چه شود
\\
گر برآری ز دل بحر غبار
&&
چون کف موسی عمران چه شود
\\
ور سلیمان بر موران آید
&&
تا شود مور سلیمان چه شود
\\
ور چو الیاس قلاووز شوی
&&
تا لب چشمه حیوان چه شود
\\
ور بروید ز گل افشانی تو
&&
همه عالم گل و ریحان چه شود
\\
آب حیوان که در آن تاریکیست
&&
پر شود شهر و بیابان چه شود
\\
ور ز خوان کرم و نعمت تو
&&
زنده گردد دو سه مهمان چه شود
\\
ور ز دلداری و جان بخشی تو
&&
جان بیابد دو سه بی‌جان چه شود
\\
ور سواره سوی میدان آیی
&&
تا شود سینه چو میدان چه شود
\\
روی چون ماهت اگر بنمایی
&&
تا رود زهره به میزان چه شود
\\
آستین کرم ار افشانی
&&
تا ندریم گریبان چه شود
\\
ور بریزی قدحی مالامال
&&
بر سر وقت خماران چه شود
\\
ور بپوشیم یکی خلعت نو
&&
ما غلامان ز تو سلطان چه شود
\\
ور چو موسی بپذیری چوبی
&&
تا شود چوب تو ثعبان چه شود
\\
رو به لطف آر و ز دشمن مشنو
&&
گر بجویی دل ایشان چه شود
\\
بس کن ای دل ز فغان جمع نشین
&&
گر نگویی تو پریشان چه شود
\\
\end{longtable}
\end{center}
