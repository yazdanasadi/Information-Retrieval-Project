\begin{center}
\section*{غزل شماره ۹۹۲: خسروانی که فتنه‌ای چینید}
\label{sec:0992}
\addcontentsline{toc}{section}{\nameref{sec:0992}}
\begin{longtable}{l p{0.5cm} r}
خسروانی که فتنه‌ای چینید
&&
فتنه برخاست هیچ ننشینید
\\
هم شما هم شما که زیبایید
&&
هم شما هم شما که شیرینید
\\
همچو عنبر حمایلیم همه
&&
بر بر سیمتان که مشکینید
\\
لذتی هست با شما گفتن
&&
هم شما داد جان مسکینید
\\
نشوم شاد اگر گمان دارم
&&
که گهی شاد و گاه غمگینید
\\
بل که بر اسب ذوق و شیرینی
&&
تا ابد خوش نشسته در زینید
\\
شاهدان فانی و شما جمله
&&
با لب لعل و جان سنگینید
\\
در صفای می شهان دیدیم
&&
که شما چون کدوی رنگینید
\\
در بهشتی که هر زمان بکریست
&&
مرد آیید اگر نه عنینید
\\
تبریزی شوید اگر در عشق
&&
بنده شمس ملت و دینید
\\
\end{longtable}
\end{center}
