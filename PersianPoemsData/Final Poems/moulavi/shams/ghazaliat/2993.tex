\begin{center}
\section*{غزل شماره ۲۹۹۳: شاها بکش قطار که شهوار می‌کشی}
\label{sec:2993}
\addcontentsline{toc}{section}{\nameref{sec:2993}}
\begin{longtable}{l p{0.5cm} r}
شاها بکش قطار که شهوار می‌کشی
&&
دامان ما گرفته به گلزار می‌کشی
\\
قطار اشتران همه مستند و کف زنان
&&
بویی ببرده‌اند که قطار می‌کشی
\\
هر اشتری میانه زنجیر می‌گزد
&&
چون شهد و چون شکر که سوی یار می‌کشی
\\
آن چشم‌های مست به چشمت که ساقی است
&&
گویند خوش بکش که به دیدار می‌کشی
\\
ما کشت تو بدیم درودی به داس عشق
&&
کردی ز که جدا و به انبار می‌کشی
\\
سکسک بدیم و توسن و در راه صدق لنگ
&&
رهوار از آن شدیم که رهوار می‌کشی
\\
هر چند سال‌ها ز چمن گل بچیده‌ایم
&&
ناگه ز چشم بد به ره خار می‌کشی
\\
ما کی غلط کنیم به هر سو کشی بکش
&&
هر سو کشی به عشرت بسیار می‌کشی
\\
شاهان کشند بنده بد را به انتقام
&&
تو جانب کرامت و ایثار می‌کشی
\\
زین لطف مجرمان را گستاخ کرده‌ای
&&
دزدان دار را خوش و بی‌دار می‌کشی
\\
هر تخمه و ملول همی‌گویدم خموش
&&
تو کرده‌ای ستیزه به گفتار می‌کشی
\\
سختی کشان ز گردش این چرخ در غم اند
&&
بر رغم جمله چرخه دوار می‌کشی
\\
ای شاه شمس مفخر تبریز نور حق
&&
تو نور نور ندره به اقطار می‌کشی
\\
\end{longtable}
\end{center}
