\begin{center}
\section*{غزل شماره ۱۹۸۱: مطربا نرمک بزن تا روح بازآید به تن}
\label{sec:1981}
\addcontentsline{toc}{section}{\nameref{sec:1981}}
\begin{longtable}{l p{0.5cm} r}
مطربا نرمک بزن تا روح بازآید به تن
&&
چون زنی بر نام شمس الدین تبریزی بزن
\\
نام شمس الدین به گوشت بهتر است از جسم و جان
&&
نام شمس الدین چو شمع و جان بنده چون لگن
\\
مطربا بهر خدا تو غیر شمس الدین مگو
&&
بر تن چون جان او بنواز تن تن تن تنن
\\
تا شود این نقش تو رقصان به سوی آسمان
&&
تا شود این جان پاکت پرده سوز و گام زن
\\
شمس دین و شمس دین و شمس دین می گوی و بس
&&
تا ببینی مردگان رقصان شده اندر کفن
\\
مطربا گر چه نیی عاشق مشو از ما ملول
&&
عشق شمس الدین کند مر جانت را چون یاسمن
\\
لاله‌ها دستک زنان و یاسمین رقصان شده
&&
سوسنک مستک شده گوید که باشد خود سمن
\\
خارها خندان شده بر گل بجسته برتری
&&
سنگ‌ها باجان شده با لعل گوید ما و من
\\
ایها الساقی ادر کأس الحمیا نصفه
&&
ان عشقی مثل خمر ان جسمی مثل دن
\\
\end{longtable}
\end{center}
