\begin{center}
\section*{غزل شماره ۳۸۶: چون نداری تاب دانش چشم بگشا در صفات}
\label{sec:0386}
\addcontentsline{toc}{section}{\nameref{sec:0386}}
\begin{longtable}{l p{0.5cm} r}
چون نداری تاب دانش چشم بگشا در صفات
&&
چون نبینی بی‌جهت را نور او بین در جهات
\\
حوریان بین نوریان بین زیر این ازرق تتق
&&
مسلمات مؤمنات قانتات تائبات
\\
هر یکی با نازباز و هر یکی عاشق نواز
&&
هر یکی شمع طراز و هر یکی صبح نجات
\\
هر یکی بسته دهان و موشکاف اندر بیان
&&
هر یکی شکرستان و هر یکی کان نبات
\\
جان کهنه می‌فشان و جان تازه می‌ستان
&&
در فقیری می‌خرام و می‌ستان ز ایشان زکات
\\
شیر جان زین مریمان خور چونک زاده ثاینی
&&
تا چو عیسی فارغ آیی از بنین و از بنات
\\
روز و شب را چون دو مجنون درکشان در سلسله
&&
ای که هر روزت چو عید و هر شبت قدر و برات
\\
چونک شه بنمود رخ را اسب شد همراه پیل
&&
عقل مسکین گشت مات و جان میان برد و مات
\\
عاشقان را وقت شورش ابله و شپشپ مبین
&&
کوه جودی عاجز آید پیش ایشان در ثبات
\\
جان جمله پیشه‌ها عشقست اما آنک او
&&
تره زار دل نبیند درفتد در ترهات
\\
من خمش کردم چو دیدم خوشتر از خود ناطقی
&&
پیش او میرم بگویم اقتلونی یا ثقات
\\
شمس تبریزی چو بگشاید دهان چون شکر
&&
از طرب در جنبش آید هم رمیم و هم رفات
\\
رو خمش کن قول کم گو بعد از این فعال باش
&&
چند گویی فاعلاتن فاعلاتن فاعلات
\\
\end{longtable}
\end{center}
