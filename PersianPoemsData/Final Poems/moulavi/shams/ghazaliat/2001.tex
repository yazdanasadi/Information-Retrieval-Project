\begin{center}
\section*{غزل شماره ۲۰۰۱: دم ده و عشوه ده ای دلبر سیمین بر من}
\label{sec:2001}
\addcontentsline{toc}{section}{\nameref{sec:2001}}
\begin{longtable}{l p{0.5cm} r}
دم ده و عشوه ده ای دلبر سیمین بر من
&&
که دمم بی‌دم تو چون اجل آمد بر من
\\
دل چو دریا شودم چون گهرت درتابد
&&
سر به گردون رسدم چونک بخاری سر من
\\
خنک آن دم که بیاری سوی من باده لعل
&&
بدرخشد ز شرارش رخ همچون زر من
\\
زان خرابم که ز اوقاف خرابات توام
&&
در خرابی است عمارت شدن مخبر من
\\
شاهد جان چو شهادت ز درون عرضه کند
&&
زود انگشت برآرد خرد کافر من
\\
پیش از آنک به حریفان دهی ای ساقی جمع
&&
از همه تشنه ترم من بده آن ساغر من
\\
بنده امر توام خاصه در آن امر که تو
&&
گوییم خیز نظر کن به سوی منظر من
\\
هین برافروز دلم را تو به نار موسی
&&
تا که افروخته ماند ابدا اخگر من
\\
من خمش کردم و در جوی تو افکندم خویش
&&
که ز جوی تو بود رونق شعر تر من
\\
\end{longtable}
\end{center}
