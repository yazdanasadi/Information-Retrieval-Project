\begin{center}
\section*{غزل شماره ۲۱۶۴: اگر بگذشت روز ای جان به شب مهمان مستان شو}
\label{sec:2164}
\addcontentsline{toc}{section}{\nameref{sec:2164}}
\begin{longtable}{l p{0.5cm} r}
اگر بگذشت روز ای جان به شب مهمان مستان شو
&&
بر خویشان و بی‌خویشان شبی تا روز مهمان شو
\\
مرو ای یوسف خوبان ز پیش چشم یعقوبان
&&
شب قدری کن این شب را چراغ بیت احزان شو
\\
اگر دوریم رحمت شو وگر عوریم خلعت شو
&&
وگر ضعفیم صحت شو وگر دردیم درمان شو
\\
اگر کفریم ایمان شو وگر جرمیم غفران شو
&&
وگر عوریم احسان شو بهشتی باش و رضوان شو
\\
برای پاسبانی را بکوب آن طبل جانی را
&&
برای دیورانی را شهب انداز شیطان شو
\\
تو بحری و جهان ماهی به گاهی چیست و بی‌گاهی
&&
حیات ماهیان خواهی بر ایشان آب حیوان شو
\\
شب تیره چه خوش باشد که مه مهمان ما باشد
&&
برای شب روان جان برآ ای ماه تابان شو
\\
خمش کن ای دل مضطر مگو دیگر ز خیر و شر
&&
چو پیش او است سر مظهر دهان بربند و پنهان شو
\\
\end{longtable}
\end{center}
