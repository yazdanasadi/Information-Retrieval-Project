\begin{center}
\section*{غزل شماره ۳۱۷۶: کردم با کان گهر آشتی}
\label{sec:3176}
\addcontentsline{toc}{section}{\nameref{sec:3176}}
\begin{longtable}{l p{0.5cm} r}
کردم با کان گهر آشتی
&&
کردم با قرص قمر آشتی
\\
خمرهٔ سرکه ز شکر صلح خواست
&&
شکر که پذرفت شکر آشتی
\\
آشتی و جنگ ز جذبهٔ حق است
&&
نیست زدم، هست ز سر آشتی
\\
رفت مسیحا به فلک ناگهان
&&
با ملکان کرد بشر آشتی
\\
ای فلک لطف، مسیح توم
&&
گر بکنی بار دگر آشتی
\\
جذبهٔ او داد عدم را وجود
&&
کرده بدان پیه نظر آشتی
\\
شاه مرا میل چو در آشتیست
&&
کرد در افلاک اثر آشتی
\\
گشت فلک دایهٔ این خاکدان
&&
ثور و اسد آمد در آشتی
\\
صلح درآ، این قدر آخر بدانک
&&
کرد کنون جبر و قدر آشتی
\\
بس کن کین صبح مرا، دایمست
&&
نیست مرا بهر سپر آشتی
\\
\end{longtable}
\end{center}
