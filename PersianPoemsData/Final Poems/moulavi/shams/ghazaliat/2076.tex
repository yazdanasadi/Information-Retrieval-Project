\begin{center}
\section*{غزل شماره ۲۰۷۶: به جان تو که از این دلشده کرانه مکن}
\label{sec:2076}
\addcontentsline{toc}{section}{\nameref{sec:2076}}
\begin{longtable}{l p{0.5cm} r}
به جان تو که از این دلشده کرانه مکن
&&
بساز با من مسکین و عزم خانه مکن
\\
بهانه‌ها بمیندیش و عذر را بگذار
&&
مرا مگیر ز بالا و خشک شانه مکن
\\
شراب حاضر و دولت ندیم و تو ساقی
&&
بده شراب و دغل‌های ساقیانه مکن
\\
نظر به روی حریفان بکن که مست تواند
&&
نظر به روزن و دهلیز و آستانه مکن
\\
بجز به حلقه عشاق روزگار مبر
&&
بجز به کوی خرابات آشیانه مکن
\\
ببین که عالم دام است و آرزو دانه
&&
به دام او مشتاب و هوای دانه مکن
\\
ز دام او چو گذشتی قدم بنه بر چرخ
&&
به زیر پای به جز چرخ آستانه مکن
\\
به آفتاب و به مهتاب التفات مکن
&&
یگانه باش و به جز قصد آن یگانه مکن
\\
مکن قرار تو بی‌او چو کاسه بر سر آب
&&
مگیر کاسه به هر مطبخی دوانه مکن
\\
زمانه روشن و تاریک و گرم و سرد شود
&&
مقام جز به سرچشمه زمانه مکن
\\
مکن ستایش بر وی عتاب را بمپوش
&&
مده قطایف و آن سیر در میانه مکن
\\
ولی چه سود که کار بتان همین باشد
&&
مگو به شعله آتش هلا زبانه مکن
\\
بگو به هرچ بسوزی بسوز جز به فراق
&&
روا نباشد و این یک ستم روانه مکن
\\
\end{longtable}
\end{center}
