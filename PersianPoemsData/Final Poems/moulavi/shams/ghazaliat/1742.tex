\begin{center}
\section*{غزل شماره ۱۷۴۲: ببسته است پری نهانیی پایم}
\label{sec:1742}
\addcontentsline{toc}{section}{\nameref{sec:1742}}
\begin{longtable}{l p{0.5cm} r}
ببسته است پری نهانیی پایم
&&
ز بند اوست که من در میان غوغایم
\\
ز کوه قافم من که غریب اطرافم
&&
به صورتم چو کبوتر به خلق عنقایم
\\
کبوترم چو شود صید چنگ باز اجل
&&
از آن سپس پر عنقای روح بگشایم
\\
ز آفتاب خرد گر چه پشت من گرم است
&&
برای سایه نشینان چو خیمه برپایم
\\
چو ابن وقت بود دامن پدر گیرد
&&
چه صوفیم که به سودای دی و فردایم
\\
مرا چو پرده درآویختی بر این درگاه
&&
هم از برای برآویختن نمی‌شایم
\\
ز لطف توست که از جغدیم برآوردی
&&
چو طوطیان ز کف تو شکر همی‌خایم
\\
اگر ز جود کف تو به بحر راه برم
&&
تمام گوهر هستی خویش بنمایم
\\
شکار درک نیم من ورای ادارکم
&&
به پای وهم نیم من درازپهنایم
\\
سخن به جای بمان خویش بین کجایی تو
&&
مرا بجوی همان جا که من همان جایم
\\
\end{longtable}
\end{center}
