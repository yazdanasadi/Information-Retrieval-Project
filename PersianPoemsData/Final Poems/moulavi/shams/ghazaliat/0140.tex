\begin{center}
\section*{غزل شماره ۱۴۰: درد ما را در جهان درمان مبادا بی‌شما}
\label{sec:0140}
\addcontentsline{toc}{section}{\nameref{sec:0140}}
\begin{longtable}{l p{0.5cm} r}
درد ما را در جهان درمان مبادا بی‌شما
&&
مرگ بادا بی‌شما و جان مبادا بی‌شما
\\
سینه‌های عاشقان جز از شما روشن مباد
&&
گلبن جان‌های ما خندان مبادا بی‌شما
\\
بشنو از ایمان که می‌گوید به آواز بلند
&&
با دو زلف کافرت کایمان مبادا بی‌شما
\\
عقل سلطان نهان و آسمان چون چتر او
&&
تاج و تخت و چتر این سلطان مبادا بی‌شما
\\
عشق را دیدم میان عاشقان ساقی شده
&&
جان ما را دیدن ایشان مبادا بی‌شما
\\
جان‌های مرده را ای چون دم عیسی شما
&&
ملک مصر و یوسف کنعان مبادا بی‌شما
\\
چون به نقد عشق شمس الدین تبریزی خوشم
&&
رخ چو زر کردم بگفتم کان مبادا بی‌شما
\\
\end{longtable}
\end{center}
