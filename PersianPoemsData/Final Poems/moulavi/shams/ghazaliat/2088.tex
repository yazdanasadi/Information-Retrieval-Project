\begin{center}
\section*{غزل شماره ۲۰۸۸: ببردی دلم را بدادی به زاغان}
\label{sec:2088}
\addcontentsline{toc}{section}{\nameref{sec:2088}}
\begin{longtable}{l p{0.5cm} r}
ببردی دلم را بدادی به زاغان
&&
گرفتم گروگان خیالت به تاوان
\\
درآیی درآیم بگیری بگیرم
&&
بگویی بگویم علامات مستان
\\
نشاید نشاید ستم کرد با من
&&
برای گریبان دریدن ز دامان
\\
بیاور بیاور شرابی که گفتی
&&
مگو که نگفتم مرنجان مرنجان
\\
شرابی شرابی که دل جمع گردد
&&
چو دل جمع گردد شود تن پریشان
\\
نخواهم نخواهم شرابی بهایی
&&
از آن بحر بگشا شراب فراوان
\\
ز تو باده دادن ز من سجده کردن
&&
ز من شکر کردن ز تو گوهرافشان
\\
چنانم کن ای جان که شکرم نماند
&&
وظیفه بیفزا دو چندان سه چندان
\\
بجوشان بجوشان شرابی ز سینه
&&
بهاری برآور از این برگ ریزان
\\
خرابم کن ای جان که از شهر ویران
&&
خراجی نجوید نه دیوان نه سلطان
\\
خمش باش ای تن که تا جان بگوید
&&
علی میر گردد چو بگذشت عثمان
\\
خمش کردم ای جان بگو نوبت خود
&&
تویی یوسف ما تویی خوب کنعان
\\
\end{longtable}
\end{center}
