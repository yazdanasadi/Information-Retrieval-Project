\begin{center}
\section*{غزل شماره ۲۴۷۰: جان به فدای عاشقان خوش هوسی است عاشقی}
\label{sec:2470}
\addcontentsline{toc}{section}{\nameref{sec:2470}}
\begin{longtable}{l p{0.5cm} r}
جان به فدای عاشقان خوش هوسی است عاشقی
&&
عشق پرست ای پسر باد هواست مابقی
\\
از می عشق سرخوشم آتش عشق مفرشم
&&
پای بنه در آتشم چند از این منافقی
\\
از سوی چرخ تا زمین سلسله‌ای است آتشین
&&
سلسله را بگیر اگر در ره خود محققی
\\
عشق مپرس چون بود عشق یکی جنون بود
&&
سلسله را زبون بود نی به طریق احمقی
\\
عشق پرست ای پسر عشق خوش است ای پسر
&&
رو که به جان صادقان صاف و لطیف و صادقی
\\
راه تو چون فنا بود خصم تو را کجا بود
&&
طاقت تو که را بود کآتش تیز مطلقی
\\
جان مرا تو بنده کن عیش مرا تو زنده کن
&&
مست کن و بیافرین بازنمای خالقی
\\
یک نفسی خموش کن در خمشی خروش کن
&&
وقت سخن تو خامشی در خمشی تو ناطقی
\\
بی‌دل و جان سخنوری شیوه گاو سامری
&&
راست نباشد ای پسر راست برو که حاذقی
\\
\end{longtable}
\end{center}
