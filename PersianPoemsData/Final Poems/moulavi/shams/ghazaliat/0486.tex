\begin{center}
\section*{غزل شماره ۴۸۶: به حق چشم خمار لطیف تابانت}
\label{sec:0486}
\addcontentsline{toc}{section}{\nameref{sec:0486}}
\begin{longtable}{l p{0.5cm} r}
به حق چشم خمار لطیف تابانت
&&
به حلقه حلقه آن طره پریشانت
\\
بدان حلاوت بی‌مر و تنگ‌های شکر
&&
که تعبیه‌ست در آن لعل شکرافشانت
\\
به کهربایی کاندر دو لعل تو درجست
&&
که گشت از آن مه و خورشید و ذره جویانت
\\
به حق غنچه و گل‌های لعل روحانی
&&
که دام بلبل عقل‌ست در گلستانت
\\
به آب حسن و به تاب جمال جان پرور
&&
کز آن گشاد دهان را انار خندانت
\\
بدان جمال الهی که قبله دل‌هاست
&&
که دم به دم ز طرب سجده می‌برد جانت
\\
تو یوسفی و تو را معجزات بسیارست
&&
ولی بس‌ست خود آن روی خوب برهانت
\\
چه جای یوسف بس یوسفان اسیر توند
&&
خدای عز و جل کی دهد بدیشانت
\\
ز هر گیاه و ز هر برگ رویدی نرگس
&&
برای دیدنت از جا بدی به بستانت
\\
چو سوخت ز آتش عشق تو جان گرم روان
&&
کجا دهد شه سردان به دست سردانت
\\
شعاع روی تو پوشیده کرد صورت تو
&&
که غرقه کرد چو خورشید نور سبحانت
\\
هزار صورت هر دم ز نور خورشیدت
&&
برآید از دل پاک و نماید احسانت
\\
درون خویش اگر خواهدت دل ناپاک
&&
ز ابلهی و خری می‌کشد به زندانت
\\
نه هیچ عاقل بفریبدت به حیلت عقل
&&
نه پای بند کند جاده هیچ سلطانت
\\
تو را که در دو جهان می‌نگنجی از عظمت
&&
ابوهریره گمان چون برد در انبانت
\\
به هر غزل که ستایم تو را ز پرده شعر
&&
دلم ز پرده ستاید هزار چندانت
\\
دلم کی باشد و من کیستم ستایش چیست
&&
ولیک جان را گلشن کنم به ریحانت
\\
بیا تو مفخر آفاق شمس تبریزی
&&
که تو غریب مهی و غریب ارکانت
\\
\end{longtable}
\end{center}
