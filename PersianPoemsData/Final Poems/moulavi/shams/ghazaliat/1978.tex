\begin{center}
\section*{غزل شماره ۱۹۷۸: ایها الساقی ادر کأس الحمیا نصف من}
\label{sec:1978}
\addcontentsline{toc}{section}{\nameref{sec:1978}}
\begin{longtable}{l p{0.5cm} r}
ایها الساقی ادر کأس الحمیا نصف من
&&
ان عشقی مثل خمر ان جسمی مثل دن
\\
مطربا نرمک بزن تا روح بازآید به تن
&&
چون زنی بر نام شمس الدین تبریزی بزن
\\
نام شمس الدین به گوشت بهتر است از جسم و جان
&&
نام شمس الدین چو شمع و جان بنده چون لگن
\\
مطربا بهر خدا تو غیر شمس الدین مگو
&&
بر تن و جان وصف او بنواز تن تن تن تنن
\\
نام شمس الدین چو شمعی همچو پروانه بسوز
&&
پیش آن چوگان نامش گوی جان را درفکن
\\
تا شود این جان تو رقاص سوی آسمان
&&
تا شود این جان پاکت پرده سوز و گام زن
\\
شمس دین و شمس دین و شمس دین می گو و بس
&&
تا ببینی مردگان رقصان شده اندر کفن
\\
مطربا گر چه نیی عاشق مشو از ما ملول
&&
عشق شمس الدین کند مر جانت را چون یاسمن
\\
یک شبی تا روز دف را تو بزن بر نام او
&&
کز جمال یوسفی دف تو شد چون پیرهن
\\
ناگهان آن گلرخم از گلستان سر برزند
&&
پیش آن گل محو گردد گلستان‌های چمن
\\
لاله‌ها دستک زنان و یاسمین رقصان شده
&&
سوسنک مستک شده گوید چه باشد خود سمن
\\
خارها خندان شده بر گل بجسته برتری
&&
سنگ‌ها تابان شده با لعل گوید ما و من
\\
\end{longtable}
\end{center}
