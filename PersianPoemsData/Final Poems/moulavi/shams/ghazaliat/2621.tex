\begin{center}
\section*{غزل شماره ۲۶۲۱: از هر چه ترنجیدی با دل تو بگو حالی}
\label{sec:2621}
\addcontentsline{toc}{section}{\nameref{sec:2621}}
\begin{longtable}{l p{0.5cm} r}
از هر چه ترنجیدی با دل تو بگو حالی
&&
کای دل تو نمی‌گفتی کز خویش شدم خالی
\\
این رنج چو در وا شد دعوی تو رسوا شد
&&
زشتی تو پیدا شد بگذار تو نکالی
\\
در صورت رنج خود نظاره بکن ای بد
&&
کی باشد با این خود آن مرتبه عالی
\\
بنگر که چه زشتی تو بس دیوسرشتی تو
&&
این است که کشتی تو پس از کی همی‌نالی
\\
گر رنج بشد مشکل نومید مشو ای دل
&&
کز غیب شود حاصل اندر عوض ابدالی
\\
از ذوق چو عوری تو هر لحظه بشوری تو
&&
کای کعبه چه دوری تو از حیزک خلخالی
\\
در بادیه مردان را کاری است نه سردان را
&&
کاین بادیه فردان را بزدود ز ارذالی
\\
در خدمت مخدومی شمس الحق تبریزی
&&
بشتاب که از فضلش در منزل اجلالی
\\
\end{longtable}
\end{center}
