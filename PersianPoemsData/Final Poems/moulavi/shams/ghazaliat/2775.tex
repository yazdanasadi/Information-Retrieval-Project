\begin{center}
\section*{غزل شماره ۲۷۷۵: مرغ دل پران مبا جز در هوای بیخودی}
\label{sec:2775}
\addcontentsline{toc}{section}{\nameref{sec:2775}}
\begin{longtable}{l p{0.5cm} r}
مرغ دل پران مبا جز در هوای بیخودی
&&
شمع جان تابان مبا جز در سرای بیخودی
\\
آفتاب لطف حق بر عاشقان تابنده باد
&&
تا بیفتد بر همه سایه همای بیخودی
\\
گر هزاران دولت و نعمت ببیند عاشقی
&&
ناید اندر چشم او الا بلای بیخودی
\\
بنگر اندر من که خود را در بلا افکنده‌ام
&&
از حلاوت‌ها که دیدم در فنای بیخودی
\\
جان و صد جان خود چه باشد گر کسی قربان کند
&&
در هوای بیخودی و از برای بیخودی
\\
عاشقا کمتر نشین با مردم غمناک تو
&&
تا غباری درنیفتد در صفای بیخودی
\\
باجفا شو با کسی کو عاشق هشیاری است
&&
تا بیابی ذوق‌ها اندر وفای بیخودی
\\
بیخودی را چون بدانی سروری کاسد شود
&&
ای سری و سروری‌ها خاک پای بیخودی
\\
خوش بود ظاهر شدن بر دشمنان بر تخت ملک
&&
لیک آن‌ها هیچ نبود جان به جای بیخودی
\\
گر تو خواهی شمس تبریزی شود مهمان تو
&&
خانه خالی کن ز خود ای کدخدای بیخودی
\\
\end{longtable}
\end{center}
