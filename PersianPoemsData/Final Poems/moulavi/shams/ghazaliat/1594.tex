\begin{center}
\section*{غزل شماره ۱۵۹۴: ایها العشاق آتش گشته چون استاره‌ایم}
\label{sec:1594}
\addcontentsline{toc}{section}{\nameref{sec:1594}}
\begin{longtable}{l p{0.5cm} r}
ایها العشاق آتش گشته چون استاره‌ایم
&&
لاجرم رقصان همه شب گرد آن مه پاره‌ایم
\\
تا بود خورشید حاضر هست استاره ستیر
&&
بی‌رخ خورشید ما می دانک ما آواره‌ایم
\\
الصلا ای عاشقان‌هان الصلا این کاریان
&&
باده کاری است این جا زانک ما این کاره‌ایم
\\
هر سحر پیغام آن پیغامبر خوبان رسد
&&
کالصلا بیچارگان ما عاشقان را چاره‌ایم
\\
نعره لبیک لبیک از همه برخاسته
&&
مصحف معنی تویی ما هر یکی سی پاره‌ایم
\\
خونبهای کشتگان چون غمزه خونی اوست
&&
در میان خون خود چون طفلک خون خواره‌ایم
\\
کوه طور از باده‌اش بیخود شد و بدمست شد
&&
ما چه کوه آهنیم آخر چه سنگ خاره‌ایم
\\
یک جو از سرش نگوییم ار همه جو جو شویم
&&
گرد خرمنگاه چرخ ار چه که ما سیاره‌ایم
\\
همچو مریم حامله نور خدایی گشته‌ایم
&&
گر چو عیسی بسته این جسم چون گهواره‌ایم
\\
از درون باره این عقل خود ما را مجو
&&
زانک در صحرای عشقش ما برون باره‌ایم
\\
عشق دیوانه‌ست و ما دیوانه دیوانه‌ایم
&&
نفس اماره‌ست و ما اماره اماره‌ایم
\\
مفخر تبریز شمس الدین تو بازآ زین سفر
&&
بهر حق یک بارگی ما عاشق یک باره‌ایم
\\
\end{longtable}
\end{center}
