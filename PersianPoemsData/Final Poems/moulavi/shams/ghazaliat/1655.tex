\begin{center}
\section*{غزل شماره ۱۶۵۵: دوش می گفت جانم کی سپهر معظم}
\label{sec:1655}
\addcontentsline{toc}{section}{\nameref{sec:1655}}
\begin{longtable}{l p{0.5cm} r}
دوش می گفت جانم کی سپهر معظم
&&
بس معلق زنانی شعله‌ها اندر اشکم
\\
بی‌گنه بی‌جنایت گردشی بی‌نهایت
&&
بر تنت در شکایت نیلیی رسم ماتم
\\
گه خوش و گاه ناخوش چون خلیل اندر آتش
&&
هم شه و هم گداوش چون براهیم ادهم
\\
صورتت سهمناکی حالتت دردناکی
&&
گردش آسیاها داری و پیچ ارقم
\\
گفت چرخ مقدس چون نترسم از آن کس
&&
کو بهشت جهان را می کند چون جهنم
\\
در کفش خاک مومی سازدش رنگ و رومی
&&
سازدش باز و بومی سازدش شکر و سم
\\
او نهانی است یارا این چنین آشکارا
&&
پیش کرده است ما را تا شود او مکتم
\\
کی شود بحر کیهان زیر خاشاک پنهان
&&
گشته خاشاک رقصان موج در زیر و در بم
\\
چون تن خاکدانت بر سر آب جانت
&&
جان تتق کرده تن را در عروسی و در غم
\\
در تتق نوعروسی تندخویی شموسی
&&
می کند خوش فسوسی بر بد و نیک عالم
\\
خاک از او سبزه زاری چرخ از او بی‌قراری
&&
هر طرف بختیاری زو معاف و مسلم
\\
عقل از او مستقینی صبر از او مستعینی
&&
عشق از او غیب بینی خاک او نقش آدم
\\
باد پویان و جویان آب‌ها دست شویان
&&
ما مسیحانه گویان خاک خامش چو مریم
\\
بحر با موج‌ها بین گرد کشتی خاکین
&&
کعبه و مکه‌ها بین در تک چاه زمزم
\\
شه بگوید تو تن زن خویش در چه میفکن
&&
که ندانی تو کردن دلو و حبل از شلولم
\\
\end{longtable}
\end{center}
