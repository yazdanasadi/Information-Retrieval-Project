\begin{center}
\section*{غزل شماره ۱۲۳: دیدم شه خوب خوش لقا را}
\label{sec:0123}
\addcontentsline{toc}{section}{\nameref{sec:0123}}
\begin{longtable}{l p{0.5cm} r}
دیدم شه خوب خوش لقا را
&&
آن چشم و چراغ سینه‌ها را
\\
آن مونس و غمگسار دل را
&&
آن جان و جهان جان فزا را
\\
آن کس که خرد دهد خرد را
&&
آن کس که صفا دهد صفا را
\\
آن سجده گه مه و فلک را
&&
آن قبله جان اولیا را
\\
هر پاره من جدا همی‌گفت
&&
کای شکر و سپاس مر خدا را
\\
موسی چو بدید ناگهانی
&&
از سوی درخت آن ضیا را
\\
گفتا که ز جست و جوی رستم
&&
چون یافتم این چنین عطا را
\\
گفت ای موسی سفر رها کن
&&
وز دست بیفکن آن عصا را
\\
آن دم موسی ز دل برون کرد
&&
همسایه و خویش و آشنا را
\\
اخلع نعلیک این بود این
&&
کز هر دو جهان ببر ولا را
\\
در خانه دل جز او نگنجد
&&
دل داند رشک انبیا را
\\
گفت ای موسی به کف چه داری
&&
گفتا که عصاست راه ما را
\\
گفتا که عصا ز کف بیفکن
&&
بنگر تو عجایب سما را
\\
افکند و عصاش اژدها شد
&&
بگریخت چو دید اژدها را
\\
گفتا که بگیر تا منش باز
&&
چوبی سازم پی شما را
\\
سازم ز عدوت دست یاری
&&
سازم دشمنت متکا را
\\
تا از جز فضل من ندانی
&&
یاران لطیف باوفا را
\\
دست و پایت چو مار گردد
&&
چون درد دهیم دست و پا را
\\
ای دست مگیر غیر ما را
&&
ای پا مطلب جز انتها را
\\
مگریز ز رنج ما که هر جا
&&
رنجیست رهی بود دوا را
\\
نگریخت کسی ز رنج الا
&&
آمد بترش پی جزا را
\\
از دانه گریز بیم آن جاست
&&
بگذار به عقل بیم جا را
\\
شمس تبریز لطف فرمود
&&
چون رفت ببرد لطف‌ها را
\\
\end{longtable}
\end{center}
