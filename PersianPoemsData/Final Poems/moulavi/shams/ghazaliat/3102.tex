\begin{center}
\section*{غزل شماره ۳۱۰۲: برست جان و دلم از خودی و از هستی}
\label{sec:3102}
\addcontentsline{toc}{section}{\nameref{sec:3102}}
\begin{longtable}{l p{0.5cm} r}
برست جان و دلم از خودی و از هستی
&&
شدست خاص شهنشاه روح در مستی
\\
زهی وجود که جان یافت در عدم ناگاه
&&
زهی بلند که جان گشت در چنین پستی
\\
درست گشت مرا آنچ می‌ندانستم
&&
چو در درستی آن مه مرا تو بشکستی
\\
چو گشت عشق تو فصاد و اکحلم بگشاد
&&
بجستم از خود و گفتم زهی سبک دستی
\\
طبیب فقر بخست و گرفت گوش مرا
&&
که مژده ده که ز رنج وجود وارستی
\\
ز انتظار رهیدی که کی صبا بوزد
&&
نه بحر را تو زبونی نه بسته شستی
\\
ز شمس تبریز این جنس‌ها بخر بفروش
&&
ز نقدهاش چو آن کیسه بر کمر بستی
\\
\end{longtable}
\end{center}
