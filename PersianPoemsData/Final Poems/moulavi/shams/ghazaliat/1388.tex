\begin{center}
\section*{غزل شماره ۱۳۸۸: ای نفس کل صورت مکن وی عقل کل بشکن قلم}
\label{sec:1388}
\addcontentsline{toc}{section}{\nameref{sec:1388}}
\begin{longtable}{l p{0.5cm} r}
ای نفس کل صورت مکن وی عقل کل بشکن قلم
&&
ای مرد طالب کم طلب بر آب جو نقش قدم
\\
ای عاشق صافی روان رو صاف چون آب روان
&&
کاین آب صافی بی‌گره جان می فزاید دم به دم
\\
از باد آب بی‌گره گر ساعتی پوشد زره
&&
بر آب جو تهمت منه کو را نه ترس است و نه غم
\\
در نقش بی‌نقشی ببین هر نقش را صد رنگ و بو
&&
در برگ بی‌برگی نگر هر شاخ را باغ ارم
\\
زان صورت صورت گسل کو منبع جان است و دل
&&
تن ریخته از شرم او بگریخته جان در حرم
\\
از باده و از باد او بس بنده و آزاد او
&&
چون کان فروبر نفس چون که برآورده شکم
\\
از بحر گویم یا ز در یا از نفاذ حکم مر
&&
نی از مقالت هم ببر می تاز تا پای علم
\\
چپ راست دان این راه را در چاه دان این چاه را
&&
چون سوی موج خون روی در خون بود خوان کرم
\\
در آتش آبی تعبیه در آب آتش تعبیه
&&
در آتشش جان در طرب در آب او دل در ندم
\\
یا من ولی انعامنا ثبت لنا اقدامنا
&&
ای بی‌تو راحت‌ها عنا ای بی‌تو صحت‌ها سقم
\\
\end{longtable}
\end{center}
