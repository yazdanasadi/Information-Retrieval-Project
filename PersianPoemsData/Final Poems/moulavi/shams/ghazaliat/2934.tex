\begin{center}
\section*{غزل شماره ۲۹۳۴: گفتی شکار گیرم رفتی شکار گشتی}
\label{sec:2934}
\addcontentsline{toc}{section}{\nameref{sec:2934}}
\begin{longtable}{l p{0.5cm} r}
گفتی شکار گیرم رفتی شکار گشتی
&&
گفتی قرار یابم خود بی‌قرار گشتی
\\
خضرت چرا نخوانم کآب حیات خوردی
&&
پیشت چرا نمیرم چون یار یار گشتی
\\
گردت چرا نگردم چون خانه خدایی
&&
پایت چرا نبوسم چون پایدار گشتی
\\
جامت چرا ننوشم چون ساقی وجودی
&&
نقلت چرا نچینیم چون قندبار گشتی
\\
فاروق چون نباشی چون از فراق رستی
&&
صدیق چون نباشی چون یار غار گشتی
\\
اکنون تو شهریاری کو را غلام گشتی
&&
اکنون شگرف و زفتی کز غم نزار گشتی
\\
هم گلشنش بدیدی صد گونه گل بچیدی
&&
هم سنبلش بسودی هم لاله زار گشتی
\\
ای چشمش الله الله خود خفته می‌زدی ره
&&
اکنون نعوذبالله چون پرخمار گشتی
\\
آنگه فقیر بودی بس خرقه‌ها ربودی
&&
پس وای بر فقیران چون ذوالفقار گشتی
\\
هین بیخ مرگ برکن زیرا که نفخ صوری
&&
گردن بزن خزان را چون نوبهار گشتی
\\
از رستخیز ایمن چون رستخیز نقدی
&&
هم از حساب رستی چون بی‌شمار گشتی
\\
از نان شدی تو فارغ چون ماهیان دریا
&&
وز آب فارغی هم چون سوسمار گشتی
\\
ای جان چون فرشته از نور حق سرشته
&&
هم ز اختیار رسته نک اختیار گشتی
\\
از کام نفس حسی روزی دو سه بریدی
&&
هم دوست کامی اکنون هم کامیار گشتی
\\
غم را شکار بودی بی‌کردگار بودی
&&
چون کردگار گشتی باکردگار گشتی
\\
گر خون خلق ریزی ور با فلک ستیزی
&&
عذرت عذار خواهد چون گلعذار گشتی
\\
نازت رسد ازیرا زیبا و نازنینی
&&
کبرت رسدهمی زان چون از کبار گشتی
\\
باش از در معانی در حلقه خموشان
&&
در گوش‌ها اگر چه چون گوشوار گشتی
\\
\end{longtable}
\end{center}
