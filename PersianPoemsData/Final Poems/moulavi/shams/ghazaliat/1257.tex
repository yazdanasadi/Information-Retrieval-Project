\begin{center}
\section*{غزل شماره ۱۲۵۷: چون تو شادی بنده گو غمخوار باش}
\label{sec:1257}
\addcontentsline{toc}{section}{\nameref{sec:1257}}
\begin{longtable}{l p{0.5cm} r}
چون تو شادی بنده گو غمخوار باش
&&
تو عزیزی صد چو ما گو خوار باش
\\
کار تو باید که باشد بر مراد
&&
کارهای عاشقان گو زار باش
\\
شاه منصوری و ملکت آن توست
&&
بنده چون منصور گو بر دار باش
\\
اشتر مستم نجویم نسترن
&&
نوشخوارم در رهت گو خار باش
\\
نشنوم من هیچ جز پیغام او
&&
هر چه خواهی گفت گو اسرار باش
\\
ای دل آن جایی تو باری که ویست
&&
از جمال یار برخوردار باش
\\
او طبیبست و به بیماران رود
&&
ای تن وامانده تو بیمار باش
\\
بر امید یار غار خلوتی
&&
ثانی اثنین برو در غار باش
\\
بر امید داد و ایثار بهار
&&
مهرها می‌کار و در ایثار باش
\\
خرمنا بر طمع ماه بانمک
&&
گم شو از دزد و در آن انبار باش
\\
بهر نطق یار خوش گفتار خویش
&&
لب ببند از گفت و کم گفتار باش
\\
\end{longtable}
\end{center}
