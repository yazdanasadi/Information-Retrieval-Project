\begin{center}
\section*{غزل شماره ۱۰۴۴: به ساقی درنگر در مست منگر}
\label{sec:1044}
\addcontentsline{toc}{section}{\nameref{sec:1044}}
\begin{longtable}{l p{0.5cm} r}
به ساقی درنگر در مست منگر
&&
به یوسف درنگر در دست منگر
\\
ایا ماهی جان در شست قالب
&&
ببین صیاد را در شست منگر
\\
بدان اصلی نگر کغاز بودی
&&
به فرعی کان کنون پیوست منگر
\\
بدان گلزار بی‌پایان نظر کن
&&
بدین خاری که پایت خست منگر
\\
همایی بین که سایه بر تو افکند
&&
به زاغی کز کف تو جست منگر
\\
چو سرو و سنبله بالاروش کن
&&
بنفشه وار سوی پست منگر
\\
چو در جویت روان شد آب حیوان
&&
به خم و کوزه گر اشکست منگر
\\
به هستی بخش و مستی بخش بگرو
&&
منال از نیست و اندر هست منگر
\\
قناعت بین که نرست و سبک رو
&&
به طمع ماده آبست منگر
\\
تو صافان بین که بر بالا دویدند
&&
به دردی کان به بن بنشست منگر
\\
جهان پر بین ز صورت‌های قدسی
&&
بدان صورت که راهت بست منگر
\\
به دام عشق مرغان شگرفند
&&
به بومی که ز دامش رست منگر
\\
به از تو ناطقی اندر کمین هست
&&
در آن کاین لحظه خاموشست منگر
\\
\end{longtable}
\end{center}
