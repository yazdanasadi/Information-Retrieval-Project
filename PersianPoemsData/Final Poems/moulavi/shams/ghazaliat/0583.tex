\begin{center}
\section*{غزل شماره ۵۸۳: رسیدم در بیابانی که عشق از وی پدید آید}
\label{sec:0583}
\addcontentsline{toc}{section}{\nameref{sec:0583}}
\begin{longtable}{l p{0.5cm} r}
رسیدم در بیابانی که عشق از وی پدید آید
&&
بیابد پاکی مطلق در او هر چه پلید آید
\\
چه مقدارست مر جان را که گردد کفو مرجان را
&&
ولی تو آفتابی بین که بر ذره پدید آید
\\
هزاران قفل و هر قفلی به عرض آسمان باشد
&&
دو سه حرف چو دندانه بر آن جمله کلید آید
\\
یکی لوحیست دل لایح در آن دریای خون سایح
&&
شود غازی ز بعد آنک صد باره شهید آید
\\
غلام موج این بحرم که هم عیدست و هم نحرم
&&
غلام ماهیم که او ز دریا مستفید آید
\\
هر آن قطره کز این دریا به ظاهر صورتی یابد
&&
یقین می‌دان که نام او جنید و بایزید آید
\\
درآ ای جان و غسلی کن در این دریای بی‌پایان
&&
که از یک قطره غسلت هزاران داد و دید آید
\\
خطر دارند کشتی‌ها ز اوج و موج هر دریا
&&
امان یابند از موجی کز این بحر سعید آید
\\
چو عارف را و عاشق را به هر ساعت بود عیدی
&&
نباشد منتظر سالی که تا ایام عید آید
\\
\end{longtable}
\end{center}
