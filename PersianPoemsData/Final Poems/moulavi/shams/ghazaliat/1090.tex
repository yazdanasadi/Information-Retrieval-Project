\begin{center}
\section*{غزل شماره ۱۰۹۰: صنما این چه گمانست فرودست حقیر}
\label{sec:1090}
\addcontentsline{toc}{section}{\nameref{sec:1090}}
\begin{longtable}{l p{0.5cm} r}
صنما این چه گمانست فرودست حقیر
&&
تا بدین حد مکن و جان مرا خوار مگیر
\\
کوه را که کند اندر نظر مرد قضا
&&
کاه را کوه کند ذاک علی الله یسیر
\\
خنک آن چشم که گوهر ز خسی بشناسد
&&
خنک آن قافله‌ای که بودش دوست خفیر
\\
حاکمی هر چه تو نامم بنهی خشنودم
&&
جان پاک تو که جان از تو شکورست و شکیر
\\
ماه را گر تو حبش نام نهی سجده کند
&&
سرو را چنبر خوانی نکند هیچ نفیر
\\
زانک دشنام تو بهتر ز ثناهای جهان
&&
ز کجا بانگ سگان و ز کجا شیر زئیر
\\
ای که بطال تو بهتر ز همه مشتغلان
&&
جز تو جمله همه لاست از آنیم فقیر
\\
تاج زرین بده و سیلی آن یار بخر
&&
ور کسی نشنود این را انما انت نذیر
\\
بر قفای تو چو باشد اثر سیلی دوست
&&
بوسه‌ها یابد رویت ز نگاران ضمیر
\\
مرد دنیا عدمی را حشمی پندارد
&&
عمر در کار عدم کی کند ای دوست بصیر
\\
رفت مردی به طبیبی به کله درد شکم
&&
گفت او را تو چه خوردی که برستست زحیر
\\
بیشتر رنج که آید همه از فعل گلوست
&&
گفت من سوخته نان خوردم از پست فطیر
\\
گفت سنقر برو آن کحل عزیزی به من آر
&&
گفت درد شکم و کحل خه ای شیخ کبیر
\\
گفت تا چشم تو مر سوخته را بشناسد
&&
تا ننوشی تو دگر سوخته ای نیم ضریر
\\
نیست را هست گمان برده‌ای از ظلمت چشم
&&
چشمت از خاک در شاه شود خوب و منیر
\\
هله ای شارح دل‌ها تو بگو شرح غزل
&&
من اگر شرح کنم نیز برنجد دل میر
\\
\end{longtable}
\end{center}
