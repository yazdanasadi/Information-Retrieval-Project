\begin{center}
\section*{غزل شماره ۵۹۹: امشب عجبست ای جان گر خواب رهی یابد}
\label{sec:0599}
\addcontentsline{toc}{section}{\nameref{sec:0599}}
\begin{longtable}{l p{0.5cm} r}
امشب عجبست ای جان گر خواب رهی یابد
&&
وان چشم کجا خسپد کو چون تو شهی یابد
\\
ای عاشق خوش مذهب زنهار مخسب امشب
&&
کان یار بهانه جو بر تو گنهی یابد
\\
من بنده آن عاشق کو نر بود و صادق
&&
کز چستی و شبخیزی از مه کلهی یابد
\\
در خدمت شه باشد شب همره مه باشد
&&
تا از ملاء اعلا چون مه سپهی یابد
\\
بر زلف شب آن غازی چون دلو رسن بازی
&&
آموخت که یوسف را در قعر چهی یابد
\\
آن اشتر بیچاره نومید شدست از جو
&&
می‌گردد در خرمن تا مشت کهی یابد
\\
بالش چو نمی‌یابد از اطلس روی تو
&&
باشد ز شب قدرت شال سیهی یابد
\\
زان نعل تو در آتش کردند در این سودا
&&
تا هر دل سودایی در خود شرهی یابد
\\
امشب شب قدر آمد خامش شو و خدمت کن
&&
تا هر دل اللهی ز الله ولهی یابد
\\
اندر پی خورشیدش شب رو پی امیدش
&&
تا ماه بلند تو با مه شبهی یابد
\\
\end{longtable}
\end{center}
