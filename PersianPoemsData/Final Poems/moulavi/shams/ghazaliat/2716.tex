\begin{center}
\section*{غزل شماره ۲۷۱۶: بگفتم با دلم آخر قراری}
\label{sec:2716}
\addcontentsline{toc}{section}{\nameref{sec:2716}}
\begin{longtable}{l p{0.5cm} r}
بگفتم با دلم آخر قراری
&&
ز آتش‌های او آخر فراری
\\
تو را می‌گویم و تو از سر طنز
&&
اشارت می‌کنی خندان که آری
\\
منم از دست تو بی‌دست و پایی
&&
تو در کوی مهی شکرعذاری
\\
دلم گفتا ندیدی آنچ دیدم
&&
تو پنداری ز اکنون است کاری
\\
منم جزوی و از خود کل کل است
&&
وی است دریای آتش من شراری
\\
ورا دیدم چو بحری موج می‌زد
&&
و جان من ز بحر او بخاری
\\
ز تبریز آفتابی رو نمودم
&&
بشد رقاص جانم ذره واری
\\
خداوند شمس دین چون یک نظر تافت
&&
بجوشید آب خوش از جان ناری
\\
ز هر قطره یکی جانی همی‌رست
&&
همی‌پرید اندر لاله زاری
\\
\end{longtable}
\end{center}
