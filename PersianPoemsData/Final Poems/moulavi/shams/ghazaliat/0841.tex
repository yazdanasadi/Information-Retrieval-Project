\begin{center}
\section*{غزل شماره ۸۴۱: باز آفتاب دولت بر آسمان برآمد}
\label{sec:0841}
\addcontentsline{toc}{section}{\nameref{sec:0841}}
\begin{longtable}{l p{0.5cm} r}
باز آفتاب دولت بر آسمان برآمد
&&
باز آرزوی جان‌ها از راه جان درآمد
\\
باز از رضای رضوان درهای خلد وا شد
&&
هر روح تا به گردن در حوض کوثر آمد
\\
باز آن شهی درآمد کو قبله شهانست
&&
باز آن مهی برآمد کز ماه برتر آمد
\\
سرگشتگان سودا جمله سوار گشتند
&&
کان شاه یک سواره در قلب لشکر آمد
\\
اجزای خاک تیره حیران شدند و خیره
&&
از لامکان شنیده خیزید محشر آمد
\\
آمد ندای بی‌چون نی از درون نه بیرون
&&
نی چپ نی راست نی پس نی از برابر آمد
\\
گویی که آن چه سویست آن سو که جست و جویست
&&
گویی کجا کنم رو آن سو که این سر آمد
\\
آن سو که میوه‌ها را این پختگی رسیدست
&&
آن سو که سنگ‌ها را اوصاف گوهر آمد
\\
آن سو که خشک ماهی شد پیش خضر زنده
&&
آن سو که دست موسی چون ماه انور آمد
\\
این سوز در دل ما چون شمع روشن آمد
&&
وین حکم بر سر ما چون تاج مفخر آمد
\\
دستور نیست جان را تا گوید این بیان را
&&
ور نی ز کفر رستی هر جا که کفر آمد
\\
کافر به وقت سختی رو آورد بدان سو
&&
این سو چو درد بیند آن سوش باور آمد
\\
با درد باش تا درد آن سوت ره نماید
&&
آن سو که بیند آن کس کز درد مضطر آمد
\\
آن پادشاه اعظم در بسته بود محکم
&&
پوشید دلق آدم امروز بر در آمد
\\
\end{longtable}
\end{center}
