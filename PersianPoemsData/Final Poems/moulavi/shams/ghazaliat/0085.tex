\begin{center}
\section*{غزل شماره ۸۵: از بهر خدا بنگر در روی چو زر جانا}
\label{sec:0085}
\addcontentsline{toc}{section}{\nameref{sec:0085}}
\begin{longtable}{l p{0.5cm} r}
از بهر خدا بنگر در روی چو زر جانا
&&
هر جا که روی ما را با خویش ببر جانا
\\
چون در دل ما آیی تو دامن خود برکش
&&
تا جامه نیالایی از خون جگر جانا
\\
ای ماه برآ آخر بر کوری مه رویان
&&
ابری سیه اندرکش در روی قمر جانا
\\
زان روز که زادی تو ای لب شکر از مادر
&&
آوه که چه کاسد شد بازار شکر جانا
\\
گفتی که سلام علیک بگرفت همه عالم
&&
دل سجده درافتاده جان بسته کمر جانا
\\
چون شمع بدم سوزان هر شب به سحر کشته
&&
امروز بنشناسم شب را ز سحر جانا
\\
شمس الحق تبریزی شاهنشه خون ریزی
&&
ای بحر کمربسته پیش تو گهر جانا
\\
\end{longtable}
\end{center}
