\begin{center}
\section*{غزل شماره ۱۵۰۲: ز زندان خلق را آزاد کردم}
\label{sec:1502}
\addcontentsline{toc}{section}{\nameref{sec:1502}}
\begin{longtable}{l p{0.5cm} r}
ز زندان خلق را آزاد کردم
&&
روان عاشقان را شاد کردم
\\
دهان اژدها را بردریدم
&&
طریق عشق را آباد کردم
\\
ز آبی من جهانی برتنیدم
&&
پس آنگه آب را پرباد کردم
\\
ببستم نقش‌ها بر آب کان را
&&
نه بر عاج و نه بر شمشاد کردم
\\
ز شادی نقش خود جان می دراند
&&
که من نقش خودش میعاد کردم
\\
ز چاهی یوسفان را برکشیدم
&&
که از یعقوب ایشان یاد کردم
\\
چو خسرو زلف شیرینان گرفتم
&&
اگر قصد یکی فرهاد کردم
\\
زهی باغی که من ترتیب کردم
&&
زهی شهری که من بنیاد کردم
\\
جهان داند که تا من شاه اویم
&&
بدادم داد ملک و داد کردم
\\
جهان داند که بیرون از جهانم
&&
تصور بهر استشهاد کردم
\\
چه استادان که من شهمات کردم
&&
چه شاگردان که من استاد کردم
\\
بسا شیران که غریدند بر ما
&&
چو روبه عاجز و منقاد کردم
\\
خمش کن آنک او از صلب عشق است
&&
بسستش اینک من ارشاد کردم
\\
ولیک آن را که طوفان بلا برد
&&
فروشد گر چه من فریاد کردم
\\
مگر از قعر طوفانش برآرم
&&
چنانک نیست را ایجاد کردم
\\
برآمد شمس تبریزی بزد تیغ
&&
زبان از تیغ او پولاد کردم
\\
\end{longtable}
\end{center}
