\begin{center}
\section*{غزل شماره ۲۱۴۱: ای تن و جان بنده او بند شکرخنده او}
\label{sec:2141}
\addcontentsline{toc}{section}{\nameref{sec:2141}}
\begin{longtable}{l p{0.5cm} r}
ای تن و جان بنده او بند شکرخنده او
&&
عقل و خرد خیره او دل شکرآکنده او
\\
چیست مراد سر ما ساغر مردافکن او
&&
چیست مراد دل ما دولت پاینده او
\\
چرخ معلق چه بود کهنه ترین خیمه او
&&
رستم و حمزه کی بود کشته و افکنده او
\\
چون سوی مردار رود زنده شود مرد بدو
&&
چون سوی درویش رود برق زند ژنده او
\\
هیچ نرفت و نرود از دل من صورت او
&&
هیچ نبود و نبود همسر و ماننده او
\\
ملک جهان چیست که تا او به جهان فخر کند
&&
فخر جهان راست که او هست خداونده او
\\
ای خنک آن دل که تویی غصه و اندیشه او
&&
ای خنک آن ره که تویی باج ستاننده او
\\
عشق بود دلبر ما نقش نباشد بر ما
&&
صورت و نقشی چه بود با دل زاینده او
\\
گفت برانم پس از این من مگسان را ز شکر
&&
خوش مگسی را که تویی مانع و راننده او
\\
نقش فلک دزد بود کیسه نگهدار از او
&&
دام بود دانه او مرده بود زنده او
\\
بس کن اگر چه که سخن سهل نماید همه را
&&
در دو هزاران نبود یک کس داننده او
\\
\end{longtable}
\end{center}
