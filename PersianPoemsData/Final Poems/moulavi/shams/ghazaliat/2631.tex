\begin{center}
\section*{غزل شماره ۲۶۳۱: تو دوش رهیدی و شب دوش رهیدی}
\label{sec:2631}
\addcontentsline{toc}{section}{\nameref{sec:2631}}
\begin{longtable}{l p{0.5cm} r}
تو دوش رهیدی و شب دوش رهیدی
&&
امروز مکن حیله که آن رفت که دیدی
\\
ما را به حکایت به در خانه ببردی
&&
بر در بنشاندی و تو بر بام دویدی
\\
صد کاسه همسایه مظلوم شکستی
&&
صد کیسه در این راه به حیلت ببریدی
\\
آن کیست که او را به دغل خفته نکردی
&&
وز زیر سر خفته گلیمی نکشیدی
\\
گفتی که از آن عالم کس بازنیامد
&&
امروز ببینی چو بدین حال رسیدی
\\
امروز ببینی که چه مرغی و چه رنگی
&&
کز زخم اجل بند قفس را بدریدی
\\
امروز ببینی که کیان را یله کردی
&&
امروز ببینی که کیان را بگزیدی
\\
یا شیر ز پستان کرامات چشیدی
&&
یا شیر ز پستان سیه دیو مکیدی
\\
ای باز کلاه از سر و روی تو برون شد
&&
خوش بنگر و خوش بشنو آنچ نشنیدی
\\
آن جا بردت پای که در سر هوسش بود
&&
و آن جا بردت دیده که آن جا نگریدی
\\
بر تو زند آن گل که به گلزار بکشتی
&&
در تو خلد آن خار که در یار خلیدی
\\
تلخی دهد امروز تو را در دل و در کام
&&
آن زهرگیایی که در این دشت چریدی
\\
آن آهن تو نرم شد امروز ببینی
&&
که قفل دری یا جهت قفل کلیدی
\\
طوق ملکی این دم اگر گوهر پاکی
&&
رد فلکی این دم اگر زشت و پلیدی
\\
گر آب حیاتی تو و گر آب سیاهی
&&
این چشم ببستی تو در آن چشمه رسیدی
\\
با جمله روان‌ها بپر روح روانی
&&
این است سزای تو گر از نفس جهیدی
\\
با خالق آرام تو آرام گرفتی
&&
وز آب و گل تیره بیگانه رمیدی
\\
امروز تو را بازخرد شعله آن نور
&&
کاین جا ز دل و جان به دل و جانش خریدی
\\
آن سیمبر اندر بر سیمین تو آید
&&
کو را چو نثار زر از این خاک بچیدی
\\
ای عشق ببخشای تو بر حال ضعیفان
&&
کز خاک همان رست که در خاک دمیدی
\\
خامش کن و منمای به هر کس سر دل ز آنک
&&
در دیده هر ذره چو خورشید پدیدی
\\
خاموش و دهان را به خموشی تو دوا کن
&&
زیرا که ز پستان سیه دیو چشیدی
\\
\end{longtable}
\end{center}
