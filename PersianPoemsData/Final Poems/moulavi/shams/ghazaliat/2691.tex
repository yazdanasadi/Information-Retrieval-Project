\begin{center}
\section*{غزل شماره ۲۶۹۱: نگفتم دوش ای زین بخاری}
\label{sec:2691}
\addcontentsline{toc}{section}{\nameref{sec:2691}}
\begin{longtable}{l p{0.5cm} r}
نگفتم دوش ای زین بخاری
&&
که نتوانی رضا دادن به خواری
\\
در آن جان‌ها که شکر روید از حق
&&
شکر باشد ز هر حسیش جاری
\\
اگر صد خنب سرکه درکشد او
&&
نه تلخی بینی او را نی نزاری
\\
خدایت چون سر مستی نداده‌ست
&&
حذر کن تا سر مستی نخاری
\\
از آن سر چون سر جان را شراب است
&&
همی‌نوشد شراب اختیاری
\\
ز تو خنده همی پنهان کند او
&&
که او خمری است و تو مسکین خماری
\\
چو داد آن خواجه را سرکه فروشی
&&
چه شیرین کرد بر وی سوکواری
\\
گوارش خر از آن رخسار چون ماه
&&
کز آن یابند مردان خوشگواری
\\
درآید در تن تو نور آن ماه
&&
چنان کاندر زمین لطف بهاری
\\
ببخشد مر تو را هم خلعت سبز
&&
رهاند مر تو را از خاکساری
\\
تصورها همه زین بوی برده
&&
برون روژیده از دل چون دراری
\\
تفضل ایها الساقی و اوفر
&&
و لکن لا براح مستعار
\\
و صبحنا بخمر مستطاب
&&
فان الیمن جما فی ابتکار
\\
و مسینا بخمر من صبوح
&&
و دم و اسلم ایا خیر المداری
\\
\end{longtable}
\end{center}
