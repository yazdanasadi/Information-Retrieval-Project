\begin{center}
\section*{غزل شماره ۳۱۸۰: اخلائی! اخلائی! صفونی عند مولایی}
\label{sec:3180}
\addcontentsline{toc}{section}{\nameref{sec:3180}}
\begin{longtable}{l p{0.5cm} r}
اخلائی! اخلائی! صفونی عند مولایی
&&
و قولوا ان ادوایی قد استولت لافنایی
\\
اخلایی اخلایی، مرا جانیست سودایی
&&
چو طوفان بر سرم بارد، غم و سودا ز بالایی
\\
و قولوا: « ایها المولی، الا یا نظرةالدنیا
&&
فجدلی نظرة احیا، اذا ما شت ابقایی
\\
اخلایی اخلایی،بشویید از دل من دست
&&
کزین اندیشه دادم دل به دست موج دریایی
\\
یقول العشق لی یا هو فصیحا فاتحا فاه
&&
فمالم تأت لقیاه متی تفرح بلقایی؟!
\\
اخلایی اخلایی، خبر آن کارفرما را
&&
که سخت از کار رفتم من، مرا کاری بفرمایی
\\
فجد بالروح یا ساقی، و رو منه اشواقی
&&
ولا تبق لنا باقی، سوی تصویر مولایی
\\
اخلایی اخلایی، امانت دست من گیرید
&&
که مستم، ره نمی‌دانم، بدان معشوق زیبایی
\\
فجد بالراح لی شکرا، ولا تبق لنا فکرا
&&
فها ان لم تکن صرفا، فما زجه ببلوایی
\\
اخلایی اخلایی، به کوی او سپاریدم
&&
بران خاکم بخسبانید کن سرمه‌ست و بینایی
\\
الا یا ساقی الواهب، ادر من خمرة الراهب
&&
فلا ندری من‌الذاهب، ولا ندری من‌الجایی
\\
اخلایی اخلایی خبر جان را که می‌دانم
&&
که تو بر راه اندیشه حریفان را همی پایی
\\
مغانی الروح! غنوالی، وبالاوتار طنوالی
&&
و بالالحان حنوالی غنا کم صفو مغنایی
\\
اخلایی اخلایی، که هر روزی یکی شوری
&&
به کوی لولیان افتد، ازان لولی سرنایی
\\
و تبریزا صفوالیها، و شمس‌الدین تالیها
&&
فهو مولی موالیها، و مولا کل علیایی
\\
اخلایی اخلایی، زبان پارسی می‌گو
&&
که نبود شرط در حلقه، شکر خوردن به تنهایی
\\
\end{longtable}
\end{center}
