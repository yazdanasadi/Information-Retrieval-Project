\begin{center}
\section*{غزل شماره ۷۹۹: سفره کهنه کجا درخور نان تو بود}
\label{sec:0799}
\addcontentsline{toc}{section}{\nameref{sec:0799}}
\begin{longtable}{l p{0.5cm} r}
سفره کهنه کجا درخور نان تو بود
&&
خرمگس هم ز کجا صاحب خوان تو بود
\\
در زمانی که بگویی هله هان تان چه کمست
&&
کو زبانی که مجابات زبان تو بود
\\
گر سیه روی بود زنگی و هندوی توست
&&
چه غمست از سیهی چونک از آن تو بود
\\
ببری در خم خویش و خوش و یک رنگ کنی
&&
تا همه روح بود فر و نشان تو بود
\\
ترس را سر ببر و گردن تعظیم بزن
&&
در مقامی که عطاها و امان تو بود
\\
ما همه بر سر راهیم و جهانی گذرست
&&
چشم روشن نفسی کان ز جهان تو بود
\\
دل اگر بی‌ادبی کرد بر این صبر مگیر
&&
طعمش بد که در این جنگ عوان تو بود
\\
سگ به هر سو که چخد نعره به کوی تو زند
&&
شیرگیرش که بود تا که زیان تو بود
\\
هین صبوحست بده می که همه مخموریم
&&
تا که جان یک نفسی مست ضمان تو بود
\\
در قدح درنگری زود فرح بخش شود
&&
گرگ چون دید سگ کهف شبان تو بود
\\
همه خفتند و دو مخمور چنین بیدارند
&&
نظری کن سوی خم‌ها که نهان تو بود
\\
سر و پا مست شود هر چه تو خواهی بشود
&&
برسد چون نرسد چونک رسان تو بود
\\
هله درویش بخور نک قدح زفت رسید
&&
سست بودن چه بود چونک اوان تو بود
\\
هله امروز نشستیم به عشرت تا شب
&&
چه کم آید می و مطرب چو بیان تو بود
\\
خاک بر سر همه را دامن این دولت گیر
&&
چو بر این خاک نشستی همه آن تو بود
\\
می او خور همه او شو سر شش گوش مباش
&&
مطلب که دو سه خر گوش کشان تو بود
\\
\end{longtable}
\end{center}
