\begin{center}
\section*{غزل شماره ۱۴۶: در صفای باده بنما ساقیا تو رنگ ما}
\label{sec:0146}
\addcontentsline{toc}{section}{\nameref{sec:0146}}
\begin{longtable}{l p{0.5cm} r}
در صفای باده بنما ساقیا تو رنگ ما
&&
محومان کن تا رهد هر دو جهان از ننگ ما
\\
باد باده برگمار از لطف خود تا برپرد
&&
در هوا ما را که تا خفت پذیرد سنگ ما
\\
بر کمیت می تو جان را کن سوار راه عشق
&&
تا چو یک گامی بود بر ما دو صد فرسنگ ما
\\
وارهان این جان ما را تو به رطلی می از آنک
&&
خون چکید از بینی و چشم دل آونگ ما
\\
ساقیا تو تیزتر رو این نمی‌بینی که بس
&&
می‌دود اندر عقب اندیشه‌های لنگ ما
\\
در طرب اندیشه‌ها خرسنگ باشد جان گداز
&&
از میان راه برگیرید این خرسنگ ما
\\
در نوای عشق شمس الدین تبریزی بزن
&&
مطرب تبریز در پرده عشاقی چنگ ما
\\
\end{longtable}
\end{center}
