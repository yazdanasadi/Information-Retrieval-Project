\begin{center}
\section*{غزل شماره ۷۸۶: آنک عکس رخ او راه ثریا بزند}
\label{sec:0786}
\addcontentsline{toc}{section}{\nameref{sec:0786}}
\begin{longtable}{l p{0.5cm} r}
آنک عکس رخ او راه ثریا بزند
&&
گر ره قافله عقل زند تا بزند
\\
آنک نقل و می او در ره صوفی نقدست
&&
رسدش گر به نظر گردن فردا بزند
\\
گر پراکنده دلی دامن دل گیر که دل
&&
خیمه امن و امان بر سر غوغا بزند
\\
عمری باید تا دیو از او بگریزد
&&
احمدی باید تا راه چلیپا بزند
\\
در هر آن کنج دلی که غم تو معتکفست
&&
نیم شب تابش خورشید بر آن جا بزند
\\
عارفا بهر سه نان دعوت جان را مگذار
&&
تا سنانت چو علی در صف هیجا بزند
\\
زین گذر کن که رسیدست شهنشاه کرم
&&
خیز تا جان تو بر عیش و تماشا بزند
\\
کف حاجت بگشا جام الهی بستان
&&
تا شعاع می جان بر رخ و سیما بزند
\\
رخ و سیمای تو زان رونق و نوری گیرد
&&
که کف شق قمر بر مه بالا بزند
\\
بر سرت بردود و عقل دهد مغز تو را
&&
عقل پرمغز تو پا بر سر جوزا بزند
\\
خواجه بربند دو گوش و بگریز از سخنم
&&
ور نه در رخت تو هم آتش یغما بزند
\\
بگریز از من و از طالع شیرافکن من
&&
کاخترم کوکبه بر آدم و حوا بزند
\\
هین خمش باش که نور تو چو بر دل‌ها زد
&&
نور محسوس شود بر سر و بر پا بزند
\\
\end{longtable}
\end{center}
