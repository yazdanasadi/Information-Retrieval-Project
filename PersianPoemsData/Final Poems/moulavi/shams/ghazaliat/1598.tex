\begin{center}
\section*{غزل شماره ۱۵۹۸: ای خوشا روزا که ما معشوق را مهمان کنیم}
\label{sec:1598}
\addcontentsline{toc}{section}{\nameref{sec:1598}}
\begin{longtable}{l p{0.5cm} r}
ای خوشا روزا که ما معشوق را مهمان کنیم
&&
دیده از روی نگارینش نگارستان کنیم
\\
گر ز داغ هجر او دردی است در دل‌های ما
&&
ز آفتاب روی او آن درد را درمان کنیم
\\
چون به دست ما سپارد زلف مشک افشان خویش
&&
پیش مشک افشان او شاید که جان قربان کنیم
\\
آن سر زلفش که بازی می کند از باد عشق
&&
میل دارد تا که ما دل را در او پیچان کنیم
\\
او به آزار دل ما هر چه خواهد آن کند
&&
ما به فرمان دل او هر چه گوید آن کنیم
\\
این کنیم و صد چنین و منتش بر جان ماست
&&
جان و دل خدمت دهیم و خدمت سلطان کنیم
\\
آفتاب رحمتش در خاک ما درتافته‌ست
&&
ذره‌های خاک خود را پیش او رقصان کنیم
\\
ذره‌های تیره را در نور او روشن کنیم
&&
چشم‌های خیره را در روی او تابان کنیم
\\
چوب خشک جسم ما را کو به مانند عصاست
&&
در کف موسی عشقش معجز ثعبان کنیم
\\
گر عجب‌های جهان حیران شود در ما رواست
&&
کاین چنین فرعون را ما موسی عمران کنیم
\\
نیمه‌ای گفتیم و باقی نیم کاران بو برند
&&
یا برای روز پنهان نیمه را پنهان کنیم
\\
\end{longtable}
\end{center}
