\begin{center}
\section*{غزل شماره ۱۱۲۵: چون سر کس نیستت فتنه مکن دل مبر}
\label{sec:1125}
\addcontentsline{toc}{section}{\nameref{sec:1125}}
\begin{longtable}{l p{0.5cm} r}
چون سر کس نیستت فتنه مکن دل مبر
&&
چونک ببردی دلی باز مرانش ز در
\\
چشم تو چون رهزند ره زده را ره نما
&&
زلفت اگر سر کشد عشوه هندو مخر
\\
عشق بود گلستان پرورش از وی ستان
&&
از شجره فقر شد باغ درون پرثمر
\\
جمله ثمر ز آفتاب پخته و شیرین شود
&&
خواب و خورم را ببر تا برسم نزد خور
\\
طبع جهان کهنه دان عاشق او کهنه دوز
&&
تازه و ترست عشق طالب او تازه تر
\\
عشق برد جوبجو تا لب دریای هو
&&
کهنه خران را بگو اسکی ببج کآمده ور
\\
هر کس یاری گزید دل سوی دلبر پرید
&&
نحس قرین زحل شمس قرین قمر
\\
دل خود از این عام نیست با کسش آرام نیست
&&
گر تو قلندردلی نیست قلندر بشر
\\
تن چو ز آب منیست آب به پستی رود
&&
اصل دل از آتشست او نرود جز زبر
\\
غیر دل و غیر تن هست تو را گوهری
&&
بی‌خبری زان گهر تا نشوی بی‌خبر
\\
\end{longtable}
\end{center}
