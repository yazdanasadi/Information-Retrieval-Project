\begin{center}
\section*{غزل شماره ۲۶۰۹: در عشق کجا باشد مانند تو عشقینی}
\label{sec:2609}
\addcontentsline{toc}{section}{\nameref{sec:2609}}
\begin{longtable}{l p{0.5cm} r}
در عشق کجا باشد مانند تو عشقینی
&&
شاهان ز هوای تو در خرقه دلقینی
\\
بر خوان تو استاده هر گوشه سلیمانی
&&
وز غایت مستی تو همکاسه مسکینی
\\
بس جان گزین بوده سلطان یقین بوده
&&
سردفتر دین بوده از عشق تو بی‌دینی
\\
کو گوهر جان بودن کو حرف بپیمودن
&&
کو سینه ره بینی کو دیده شه بینی
\\
هر مست میت خورده دو دست برآورده
&&
کاین عشق فزون بادا وز هر طرف آمینی
\\
گویند بخوان یاسین تا عشق شود تسکین
&&
جانی که به لب آمد چه سود ز یاسینی
\\
آن دلشده خاکی کز عشق زمین بوسد
&&
در دولت تو بنهد بر پشت فلک زینی
\\
آوه خنک آن دل را کو لازم آن جان شد
&&
گه باده جان گیرد گه طره مشکینی
\\
هرگز نکند ما را عالم به جوال اندر
&&
کز شمس حق تبریز پر کردم خرجینی
\\
\end{longtable}
\end{center}
