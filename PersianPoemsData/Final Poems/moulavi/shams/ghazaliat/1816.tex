\begin{center}
\section*{غزل شماره ۱۸۱۶: آینه‌ای بزدایم از جهت منظر من}
\label{sec:1816}
\addcontentsline{toc}{section}{\nameref{sec:1816}}
\begin{longtable}{l p{0.5cm} r}
آینه‌ای بزدایم از جهت منظر من
&&
وای از این خاک تنم تیره دل اکدر من
\\
رفت شب و این دل من پاک نشد از گل من
&&
ساقی مستقبل من کو قدح احمر من
\\
رفت دریغا خر من مرد به ناگه خر من
&&
شکر که سرگین خری دور شده‌ست از در من
\\
مرگ خران سخت بود در حق من بخت بود
&&
زانک چو خر دور شود باشد عیسی بر من
\\
از پی غربیل علف چند شدم مات و تلف
&&
چند شدم لاغر و کژ بهر خر لاغر من
\\
آنچ که خر کرد به من گرگ درنده نکند
&&
رفت ز درد و غم او حق خدا اکثر من
\\
تلخی من خامی من خواری و بدنامی من
&&
خون دل آشامی من خاک از او بر سر من
\\
شارق من فارق من از نظر خالق من
&&
شمع کشی دیده کنی در نظر و منظر من
\\
\end{longtable}
\end{center}
