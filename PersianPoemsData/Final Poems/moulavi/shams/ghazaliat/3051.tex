\begin{center}
\section*{غزل شماره ۳۰۵۱: به عاقبت بپریدی و در نهان رفتی}
\label{sec:3051}
\addcontentsline{toc}{section}{\nameref{sec:3051}}
\begin{longtable}{l p{0.5cm} r}
به عاقبت بپریدی و در نهان رفتی
&&
عجب عجب به کدامین ره از جهان رفتی
\\
بسی زدی پر و بال و قفس دراشکستی
&&
هوا گرفتی و سوی جهان جان رفتی
\\
تو باز خاص بدی در وثاق پیرزنی
&&
چو طبل باز شنیدی به لامکان رفتی
\\
بدی تو بلبل مستی میانه جغدان
&&
رسید بوی گلستان به گل ستان رفتی
\\
بسی خمار کشیدی از این خمیر ترش
&&
به عاقبت به خرابات جاودان رفتی
\\
پی نشانه دولت چو تیر راست شدی
&&
بدان نشانه پریدی و زین کمان رفتی
\\
نشان‌های کژت داد این جهان چو غول
&&
نشان گذاشتی و سوی بی‌نشان رفتی
\\
تو تاج را چه کنی چونک آفتاب شدی
&&
کمر چرا طلبی چونک از میان رفتی
\\
دو چشم کشته شنیدم که سوی جان نگرد
&&
چرا به جان نگری چون به جان جان رفتی
\\
دلا چه نادره مرغی که در شکار شکور
&&
تو با دو پر چو سپر جانب سنان رفتی
\\
گل از خزان بگریزد عجب چه شوخ گلی
&&
که پیش باد خزانی خزان خزان رفتی
\\
ز آسمان تو چو باران به بام عالم خاک
&&
به هر طرف بدویدی به ناودان رفتی
\\
خموش باش مکش رنج گفت و گوی بخسب
&&
که در پناه چنان یار مهربان رفتی
\\
\end{longtable}
\end{center}
