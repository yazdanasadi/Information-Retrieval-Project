\begin{center}
\section*{غزل شماره ۳۱۷۱: خشم مرو خواجه! پشیمان شوی}
\label{sec:3171}
\addcontentsline{toc}{section}{\nameref{sec:3171}}
\begin{longtable}{l p{0.5cm} r}
خشم مرو خواجه! پشیمان شوی
&&
جمع نشین، ورنه پریشان شوی
\\
طیره مشو خیره مرو زین چمن
&&
ورنه چو جغدان سوی ویران شوی
\\
گر بگریزی ز خراجات شهر
&&
بارکش غول بیابان شوی
\\
گر تو ز خورشید حمل سر کشی
&&
بفسری و برف زمستان شوی
\\
روی به جنگ آر و به صف شیروار
&&
ورنه چو گربه تو در انبان شوی
\\
کم خور ازین پاچهٔ گاو، ای ملک
&&
سیر چریدی، خر شیطان شوی
\\
کافر نفست چو زبون تو شد
&&
گر همه کفری همه ایمان شوی
\\
روی مکن ترش ز تلخی یار
&&
تا ز عنایت گل خندان شوی
\\
دست و دهان را چو بشویی ز حرص
&&
صاحب و همکاسهٔ سلطان شوی
\\
ای دل، یک لحظه تو دیوانهٔ
&&
با دمی خواجهٔ دیوان شوی
\\
گاه بدزدی، ره ایرن زنی
&&
گاه روی شحنهٔ توران شوی
\\
گه ز (سپاهان) و حجاز) و (عراق)
&&
مطرب آن ماه خراسان شوی
\\
بوقلمونی چه شود گر چو عقل
&&
یک صفت و یک دل و یکسان شوی؟
\\
گر نکنی این همه خاموش باش
&&
تا به خموشی همگی جان شوی
\\
روی به شمس الحق تبریز کن
&&
تا ملک ملک سلیمان شوی
\\
\end{longtable}
\end{center}
