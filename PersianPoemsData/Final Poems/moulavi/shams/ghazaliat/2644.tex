\begin{center}
\section*{غزل شماره ۲۶۴۴: یک روز مرا بر لب خود میر نکردی}
\label{sec:2644}
\addcontentsline{toc}{section}{\nameref{sec:2644}}
\begin{longtable}{l p{0.5cm} r}
یک روز مرا بر لب خود میر نکردی
&&
وز لعل لبت جامگی تقریر نکردی
\\
زان شب که سر زلف تو در خواب بدیدم
&&
حیران و پریشانم و تعبیر نکردی
\\
یک عالم و عاقل به جهان نیست که او را
&&
دیوانه آن زلف چو زنجیر نکردی
\\
بگریست بسی از غم تو طفل دو چشمم
&&
وز سنگ دلی در دهنش شیر نکردی
\\
در کعبه خوبی تو احرام ببستیم
&&
بس تلبیه گفتیم و تو تکبیر نکردی
\\
بگرفت دلم در غمت ای سرو جوان بخت
&&
شد پیر دلم پیروی پیر نکردی
\\
با قوس دو ابروی تو یک دل به جهان نیست
&&
تا خسته بدان غمزه چون تیر نکردی
\\
بس عقل که در آیت حسن تو فروماند
&&
وز وی به کرم روزی تفسیر نکردی
\\
در بردن جان‌ها و در آزردن جان‌ها
&&
الحق صنما هیچ تو تقصیر نکردی
\\
در کشتنم ای دلبر خون خوار بکردم
&&
صد لابه و یک ساعت تأخیر نکردی
\\
در آتش عشق تو دلم سوخت به یک بار
&&
وز بهر دوا قرص تباشیر نکردی
\\
بیمار شدم از غم هجر تو و روزی
&&
از بهر من خسته تو تدبیر نکردی
\\
خورشید رخت با زحل زلف سیاهت
&&
صد بار قران کرد و تو تأثیر نکردی
\\
بر خاک درت روی نهادم ز سر عجز
&&
وز قصه هجرانم تحریر نکردی
\\
خامش شوم و هیچ نگویم پس از این من
&&
هر چاکر دیرینه چو توفیر نکردی
\\
\end{longtable}
\end{center}
