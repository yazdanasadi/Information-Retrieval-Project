\begin{center}
\section*{غزل شماره ۱۱۱۷: دل ناظر جمال تو آن گاه انتظار}
\label{sec:1117}
\addcontentsline{toc}{section}{\nameref{sec:1117}}
\begin{longtable}{l p{0.5cm} r}
دل ناظر جمال تو آن گاه انتظار
&&
جان مست گلستان تو آن گاه خار خار
\\
هر دم ز پرتو نظر او به سوی دل
&&
حوریست بر یمین و نگاریست بر یسار
\\
هر صبحدم که دام شب و روز بردریم
&&
از دوست بوسه‌ای و ز ما سجده صد هزار
\\
امسال حلقه ایست ز سودای عاشقان
&&
گر نیست بازگشت در این عشق عمر پار
\\
بنواز چنگ عشق تو به نغمات لم یزل
&&
کز چنگ‌های عشق تو جانست تار تار
\\
اندر هوای عشق تو از تابش حیات
&&
بگرفته بیخ‌های درخت و دهد ثمار
\\
غوطی بخورد جان به تک بحر و شد گهر
&&
این بحر و این گهر ز پی لعل توست زار
\\
از نغمه‌های طوطی شکرستان توست
&&
در رقص شاخ بید و دو دستک زنان چنار
\\
از بعد ماجرای صفا صوفیان عشق
&&
گیرند یک دگر را چون مستیان کنار
\\
مستانه جان برون جهد از وحدت الست
&&
چون سیل سوی بحر نه آرام و نه قرار
\\
جزوی چو تیر جسته ز قبضه کمان کل
&&
او را نشانه نیست به جز کل و نی گذار
\\
جانیست خوش برون شده از صد هزار پوست
&&
در چاربالش ابد او راست کار و بار
\\
جان‌های صادقان همه در وی زنند چنگ
&&
تا بانوا شوند از آن جان نامدار
\\
جان‌ها گرفته دامنش از عشق و او چو قطب
&&
بگرفته دامن ازل محض مردوار
\\
تبریز رو دلا و ز شمس حق این بپرس
&&
تا بر براق سر معانی شوی سوار
\\
\end{longtable}
\end{center}
