\begin{center}
\section*{غزل شماره ۶۸۷: کی باشد کاین قفص چمن گردد}
\label{sec:0687}
\addcontentsline{toc}{section}{\nameref{sec:0687}}
\begin{longtable}{l p{0.5cm} r}
کی باشد کاین قفس چمن گردد
&&
و اندرخور گام و کام من گردد
\\
این زهر کشنده انگبین بخشد
&&
وین خار خلنده یاسمن گردد
\\
آن ماه دو هفته در کنار آید
&&
وز غصه حسود ممتحن گردد
\\
آن یوسف مصر الصلا گوید
&&
یعقوب قرین پیرهن گردد
\\
بر ما خورشید سایه اندازد
&&
وان شمع مقیم این لگن گردد
\\
آن چنگ نشاط ساز نو یابد
&&
وین گوش حریف تن تنن گردد
\\
در خرمن ماه سنبله کوبیم
&&
چون نور سهیل در یمن گردد
\\
خم‌های شراب عشق برجوشد
&&
هنگام کباب و بابزن گردد
\\
سیمرغ هوای ما ز قاف آید
&&
دام شبلی و بوالحسن گردد
\\
هر ذره مثال آفتاب آید
&&
هر قطره به موهبت عدن گردد
\\
هر بره ز گرگ شیر آشامد
&&
هر پیل انیس کرگدن گردد
\\
ز انبوهی دلبران و مه رویان
&&
هر گوشه شهر ما ختن گردد
\\
هر عاشق بی‌مراد سرگشته
&&
مستغرق عشق باختن گردد
\\
چون قالب مرده جان نو یابد
&&
فارغ ز لفافه و کفن گردد
\\
آن عقل فضول در جنون آید
&&
هوش از بن گوش مرتهن گردد
\\
جان و دل صد هزار دیوانه
&&
از بوسه یار خوش دهن گردد
\\
آن روز که جان جمله مخموران
&&
ساقی هزار انجمن گردد
\\
وان کس که سبال می‌زدی بر عشق
&&
در عشق شهیر مرد و زن گردد
\\
در چاه فراق هر کی افتاده‌ست
&&
ره یابد و همره رسن گردد
\\
باقیش مگو درون دل می‌دار
&&
آن به که سخن در آن وطن گردد
\\
\end{longtable}
\end{center}
