\begin{center}
\section*{غزل شماره ۱۰۹: بکت عینی غداه البین دمعا}
\label{sec:0109}
\addcontentsline{toc}{section}{\nameref{sec:0109}}
\begin{longtable}{l p{0.5cm} r}
بکت عینی غداه البین دمعا
&&
و اخری بالبکا بخلت علینا
\\
فعاقبت التی بخلت علینا
&&
بان غمضتها یوم التقینا
\\
چه مرد آن عتابم خیز یارا
&&
بده آن جام مالامال صهبا
\\
نرنجم ز آنچ مردم می‌برنجند
&&
که پیشم جمله جان‌ها هست یکتا
\\
اگر چه پوستینی بازگونه
&&
بپوشیدست این اجسام بر ما
\\
تو را در پوستین من می‌شناسم
&&
همان جان منی در پوست جانا
\\
بدرم پوست را تو هم بدران
&&
چرا سازیم با خود جنگ و هیجا
\\
یکی جانیم در اجسام مفرق
&&
اگر خردیم اگر پیریم و برنا
\\
چراغک‌هاست کآتش را جدا کرد
&&
یکی اصلست ایشان را و منش
\\
یکی طبع و یکی رنگ و یکی خوی
&&
که سرهاشان نباشد غیر پاها
\\
در این تقریر برهان‌هاست در دل
&&
به سر با تو بگویم یا به اخفا
\\
غلط خود تو بگویی با تو آن را
&&
چه تو بر توست بنگر این تماشا
\\
\end{longtable}
\end{center}
