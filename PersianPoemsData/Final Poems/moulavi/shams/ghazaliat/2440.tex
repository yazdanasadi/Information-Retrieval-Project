\begin{center}
\section*{غزل شماره ۲۴۴۰: ای آفتاب سرکشان با کهکشان آمیختی}
\label{sec:2440}
\addcontentsline{toc}{section}{\nameref{sec:2440}}
\begin{longtable}{l p{0.5cm} r}
ای آفتاب سرکشان با کهکشان آمیختی
&&
مانند شیر و انگبین با بندگان آمیختی
\\
یا چون شراب جان فزا هر جزو را دادی طرب
&&
یا همچو یاران کرم با خاکدان آمیختی
\\
یا همچو عشق جان فدا در لاابالی ماردی
&&
با عقل پرحرص شحیح خرده دان آمیختی
\\
ای آتش فرمانروا در آب مسکن ساختی
&&
وی نرگس عالی نظر با ارغوان آمیختی
\\
چندان در آتش درشدی کآتش در آتش درزدی
&&
چندان نشان جستی که تو با بی‌نشان آمیختی
\\
ای سر الله الصمد ای بازگشت نیک و بد
&&
پهلو تهی کردی ز خود با پهلوان آمیختی
\\
جان‌ها بجستندت بسی بویی نبرد از تو کسی
&&
آیس شدند و خسته دل خود ناگهان آمیختی
\\
از جنس نبود حیرتی بی‌جنس نبود الفتی
&&
تو این نه‌ای و آن نه‌ای با این و آن آمیختی
\\
هر دو جهان مهمان تو بنشسته گرد خوان تو
&&
صد گونه نعمت ریختی با میهمان آمیختی
\\
آمیختی چندانک او خود را نمی‌داند ز تو
&&
آری کجا داند چو تو با تن چو جان آمیختی
\\
پیرا جوان گردی چو تو سرسبز این گلشن شدی
&&
تیرا به صیدی دررسی چون با کمان آمیختی
\\
ای دولت و بخت همه دزدیده‌ای رخت همه
&&
چالاک رهزن آمدی با کاروان آمیختی
\\
چرخ و فلک ره می‌رود تا تو رهش آموختی
&&
جان و جهان بر می‌پرد تا با جهان آمیختی
\\
حیرانم اندر لطف تو کاین قهر چون سر می‌کشد
&&
گردن چو قصابان مگر با گردران آمیختی
\\
خوبان یوسف چهره را آموختی عاشق کشی
&&
و آن خار چون عفریت را با گلستان آمیختی
\\
این را رها کن عارفا آن را نظر کن کز صفا
&&
رستی ز اجزای زمین با آسمان آمیختی
\\
رستی ز دام ای مرغ جان در شاخ گل آویختی
&&
جستی ز وسواس جنان و اندر جنان آمیختی
\\
از بام گردون آمدی ای آب آب زندگی
&&
از بام ما جولان زدی با ناودان آمیختی
\\
شب دزد کی یابد تو را چون نیستی اندر سرا
&&
بر بام چوبک می‌زنی با پاسبان آمیختی
\\
اسرار این را مو به مو بی‌پرده و حرفی بگو
&&
ای آنک حرف و لحن را اندر بیان آمیختی
\\
\end{longtable}
\end{center}
