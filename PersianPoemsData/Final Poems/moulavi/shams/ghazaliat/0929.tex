\begin{center}
\section*{غزل شماره ۹۲۹: ز عشق آن رخ خوب تو ای اصول مراد}
\label{sec:0929}
\addcontentsline{toc}{section}{\nameref{sec:0929}}
\begin{longtable}{l p{0.5cm} r}
ز عشق آن رخ خوب تو ای اصول مراد
&&
هر آن که توبه کند توبه‌اش قبول مباد
\\
هزار شکر و هزاران سپاس یزدان را
&&
که عشق تو به جهان پر و بال بازگشاد
\\
در آرزوی صباح جمال تو عمری
&&
جهان پیر همی‌خواند هر سحر اوراد
\\
برادری بنمودی شهنشهی کردی
&&
چه داد ماند که آن حسن و خوبی تو نداد
\\
شنیده‌ایم که یوسف نخفت شب ده سال
&&
برادران را از حق بخواست آن شه زاد
\\
که ای خدای اگر عفوشان کنی کردی
&&
وگر نه درفکنم صد فغان در این بنیاد
\\
مگیر یا رب از ایشان که بس پشیمانند
&&
از آن گناه کز ایشان به ناگهان افتاد
\\
دو پای یوسف آماس کرد از شبخیز
&&
به درد آمد چشمش ز گریه و فریاد
\\
غریو در ملکوت و فرشتگان افتاد
&&
که بهر لطف بجوشید و بندها بگشاد
\\
رسید چارده خلعت که هر چهارده تان
&&
پیمبرید و رسولید و سرور عباد
\\
چنین بود شب و روز اجتهاد پیران را
&&
که خلق را برهانند از عذاب و فساد
\\
کنند کار کسی را تمام و برگذرند
&&
که جز خدای نداند زهی کریم و جواد
\\
چو خضر سوی بحار ایلیاس در خشکی
&&
برای گم شدگان می‌کنند استمداد
\\
دهند گنج روان و برند رنج روان
&&
دهند خلعت اطلس برون کنند لباد
\\
بس است باقی این را بگویمت فردا
&&
شب ار چه ماه بود نیست بی‌ظلام و سواد
\\
\end{longtable}
\end{center}
