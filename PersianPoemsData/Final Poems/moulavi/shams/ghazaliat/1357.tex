\begin{center}
\section*{غزل شماره ۱۳۵۷: پیام کرد مرا بامداد بحر عسل}
\label{sec:1357}
\addcontentsline{toc}{section}{\nameref{sec:1357}}
\begin{longtable}{l p{0.5cm} r}
پیام کرد مرا بامداد بحر عسل
&&
که موج موج عسل بین به چشم خلق غزل
\\
به روزه دار نیاید ز آب جز بانگی
&&
ولیک عاقبت آن بانگ هم رسد به عمل
\\
سماع شرفه آبست و تشنگان در رقص
&&
حیات یابی از این بانگ آب اقل اقل
\\
بگوید آب ز من رسته‌ای به من آیی
&&
به آخر آن جا آیی که بوده‌ای اول
\\
به جان و سر که از این آب بر سر ار ریزد
&&
هزار طره بروید ز مشک بر سر کل
\\
شراب خوار که نامیخت با شراب این آب
&&
کشد خمار پیاپی تو باش لاتعجل
\\
\end{longtable}
\end{center}
