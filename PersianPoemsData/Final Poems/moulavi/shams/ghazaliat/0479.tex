\begin{center}
\section*{غزل شماره ۴۷۹: ستیزه کن که ز خوبان ستیزه شیرینست}
\label{sec:0479}
\addcontentsline{toc}{section}{\nameref{sec:0479}}
\begin{longtable}{l p{0.5cm} r}
ستیزه کن که ز خوبان ستیزه شیرینست
&&
بهانه کن که بتان را بهانه آیینست
\\
از آن لب شکرینت بهانه‌های دروغ
&&
به جای فاتحه و کاف‌ها و یاسینست
\\
وفا طمع نکنم زانک جور خوبان را
&&
طبیعت است و سرشت است و عادت و دینست
\\
اگر ترش کنی و رو ز ما بگردانی
&&
به قاصد است و به مکر است و آن دروغینست
\\
ز دست غیر تو اندر دهان من حلوا
&&
به جان پاک عزیزان که گرز رویینست
\\
هزار وعده ده آنگه خلاف کن همه را
&&
که آن سراب که ارزد صد آب خوش اینست
\\
زر او دهد که رخش از فراق همچو زر است
&&
چرا دهد زر و سیم آن پری که سیمینست
\\
جواب همچو شکر او دهد که محتاج است
&&
جواب تلخ تو را صد هزار تمکینست
\\
جمال و حسن تو گنج است و خوی بد چون مار
&&
بقای گنج تو بادا که آن برونینست
\\
قماش هستی ما را به ناز خویش بسوز
&&
که آن زکات لطیفت نصیب مسکینست
\\
برون در همه را چون سگان کو بنشان
&&
که در شرف سر کوی تو طور سینینست
\\
خورند چوب خلیفه شهان چو شاه شوند
&&
جفای عشق کشیدن فن سلاطین است
\\
امام فاتحه خواند ملک کند آمین
&&
مرا چو فاتحه خواندم امید آمینست
\\
هر آن فریب کز اندیشه تو می‌زاید
&&
هزار گوهر و لعلش بها و کابینست
\\
چنانک مدرسه فقه را برون شوها است
&&
بدانک مدرسه عشق را قوانینست
\\
خمش کنیم که تا شرح آن بگوید شاه
&&
که زنده شخص جهان زان گزیده تلقینست
\\
\end{longtable}
\end{center}
