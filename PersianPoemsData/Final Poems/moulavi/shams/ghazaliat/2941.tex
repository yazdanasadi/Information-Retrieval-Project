\begin{center}
\section*{غزل شماره ۲۹۴۱: ای گوهر خدایی آیینه معانی}
\label{sec:2941}
\addcontentsline{toc}{section}{\nameref{sec:2941}}
\begin{longtable}{l p{0.5cm} r}
ای گوهر خدایی آیینه معانی
&&
هر دم ز تاب رویت بر عرش ارمغانی
\\
عرش از خدای پرسد کاین تاب کیست بر من
&&
فرمایدش ز غیرت کاین تاب را ندانی
\\
از غیرت الهی در عرش حیرت افتد
&&
زیرا ز غیرت آمد پیغام لن ترانی
\\
زان تاب اگر شعاعی بر آسمان رسیدی
&&
از آسمان نمودی صد ماه آسمانی
\\
اندر جمال هر مه لطف ازل نمودی
&&
هر عاشقی بدیدی مقصودهای جانی
\\
در راه ره روان را رنج و طلب نبودی
&&
خوف فنا نبودی اندر جهان فانی
\\
یک بار دردمیدی تا جان گرفت قالب
&&
دردم تو بار دیگر تا جان شود عیانی
\\
از یک شعاع رویت چون لامکان مکان شد
&&
هم برق تو رساند او را به لامکانی
\\
انگشتری لعلت بر نقد عرضه فرما
&&
تا نعره‌ها برآید از لعل‌های کانی
\\
یک جام مان بدادی تا رخت‌ها گرو شد
&&
جامی دگر از آن می هم چاره کن تو دانی
\\
جانی رسید ما را از شمس حق تبریز
&&
کان جان همی‌نماید در غیب دلستانی
\\
\end{longtable}
\end{center}
