\begin{center}
\section*{غزل شماره ۱۲۰۵: سیر نگشت جان من بس مکن و مگو که بس}
\label{sec:1205}
\addcontentsline{toc}{section}{\nameref{sec:1205}}
\begin{longtable}{l p{0.5cm} r}
سیر نگشت جان من بس مکن و مگو که بس
&&
گر چه ملول گشته‌ای کم نزنی ز هیچ کس
\\
چونک رسول از قنق گشت ملول و شد ترش
&&
ناصح ایزدی ورا کرد عتاب در عبس
\\
گر نکنی موافقت درد دلی بگیردت
&&
همنفسی خوش است خوش هین مگریز یک نفس
\\
ذوق گرفت هر چه او پخت میان جنس خود
&&
ما بپزیم هم به هم ما نه کمیم از عدس
\\
من نبرم ز سرخوشان خاصه از این شکرکشان
&&
مرگ بود فراقشان مرگ که را بود هوس
\\
دوش حریف مست من داد سبو به دست من
&&
بشکنم آن سبوی را بر سر نفس مرتبس
\\
نفس ضعیف معده را من نکنم حریف خود
&&
زانک خدوک می‌شود خوان مرا از این مگس
\\
من پس و پیش ننگرم پرده شرم بردرم
&&
زانک کمند سکر می می‌کشدم ز پیش و پس
\\
خوش سحری که روی او باشد آفتاب ما
&&
شاد شبی که باشد او بر سر کوی دل عسس
\\
آمد عشق چاشتی شکل طبیب پیش من
&&
دست نهاد بر رگم گفت ضعیف شد مجس
\\
گفت کباب خور پی قوت دل بگفتمش
&&
دل همگی کباب شد سوی شراب ران فرس
\\
گفت شراب اگر خوری از کف هر خسی مخور
&&
باده منت دهم گزین صاف شده ز خاک و خس
\\
گفتم اگر بیابمت من چه کنم شراب را
&&
نیست روا تیممی بر لب نیل و بر ارس
\\
خامش باش ای سقا کاین فرس الحیات تو
&&
آب حیات می‌کشد بازگشا از او جرس
\\
آب حیات از شرف خود نرسد به هر خلف
&&
زین سببست مختفی آب حیات در غلس
\\
\end{longtable}
\end{center}
