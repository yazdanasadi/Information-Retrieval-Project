\begin{center}
\section*{غزل شماره ۱۸۰۴: با آن سبک روحی گل وان لطف شه برگ سمن}
\label{sec:1804}
\addcontentsline{toc}{section}{\nameref{sec:1804}}
\begin{longtable}{l p{0.5cm} r}
با آن سبک روحی گل وان لطف شه برگ سمن
&&
چون او ببیند روی تو هر برگ او گردد سه من
\\
ای گلشن تو زندگی وی زخم تو فرخندگی
&&
وی بنده‌ات را بندگی بهتر ز ملک انجمن
\\
گفتی که جان بخشم تو را نی نی بگو بکشم تو را
&&
تا زنده‌ای باشم تو را چون شمع در گردن زدن
\\
زاهد چه جوید رحم تو عاشق چه جوید زخم تو
&&
آن مرده‌ای اندر قبا وین زنده‌ای اندر کفن
\\
آن در خلاص جان دود وین عشق را قربان شود
&&
آن سر نهد تا جان برد وین خصم جان خویشتن
\\
ای تافته در جان من چون آفتاب اندر حمل
&&
وی من ز تاب روی تو همچون عقیق اندر یمن
\\
\end{longtable}
\end{center}
