\begin{center}
\section*{غزل شماره ۲۹۴۹: در غیب هست عودی کاین عشق از او است دودی}
\label{sec:2949}
\addcontentsline{toc}{section}{\nameref{sec:2949}}
\begin{longtable}{l p{0.5cm} r}
در غیب هست عودی کاین عشق از او است دودی
&&
یک هست نیست رنگی کز او است هر وجودی
\\
هستی ز غیب رسته بر غیب پرده بسته
&&
و آن غیب همچو آتش در پرده‌های دودی
\\
دود ار چه زاد ز آتش هم دود شد حجابش
&&
بگذر ز دود هستی کز دود نیست سودی
\\
از دود گر گذشتی جان عین نور گشتی
&&
جان شمع و تن چو طشتی جان آب و تن چو رودی
\\
گر گرد پست شستی قرص فلک شکستی
&&
در نیست برشکستی بر هست‌ها فزودی
\\
بشکستی از نری او سد سکندری او
&&
ز افرشته و پری او روبندها گشودی
\\
ملکش شدی مهیا از عرش تا ثریا
&&
از زیر هفت دریا در بقا ربودی
\\
رفتی لطیف و خرم زان سو ز خشک و از نم
&&
در عشق گشته محرم با شاهدی به سودی
\\
تبریز شمس دینی گر داردش امینی
&&
با دیده یقینی در غیب وانمودی
\\
\end{longtable}
\end{center}
