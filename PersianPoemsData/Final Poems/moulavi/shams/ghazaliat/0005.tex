\begin{center}
\section*{غزل شماره ۵: آن شکل بین وان شیوه بین وان قد و خد و دست و پا}
\label{sec:0005}
\addcontentsline{toc}{section}{\nameref{sec:0005}}
\begin{longtable}{l p{0.5cm} r}
آن شکل بین وان شیوه بین وان قد و خد و دست و پا
&&
آن رنگ بین وان هنگ بین وان ماه بدر اندر قبا
\\
از سرو گویم یا چمن از لاله گویم یا سمن
&&
از شمع گویم یا لگن یا رقص گل پیش صبا
\\
ای عشق چون آتشکده در نقش و صورت آمده
&&
بر کاروان دل زده یک دم امان ده یا فتی
\\
در آتش و در سوز من شب می‌برم تا روز من
&&
ای فرخ پیروز من از روی آن شمس الضحی
\\
بر گرد ماهش می‌تنم بی‌لب سلامش می‌کنم
&&
خود را زمین برمی‌زنم زان پیش کو گوید صلا
\\
گلزار و باغ عالمی چشم و چراغ عالمی
&&
هم درد و داغ عالمی چون پا نهی اندر جفا
\\
آیم کنم جان را گرو گویی مده زحمت برو
&&
خدمت کنم تا واروم گویی که ای ابله بیا
\\
گشته خیال همنشین با عاشقان آتشین
&&
غایب مبادا صورتت یک دم ز پیش چشم ما
\\
ای دل قرار تو چه شد وان کار و بار تو چه شد
&&
خوابت که می‌بندد چنین اندر صباح و در مسا
\\
دل گفت حسن روی او وان نرگس جادوی او
&&
وان سنبل ابروی او وان لعل شیرین ماجرا
\\
ای عشق پیش هر کسی نام و لقب داری بسی
&&
من دوش نام دیگرت کردم که درد بی‌دوا
\\
ای رونق جانم ز تو چون چرخ گردانم ز تو
&&
گندم فرست ای جان که تا خیره نگردد آسیا
\\
دیگر نخواهم زد نفس این بیت را می‌گوی و بس
&&
بگداخت جانم زین هوس ارفق بنا یا ربنا
\\
\end{longtable}
\end{center}
