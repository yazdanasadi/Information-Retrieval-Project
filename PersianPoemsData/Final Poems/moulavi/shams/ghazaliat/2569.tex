\begin{center}
\section*{غزل شماره ۲۵۶۹: آن زلف مسلسل را گر دام کنی حالی}
\label{sec:2569}
\addcontentsline{toc}{section}{\nameref{sec:2569}}
\begin{longtable}{l p{0.5cm} r}
آن زلف مسلسل را گر دام کنی حالی
&&
در عشق جهانی را بدنام کنی حالی
\\
می‌جوش ز سر گیرد خمخانه به رقص آید
&&
گر از شکرقندت در جام کنی حالی
\\
از چشم چو بادامت در مجلس یک رنگی
&&
هر نقل که پیش آید بادام کنی حالی
\\
حاشا ز عطای تو کان نسیه بود ای جان
&&
گر تشنه بود صادق انعام کنی حالی
\\
ای ماه فلک پیما از منزل ما تا تو
&&
صدساله ره ار باشد یک گام کنی حالی
\\
از لطف تو از عقرب صد شیر بجوشیده
&&
و آن کره گردون را هم رام کنی حالی
\\
بر بام فلک صد در بگشاید و بنماید
&&
گر حارس بامت را بر بام کنی حالی
\\
هر خام شود پخته هم خوانده شود تخته
&&
گر صبح رخت جلوه در شام کنی حالی
\\
\end{longtable}
\end{center}
