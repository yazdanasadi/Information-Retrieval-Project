\begin{center}
\section*{غزل شماره ۲۶۴۸: دریغا کز میان ای یار رفتی}
\label{sec:2648}
\addcontentsline{toc}{section}{\nameref{sec:2648}}
\begin{longtable}{l p{0.5cm} r}
دریغا کز میان ای یار رفتی
&&
به درد و حسرت بسیار رفتی
\\
بسی زنهار گفتی لابه کردی
&&
چه سود از حکم بی‌زنهار رفتی
\\
به هر سو چاره جستی حیله کردی
&&
ندیده چاره و ناچار رفتی
\\
کنار پرگل و روی چو ماهت
&&
چه شد چون در زمین خوار رفتی
\\
ز حلقه دوستان و همنشینان
&&
میان خاک و مور و مار رفتی
\\
چه شد آن نکته‌ها و آن سخن‌ها
&&
چه شد عقلی که در اسرار رفتی
\\
چه شد دستی که دست ما گرفتی
&&
چه شد پایی که در گلزار رفتی
\\
لطیف و خوب و مردم دار بودی
&&
درون خاک مردم خوار رفتی
\\
چه اندیشه که می‌کردی و ناگاه
&&
به راه دور و ناهموار رفتی
\\
فلک بگریست و مه را رو خراشید
&&
در آن ساعت که زار زار رفتی
\\
دلم خون شد چه پرسم من چه دانم
&&
بگو باری عجب بیدار رفتی
\\
چو رفتی صحبت پاکان گزیدی
&&
و یا محروم و باانکار رفتی
\\
جوابک‌های شیرینت کجا شد
&&
خمش کردی و از گفتار رفتی
\\
زهی داغ و زهی حسرت که ناگه
&&
سفر کردی مسافروار رفتی
\\
کجا رفتی که پیدا نیست گردت
&&
زهی پرخون رهی کاین بار رفتی
\\
\end{longtable}
\end{center}
