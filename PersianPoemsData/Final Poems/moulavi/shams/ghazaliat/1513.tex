\begin{center}
\section*{غزل شماره ۱۵۱۳: اگر سرمست اگر مخمور باشم}
\label{sec:1513}
\addcontentsline{toc}{section}{\nameref{sec:1513}}
\begin{longtable}{l p{0.5cm} r}
اگر سرمست اگر مخمور باشم
&&
مهل کز مجلس تو دور باشم
\\
رخم از قبله جان نور گیرد
&&
چو با یاد تو اندر گور باشم
\\
قرارم کی بود خود در تک گور
&&
چو بر دمگاه نفخ صور باشم
\\
صد افسنتین و داروهای نافع
&&
تویی جان را چو من رنجور باشم
\\
شوم شیرین ز لطف گوهر تو
&&
اگر چون بحر تلخ و شور باشم
\\
اگر غم همچو شب عالم بگیرد
&&
برآ ای صبح تا منصور باشم
\\
تویی روز و منم استاره روز
&&
عجب نبود اگر مشهور باشم
\\
به من شادند جمله روزجویان
&&
چو پیش آهنگ چون تو نور باشم
\\
مرا مخمور می داری نه از بخل
&&
ولی تا ساکن و مستور باشم
\\
بدان مستور می داری چو حوتم
&&
که تا از عقربت مهجور باشم
\\
چه غم دارم ز نیش عقرب ای ماه
&&
چو غرق شهد چون زنبور باشم
\\
خمش کردم ولیکن عشق خواهد
&&
که پیش زخمه‌اش طنبور باشم
\\
\end{longtable}
\end{center}
