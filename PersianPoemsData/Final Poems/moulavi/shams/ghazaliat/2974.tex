\begin{center}
\section*{غزل شماره ۲۹۷۴: آن دم که دل کند سوی دلبر اشارتی}
\label{sec:2974}
\addcontentsline{toc}{section}{\nameref{sec:2974}}
\begin{longtable}{l p{0.5cm} r}
آن دم که دل کند سوی دلبر اشارتی
&&
زان سر رسد به بی‌سر و باسر اشارتی
\\
زان رنگ اشارتی که به روز الست بود
&&
کآمد به جان مؤمن و کافر اشارتی
\\
زیرا که قهر و لطف کز آن بحر دررسید
&&
بر سنگ اشارتی است و به گوهر اشارتی
\\
بر سنگ اشارتی است که بر حال خویش باش
&&
بر گوهر است هر دم دیگر اشارتی
\\
بر سنگ کرده نقشی و آن نقش بند او است
&&
هر لحظه سوی نقش ز آزر اشارتی
\\
چون در گهر رسید اشارت گداخت او
&&
احسنت آفرین چه منور اشارتی
\\
بعد از گداز کرد گهر صد هزار جوش
&&
چون می‌رسید از تف آذر اشارتی
\\
جوشید و بحر گشت و جهان در جهان گرفت
&&
چون آمدش ز ایزد اکبر اشارتی
\\
ما را اشارتی است ز تبریز و شمس دین
&&
چون تشنه را ز چشمه کوثر اشارتی
\\
\end{longtable}
\end{center}
