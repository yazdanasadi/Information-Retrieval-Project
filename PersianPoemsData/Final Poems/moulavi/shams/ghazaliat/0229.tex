\begin{center}
\section*{غزل شماره ۲۲۹: شراب داد خدا مر مرا تو را سرکا}
\label{sec:0229}
\addcontentsline{toc}{section}{\nameref{sec:0229}}
\begin{longtable}{l p{0.5cm} r}
شراب داد خدا مر مرا تو را سرکا
&&
چو قسمتست چه جنگست مر مرا و تو را
\\
شراب آن گل است و خمار حصه خار
&&
شناسد او همه را و سزا دهد به سزا
\\
شکر ز بهر دل تو ترش نخواهد شد
&&
که هست جا و مقام شکر دل حلوا
\\
تو را چو نوحه گری داد نوحه‌ای می‌کن
&&
مرا چو مطرب خود کرد دردمم سرنا
\\
شکر شکر چه بخندد به روی من دلدار
&&
به روی او نگرم وارهم ز رو و ریا
\\
اگر بدست ترش شکری تو از من نیز
&&
طمع کن ای ترش ار نه محال را مفزا
\\
وگر گریست به عالم گلی که تا من نیز
&&
بگریم و بکنم نوحه‌ای چو آن گل‌ها
\\
حقم نداد غمی جز که قافیه طلبی
&&
ز بهر شعر و از آن هم خلاص داد مرا
\\
بگیر و پاره کن این شعر را چو شعر کهن
&&
که فارغست معانی ز حرف و باد و هوا
\\
\end{longtable}
\end{center}
