\begin{center}
\section*{غزل شماره ۱۴۹۷: به جان جمله مستان که مستم}
\label{sec:1497}
\addcontentsline{toc}{section}{\nameref{sec:1497}}
\begin{longtable}{l p{0.5cm} r}
به جان جمله مستان که مستم
&&
بگیر ای دلبر عیار دستم
\\
به جان جمله جانبازان که جانم
&&
به جان رستگارانش که رستم
\\
عطاردوار دفترباره بودم
&&
زبردست ادیبان می نشستم
\\
چو دیدم لوح پیشانی ساقی
&&
شدم مست و قلم‌ها را شکستم
\\
جمال یار شد قبله نمازم
&&
ز اشک رشک او شد آبدستم
\\
ز حسن یوسفی سرمست بودم
&&
که حسنش هر دمی گوید الستم
\\
در آن مستی ترنجی می بریدم
&&
ترنج اینک درست و دست خستم
\\
مبادم سر اگر جز تو سرم هست
&&
بسوزا هستیم گر بی‌تو هستم
\\
تویی معبود در کعبه و کنشتم
&&
تویی مقصود از بالا و پستم
\\
شکار من بود ماهی و یونس
&&
چو حاصل شد ز جعدت شصت شستم
\\
چو دیدم خوان تو بس چشم سیرم
&&
چو خوردم ز آب تو زین جوی جستم
\\
برای طبع لنگان لنگ رفتم
&&
ز بیم چشم بد سر نیز بستم
\\
همان ارزد کسی کش می پرستد
&&
زهی من که مر او را می پرستم
\\
ببرد از کسی کآخر ببرد
&&
به سوی عدل بگریزید ز استم
\\
چو ری با سین و تی و میم پیوست
&&
بدین پیوند رو بنمود رستم
\\
یقین شد که جماعت رحمت آمد
&&
جماعت را به جان من چاکرستم
\\
خمش کردم شکار شیر باشم
&&
که تا گوید شکار مفترستم
\\
\end{longtable}
\end{center}
