\begin{center}
\section*{غزل شماره ۶۸۹: دوش از بت من جهان چه می‌شد}
\label{sec:0689}
\addcontentsline{toc}{section}{\nameref{sec:0689}}
\begin{longtable}{l p{0.5cm} r}
دوش از بت من جهان چه می‌شد
&&
وز ماه من آسمان چه می‌شد
\\
در پیش رخش چه رقص می‌کرد
&&
وز آتش عشق جان چه می‌شد
\\
چشم از نظرش چه مست می‌گشت
&&
وز قند لبش دهان چه می‌شد
\\
از تیر مژه چه صید می‌کرد
&&
وان ابروی چون کمان چه می‌شد
\\
می‌شد که به لاله رنگ بخشد
&&
ور نی سوی گلستان چه می‌شد
\\
آن لحظه به سبزه گل چه می‌گفت
&&
وز نرگسش ارغوان چه می‌شد
\\
جز از پی نور بخش کردن
&&
بر چرخ دوان دوان چه می‌شد
\\
گر زانک نه لطف بی‌کران داشت
&&
آن ماه در این میان چه می‌شد
\\
بنمود ز لامکان جمالی
&&
یا رب که از او مکان چه می‌شد
\\
بگشاد نقاب بی‌نشانی
&&
وین عالم بانشان چه می‌شد
\\
شب رفت و بماند روز مطلق
&&
وین عقل چو پاسبان چه می‌شد
\\
از دیده غیب شمس تبریز
&&
این دیده غیب دان چه می‌شد
\\
\end{longtable}
\end{center}
