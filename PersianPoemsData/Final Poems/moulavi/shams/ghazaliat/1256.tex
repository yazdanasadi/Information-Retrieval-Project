\begin{center}
\section*{غزل شماره ۱۲۵۶: آنک جانش داده‌ای آن را مکش}
\label{sec:1256}
\addcontentsline{toc}{section}{\nameref{sec:1256}}
\begin{longtable}{l p{0.5cm} r}
آنک جانش داده‌ای آن را مکش
&&
ور ندادی نقش بی‌جان را مکش
\\
آن دو زلف کافر خود را بگو
&&
کای یگانه اهل ایمان را مکش
\\
آفتابا روی خود جلوه مکن
&&
چند روزی ماه تابان را مکش
\\
چون تو سیمرغی به قاف ذوالجلال
&&
بازگرد و جمله مرغان را مکش
\\
در میان خون هر مسکین مرو
&&
جز قباد و شاه خاقان را مکش
\\
گر مرا دربان عشقت بار داد
&&
از سر غیرت تو دربان را مکش
\\
گر فضولم من که مهمان توام
&&
شرط نبود هیچ مهمان را مکش
\\
مست میدانم ز می‌دانم خراب
&&
شیشه مشکن مست میدان را مکش
\\
شمس تبریزی تویی سلطان من
&&
بازگشتم باز سلطان را مکش
\\
\end{longtable}
\end{center}
