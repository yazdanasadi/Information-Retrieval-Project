\begin{center}
\section*{غزل شماره ۳۲۰۰: هذا طبیبی، عند الدوآء}
\label{sec:3200}
\addcontentsline{toc}{section}{\nameref{sec:3200}}
\begin{longtable}{l p{0.5cm} r}
هذا طبیبی، عند الدوآء
&&
هذا حبیبی، عند الوء
\\
هذا لباسی، هذا کناسی
&&
هذا شرابی، هذا غذایی
\\
هذا انیسی، عندالفراق
&&
هذا خلاصی، عند البء
\\
قالوا تسلی، حاشا و کلا
&&
قلبی مقیم، وسط الوفء
\\
این کان احمد، قلبی تعمد
&&
روحی فداه، عند الفنء
\\
ان کان شاکی، یبغی هلاکی
&&
سمعا و طاعه ذا مشتهایی
\\
هذا سلحدار، لایدخل الدار
&&
الا بدینار، عند الابء
\\
مونی حیاتی، حصدی نباتی
&&
حبسی نجاتی، مقتی بقایی
\\
یا من یلمنی، مالک و مالی
&&
صبری محال فی الاتقء
\\
روحی مصیب، قلبی مصاف
&&
صبری مذاب، فی حرنایی
\\
انا نسینا، ما قد لقینا
&&
لما راینا، بدر الضیء
\\
یا ذوفنونی، ابصر جنونی
&&
فوق‌الظنون، خرق الحیاء
\\
امروز دلبر یکبار دیگر
&&
آمد که گیرد مرغ هوایی
\\
گر او پذیرد، ده ده بگیرد
&&
لیکن بخیلست، در رخ نمایی
\\
بر گرد دلبر، پانصد کبوتر
&&
پر می‌فشانند، بهر گوایی
\\
ای نیم مرده، پران شو اینجا
&&
کاینجا نماند، بی‌اشتهایی
\\
مستان کم زن، رستند از تن
&&
دزدم گلیمی، من از کسایی
\\
\end{longtable}
\end{center}
