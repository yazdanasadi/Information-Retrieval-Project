\begin{center}
\section*{غزل شماره ۳۱۹۶: بار منست او بچه نغزی، خواجه اگرچه همه مغزی}
\label{sec:3196}
\addcontentsline{toc}{section}{\nameref{sec:3196}}
\begin{longtable}{l p{0.5cm} r}
بار منست او بچه نغزی، خواجه اگرچه همه مغزی
&&
چون گذری بر سر کویش، پای نکونه که نلغزی
\\
حدثنی صاحب قلبی، طهرلی جلدة کلبی
&&
اضحکنی نور فادی، اسکرنی شربة ربی
\\
وز در بسته چو برنجی، شیوه کنی زود بقنجی؟!
&&
شیوه مکن، قنج رها کن، پست کن آن سر، که بگنجی
\\
طاب لحبی حرکاتی، صار خساری برکاتی
&&
انت حیاتی و تعدی، طال حیاتی بحیاتی
\\
جان دل تو، دل جانی، قبلهٔ نظاره کنانی
&&
چونک شود خیره نظرشان، از ره دلشان بکشانی
\\
عمرک یا عمر و تولی، زادک یا زید تجلی
&&
کم تنم‌اللیل؟! تنبه! قد ظهرالصبح، تجلی
\\
خانهٔ دل را دو دری کن، جانب جان راه‌بری کن
&&
طالب دریای حیاتی، سنگ دلا، رو گهری کن
\\
یا سندی انت جمالی ، انت دلیلی ودلالی
&&
کیف تجوز و ترجی، تعرض عنی لملالی
\\
جان و روان خیز روان کن، با شه شاهان سیران کن
&&
هیچ بطی جوید کشتی؟! جان شدهٔ ترک مکان کن
\\
قد طلع‌البدر علینا، قد وصل‌الوصل الینا
&&
یا فئتی وافق بدر فیه نذرنا والینا
\\
ای طربستان، چه لطیفی؟! ای سرمستان چه ظریفی؟!
&&
ده بخوری تو بدهی یک، کی بود این شرط حریفی؟!
\\
کل مساء و صباح یسکرناالعشق براح
&&
قد یس‌المحزن منا، التحق الحزن بصاح
\\
بس کن گفتار رها کن، باز شهی قصد هوا کن
&&
باز رو ای باز بدان شه، با شه خود عهد و وفا کن
\\
بسکم‌الهجر فعودوا، فی طلب‌الوصل سعود
&&
امتنع‌الوصل بشح، اجتنبواالشح، وجودوا
\\
\end{longtable}
\end{center}
