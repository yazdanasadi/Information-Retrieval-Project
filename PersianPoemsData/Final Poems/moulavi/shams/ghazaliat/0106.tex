\begin{center}
\section*{غزل شماره ۱۰۶: مرا حلوا هوس کردست حلوا}
\label{sec:0106}
\addcontentsline{toc}{section}{\nameref{sec:0106}}
\begin{longtable}{l p{0.5cm} r}
مرا حلوا هوس کردست حلوا
&&
میفکن وعده حلوا به فردا
\\
دل و جانم بدان حلواست پیوست
&&
که صوفی را صفا آرد نه صفرا
\\
زهی حلوای گرم و چرب و شیرین
&&
که هر دم می‌رسد بویش ز بالا
\\
دهانی بسته حلوا خور چو انجیر
&&
ز دل خور هیچ دست و لب میالا
\\
از آن دستست این حلوا از آن دست
&&
بخور زان دست ای بی‌دست و بی‌پا
\\
دمی با مصطفا و کاسه باشیم
&&
که او می خورد از آن جا شیر و خرما
\\
از آن خرما که مریم را ندا کرد
&&
کلی و اشربی و قری عینا
\\
دلیل آنک زاده عقل کلیم
&&
ندایش می‌رسد کای جان بابا
\\
همی‌خواند که فرزندان بیایید
&&
که خوان آراسته‌ست و یار تنها
\\
\end{longtable}
\end{center}
