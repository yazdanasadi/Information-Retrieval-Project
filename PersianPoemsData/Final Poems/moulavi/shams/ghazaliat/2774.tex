\begin{center}
\section*{غزل شماره ۲۷۷۴: در فنای محض افشانند مردان آستی}
\label{sec:2774}
\addcontentsline{toc}{section}{\nameref{sec:2774}}
\begin{longtable}{l p{0.5cm} r}
در فنای محض افشانند مردان آستی
&&
دامن خود برفشاند از دروغ و راستی
\\
مرد مطلق دست خود را کی بیالاید به جان
&&
آخر ای جان قلندر از چه پهلو خاستی
\\
سالکی جان مجرد بر قلندر عرضه داد
&&
گفت در گوشش قلندر کان طرف می واستی
\\
کاین طرف هر چند سوزی در شرار عشق خویش
&&
لیک هم مطلق نه‌ای زیرا که در غوغاستی
\\
در جمال لم یزل چشم ازل حیران شده
&&
نی فزودی از دو عالم نی ز نفیش کاستی
\\
تو نه این جایی نه آن جا لیک عشاق از هوس
&&
می‌کنند آن جا نظر کان جاستی آن جاستی
\\
ای که از الا تو لافیدی بدین زفتی مباش
&&
چشم‌ها را پاک کن بنگر که هم در لاستی
\\
مرحبا جان عدم رنگ وجودآمیز خوش
&&
فارغ از هست و عدم مر هر دو را آراستی
\\
پاکی چشمت نباشد جز شه تبریزیان
&&
شمس دین گر او بخواهد لیک نی زان‌هاستی
\\
\end{longtable}
\end{center}
