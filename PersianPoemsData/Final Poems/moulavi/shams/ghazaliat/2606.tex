\begin{center}
\section*{غزل شماره ۲۶۰۶: مانده شدم از گفتن تا تو بر ما مانی}
\label{sec:2606}
\addcontentsline{toc}{section}{\nameref{sec:2606}}
\begin{longtable}{l p{0.5cm} r}
مانده شدم از گفتن تا تو بر ما مانی
&&
خویش من و پیوندی نی همره و مهمانی
\\
شیری است که می‌جوشد خونی است نمی‌خسبد
&&
خربنده چرا گشتی شه زاده ارکانی
\\
زر دارد و زر بدهد زین واخردت این دم
&&
آن کس که رهانید از بسیار پریشانی
\\
اشتر ز سوی بیشه بی‌جهد نمی‌آید
&&
کی آمده‌ای ای جان زان خاک به آسانی
\\
صد جا بترنجیدی گفتی نروم زین جا
&&
گوش تو کشان کردم تا جوهر انسانی
\\
در چرخ درآوردم نه گنبد نیلی را
&&
استیزه چه می‌بافی ای شیخ لت انبانی
\\
چون دیگ سیه پوشی اندر پی تتماجی
&&
کو نخوت کرمنا کو همت سلطانی
\\
تو مرد لب قدری نی مرد شب قدری
&&
تو طفل سر خوانی نی پیر پری خوانی
\\
سخت است بلی پندت اما نگذارندت
&&
سیلی زندت آرد استاد دبستانی
\\
هر لحظه کمندی نو در گردنت اندازد
&&
روزی که به جد گیرد گردن ز کی پیچانی
\\
بنگر تو در این اجزا که همرهشان بودی
&&
در خود بترنجیده از نامی و ارکانی
\\
زان جا بکشانمشان مانند تو تا این جا
&&
و اندر پس این منزل صد منزل روحانی
\\
چون بز همه را گویم هین برجه و خدمت کن
&&
ریشت پی آن دادم تا ریش بجنبانی
\\
گر ریش نجنبانی یک یک بکنم ریشت
&&
ریش کی رهید از من تا تو دبه برهانی
\\
یک لحظه شدی شانه در ریش درافتادی
&&
یک لحظه شو آیینه چون حلقه گردانی
\\
هم شانه و هم مویی هم آینه هم رویی
&&
هم شیر و هم آهویی هم اینی و هم آنی
\\
هم فرقی و هم زلفی مفتاحی و هم قلفی
&&
بی‌رنج چه می‌سلفی آواز چه لرزانی
\\
خاموش کن از گفتن هین بازی دیگر کن
&&
صد بازی نو داری ای نر بز لحیانی
\\
\end{longtable}
\end{center}
