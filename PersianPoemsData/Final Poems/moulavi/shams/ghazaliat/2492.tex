\begin{center}
\section*{غزل شماره ۲۴۹۲: هر بشری که صاف شد در دو جهان ورا دلی}
\label{sec:2492}
\addcontentsline{toc}{section}{\nameref{sec:2492}}
\begin{longtable}{l p{0.5cm} r}
هر بشری که صاف شد در دو جهان ورا دلی
&&
دید غرض که فقر بد بانگ الست را بلی
\\
عالم خاک همچو تل فقر چو گنج زیر او
&&
شادی کودکان بود بازی و لاغ بر تلی
\\
چشم هر آنک بسته شد تابش حرص خسته شد
&&
و آنک ز گنج رسته شد گشت گران و کاهلی
\\
گنج جمال همچو مه جانش بدیده گفته خه
&&
بر ره او هزار شه آه شگرف حاصلی
\\
وصف لبش بگفتمی چهره جان شکفتمی
&&
راه بیان برفتمی لیک کجاست واصلی
\\
جان بجهان و هم بجه سر بمکش سرک بنه
&&
گر چه درون هر دو ده نیست درون قابلی
\\
ای تبریز مشتهر بند به شمس دین کمر
&&
ز آنک مبارک است سر بر کف پای کاملی
\\
\end{longtable}
\end{center}
