\begin{center}
\section*{غزل شماره ۳۷۲: من سر نخورم که سر گران‌ست}
\label{sec:0372}
\addcontentsline{toc}{section}{\nameref{sec:0372}}
\begin{longtable}{l p{0.5cm} r}
من سر نخورم که سر گران‌ست
&&
پاچه نخورم که استخوان‌ست
\\
بریان نخورم که هم زیان‌ست
&&
من نور خورم که قوت جان‌ست
\\
من سر نخوهم که باکلاهند
&&
من زر نخوهم که بازخواهند
\\
من خر نخوهم که بند کاهند
&&
من کبک خورم که صید شاهند
\\
بالا نپرم نه لک لکم من
&&
کس را نگزم که نی سگم من
\\
لنگی نکنم نه بدتکم من
&&
که عاشق روی ایبکم من
\\
ترشی نکنم نه سرکه‌ام من
&&
پرنم نشوم نه برکه‌ام من
\\
سرکش نشوم نه عکه‌ام من
&&
قانع بزیم که مکه‌ام من
\\
دستار مرا گرو نهادی
&&
یک کوزه مثلثم ندادی
\\
انصاف بده عوان نژادی
&&
ما را کم نیست هیچ شادی
\\
سالار دهی و خواجه ده
&&
آن باده که گفته‌ای به من ده
\\
ور دفع دهی تو و برون جه
&&
در کس زنان خویشتن نه
\\
من عشق خورم که خوشگوارست
&&
ذوق دهنست و نشو جان‌ست
\\
خوردم ز ثرید و پاچه یک چند
&&
از پاچه سر مرا زیانست
\\
زین پس سر پاچه نیست ما را
&&
ما را و کسی که اهل خوانست
\\
\end{longtable}
\end{center}
