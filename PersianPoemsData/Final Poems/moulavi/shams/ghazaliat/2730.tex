\begin{center}
\section*{غزل شماره ۲۷۳۰: آورد خبر شکرستایی}
\label{sec:2730}
\addcontentsline{toc}{section}{\nameref{sec:2730}}
\begin{longtable}{l p{0.5cm} r}
آورد خبر شکرستانی
&&
کز مصر رسید کاروانی
\\
صد اشتر جمله شکر و قند
&&
یا رب چه لطیف ارمغانی
\\
در نیم شبی رسید شمعی
&&
در قالب مرده رفت جانی
\\
گفتم که بگو سخن گشاده
&&
گفتا که رسید آن فلانی
\\
دل از سبکی ز جای برجست
&&
بنهاد ز عقل نردبانی
\\
بر بام دوید از سر عشق
&&
می‌جست از این خبر نشانی
\\
ناگاه بدید از سر بام
&&
بیرون ز جهان ما جهانی
\\
دریای محیط در سبویی
&&
در صورت خاک آسمانی
\\
بر بام نشسته پادشاهی
&&
پوشیده لباس پاسبانی
\\
باغی و بهشت بی‌نهایت
&&
در سینه مرد باغبانی
\\
می‌گشت به سینه‌ها خیالش
&&
می‌کرد ز شاه دل بیانی
\\
مگریز ز چشمم ای خیالش
&&
تا تازه شود دلم زمانی
\\
شمس تبریز لامکان دید
&&
برساخت ز لامکان مکانی
\\
\end{longtable}
\end{center}
