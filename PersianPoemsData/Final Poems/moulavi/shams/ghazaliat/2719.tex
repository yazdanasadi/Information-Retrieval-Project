\begin{center}
\section*{غزل شماره ۲۷۱۹: مگر تو یوسفان را دلستانی}
\label{sec:2719}
\addcontentsline{toc}{section}{\nameref{sec:2719}}
\begin{longtable}{l p{0.5cm} r}
مگر تو یوسفان را دلستانی
&&
مگر تو رشک ماه آسمانی
\\
مها از بس عزیزی و لطیفی
&&
غریب این جهان و آن جهانی
\\
روان‌هایی که روز تو شنیدند
&&
به طمع تو گرفته شب گرانی
\\
ز شب رفتن ز چالاکی چه آید
&&
چو ذوالعرشت کند می پاسبانی
\\
منم آن کز دم عیسی بمردم
&&
مرا کشته‌ست آب زندگانی
\\
چنین مرگی که مردم زنده گردم
&&
گرت بینم ایا فخر الزمانی
\\
دلم از هجر تو خون گشت لیکن
&&
از آن خون رست صورت‌های جانی
\\
ز درد تو رواق صاف جوشید
&&
ز درد خم‌های خسروانی
\\
خداوندی است شمس الدین تبریز
&&
که او را نیست در آفاق ثانی
\\
برید آفرینش در دو عالم
&&
نیاورده‌ست چون او ارمغانی
\\
هزاران جان نثار جان او باد
&&
که تا گردند جان‌ها جاودانی
\\
دریغا لفظ‌ها بودی نوآیین
&&
کز این الفاظ ناقص شد معانی
\\
\end{longtable}
\end{center}
