\begin{center}
\section*{غزل شماره ۲۶۷۱: بر من نیستی یارا کجایی}
\label{sec:2671}
\addcontentsline{toc}{section}{\nameref{sec:2671}}
\begin{longtable}{l p{0.5cm} r}
بر من نیستی یارا کجایی
&&
به هر جایی که هستی جان فزایی
\\
ز خشم من به هر ناکس بسازی
&&
به رغم من به هر آتش درآیی
\\
چو بینی مر مرا نادیده آری
&&
چنین باشد وفا و آشنایی
\\
عزیزی بودم خوارم ز عشقت
&&
در این خواری نگر کبر خدایی
\\
برای تو جدا گردم ز عالم
&&
که تا ناید مرا بوی جدایی
\\
سبک روحا گران کردی تو رو را
&&
که یعنی قصد دارم بی‌وفایی
\\
تو در دل جورها داری همی‌کن
&&
که تا روز قیامت جان مایی
\\
الا ای چرخ زاینده چنین ماه
&&
نزایی و نزایی و نزایی
\\
به کوه قاف شمس الدین تبریز
&&
همایی و همایی و همایی
\\
\end{longtable}
\end{center}
