\begin{center}
\section*{غزل شماره ۲۰۲۷: ای امتان باطل بر نان زنید بر نان}
\label{sec:2027}
\addcontentsline{toc}{section}{\nameref{sec:2027}}
\begin{longtable}{l p{0.5cm} r}
ای امتان باطل بر نان زنید بر نان
&&
وی امتان مقبل بر جان زنید بر جان
\\
حیوان علف کشاند غیر علف نداند
&&
آن آدمی بود کو جوید عقیق و مرجان
\\
آن باغ‌ها بخفته وین باغ‌ها شکفته
&&
وین قسمتی است رفته در بارگاه سلطان
\\
جان‌هاست نارسیده در دام‌ها خزیده
&&
جان‌هاست برپریده ره برده تا به جانان
\\
جانی ز شرح افزون بالای چرخ گردون
&&
چست و لطیف و موزون چون مه به برج میزان
\\
جانی دگر چو آتش تند و حرون و سرکش
&&
کوتاه عمر و ناخوش همچون خیال شیطان
\\
ای خواجه تو کدامی یا پخته یا که خامی
&&
سرمست نقل و جامی یا شهسوار میدان
\\
روزی به سوی صحرا دیدم یکی معلا
&&
اندر هوا به بالا می‌کرد رقص و جولان
\\
هر سو از او خروشی او ساکن و خموشی
&&
سرسبز و سبزپوشی جانم بماند حیران
\\
گفتم که در چه شوری کز وهم خلق دوری
&&
تو نور نور نوری یا آفتاب تابان
\\
گفتا دلم تنگ شد تن نیز هم سبک شد
&&
تا پاگشاده گشتم از چارمیخ ارکان
\\
گفتم که ای امیرم شادت کنار گیرم
&&
بسیار لابه کردم گفتا که نیست امکان
\\
گفتم بیا وفا کن وین ناز را رها کن
&&
شاخی شکر سخا کن چه کم شود از آن کان
\\
گفتا که من فنایم اندر کنار نایم
&&
نقشی همی‌نمایم از بهر درد و درمان
\\
گفتم تو را نباید خود دفع کم نیاید
&&
پنجه بهانه زاید از طبعت ای سخندان
\\
گفتا ز سر یک تو باور کجا کنی تو
&&
طفلی و درست ابجد برگیر لوح و می‌خوان
\\
گفتم همین سیاست می‌کن حلال بادت
&&
صد گونه دفع می‌ده می‌کش مرا به هجران
\\
زود از زبان دیگر صد پاسخ چو شکر
&&
برخواند بر من از بر گشتم خراب و سکران
\\
بسیار اشک راندم تا دیر مست ماندم
&&
تا که برون شد آن شه چون جان ز نقش انسان
\\
داغی بماند حاصل زان صحبت اندر این دل
&&
داغی که از لذیذی ارزد هزار احسان
\\
فرمود مشکلاتی در وی عجب عظاتی
&&
خامش در زبان‌ها آن می نیاید آسان
\\
\end{longtable}
\end{center}
