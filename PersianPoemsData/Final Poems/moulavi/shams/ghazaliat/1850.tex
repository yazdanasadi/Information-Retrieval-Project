\begin{center}
\section*{غزل شماره ۱۸۵۰: چراغ عالم افروزم نمی‌تابد چنین روشن}
\label{sec:1850}
\addcontentsline{toc}{section}{\nameref{sec:1850}}
\begin{longtable}{l p{0.5cm} r}
چراغ عالم افروزم نمی‌تابد چنین روشن
&&
عجب این عیب از چشم است یا از نو یا روزن
\\
مگر گم شد سر رشته چه شد آن حال بگذشته
&&
که پوشیده نمی‌ماند در آن حالت سر سوزن
\\
خنک آن دم که فراش فرشنا اندر این مسجد
&&
در این قندیل دل ریزد ز زیتون خدا روغن
\\
دلا در بوته آتش درآ مردانه بنشین خوش
&&
که از تأثیر این آتش چنان آیینه شد آهن
\\
چو ابراهیم در آذر درآمد همچو نقد زر
&&
برویید از رخ آتش سمن زار و گل و سوسن
\\
اگر دل را از این غوغا نیاری اندر این سودا
&&
چه خواهی کرد این دل را بیا بنشین بگو با من
\\
اگر در حلقه مردان نمی‌آیی ز نامردی
&&
چو حلقه بر در مردان برون می باش و در می زن
\\
چو پیغامبر بگفت الصوم جنه پس بگیر آن را
&&
به پیش نفس تیرانداز زنهار این سپر مفکن
\\
سپر باید در این خشکی چو در دریا رسی آنگه
&&
چو ماهی بر تنت روید به دفع تیر او جوشن
\\
\end{longtable}
\end{center}
