\begin{center}
\section*{غزل شماره ۷۴۰: مشک و عنبر گر ز مشک زلف یارم بو کند}
\label{sec:0740}
\addcontentsline{toc}{section}{\nameref{sec:0740}}
\begin{longtable}{l p{0.5cm} r}
مشک و عنبر گر ز مشک زلف یارم بو کند
&&
بوی خود را واهلد در حال و زلفش بو کند
\\
کافر و مؤمن گر از خوی خوشش واقف شوند
&&
خوی را خود واکند در حین و خو با او کند
\\
آفتابی ناگهان از روی او تابان شود
&&
پردها را بردرد وین کار را یک سو کند
\\
چنگ تن‌ها را به دست روح‌ها زان داد حق
&&
تا بیان سر حق لایزالی او کند
\\
تارهای خشم و عشق و حقد و حاجت می‌زند
&&
تا ز هر یک بانگ دیگر در حوادث رو کند
\\
شاد با چنگ تنی کز دست جان حق بستدش
&&
بر کنار خود نهاد و ساز آن را هو کند
\\
اوستاد چنگ‌ها آن چنگ باشد در جهان
&&
وای آن چنگی که با آن چنگ حق پهلو کند
\\
باز هم در چنگ حق تاریست بس پنهان و خوش
&&
کو به ناگه وصف آن دو نرگس جادو کند
\\
نرگسان مست شمس الدین تبریزی که هست
&&
چشم آهو تا شکار شیر آن آهو کند
\\
\end{longtable}
\end{center}
