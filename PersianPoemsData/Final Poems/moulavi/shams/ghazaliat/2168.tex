\begin{center}
\section*{غزل شماره ۲۱۶۸: نمی‌گفتی مرا روزی که ما را یار غاری تو}
\label{sec:2168}
\addcontentsline{toc}{section}{\nameref{sec:2168}}
\begin{longtable}{l p{0.5cm} r}
نمی‌گفتی مرا روزی که ما را یار غاری تو
&&
درون باغ عشق ما درخت پایداری تو
\\
ایا شیر خدا آخر بفرمودی به صید اندر
&&
که خه مر آهوی ما را چو آهو خوش شکاری تو
\\
شکفته داشتی چون گل دل و جانم دلاراما
&&
کنونم خود نمی‌گویی کز آن گلزار خاری تو
\\
ز نازی کز تو در سر بد تهی کرد از دماغم غم
&&
مرا زنهار از هجرت که بس بی‌زینهاری تو
\\
چو فتوی داد عشق تو به خون من نمی‌دانم
&&
چه جوهردار تیغی تو چه سنگین دل نگاری تو
\\
ایا اومید در دستم عصای موسوی بودی
&&
ز هجران چو فرعونش کنون جان در چو ماری تو
\\
چو از افلاک نورانی وصال شاه افتادی
&&
چو آدم اندر این پستی در این اقلیم ناری تو
\\
کنار وصل دربودی یکی چندی تو ای دیده
&&
کنار از اشک پر کن تو چو از شه برکناری تو
\\
الا ای مو سیه پوشی به هنگام طرب وآنگه
&&
سپیدت جامه باشد چون در این غم سوگواری تو
\\
به نظم و نثر عذر من سمر شد در جهان اکنون
&&
که یک عذرم نپذرفتی چگونه خوش عذاری تو
\\
تو ای جان سنگ خارایی که از آب حیات او
&&
جدا گشتی و محرومی وآنگه برقراری تو
\\
رمیدستی از این قالب ولیکن علقه‌ای داری
&&
کز آن بحر کرم در گوش در شاهواری تو
\\
در این اومید پژمرده بپژمردی چو باغ از دی
&&
ز دی بگذر سبک برپر که نی جان بهاری تو
\\
بخارای جهان جان که معدنگاه علم آن است
&&
سفر کن جان باعزت که نی جان بخاری تو
\\
مزن فال بدی زیرا به فال سعد وصل آید
&&
مگو دورم ز شاه خود که نیک اندر جواری تو
\\
چو دانستی که دیوانه شدی عقل است این دانش
&&
چو می‌دانی که تو مستی پس اکنون هشیاری تو
\\
هزاران شکر آن شه را که فرزین بند او گشتی
&&
هزاران منت آن می را که از وی در خماری تو
\\
همه فخر و همه دولت برای شاه می‌زیبد
&&
چرا در قید فخری تو چرا دربند عاری تو
\\
فراق من شده فربه ز خون تو که خورد ای دل
&&
چرا قربان شدی ای دل چو شیشاک نزاری تو
\\
چو سرنایی تو نه چشم از برای انتظار لب
&&
چو آن لب را نمی‌بینی در آن پرده چه زاری تو
\\
چو دف از ضربت هجرت چو چنبر گشت پشت من
&&
چرا بر دست این دل هم مثال دف نداری تو
\\
هزاران منتت بر جان ز عشق شاه شمس الدین
&&
تو بادی ریش درکرده که یعنی حق گزاری تو
\\
الا ای شاه تبریزم در این دریای خون ریزم
&&
چه باشد گر چو موسی گرد از دریا برآری تو
\\
ایا خوبی و لطف شه شمردم رمزکی از تو
&&
شمردن از کجا تانم که بی‌حد و شماری تو
\\
\end{longtable}
\end{center}
