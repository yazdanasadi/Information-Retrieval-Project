\begin{center}
\section*{غزل شماره ۴۴۳: از دل به دل برادر گویند روزنیست}
\label{sec:0443}
\addcontentsline{toc}{section}{\nameref{sec:0443}}
\begin{longtable}{l p{0.5cm} r}
از دل به دل برادر گویند روزنیست
&&
روزن مگیر گیر که سوراخ سوزنیست
\\
هر کس که غافل آمد از این روزن ضمیر
&&
گر فاضل زمانه بود گول و کودنیست
\\
زان روزنه نظر کن در خانه جلیس
&&
بنگر که ظلمت است در او یا که روشنیست
\\
گر روشن است و بر تو زند برق روشنش
&&
می‌دان که کان لعل و عقیق است و معدنیست
\\
پهلوی او نشین که امیر است و پهلوان
&&
گل در رهش بکار که سروی و سوسنی است
\\
در گردنش درآر دو دست و کنار گیر
&&
برخور از آن کنار که مرفوع گردنیست
\\
رو رخت سوی او کش و پهلوش خانه گیر
&&
کان جا فرشتگان را آرام و مسکنیست
\\
خواهم که شرح گویم می‌لرزد این دلم
&&
زیرا غریب و نادر و بی‌ما و بی‌منیست
\\
آن جا که او نباشد این جان و این بدن
&&
از همدگر رمیده چو آبی و روغنیست
\\
خواهی بلرز و خواه ملرز اینت گفتنیست
&&
گر بر لب و دهانم خود بند آهنیست
\\
آهن شکافتن بر داوود عشق چیست
&&
خامش که شاه عشق عجایب تهمتنیست
\\
\end{longtable}
\end{center}
