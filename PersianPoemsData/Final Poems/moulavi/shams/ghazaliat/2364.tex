\begin{center}
\section*{غزل شماره ۲۳۶۴: ای ز هجرانت زمین و آسمان بگریسته}
\label{sec:2364}
\addcontentsline{toc}{section}{\nameref{sec:2364}}
\begin{longtable}{l p{0.5cm} r}
ای ز هجرانت زمین و آسمان بگریسته
&&
دل میان خون نشسته عقل و جان بگریسته
\\
چون به عالم نیست یک کس مر مکانت را عوض
&&
در عزای تو مکان و لامکان بگریسته
\\
جبرئیل و قدسیان را بال و پر ازرق شده
&&
انبیا و اولیا را دیدگان بگریسته
\\
اندر این ماتم دریغا تاب گفتارم نماند
&&
تا مثالی وانمایم کان چنان بگریسته
\\
چون از این خانه برفتی سقف دولت درشکست
&&
لاجرم دولت بر اهل امتحان بگریسته
\\
در حقیقت صد جهان بودی نبودی یک کسی
&&
دوش دیدم آن جهان بر این جهان بگریسته
\\
چو ز دیده دور گشتی رفت دیده در پیت
&&
جان پی دیده بمانده خون چکان بگریسته
\\
غیرت تو گر نبودی اشک‌ها باریدمی
&&
همچنین به خون چکان دل در نهان بگریسته
\\
مشک‌ها باید چه جای اشک‌ها در هجر تو
&&
هر نفس خونابه گشته هر زمان بگریسته
\\
ای دریغا ای دریغا ای دریغا ای دریغ
&&
بر چنان چشم عیان چشم گمان بگریسته
\\
شه صلاح الدین برفتی ای همای گرم رو
&&
از کمان جستی چو تیر و آن کمان بگریسته
\\
بر صلاح الدین چه داند هر کسی بگریستن
&&
هم کسی باید که داند بر کسان بگریسته
\\
\end{longtable}
\end{center}
