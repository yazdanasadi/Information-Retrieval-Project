\begin{center}
\section*{غزل شماره ۱۴۷: آخر از هجران به وصلش دررسیدستی دلا}
\label{sec:0147}
\addcontentsline{toc}{section}{\nameref{sec:0147}}
\begin{longtable}{l p{0.5cm} r}
آخر از هجران به وصلش دررسیدستی دلا
&&
صد هزاران سر سر جان شنیدستی دلا
\\
از ورای پرده‌ها تو گشته‌ای چون می از او
&&
پرده خوبان مه رو را دریدستی دلا
\\
از قوام قامتش در قامت تو کژ بماند
&&
همچو چنگ از بهر سرو تر خمیدستی دلا
\\
ز آن سوی هست و عدم چون خاص خاص خسروی
&&
همچو ادبیران چه در هستی خزیدستی دلا
\\
باز جانی شسته‌ای بر ساعد خسرو به ناز
&&
پای بندت با ویست ار چه پریدستی دلا
\\
ور نباشد پای بندت تا نپنداری که تو
&&
از چنان آرام جان‌ها دررمیدستی دلا
\\
بلک چون ماهی به دریا بلک چون قالب به جان
&&
در هوای عشق آن شه آرمیدستی دلا
\\
چون تو را او شاه از شاهان عالم برگزید
&&
تو ز قرآن گزینش برگزیدستی دلا
\\
چون لب اقبال دولت تو گزیدی باک نیست
&&
گر ز زخم خشم دست خود گزیدستی دلا
\\
پای خود بر چرخ تا ننهی تو از عزت از آنک
&&
در رکاب صدر شمس الدین دویدستی دلا
\\
تو ز جام خاص شاهان تا نیاشامی مدام
&&
کز مدام شمس تبریزی چشیدستی دلا
\\
\end{longtable}
\end{center}
