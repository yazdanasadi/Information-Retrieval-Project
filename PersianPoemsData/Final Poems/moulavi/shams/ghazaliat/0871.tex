\begin{center}
\section*{غزل شماره ۸۷۱: آمد بهار خرم و رحمت نثار شد}
\label{sec:0871}
\addcontentsline{toc}{section}{\nameref{sec:0871}}
\begin{longtable}{l p{0.5cm} r}
آمد بهار خرم و رحمت نثار شد
&&
سوسن چو ذوالفقار علی آبدار شد
\\
اجزای خاک حامله بودند از آسمان
&&
نه ماه گشت حامله زان بی‌قرار شد
\\
گلنار پرگره شد و جوبار پرزره
&&
صحرا پر از بنفشه و که لاله زار شد
\\
اشکوفه لب گشاد که هنگام بوسه گشت
&&
بگشاد سر و دست که وقت کنار شد
\\
گلزار چرخ چونک گلستان دل بدید
&&
در رو کشید ابر و ز دل شرمسار شد
\\
آن خار می‌گریست که ای عیب پوش خلق
&&
شد مستجاب دعوت او گلعذار شد
\\
شاه بهار بست کمر را به معذرت
&&
هر شاخ و هر درخت از او تاجدار شد
\\
هر چوب در تجمل چون بزم میر گشت
&&
گر در دو دست موسی یک چوب مار شد
\\
زنده شدند بار دگر کشتگان دی
&&
تا منکر قیامت بی‌اعتبار شد
\\
اصحاب کهف باغ ز خواب اندرآمدند
&&
چون لطف روح بخش خدا یار غار شد
\\
ای زنده گشتگان به زمستان کجا بدیت
&&
آن سو که وقت خواب روان را مطار شد
\\
آن سو که هر شبی بپرد این حواس و روح
&&
آن سو که هر شبی نظر و انتظار شد
\\
مه چون هلال بود سفر کرد آن طرف
&&
بدری منور آمد و شمع دیار شد
\\
این پنج حس ظاهر و پنج دگر نهان
&&
لنگ و ملول رفت و سحر راهوار شد
\\
بربند این دهان و مپیمای باد بیش
&&
کز باد گفت راه نظر پرغبار شد
\\
\end{longtable}
\end{center}
