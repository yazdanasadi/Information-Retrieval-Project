\begin{center}
\section*{غزل شماره ۱۹۳۱: عقل از کف عشق خورد افیون}
\label{sec:1931}
\addcontentsline{toc}{section}{\nameref{sec:1931}}
\begin{longtable}{l p{0.5cm} r}
عقل از کف عشق خورد افیون
&&
هش دار جنون عقل اکنون
\\
عشق مجنون و عقل عاقل
&&
امروز شدند هر دو مجنون
\\
جیحون که به عشق بحر می رفت
&&
دریا شد و محو گشت جیحون
\\
در عشق رسید بحر خون دید
&&
بنشست خرد میانه خون
\\
بر فرق گرفت موج خونش
&&
می برد ز هر سوی به بی‌سون
\\
تا گم کردش تمام از خود
&&
تا گشت به عشق چست و موزون
\\
در گم شدگی رسید جایی
&&
کان جا نه زمین بود نه گردون
\\
گر پیش رود قدم ندارد
&&
ور بنشیند پس او است مغبون
\\
ناگاه بدید زان سوی محو
&&
زان سوی جهان نور بی‌چون
\\
یک سنجق و صد هزار نیزه
&&
از نور لطیف گشت مفتون
\\
آن پای گرفته‌اش روان شد
&&
می رفت در آن عجیب‌هامون
\\
تا بو که رسد قدم بدان جا
&&
تا رسته شود ز خویش و مادون
\\
پیش آمد در رهش دو وادی
&&
یک آتش بد یکیش گلگون
\\
آواز آمد که رو در آتش
&&
تا یافت شوی به گلستان هون
\\
ور زانک به گلستان درآیی
&&
خود را بینی در آتش و تون
\\
بر پشت فلک پری چو عیسی
&&
و اندر بالا فرو چو قارون
\\
بگریز و امان شاه جان جو
&&
از جمله عقیله‌ها تو بیرون
\\
آن شمس الدین و فخر تبریز
&&
کز هر چه صفت کنیش افزون
\\
\end{longtable}
\end{center}
