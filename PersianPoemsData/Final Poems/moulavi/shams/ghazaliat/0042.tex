\begin{center}
\section*{غزل شماره ۴۲: کار تو داری صنما قدر تو باری صنما}
\label{sec:0042}
\addcontentsline{toc}{section}{\nameref{sec:0042}}
\begin{longtable}{l p{0.5cm} r}
کار تو داری صنما قدر تو باری صنما
&&
ما همه پابسته تو شیر شکاری صنما
\\
دلبر بی‌کینه ما شمع دل سینه ما
&&
در دو جهان در دو سرا کار تو داری صنما
\\
ذره به ذره بر تو سجده کنان بر در تو
&&
چاکر و یاری گر تو آه چه یاری صنما
\\
هر نفسی تشنه ترم بسته جوع البقرم
&&
گفت که دریا بخوری گفتم کری صنما
\\
هر کی ز تو نیست جدا هیچ نمیرد به خدا
&&
آنگه اگر مرگ بود پیش تو باری صنما
\\
نیست مرا کار و دکان هستم بی‌کار جهان
&&
زان که ندانم جز تو کارگزاری صنما
\\
خواه شب و خواه سحر نیستم از هر دو خبر
&&
کیست خبر چیست خبر روزشماری صنما
\\
روز مرا دیدن تو شب غم ببریدن تو
&&
از تو شبم روز شود همچو نهاری صنما
\\
باغ پر از نعمت من گلبن بازینت من
&&
هیچ ندید و نبود چون تو بهاری صنما
\\
جسم مرا خاک کنی خاک مرا پاک کنی
&&
باز مرا نقش کنی ماه عذاری صنما
\\
فلسفیک کور شود نور از او دور شود
&&
زو ندمد سنبل دین چونک نکاری صنما
\\
فلسفی این هستی من عارف تو مستی من
&&
خوبی این زشتی آن هم تو نگاری صنما
\\
\end{longtable}
\end{center}
