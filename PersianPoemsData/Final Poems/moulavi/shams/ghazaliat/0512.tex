\begin{center}
\section*{غزل شماره ۵۱۲: صبر مرا آینه بیماریست}
\label{sec:0512}
\addcontentsline{toc}{section}{\nameref{sec:0512}}
\begin{longtable}{l p{0.5cm} r}
صبر مرا آینه بیماریست
&&
آینه عاشق غمخواریست
\\
درد نباشد ننماید صبور
&&
که دل او روشن یا تاریست
\\
آینه جویی‌ست نشان جمال
&&
که رخم از عیب و کلف عاریست
\\
ور کلفی باشد عاریتیست
&&
قابل داروست و تب افشاریست
\\
آینه رنج ز فرعون دور
&&
کان رخ او رنگی و زنگاریست
\\
چند هزاران سر طفلان برید
&&
کم ز قضا دردسری ساریست
\\
من در آن خوف ببندم تمام
&&
چون که مرا حکم و شهی جاریست
\\
گفت قضا بر سر و سبلت مخند
&&
کاین قلمی رفته ز جباریست
\\
کور شو امروز که موسی رسید
&&
در کف او خنجر قهاریست
\\
حلق بکش پیش وی و سر مپیچ
&&
کاین نه زمان فن و مکاریست
\\
سبط که سرشان بشکستی به ظلم
&&
بعد توشان دولت و پاداریست
\\
خار زدی در دل و در دیدشان
&&
این دمشان نوبت گلزاریست
\\
خلق مرا زهر خورانیده‌ای
&&
از منشان داد شکرباریست
\\
از تو کشیدند خمار دراز
&&
تا به ابدشان می و خماریست
\\
هیزم دیک فقرا ظالمست
&&
پخته بدو گردد کو ناریست
\\
دم نزدم زان که دم من سکست
&&
نوبت خاموشی و ستاریست
\\
خامش کن که تا بگوید حبیب
&&
آن سخنان کز همه متواریست
\\
\end{longtable}
\end{center}
