\begin{center}
\section*{غزل شماره ۲۸۱۵: بمشو همره مرغان که چنین بی‌پر و بالی}
\label{sec:2815}
\addcontentsline{toc}{section}{\nameref{sec:2815}}
\begin{longtable}{l p{0.5cm} r}
بمشو همره مرغان که چنین بی‌پر و بالی
&&
چو نه میری نه وزیری بن سبلت به چه مالی
\\
چو هیاهوی برآری و نبینند سپاهی
&&
بشناسند همه کس که تو طبلی و دوالی
\\
چو خلیفه پسری تو بنه آن طبل ز گردن
&&
بستان خنجر و جوشن که سپهدار جلالی
\\
به خدا صاحب باغی تو ز هر باغ چه دزدی
&&
بفروش از رز خویشت همه انگور حلالی
\\
تو نه آن بدر کمالی که دهی نور و نگیری
&&
بستان نور چو سائل که تو امروز هلالی
\\
هله ای عشق برافشان گهر خویش بر اختر
&&
که همه اختر و ماهند و تو خورشیدمثالی
\\
بده آن دست به دستم مکشان دست که مستم
&&
که شراب است و کباب است و یکی گوشه‌ای خالی
\\
بدوان مست و خرامان به سوی مجلس سلطان
&&
بنگر مجلس عالی که تویی مجلس عالی
\\
نه صداعی نه خماری نه غمت ماند نه زاری
&&
عسسی دان غم خود را به در شحنه و والی
\\
عسس و شحنه چه گویند حریفان ملک را
&&
همه در روی درافتند که بس خوب خصالی
\\
\end{longtable}
\end{center}
