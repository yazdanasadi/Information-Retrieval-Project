\begin{center}
\section*{غزل شماره ۲۲۳: کجاست مطرب جان تا ز نعره‌های صلا}
\label{sec:0223}
\addcontentsline{toc}{section}{\nameref{sec:0223}}
\begin{longtable}{l p{0.5cm} r}
کجاست مطرب جان تا ز نعره‌های صلا
&&
درافکند دم او در هزار سر سودا
\\
بگفته‌ام که نگویم ولیک خواهم گفت
&&
من از کجا و وفاهای عهدها ز کجا
\\
اگر زمین به سراسر بروید از توبه
&&
به یک دم آن همه را عشق بدرود چو گیا
\\
از آنک توبه چو بندست بند نپذیرد
&&
علو موج چو کهسار و غره دریا
\\
میان ابروت ای عشق این زمان گرهیست
&&
که نیست لایق آن روی خوب از آن بازآ
\\
مرا به جمله جهان کار کس نیاید خوش
&&
که کارهای تو دیدم مناسب و همتا
\\
چو آفتاب جمالت برآمد از مشرق
&&
ز ذره ذره شنیدم که نعم مولانا
\\
حلاوتیست در آن آب بحر زخارت
&&
که شد از او جگر آب را هم استسقا
\\
خدای پهلوی هر درد دارویی بنهاد
&&
چو درد عشق قدیمست ماند بی ز دوا
\\
وگر دوا بود این را تو خود روا داری
&&
به کاه گل که بیندوده است بام سما
\\
کسی که نوبت الفقر فخر زد جانش
&&
چه التفات نماید به تاج و تخت و لوا
\\
چو باغ و راغ حقایق جهان گرفت همه
&&
میان زهرگیاهی چرا چرند چرا
\\
دهان پرست سخن لیک گفت امکان نیست
&&
به جان جمله مردان بگو تو باقی را
\\
\end{longtable}
\end{center}
