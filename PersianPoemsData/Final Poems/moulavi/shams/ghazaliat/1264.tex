\begin{center}
\section*{غزل شماره ۱۲۶۴: می‌گفت چشم شوخش با طره سیاهش}
\label{sec:1264}
\addcontentsline{toc}{section}{\nameref{sec:1264}}
\begin{longtable}{l p{0.5cm} r}
می‌گفت چشم شوخش با طره سیاهش
&&
من دم دهم فلان را تو درربا کلاهش
\\
یعقوب را بگویم یوسف به قعر چاهست
&&
چون بر سر چه آید تو درفکن به چاهش
\\
ما شکل حاجیانیم جاسوس و رهزنانیم
&&
حاجی چو در ره آید ما خود زنیم راهش
\\
ما شاخ ارغوانیم در آب و می‌نماییم
&&
با نعل بازگونه چون ماه و چون سپاهش
\\
روباه دید دنبه در سبزه زار و می‌گفت
&&
هرگز کی دید دنبه بی‌دام در گیاهش
\\
وان گرگ از حریصی در دنبه چون نمک شد
&&
از دام بی‌خبر بد آن خاطر تباهش
\\
ابله چو اندرافتد گوید که بی‌گناهم
&&
بس نیست ای برادر آن ابلهی گناهش
\\
ابله کننده عشقست عشقی گزین تو باری
&&
کابله شدن بیرزد حسن و جمال و جاهش
\\
پای تو درد گیرد افسون جان بر او خوان
&&
آن پای گاو باشد کافسون اوست کاهش
\\
حلق تو درد گیرد همراه دم پذیرد
&&
خود حلق کی گشاید بی‌آه غصه کاهش
\\
تا پیشگاه عشقش چون باشد و چه باشد
&&
چون ما ز دست رفتیم از پای گاه جاهش
\\
تا چه جمال دارد آن نادره مطرز
&&
که سوخت جان ما را آن نقش کارگاهش
\\
ز اندیشه می‌گذارم تا خود چه حیله سازم
&&
با او که مکر و حیله تلقین کند الهش
\\
آن کس که گم کند ره با عقل بازگردد
&&
وان را که عقل گم شد از کی بود پناهش
\\
نی ما از آن شاهیم ما عقل و جان نخواهیم
&&
چه عقل و بند و پندش چه جان و آه آهش
\\
مستی فزود خامش تا نکته‌ای نرانی
&&
ای رفته لاابالی در خون نیکخواهش
\\
\end{longtable}
\end{center}
