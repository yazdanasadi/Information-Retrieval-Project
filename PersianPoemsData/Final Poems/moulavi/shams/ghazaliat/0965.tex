\begin{center}
\section*{غزل شماره ۹۶۵: دل گردون خلل کند چو مه تو نهان شود}
\label{sec:0965}
\addcontentsline{toc}{section}{\nameref{sec:0965}}
\begin{longtable}{l p{0.5cm} r}
دل گردون خلل کند چو مه تو نهان شود
&&
چو رسد تیر غمزه‌ات همه قدها کمان شود
\\
چو تو دلداریی کنی دو جهان جمله دل شود
&&
دل ما چون جهان شود همه دل‌ها جهان شود
\\
فتد آتش در این فلک که بنالد از آن ملک
&&
چو غم و دود عاشقان به سوی آسمان شود
\\
نبود رشک عشق تو بجهد خون عاشقان
&&
چو شفق بر سر افق همه گردون نشان شود
\\
چه زمان باشد آن زمان که بلرزد ز تو زمین
&&
چه عجب باشد آن مکان چو مکان لامکان شود
\\
ز خیال نگار من چو بخندد بهار من
&&
رخ او گلفشان شود نظرم گلستان شود
\\
بفشان گل که گلشنی همه را چشم روشنی
&&
به کرم گر نظر کنی چه شود چه زیان شود
\\
خوشم ار سر بداده‌ام چو درختان به باد من
&&
که به باغ جمال تو نظرم باغبان شود
\\
چه عجب گر ز مستیت خرف و سرگران شوم
&&
چو درختی که میوه‌اش بپزد سرگران شود
\\
چو بنفشه دوتا شدم چو سمن بی‌وفا شدم
&&
که دل لاله‌ها سیه ز غم ارغوان شود
\\
رخ یارم چو گلستان رخ زارم چو زعفران
&&
رخ او چون چنین بود رخ عاشق چنان شود
\\
همه نرگس شود رزان ز پی دید گلستان
&&
گل تو بهر بوسه‌اش همه شکل دهان شود
\\
به وصال بهار او چو بخندد دل چمن
&&
ز غم هجر جوی‌ها چو سرشکم روان شود
\\
چو پرست از محبتش دل آن عالم خل
&&
که درختش ز شکر دوست سراسر زبان شود
\\
چو سر از خاک برزنند ز درختان ندا رسد
&&
که تو هر چه نهان کنی همه روزی عیان شود
\\
گل سوری گشاد رخ به لجاج گل سه تو
&&
گل گفتش نمایمت چو گه امتحان شود
\\
ز تک خاک دانه‌ها سوی بالا برآمده
&&
که عنایت فتاده را به علی نردبان شود
\\
تو زمین خورنده بین بخورد دانه پرورد
&&
عجب این گرگ گرسنه رمه را چون شبان شود
\\
همه گرگان شبان شده همه دزدان چو پاسبان
&&
چه برد دزد عاشقان چو خدا پاسبان شود
\\
مشتاب ار چه باغ را ز کرم سفره سبز شد
&&
بنشین منتظر دمی که کنون وقت خوان شود
\\
ز رفیقان گلستان مرم از زخم خاربن
&&
که رفیق سلاح کش مدد کاروان شود
\\
خمش ای دل که گر کسی بود او صادق طلب
&&
جهت صدق طالبان خمشی‌ها بیان شود
\\
\end{longtable}
\end{center}
