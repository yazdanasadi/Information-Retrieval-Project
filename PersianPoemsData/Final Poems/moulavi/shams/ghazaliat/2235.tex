\begin{center}
\section*{غزل شماره ۲۲۳۵: آمد خیال آن رخ چون گلستان تو}
\label{sec:2235}
\addcontentsline{toc}{section}{\nameref{sec:2235}}
\begin{longtable}{l p{0.5cm} r}
آمد خیال آن رخ چون گلستان تو
&&
و آورد قصه‌های شکر از لبان تو
\\
گفتم بدو چه باخبری از ضمیر جان
&&
جان و جهان چه بی‌خبرند از جهان تو
\\
آخر چه بوده‌ای و چه بوده‌ست اصل تو
&&
آخر چه گوهری و چه بوده‌ست کان تو
\\
دلاله عشق بود و مرا سوی تو کشید
&&
اول غلام عشقم و آن گاه آن تو
\\
بنهاد دست بر دل پرخون که آن کیست
&&
هر چند شرم بود بگفتم کز آن تو
\\
بر چشم من فتاد ورا چشم گفت چیست
&&
گفتم مها دو ابر تر درفشان تو
\\
از خون به زعفران دلم دید لاله زار
&&
گفتم که گلرخا همه نقش و نشان تو
\\
هر جا که بوی کرد ز من بوی خویش یافت
&&
گفتم نکو نگر که چنینم به جان تو
\\
ای شمس دین مفخر تبریز جان ماست
&&
در حلقه وفا بر دردی کشان تو
\\
\end{longtable}
\end{center}
