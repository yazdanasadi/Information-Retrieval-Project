\begin{center}
\section*{غزل شماره ۱۰۴۸: خداوند خداوندان اسرار}
\label{sec:1048}
\addcontentsline{toc}{section}{\nameref{sec:1048}}
\begin{longtable}{l p{0.5cm} r}
خداوند خداوندان اسرار
&&
زهی خورشید در خورشید انوار
\\
ز عشق حسن تو خوبان مه رو
&&
به رقص اندر مثال چرخ دوار
\\
چو بنمایی ز خوبی دست بردی
&&
بماند دست و پای عقل از کار
\\
گشاده ز آتش او آب حیوان
&&
که آبش خوشترست ای دوست یا نار
\\
از آن آتش بروییدست گلزار
&&
و زان گلزار عالم‌های دل زار
\\
از آن گل‌ها که هر دم تازه‌تر شد
&&
نه زان گل‌ها که پژمردست پیرار
\\
نتاند کرد عشقش را نهان کس
&&
اگر چه عشق او دارد ز ما عار
\\
یکی غاریست هجرانش پرآتش
&&
عجب روزی برآرم سر از این غار
\\
ز انکارت بروید پرده‌هایی
&&
مکن در کار آن دلبر تو انکار
\\
چو گرگی می‌نمودی روی یوسف
&&
چون آن پرده غرض می‌گشت اظهار
\\
ز جان آدمی زاید حسدها
&&
ملک باش و به آدم ملک بسپار
\\
غذای نفس تخم آن غرض‌هاست
&&
چو کاریدی بروید آن به ناچار
\\
نداند گاو کردن بانگ بلبل
&&
نداند ذوق مستی عقل هشیار
\\
نزاید گرگ لطف روی یوسف
&&
و نی طاووس زاید بیضه مار
\\
به طراری ربود این عمرها را
&&
به پس فردا و فردا نفس طرار
\\
همه عمرت هم امروزست لاغیر
&&
تو مشنو وعده این طبع عیار
\\
کمر بگشا ز هستی و کمر بند
&&
به خدمت تا رهی زین نفس اغیار
\\
نمازت کی روا باشد که رویت
&&
به هنگام نمازست سوی بلغار
\\
در آن صحرا بچر گر مشک خواهی
&&
که می‌چرد در آن آهوی تاتار
\\
نمی‌بینی تغیرها و تحویل
&&
در افلاک و زمین و اندر آثار
\\
کی داند جوهر خوبت بگردد
&&
به خاکی کش ندارد سود غمخوار
\\
چو تو خربنده باشی نفس خود را
&&
به حلقه نازنینان باشی بس خوار
\\
اگر خواهی عطای رایگانی
&&
ز عالم‌های باقی ملک بسیار
\\
چنان جامی که ویرانی هوش است
&&
ز شمس حق و دین بستان و هش دار
\\
خداوند خداوندان باقی
&&
که نبودشان به مخدومیش انکار
\\
ز لطف جان او رفته بکارت
&&
چو دیدندنش ز جنت حور ابکار
\\
اگر نه پرده رشک الهی
&&
بپوشیدیش از دار و ز دیار
\\
که سنگ و خاک و آب و باد و آتش
&&
همه روحی شدندی مست و سیار
\\
به بازار بتان و عاشقان در
&&
ز نقش او بسوزد جمله بازار
\\
دو ده دان هر دو کون دو جهان را
&&
چه باشد ده که باشد اوش سالار
\\
که روح القدس پایش می ببوسید
&&
ندا آمد که پایش را مه آزار
\\
چه کم عقلی بود آن کس که این را
&&
برای جاه او گوید که مکثار
\\
به حق آنک آن شیر حقیقی
&&
چنین صید دلم کردست اشکار
\\
که از تبریز پیغامی فرستی
&&
که اینست لابه ما اندر اسحار
\\
\end{longtable}
\end{center}
