\begin{center}
\section*{غزل شماره ۱۶۲۵: دو هزار عهد کردم که سر جنون نخارم}
\label{sec:1625}
\addcontentsline{toc}{section}{\nameref{sec:1625}}
\begin{longtable}{l p{0.5cm} r}
دو هزار عهد کردم که سر جنون نخارم
&&
ز تو درشکست عهدم ز تو باد شد قرارم
\\
ز ره زیاده جویی به طریق خیره رویی
&&
بروم که کدخدایم غله بدروم بکارم
\\
همه حل و عقد عالم چو به دست غیب آمد
&&
من بوالفضول معجب تو بگو که بر چه کارم
\\
چو قضا به سخره خواهد که ز سبلتی بخندد
&&
سگ لنگ را بگوید که برس بدان شکارم
\\
چو بر اوش رحم آید خبرش کند که بنشین
&&
بهل اختیار خود را تو به پیش اختیارم
\\
اگرت شکار باید ز منت شکار خوشتر
&&
همه صیدهای جان را به نثار بر تو بارم
\\
نه ز دام من ملالی نه ز جام من وبالی
&&
نه نظیر من جمالی چه غریب و ندره یارم
\\
خمش ار دگر بگویم ز مقالت خوش او
&&
بپرد کبوتر دل سوی اولین مطارم
\\
تبریز و شمس دین شد سبب فروغ اختر
&&
رخ شمس از او منور به فراز سبز طارم
\\
\end{longtable}
\end{center}
