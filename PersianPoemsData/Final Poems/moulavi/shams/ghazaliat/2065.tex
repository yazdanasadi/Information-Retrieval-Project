\begin{center}
\section*{غزل شماره ۲۰۶۵: باز درآمد ز راه فتنه برانگیز من}
\label{sec:2065}
\addcontentsline{toc}{section}{\nameref{sec:2065}}
\begin{longtable}{l p{0.5cm} r}
باز درآمد ز راه فتنه برانگیز من
&&
باز کمر بست سخت یار به استیز من
\\
مطبخ دل را نگار باز قباله گرفت
&&
می‌شکند دیگ من کاسه و کفلیز من
\\
خانه خرابی گرفت ز آنک قنق زفت بود
&&
هیچ نگنجد فلک در در و دهلیز من
\\
راه قنق را گرفت غیرت و گفتش مرو
&&
جمله افق را گرفت ابر شکرریز من
\\
سر کن ای بوالفضول ای ز کشاکش ملول
&&
جاذبه خیزان او منگر در خیز من
\\
منت او را که او منت و شکر آفرید
&&
کز کف کفران گذشت مرکب شبدیز من
\\
رست رخم از عبس کاسه ز ننگ عدس
&&
آخر کاری بکرد اشک غم آمیز من
\\
اصل همه باغ‌ها جان همه لاغ‌ها
&&
چیست اگر زیرکی لاغ دلاویز من
\\
ای خضر راستین گوهر دریاست این
&&
از تو در این آستین همچو فراویز من
\\
چونک مرا یار خواند دست سوی من فشاند
&&
تیز فرس پیش راند خاطر سرتیز من
\\
چند نهان می‌کنم شمس حق مغتنم
&&
خواجگیی می‌کند خواجه تبریز من
\\
\end{longtable}
\end{center}
