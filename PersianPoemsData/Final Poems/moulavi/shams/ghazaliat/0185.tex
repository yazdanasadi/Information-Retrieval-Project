\begin{center}
\section*{غزل شماره ۱۸۵: از بس که ریخت جرعه بر خاک ما ز بالا}
\label{sec:0185}
\addcontentsline{toc}{section}{\nameref{sec:0185}}
\begin{longtable}{l p{0.5cm} r}
از بس که ریخت جرعه بر خاک ما ز بالا
&&
هر ذره خاک ما را آورد در علالا
\\
سینه شکاف گشته دل عشق باف گشته
&&
چون شیشه صاف گشته از جام حق تعالی
\\
اشکوفه‌ها شکفته وز چشم بد نهفته
&&
غیرت مرا بگفته می خور دهان میالا
\\
ای جان چو رو نمودی جان و دلم ربودی
&&
چون مشتری تو بودی قیمت گرفت کالا
\\
ابرت نبات بارد جورت حیات آرد
&&
درد تو خوش گوارد تو درد را مپالا
\\
ای عشق با توستم وز باده تو مستم
&&
وز تو بلند و پستم وقت دنا تدلی
\\
ماهت چگونه خوانم مه رنج دق دارد
&&
سروت اگر بخوانم آن راستست الا
\\
سرو احتراق دارد مه هم محاق دارد
&&
جز اصل اصل جان‌ها اصلی ندارد اصلا
\\
خورشید را کسوفی مه را بود خسوفی
&&
گر تو خلیل وقتی این هر دو را بگو لا
\\
گویند جمله یاران باطل شدند و مردند
&&
باطل نگردد آن کو بر حق کند تولا
\\
این خنده‌های خلقان برقیست دم بریده
&&
جز خنده‌ای که باشد در جان ز رب اعلا
\\
آب حیات حقست وان کو گریخت در حق
&&
هم روح شد غلامش هم روح قدس لالا
\\
\end{longtable}
\end{center}
