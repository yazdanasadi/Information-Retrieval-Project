\begin{center}
\section*{غزل شماره ۱۳۹۵: مطرب عشق ابدم زخمه عشرت بزنم}
\label{sec:1395}
\addcontentsline{toc}{section}{\nameref{sec:1395}}
\begin{longtable}{l p{0.5cm} r}
مطرب عشق ابدم زخمه عشرت بزنم
&&
ریش طرب شانه کنم سبلت غم را بکنم
\\
تا همه جان ناز شود چونک طرب ساز شود
&&
تا سر خم باز شود گل ز سرش دور کنم
\\
چونک خلیلی بده‌ام عاشق آتشکده‌ام
&&
عاشق جان و خردم دشمن نقش وثنم
\\
وقت بهارست و عمل جفتی خورشید و حمل
&&
جوش کند خون دلم آب شود برف تنم
\\
ای مه تابان شده‌ای از چه گدازان شده‌ای
&&
گفت گرفتار دلم عاشق روی حسنم
\\
عشق کسی می کشدم گوش کشان می بردم
&&
تیر بلا می رسدم زان همه تن چون مجنم
\\
گر چه در این شور و شرم غرقه بحر شکرم
&&
گر چه اسیر سفرم تازه به بوی وطنم
\\
یار وصالی بده‌ام جفت جمالی بده‌ام
&&
فلسفه برخواند قضا داد جدایی به فنم
\\
تا که رگی در تن من جنبد من سوی وطن
&&
باشم پران و دوان ای شه شیرین ذقنم
\\
دم به دم آن بوی خوشش وان طلب گوش کشش
&&
آب روان کرد مرا ساقی سرو و سمنم
\\
همره یعقوب شدم فتنه آن خوب شدم
&&
هدیه فرستد به کرم یوسف جان پیرهنم
\\
الحق جانا چه خوشی قوس وفا را تو کشی
&&
در دو جهان دیده بود هیچ کسی چون تو صنم
\\
بر بر او بربزنم گر چه برابر نزنم
&&
شیشه بر آن سنگ زنم بنده شیشه شکنم
\\
پیل به خرطوم جفا قاصد کعبه شده است
&&
من چو ابابیل حقم یاور هر کرگدنم
\\
صیقل هر آینه‌ام رستم هر میمنه‌ام
&&
قوت هر گرسنه‌ام انجم هر انجمنم
\\
معنی هر قد و خدم سایه لطف احدم
&&
کعبه هر نیک و بدم دایه باغ و چمنم
\\
آتش بدخوی بود سوزش هر کوی بود
&&
چونک نکوروی بود باشد خوب ختنم
\\
گر تو بدین کژ نگری کاسه زنی کوزه خوری
&&
سایه عدل صمدم جز که مناسب نتنم
\\
وقت شد ای شاه شهان سرور خوبان جهان
&&
که به کرم شرح کنی آنک نگوید دهنم
\\
\end{longtable}
\end{center}
