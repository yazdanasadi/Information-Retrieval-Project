\begin{center}
\section*{غزل شماره ۵۸۰: صلا یا ایها العشاق کان مه رو نگار آمد}
\label{sec:0580}
\addcontentsline{toc}{section}{\nameref{sec:0580}}
\begin{longtable}{l p{0.5cm} r}
صلا یا ایها العشاق کان مه رو نگار آمد
&&
میان بندید عشرت را که یار اندر کنار آمد
\\
بشارت می پرستان را که کار افتاد مستان را
&&
که بزم روح گستردند و باده بی‌خمار آمد
\\
قیامت در قیامت بین نگار سروقامت بین
&&
کز او عالم بهشتی شد هزاران نوبهار آمد
\\
چو او آب حیات آمد چرا آتش برانگیزد
&&
چو او باشد قرار جان چرا جان بی‌قرار آمد
\\
درآ ساقی دگرباره بکن عشاق را چاره
&&
که آهوچشم خون خواره چو شیر اندر شکار آمد
\\
چو کار جان به جان آمد ندای الامان آمد
&&
که لشکرهای عشق او به دروازه حصار آمد
\\
رود جان بداندیشش به شمشیر و کفن پیشش
&&
که هرک از عشق برگردد به آخر شرمسار آمد
\\
نه اول ماند و نی آخر مرا در عشق آن فاخر
&&
که عاشق همچو نی آمد و عشق او چو نار آمد
\\
اگر چه لطف شمس الدین تبریزی گذر دارد
&&
ز باد و آب و خاک و نار جان هر چهار آمد
\\
\end{longtable}
\end{center}
