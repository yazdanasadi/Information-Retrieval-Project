\begin{center}
\section*{غزل شماره ۲۶۵۱: کریما تو گلی یا جمله قندی}
\label{sec:2651}
\addcontentsline{toc}{section}{\nameref{sec:2651}}
\begin{longtable}{l p{0.5cm} r}
کریما تو گلی یا جمله قندی
&&
که چون بینی مرا چون گل بخندی
\\
عزیزا تو به بستان آن درختی
&&
که چون دیدم تو را بیخم بکندی
\\
چه کم گردد ز جاهت گر بپرسی
&&
که چونی در فراقم دردمندی
\\
من آنم کز فراقت مستمندم
&&
تو آنی که خلاص مستمندی
\\
در این مطبخ هزاران جان به خرج است
&&
ببین تو ای دل پرخون که چندی
\\
چو حلقه بر درت گر چه مقیمم
&&
چه چاره چون تو بر بام بلندی
\\
بیا ای زلف چوگان حکم داری
&&
که چون گویم در این میدان فکندی
\\
سپند از بهر آن باشد که سوزد
&&
دلا می‌سوز دلبر را سپندی
\\
بیا ای جام عشق شمس تبریز
&&
که درد کهنه را تو سودمندی
\\
\end{longtable}
\end{center}
