\begin{center}
\section*{غزل شماره ۱۳۷۲: این بار من یک بارگی در عاشقی پیچیده‌ام}
\label{sec:1372}
\addcontentsline{toc}{section}{\nameref{sec:1372}}
\begin{longtable}{l p{0.5cm} r}
این بار من یک بارگی در عاشقی پیچیده‌ام
&&
این بار من یک بارگی از عافیت ببریده‌ام
\\
دل را ز خود برکنده‌ام با چیز دیگر زنده‌ام
&&
عقل و دل و اندیشه را از بیخ و بن سوزیده‌ام
\\
ای مردمان ای مردمان از من نیاید مردمی
&&
دیوانه هم نندیشد آن کاندر دل اندیشیده‌ام
\\
دیوانه کوکب ریخته از شور من بگریخته
&&
من با اجل آمیخته در نیستی پریده‌ام
\\
امروز عقل من ز من یک بارگی بیزار شد
&&
خواهد که ترساند مرا پنداشت من نادیده‌ام
\\
من خود کجا ترسم از او شکلی بکردم بهر او
&&
من گیج کی باشم ولی قاصد چنین گیجیده‌ام
\\
از کاسهٔ استارگان وز خون گردون فارغم
&&
بهر گدارویان بسی من کاسه‌ها لیسیده‌ام
\\
من از برای مصلحت در حبس دنیا مانده‌ام
&&
حبس از کجا من از کجا مال که را دزدیده‌ام
\\
در حبس تن غرقم به خون وز اشک چشم هر حرون
&&
دامان خون آلود را در خاک می مالیده‌ام
\\
مانند طفلی در شکم من پرورش دارم ز خون
&&
یک بار زاید آدمی من بارها زاییده‌ام
\\
چندانک خواهی درنگر در من که نشناسی مرا
&&
زیرا از آن کم دیده‌ای من صدصفت گردیده‌ام
\\
در دیده من اندرآ وز چشم من بنگر مرا
&&
زیرا برون از دیده‌ها منزلگهی بگزیده‌ام
\\
تو مست مست سرخوشی من مست بی‌سر سرخوشم
&&
تو عاشق خندان لبی من بی‌دهان خندیده‌ام
\\
من طرفه مرغم کز چمن با اشتهای خویشتن
&&
بی‌دام و بی‌گیرنده‌ای اندر قفس خیزیده‌ام
\\
زیرا قفس با دوستان خوشتر ز باغ و بوستان
&&
بهر رضای یوسفان در چاه آرامیده‌ام
\\
در زخم او زاری مکن دعوی بیماری مکن
&&
صد جان شیرین داده‌ام تا این بلا بخریده‌ام
\\
چون کرم پیله در بلا در اطلس و خز می روی
&&
بشنو ز کرم پیله هم کاندر قبا پوسیده‌ام
\\
پوسیده‌ای در گور تن رو پیش اسرافیل من
&&
کز بهر من در صور دم کز گور تن ریزیده‌ام
\\
نی نی چو باز ممتحن بردوز چشم از خویشتن
&&
مانند طاووسی نکو من دیبه‌ها پوشیده‌ام
\\
پیش طبیبش سر بنه یعنی مرا تریاق ده
&&
زیرا در این دام نزه من زهرها نوشیده‌ام
\\
تو پیش حلوایی جان شیرین و شیرین جان شوی
&&
زیرا من از حلوای جان چون نیشکر بالیده‌ام
\\
عین تو را حلوا کند به زانک صد حلوا دهد
&&
من لذت حلوای جان جز از لبش نشنیده‌ام
\\
خاموش کن کاندر سخن حلوا بیفتد از دهن
&&
بی گفت مردم بو برد زان سان که من بوییده‌ام
\\
هر غوره‌ای نالان شده کای شمس تبریزی بیا
&&
کز خامی و بی‌لذتی در خویشتن چغزیده‌ام
\\
\end{longtable}
\end{center}
