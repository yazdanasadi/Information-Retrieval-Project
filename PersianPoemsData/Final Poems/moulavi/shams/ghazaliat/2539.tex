\begin{center}
\section*{غزل شماره ۲۵۳۹: یکی طوطی مژده آور یکی مرغی خوش آوازی}
\label{sec:2539}
\addcontentsline{toc}{section}{\nameref{sec:2539}}
\begin{longtable}{l p{0.5cm} r}
یکی طوطی مژده آور یکی مرغی خوش آوازی
&&
چه باشد گر به سوی ما کند هر روز پروازی
\\
دراندازد به جان عاقلان بی‌خبر سوزی
&&
بسازد بهر مشتاقان به رسم مطربان سازی
\\
کند هنبازی طوطی صبا را از برای شه
&&
که او را نیست در پاکی و بیناییش هنبازی
\\
بجوشد بار دیگر از جمالش شادی تازه
&&
درآید بار دیگر از وصالش در فلک تازی
\\
به ناگاهان نماید روی آن پشت و پناه من
&&
ببینی عقل ترسان را به پای عشق سربازی
\\
همه عاشق شوندش زار هم بی‌دین و هم بادین
&&
همه صادق شوند او را نماند هیچ طنازی
\\
شود گوش طبیعت هم ز سر غیب‌ها واقف
&&
شود دیده فروبسته ز خاک پای او بازی
\\
شود بازار مه رویان از آن مه رو فروبسته
&&
شود دروازه عشرت از آن می‌روی در بازی
\\
شود شب‌های تاریک فراق آن صنم روشن
&&
بگوید وصل خوش نکته به گوش هجر یک رازی
\\
که رسم و قاعده غم‌ها ز جان خلق بردارند
&&
رسیده عمر ما آخر نهد از عیش آغازی
\\
درون بحر بی‌پایان مرگ و نیستی جان‌ها
&&
بود ایمن چو بر دریا بود مرغاب یا قازی
\\
به غیر ناطقه غیرت نبودت هیچ بدگویی
&&
نبودستت به جز هم مشک زلفین تو غمازی
\\
که از عشقت بسی جان‌ها چو چوب خشک می‌سوزد
&&
ز غیرت گشته با خلقان یکی بدگو و همازی
\\
الا ای آنک یک پرتو از آن رخسار بنمایی
&&
خنک گردد همه دل‌ها نماند حسرت و آزی
\\
الا ای کان ربانی شمس الدین تبریزی
&&
رخ همچون زرم دارد برای وصل تو گازی
\\
\end{longtable}
\end{center}
