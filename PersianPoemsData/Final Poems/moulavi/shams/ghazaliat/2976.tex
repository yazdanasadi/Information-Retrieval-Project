\begin{center}
\section*{غزل شماره ۲۹۷۶: شد جادوی حرام و حق از جادوی بری}
\label{sec:2976}
\addcontentsline{toc}{section}{\nameref{sec:2976}}
\begin{longtable}{l p{0.5cm} r}
شد جادوی حرام و حق از جادوی بری
&&
بر تو حرام نیست که محبوب ساحری
\\
می‌بند و می‌گشا که همین است جادوی
&&
می‌بخش و می‌ربا که همین است داوری
\\
دریا بدیده‌ایم که در وی گهر بود
&&
دریا درون گوهر کی کرد باوری
\\
سحر حلال آمد بگشاد پر و بال
&&
افسانه گشت بابل و دستان سامری
\\
همیان زر نهاده و معیوب می‌خرد
&&
ای عاشقان کی دید که شد ماه مشتری
\\
امروز می‌گزید ز بازار اسپ او
&&
اسپان پشت ریش و یدک‌های لاغری
\\
گفتم که اسب مرده چنین راه کی برد
&&
گفتا که راه ما نتوان شد به لمتری
\\
کشتی شکسته باید در آبگیر خضر
&&
کشتی چو نشکنی تو نه کشتی که لنگری
\\
دنیا چو قنطره‌ست گذر کن چو پا شکست
&&
با پای ناشکسته از این پول نگذری
\\
زیرا رجوع ضد قدوم است و عکس او است
&&
فرمان ارجعی را منیوش سرسری
\\
\end{longtable}
\end{center}
