\begin{center}
\section*{غزل شماره ۶۶۸: رجب بیرون شد و شعبان درآمد}
\label{sec:0668}
\addcontentsline{toc}{section}{\nameref{sec:0668}}
\begin{longtable}{l p{0.5cm} r}
رجب بیرون شد و شعبان درآمد
&&
برون شد جان ز تن جانان درآمد
\\
دم جهل و دم غفلت برون شد
&&
دم عشق و دم غفران درآمد
\\
بروید دل گل و نسرین و ریحان
&&
چو از ابر کرم باران درآمد
\\
دهان جمله غمگینان بخندد
&&
بدین قندی که در دندان درآمد
\\
چو خورشید آدمی زربفت پوشد
&&
چو آن مه روی زرافشان درآمد
\\
بزن دست و بگو ای مطرب عشق
&&
که آن سرفتنه پاکوبان درآمد
\\
اگر دی رفت باقی باد امروز
&&
وگر عمر بشد عثمان درآمد
\\
همه عمر گذشته بازآید
&&
چو این اقبال جاویدان درآمد
\\
چو در کشتی نوحی مست خفته
&&
چه غم داری اگر طوفان درآمد
\\
منور شد چو گردون خاک تبریز
&&
چو شمس الدین در آن میدان درآمد
\\
\end{longtable}
\end{center}
