\begin{center}
\section*{غزل شماره ۱۱۷۵: انجیرفروش را چه بهتر}
\label{sec:1175}
\addcontentsline{toc}{section}{\nameref{sec:1175}}
\begin{longtable}{l p{0.5cm} r}
انجیرفروش را چه بهتر
&&
انجیرفروشی ای برادر
\\
یا ساقی عشقنا تذکر
&&
فالعیش بلا نداک ابتر
\\
ما را سر صنعت و دکان نیست
&&
ای ساقی جان کجاست ساغر
\\
لا تترکنا سدی صحایا
&&
الخیر ینال لا یوخر
\\
کم جوی وفا عتاب کم کن
&&
ای زنده کن هزار مضطر
\\
الحنطه حیث کان حنطه
&&
اذ کان کذاک یوم بیدر
\\
چون پیشه مرد زرگری شد
&&
هر شهر که رفت کیست زرگر
\\
ابرارک یشربون خمراً
&&
فی ظل سخایک المخیر
\\
خود دل دهدت که برنهی بار
&&
بر مرکب پشت ریش لاغر
\\
من کاسک للثری نصیب
&&
و الارض بذاک صار اخضر
\\
بگذار که می‌چرد ضعیفی
&&
در روضه رحمتت محرر
\\
یا ساقی‌هات لا تقصر
&&
یا طول حیاتنا المقصر
\\
در سایه دوست چون بود جان
&&
همچون ماهی میان کوثر
\\
طهر خطراتنا و طیب
&&
من کأس مدامک المطهر
\\
ما را بمران وگر برانی
&&
هم بر تو تنیم چون کبوتر
\\
و الفجر لذی لیال عشر
&&
من نهر رحیقک المفرج
\\
آمد عثمان شهاب دین هین
&&
واگو غزل مرا مکرر
\\
\end{longtable}
\end{center}
