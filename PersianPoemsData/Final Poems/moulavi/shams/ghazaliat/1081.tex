\begin{center}
\section*{غزل شماره ۱۰۸۱: ای صبا حالی ز خد و خال شمس الدین بیار}
\label{sec:1081}
\addcontentsline{toc}{section}{\nameref{sec:1081}}
\begin{longtable}{l p{0.5cm} r}
ای صبا حالی ز خد و خال شمس الدین بیار
&&
عنبر و مشک ختن از چین به قسطنطین بیار
\\
گر سلامی از لب شیرین او داری بگو
&&
ور پیامی از دل سنگین او داری بیار
\\
سر چه باشد تا فدای پای شمس الدین کنم
&&
نام شمس الدین بگو تا جان کنم بر او نثار
\\
خلعت خیر و لباس از عشق او دارد دلم
&&
حسن شمس الدین دثار و عشق شمس الدین شعار
\\
ما به بوی شمس دین سرخوش شدیم و می‌رویم
&&
ما ز جام شمس دین مستیم ساقی می میار
\\
ما دماغ از بوی شمس الدین معطر کرده‌ایم
&&
فارغیم از بوی عود و عنبر و مشک تتار
\\
شمس دین بر دل مقیم و شمس دین بر جان کریم
&&
شمس دین در یتیم و شمس دین نقد عیار
\\
من نه تنها می‌سرایم شمس دین و شمس دین
&&
می‌سراید عندلیب از باغ و کبک از کوهسار
\\
حسن حوران شمس دین و باغ رضوان شمس دین
&&
عین انسان شمس دین و شمس دین فخر کبار
\\
روز روشن شمس دین و چرخ گردان شمس دین
&&
گوهر کان شمس دین و شمس دین لیل و نهار
\\
شمس دین جام جمست و شمس دین بحر عظیم
&&
شمس دین عیسی دم است و شمس دین یوسف عذار
\\
از خدا خواهم ز جان خوش دولتی با او نهان
&&
جان ما اندر میان و شمس دین اندر کنار
\\
شمس دین خوشتر ز جان و شمس دین شکرستان
&&
شمس دین سرو روان و شمس دین باغ و بهار
\\
شمس دین نقل و شراب و شمس دین چنگ و رباب
&&
شمس دین خمر و خمار و شمس دین هم نور و نار
\\
نی خماری کز وی آید انده و حزن و ندم
&&
آن خمار شمس دین کز وی فزاید افتخار
\\
ای دلیل بی‌دلان و ای رسول عاشقان
&&
شمس تبریزی بیا زنهار دست از ما مدار
\\
\end{longtable}
\end{center}
