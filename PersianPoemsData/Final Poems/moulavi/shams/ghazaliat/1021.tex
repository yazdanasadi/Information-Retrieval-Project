\begin{center}
\section*{غزل شماره ۱۰۲۱: گرم درآ و دم مده باده بیار و غم ببر}
\label{sec:1021}
\addcontentsline{toc}{section}{\nameref{sec:1021}}
\begin{longtable}{l p{0.5cm} r}
گرم درآ و دم مده باده بیار و غم ببر
&&
ای دل و جان هر طرف چشم و چراغ هر سحر
\\
هم طرب سرشته‌ای هم طلب فرشته‌ای
&&
هم عرصات گشته‌ای پر ز نبات و نیشکر
\\
خیز که رسته خیز شد روز نبات ریز شد
&&
با خردم ستیز شد هین بربا از او خبر
\\
خوش خبران غلام تو رطل گران سلام تو
&&
چون شنوند نام تو یاوه کنند پا و سر
\\
خیز که روز می‌رود فصل تموز می‌رود
&&
رفت و هنوز می‌رود دیو ز سایه عمر
\\
ای بشنیده آه جان باده رسان ز راه جان
&&
پشت دل و پناه جان پیش درآ چو شیر نر
\\
مست و خراب و شاد و خوش می‌گذری ز پنج و شش
&&
قافله را بکش بکش خوش سفریست این سفر
\\
لحظه به لحظه دم به دم می بده و بسوز غم
&&
نوبت تست ای صنم دور توست ای قمر
\\
عقل رباست و دلربا در تبریز شمس دین
&&
آن تبریز چون بصر شمس در اوست چون نظر
\\
گر چه بصر عیان بود نور در او نهان بود
&&
دیده نمی‌شود نظر جز به بصیرتی دگر
\\
\end{longtable}
\end{center}
