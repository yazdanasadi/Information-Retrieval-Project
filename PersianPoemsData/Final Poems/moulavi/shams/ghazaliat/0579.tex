\begin{center}
\section*{غزل شماره ۵۷۹: دگرباره سر مستان ز مستی در سجود آمد}
\label{sec:0579}
\addcontentsline{toc}{section}{\nameref{sec:0579}}
\begin{longtable}{l p{0.5cm} r}
دگرباره سر مستان ز مستی در سجود آمد
&&
مگر آن مطرب جان‌ها ز پرده در سرود آمد
\\
سراندازان و جانبازان دگرباره بشوریدند
&&
وجود اندر فنا رفت و فنا اندر وجود آمد
\\
دگرباره جهان پر شد ز بانگ صور اسرافیل
&&
امین غیب پیدا شد که جان را زاد و بود آمد
\\
ببین اجزای خاکی را که جان تازه پذرفتند
&&
همه خاکیش پاکی شد زیان‌ها جمله سود آمد
\\
ندارد رنگ آن عالم ولیک از تابه دیده
&&
چو نور از جان رنگ آمیز این سرخ و کبود آمد
\\
نصیب تن از این رنگست نصیب جان از این لذت
&&
ازیرا ز آتش مطبخ نصیب دیگ دود آمد
\\
بسوز ای دل که تا خامی نیاید بوی دل از تو
&&
کجا دیدی که بی‌آتش کسی را بوی عود آمد
\\
همیشه بوی با عودست نه رفت از عود و نه آمد
&&
یکی گوید که دیر آمد یکی گوید که زود آمد
\\
ز صف نگریخت شاهنشه ولی خود و زره پرده‌ست
&&
حجاب روی چون ماهش ز زخم خلق خود آمد
\\
\end{longtable}
\end{center}
