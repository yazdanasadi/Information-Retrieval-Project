\begin{center}
\section*{غزل شماره ۲۲۶۵: الیوم من الوصل نسیم و سعود}
\label{sec:2265}
\addcontentsline{toc}{section}{\nameref{sec:2265}}
\begin{longtable}{l p{0.5cm} r}
الیوم من الوصل نسیم و سعود
&&
الیوم اری الحب علی العهد فعودوا
\\
رفته‌ست رقیب و بر آن یار نبود او
&&
بی‌زحمت دشمن دم عشاق شنود او
\\
یا قلب ابشرک به وصل و رحیق
&&
ما فاتک من دهرک الیوم یعود
\\
شکر است عدو رفته و ما همدم جامیم
&&
ما سرخ و سپید از طرب و کور و کبود او
\\
یا حب حنا نیک تجلیت بوصل
&&
الروح فدا روحک بالروح تجود
\\
ما را که برای دل حساد جفا گفت
&&
امروز چو خلوت شد ما را بستود او
\\
هذا قمر قد غلب الشمس بنور
&&
من طالعه الیوم علی الشمس یسود
\\
امروز نقاب از رخ خود ماه برانداخت
&&
بر طلعت خورشید و مه و زهره فزود او
\\
ما اکثر ما قد خفض العیش به هجر
&&
للعیش من الیوم نهوض و صعود
\\
پیوسته ز خورشید ستاند مه نو نور
&&
این مه که به خورشید دهد نور چه بود او
\\
یا قلب تمتع و طب ان شکورا
&&
الحب شفیق لک و الله ودود
\\
این دم سپه عشق چه خوش دست گشادند
&&
چون یک گره از طره پربند گشود او
\\
الحب الی المجلس والله سقانا
&&
و السکر من القهوه کالدهر ولود
\\
آن غم که ز عشاق بسی گرد برآورد
&&
بیرون ز در است این دم و از بام فرود او
\\
الیوم من العیش لقاء و شفا
&&
الیوم من السکر رکوع و سجود
\\
آن ساغر لاغرشده را داروی دل ده
&&
دیر است که محروم شد از ذوق وجود او
\\
یا قوم الی العشق انیبوا و اجیبوا
&&
لما کتب الله علی العشق خلود
\\
امروز صلا می‌زند این خفته دلان را
&&
آن عشق سماوی که نخفت و نغنود او
\\
العشق من الکون حیات و لباب
&&
و العیش سوی العشق قشور و جلود
\\
هر دوست که از عشق به دنیات کشاند
&&
خود دشمن تو او است یقین دان و حسود او
\\
لا تنطق فی العشق و یکفیک انین
&&
فالمخلص للعاشق صبر و جحود
\\
بس کن تو مگو هیچ که تا اشک بگوید
&&
دل خود چو بسوزد بدهد بوی چو عود او
\\
\end{longtable}
\end{center}
