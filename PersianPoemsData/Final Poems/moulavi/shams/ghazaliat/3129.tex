\begin{center}
\section*{غزل شماره ۳۱۲۹: رضیت بما قسم‌الله لی}
\label{sec:3129}
\addcontentsline{toc}{section}{\nameref{sec:3129}}
\begin{longtable}{l p{0.5cm} r}
رضیت بما قسم‌الله لی
&&
و فوضت امری دلی خالقی
\\
لقد احسن‌الله فیما مضی
&&
کذالک یحسن فیما بقی
\\
ایا ساقی جان هر متقی
&&
بگردان چو مردان، می راوقی
\\
بخر جان و دلرا ز اندیشها
&&
که بر جانها حاکم مطلقی
\\
بهشت رخت گر تجلی کند
&&
نه دوزخ بماند، نه در وی شقی
\\
اگر تو گریزی ز ما، سابقی
&&
ور از تو گریزیم، تولا حقی
\\
میان شب و روز فرقی نماند
&&
چو ماهت نه غربیست، نی مشرقی
\\
به صد لابه مخمور را می دهی
&&
کی دیدست ساقی بدین مشفقی؟!
\\
شراب سخن بخش رقاص کن
&&
که گردد کلوخ از تفش منطقی
\\
چو حق گول جستست و قلب سلیم
&&
دلا زیرکی می‌کنی؟ احمقی
\\
ز فکرت دل و جان گر آرام داشت
&&
چرا رفت در سکر و در موسقی؟!
\\
تو تنها چرایی اگر خوش خویی؟!
&&
تو عذرا چرایی اگر وامقی؟!
\\
جعل وش ز گل خویشتن در کشی
&&
همان چرک می‌کش، بدان لایقی
\\
همه خارکس دان، اگر پادشاست
&&
بجز خار خار، و غم عاشقی
\\
خمش کن، ببین حق را فتح باب
&&
چهددر فکرت نکتهٔ مغلقی؟!
\\
\end{longtable}
\end{center}
