\begin{center}
\section*{غزل شماره ۳۴۱: بیا کامروز ما را روز عیدست}
\label{sec:0341}
\addcontentsline{toc}{section}{\nameref{sec:0341}}
\begin{longtable}{l p{0.5cm} r}
بیا کامروز ما را روز عیدست
&&
از این پس عیش و عشرت بر مزیدست
\\
بزن دستی بگو کامروز شادی‌ست
&&
که روز خوش هم از اول پدیدست
\\
چو یار ما در این عالم کی باشد
&&
چنین عیدی به صد دوران کی دیدست
\\
زمین و آسمان‌ها پرشکر شد
&&
به هر سویی شکرها بردمیدست
\\
رسید آن بانگ موج گوهرافشان
&&
جهان پرموج و دریا ناپدیدست
\\
محمد باز از معراج آمد
&&
ز چارم چرخ عیسی دررسیدست
\\
هر آن نقدی کز این جا نیست قلبست
&&
میی کز جام جان نبود پلیدست
\\
زهی مجلس که ساقی بخت باشد
&&
حریفانش جنید و بایزیدست
\\
خماری داشتم من در ارادت
&&
ندانستم که حق ما را مریدست
\\
کنون من خفتم و پاها کشیدم
&&
چو دانستم که بختم می کشیدست
\\
\end{longtable}
\end{center}
