\begin{center}
\section*{غزل شماره ۱۰۵۲: نوریست میان شعر احمر}
\label{sec:1052}
\addcontentsline{toc}{section}{\nameref{sec:1052}}
\begin{longtable}{l p{0.5cm} r}
نوریست میان شعر احمر
&&
از دیده و وهم و روح برتر
\\
خواهی خود را بدو بدوزی
&&
برخیز و حجاب نفس بردر
\\
آن روح لطیف صورتی شد
&&
با ابرو و چشم و رنگ اسمر
\\
بنمود خدای بی چگونه
&&
بر صورت مصطفی پیمبر
\\
آن صورت او فنای صورت
&&
وان نرگس او چو روز محشر
\\
هر گه که به خلق بنگریدی
&&
گشتی ز خدا گشاده صد در
\\
چون صورت مصطفی فنا شد
&&
عالم بگرفت الله اکبر
\\
\end{longtable}
\end{center}
