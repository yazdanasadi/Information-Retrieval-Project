\begin{center}
\section*{غزل شماره ۴۵۰: از بامداد روی تو دیدن حیات ماست}
\label{sec:0450}
\addcontentsline{toc}{section}{\nameref{sec:0450}}
\begin{longtable}{l p{0.5cm} r}
از بامداد روی تو دیدن حیات ماست
&&
امروز روی خوب تو یا رب چه دلرباست
\\
امروز در جمال تو خود لطف دیگرست
&&
امروز هر چه عاشق شیدا کند سزاست
\\
امروز آن کسی که مرا دی بداد پند
&&
چون روی تو بدید ز من عذرها بخواست
\\
صد چشم وام خواهم تا در تو بنگرم
&&
این وام از کی خواهم و آن چشم خود که راست
\\
در پیش بود دولت امروز لاجرم
&&
می‌جست و می‌طپید دل بنده روزهاست
\\
از عشق شرم دارم اگر گویمش بشر
&&
می‌ترسم از خدای که گویم که این خداست
\\
ابروم می‌جهید و دل بنده می‌طپید
&&
این می‌نمود رو که چنین بخت در قفاست
\\
رقاصتر درخت در این باغ‌ها منم
&&
زیرا درخت بختم و اندر سرم صباست
\\
چون باشد آن درخت که برگش تو داده‌ای
&&
چون باشد آن غریب که همسایه هماست
\\
در ظل آفتاب تو چرخی همی‌زنیم
&&
کوری آنک گوید ظل از شجر جداست
\\
جان نعره می‌زند که زهی عشق آتشین
&&
کب حیات دارد با تو نشست و خاست
\\
چون بگذرد خیال تو در کوی سینه‌ها
&&
پای برهنه دل به در آید که جان کجاست
\\
روی زمین چو نور بگیرد ز ماه تو
&&
گویی هزار زهره و خورشید بر سماست
\\
در روزن دلم نظری کن چو آفتاب
&&
تا آسمان نگوید کان ماه بی‌وفاست
\\
قدم کمان شد از غم و دادم نشان کژ
&&
با عشق همچو تیرم اینک نشان راست
\\
در دل خیال خطه تبریز نقش بست
&&
کان خانه اجابت و دل خانه دعاست
\\
\end{longtable}
\end{center}
