\begin{center}
\section*{غزل شماره ۲۷۲۸: باغ است و بهار و سرو عالی}
\label{sec:2728}
\addcontentsline{toc}{section}{\nameref{sec:2728}}
\begin{longtable}{l p{0.5cm} r}
باغ است و بهار و سرو عالی
&&
ما می‌نرویم از این حوالی
\\
بگشای نقاب و در فروبند
&&
ماییم و تویی و خانه خالی
\\
امروز حریف خاص عشقیم
&&
برداشته جام لاابالی
\\
ای مطرب خوش نوای خوش نی
&&
باید که عظیم خوش بنالی
\\
ای ساقی شادکام خوش حال
&&
پیش آر شراب را تو حالی
\\
تا خوش بخوریم و خوش بخسبیم
&&
در سایه لطف لایزالی
\\
خوردی نه ز راه حلق و اشکم
&&
خوابی نه نتیجه لیالی
\\
ای دل خواهم که آن قدح را
&&
بر دیده و چشم خود بمالی
\\
چون نیست شوی تمام در می
&&
آن ساعت هست بر کمالی
\\
پاینده شوی از آن سقاهم
&&
بی مرگ و فنا و انتقالی
\\
دزدی بگذار و خوش همی‌رو
&&
ایمن ز شکنجه‌های والی
\\
گویی بنما که ایمنی کو
&&
رو رو که هنوز در سؤالی
\\
ای روز بدین خوشی چه روزی
&&
ای روز به از هزار سالی
\\
ای جمله روزها غلامت
&&
ایشان هجرند و تو وصالی
\\
ای روز جمال تو کی بیند
&&
ای روز عظیم باجمالی
\\
هم خود بینی جمال خود را
&&
و آن چشم که گوش او بمالی
\\
ای روز نه روز آفتابی
&&
تو روز ز نور ذوالجلالی
\\
خورشید کند سجود هر شام
&&
می‌خواهد از مهت هلالی
\\
ای روز میان روز پنهان
&&
ای روز مقیم لایزالی
\\
ای روزی روزها و شب‌ها
&&
ای لطف جنوبی و شمالی
\\
خامش کنم از کمال گفتن
&&
زیرا تو ورای هر کمالی
\\
پیدا نشوی به قال زیرا
&&
تو پیداتر ز قیل و قالی
\\
از قال شود خیال پیدا
&&
تو فوق توهم و خیالی
\\
و آن وهم و خیال تشنه توست
&&
ای داده تو آب را زلالی
\\
این هر دو در آب جان دهن خشک
&&
در عالم پر ز خویش خالی
\\
باقی غزل ورای پرده
&&
محجوب ز تو که در ملالی
\\
\end{longtable}
\end{center}
