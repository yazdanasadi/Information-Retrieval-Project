\begin{center}
\section*{غزل شماره ۳۰۸۸: حرام گشت از این پس فغان و غمخواری}
\label{sec:3088}
\addcontentsline{toc}{section}{\nameref{sec:3088}}
\begin{longtable}{l p{0.5cm} r}
حرام گشت از این پس فغان و غمخواری
&&
بهشت گشت جهان زانک تو جهان داری
\\
مثال ده که نروید ز سینه خار غمی
&&
مثال ده که کند ابر غم گهرباری
\\
مثال ده که نیاید ز صبح غمازی
&&
مثال ده که نگردد جهان به شب تاری
\\
مثال ده که نریزد گلی ز شاخ درخت
&&
مثال ده که کند توبه خار از خاری
\\
مثال ده که رهد حرص از گداچشمی
&&
مثال ده که طمع وارهد ز طراری
\\
مثال گر ندهی حسن بی‌مثال تو بس
&&
که مستی دل و جانست و خصم هشیاری
\\
چو شب به خلوت معراج تو مشرف شد
&&
به آفتاب نظر می‌کند به صد خواری
\\
ز رشک نیشکرت نی هزار ناله کند
&&
ز چنگ هجر تو گیرند چنگ‌ها زاری
\\
ز تف عشق تو سوزی است در دل آتش
&&
هم از هوای تو دارد هوا سبکساری
\\
برای خدمت تو آب در سجود رود
&&
ز درد توست بر این خاک رنگ بیماری
\\
ز عشق تابش خورشید تو به وقت طلوع
&&
بلند کرد سر آن کوه نی ز جباری
\\
که تا نخست برو تابد آن تف خورشید
&&
نخست او کند آن نور را خریداری
\\
تنا ز کوه بیاموز سر به بالا دار
&&
که کان عشق خدایی نه کم ز کهساری
\\
مکن به زیر و به بالا به لامکان کن سر
&&
که هست شش جهت آن جا تو را نگوساری
\\
به دل نگر که دل تو برون شش جهت است
&&
که دل تو را برهاند از این جگرخواری
\\
روانه باش به اسرار و می تماشا کن
&&
ز آسمان بپذیر این لطیف رفتاری
\\
چو غوره از ترشی رو به سوی انگوری
&&
چو نی برو ز نیی جانب شکرباری
\\
حلاوت شکر او گلوی من بگرفت
&&
بماندم از رخ خوبش ز خوب گفتاری
\\
بگو به عشق که ای عشق خوش گلوگیری
&&
گه جفا و وفا خوب و خوب کرداری
\\
گلو چو سخت بگیری سبک برآید جان
&&
درآیدم ز تو جان چون گلوم افشاری
\\
گلوی خود به رسن زان سپرد خوش منصور
&&
دلا چو بوی بری صد گلو تو بسپاری
\\
ز کودکی تو به پیری روانه‌ای و دوان
&&
ولیکن آن حرکت نیست فاش و اظهاری
\\
\end{longtable}
\end{center}
