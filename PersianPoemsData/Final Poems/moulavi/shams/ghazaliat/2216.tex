\begin{center}
\section*{غزل شماره ۲۲۱۶: تن مزن ای پسر خوش دم خوش کام بگو}
\label{sec:2216}
\addcontentsline{toc}{section}{\nameref{sec:2216}}
\begin{longtable}{l p{0.5cm} r}
تن مزن ای پسر خوش دم خوش کام بگو
&&
بهر آرام دلم نام دلارام بگو
\\
پرده من مدران و در احسان بگشا
&&
شیشه دل مشکن قصه آن جام بگو
\\
ور در لطف ببستی در اومید مبند
&&
بر سر بام برآ و ز سر بام بگو
\\
ور حدیث و صفت او شر و شوری دارد
&&
صفت این دل تنگ شررآشام بگو
\\
چونک رضوان بهشتی تو صلایی درده
&&
چونک پیغامبر عشقی هله پیغام بگو
\\
آه زندانی این دام بسی بشنودیم
&&
حال مرغی که برسته‌ست از این دام بگو
\\
سخن بند مگو و صفت قند بگو
&&
صفت راه مگو و ز سرانجام بگو
\\
شرح آن بحر که واگشت همه جان‌ها او است
&&
که فزون است ز ایام و ز اعوام بگو
\\
ور تنور تو بود گرم و دعای تو قبول
&&
غم هر ممتحن سوخته خام بگو
\\
شکر آن بهره که ما یافته‌ایم از در فضل
&&
فرصت ار دست دهد هم بر بهرام بگو
\\
وگر از عام بترسی که سخن فاش کنی
&&
سخن خاص نهان در سخن عام بگو
\\
ور از آن نیز بترسی هله چون مرغ چمن
&&
دم به دم زمزمه بی‌الف و لام بگو
\\
همچو اندیشه که دانی تو و دانای ضمیر
&&
سخنی بی‌نقط و بی‌مد و ادغام بگو
\\
\end{longtable}
\end{center}
