\begin{center}
\section*{غزل شماره ۱۳۳۵: بانگ زدم نیم شبان کیست در این خانه دل}
\label{sec:1335}
\addcontentsline{toc}{section}{\nameref{sec:1335}}
\begin{longtable}{l p{0.5cm} r}
بانگ زدم نیم شبان کیست در این خانه دل
&&
گفت منم کز رخ من شد مه و خورشید خجل
\\
گفت که این خانه دل پر همه نقشست چرا
&&
گفتم این عکس تو است ای رخ تو رشک چگل
\\
گفت که این نقش دگر چیست پر از خون جگر
&&
گفتم این نقش من خسته دل و پای به گل
\\
بستم من گردن جان بردم پیشش به نشان
&&
مجرم عشق است مکن مجرم خود را تو بحل
\\
داد سر رشته به من رشته پرفتنه و فن
&&
گفت بکش تا بکشم هم بکش و هم مگسل
\\
تافت از آن خرگه جان صورت ترکم به از آن
&&
دست ببردم سوی او دست مرا زد که بهل
\\
گفتم تو همچو فلان ترش شدی گفت بدان
&&
من ترش مصلحتم نی ترش کینه و غل
\\
هر کی درآید که منم بر سر شاخش بزنم
&&
کاین حرم عشق بود ای حیوان نیست اغل
\\
هست صلاح دل و دین صورت آن ترک یقین
&&
چشم فرومال و ببین صورت دل صورت دل
\\
\end{longtable}
\end{center}
