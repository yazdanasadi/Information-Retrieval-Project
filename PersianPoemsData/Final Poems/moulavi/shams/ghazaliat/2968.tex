\begin{center}
\section*{غزل شماره ۲۹۶۸: ای چنگیان غیبی از راه خوش نوایی}
\label{sec:2968}
\addcontentsline{toc}{section}{\nameref{sec:2968}}
\begin{longtable}{l p{0.5cm} r}
ای چنگیان غیبی از راه خوش نوایی
&&
تشنه دلان خود را کردید بس سقایی
\\
جان تشنه ابد شد وین تشنگی ز حد شد
&&
یا ضربت جدایی یا شربت عطایی
\\
ای زهره مزین زین هر دو یک نوا زن
&&
یا پرده رهاوی یا پرده رهایی
\\
گر چنگ کژ نوازی در چنگ غم گدازی
&&
خوش زن نوا اگر نی مردی ز بی‌نوایی
\\
بی زخمه هیچ چنگی آب و نوا ندارد
&&
می‌کش تو زخمه زخمه گر چنگ بوالوفایی
\\
گر بگسلند تارت گیرند بر کنارت
&&
پیوند نو دهندت چندین دژم چرایی
\\
تو خود عزیز یاری پیوسته در کناری
&&
در بزم شهریاری بیرون ز جان و جایی
\\
خامش که سخت مستم بربند هر دو دستم
&&
ور نه قدح شکستم گر لحظه‌ای بپایی
\\
من پیر منبلانم بر خویش زخم رانم
&&
من مصلحت ندانم با ما تو برنیایی
\\
هم پاره پاره باشم هم خصم چاره باشم
&&
هم سنگ خاره باشم در صبر و بی‌نوایی
\\
از بس که تند و عاقم در دوزخ فراقم
&&
دوزخ ز احتراقم گیرد گریزپایی
\\
چون دید شور ما را عطار آشکارا
&&
بشکست طبل‌ها را در بزم کبریایی
\\
تبریز چون برفتم با شمس دین بگفتم
&&
بی حرف صد مقالت در وحدت خدایی
\\
\end{longtable}
\end{center}
