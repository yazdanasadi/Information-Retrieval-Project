\begin{center}
\section*{غزل شماره ۲۸۲۹: تو نفس نفس بر این دل هوسی دگر گماری}
\label{sec:2829}
\addcontentsline{toc}{section}{\nameref{sec:2829}}
\begin{longtable}{l p{0.5cm} r}
تو نفس نفس بر این دل هوسی دگر گماری
&&
چه خوش است این صبوری چه کنم نمی‌گذاری
\\
سر این خدای داند که مرا چه می‌دواند
&&
تو چه دانی ای دل آخر تو بر این چه دست داری
\\
به شکارگاه بنگر که زبون شدند شیران
&&
تو کجا گریزی آخر که چنین زبون شکاری
\\
تو از او نمی‌گریزی تو بدو همی‌گریزی
&&
غلطی غلط از آنی که میان این غباری
\\
ز شه ار خبر نداری که همی‌کند شکارت
&&
بنگر تو لحظه لحظه که شکار بی‌قراری
\\
چو به ترس هر کسی را طرفی همی‌دواند
&&
اگر او محیط نبود ز کجاست ترسگاری
\\
ز کسی است ترس لابد که ز خود کسی نترسد
&&
همه را مخوف دیدی جز از این همه‌ست باری
\\
به هلاک می‌دواند به خلاص می‌دواند
&&
به از این نباشد ای جان که تو دل بدو سپاری
\\
بنمایمت سپردن دل اگر دلم بخواهد
&&
دل خود بدو سپردم هم از او طلب تو یاری
\\
\end{longtable}
\end{center}
