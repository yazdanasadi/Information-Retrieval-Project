\begin{center}
\section*{غزل شماره ۲۹۵۷: اندر شکست جان شد پیدا لطیف جانی}
\label{sec:2957}
\addcontentsline{toc}{section}{\nameref{sec:2957}}
\begin{longtable}{l p{0.5cm} r}
اندر شکست جان شد پیدا لطیف جانی
&&
چون این جهان فروشد وا شد دگر جهانی
\\
بازار زرگران بین کز نقد زر چه پر شد
&&
گر چه ز زخم تیشه درهم شکست کانی
\\
تا تو خمش نکردی اندیشه گرد نامد
&&
وا شد دهان دل چون بربسته شد دهانی
\\
چندین هزار خانه کی گشت از زمانه
&&
تا در دل مهندس نقشش نشد نهانی
\\
سری است زان نهانتر صد نقش از آن مصور
&&
در خاطر مهندس و اندر دل فلانی
\\
چون دل صفا پذیرد آن سر جهان بگیرد
&&
وآنگه کسی نمیرد در دور لامکانی
\\
تبریز شمس دین را از لطف لابه‌ای کن
&&
کز باغ بی‌زمانی در ما نگر زمانی
\\
\end{longtable}
\end{center}
