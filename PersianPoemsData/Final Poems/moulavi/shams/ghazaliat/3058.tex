\begin{center}
\section*{غزل شماره ۳۰۵۸: ز بامداد درآورد دلبرم جامی}
\label{sec:3058}
\addcontentsline{toc}{section}{\nameref{sec:3058}}
\begin{longtable}{l p{0.5cm} r}
ز بامداد درآورد دلبرم جامی
&&
به ناشتاب چشانید خام را خامی
\\
نه باده‌اش ز عصیر و نه جام او ز زجاج
&&
نه نقل او چو خسیسان به قند و بادامی
\\
به باد باده مرا داد همچو که بر باد
&&
به آب گرم مرا کرد یار اکرامی
\\
بسی نمودم سالوس و او مرا می‌گفت
&&
مکن مکن که کم افتد چنین به ایامی
\\
طریق ناز گرفتم که نی برو امروز
&&
ستیزه کرد و مرا داد چند دشنامی
\\
چنین شراب و چو من ساقی و تو گویی نی
&&
کی گوید این نه مگر جاهلی و یا عامی
\\
هزار می‌نکند آنچ کرد دشنامش
&&
خراب گشتم نی ننگ ماند و نی نامی
\\
چگونه مست نگردی ز لطف آن شاهی
&&
که او خراب کند عالمی به پیغامی
\\
دلی بیابد تا این سخن تمام کنم
&&
خراب کرد دلم را چنان دلارامی
\\
سری نهادم بر پای او چو مستان من
&&
پدید شد سر مست مرا سرانجامی
\\
سر مرا به بر اندرگرفت و خوش بنواخت
&&
غریب دلبریی و بدیع انعامی
\\
وانگه از سر دقت به حاضران می‌گفت
&&
نه درخورست چنین مرغ با چنین دامی
\\
به باغ بلبل مستم صفیر من بشنو
&&
مباش در قفسی و کناره بامی
\\
فروکشیدم و باقی غزل نخواهم گفت
&&
مگر بیابم چون خویش دوزخ آشامی
\\
\end{longtable}
\end{center}
