\begin{center}
\section*{غزل شماره ۳۰۷۸: ز آب تشنه گرفته‌ست خشم می‌بینی}
\label{sec:3078}
\addcontentsline{toc}{section}{\nameref{sec:3078}}
\begin{longtable}{l p{0.5cm} r}
ز آب تشنه گرفته‌ست خشم می‌بینی
&&
گرسنه آمد و با نان همی‌کند بینی
\\
ز آفتاب گرفته‌ست خشم گازر نیز
&&
زهی حماقت و ادبیر و جهل و گر کینی
\\
تو را که معدن زر پیش خود همی‌خواند
&&
نمی‌روی و قراضه ز خاک می‌چینی
\\
قراضه‌هاست ز حسن ازل در این خوبان
&&
در آب و گل به چه آمد پی خوش آیینی
\\
چو کان حسن بچیند قراضه‌ها ز بتان
&&
به آب و گل بنماید که آن نه‌ای اینی
\\
تو جهد کن که سراسر همه قراضه شوی
&&
روی به معدن خود زانک جمله زرینی
\\
به شهد جذبه من آب جفا بیامیزم
&&
که شهد صرف گلو گیردت ز شیرینی
\\
کشیدمت نه دعاها کشند آمین را
&&
کشانه شو سوی من گر چه لنگ تخمینی
\\
به سوی بحر رو ای ماهی و مکش خود را
&&
تو با سعادت و اقبال خود چه در کینی
\\
اگر تو می‌نروی آن کرم تو را بکشد
&&
چنین کند کرم و رحمت سلاطینی
\\
وگر درشت کشد مر تو را مترسان دل
&&
که یوسفست کشنده تو ابن یامینی
\\
به تهمت و به درشتی و دزدیش بکشید
&&
که صاع زر تو ببردی به بد تو تعیینی
\\
چو خلوت آمد گفتش که من قرین توام
&&
تو لایقی بر من من دعا تو آمینی
\\
در آن مکان که مکان نیست قصرها داری
&&
در این مکان فنا چون حریص تمکینی
\\
هزار بارت گفتم خمش کن و تن زن
&&
تو از لجاج کنون احمدی و پارینی
\\
فداح روح حیاتی فانت تحیینی
&&
و انت تخلص دیباجتی من الطین
\\
و انت تلبس روحی مکرما حللا
&&
بها اعیش و تکفیننی لتکفینی
\\
ایا مفجر عین تقر عینینی
&&
سقاها سکراتی و شربها دینی
\\
\end{longtable}
\end{center}
