\begin{center}
\section*{غزل شماره ۲۵۵۱: دلی یا دیده عقلی تو یا نور خدابینی}
\label{sec:2551}
\addcontentsline{toc}{section}{\nameref{sec:2551}}
\begin{longtable}{l p{0.5cm} r}
دلی یا دیده عقلی تو یا نور خدابینی
&&
چراغ افروز عشاقی تو یا خورشیدآیینی
\\
چو نامت بشنود دل‌ها نگنجد در منازل‌ها
&&
شود حل جمله مشکل‌ها به نور لم یزل بینی
\\
بگفتم آفتابا تو مرا همراه کن با تو
&&
که جمله دردها را تو شفا گشتی و تسکینی
\\
بگفتا جان ربایم من قدم بر عرش سایم من
&&
به آب و گل کم آیم من مگر در وقت و هر حینی
\\
چو تو از خویش آگاهی ندانی کرد همراهی
&&
که آن معراج اللهی نیابد جز که مسکینی
\\
تو مسکینی در این ظاهر درونت نفس بس قاهر
&&
یکی سالوسک کافر که رهزن گشت و ره شینی
\\
مکن پوشیده از پیری چنین مو در چنین شیری
&&
یکی پیری که علم غیب زیر او است بالینی
\\
طبیب عاشقان است او جهان را همچو جان است او
&&
گداز آهنان است او به آهن داده تلبینی
\\
کند در حال گل را زر دهد در حال تن را سر
&&
از او انوار دین یابد روان و جان بی‌دینی
\\
در آن دهلیز و ایوانش بیا بنگر تو برهانش
&&
شده هر مرده از جانش یکی ویسی و رامینی
\\
ز شمس الدین تبریزی دلا این حرف می‌بیزی
&&
به امیدی که بازآید از آن خوش شاه شاهینی
\\
\end{longtable}
\end{center}
