\begin{center}
\section*{غزل شماره ۲۵۱۳: بیا ای عارف مطرب چه باشد گر ز خوش خویی}
\label{sec:2513}
\addcontentsline{toc}{section}{\nameref{sec:2513}}
\begin{longtable}{l p{0.5cm} r}
بیا ای عارف مطرب چه باشد گر ز خوش خویی
&&
چو شعری نور افشانی و زان اشعار برگویی
\\
به جان جمله مردان به درد جمله بادردان
&&
که برگو تا چه می‌خواهی و زین حیران چه می‌جویی
\\
از آن روی چو ماه او ز عشق حسن خواه او
&&
بیاموزید ای خوبان رخ افروزی و مه رویی
\\
از آن چشم سیاه او وزان زلف سه تاه او
&&
الا ای اهل هندستان بیاموزید هندویی
\\
ز غمزه تیراندازش کرشمه ساحری سازش
&&
هلا هاروت و ماروتم بیاموزید جادویی
\\
ایا اصحاب و خلوتیان شده دل را چنان جویان
&&
ز لعل جان فزای او بیاموزید دلجویی
\\
ز خرمنگاه شش گوشه نخواهی یافتن خوشه
&&
روان شو سوی بی‌سویان رها کن رسم شش سویی
\\
همه عالم ز تو نالان تو باری از چه می‌نالی
&&
چو از تو کم نشد یک مو نمی‌دانم چه می‌مویی
\\
فدایم آن کبوتر را که بر بام تو می‌پرد
&&
کجایی ای سگ مقبل که اهل آن چنان کویی
\\
چو آن عمر عزیز آمد چرا عشرت نمی‌سازی
&&
چو آن استاد جان آمد چرا تخته نمی‌شویی
\\
در این دام است آن آهو تو در صحرا چه می‌گردی
&&
گهر در خانه گم کردی به هر ویران چه می‌پویی
\\
به هر روزی در این خانه یکی حجره نوی یابی
&&
تو یک تو نیستی ای جان تفحص کن که صدتویی
\\
اگر کفری و گر دینی اگر مهری و گر کینی
&&
همو را بین همو را دان یقین می‌دان که با اویی
\\
بماند آن نادره دستان ولیکن ساقی مستان
&&
گرفت این دم گلوی من که بفشارم گر افزویی
\\
\end{longtable}
\end{center}
