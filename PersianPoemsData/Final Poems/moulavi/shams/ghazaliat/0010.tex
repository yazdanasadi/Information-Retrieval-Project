\begin{center}
\section*{غزل شماره ۱۰: مهمان شاهم هر شبی بر خوان احسان و وفا}
\label{sec:0010}
\addcontentsline{toc}{section}{\nameref{sec:0010}}
\begin{longtable}{l p{0.5cm} r}
مهمان شاهم هر شبی بر خوان احسان و وفا
&&
مهمان صاحب دولتم که دولتش پاینده با
\\
بر خوان شیران یک شبی بوزینه‌ای همراه شد
&&
استیزه رو گر نیستی او از کجا شیر از کجا
\\
بنگر که از شمشیر شه در قهرمان خون می‌چکد
&&
آخر چه گستاخی است این والله خطا والله خطا
\\
گر طفل شیری پنجه زد بر روی مادر ناگهان
&&
تو دشمن خود نیستی بر وی منه تو پنجه را
\\
آن کو ز شیران شیر خورد او شیر باشد نیست مرد
&&
بسیار نقش آدمی دیدم که بود آن اژدها
\\
نوح ار چه مردم وار بد طوفان مردم خوار بد
&&
گر هست آتش ذره‌ای آن ذره دارد شعله‌ها
\\
شمشیرم و خون ریز من هم نرمم و هم تیز من
&&
همچون جهان فانیم ظاهر خوش و باطن بلا
\\
\end{longtable}
\end{center}
