\begin{center}
\section*{غزل شماره ۱۶۲۱: چو غلام آفتابم هم از آفتاب گویم}
\label{sec:1621}
\addcontentsline{toc}{section}{\nameref{sec:1621}}
\begin{longtable}{l p{0.5cm} r}
چو غلام آفتابم هم از آفتاب گویم
&&
نه شبم نه شب پرستم که حدیث خواب گویم
\\
چو رسول آفتابم به طریق ترجمانی
&&
پنهان از او بپرسم به شما جواب گویم
\\
به قدم چو آفتابم به خرابه‌ها بتابم
&&
بگریزم از عمارت سخن خراب گویم
\\
به سر درخت مانم که ز اصل دور گشتم
&&
به میانه قشورم همه از لباب گویم
\\
من اگر چه سیب شیبم ز درخت بس بلندم
&&
من اگر خراب و مستم سخن صواب گویم
\\
چو دلم ز خاک کویش بکشیده است بویش
&&
خجلم ز خاک کویش که حدیث آب گویم
\\
بگشا نقاب از رخ که رخ تو است فرخ
&&
تو روا مبین که با تو ز پس نقاب گویم
\\
چو دلت چو سنگ باشد پر از آتشم چو آهن
&&
تو چو لطف شیشه گیری قدح و شراب گویم
\\
ز جبین زعفرانی کر و فر لاله گویم
&&
به دو چشم ناودانی صفت سحاب گویم
\\
چو ز آفتاب زادم به خدا که کیقبادم
&&
نه به شب طلوع سازم نه ز ماهتاب گویم
\\
اگرم حسود پرسد دل من ز شکر ترسد
&&
به شکایت اندرآیم غم اضطراب گویم
\\
بر رافضی چگونه ز بنی قحافه لافم
&&
بر خارجی چگونه غم بوتراب گویم
\\
چو رباب از او بنالد چو کمانچه رو درافتم
&&
چو خطیب خطبه خواند من از آن خطاب گویم
\\
به زبان خموش کردم که دل کباب دارم
&&
دل تو بسوزد ار من ز دل کباب گویم
\\
\end{longtable}
\end{center}
