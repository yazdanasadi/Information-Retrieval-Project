\begin{center}
\section*{غزل شماره ۲۶۸۸: صلا ای صوفیان کامروز باری}
\label{sec:2688}
\addcontentsline{toc}{section}{\nameref{sec:2688}}
\begin{longtable}{l p{0.5cm} r}
صلا ای صوفیان کامروز باری
&&
سماع است و شراب و عیش آری
\\
صلا که ساعتی دیگر نیابی
&&
ز مشرق تا به مغرب هوشیاری
\\
چنان در بحر مستی غرق گردند
&&
که دل در عشق خوبی خوش عذاری
\\
از این مستان ننوشی های و هویی
&&
وزین خوبان نبینی گوشواری
\\
در این مستان کجا وهمی رسیدی
&&
گر این مستان ننالند از خماری
\\
به صد عالم نگنجد از جلالت
&&
چنین سلطان و اعظم شهریاری
\\
ولیکن چون غبار انگیخت اسپش
&&
به وهم آمد کر و فر سواری
\\
دهان بربند کاین جا یک نظر نیست
&&
که بشناسد سواری از غباری
\\
\end{longtable}
\end{center}
