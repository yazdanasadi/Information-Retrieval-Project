\begin{center}
\section*{غزل شماره ۱۷۲۸: مرا اگر تو نخواهی منت به جان خواهم}
\label{sec:1728}
\addcontentsline{toc}{section}{\nameref{sec:1728}}
\begin{longtable}{l p{0.5cm} r}
مرا اگر تو نخواهی منت به جان خواهم
&&
وگر درم نگشایی مقیم درگاهم
\\
چو ماهیم که بیفکند موج بیرونش
&&
به غیر آب نباشد پناه و دلخواهم
\\
کجا روم به سر خویش کی دلی دارم
&&
من و تن و دل من سایه شهنشاهم
\\
به توست بیخودیم گر خراب و سرمستم
&&
به توست آگهی من اگر من آگاهم
\\
نه دلربام تویی گر مرا دلی باقی است
&&
نه کهربام تویی گر مثل پر کاهم
\\
نه از حلاوت حلوای بی‌حد لب توست
&&
که چون کلیچه فتاده کنون در افواهم
\\
ز هر دو عالم پهلوی خود تهی کردم
&&
چو هی نشسته به پهلوی لام اللهم
\\
ز جاه و سلطنت و سروری نیندیشم
&&
بس است دولت عشق تو منصب و جاهم
\\
چو قل هو الله مجموع غرق تنزیهم
&&
نه چون مشبهیان سرنگون اشباهم
\\
اگر تتار غمت خشم و ترکیی آرد
&&
به عشق و صبر کمربسته همچو خرگاهم
\\
اگر چه کاهل و بی‌گاه خیز قافله‌ام
&&
به سوی توست سفرهای گاه و بی‌گاهم
\\
برآ چو ماه تمام و تمام این تو بگو
&&
که زیر عقده هجرت بمانده چون ماهم
\\
\end{longtable}
\end{center}
