\begin{center}
\section*{غزل شماره ۱۹۳۷: هر خوشی که فوت شد از تو مباش اندوهگین}
\label{sec:1937}
\addcontentsline{toc}{section}{\nameref{sec:1937}}
\begin{longtable}{l p{0.5cm} r}
هر خوشی که فوت شد از تو مباش اندوهگین
&&
کو به نقشی دیگر آید سوی تو می دان یقین
\\
نی خوشی مر طفل را از دایگان و شیر بود
&&
چون برید از شیر آمد آن ز خمر و انگبین
\\
این خوشی چیزی است بی‌چون کآید اندر نقش‌ها
&&
گردد از حقه به حقه در میان آب و طین
\\
لطف خود پیدا کند در آب باران ناگهان
&&
باز در گلشن درآید سر برآرد از زمین
\\
گه ز راه آب آید گه ز راه نان و گوشت
&&
گه ز راه شاهد آید گه ز راه اسب و زین
\\
از پس این پرده‌ها ناگاه روزی سر کند
&&
جمله بت‌ها بشکند آنک نه آن است و نه این
\\
جان به خواب از تن برآید در خیال آید بدید
&&
تن شود معزول و عاطل صورتی دیگر مبین
\\
گویی اندر خواب دیدم همچو سروی خویش را
&&
روی من چون لاله زار و تن چو ورد و یاسمین
\\
آن خیال سرو رفت و جان به خانه بازگشت
&&
ان فی هذا و ذاک عبرة للعالمین
\\
ترسم از فتنه وگر نی گفتنی‌ها گفتمی
&&
حق ز من خوشتر بگوید تو مهل فتراک دین
\\
فاعلاتن فاعلاتن فاعلاتن فاعلات
&&
نان گندم گر نداری گو حدیث گندمین
\\
آخر ای تبریز جان اندر نجوم دل نگر
&&
تا ببینی شمس دنیا را تو عکس شمس دین
\\
\end{longtable}
\end{center}
