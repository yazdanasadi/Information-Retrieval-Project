\begin{center}
\section*{غزل شماره ۱۶۲: تو مرا جان و جهانی چه کنم جان و جهان را}
\label{sec:0162}
\addcontentsline{toc}{section}{\nameref{sec:0162}}
\begin{longtable}{l p{0.5cm} r}
تو مرا جان و جهانی چه کنم جان و جهان را
&&
تو مرا گنج روانی چه کنم سود و زیان را
\\
نفسی یار شرابم نفسی یار کبابم
&&
چو در این دور خرابم چه کنم دور زمان را
\\
ز همه خلق رمیدم ز همه بازرهیدم
&&
نه نهانم نه بدیدم چه کنم کون و مکان را
\\
ز وصال تو خمارم سر مخلوق ندارم
&&
چو تو را صید و شکارم چه کنم تیر و کمان را
\\
چو من اندر تک جویم چه روم آب چه جویم
&&
چه توان گفت چه گویم صفت این جوی روان را
\\
چو نهادم سر هستی چه کشم بار کهی را
&&
چو مرا گرگ شبان شد چه کشم ناز شبان را
\\
چه خوشی عشق چه مستی چو قدح بر کف دستی
&&
خنک آن جا که نشستی خنک آن دیده جان را
\\
ز تو هر ذره جهانی ز تو هر قطره چو جانی
&&
چو ز تو یافت نشانی چه کند نام و نشان را
\\
جهت گوهر فایق به تک بحر حقایق
&&
چو به سر باید رفتن چه کنم پای دوان را
\\
به سلاح احد تو ره ما را بزدی تو
&&
همه رختم ستدی تو چه دهم باج ستان را
\\
ز شعاع مه تابان ز خم طره پیچان
&&
دل من شد سبک ای جان بده آن رطل گران را
\\
منگر رنج و بلا را بنگر عشق و ولا را
&&
منگر جور و جفا را بنگر صد نگران را
\\
غم را لطف لقب کن ز غم و درد طرب کن
&&
هم از این خوب طلب کن فرج و امن و امان را
\\
بطلب امن و امان را بگزین گوشه گران را
&&
بشنو راه دهان را مگشا راه دهان را
\\
\end{longtable}
\end{center}
