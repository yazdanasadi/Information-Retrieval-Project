\begin{center}
\section*{غزل شماره ۱۶۷۵: دوش عشق شمس دین می باختیم}
\label{sec:1675}
\addcontentsline{toc}{section}{\nameref{sec:1675}}
\begin{longtable}{l p{0.5cm} r}
دوش عشق شمس دین می باختیم
&&
سوی رفعت روح می افراختیم
\\
در فراق روی آن معشوق جان
&&
ماحضر با عشق او می ساختیم
\\
در نثار عشق جان افزای او
&&
قالب از جان هر زمان پرداختیم
\\
عشق او صد جان دیگر می بداد
&&
ما در این داد و ستد پرداختیم
\\
همچو چنگ از حال خود خالی شدیم
&&
پرده عشاق را بنواختیم
\\
اندر آن پرده بده یک پردگی
&&
کز شعاعش پرده‌ها بشناختیم
\\
هر زمان خود را به سوی پرده‌ای
&&
حیله حیله پیشتر انداختیم
\\
برج برج و پرده پرده بعد از آن
&&
همچو ماه چارده می تاختیم
\\
رو نمود از سوی تبریز آفتاب
&&
تا دل از رخت طبیعت آختیم
\\
\end{longtable}
\end{center}
