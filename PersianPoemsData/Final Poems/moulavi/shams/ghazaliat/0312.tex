\begin{center}
\section*{غزل شماره ۳۱۲: به جان تو که مرو از میان کار مخسب}
\label{sec:0312}
\addcontentsline{toc}{section}{\nameref{sec:0312}}
\begin{longtable}{l p{0.5cm} r}
به جان تو که مرو از میان کار مخسب
&&
ز عمر یک شب کم گیر و زنده دار مخسب
\\
هزار شب تو برای هوای خود خفتی
&&
یکی شبی چه شود از برای یار مخسب
\\
برای یار لطیفی که شب نمی‌خسبد
&&
موافقت کن و دل را بدو سپار مخسب
\\
بترس از آن شب رنجوریی که تو تا روز
&&
فغان و یارب و یارب کنی به زار مخسب
\\
شبی که مرگ بیاید قنق کرک گوید
&&
به حق تلخی آن شب که ره سپار مخسب
\\
از آن زلازل هیبت که سنگ آب شود
&&
اگر تو سنگ نه‌ای آن به یاد آر مخسب
\\
اگر چه زنگی شب سخت ساقی چستست
&&
مگیر جام وی و ترس از آن خمار مخسب
\\
خدای گفت که شب دوستان نمی‌خسبند
&&
اگر خجل شده‌ای زین و شرمسار مخسب
\\
بترس از آن شب سخت عظیم بی‌زنهار
&&
ذخیره ساز شبی را و زینهار مخسب
\\
شنیده‌ای که مهان کام‌ها به شب یابند
&&
برای عشق شهنشاه کامیار مخسب
\\
چو مغز خشک شود تازه مغزیت بخشد
&&
که جمله مغز شوی ای امیدوار مخسب
\\
هزار بارت گفتم خموش و سودت نیست
&&
یکی بیار و عوض گیر صد هزار مخسب
\\
\end{longtable}
\end{center}
