\begin{center}
\section*{غزل شماره ۱۹۴۳: می گزید او آستین را شرمگین در آمدن}
\label{sec:1943}
\addcontentsline{toc}{section}{\nameref{sec:1943}}
\begin{longtable}{l p{0.5cm} r}
می گزید او آستین را شرمگین در آمدن
&&
بر سر کویی که پوشد جان‌ها حله بدن
\\
آن طرف رندان همه شب جامه‌ها را می کنند
&&
تا ببینی روز روشن ما و من بی‌ما و من
\\
رومیانش جامه دزد و زنگیانش جامه دوز
&&
شاد باش ای جامه دزد و آفرین ای جامه کن
\\
سرفرازی کار شمع و سرسپاری کار او
&&
شرط باشد هر دو کارش هر کی شد شمع لگن
\\
در سپردن هر کی زودتر در فروزش بیشتر
&&
سر بنه در زیر پای و دستکی بر هم بزن
\\
چون درآرد ماه رویی دست خود در گردنت
&&
ترک کن سالوس را تو خویش را بر وی فکن
\\
تا بریزی و برویی آن زمان در باغ او
&&
روی گل بر روی گل هم یاسمن بر یاسمن
\\
عاشقان اندرربوده از بتان روبندها
&&
زانک در وحدت نباشد نقش‌های مرد و زن
\\
بر سر گور بدن بین روح‌ها رقصان شده
&&
تا بدیده صد هزاران خویشتن بی‌خویشتن
\\
زلف عنبرسای او گوید به جان لولیان
&&
خیز لولی تا رسن بازی کنیم اینک رسن
\\
مرتضای عشق شمس الدین تبریزی ببین
&&
چون حسینم خون خود در زهر کش همچون حسن
\\
\end{longtable}
\end{center}
