\begin{center}
\section*{غزل شماره ۱۸۵۸: چو افتم من ز عشق دل به پای دلربای من}
\label{sec:1858}
\addcontentsline{toc}{section}{\nameref{sec:1858}}
\begin{longtable}{l p{0.5cm} r}
چو افتم من ز عشق دل به پای دلربای من
&&
از آن شادی بیاید جان نهان افتد به پای من
\\
وگر روزی در آن خدمت کنم تقصیر چون خامان
&&
شود دل خصم جان من کند هجران سزای من
\\
سحرگاهان دعا کردم که این جان باد خاک او
&&
شنیدم نعره آمین ز جان اندر دعای من
\\
چگونه راه برد این دل به سوی دلبر پنهان
&&
چگونه بوی برد این جان که هست او جان فزای من
\\
یکی جامی به پیش آورد من از ناز گفتم نی
&&
بگفتا نی مگو بستان برای اقتضای من
\\
چو از صافش چشیدم من مرا درداد یک دردی
&&
یکی دردی گران خواری که کامل شد صفای من
\\
\end{longtable}
\end{center}
