\begin{center}
\section*{غزل شماره ۸۵۸: وقتی خوشست ما را لابد نبید باید}
\label{sec:0858}
\addcontentsline{toc}{section}{\nameref{sec:0858}}
\begin{longtable}{l p{0.5cm} r}
وقتی خوشست ما را لابد نبید باید
&&
وقتی چنین به جانی جامی خرید باید
\\
ما را نبید و باده از خم غیب آید
&&
ما را مقام و مجلس عرش مجید باید
\\
هر جا فقیر بینی با وی نشست باید
&&
هر جا زحیر بینی از وی برید باید
\\
بگریز از آن فقیری کو بند لوت باشد
&&
ما را فقیر معنی چون بایزید باید
\\
از نور پاک چون زاد او باز پاک خواهد
&&
و آنک از حدث بزاید او را پلید باید
\\
اما چو قلب و نیکو ماننده‌اند با هم
&&
پیش چراغ یزدان آن را گزید باید
\\
بر دل نهاد قفلی یزدان و ختم کردش
&&
از بهر فتح این در در غم طپید باید
\\
سگ چون به کوی خسبد از قفل در چه باکش
&&
اصحاب خانه‌ها را فتح کلید باید
\\
سالی دو عید کردن کار عوام باشد
&&
ما صوفیان جان را هر دم دو عید باید
\\
جان گفت من مریدم زاینده جدیدم
&&
زایندگان نو را رزق جدید باید
\\
ما را از آن مفازه عیشیست تازه تازه
&&
آن را که تازه نبود او را قدید باید
\\
ای آمده چو سردان اندر سماع مردان
&&
زنده ز شخص مرده آخر بدید باید
\\
گر زانک چوب خشکی جز ز آتشی نخنبی
&&
ور زانک شاخ سبزی آخر خمید باید
\\
آن ذوق را گرفتم پستان مادر آمد
&&
بنهاد در دهانت آخر مکید باید
\\
خامش که در فصاحت عمر عزیز بردی
&&
در روضه خموشان چندی چرید باید
\\
ای شمس حق تبریز در گفتنم کشیدی
&&
روزی دو در خموشی دم درکشید باید
\\
\end{longtable}
\end{center}
