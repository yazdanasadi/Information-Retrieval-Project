\begin{center}
\section*{غزل شماره ۱۶۰۰: از شهنشه شمس دین من ساغری را یافتم}
\label{sec:1600}
\addcontentsline{toc}{section}{\nameref{sec:1600}}
\begin{longtable}{l p{0.5cm} r}
از شهنشه شمس دین من ساغری را یافتم
&&
در درون ساغرش چشمه خوری را یافتم
\\
تابش سینه و برت را خود ندارد چشم تاب
&&
شکر ایزد را که من زین دلبری را یافتم
\\
میرداد قهر چون ماری فروکوبد سرش
&&
آنک گوید در دو کونش هم سری را یافتم
\\
چون درون طره‌اش دریافتم دل را عجب
&&
در درون مشک رفتم عنبری را یافتم
\\
گر ببینی طوطی جان مرا گرد لبش
&&
می پرد پرک زنان که شکری را یافتم
\\
گر بپرسندت حکایت کن که من بر جام لعل
&&
عاشقی مستی جوانی می خوری را یافتم
\\
گر کسی منکر شود تو گردن او را ببند
&&
می کشانش روسیه که منکری را یافتم
\\
در میان طره‌اش رخسار چون آتش ببین
&&
گو میان مشک و عنبر مجمری را یافتم
\\
چون گشاید لعل را او تا نثار در کند
&&
گو که در خورشید از رحمت دری را یافتم
\\
چون دکان سرپزان سرها و دل‌ها پیش او
&&
هست بی‌پایان در آن سرها سری را یافتم
\\
چون نگه کردم سر من بود پر از عشق او
&&
من برون از هر دو عالم منظری را یافتم
\\
من به برج ثور دیدم منکر آن آفتاب
&&
گاو جستم من ز ثور و خود خری را یافتم
\\
من صف رستم دلان جستم بدیدم شاه را
&&
ترک آن کردم چو بی‌صف صفدری را یافتم
\\
من همی‌کشتی سوی تبریز راندم می نرفت
&&
پس ز جان بر کشتی خود لنگری را یافتم
\\
\end{longtable}
\end{center}
