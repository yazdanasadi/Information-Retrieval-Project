\begin{center}
\section*{غزل شماره ۲۵۲۳: دل آتش پرست من که در آتش چو گوگردی}
\label{sec:2523}
\addcontentsline{toc}{section}{\nameref{sec:2523}}
\begin{longtable}{l p{0.5cm} r}
دل آتش پرست من که در آتش چو گوگردی
&&
به ساقی گو که زود آخر هم از اول قدح دردی
\\
بیا ای ساقی لب گز تو خامان را بدان می‌پز
&&
زهی بستان و باغ و رز کز آن انگور افشردی
\\
نشان بدهم که کس ندهد نشان این است ای خوش قد
&&
که آن شب بردیم بیخود بدان مه روم بسپردی
\\
تو عقلا یاد می‌داری که شاه عقلم از یاری
&&
چو داد آن باده ناری به اول دم فرومردی
\\
دو طشت آورد آن دلبر یکی ز آتش یکی پرزر
&&
چو زر گیری بود آذر ور آتش برزنی بردی
\\
ببین ساقی سرکش را بکش آن آتش خوش را
&&
چه دانی قدر آتش را که آن جا کودک خردی
\\
ز آتش شاد برخیزی ز شمس الدین تبریزی
&&
ور اندر زر تو بگریزی مثال زر بیفسردی
\\
\end{longtable}
\end{center}
