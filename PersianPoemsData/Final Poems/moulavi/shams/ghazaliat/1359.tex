\begin{center}
\section*{غزل شماره ۱۳۵۹: ز خود شدم ز جمال پر از صفا ای دل}
\label{sec:1359}
\addcontentsline{toc}{section}{\nameref{sec:1359}}
\begin{longtable}{l p{0.5cm} r}
ز خود شدم ز جمال پر از صفا ای دل
&&
بگفتمش که زهی خوبی خدا ای دل
\\
غلام تست هزار آفتاب و چشم و چراغ
&&
ز پرتو تو ظلالست جان‌ها ای دل
\\
نهایتیست که خوبی از آن گذر نکند
&&
گذشت حسن تو از حد و منتها ای دل
\\
پری و دیو به پیش تو بسته‌اند کمر
&&
ملک سجود کند و اختر و سما ای دل
\\
کدام دل که بر او داغ بندگی تو نیست
&&
کدام داغ غمی کش نه‌ای دوا ای دل
\\
به حکم تست همه گنج‌های لم یزلی
&&
چه گنج‌ها که نداری تو در فنا ای دل
\\
نظر ز سوختگان وامگیر کز نظرت
&&
چه کوثرست و دوا دفع سوز را ای دل
\\
بگفتم این مه ماند به شمس تبریزی
&&
بگفت دل که کجایست تا کجا ای دل
\\
\end{longtable}
\end{center}
