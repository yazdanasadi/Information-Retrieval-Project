\begin{center}
\section*{غزل شماره ۵۹۵: آن را که درون دل عشق و طلبی باشد}
\label{sec:0595}
\addcontentsline{toc}{section}{\nameref{sec:0595}}
\begin{longtable}{l p{0.5cm} r}
آن را که درون دل عشق و طلبی باشد
&&
چون دل نگشاید در آن را سببی باشد
\\
رو بر در دل بنشین کان دلبر پنهانی
&&
وقت سحری آید یا نیم شبی باشد
\\
جانی که جدا گردد جویای خدا گردد
&&
او نادره‌ای باشد او بوالعجبی باشد
\\
آن دیده کز این ایوان ایوان دگر بیند
&&
صاحب نظری باشد شیرین لقبی باشد
\\
آن کس که چنین باشد با روح قرین باشد
&&
در ساعت جان دادن او را طربی باشد
\\
پایش چو به سنگ آید دریش به چنگ آید
&&
جانش چو به لب آید با قندلبی باشد
\\
چون تاج ملوکاتش در چشم نمی‌آید
&&
او بی‌پدر و مادر عالی نسبی باشد
\\
خاموش کن و هر جا اسرار مکن پیدا
&&
در جمع سبک روحان هم بولهبی باشد
\\
\end{longtable}
\end{center}
