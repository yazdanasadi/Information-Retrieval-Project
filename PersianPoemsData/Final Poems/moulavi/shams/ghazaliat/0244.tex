\begin{center}
\section*{غزل شماره ۲۴۴: چه شدی گر تو همچون من شدییی عاشق ای فتا}
\label{sec:0244}
\addcontentsline{toc}{section}{\nameref{sec:0244}}
\begin{longtable}{l p{0.5cm} r}
چه شدی گر تو همچون من شدییی عاشق ای فتا
&&
همه روز اندر آن جنون همه شب اندر این بکا
\\
ز دو چشمت خیال او نشدی یک دمی نهان
&&
که دو صد نور می‌رسد به دو دیده از آن لقا
\\
ز رفیقان گسستیی ز جهان دست شستیی
&&
که مجرد شدم ز خود که مسلم شدم تو را
\\
چو بر این خلق می‌تنم مثل آب و روغنم
&&
ز برونیم متصل به درونه ز هم جدا
\\
ز هوس‌ها گذشتیی به جنون بسته گشتیی
&&
نه جنونی ز خلط و خون که طبیبش دهد دوا
\\
که طبیبان اگر دمی‌بچشندی از این غمی
&&
بجهندی ز بند خود بدرندی کتاب‌ها
\\
هله زین جمله درگذر بطلب معدن شکر
&&
که شوی محو آن شکر چو لبن در زلوبیا
\\
\end{longtable}
\end{center}
