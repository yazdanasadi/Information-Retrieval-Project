\begin{center}
\section*{غزل شماره ۱۱۵۷: گر تو خواهی وطن پر از دلدار}
\label{sec:1157}
\addcontentsline{toc}{section}{\nameref{sec:1157}}
\begin{longtable}{l p{0.5cm} r}
گر تو خواهی وطن پر از دلدار
&&
خانه را رو تهی کن از اغیار
\\
ور تو خواهی سماع را گیرا
&&
دور دارش ز دیده انکار
\\
هر که او را سماع مست نکرد
&&
منکرش دان اگر چه کرد اقرار
\\
هر که اقرار کرد و باده شناخت
&&
عاقلش نام نه مگو خمار
\\
به بهانه به ره کن آن‌ها را
&&
تا شوی از سماع برخوردار
\\
وز میان خویش را برون کن تیز
&&
تا بگیری تو خویش را به کنار
\\
سایه یار به که ذکر خدای
&&
این چنین گفتست صدر کبار
\\
تا نگویی که گل هم از خارست
&&
زانک هر خار گل نیارد بار
\\
خار بیگانه را ز دل برکن
&&
خار گل را به جان و دل می‌دار
\\
موسی اندر درخت آتش دید
&&
سبزتر می‌شد آن درخت از نار
\\
شهوت و حرص مرد صاحب دل
&&
همچنین دان و همچنین پندار
\\
صورت شهوتست لیکن هست
&&
همچو نار خلیل پرانوار
\\
شمس تبریز را بشر بینند
&&
چون گشایند دیده‌ها کفار
\\
\end{longtable}
\end{center}
