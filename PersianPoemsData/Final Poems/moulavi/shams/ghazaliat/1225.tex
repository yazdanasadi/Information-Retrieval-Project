\begin{center}
\section*{غزل شماره ۱۲۲۵: ریاضت نیست پیش ما همه لطفست و بخشایش}
\label{sec:1225}
\addcontentsline{toc}{section}{\nameref{sec:1225}}
\begin{longtable}{l p{0.5cm} r}
ریاضت نیست پیش ما همه لطفست و بخشایش
&&
همه مهرست و دلداری همه عیش است و آسایش
\\
هر آنچ از فقر کار آید به باغ جان به بار آید
&&
به ما از شهریار آید و باقی جمله آرایش
\\
همه دیدست در راهش همه صدرست درگاهش
&&
وگر تن هست در کاهش ببین جان را تو افزایش
\\
ببین تو لطف پاکی را امیر سهمناکی را
&&
که او یک مشت خاکی را کند در لامکان جایش
\\
بسی کوران و ره شینان از او گشتند ره بینان
&&
بسی جان‌های غمگینان چو طوطی شد شکرخایش
\\
بسی زخمست بی‌دشنه ز پنج و چار وز شش نه
&&
ز عشق آتش تشنه که جز خون نیست سقایش
\\
زهی شیرین که می‌سوزم چو از شمعش برافروزم
&&
زهی شادی امروزم ز دولت‌های فردایش
\\
چرا من خاکی و پستم ازیرا عاشق و مستم
&&
چرا من جمله جانستم ز عشق جسم فرسایش
\\
به پیش عاشقان صف صف برآورده به حاجب کف
&&
ز زخم اوست دل چون دف دهان از ناله سرنایش
\\
از او چونست این دل چون کز او غرقست ره ره خون
&&
وز او غوغاست در گردون و ناله جان ز هیهایش
\\
دلا تا چند پرهیزی بگو تو شمس تبریزی
&&
بنه سر تو ز سرتیزی برای فخر بر پایش
\\
\end{longtable}
\end{center}
