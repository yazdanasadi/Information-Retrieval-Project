\begin{center}
\section*{غزل شماره ۲۹۰۰: بوی مشکی در جهان افکنده‌ای}
\label{sec:2900}
\addcontentsline{toc}{section}{\nameref{sec:2900}}
\begin{longtable}{l p{0.5cm} r}
بوی مشکی در جهان افکنده‌ای
&&
مشک را در لامکان افکنده‌ای
\\
صد هزاران غلغله زین بوی مشک
&&
در زمین و آسمان افکنده‌ای
\\
از شعاع نور و نار خویشتن
&&
آتشی در عقل و جان افکنده‌ای
\\
از کمال لعل جان افزای خویش
&&
شورشی در بحر و کان افکنده‌ای
\\
تو نهادی قاعده عاشق کشی
&&
در دل عاشق کشان افکنده‌ای
\\
صد هزاران روح رومی روی را
&&
در میان زنگیان افکنده‌ای
\\
با یقین پاکشان بسرشته‌ای
&&
چونشان اندر گمان افکنده‌ای
\\
چون به دست خویششان کردی خمیر
&&
چونشان در قید نان افکنده‌ای
\\
هم شکار و هم شکاری گیر را
&&
زیر این دام گران افکنده‌ای
\\
پردلان را همچو دل بشکسته‌ای
&&
بی دلان را در فغان افکنده‌ای
\\
جان سلطان زادگان را بنده وار
&&
پیش عقل پاسبان افکنده‌ای
\\
\end{longtable}
\end{center}
