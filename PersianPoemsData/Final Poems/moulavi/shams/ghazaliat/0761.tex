\begin{center}
\section*{غزل شماره ۷۶۱: چو سحرگاه ز گلشن مه عیار برآمد}
\label{sec:0761}
\addcontentsline{toc}{section}{\nameref{sec:0761}}
\begin{longtable}{l p{0.5cm} r}
چو سحرگاه ز گلشن مه عیار برآمد
&&
چه بسی نعره مستان که ز گلزار برآمد
\\
ز رخ ماه خصالش ز لطیفی وصالش
&&
همه را بخت فزون شد همه را کار برآمد
\\
ز دو صد روضه رضوان ز دو صد چشمه حیوان
&&
دو هزاران گل خندان ز دل خار برآمد
\\
غم چون دزد که در دل همه شب دارد منزل
&&
به کف شحنه وصلش به سر دار برآمد
\\
ز پس ظلم رسیده همه امید بریده
&&
مثل دولت تابان دل بیدار برآمد
\\
تن و جان از پس پیری ز وصالش چه جوان شد
&&
همه را بعد کسادی چه خریدار برآمد
\\
چو صلاح دل و دین را همه دیدیت بگویید
&&
که چه خورشید عجایب که ز اسرار برآمد
\\
\end{longtable}
\end{center}
