\begin{center}
\section*{غزل شماره ۱۲۶۸: صد سال اگر گریزی و نایی بتا به پیش}
\label{sec:1268}
\addcontentsline{toc}{section}{\nameref{sec:1268}}
\begin{longtable}{l p{0.5cm} r}
صد سال اگر گریزی و نایی بتا به پیش
&&
برهم زنیم کار تو را همچو کار خویش
\\
مگریز که ز چنبر چرخت گذشتنیست
&&
گر شیر شرزه باشی ور سفله گاومیش
\\
تن دنبلیست بر کتف جان برآمده
&&
چون پر شود تهی شود آخر ز زخم نیش
\\
ای شاد باطلی که گریزد ز باطلی
&&
بر عشق حق بچفسد بی‌صمغ و بی‌سریش
\\
گز می‌کنند جامه عمرت به روز و شب
&&
هم آخر آرد او را یا روز یا شبیش
\\
بیچاره آدمی که زبونست عشق را
&&
زفت آمد این سوار بر این اسب پشت ریش
\\
خاموش باش و در خمشی گم شو از وجود
&&
کان عشق راست کشتن عشاق دین و کیش
\\
\end{longtable}
\end{center}
