\begin{center}
\section*{غزل شماره ۳۱۶۹: گر نه شکار غم دلدارمی}
\label{sec:3169}
\addcontentsline{toc}{section}{\nameref{sec:3169}}
\begin{longtable}{l p{0.5cm} r}
گر نه شکار غم دلدارمی
&&
گردن شیر فلک افشارمی
\\
دست مرا بست، وگر نی کنون
&&
من سر تو بهتر ازین خارمی
\\
گر نبدی رشک رخ چون گلشن
&&
بلبل هر گلشن و گلزارمی
\\
گر گل او در نگشادی، چرا
&&
خار صفت بر سر دیوارمی؟
\\
نیست یکی کار که او آن نکرد
&&
ورنه چرا کاهل و بی‌کارمی؟
\\
عشق طبیبست که رنجور جوست
&&
ورنه چرا خسته و بیمارمی؟
\\
کشت خلیل از پی او چار مرغ
&&
کاش به قربانیش آن چارمی
\\
تا پی خوردن به شکر خوردنش
&&
طوطی با صد سر و منقارمی
\\
وز جهت قوت دگر طوطیان
&&
چون لب او جمله شکر کارمی
\\
گر نه دلی داد چو دریا مرا
&&
چون دگران تند و جگر خوارمی
\\
در سر من عشق بپیچید سخت
&&
ورنه چرا بی‌دل و دستارمی؟
\\
بر لب من دوش ببوسید یار
&&
ورنه چرا با مزه گفتارمی؟
\\
بر خط من نقطهٔ دولت نهاد
&&
ورنه چه گردنده چو پرگارمی؟
\\
گر نه‌امی پست، که دیدی مرا؟!
&&
ورنه امی مست بهنجارمی
\\
چونک ز مستی کژ و مژ می‌روم
&&
کاش که من بر ره هموارمی
\\
یا مثل لاله رخان خوشش
&&
معتزلی بر سر کهسارمی
\\
بس! که گرین بانگ دهل نیستی
&&
همچو خیالات در اسرارمی
\\
\end{longtable}
\end{center}
