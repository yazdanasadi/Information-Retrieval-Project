\begin{center}
\section*{غزل شماره ۴۷: ای که تو ماه آسمان ماه کجا و تو کجا}
\label{sec:0047}
\addcontentsline{toc}{section}{\nameref{sec:0047}}
\begin{longtable}{l p{0.5cm} r}
ای که تو ماه آسمان ماه کجا و تو کجا
&&
در رخ مه کجا بود این کر و فر و کبریا
\\
جمله به ماه عاشق و ماه اسیر عشق تو
&&
ناله کنان ز درد تو لابه کنان که ای خدا
\\
سجده کنند مهر و مه پیش رخ چو آتشت
&&
چونک کند جمال تو با مه و مهر ماجرا
\\
آمد دوش مه که تا سجده برد به پیش تو
&&
غیرت عاشقان تو نعره زنان که رو میا
\\
خوش بخرام بر زمین تا شکفند جان‌ها
&&
تا که ملک فروکند سر ز دریچه سما
\\
چونک شوی ز روی تو برق جهنده هر دلی
&&
دست به چشم برنهد از پی حفظ دیده‌ها
\\
هر چه بیافت باغ دل از طرب و شکفتگی
&&
از دی این فراق شد حاصل او همه هبا
\\
زرد شدست باغ جان از غم هجر چون خزان
&&
کی برسد بهار تو تا بنماییش نما
\\
بر سر کوی تو دلم زار نزار خفت دی
&&
کرد خیال تو گذر دید بدان صفت ورا
\\
گفت چگونه‌ای از این عارضه گران بگو
&&
کز تنکی ز دیده‌ها رفت تن تو در خفا
\\
گفت و گذشت او ز من لیک ز ذوق آن سخن
&&
صحت یافت این دلم یا رب تش دهی جزا
\\
\end{longtable}
\end{center}
