\begin{center}
\section*{غزل شماره ۲۳۶۲: ای دو چشمت جاودان را نکته‌ها آموخته}
\label{sec:2362}
\addcontentsline{toc}{section}{\nameref{sec:2362}}
\begin{longtable}{l p{0.5cm} r}
ای دو چشمت جاودان را نکته‌ها آموخته
&&
جان‌ها را شیوه‌های جان فزا آموخته
\\
هر چه در عالم دری بسته‌ست مفتاحش تویی
&&
عشق شاگرد تو است و درگشا آموخته
\\
از برای صوفیان صاف بزم آراسته
&&
وانگهانی صوفیان را الصلا آموخته
\\
وز میان صوفیان آن صوفی محبوب را
&&
سر معشوقی مطلق در خلاء آموخته
\\
و آن دگر را ز امتحان اندر فراق انداخته
&&
سر سر عاشقانش در بلا آموخته
\\
عشق را نیمی نیاز و نیم دیگر بی‌نیاز
&&
این اجابت یافته و آن خود دعا آموخته
\\
پیش آب لطف او بین آتشی زانو زده
&&
همچو افلاطون حکمت صد دوا آموخته
\\
با دعا و با اجابت نقب کرده نیم شب
&&
سوی عیاران رند و صد دغا آموخته
\\
پرجفایانی که ایشان با همه کافردلی
&&
مر وفا را گوش مالیده وفا آموخته
\\
زخم و آتش‌های پنهانی است اندر چشمشان
&&
کهنان را همچو آیینه صفا آموخته
\\
جمله ایشان بندگان شمس تبریزی شده
&&
در تجلی‌های او نور لقا آموخته
\\
\end{longtable}
\end{center}
