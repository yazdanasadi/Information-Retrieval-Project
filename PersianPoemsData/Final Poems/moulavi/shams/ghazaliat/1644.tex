\begin{center}
\section*{غزل شماره ۱۶۴۴: جز ز فتان دو چشمت ز کی مفتون باشیم}
\label{sec:1644}
\addcontentsline{toc}{section}{\nameref{sec:1644}}
\begin{longtable}{l p{0.5cm} r}
جز ز فتان دو چشمت ز کی مفتون باشیم
&&
جز ز زنجیر دو زلفت ز کی مجنون باشیم
\\
جز از آن روی چو ماهت که مهش جویان است
&&
دگر از بهر که سرگشته چو گردون باشیم
\\
نار خندان تو ما را صنما گریان کرد
&&
تا چو نار از غم تو با دل پرخون باشیم
\\
چشم مست تو قدح بر سر ما می ریزد
&&
ما چه موقوف شراب و می و افیون باشیم
\\
گلفشان رخ تو خرمن گل می بخشد
&&
ما چه موقوف بهار و گل گلگون باشیم
\\
همچو موسی ز درخت تو حریف نوریم
&&
ما چرا عاشق برگ و زر قارون باشیم
\\
هر زمان عشق درآید که حریفان چونید
&&
ما ز چون گفتن او واله و بی‌چون باشیم
\\
ما چو زاییده و پرورده آن دریاییم
&&
صاف و تابنده و خوش چون در مکنون باشیم
\\
ما ز نور رخ خورشید چو اجرا داریم
&&
همچو مه تیزرو و چابک و موزون باشیم
\\
به دعا نوح خیالت یم و جیحون خواهد
&&
بهر این سابح و با چشم چو جیحون باشیم
\\
همچو عشقیم درون دل هر سودایی
&&
لیک چون عشق ز وهم همه بیرون باشیم
\\
چونک در مطبخ دل لوت طبق بر طبق است
&&
ما چرا کاسه کش مطبخ هر دون باشیم
\\
وقف کردیم بر این باده جان کاسه سر
&&
تا حریف سری و شبلی و ذاالنون باشیم
\\
شمس تبریز پی نور تو زان ذره شدیم
&&
تا ز ذرات جهان در عدد افزون باشیم
\\
\end{longtable}
\end{center}
