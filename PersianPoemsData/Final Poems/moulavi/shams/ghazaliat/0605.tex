\begin{center}
\section*{غزل شماره ۶۰۵: ای دوست شکر خوشتر یا آنک شکر سازد}
\label{sec:0605}
\addcontentsline{toc}{section}{\nameref{sec:0605}}
\begin{longtable}{l p{0.5cm} r}
ای دوست شکر خوشتر یا آنک شکر سازد
&&
ای دوست قمر خوشتر یا آنک قمر سازد
\\
بگذار شکرها را بگذار قمرها را
&&
او چیز دگر داند او چیز دگر سازد
\\
در بحر عجایب‌ها باشد به جز از گوهر
&&
اما نه چو سلطانی کو بحر و درر سازد
\\
جز آب دگر آبی از نادره دولابی
&&
بی شبهه و بی‌خوابی او قوت جگر سازد
\\
بی عقل نتان کردن یک صورت گرمابه
&&
چون باشد آن علمی کو عقل و خبر سازد
\\
بی علم نمی‌تانی کز پیه کشی روغن
&&
بنگر تو در آن علمی کز پیه نظر سازد
\\
جان‌ها است برآشفته ناخورده و ناخفته
&&
از بهر عجب بزمی کو وقت سحر سازد
\\
ای شاد سحرگاهی کان حسرت هر ماهی
&&
بر گرد میان من دو دست کمر سازد
\\
می‌خندد این گردون بر سبلت آن مفتون
&&
خود را پی دو سه خر آن مسخره خر سازد
\\
آن خر به مثال جو در زر فکند خود را
&&
غافل بود از شاهی کز سنگ گهر سازد
\\
بس کردم و بس کردم من ترک نفس کردم
&&
خود گوید جانانی کز گوش بصر سازد
\\
\end{longtable}
\end{center}
