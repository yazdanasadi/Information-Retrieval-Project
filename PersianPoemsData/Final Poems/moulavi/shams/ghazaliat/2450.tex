\begin{center}
\section*{غزل شماره ۲۴۵۰: در دل خیالش زان بود تا تو به هر سو ننگری}
\label{sec:2450}
\addcontentsline{toc}{section}{\nameref{sec:2450}}
\begin{longtable}{l p{0.5cm} r}
در دل خیالش زان بود تا تو به هر سو ننگری
&&
و آن لطف بی‌حد زان کند تا هیچ از حد نگذری
\\
با صوفیان صاف دین در وجد گردی همنشین
&&
گر پای در بیرون نهی زین خانقاه شش دری
\\
داری دری پنهان صفت شش در مجو و شش جهت
&&
پنهان دری که هر شبی زان در همی‌بیرون پری
\\
چون می‌پری بر پای تو رشته خیالی بسته‌اند
&&
تا واکشندت صبحدم تا برنپری یک سری
\\
بازآ به زندان رحم تا خلقتت کامل شدن
&&
هست این جهان همچون رحم این جمله خون زان می‌خوری
\\
جان را چو بررویید پر شد بیضه تن را شکست
&&
جان جعفر طیار شد تا می‌نماید جعفری
\\
\end{longtable}
\end{center}
