\begin{center}
\section*{غزل شماره ۱۶۵۰: خوش بنوشم تو اگر زهر نهی در جامم}
\label{sec:1650}
\addcontentsline{toc}{section}{\nameref{sec:1650}}
\begin{longtable}{l p{0.5cm} r}
خوش بنوشم تو اگر زهر نهی در جامم
&&
پخته و خام تو را گر نپذیرم خامم
\\
عاشق هدیه نیم عاشق آن دست توام
&&
سنقر دانه نیم ایبک بند دامم
\\
از تغار تو اگر خون رسدم همچو سگان
&&
گر من آن را قدح خاص ندانم عامم
\\
غنچه و خار تو را دایه شوم همچو زمین
&&
تا سمعنا و اطعنا کنی ای جان نامم
\\
ملخ حکم تو تا مزرعه‌ام را بچرید
&&
گر نگردم تلف تو علف ایامم
\\
ساقی صبر بیا رطل گرانم درده
&&
تا چو ریگش به یکی بار فروآشامم
\\
گوییم شپشپی و چون پشه بی‌آرامی
&&
چون دلارام نیابم به چه چیز آرامم
\\
همچو دزدان ز عسس من همه شب در بیمم
&&
همچو خورشیدپرستان به سحر بر بامم
\\
مهر غیر تو بود در دل من مهر ضلال
&&
شکر غیر تو بود در سر من سرسامم
\\
به زبان گر نکنم یاد شکرخانه تو
&&
کام و ناکام بود لذت آن در کامم
\\
خبر رشک تو می آرد اشک تر من
&&
نه به تقلید بل از دیده دهد پیغامم
\\
\end{longtable}
\end{center}
