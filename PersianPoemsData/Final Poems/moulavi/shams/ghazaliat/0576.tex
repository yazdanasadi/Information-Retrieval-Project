\begin{center}
\section*{غزل شماره ۵۷۶: دل من چون صدف باشد خیال دوست در باشد}
\label{sec:0576}
\addcontentsline{toc}{section}{\nameref{sec:0576}}
\begin{longtable}{l p{0.5cm} r}
دل من چون صدف باشد خیال دوست در باشد
&&
کنون من هم نمی‌گنجم کز او این خانه پر باشد
\\
ز شیرینی حدیثش شب شکافیدست جان را لب
&&
عجب دارم که می‌گوید حدیث حق مر باشد
\\
غذاها از برون آید غذای عاشق از باطن
&&
برآرد از خود و خاید که عاق چون شتر باشد
\\
سبک رو همچو پریان شو ز جسم خویش عریان شو
&&
مسلم نیست عریانی مر آن کس را که عر باشد
\\
صلاح الدین به صید آمد همه شیران بود صیدش
&&
غلام او کسی باشد که از دو کون حر باشد
\\
\end{longtable}
\end{center}
