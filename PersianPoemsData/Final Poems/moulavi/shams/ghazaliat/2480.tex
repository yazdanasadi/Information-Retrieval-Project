\begin{center}
\section*{غزل شماره ۲۴۸۰: ای دل بی‌قرار من راست بگو چه گوهری}
\label{sec:2480}
\addcontentsline{toc}{section}{\nameref{sec:2480}}
\begin{longtable}{l p{0.5cm} r}
ای دل بی‌قرار من راست بگو چه گوهری
&&
آتشیی تو آبیی آدمیی تو یا پری
\\
از چه طرف رسیده‌ای وز چه غذا چریده‌ای
&&
سوی فنا چه دیده‌ای سوی فنا چه می‌پری
\\
بیخ مرا چه می‌کنی قصد فنا چه می‌کنی
&&
راه خرد چه می‌زنی پرده خود چه می‌دری
\\
هر حیوان و جانور از عدمند بر حذر
&&
جز تو که رخت خویش را سوی عدم همی‌بری
\\
گرم و شتاب می‌روی مست و خراب می‌روی
&&
گوش به پند کی نهی عشوه خلق کی خوری
\\
از سر کوه این جهان سیل تویی روان روان
&&
جانب بحر لامکان از دم من روانتری
\\
باغ و بهار خیره سر کز چه نسیم می‌وزی
&&
سوسن و سرو مست تو تا چه گلی چه عبهری
\\
بانک دفی که صنج او نیست حریف چنبرش
&&
درنرود به گوش ما چون هذیان کافری
\\
موسی عشق تو مرا گفت که لامساس شو
&&
چون نگریزم از همه چون نرمم ز سامری
\\
از همه من گریختم گر چه میان مردمم
&&
چون به میان خاک کان نقده زر جعفری
\\
گر دو هزار بار زر نعره زند که من زرم
&&
تا نرود ز کان برون نیست کسیش مشتری
\\
\end{longtable}
\end{center}
