\begin{center}
\section*{غزل شماره ۲۸۶۲: هله هشدار که با بی‌خبران نستیزی}
\label{sec:2862}
\addcontentsline{toc}{section}{\nameref{sec:2862}}
\begin{longtable}{l p{0.5cm} r}
هله هشدار که با بی‌خبران نستیزی
&&
پیش مستان چنان رطل گران نستیزی
\\
گر نخواهی که کمان وار ابد کژ مانی
&&
چون کشندت سوی خود همچو کمان نستیزی
\\
گر نخواهی که تو را گرگ هوا بردرد
&&
چون تو را خواند سوی خویش شبان نستیزی
\\
عجمی وار نگویی تو شهان را که کیید
&&
چون نمایند تو را نقش و نشان نستیزی
\\
از میان دل و جان تو چو سر برکردند
&&
جان به شکرانه نهی تو به میان نستیزی
\\
چو به ظاهر تو سمعنا و اطعنا گفتی
&&
ظاهر آنگه شود این که به نهان نستیزی
\\
در گمانی ز معاد خود و از مبدا خود
&&
شودت عین چو با اهل عیان نستیزی
\\
در تجلی بنماید دو جهان چون ذرات
&&
گر شوی ذره و چون کوه گران نستیزی
\\
ز زمان و ز مکان بازرهی گر تو ز خود
&&
چو زمان برگذری و چو مکان نستیزی
\\
مثل چرخ تو در گردش و در کار آیی
&&
گر چو دولاب تو با آب روان نستیزی
\\
چون جهان زهره ندارد که ستیزد با شاه
&&
الله الله که تو با شاه جهان نستیزی
\\
هم به بغداد رسی روی خلیفه بینی
&&
گر کنی عزم سفر در همدان نستیزی
\\
حیله و زوبعی و شیوه و روبه بازی
&&
راست آید چو تو با شیر ژیان نستیزی
\\
همچو آیینه شوی خامش و گویا تو اگر
&&
همه دل گردی و بر گفت زبان نستیزی
\\
\end{longtable}
\end{center}
