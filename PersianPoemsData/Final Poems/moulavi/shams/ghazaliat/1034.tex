\begin{center}
\section*{غزل شماره ۱۰۳۴: مکن یار مکن یار مرو ای مه عیار}
\label{sec:1034}
\addcontentsline{toc}{section}{\nameref{sec:1034}}
\begin{longtable}{l p{0.5cm} r}
مکن یار مکن یار مرو ای مه عیار
&&
رخ فرخ خود را مپوشان به یکی بار
\\
تو دریای الهی همه خلق چو ماهی
&&
چو خشک آوری ای دوست بمیرند به ناچار
\\
مگو با دل شیدا دگر وعده فردا
&&
که بر چرخ رسیدست ز فردای تو زنهار
\\
چو در دست تو باشیم ندانیم سر از پای
&&
چو سرمست تو باشیم بیفتد سر و دستار
\\
عطاهای تو نقدست شکایت نتوان کرد
&&
ولیکن گله کردیم برای دل اغیار
\\
مرا عشق بپرسید که ای خواجه چه خواهی
&&
چه خواهد سر مخمور به غیر در خمار
\\
سراسر همه عیبیم بدیدی و خریدی
&&
زهی کاله پرعیب زهی لطف خریدار
\\
ملوکان همه زربخش تویی خسرو سربخش
&&
سر از گور برآورد ز تو مرده پیرار
\\
ملالت نفزایید دلم را هوس دوست
&&
اگر رهزندم جان ز جان گردم بیزار
\\
چو ابر تو ببارید بروید سمن از ریگ
&&
چو خورشید تو درتافت بروید گل و گلزار
\\
ز سودای خیال تو شدستیم خیالی
&&
کی داند چه شویم از تو چو باشد گه دیدار
\\
همه شیشه شکستیم کف پای بخستیم
&&
حریفان همه مستیم مزن جز ره هموار
\\
\end{longtable}
\end{center}
