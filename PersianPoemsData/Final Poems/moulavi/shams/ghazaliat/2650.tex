\begin{center}
\section*{غزل شماره ۲۶۵۰: تو آن ماهی که در گردون نگنجی}
\label{sec:2650}
\addcontentsline{toc}{section}{\nameref{sec:2650}}
\begin{longtable}{l p{0.5cm} r}
تو آن ماهی که در گردون نگنجی
&&
تو آن آبی که در جیحون نگنجی
\\
تو آن دری که از دریا فزونی
&&
تو آن کوهی که در هامون نگنجی
\\
چه خوانم من فسون ای شاه پریان
&&
که تو در شیشه و افسون نگنجی
\\
تو لیلیی ولیک از رشک مولی
&&
به کنج خاطر مجنون نگنجی
\\
تو خورشیدی قبایت نور سینه است
&&
تو اندر اطلس و اکسون نگنجی
\\
تویی شاگرد جان افزا طبیبی
&&
در استدلال افلاطون نگنجی
\\
تو معجونی که نبود در ذخیره
&&
ذخیره چیست در قانون نگنجی
\\
بگوید خصم تا خود چون بود این
&&
تو از بی‌چونی و در چون نگنجی
\\
چنین بودی در اشکمگاه دنیا
&&
بگنجیدی ولی اکنون نگنجی
\\
مخوان در گوش‌ها این را خمش کن
&&
تو اندر گوش هر مفتون نگنجی
\\
\end{longtable}
\end{center}
