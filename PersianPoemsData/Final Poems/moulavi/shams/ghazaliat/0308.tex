\begin{center}
\section*{غزل شماره ۳۰۸: خوابم ببسته‌ای بگشا ای قمر نقاب}
\label{sec:0308}
\addcontentsline{toc}{section}{\nameref{sec:0308}}
\begin{longtable}{l p{0.5cm} r}
خوابم ببسته‌ای بگشا ای قمر نقاب
&&
تا سجده‌های شکر کند پیشت آفتاب
\\
دامان تو گرفتم و دستم بتافتی
&&
هین دست درکشیدم روی از وفا متاب
\\
گفتی مکن شتاب که آن هست فعل دیو
&&
دیو او بود که می‌نکند سوی تو شتاب
\\
یا رب کنم ببینم بر درگه نیاز
&&
چندین هزار یا رب مشتاق آن جواب
\\
از خاک بیشتر دل و جان‌های آتشین
&&
مستسقیانه کوزه گرفته که آب آب
\\
بر خاک رحم کن که از این چار عنصر او
&&
بی دست و پاتر آمد در سیر و انقلاب
\\
وقتی که او سبک شود آن باد پای اوست
&&
لنگانه برجهد دو سه گامی پی سحاب
\\
تا خنده گیرد از تک آن لنگ برق را
&&
و اندر شفاعت آید آن رعد خوش خطاب
\\
با ساقیان ابر بگوید که برجهید
&&
کز تشنگان خاک بجوشید اضطراب
\\
گیرم که من نگویم آخر نمی‌رسد
&&
اندر مشام رحمت بوی دل کباب
\\
پس ساقیان ابر همان دم روان شوند
&&
با جره و قنینه و با مشک پرشراب
\\
خاموش و در خراب همی‌جوی گنج عشق
&&
کاین گنج در بهار برویید از خراب
\\
\end{longtable}
\end{center}
