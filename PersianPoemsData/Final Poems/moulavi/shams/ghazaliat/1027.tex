\begin{center}
\section*{غزل شماره ۱۰۲۷: یغمابک ترکستان بر زنگ بزد لشکر}
\label{sec:1027}
\addcontentsline{toc}{section}{\nameref{sec:1027}}
\begin{longtable}{l p{0.5cm} r}
یغمابک ترکستان بر زنگ بزد لشکر
&&
در قلعه بی‌خویشی بگریز هلا زوتر
\\
تا کی ز شب زنگی بر عقل بود تنگی
&&
شاهنشه صبح آمد زد بر سر او خنجر
\\
گاو سیه شب را قربان سحر کردند
&&
مؤذن پی این گوید کالله هو الاکبر
\\
آورد برون گردون از زیر لگن شمعی
&&
کز خجلت نور او بر چرخ نماند اختر
\\
خورشید گر از اول بیمارصفت باشد
&&
هم از دل خود گردد در هر نفسی خوشتر
\\
ای چشم که پردردی در سایه او بنشین
&&
زنهار در این حالت در چهره او بنگر
\\
آن واعظ روشن دل کو ذره به رقص آرد
&&
بس نور که بفشاند او از سر این منبر
\\
شاباش زهی نوری بر کوری هر کوری
&&
زان پس که بر آرد سر کور وی نپوشاند
\\
شمس الحق تبریزی در آینه صافت
&&
گر غیر خدا بینم باشم بتر از کافر
\\
\end{longtable}
\end{center}
