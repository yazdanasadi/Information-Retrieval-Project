\begin{center}
\section*{غزل شماره ۲۰۰۶: هر کجا که پا نهی ای جان من}
\label{sec:2006}
\addcontentsline{toc}{section}{\nameref{sec:2006}}
\begin{longtable}{l p{0.5cm} r}
هر کجا که پا نهی ای جان من
&&
بردمد لاله و بنفشه و یاسمن
\\
پاره گل برکنی بر وی دمی
&&
بازگردد یا کبوتر یا زغن
\\
در تغاری دست شویی آن تغار
&&
ز آب دست تو شود زرین لگن
\\
بر سر گوری بخوانی فاتحه
&&
بوالفتوحی سر برآرد از کفن
\\
دامنت بر چنگل خاری زند
&&
چنگلش چنگی شود با تن تنن
\\
هر بتی را که شکستی ای خلیل
&&
جان پذیرد عقل یابد زان شکن
\\
تا مه تو تافت بر بداختری
&&
سعد اکبر گشت و وارست از محن
\\
هر دمی از صحن سینه برجهد
&&
همچو آدم زاده‌ای بی‌مرد و زن
\\
وآنگه از پهلوی او وز پشت او
&&
پر شوند آدمچگان اندر زمن
\\
خواستم گفتن بر این پنجاه بیت
&&
لب ببستم تا گشایی تو دهن
\\
\end{longtable}
\end{center}
