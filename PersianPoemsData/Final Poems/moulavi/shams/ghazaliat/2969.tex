\begin{center}
\section*{غزل شماره ۲۹۶۹: بوی کباب داری تو نیز دل کبابی}
\label{sec:2969}
\addcontentsline{toc}{section}{\nameref{sec:2969}}
\begin{longtable}{l p{0.5cm} r}
بوی کباب داری تو نیز دل کبابی
&&
در تو هر آنچ گم شد در ماش بازیابی
\\
زین سر چو زنده باشی تو سرفکنده باشی
&&
خود را چو بنده باشی ما را دگر نیابی
\\
ای خواجه ترک ره کن ما را حدیث شه کن
&&
بگشا دهان و اه کن گر مست آن شرابی
\\
دوشم نگار دلبر می‌داد جام از زر
&&
گفتا بکش تو دیگر گر مست نیم خوابی
\\
گفتم که برنخیزم گفتا که برستیزم
&&
هم بر سرت بریزم گر مستی و خرابی
\\
چون ریخت بر من آن را دیدم فنا جهان را
&&
عالم چو بحر جوشان من گشته مرغ آبی
\\
ای خواجه خشم بنشان سر را دگر مپیچان
&&
ما را چه جرم باشد گر ز آنک درنیابی
\\
سر اله گفتم در قعر چاه گفتم
&&
مه را سیاه گفتم چون محرم نقابی
\\
ای خواجه صدر عالی تا تو در این حوالی
&&
گه بسته سؤالی گه خسته جوابی
\\
ای شمس حق تبریز بستم دهان ازیرا
&&
هر دیده برنتابد نورت چو آفتابی
\\
\end{longtable}
\end{center}
