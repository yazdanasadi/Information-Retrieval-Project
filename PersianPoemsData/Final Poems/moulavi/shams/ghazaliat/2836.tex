\begin{center}
\section*{غزل شماره ۲۸۳۶: به چه روی پشت آرم به کسی که از گزینی}
\label{sec:2836}
\addcontentsline{toc}{section}{\nameref{sec:2836}}
\begin{longtable}{l p{0.5cm} r}
به چه روی پشت آرم به کسی که از گزینی
&&
سوی او کند خدا رو به حدیث و همنشینی
\\
نه که روی و پشت عالم همه رو به قبله دارد
&&
که ز کیمیاست مس را برهیدن از مسینی
\\
همگان ز خود گریزان سوی حق و نعل ریزان
&&
که ز کاسدی رسانمان به لطافت و ثمینی
\\
نه زمین ستان بخفته ز رخ فلک شکفته
&&
ز فلک نبات یابد برهد از این زمینی
\\
دهد آن حبوب علوی به زمین خوشی و حلوی
&&
به بهار امانتی‌ها بنماید از امینی
\\
هله ای حیات حسی بگریز هم ز مسی
&&
سوی آسمان قدسی که تو عاشق مهینی
\\
ز برای دعوت جان برسیده‌اند خوبان
&&
که بیا به معدن و کان بهل این قراضه چینی
\\
به خدا که ماه رویی به خدا فرشته خویی
&&
به خدا که مشک بویی به خدا که این چنینی
\\
تو که یوسف زمانی چه میان هندوانی
&&
برو آینه طلب کن بنگر که روی بینی
\\
به صفا چو آسمانی به ملاطفت چو جانی
&&
به شکفتگی چنانی به نهفتگی چنینی
\\
به خزینه خوب رختی ز قدیم نیکبختی
&&
به نبات چون درختی به ثبات چون یقینی
\\
شده‌ام چو موم ای جان به هوای مهر سلطان
&&
برسان به موم مهرش که گزیده‌تر نگینی
\\
هله بس که کاسه‌ها را به طعام او است قیمت
&&
و اگر نه خاک نه ارزد همه کاسه‌های چینی
\\
\end{longtable}
\end{center}
