\begin{center}
\section*{غزل شماره ۵۰۴: خیز که امروز جهان آن ماست}
\label{sec:0504}
\addcontentsline{toc}{section}{\nameref{sec:0504}}
\begin{longtable}{l p{0.5cm} r}
خیز که امروز جهان آن ماست
&&
جان و جهان ساقی و مهمان ماست
\\
در دل و در دیده دیو و پری
&&
دبدبه فر سلیمان ماست
\\
رستم دستان و هزاران چو او
&&
بنده و بازیچه دستان ماست
\\
بس نبود مصر مرا این شرف
&&
این که شهش یوسف کنعان ماست
\\
خیز که فرمان ده جان و جهان
&&
از کرم امروز به فرمان ماست
\\
زهره و مه دف زن شادی ماست
&&
بلبل جان مست گلستان ماست
\\
کاسه ارزاق پیاپی شده‌ست
&&
کیسه اقبال حرمدان ماست
\\
شاه شهی بخش طرب ساز ماست
&&
یار پری روی پری خوان ماست
\\
آن ملک مفخر چوگان و گوی
&&
شکر که امروز به میدان ماست
\\
آن ملک مملکت جان و دل
&&
در دل و در جان پریشان ماست
\\
کیست در آن گوشه دل تن زده
&&
پیش کشش کو شکرستان ماست
\\
خازن رضوان که مه جنت‌ست
&&
مست رضای دل رضوان ماست
\\
شور درافکنده و پنهان شده
&&
او نمک عمر و نمکدان ماست
\\
گوشه گرفتست و جهان مست اوست
&&
او خضر و چشمه حیوان ماست
\\
چون نمک دیگ و چو جان در بدن
&&
از همه ظاهرتر و پنهان ماست
\\
نیست نماینده و خود جمله اوست
&&
خود همه ماییم چو او آن ماست
\\
بیش مگو حجت و برهان که عشق
&&
در خمشی حجت و برهان ماست
\\
\end{longtable}
\end{center}
