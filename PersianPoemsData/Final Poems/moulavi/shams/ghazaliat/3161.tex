\begin{center}
\section*{غزل شماره ۳۱۶۱: خامشی ناطقی مگر جانی}
\label{sec:3161}
\addcontentsline{toc}{section}{\nameref{sec:3161}}
\begin{longtable}{l p{0.5cm} r}
خامشی ناطقی مگر جانی
&&
می‌زنی نعره‌های پنهانی
\\
تو چو باغی و صورتت برگی
&&
باغ چه صد هزار چندانی
\\
بی تو باغ حیات زندانیست
&&
هست مردن خلاص زندانی
\\
چون تو بحری و صورتت ابرست
&&
فیض دل قطره‌های مرجانی
\\
ای یکی گو شده یکی گویان
&&
پیش حکمت که شاه چوگانی
\\
تا یکی گو نشد اگر چه زرست
&&
گر چه نیکوست نیست میدانی
\\
پهلوی اعتراض را بتراش
&&
گر تو چون گوی چست و گردانی
\\
پهلوی اعتراض در ابلیس
&&
گشت مردود رد ربانی
\\
پس به خراط خویش را بسپار
&&
تا یکی گو شوی اگر آنی
\\
مانعست اعتراض ابلیسی
&&
از یکی گویی و یکی دانی
\\
\end{longtable}
\end{center}
