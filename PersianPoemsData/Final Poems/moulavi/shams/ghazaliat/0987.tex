\begin{center}
\section*{غزل شماره ۹۸۷: هر که را ذوق دین پدید آید}
\label{sec:0987}
\addcontentsline{toc}{section}{\nameref{sec:0987}}
\begin{longtable}{l p{0.5cm} r}
هر که را ذوق دین پدید آید
&&
شهد دنیاش کی لذیذ آید
\\
آن چنان عقل را چه خواهی کرد
&&
که نگوسار یک نبیذ آید
\\
عقل بفروش و جمله حیرت خر
&&
که تو را سود از این خرید آید
\\
نه از آن حالتیست ای عاقل
&&
که در او عقل کس بدید آید
\\
نشود باز این چنین قفلی
&&
گر همه عقل‌ها کلید آید
\\
گر درآیند ذره ذره به بانگ
&&
آن همه بانگ ناشنید آید
\\
چه شود بیش و کم از این دریا
&&
بنده گر پاک وگر پلید آید
\\
هر که رو آورد بدین دریا
&&
گر یزیدست بایزید آید
\\
\end{longtable}
\end{center}
