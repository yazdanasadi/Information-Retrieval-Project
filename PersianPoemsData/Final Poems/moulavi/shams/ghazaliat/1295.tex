\begin{center}
\section*{غزل شماره ۱۲۹۵: بیا بیا که تویی جان جان سماع}
\label{sec:1295}
\addcontentsline{toc}{section}{\nameref{sec:1295}}
\begin{longtable}{l p{0.5cm} r}
بیا بیا که تویی جان جان جان سماع
&&
بیا که سرو روانی به بوستان سماع
\\
بیا که چون تو نبودست و هم نخواهد بود
&&
بیا که چون تو ندیدست دیدگان سماع
\\
بیا که چشمه خورشید زیر سایه تست
&&
هزار زهره تو داری بر آسمان سماع
\\
سماع شکر تو گوید به صد زبان فصیح
&&
یکی دو نکته بگویم من از زبان سماع
\\
برون ز هر دو جهانی چو در سماع آیی
&&
برون ز هر دو جهانست این جهان سماع
\\
اگر چه بام بلندست بام هفتم چرخ
&&
گذشته است از این بام نردبان سماع
\\
به زیر پای بکوبید هر چه غیر ویست
&&
سماع از آن شما و شما از آن سماع
\\
چو عشق دست درآرد به گردنم چه کنم
&&
کنار درکشمش همچنین میان سماع
\\
کنار ذره چو پر شد ز پرتو خورشید
&&
همه به رقص درآیند بی‌فغان سماع
\\
بیا که صورت عشقست شمس تبریزی
&&
که باز ماند ز عشق لبش دهان سماع
\\
\end{longtable}
\end{center}
