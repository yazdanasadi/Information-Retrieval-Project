\begin{center}
\section*{غزل شماره ۵۲۵: بی گاه شد بی‌گاه شد خورشید اندر چاه شد}
\label{sec:0525}
\addcontentsline{toc}{section}{\nameref{sec:0525}}
\begin{longtable}{l p{0.5cm} r}
بی گاه شد بی‌گاه شد خورشید اندر چاه شد
&&
خیزید ای خوش طالعان وقت طلوع ماه شد
\\
ساقی به سوی جام رو ای پاسبان بر بام رو
&&
ای جان بی‌آرام رو کان یار خلوت خواه شد
\\
اشکی که چشم افروختی صبری که خرمن سوختی
&&
عقلی که راه آموختی در نیم شب گمراه شد
\\
جان‌های باطن روشنان شب را به دل روشن کنان
&&
هندوی شب نعره زنان کان ترک در خرگاه شد
\\
باشد ز بازی‌های خوش بی‌ذوق رود فرزین شود
&&
در سایه فرخ رخی بیدق برفت و شاه شد
\\
شب روح‌ها واصل شود مقصودها حاصل شود
&&
چون روز روشن دل شود هر کو ز شب آگاه شد
\\
ای روز چون حشری مگر وی شب شب قدری مگر
&&
یا چون درخت موسیی کو مظهر الله شد
\\
شب ماه خرمن می‌کند ای روز زین بر گاو نه
&&
بنگر که راه کهکشان از سنبله پرکاه شد
\\
در چاه شب غافل مشو در دلو گردون دست زن
&&
یوسف گرفت آن دلو را از چاه سوی جاه شد
\\
در تیره شب چون مصطفی می‌رو طلب می‌کن صفا
&&
کان شه ز معراج شبی بی‌مثل و بی‌اشباه شد
\\
خاموش شد عالم به شب تا چست باشی در طلب
&&
زیرا که بانگ و عربده تشویش خلوتگاه شد
\\
ای شمس تبریزی که تو از پرده شب فارغی
&&
لاشرقی و لاغربیی اکنون سخن کوتاه شد
\\
\end{longtable}
\end{center}
