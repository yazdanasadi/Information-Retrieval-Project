\begin{center}
\section*{غزل شماره ۶۱۶: چونی و چه باشد چون تا قدر تو را داند}
\label{sec:0616}
\addcontentsline{toc}{section}{\nameref{sec:0616}}
\begin{longtable}{l p{0.5cm} r}
چونی و چه باشد چون تا قدر تو را داند
&&
جز پادشه بی‌چون قدر تو کجا داند
\\
عالم ز تو پرنورست ای دلبر دور از تو
&&
حق تو زمین داند یا چرخ سما داند
\\
این پرده نیلی را بادیست که جنباند
&&
این باد هوایی نی بادی که خدا داند
\\
خرقه غم و شادی را دانی که که می‌دوزد
&&
وین خرقه ز دوزنده خود را چه جدا داند
\\
اندر دل آیینه دانی که چه می‌تابد
&&
داند چه خیالست آن آن کس که صفا داند
\\
شقه علم عالم هر چند که می‌رقصد
&&
چشم تو علم بیند جان تو هوا داند
\\
وان کس که هوا را هم داند که چه بیچارست
&&
جز حضرت الاالله باقی همه لا داند
\\
شمس الحق تبریزی این مکر که حق دارد
&&
بی مهره تو جانم کی نرد دغا داند
\\
\end{longtable}
\end{center}
