\begin{center}
\section*{غزل شماره ۲۱۳۹: والله ملولم من کنون از جام و سغراق و کدو}
\label{sec:2139}
\addcontentsline{toc}{section}{\nameref{sec:2139}}
\begin{longtable}{l p{0.5cm} r}
والله ملولم من کنون از جام و سغراق و کدو
&&
کو ساقی دریادلی تا جام سازد از سبو
\\
با آنچ خو کردی مرا اندرمدزد آن ده مها
&&
با توست آن حیله مکن این جا مجو آن جا مجو
\\
هر بار بفریبی مرا گویی که در مجلس درآ
&&
هر آرزو که باشدت پیش آ و در گوشم بگو
\\
خوش من فریب تو خورم نندیشم و این ننگرم
&&
که من چو حلقه بر درم چون لب نهم بر گوش تو
\\
من بر درم تو واصلی حاتم کف و دریادلی
&&
بالله رها کن کاهلی می‌ریز چون خون عدو
\\
تا هوش باشد یار من باطل شود گفتار من
&&
هر دم خیالی باطلی سر برزند در پیش او
\\
آن کز میت گلگون بود یا رب چه روزافزون بود
&&
کز آب حیوان می‌کند آن خضر هر ساعت وضو
\\
از آسمان آمد ندا کای بزمتان را ما فدا
&&
طوبی لکم طوبی لکم طیبوا کراما و اشربوا
\\
سقیا لهذا المفتتح القوم غرقی فی الفرح
&&
زین سو قدح زان سو قدح تا شد شکم‌ها چارسو
\\
کس را نماند از خود خبر بربند در بگشا کمر
&&
از دست رفتیم ای پسر رو دست‌ها از ما بشو
\\
من مست چشم شنگ تو و آن طره آونگ تو
&&
کز باده گلرنگ تو وارسته‌ایم از رنگ و بو
\\
خامش کن کز بیخودی گر های و هویی می‌زدی
&&
این جا به فضل ایزدی نی های می گنجد نه هو
\\
می‌گشته‌ام بی‌هوش من تا روز روشن دوش من
&&
یک ساعتی ساران کو یک ساعتی پایان کو
\\
ای شمس تبریزی بیا ای جان و دل چاکر تو را
&&
گر چه نبشتی از جفا نام مرا بر آب جو
\\
\end{longtable}
\end{center}
