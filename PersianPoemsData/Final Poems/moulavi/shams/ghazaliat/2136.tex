\begin{center}
\section*{غزل شماره ۲۱۳۶: ساقی اگر کم شد میت دستار ما بستان گرو}
\label{sec:2136}
\addcontentsline{toc}{section}{\nameref{sec:2136}}
\begin{longtable}{l p{0.5cm} r}
ساقی اگر کم شد میت دستار ما بستان گرو
&&
چون می ز داد تو بود شاید نهادن جان گرو
\\
بس اکدش و بس کدخدا کز شور می‌های خدا
&&
کرده‌ست اندر شهر ما دکان و خان و مان گرو
\\
آن شاه ابراهیم بین کادهم به دستش معرفت
&&
مر تخت را و تاج را کرده‌ست آن سلطان گرو
\\
بوبکر سر کرده گرو عمر پسر کرده گرو
&&
عثمان جگر کرده گرو و آن بوهریره انبان گرو
\\
پس چه عجب آید تو را چون با شهان این می‌کند
&&
گر ز آنک درویشی کند از بهر می خلقان گرو
\\
آن شاهد فرد احد یک جرعه‌ای در بت نهد
&&
در عشق آن سنگ سیه کافر کند ایمان گرو
\\
من مست آن میخانه‌ام در دام آن دردانه‌ام
&&
در هیچ دامی پر خود ننهاده چون مرغان گرو
\\
بهر چه لرزی بر گرو در کار او جان گو برو
&&
جان شد گرو ای کاشکی گشتی دو صد چندان گرو
\\
خامش رها کن بلبلی در گلشن آی و درنگر
&&
بلبل نهاده پر و سر پیش گل خندان گرو
\\
\end{longtable}
\end{center}
